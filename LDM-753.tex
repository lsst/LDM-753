\documentclass[DM,lsstdraft,STS,toc]{lsstdoc}
\usepackage{enumitem}
\usepackage{booktabs}
\usepackage{arydshln}
\usepackage{afterpage}
\usepackage{pdflscape}
\usepackage{graphicx}
\usepackage{lipsum}
\usepackage[normalem]{ulem}


\input{meta.tex}

%\input{aglossary.tex}
%\makeglossaries

\DeclareRobustCommand{\cmdkey}{\raisebox{-.035em}{\includegraphics[height=.75em]{figures/cmdkey}}}

\setcounter{tocdepth}{2}

\begin{document}

\providecommand{\tightlist}{%
  \setlength{\itemsep}{0pt}\setlength{\parskip}{0pt}}

\def\product{LSST Data Management}

\setDocCompact{true}

\title[DM Infrastructure Verification Document]{Vera C. Rubin Observatory DM Service Verification Document}

\author{Jeff Carlin}
\setDocRef{\lsstDocType-\lsstDocNum}
\setDocDate{\vcsDate}

\setDocAbstract {
Data Management Service Verification Elements Baseline.
}

% Most recent last
\setDocChangeRecord{%
	\addtohist{}{2020-08-11}{First draft}{}
}

\setDocUpstreamLocation{\url{https://github.com/lsst/ldm-753}}
\setDocUpstreamVersion{\vcsRevision}

\maketitle


\section{Introduction}\label{sec:intro}


\subsection{Scope}\label{sec:scope}

The scope of this document is to capture the content and details of all DM Verification Elements
categorized in the \textbf{Service} sub-component. This will make it possibile to:

\begin{itemize}
\item provide to users and stakeholders the verification elements details, without the need to access Jira
\item approve changes to the verification elements
\end{itemize}

\subsection{Specification Flow-down}\label{sec:sepcflowd}


\subsection{LSST Verification and Validation JIRA Project (LVV)}\label{sec:lvv}

The LSST Verification and Validation JIRA Project contains the detailed specifications within or derived from,
and traceable to, the DMSR specifications, in Verification Elements. Verification Elements also specify
the verification methods, the responsible parties, and additional notes regarding verification,
as per the \citeds{LSE-160} LSST Verification and Validation Process.

The Verification Elements have one or more 
Test Cases associated with them that describe the implementation of the verification activities in terms 
of specific tests to be executed.  Those Test Cases are then scheduled via Test Plans and Campaigns, 
and executed with results reported in Test Cycles.


\subsection{Verification and Validation Schedule and Resources}\label{sec:schedule}

The schedule and resources required for the verification are defined in the LSST 
Project Management Control System (PMCS). 


\subsection{Applicable Documents}
\label{sec:docs}

\begin{tabular}[htb]{l l}
\citeds{LSE-61}  & LSST DM Subsystem Requirements \\
\citeds{LSE-160} & Verification and Validation Process \\
\end{tabular}


\newpage
%%%%%%%%%%%%%%%%%%%%%%%%%%%%%%%%%%%%%%%%%%%%%%%%%%%%%%%%%%%%%%%%%%%%%%%%%%%%%%%%%%%%%%%%%%%%%%%
% generated from JIRA project LVV
% using template at <template>.
% using docsteady version 1.2rc21.post0+g596756a.d20200811
% Please do not edit -- update information in Jira instead
%%%%%%%%%%%%%%%%%%%%%%%%%%%%%%%%%%%%%%%%%%%%%%%%%%%%%%%%%%%%%%%%%%%%%%%%%%%%%%%%%%%%%%%%%%%%%%%

\section{ DM Service Verification Elements }
\label{sec:ves}

The following is the list of verification elements defined in the context of the Service component of the DM subsystem.


\subsection{[LVV-32] DMS-REQ-0074-V-01: Difference Exposure Attributes }\label{lvv-32}

\begin{longtable}{cccc}
\hline
\textbf{Jira Link} & \textbf{Assignee} & \textbf{Status} & \textbf{Test Cases}\\ \hline
\href{https://jira.lsstcorp.org/browse/LVV-32}{LVV-32} &
Eric Bellm & Not Covered &
\begin{tabular}{c}
LVV-T20 \\
LVV-T37 \\
\end{tabular}
\\
\hline
\end{longtable}

\textbf{Verification Element Description:} \\
Demonstrate that all the noted information can be retrieved from the
database system. Requirement needs to be adjusted as PSF matching kernel
might not exist.

{\footnotesize
\begin{longtable}{p{2.5cm}p{13.5cm}}
\hline
\multicolumn{2}{c}{\textbf{Requirement Details}}\\ \hline
Requirement ID & DMS-REQ-0074 \\ \cdashline{1-2}
Requirement Description &
\begin{minipage}[]{13cm}
\textbf{Specification:} For each Difference Exposure, the DMS shall
store: the identify of the input exposures and related provenance
information, and a set of metadata attributes including at least a
representation of the PSF matching kernel used in the differencing.
\end{minipage}
\\ \cdashline{1-2}
Requirement Priority & 1b \\ \cdashline{1-2}
Upper Level Requirement &
\begin{tabular}{cl}
OSS-REQ-0122 & Provenance \\
DMS-REQ-0066 & Keep Exposure Archive \\
\end{tabular}
\\ \hline
\end{longtable}
}


\subsubsection{Test Cases Summary}
\begin{longtable}{p{3cm}p{2.5cm}p{2.5cm}p{3cm}p{4cm}}
\toprule
\href{https://jira.lsstcorp.org/secure/Tests.jspa\#/testCase/LVV-T20}{LVV-T20} & \multicolumn{4}{p{12cm}}{ AG-00-15: Scientific Verification of Difference Images } \\ \hline
\textbf{Owner} & \textbf{Status} & \textbf{Version} & \textbf{Critical Event} & \textbf{Verification Type} \\ \hline
Eric Bellm & Approved & 1 & false & Test \\ \hline
\end{longtable}
{\scriptsize
\textbf{Objective:}\\
This test will check that the difference images delivered by the Alert
Generation science pay- load meet the requirements laid down by
\citeds{LSE-61}.\\
Specifically, this will demonstrate that:

\begin{itemize}
\tightlist
\item
  Difference images have been generated and persisted during payload
  execution;
\item
  Each difference image includes information about the identity of the
  input exposures, and metadata such as a representation of the PSF
  matching kernel (DMS-REQ-0074);
\item
  Masks are correctly propagated from the input images.
\end{itemize}

This test does not include quantitative targets for the science quality
criteria.
}
\begin{longtable}{p{3cm}p{2.5cm}p{2.5cm}p{3cm}p{4cm}}
\toprule
\href{https://jira.lsstcorp.org/secure/Tests.jspa\#/testCase/LVV-T37}{LVV-T37} & \multicolumn{4}{p{12cm}}{ Verify implementation of Difference Exposure Attributes } \\ \hline
\textbf{Owner} & \textbf{Status} & \textbf{Version} & \textbf{Critical Event} & \textbf{Verification Type} \\ \hline
Eric Bellm & Draft & 1 & false & Test \\ \hline
\end{longtable}
{\scriptsize
\textbf{Objective:}\\
Verify that for each Difference Exposure the DMS stores\\
1. The identify of the input exposures and related provenance
information\\
2. Metadata attributes of the subtraction, including the PSF-matching
kernel used.
}
  
 \newpage 
\subsection{[LVV-34] DMS-REQ-0077-V-01: Maintain Archive Publicly Accessible }\label{lvv-34}

\begin{longtable}{cccc}
\hline
\textbf{Jira Link} & \textbf{Assignee} & \textbf{Status} & \textbf{Test Cases}\\ \hline
\href{https://jira.lsstcorp.org/browse/LVV-34}{LVV-34} &
Colin Slater & Not Covered &
\begin{tabular}{c}
LVV-T150 \\
\end{tabular}
\\
\hline
\end{longtable}

\textbf{Verification Element Description:} \\
For a system with 3 precursor data releases. Verify that queries can be
performed on the 2 active DRs and that the DR1 can be downloaded in
bulk. No requirement for DR1 to be queryable.

{\footnotesize
\begin{longtable}{p{2.5cm}p{13.5cm}}
\hline
\multicolumn{2}{c}{\textbf{Requirement Details}}\\ \hline
Requirement ID & DMS-REQ-0077 \\ \cdashline{1-2}
Requirement Description &
\begin{minipage}[]{13cm}
\textbf{Specification:} All releases of the DMS catalog archive shall be
maintained and preserved in a publicly accessible state for the entire
operational life of the LSST observatory.
\end{minipage}
\\ \cdashline{1-2}
Requirement Discussion &
\begin{minipage}[]{13cm}
\textbf{Discussion:} The scientific intent is satisfied by keeping data
products from the current DRP release and the one prior available with
low-latency, in a form readily queryable by the public. Earlier releases
may be available from deep-store with potentially high latency, for bulk
download by users.
\end{minipage}
\\ \cdashline{1-2}
Requirement Priority & 1b \\ \cdashline{1-2}
Upper Level Requirement &
\begin{tabular}{cl}
DMS-REQ-0076 & Keep Science Data Archive \\
OSS-REQ-0186 & Access to Previous Data Releases \\
\end{tabular}
\\ \hline
\end{longtable}
}


\subsubsection{Test Cases Summary}
\begin{longtable}{p{3cm}p{2.5cm}p{2.5cm}p{3cm}p{4cm}}
\toprule
\href{https://jira.lsstcorp.org/secure/Tests.jspa\#/testCase/LVV-T150}{LVV-T150} & \multicolumn{4}{p{12cm}}{ Verify implementation of Maintain Archive Publicly Accessible } \\ \hline
\textbf{Owner} & \textbf{Status} & \textbf{Version} & \textbf{Critical Event} & \textbf{Verification Type} \\ \hline
Colin Slater & Defined & 1 & false & Test \\ \hline
\end{longtable}
{\scriptsize
\textbf{Objective:}\\
Verify that prior data releases remain accessible.
}
  
 \newpage 
\subsection{[LVV-35] DMS-REQ-0078-V-01: Catalog Export Formats }\label{lvv-35}

\begin{longtable}{cccc}
\hline
\textbf{Jira Link} & \textbf{Assignee} & \textbf{Status} & \textbf{Test Cases}\\ \hline
\href{https://jira.lsstcorp.org/browse/LVV-35}{LVV-35} &
Colin Slater & Not Covered &
\begin{tabular}{c}
LVV-T151 \\
LVV-T1232 \\
\end{tabular}
\\
\hline
\end{longtable}

\textbf{Verification Element Description:} \\
Using TAP server, form ADQL query and verify that results can be
retrieved in the specified formats.

{\footnotesize
\begin{longtable}{p{2.5cm}p{13.5cm}}
\hline
\multicolumn{2}{c}{\textbf{Requirement Details}}\\ \hline
Requirement ID & DMS-REQ-0078 \\ \cdashline{1-2}
Requirement Description &
\begin{minipage}[]{13cm}
\textbf{Specification:} The DMS catalog archive shall provide catalog
data and associated metadata on request in community standard formats: *
Comma-separated ASCII text * eXtensible Markup Language (XML) format,
including VOTable
(\url{http://www.ivoa.net/twiki/bin/view/IVOA/IvoaVOTable}), and * FITS
tables.
\end{minipage}
\\ \cdashline{1-2}
Requirement Priority & 1a \\ \cdashline{1-2}
Upper Level Requirement &
\begin{tabular}{cl}
DMS-REQ-0076 & Keep Science Data Archive \\
OSS-REQ-0176 & Data Access \\
\end{tabular}
\\ \hline
\end{longtable}
}


\subsubsection{Test Cases Summary}
\begin{longtable}{p{3cm}p{2.5cm}p{2.5cm}p{3cm}p{4cm}}
\toprule
\href{https://jira.lsstcorp.org/secure/Tests.jspa\#/testCase/LVV-T151}{LVV-T151} & \multicolumn{4}{p{12cm}}{ Verify Implementation of Catalog Export Formats From the Notebook Aspect } \\ \hline
\textbf{Owner} & \textbf{Status} & \textbf{Version} & \textbf{Critical Event} & \textbf{Verification Type} \\ \hline
Colin Slater & Defined & 1 & false & Test \\ \hline
\end{longtable}
{\scriptsize
\textbf{Objective:}\\
Verify that catalog data is exportable from the notebook aspect in a
variety of community-standard formats.
}
\begin{longtable}{p{3cm}p{2.5cm}p{2.5cm}p{3cm}p{4cm}}
\toprule
\href{https://jira.lsstcorp.org/secure/Tests.jspa\#/testCase/LVV-T1232}{LVV-T1232} & \multicolumn{4}{p{12cm}}{ Verify Implementation of Catalog Export Formats From the Portal Aspect } \\ \hline
\textbf{Owner} & \textbf{Status} & \textbf{Version} & \textbf{Critical Event} & \textbf{Verification Type} \\ \hline
Colin Slater & Defined & 1 & false & Test \\ \hline
\end{longtable}
{\scriptsize
\textbf{Objective:}\\
Verify that catalog data is exportable from the portal aspect in a
variety of community-standard formats.
}
  
 \newpage 
\subsection{[LVV-37] DMS-REQ-0094-V-01: Keep Historical Alert Archive }\label{lvv-37}

\begin{longtable}{cccc}
\hline
\textbf{Jira Link} & \textbf{Assignee} & \textbf{Status} & \textbf{Test Cases}\\ \hline
\href{https://jira.lsstcorp.org/browse/LVV-37}{LVV-37} &
Eric Bellm & Not Covered &
\begin{tabular}{c}
LVV-T152 \\
\end{tabular}
\\
\hline
\end{longtable}

\textbf{Verification Element Description:} \\
Show that alerts go into the L1 live database. Show that it is generated
and inspect access policies, retention policies, and disaster recovery
scheme. Can not demonstrate that we are keeping it updating for the
entire survey.

{\footnotesize
\begin{longtable}{p{2.5cm}p{13.5cm}}
\hline
\multicolumn{2}{c}{\textbf{Requirement Details}}\\ \hline
Requirement ID & DMS-REQ-0094 \\ \cdashline{1-2}
Requirement Description &
\begin{minipage}[]{13cm}
\textbf{Specification:} The DMS shall preserve and keep in an accessible
state an alert archive with all issued alerts for a historical record
and for false alert analysis.
\end{minipage}
\\ \cdashline{1-2}
Requirement Priority & 1b \\ \cdashline{1-2}
Upper Level Requirement &
\begin{tabular}{cl}
DMS-REQ-0092 & Alert Attributes \\
OSS-REQ-0128 & Alerts \\
\end{tabular}
\\ \hline
\end{longtable}
}


\subsubsection{Test Cases Summary}
\begin{longtable}{p{3cm}p{2.5cm}p{2.5cm}p{3cm}p{4cm}}
\toprule
\href{https://jira.lsstcorp.org/secure/Tests.jspa\#/testCase/LVV-T152}{LVV-T152} & \multicolumn{4}{p{12cm}}{ Verify implementation of Keep Historical Alert Archive } \\ \hline
\textbf{Owner} & \textbf{Status} & \textbf{Version} & \textbf{Critical Event} & \textbf{Verification Type} \\ \hline
Eric Bellm & Draft & 1 & false & Test \\ \hline
\end{longtable}
{\scriptsize
\textbf{Objective:}\\
Verify that the DMS preserves and makes accessible an Alert Archive for
reference and for false alert analyses
}
  
 \newpage 
\subsection{[LVV-45] DMS-REQ-0103-V-01: Produce Images for EPO }\label{lvv-45}

\begin{longtable}{cccc}
\hline
\textbf{Jira Link} & \textbf{Assignee} & \textbf{Status} & \textbf{Test Cases}\\ \hline
\href{https://jira.lsstcorp.org/browse/LVV-45}{LVV-45} &
Leanne Guy & Not Covered &
\begin{tabular}{c}
LVV-T63 \\
\end{tabular}
\\
\hline
\end{longtable}

\textbf{Verification Element Description:} \\
Requirement is too vague and open-ended. Might include healpix RGB
multi-scale images. Might just be coadds? Is generation under control of
EPO for ``on demand'' generation? Or are they part of DRP? This
requirement needs to be removed and replaced with real requirements from
EPO.

{\footnotesize
\begin{longtable}{p{2.5cm}p{13.5cm}}
\hline
\multicolumn{2}{c}{\textbf{Requirement Details}}\\ \hline
Requirement ID & DMS-REQ-0103 \\ \cdashline{1-2}
Requirement Description &
\begin{minipage}[]{13cm}
\textbf{Specification:} The DMS shall produce images for EPO purposes,
according to the requirements in the DM-EPO ICD.
\end{minipage}
\\ \cdashline{1-2}
Requirement Discussion &
\begin{minipage}[]{13cm}
\textbf{Discussion:} This is expected to include polychromatic (e.g.,
RGB JPEG) images for casual users. The DM-EPO ICD is \citeds{LSE-131}.
\end{minipage}
\\ \cdashline{1-2}
Requirement Priority & 1b \\ \cdashline{1-2}
Upper Level Requirement &
\begin{tabular}{cl}
OSS-REQ-0136 & Co-added Exposures \\
\end{tabular}
\\ \hline
\end{longtable}
}


\subsubsection{Test Cases Summary}
\begin{longtable}{p{3cm}p{2.5cm}p{2.5cm}p{3cm}p{4cm}}
\toprule
\href{https://jira.lsstcorp.org/secure/Tests.jspa\#/testCase/LVV-T63}{LVV-T63} & \multicolumn{4}{p{12cm}}{ Verify implementation of Produce Images for EPO } \\ \hline
\textbf{Owner} & \textbf{Status} & \textbf{Version} & \textbf{Critical Event} & \textbf{Verification Type} \\ \hline
Gregory Dubois-Felsmann & Draft & 1 & false & Test \\ \hline
\end{longtable}
{\scriptsize
\textbf{Objective:}\\
This test will verify that the DRP pipelines produce the image data
products called out in \citeds{LSE-131}. ~Currently this is limited to a color
all-sky HiPS map. ~This will be verified (1) by inspection of pipeline
configurations and (2) in operations rehearsals on precursor data. ~The
production of a usable HiPS map will be verified by browsing it with
community tools.
}
  
 \newpage 
\subsection{[LVV-47] DMS-REQ-0119-V-01: DAC resource allocation for Level 3 processing }\label{lvv-47}

\begin{longtable}{cccc}
\hline
\textbf{Jira Link} & \textbf{Assignee} & \textbf{Status} & \textbf{Test Cases}\\ \hline
\href{https://jira.lsstcorp.org/browse/LVV-47}{LVV-47} &
Colin Slater & Not Covered &
\begin{tabular}{c}
LVV-T117 \\
\end{tabular}
\\
\hline
\end{longtable}

\textbf{Verification Element Description:} \\
Create L3 instance. Submit a number of L3 processing jobs and
demonstrate that prioritization and resource allocation happens
correctly when limits are set lower than normal.

{\footnotesize
\begin{longtable}{p{2.5cm}p{13.5cm}}
\hline
\multicolumn{2}{c}{\textbf{Requirement Details}}\\ \hline
Requirement ID & DMS-REQ-0119 \\ \cdashline{1-2}
Requirement Description &
\begin{minipage}[]{13cm}
\textbf{Specification:} The DMS shall provide a resource allocation
mechanism for the DACs that allows the prioritization and allocation of
the resources defined in DMS-REQ-0396 to a variety of Level 3 processing
and storage activities based on user identity and group membership.
\end{minipage}
\\ \cdashline{1-2}
Requirement Discussion &
\begin{minipage}[]{13cm}
\textbf{Discussion:} It is assumed that the DAC Level 3 processing
resources will likely be oversubscribed, making this necessary. This
technical mechanism is intended to be used to implement the decisions
made by an anticipated administrative mechanism in the operations
organization, such as an allocation committee.
\end{minipage}
\\ \cdashline{1-2}
Requirement Priority & 2 \\ \cdashline{1-2}
Upper Level Requirement &
\begin{tabular}{cl}
OSS-REQ-0143 & Resource Allocation \\
\end{tabular}
\\ \hline
\end{longtable}
}


\subsubsection{Test Cases Summary}
\begin{longtable}{p{3cm}p{2.5cm}p{2.5cm}p{3cm}p{4cm}}
\toprule
\href{https://jira.lsstcorp.org/secure/Tests.jspa\#/testCase/LVV-T117}{LVV-T117} & \multicolumn{4}{p{12cm}}{ Verify implementation of DAC resource allocation for Level 3 processing } \\ \hline
\textbf{Owner} & \textbf{Status} & \textbf{Version} & \textbf{Critical Event} & \textbf{Verification Type} \\ \hline
Colin Slater & Draft & 1 & false & Test \\ \hline
\end{longtable}
{\scriptsize
\textbf{Objective:}\\
Verify that compute time and storage space allocations can be granted to
science users.
}
  
 \newpage 
\subsection{[LVV-50] DMS-REQ-0122-V-01: Access to catalogs for external Level 3 processing }\label{lvv-50}

\begin{longtable}{cccc}
\hline
\textbf{Jira Link} & \textbf{Assignee} & \textbf{Status} & \textbf{Test Cases}\\ \hline
\href{https://jira.lsstcorp.org/browse/LVV-50}{LVV-50} &
Simon Krughoff & Not Covered &
\begin{tabular}{c}
LVV-T204 \\
\end{tabular}
\\
\hline
\end{longtable}

\textbf{Verification Element Description:} \\
Show that a catalog can be exported. Verify that content matches the
archive values. Demonstrate that catalog export will work with multiple
data releases.

{\footnotesize
\begin{longtable}{p{2.5cm}p{13.5cm}}
\hline
\multicolumn{2}{c}{\textbf{Requirement Details}}\\ \hline
Requirement ID & DMS-REQ-0122 \\ \cdashline{1-2}
Requirement Description &
\begin{minipage}[]{13cm}
\textbf{Specification:} The DMS shall facilitate Level 3 catalog
processing that may take place at external facilities outside the DACs.
This will principally be by facilitating the export of catalogs and the
provision of tools for maintaining and validating exported data.
\end{minipage}
\\ \cdashline{1-2}
Requirement Priority & 2 \\ \cdashline{1-2}
Upper Level Requirement &
\begin{tabular}{cl}
OSS-REQ-0180 & Data Products Query and Download Availability \\
OSS-REQ-0140 & Production \\
\end{tabular}
\\ \hline
\end{longtable}
}


\subsubsection{Test Cases Summary}
\begin{longtable}{p{3cm}p{2.5cm}p{2.5cm}p{3cm}p{4cm}}
\toprule
\href{https://jira.lsstcorp.org/secure/Tests.jspa\#/testCase/LVV-T204}{LVV-T204} & \multicolumn{4}{p{12cm}}{ Verify implementation of Access to catalogs for external Level 3
processing } \\ \hline
\textbf{Owner} & \textbf{Status} & \textbf{Version} & \textbf{Critical Event} & \textbf{Verification Type} \\ \hline
Kian-Tat Lim & Draft & 1 & false & Test \\ \hline
\end{longtable}
{\scriptsize
\textbf{Objective:}\\
Verify that catalog export, and maintenance/validation tools for Level 3
products to outside of the Data Access Centers.
}
  
 \newpage 
\subsection{[LVV-51] DMS-REQ-0123-V-01: Access to input catalogs for DAC-based Level 3
processing }\label{lvv-51}

\begin{longtable}{cccc}
\hline
\textbf{Jira Link} & \textbf{Assignee} & \textbf{Status} & \textbf{Test Cases}\\ \hline
\href{https://jira.lsstcorp.org/browse/LVV-51}{LVV-51} &
Colin Slater & Not Covered &
\begin{tabular}{c}
LVV-T205 \\
\end{tabular}
\\
\hline
\end{longtable}

\textbf{Verification Element Description:} \\
Show that a L3 job can access L1 and L2 catalogs.

{\footnotesize
\begin{longtable}{p{2.5cm}p{13.5cm}}
\hline
\multicolumn{2}{c}{\textbf{Requirement Details}}\\ \hline
Requirement ID & DMS-REQ-0123 \\ \cdashline{1-2}
Requirement Description &
\begin{minipage}[]{13cm}
\textbf{Specification:} The DMS shall provide access to all Level 1 and
Level 2 catalog products through the LSST project's Data Access Centers,
and any others that have been established and funded, for Level 3
processing that takes place at the DACs.
\end{minipage}
\\ \cdashline{1-2}
Requirement Priority & 2 \\ \cdashline{1-2}
Upper Level Requirement &
\begin{tabular}{cl}
OSS-REQ-0140 & Production \\
\end{tabular}
\\ \hline
\end{longtable}
}


\subsubsection{Test Cases Summary}
\begin{longtable}{p{3cm}p{2.5cm}p{2.5cm}p{3cm}p{4cm}}
\toprule
\href{https://jira.lsstcorp.org/secure/Tests.jspa\#/testCase/LVV-T205}{LVV-T205} & \multicolumn{4}{p{12cm}}{ Verify implementation of Access to input catalogs for DAC-based Level 3
processing } \\ \hline
\textbf{Owner} & \textbf{Status} & \textbf{Version} & \textbf{Critical Event} & \textbf{Verification Type} \\ \hline
Robert Gruendl & Draft & 1 & false & Test \\ \hline
\end{longtable}
{\scriptsize
\textbf{Objective:}\\
Verify that data products are available at the Data Access Centers for
use in Level 3 processing.
}
  
 \newpage 
\subsection{[LVV-55] DMS-REQ-0127-V-01: Access to input images for DAC-based Level 3
processing }\label{lvv-55}

\begin{longtable}{cccc}
\hline
\textbf{Jira Link} & \textbf{Assignee} & \textbf{Status} & \textbf{Test Cases}\\ \hline
\href{https://jira.lsstcorp.org/browse/LVV-55}{LVV-55} &
Colin Slater & Not Covered &
\begin{tabular}{c}
LVV-T208 \\
\end{tabular}
\\
\hline
\end{longtable}

\textbf{Verification Element Description:} \\
Show that a L3 job can access L1 and L2 image products.

{\footnotesize
\begin{longtable}{p{2.5cm}p{13.5cm}}
\hline
\multicolumn{2}{c}{\textbf{Requirement Details}}\\ \hline
Requirement ID & DMS-REQ-0127 \\ \cdashline{1-2}
Requirement Description &
\begin{minipage}[]{13cm}
\textbf{Specification:} The DMS shall provide access to all Level 1 and
Level 2 image products through the LSST project's Data Access Centers,
and any others that have been established and funded, for Level 3
processing that takes place at the DACs.
\end{minipage}
\\ \cdashline{1-2}
Requirement Priority & 2 \\ \cdashline{1-2}
Upper Level Requirement &
\begin{tabular}{cl}
OSS-REQ-0140 & Production \\
\end{tabular}
\\ \hline
\end{longtable}
}


\subsubsection{Test Cases Summary}
\begin{longtable}{p{3cm}p{2.5cm}p{2.5cm}p{3cm}p{4cm}}
\toprule
\href{https://jira.lsstcorp.org/secure/Tests.jspa\#/testCase/LVV-T208}{LVV-T208} & \multicolumn{4}{p{12cm}}{ Verify implementation of Access to input images for DAC-based Level 3
processing } \\ \hline
\textbf{Owner} & \textbf{Status} & \textbf{Version} & \textbf{Critical Event} & \textbf{Verification Type} \\ \hline
Kian-Tat Lim & Draft & 1 & false & Test \\ \hline
\end{longtable}
{\scriptsize
\textbf{Objective:}\\
Verify that prompt processing and DRP products are available at the DACs
for Level 3 processing at the DACs.
}
  
 \newpage 
\subsection{[LVV-58] DMS-REQ-0131-V-01: Time allowed to process calibs }\label{lvv-58}

\begin{longtable}{cccc}
\hline
\textbf{Jira Link} & \textbf{Assignee} & \textbf{Status} & \textbf{Test Cases}\\ \hline
\href{https://jira.lsstcorp.org/browse/LVV-58}{LVV-58} &
Robert Lupton & Not Covered &
\begin{tabular}{c}
LVV-T106 \\
\end{tabular}
\\
\hline
\end{longtable}

\textbf{Verification Element Description:} \\
With calibration observation data that requires the most processing,
ensure that it can be processed and stored on the correct timescale.
Simulate a ``worst possible'' night's observing and inspect the daily
operations plan.

Associated element
(\href{https://jira.lsstcorp.org/browse/LVV-9745}{LVV-9745}) satisfies
the number of calibs to be processed in the allotted time.

{\footnotesize
\begin{longtable}{p{2.5cm}p{13.5cm}}
\hline
\multicolumn{2}{c}{\textbf{Requirement Details}}\\ \hline
Requirement ID & DMS-REQ-0131 \\ \cdashline{1-2}
Requirement Description &
\begin{minipage}[]{13cm}
\textbf{Specification:} Calibration products from a group of up to
\textbf{nCalExpProc} related exposures that should be processed
together, shall be available from the DMS image archive within
\textbf{calProcTime} of the end of the acquisition of images/data for
that group.
\end{minipage}
\\ \cdashline{1-2}
Requirement Parameters & {[}\textbf{nCalExpProc = 25{{[}integer{]}}} Maximum number of
calibration exposures that can be processed together within time
calProcTime., \textbf{calProcTime = 1200{{[}second{]}}} Time allowed to
process nCalExpProc calibration exposures and have them available within
the DMS.{]} \\ \cdashline{1-2}
Requirement Discussion &
\begin{minipage}[]{13cm}
\textbf{Discussion:} The motivation here is that calibration images will
be needed at least 1 hour prior to the start of observing and this
requirement allows the calibration observations to be planned
accordingly.
\end{minipage}
\\ \cdashline{1-2}
Requirement Priority & 2 \\ \cdashline{1-2}
Upper Level Requirement &
\begin{tabular}{cl}
OSS-REQ-0046 & Calibration \\
OSS-REQ-0021 & Base Site \\
OSS-REQ-0194 & Calibration Exposures Per Day \\
DMS-REQ-0130 & Calibration Data Products \\
\end{tabular}
\\ \hline
\end{longtable}
}


\subsubsection{Test Cases Summary}
\begin{longtable}{p{3cm}p{2.5cm}p{2.5cm}p{3cm}p{4cm}}
\toprule
\href{https://jira.lsstcorp.org/secure/Tests.jspa\#/testCase/LVV-T106}{LVV-T106} & \multicolumn{4}{p{12cm}}{ Verify implementation of Calibration Images Available Within Specified
Time } \\ \hline
\textbf{Owner} & \textbf{Status} & \textbf{Version} & \textbf{Critical Event} & \textbf{Verification Type} \\ \hline
Kian-Tat Lim & Draft & 1 & false & Test \\ \hline
\end{longtable}
{\scriptsize
\textbf{Objective:}\\
Execute single-day operations rehearsal, observe data products generated
}
  
 \newpage 
\subsection{[LVV-60] DMS-REQ-0155-V-01: Provide Data Access Services }\label{lvv-60}

\begin{longtable}{cccc}
\hline
\textbf{Jira Link} & \textbf{Assignee} & \textbf{Status} & \textbf{Test Cases}\\ \hline
\href{https://jira.lsstcorp.org/browse/LVV-60}{LVV-60} &
Gregory Dubois-Felsmann & Not Covered &
\begin{tabular}{c}
\end{tabular}
\\
\hline
\end{longtable}

\textbf{Verification Element Description:} \\
Undefined

{\footnotesize
\begin{longtable}{p{2.5cm}p{13.5cm}}
\hline
\multicolumn{2}{c}{\textbf{Requirement Details}}\\ \hline
Requirement ID & DMS-REQ-0155 \\ \cdashline{1-2}
Requirement Description &
\begin{minipage}[]{13cm}

\end{minipage}
\\ \cdashline{1-2}
Requirement Discussion &
\begin{minipage}[]{13cm}
(This is a composite requirement in the SysML model, which simply
aggregates its children.)
\end{minipage}
\\ \cdashline{1-2}
Requirement Priority & 1a \\ \cdashline{1-2}
Upper Level Requirement &
\begin{tabular}{cl}
OSS-REQ-0176 & Data Access \\
\end{tabular}
\\ \hline
\end{longtable}
}

\subsubsection{Verified By}
\begin{itemize}
\item . LVV-129 (\ref{lvv-129}) DMS-REQ-0298-V-01: Data Product and Raw Data Access
\item . LVV-130 (\ref{lvv-130}) DMS-REQ-0299-V-01: Data Product Ingest
\item . LVV-131 (\ref{lvv-131}) DMS-REQ-0300-V-01: Bulk Download Service
\end{itemize}

  
 \newpage 
\subsection{[LVV-61] DMS-REQ-0156-V-01: Provide Pipeline Execution Services }\label{lvv-61}

\begin{longtable}{cccc}
\hline
\textbf{Jira Link} & \textbf{Assignee} & \textbf{Status} & \textbf{Test Cases}\\ \hline
\href{https://jira.lsstcorp.org/browse/LVV-61}{LVV-61} &
Robert Gruendl & Not Covered &
\begin{tabular}{c}
\end{tabular}
\\
\hline
\end{longtable}

\textbf{Verification Element Description:} \\
Undefined

{\footnotesize
\begin{longtable}{p{2.5cm}p{13.5cm}}
\hline
\multicolumn{2}{c}{\textbf{Requirement Details}}\\ \hline
Requirement ID & DMS-REQ-0156 \\ \cdashline{1-2}
Requirement Description &
\begin{minipage}[]{13cm}

\end{minipage}
\\ \cdashline{1-2}
Requirement Discussion &
\begin{minipage}[]{13cm}
(This is a composite requirement in the SysML model, which simply
aggregates its children.)
\end{minipage}
\\ \cdashline{1-2}
Requirement Priority & 1a \\ \cdashline{1-2}
Upper Level Requirement &
\begin{tabular}{cl}
OSS-REQ-0117 & Automated Production \\
OSS-REQ-0037 & Observatory Control System Definition \\
\end{tabular}
\\ \hline
\end{longtable}
}

\subsubsection{Verified By}
\begin{itemize}
\item . LVV-133 (\ref{lvv-133}) DMS-REQ-0302-V-01: Production Orchestration
\item . LVV-134 (\ref{lvv-134}) DMS-REQ-0303-V-01: Production Monitoring
\item . LVV-135 (\ref{lvv-135}) DMS-REQ-0304-V-01: Production Fault Tolerance
\end{itemize}

  
 \newpage 
\subsection{[LVV-63] DMS-REQ-0160-V-01: Provide User Interface Services }\label{lvv-63}

\begin{longtable}{cccc}
\hline
\textbf{Jira Link} & \textbf{Assignee} & \textbf{Status} & \textbf{Test Cases}\\ \hline
\href{https://jira.lsstcorp.org/browse/LVV-63}{LVV-63} &
Gregory Dubois-Felsmann & Not Covered &
\begin{tabular}{c}
LVV-T131 \\
LVV-T368 \\
LVV-T368 \\
\end{tabular}
\\
\hline
\end{longtable}

\textbf{Verification Element Description:} \\
Show that the SUI can handle these queries and interactions. Need to be
more explicit on resampling/re-project. Healpix?

{\footnotesize
\begin{longtable}{p{2.5cm}p{13.5cm}}
\hline
\multicolumn{2}{c}{\textbf{Requirement Details}}\\ \hline
Requirement ID & DMS-REQ-0160 \\ \cdashline{1-2}
Requirement Description &
\begin{minipage}[]{13cm}
\textbf{Specification:} The DMS shall provide software for User
Interface Services, including services to: browse LSST data products
through astronomical views or visualizations; create and serve ``best''
images of selectable regions of the sky; resample and re-project images,
and visualize catalog content.
\end{minipage}
\\ \cdashline{1-2}
Requirement Priority & 1b \\ \cdashline{1-2}
Upper Level Requirement &
\begin{tabular}{cl}
OSS-REQ-0057 & Image Visualization \\
\end{tabular}
\\ \hline
\end{longtable}
}


\subsubsection{Test Cases Summary}
\begin{longtable}{p{3cm}p{2.5cm}p{2.5cm}p{3cm}p{4cm}}
\toprule
\href{https://jira.lsstcorp.org/secure/Tests.jspa\#/testCase/LVV-T131}{LVV-T131} & \multicolumn{4}{p{12cm}}{ Verify implementation of Provide User Interface Services } \\ \hline
\textbf{Owner} & \textbf{Status} & \textbf{Version} & \textbf{Critical Event} & \textbf{Verification Type} \\ \hline
Gregory Dubois-Felsmann & Defined & 1 & false & Test \\ \hline
\end{longtable}
{\scriptsize
\textbf{Objective:}\\
Verify the availability and functionality of the broad range of user
interface services called for in the requirement, as applied to both
Nightly and DRP data. ~This will primarily be done by verifications
performed at the LSST Science Platform level, based on the requirements
in \citeds{LDM-554}; however, a high-level set of tests corresponding to the
DMS-REQ-0160 requirement are defined below.
}
\begin{longtable}{p{3cm}p{2.5cm}p{2.5cm}p{3cm}p{4cm}}
\toprule
\href{https://jira.lsstcorp.org/secure/Tests.jspa\#/testCase/LVV-T368}{LVV-T368} & \multicolumn{4}{p{12cm}}{ Loading and processing Camera test data } \\ \hline
\textbf{Owner} & \textbf{Status} & \textbf{Version} & \textbf{Critical Event} & \textbf{Verification Type} \\ \hline
John Swinbank & Approved & 2 & false & Test \\ \hline
\end{longtable}
{\scriptsize
\textbf{Objective:}\\
This test will check:

\begin{itemize}
\tightlist
\item
  That Camera test data is available for processing in the LSST Data
  Facility, and accessible through the LSST Science Platform;
\item
  That the Data Management I/O abstraction (the ``Data Butler'') can
  load that data into the Science Platform environment;
\item
  That Data Management algorithmic ``tasks'' can be executed to process
  that data;
\item
  That results can be displayed in the Firefly display tool.
\end{itemize}
}
\begin{longtable}{p{3cm}p{2.5cm}p{2.5cm}p{3cm}p{4cm}}
\toprule
\href{https://jira.lsstcorp.org/secure/Tests.jspa\#/testCase/LVV-T368}{LVV-T368} & \multicolumn{4}{p{12cm}}{ Loading and processing Camera test data } \\ \hline
\textbf{Owner} & \textbf{Status} & \textbf{Version} & \textbf{Critical Event} & \textbf{Verification Type} \\ \hline
John Swinbank & Approved & 2 & false & Test \\ \hline
\end{longtable}
{\scriptsize
\textbf{Objective:}\\
This test will check:

\begin{itemize}
\tightlist
\item
  That Camera test data is available for processing in the LSST Data
  Facility, and accessible through the LSST Science Platform;
\item
  That the Data Management I/O abstraction (the ``Data Butler'') can
  load that data into the Science Platform environment;
\item
  That Data Management algorithmic ``tasks'' can be executed to process
  that data;
\item
  That results can be displayed in the Firefly display tool.
\end{itemize}
}
  
 \newpage 
\subsection{[LVV-129] DMS-REQ-0298-V-01: Data Product and Raw Data Access }\label{lvv-129}

\begin{longtable}{cccc}
\hline
\textbf{Jira Link} & \textbf{Assignee} & \textbf{Status} & \textbf{Test Cases}\\ \hline
\href{https://jira.lsstcorp.org/browse/LVV-129}{LVV-129} &
Robert Gruendl & Not Covered &
\begin{tabular}{c}
LVV-T136 \\
LVV-T368 \\
LVV-T374 \\
LVV-T368 \\
\end{tabular}
\\
\hline
\end{longtable}

\textbf{Verification Element Description:} \\
Some of this is handled by the SUI requirement (DMS-REQ-0160). A key
demonstration is to run a test suite that does each of the requests in
turn and verifies against reference results.

{\footnotesize
\begin{longtable}{p{2.5cm}p{13.5cm}}
\hline
\multicolumn{2}{c}{\textbf{Requirement Details}}\\ \hline
Requirement ID & DMS-REQ-0298 \\ \cdashline{1-2}
Requirement Description &
\begin{minipage}[]{13cm}
\textbf{Specification:} The DMS shall provide software for Data Access
Services to list and retrieve image, file, and catalog data products
(including raw telescope images and calibration data), their associated
metadata, their provenance, or any combination thereof, independent of
their actual storage location.
\end{minipage}
\\ \cdashline{1-2}
Requirement Priority & 1a \\ \cdashline{1-2}
Upper Level Requirement &
\begin{tabular}{cl}
OSS-REQ-0176 & Data Access \\
\end{tabular}
\\ \hline
\end{longtable}
}


\subsubsection{Test Cases Summary}
\begin{longtable}{p{3cm}p{2.5cm}p{2.5cm}p{3cm}p{4cm}}
\toprule
\href{https://jira.lsstcorp.org/secure/Tests.jspa\#/testCase/LVV-T136}{LVV-T136} & \multicolumn{4}{p{12cm}}{ Verify implementation of Data Product and Raw Data Access } \\ \hline
\textbf{Owner} & \textbf{Status} & \textbf{Version} & \textbf{Critical Event} & \textbf{Verification Type} \\ \hline
Colin Slater & Defined & 1 & false & Test \\ \hline
\end{longtable}
{\scriptsize
\textbf{Objective:}\\
Verify that available image, file, and catalog data products, and their
metadata and provenance information, can be listed and retrieved.
}
\begin{longtable}{p{3cm}p{2.5cm}p{2.5cm}p{3cm}p{4cm}}
\toprule
\href{https://jira.lsstcorp.org/secure/Tests.jspa\#/testCase/LVV-T368}{LVV-T368} & \multicolumn{4}{p{12cm}}{ Loading and processing Camera test data } \\ \hline
\textbf{Owner} & \textbf{Status} & \textbf{Version} & \textbf{Critical Event} & \textbf{Verification Type} \\ \hline
John Swinbank & Approved & 2 & false & Test \\ \hline
\end{longtable}
{\scriptsize
\textbf{Objective:}\\
This test will check:

\begin{itemize}
\tightlist
\item
  That Camera test data is available for processing in the LSST Data
  Facility, and accessible through the LSST Science Platform;
\item
  That the Data Management I/O abstraction (the ``Data Butler'') can
  load that data into the Science Platform environment;
\item
  That Data Management algorithmic ``tasks'' can be executed to process
  that data;
\item
  That results can be displayed in the Firefly display tool.
\end{itemize}
}
\begin{longtable}{p{3cm}p{2.5cm}p{2.5cm}p{3cm}p{4cm}}
\toprule
\href{https://jira.lsstcorp.org/secure/Tests.jspa\#/testCase/LVV-T374}{LVV-T374} & \multicolumn{4}{p{12cm}}{ Ingesting Camera test data } \\ \hline
\textbf{Owner} & \textbf{Status} & \textbf{Version} & \textbf{Critical Event} & \textbf{Verification Type} \\ \hline
John Swinbank & Approved & 1 & false & Test \\ \hline
\end{longtable}
{\scriptsize
\textbf{Objective:}\\
This test will check:

\begin{itemize}
\tightlist
\item
  That raw Camera test data is available on a filesystem in the LSST
  Data Facility;
\item
  That raw Camera test data can be ingested and made available through
  the Data Management I/O abstraction (the ``Data Butler'').
\end{itemize}
}
\begin{longtable}{p{3cm}p{2.5cm}p{2.5cm}p{3cm}p{4cm}}
\toprule
\href{https://jira.lsstcorp.org/secure/Tests.jspa\#/testCase/LVV-T368}{LVV-T368} & \multicolumn{4}{p{12cm}}{ Loading and processing Camera test data } \\ \hline
\textbf{Owner} & \textbf{Status} & \textbf{Version} & \textbf{Critical Event} & \textbf{Verification Type} \\ \hline
John Swinbank & Approved & 2 & false & Test \\ \hline
\end{longtable}
{\scriptsize
\textbf{Objective:}\\
This test will check:

\begin{itemize}
\tightlist
\item
  That Camera test data is available for processing in the LSST Data
  Facility, and accessible through the LSST Science Platform;
\item
  That the Data Management I/O abstraction (the ``Data Butler'') can
  load that data into the Science Platform environment;
\item
  That Data Management algorithmic ``tasks'' can be executed to process
  that data;
\item
  That results can be displayed in the Firefly display tool.
\end{itemize}
}
  
 \newpage 
\subsection{[LVV-139] DMS-REQ-0308-V-01: Software Architecture to Enable Community Re-Use }\label{lvv-139}

\begin{longtable}{cccc}
\hline
\textbf{Jira Link} & \textbf{Assignee} & \textbf{Status} & \textbf{Test Cases}\\ \hline
\href{https://jira.lsstcorp.org/browse/LVV-139}{LVV-139} &
Simon Krughoff & Not Covered &
\begin{tabular}{c}
LVV-T10 \\
LVV-T17 \\
LVV-T124 \\
LVV-T216 \\
LVV-T362 \\
LVV-T363 \\
\end{tabular}
\\
\hline
\end{longtable}

\textbf{Verification Element Description:} \\
Show that a processing run (of limited size) can be performed on a
desktop and archive centre. Are we meant to be verifying ``high
throughput''?

{\footnotesize
\begin{longtable}{p{2.5cm}p{13.5cm}}
\hline
\multicolumn{2}{c}{\textbf{Requirement Details}}\\ \hline
Requirement ID & DMS-REQ-0308 \\ \cdashline{1-2}
Requirement Description &
\begin{minipage}[]{13cm}
\textbf{Specification:} The DMS software architecture shall be designed
to enable high throughput on high-performance compute platforms, while
also enabling the use of science-specific algorithms by science users on
commodity desktop compute platforms.
\end{minipage}
\\ \cdashline{1-2}
Requirement Discussion &
\begin{minipage}[]{13cm}
\textbf{Discussion:} The high data volume and short processing timeline
for LSST Productions anticipates the use of high-performance compute
infrastructure, while the need to make the science algorithms
immediately applicable to science teams for Level-3 processing drives
the need for easy interoperability with desktop compute environments.
\end{minipage}
\\ \cdashline{1-2}
Requirement Priority & 1b \\ \cdashline{1-2}
Upper Level Requirement &
\begin{tabular}{cl}
OSS-REQ-0121 & Open Source, Open Configuration \\
\end{tabular}
\\ \hline
\end{longtable}
}


\subsubsection{Test Cases Summary}
\begin{longtable}{p{3cm}p{2.5cm}p{2.5cm}p{3cm}p{4cm}}
\toprule
\href{https://jira.lsstcorp.org/secure/Tests.jspa\#/testCase/LVV-T10}{LVV-T10} & \multicolumn{4}{p{12cm}}{ DRP-00-00: Installation of the Data Release Production v14.0 science
payload } \\ \hline
\textbf{Owner} & \textbf{Status} & \textbf{Version} & \textbf{Critical Event} & \textbf{Verification Type} \\ \hline
Jim Bosch & Approved & 1 & false & Test \\ \hline
\end{longtable}
{\scriptsize
\textbf{Objective:}\\
This test will check:

\begin{itemize}
\tightlist
\item
  That the Data Release Production science payload is available for
  distribution from documented channels;
\item
  That the Data Release Production science payload can be installed on
  LSST Data Facility-managed systems.
\end{itemize}
}
\begin{longtable}{p{3cm}p{2.5cm}p{2.5cm}p{3cm}p{4cm}}
\toprule
\href{https://jira.lsstcorp.org/secure/Tests.jspa\#/testCase/LVV-T17}{LVV-T17} & \multicolumn{4}{p{12cm}}{ AG-00-00: Installation of the Alert Generation v16.0 science payload. } \\ \hline
\textbf{Owner} & \textbf{Status} & \textbf{Version} & \textbf{Critical Event} & \textbf{Verification Type} \\ \hline
Eric Bellm & Approved & 1 & false & Test \\ \hline
\end{longtable}
{\scriptsize
\textbf{Objective:}\\
This test will check:

\begin{itemize}
\tightlist
\item
  That the Alert Generation science payload is available for
  distribution from documented channels;
\item
  That the Alert Generation science payload can be installed on LSST
  Data Facility-managed systems.
\end{itemize}
}
\begin{longtable}{p{3cm}p{2.5cm}p{2.5cm}p{3cm}p{4cm}}
\toprule
\href{https://jira.lsstcorp.org/secure/Tests.jspa\#/testCase/LVV-T124}{LVV-T124} & \multicolumn{4}{p{12cm}}{ Verify implementation of Software Architecture to Enable Community
Re-Use } \\ \hline
\textbf{Owner} & \textbf{Status} & \textbf{Version} & \textbf{Critical Event} & \textbf{Verification Type} \\ \hline
Simon Krughoff & Defined & 1 & false & Test \\ \hline
\end{longtable}
{\scriptsize
\textbf{Objective:}\\
Show that the LSST software is capable of being executed in multiple
contexts: single user instance, batch processing, continuous
integration.\\
Also show that the algorithms can be reconfigured and, if desired,
completely replaced at run time.
}
\begin{longtable}{p{3cm}p{2.5cm}p{2.5cm}p{3cm}p{4cm}}
\toprule
\href{https://jira.lsstcorp.org/secure/Tests.jspa\#/testCase/LVV-T216}{LVV-T216} & \multicolumn{4}{p{12cm}}{ Installation of the Alert Distribution payloads. } \\ \hline
\textbf{Owner} & \textbf{Status} & \textbf{Version} & \textbf{Critical Event} & \textbf{Verification Type} \\ \hline
Eric Bellm & Approved & 1 & false & Test \\ \hline
\end{longtable}
{\scriptsize
\textbf{Objective:}\\
This test will check:\\

\begin{itemize}
\tightlist
\item
  That the Alert Distribution payloads are available from documented
  channels.
\item
  That the Alert Distribution payloads can be installed on LSST Data
  Facility-managed systems.
\item
  That the Alert Distribution payloads can be executed by LSST Data
  Facility-managed systems.
\end{itemize}
}
\begin{longtable}{p{3cm}p{2.5cm}p{2.5cm}p{3cm}p{4cm}}
\toprule
\href{https://jira.lsstcorp.org/secure/Tests.jspa\#/testCase/LVV-T362}{LVV-T362} & \multicolumn{4}{p{12cm}}{ Installation of the LSST Science Pipelines Payloads } \\ \hline
\textbf{Owner} & \textbf{Status} & \textbf{Version} & \textbf{Critical Event} & \textbf{Verification Type} \\ \hline
John Swinbank & Draft & 1 & false & Test \\ \hline
\end{longtable}
{\scriptsize
\textbf{Objective:}\\
This test will check that:

\begin{itemize}
\tightlist
\item
  The Alert Production Pipeline payload is available for installation
  from documented channels;
\item
  The Data Release Production Pipeline payload is available for
  installation from documented channels;
\item
  The Calibration Products Production Pipeline payload is available for
  installation from documented channels;
\item
  These payloads can be installed on systems at the LSST Data Facility
  following available documentation;
\item
  The installed pipeline payloads are capable of successfully executing
  basic integration tests.
\end{itemize}

Note that this test assumes a 2018-era packaging of the Science
Pipelines software, in which all the above payloads are represented by a
single ``meta-package'', lsst\_distrib.
}
\begin{longtable}{p{3cm}p{2.5cm}p{2.5cm}p{3cm}p{4cm}}
\toprule
\href{https://jira.lsstcorp.org/secure/Tests.jspa\#/testCase/LVV-T363}{LVV-T363} & \multicolumn{4}{p{12cm}}{ Science Pipelines Release Documentation } \\ \hline
\textbf{Owner} & \textbf{Status} & \textbf{Version} & \textbf{Critical Event} & \textbf{Verification Type} \\ \hline
John Swinbank & Draft & 1 & false & Inspection \\ \hline
\end{longtable}
{\scriptsize
\textbf{Objective:}\\
This test will check:

\begin{itemize}
\tightlist
\item
  That a particular Science Pipelines release is adequately described by
  documentation at the https://pipelines.lsst.io/ site;
\item
  That the Science Pipelines release is accompanied by a
  characterization report which describes its scientific performance.
\end{itemize}
}
  
 \newpage 
\subsection{[LVV-143] DMS-REQ-0312-V-01: Level 1 Data Product Access }\label{lvv-143}

\begin{longtable}{cccc}
\hline
\textbf{Jira Link} & \textbf{Assignee} & \textbf{Status} & \textbf{Test Cases}\\ \hline
\href{https://jira.lsstcorp.org/browse/LVV-143}{LVV-143} &
Eric Bellm & Not Covered &
\begin{tabular}{c}
LVV-T157 \\
\end{tabular}
\\
\hline
\end{longtable}

\textbf{Verification Element Description:} \\
Do a real-time L1 test run. Demonstrate that an end-user can see the L1
database being updated live.

The term ''live'' Level 1 Database refers to the Prompt Products
Database being updated within L1PublicT, and while it is updated as a
result of Alert Production it does not contain copies of the alert
packets, which are stored elsewhere
(\href{https://jira.lsstcorp.org/browse/LVV-1456}{LVV-1456}).

{\footnotesize
\begin{longtable}{p{2.5cm}p{13.5cm}}
\hline
\multicolumn{2}{c}{\textbf{Requirement Details}}\\ \hline
Requirement ID & DMS-REQ-0312 \\ \cdashline{1-2}
Requirement Description &
\begin{minipage}[]{13cm}
\textbf{Specification:} The DMS shall maintain a ``live'' Level 1
Database for query by science users, updated as a result of Alert
Production processing.
\end{minipage}
\\ \cdashline{1-2}
Requirement Priority & 1b \\ \cdashline{1-2}
Upper Level Requirement &
\begin{tabular}{cl}
OSS-REQ-0185 & Transient Alert Query \\
OSS-REQ-0127 & Level 1 Data Product Availability \\
\end{tabular}
\\ \hline
\end{longtable}
}


\subsubsection{Test Cases Summary}
\begin{longtable}{p{3cm}p{2.5cm}p{2.5cm}p{3cm}p{4cm}}
\toprule
\href{https://jira.lsstcorp.org/secure/Tests.jspa\#/testCase/LVV-T157}{LVV-T157} & \multicolumn{4}{p{12cm}}{ Verify implementation Level 1 Data Product Access } \\ \hline
\textbf{Owner} & \textbf{Status} & \textbf{Version} & \textbf{Critical Event} & \textbf{Verification Type} \\ \hline
Colin Slater & Draft & 1 & false & Test \\ \hline
\end{longtable}
{\scriptsize
\textbf{Objective:}\\
Verify that Level 1 Data Products are accessible by science users.
}
  
 \newpage 
\subsection{[LVV-144] DMS-REQ-0313-V-01: Level 1 \& 2 Catalog Access }\label{lvv-144}

\begin{longtable}{cccc}
\hline
\textbf{Jira Link} & \textbf{Assignee} & \textbf{Status} & \textbf{Test Cases}\\ \hline
\href{https://jira.lsstcorp.org/browse/LVV-144}{LVV-144} &
Colin Slater & Not Covered &
\begin{tabular}{c}
LVV-T158 \\
\end{tabular}
\\
\hline
\end{longtable}

\textbf{Verification Element Description:} \\
Can only really be demonstrated when the 3rd data release is created.
This could be done using precursor survey data by demonstrating that DR1
is deleted and moved to tape when DR3 is released. It may be that for
commissioning we can only show this by inspection of release policy
document.

{\footnotesize
\begin{longtable}{p{2.5cm}p{13.5cm}}
\hline
\multicolumn{2}{c}{\textbf{Requirement Details}}\\ \hline
Requirement ID & DMS-REQ-0313 \\ \cdashline{1-2}
Requirement Description &
\begin{minipage}[]{13cm}
\textbf{Specification:} The DMS shall maintain both the Level-2 catalog
and the reprocessed Level-1 catalog from the most recent two Data
Releases for query by science users, as well as versions of the most
recent catalogs generated from Special Programs data.
\end{minipage}
\\ \cdashline{1-2}
Requirement Discussion &
\begin{minipage}[]{13cm}
\textbf{Discussion:} There is no requirement for older data releases to
be queryable.
\end{minipage}
\\ \cdashline{1-2}
Requirement Priority & 1b \\ \cdashline{1-2}
Upper Level Requirement &
\begin{tabular}{cl}
OSS-REQ-0186 & Access to Previous Data Releases \\
\end{tabular}
\\ \hline
\end{longtable}
}


\subsubsection{Test Cases Summary}
\begin{longtable}{p{3cm}p{2.5cm}p{2.5cm}p{3cm}p{4cm}}
\toprule
\href{https://jira.lsstcorp.org/secure/Tests.jspa\#/testCase/LVV-T158}{LVV-T158} & \multicolumn{4}{p{12cm}}{ Verify implementation Level 1 and 2 Catalog Access } \\ \hline
\textbf{Owner} & \textbf{Status} & \textbf{Version} & \textbf{Critical Event} & \textbf{Verification Type} \\ \hline
Colin Slater & Draft & 1 & false & Test \\ \hline
\end{longtable}
{\scriptsize
\textbf{Objective:}\\
Verify that Data Release Products are accessible by science users.
}
  
 \newpage 
\subsection{[LVV-151] DMS-REQ-0320-V-01: Processing of Data From Special Programs }\label{lvv-151}

\begin{longtable}{cccc}
\hline
\textbf{Jira Link} & \textbf{Assignee} & \textbf{Status} & \textbf{Test Cases}\\ \hline
\href{https://jira.lsstcorp.org/browse/LVV-151}{LVV-151} &
Melissa Graham & Not Covered &
\begin{tabular}{c}
LVV-T92 \\
\end{tabular}
\\
\hline
\end{longtable}

\textbf{Verification Element Description:} \\
Fo a simulated night of observing that includes some special program
observations, show that the special programs observations are reduced
using a specialized recipe.

{\footnotesize
\begin{longtable}{p{2.5cm}p{13.5cm}}
\hline
\multicolumn{2}{c}{\textbf{Requirement Details}}\\ \hline
Requirement ID & DMS-REQ-0320 \\ \cdashline{1-2}
Requirement Description &
\begin{minipage}[]{13cm}
\textbf{Specification:} It shall be possible for special programs to
trigger their own data processing recipes, during the night instead of
the nightly Alert Processing (but the recipes may still issue Alerts),
or on alternative timescales.
\end{minipage}
\\ \cdashline{1-2}
Requirement Discussion &
\begin{minipage}[]{13cm}
\textbf{Discussion:} LSST will provide these recipes for processing
Special Programs data when possible, which includes cases where DM can
run original or reconfigured versions of existing pipelines, and
excludes cases where the development of new algorithms, or the
allocation of significant additional computational resources, are
required. An example of an alternative timescale is a nightly trigger to
coadd all the deep-drilling field images. Decisions about which recipes
are applied to which Special Programs will be made by the Operations
team, after consideration of the scientific goals, computational
resources, and data rights policy.
\end{minipage}
\\ \cdashline{1-2}
Requirement Priority & 2 \\ \cdashline{1-2}
Upper Level Requirement &
\begin{tabular}{cl}
LSR-REQ-0075 & Survey Time Allocation \\
OSS-REQ-0392 & Data Products Handling for Special Programs \\
\end{tabular}
\\ \hline
\end{longtable}
}


\subsubsection{Test Cases Summary}
\begin{longtable}{p{3cm}p{2.5cm}p{2.5cm}p{3cm}p{4cm}}
\toprule
\href{https://jira.lsstcorp.org/secure/Tests.jspa\#/testCase/LVV-T92}{LVV-T92} & \multicolumn{4}{p{12cm}}{ Verify implementation of Processing of Data From Special Programs } \\ \hline
\textbf{Owner} & \textbf{Status} & \textbf{Version} & \textbf{Critical Event} & \textbf{Verification Type} \\ \hline
Melissa Graham & Draft & 1 & false & Test \\ \hline
\end{longtable}
{\scriptsize
\textbf{Objective:}\\
For a simulated night of observing that includes some special program
observations, show that the SP observations are reduced using their
designated reconfigured pipelines (i.e., that the image metadata is
sufficient to trigger the processing and include all other relevant
images in the processing).
}
  
 \newpage 
\subsection{[LVV-171] DMS-REQ-0340-V-01: Access Controls of Level 3 Data Products }\label{lvv-171}

\begin{longtable}{cccc}
\hline
\textbf{Jira Link} & \textbf{Assignee} & \textbf{Status} & \textbf{Test Cases}\\ \hline
\href{https://jira.lsstcorp.org/browse/LVV-171}{LVV-171} &
Simon Krughoff & Not Covered &
\begin{tabular}{c}
LVV-T123 \\
\end{tabular}
\\
\hline
\end{longtable}

\textbf{Verification Element Description:} \\
Create some L3 data products. Adjust permissions and show that retrieval
fails if permissions are not suitable.

{\footnotesize
\begin{longtable}{p{2.5cm}p{13.5cm}}
\hline
\multicolumn{2}{c}{\textbf{Requirement Details}}\\ \hline
Requirement ID & DMS-REQ-0340 \\ \cdashline{1-2}
Requirement Description &
\begin{minipage}[]{13cm}
\textbf{Specification:} All Level 3 data products shall be configured to
have the ability to have access restricted to the owner, a list of
people, a named group, or be completely public.
\end{minipage}
\\ \cdashline{1-2}
Requirement Discussion &
\begin{minipage}[]{13cm}
\textbf{Discussion:} These features are supported by VOSpace.
\end{minipage}
\\ \cdashline{1-2}
Requirement Priority & 2 \\ \cdashline{1-2}
Upper Level Requirement &
\begin{tabular}{cl}
OSS-REQ-0176 & Data Access \\
OSS-REQ-0187 & Information Security \\
OSS-REQ-0142 & Access \\
\end{tabular}
\\ \hline
\end{longtable}
}


\subsubsection{Test Cases Summary}
\begin{longtable}{p{3cm}p{2.5cm}p{2.5cm}p{3cm}p{4cm}}
\toprule
\href{https://jira.lsstcorp.org/secure/Tests.jspa\#/testCase/LVV-T123}{LVV-T123} & \multicolumn{4}{p{12cm}}{ Verify implementation of Access Controls of Level 3 Data Products } \\ \hline
\textbf{Owner} & \textbf{Status} & \textbf{Version} & \textbf{Critical Event} & \textbf{Verification Type} \\ \hline
Robert Gruendl & Draft & 1 & false & Test \\ \hline
\end{longtable}
{\scriptsize
\textbf{Objective:}\\
This test touches upon the interface between the following areas: IT
Security, Identity Management, LSP Portal, and Parallel Distributed
Database. ~The purpose is to show that access to user generated data
products (previously Level 3) can have a variety of access restrictions
varying from single-user, a list, a named group, or open access.
}
  
 \newpage 
\subsection{[LVV-172] DMS-REQ-0341-V-01: Max elapsed time for precovery results }\label{lvv-172}

\begin{longtable}{cccc}
\hline
\textbf{Jira Link} & \textbf{Assignee} & \textbf{Status} & \textbf{Test Cases}\\ \hline
\href{https://jira.lsstcorp.org/browse/LVV-172}{LVV-172} &
Robert Gruendl & Not Covered &
\begin{tabular}{c}
LVV-T160 \\
\end{tabular}
\\
\hline
\end{longtable}

\textbf{Verification Element Description:} \\
Submit precovery request and compare results with expected values.

Associated element
(\href{https://jira.lsstcorp.org/browse/LVV-9749}{LVV-9749}) satisfies
the minimum number of precovery service connections that must be
supported.

{\footnotesize
\begin{longtable}{p{2.5cm}p{13.5cm}}
\hline
\multicolumn{2}{c}{\textbf{Requirement Details}}\\ \hline
Requirement ID & DMS-REQ-0341 \\ \cdashline{1-2}
Requirement Description &
\begin{minipage}[]{13cm}
\textbf{Specification:} A ``precovery service'' shall be available to
end-users to request precovery for a provided sky location across all
previous visits, making the results available within
\textbf{precoveryServiceElapsed} hours of the request and supporting at
least \textbf{precoveryServicePeakUsers} submissions per hour.
\end{minipage}
\\ \cdashline{1-2}
Requirement Parameters & {[}\textbf{precoveryServiceElapsed = 24{{[}hour{]}}} Maximum time
between submitting a request and receiving the results.,
\textbf{precoveryServicePeakUsers = 10{{[}integer{]}}} Minimum number of
precovery service connections to be supported per hour.{]} \\ \cdashline{1-2}
Requirement Discussion &
\begin{minipage}[]{13cm}
\textbf{Discussion:} This is forced photometry on difference images from
each visit. This will include a web interface and scriptable APIs.
\end{minipage}
\\ \cdashline{1-2}
Requirement Priority & 1b \\ \cdashline{1-2}
Upper Level Requirement &
\begin{tabular}{cl}
OSS-REQ-0126 & Level 1 Data Products \\
\end{tabular}
\\ \hline
\end{longtable}
}


\subsubsection{Test Cases Summary}
\begin{longtable}{p{3cm}p{2.5cm}p{2.5cm}p{3cm}p{4cm}}
\toprule
\href{https://jira.lsstcorp.org/secure/Tests.jspa\#/testCase/LVV-T160}{LVV-T160} & \multicolumn{4}{p{12cm}}{ Verify implementation of Providing a Precovery Service } \\ \hline
\textbf{Owner} & \textbf{Status} & \textbf{Version} & \textbf{Critical Event} & \textbf{Verification Type} \\ \hline
Gregory Dubois-Felsmann & Draft & 1 & false & Test \\ \hline
\end{longtable}
{\scriptsize
\textbf{Objective:}\\
Verify that a technical capability to perform user-directed precovery
analyses on difference images exists and that it is exposed through the
LSST Science Platform. ~Verified by testing against precursor
datasets.\\
(Involves: LSP Portal, MOPS and Forced Photometry)
}
  
 \newpage 
\subsection{[LVV-173] DMS-REQ-0342-V-01: Alert Filtering Service }\label{lvv-173}

\begin{longtable}{cccc}
\hline
\textbf{Jira Link} & \textbf{Assignee} & \textbf{Status} & \textbf{Test Cases}\\ \hline
\href{https://jira.lsstcorp.org/browse/LVV-173}{LVV-173} &
Eric Bellm & Not Covered &
\begin{tabular}{c}
LVV-T112 \\
LVV-T218 \\
\end{tabular}
\\
\hline
\end{longtable}

\textbf{Verification Element Description:} \\
In simulated L1 system, register a simple filter and verify that the
filter triggers for the correct alerts.

{\footnotesize
\begin{longtable}{p{2.5cm}p{13.5cm}}
\hline
\multicolumn{2}{c}{\textbf{Requirement Details}}\\ \hline
Requirement ID & DMS-REQ-0342 \\ \cdashline{1-2}
Requirement Description &
\begin{minipage}[]{13cm}
\textbf{Specification:} A basic, limited capacity, alert filtering
service shall be provided that can be given user defined filters to
reduce the alert stream to manageable levels.
\end{minipage}
\\ \cdashline{1-2}
Requirement Priority & 2 \\ \cdashline{1-2}
Upper Level Requirement &
\begin{tabular}{cl}
LSR-REQ-0025 & Transient Filtering \\
\end{tabular}
\\ \hline
\end{longtable}
}


\subsubsection{Test Cases Summary}
\begin{longtable}{p{3cm}p{2.5cm}p{2.5cm}p{3cm}p{4cm}}
\toprule
\href{https://jira.lsstcorp.org/secure/Tests.jspa\#/testCase/LVV-T112}{LVV-T112} & \multicolumn{4}{p{12cm}}{ Verify implementation of Alert Filtering Service } \\ \hline
\textbf{Owner} & \textbf{Status} & \textbf{Version} & \textbf{Critical Event} & \textbf{Verification Type} \\ \hline
Eric Bellm & Defined & 1 & false & Test \\ \hline
\end{longtable}
{\scriptsize
\textbf{Objective:}\\
Verify that user-defined filters can be used to generate a basic alert
filtering service.
}
\begin{longtable}{p{3cm}p{2.5cm}p{2.5cm}p{3cm}p{4cm}}
\toprule
\href{https://jira.lsstcorp.org/secure/Tests.jspa\#/testCase/LVV-T218}{LVV-T218} & \multicolumn{4}{p{12cm}}{ Simple Filtering of the LSST Alert Stream } \\ \hline
\textbf{Owner} & \textbf{Status} & \textbf{Version} & \textbf{Critical Event} & \textbf{Verification Type} \\ \hline
Eric Bellm & Approved & 1 & false & Test \\ \hline
\end{longtable}
{\scriptsize
\textbf{Objective:}\\
This test will demonstrate the LSST Alert Filtering Service that returns
a subset of alerts from the full stream identified by user-provided
filters.\\[2\baselineskip]Specifically, this will demonstrate that:\\

\begin{itemize}
\tightlist
\item
  The filtering service can retrieve alerts from the full alert stream
  and filter them according to their contents; ~ ~
\item
  The filtered subset can be delivered to science users.
\end{itemize}
}
  
 \newpage 
\subsection{[LVV-174] DMS-REQ-0343-V-01: Number of full-size alerts }\label{lvv-174}

\begin{longtable}{cccc}
\hline
\textbf{Jira Link} & \textbf{Assignee} & \textbf{Status} & \textbf{Test Cases}\\ \hline
\href{https://jira.lsstcorp.org/browse/LVV-174}{LVV-174} &
Eric Bellm & Not Covered &
\begin{tabular}{c}
LVV-T113 \\
LVV-T218 \\
\end{tabular}
\\
\hline
\end{longtable}

\textbf{Verification Element Description:} \\
In simulated L1 system, register numBrokerUsers distinct filter codes
and verify that they receive the expected throughput.

Additional element
(\href{https://jira.lsstcorp.org/browse/LVV-9748}{LVV-9748}) satisfies
the constraint on the number of simultaneous users.

{\footnotesize
\begin{longtable}{p{2.5cm}p{13.5cm}}
\hline
\multicolumn{2}{c}{\textbf{Requirement Details}}\\ \hline
Requirement ID & DMS-REQ-0343 \\ \cdashline{1-2}
Requirement Description &
\begin{minipage}[]{13cm}
\textbf{Specification:} The LSST alert filtering service shall support
\textbf{numBrokerUsers} simultaneous users with each user allocated a
bandwidth capable of receiving the equivalent of
\textbf{numBrokerAlerts} alerts per visit.
\end{minipage}
\\ \cdashline{1-2}
Requirement Parameters & {[}\textbf{numBrokerUsers = 100{{[}integer{]}}} Supported number of
simultaneous users connected to the LSST alert filtering system.,
\textbf{numBrokerAlerts = 20{{[}integer{]}}} Number of full-sized alerts
that can be received per visit per user.{]} \\ \cdashline{1-2}
Requirement Discussion &
\begin{minipage}[]{13cm}
\textbf{Discussion:} The constraint on number of alerts is specified for
the full VOEvent alert content, but could also be satisfied by all
alerts being received with minimal alert content.
\end{minipage}
\\ \cdashline{1-2}
Requirement Priority & 2 \\ \cdashline{1-2}
Upper Level Requirement &
\begin{tabular}{cl}
OSS-REQ-0193 & Alerts per Visit \\
OSS-REQ-0184 & Transient Alert Publication \\
\end{tabular}
\\ \hline
\end{longtable}
}


\subsubsection{Test Cases Summary}
\begin{longtable}{p{3cm}p{2.5cm}p{2.5cm}p{3cm}p{4cm}}
\toprule
\href{https://jira.lsstcorp.org/secure/Tests.jspa\#/testCase/LVV-T113}{LVV-T113} & \multicolumn{4}{p{12cm}}{ Verify implementation of Performance Requirements for LSST Alert
Filtering Service } \\ \hline
\textbf{Owner} & \textbf{Status} & \textbf{Version} & \textbf{Critical Event} & \textbf{Verification Type} \\ \hline
Eric Bellm & Defined & 1 & false & Test \\ \hline
\end{longtable}
{\scriptsize
\textbf{Objective:}\\
Verify that the DMS alert filter service provides sufficient bandwidth
for \textbf{numBrokerUsers = 100} simultaneously-operating brokers to
receive up to \textbf{numBrokerAlerts = 20} alerts per visit.
}
\begin{longtable}{p{3cm}p{2.5cm}p{2.5cm}p{3cm}p{4cm}}
\toprule
\href{https://jira.lsstcorp.org/secure/Tests.jspa\#/testCase/LVV-T218}{LVV-T218} & \multicolumn{4}{p{12cm}}{ Simple Filtering of the LSST Alert Stream } \\ \hline
\textbf{Owner} & \textbf{Status} & \textbf{Version} & \textbf{Critical Event} & \textbf{Verification Type} \\ \hline
Eric Bellm & Approved & 1 & false & Test \\ \hline
\end{longtable}
{\scriptsize
\textbf{Objective:}\\
This test will demonstrate the LSST Alert Filtering Service that returns
a subset of alerts from the full stream identified by user-provided
filters.\\[2\baselineskip]Specifically, this will demonstrate that:\\

\begin{itemize}
\tightlist
\item
  The filtering service can retrieve alerts from the full alert stream
  and filter them according to their contents; ~ ~
\item
  The filtered subset can be delivered to science users.
\end{itemize}
}
  
 \newpage 
\subsection{[LVV-175] DMS-REQ-0004-V-01: Time to L1 public release }\label{lvv-175}

\begin{longtable}{cccc}
\hline
\textbf{Jira Link} & \textbf{Assignee} & \textbf{Status} & \textbf{Test Cases}\\ \hline
\href{https://jira.lsstcorp.org/browse/LVV-175}{LVV-175} &
Melissa Graham & Not Covered &
\begin{tabular}{c}
LVV-T35 \\
LVV-T95 \\
\end{tabular}
\\
\hline
\end{longtable}

\textbf{Verification Element Description:} \\
This is 3 distinct requirements. OTT1 can be tested with simulated data.
L1 Data Products can be created with precursor data but requires that we
include some ``worst case'' datasets (in terms of density and night
length). SSObject orbit determination can be done to a certain extent
with simulated data. Will need to be verified again during
commissioning.

Associated element
(\href{https://jira.lsstcorp.org/browse/LVV-9740}{LVV-9740}) satisfies
the latency of reporting transients.\\
Associated element
(\href{https://jira.lsstcorp.org/browse/LVV-9803}{LVV-9803}) satisfies
the availability of Solar System Object orbits.

Associated element
(\href{https://jira.lsstcorp.org/browse/LVV-9744}{LVV-9744}) satisfies
the latency of reporting optical transients.

{\footnotesize
\begin{longtable}{p{2.5cm}p{13.5cm}}
\hline
\multicolumn{2}{c}{\textbf{Requirement Details}}\\ \hline
Requirement ID & DMS-REQ-0004 \\ \cdashline{1-2}
Requirement Description &
\begin{minipage}[]{13cm}
\textbf{Specification:} With the exception of alerts and Solar System
Objects, all Level 1 Data Products shall be made public within time
\textbf{L1PublicT} of the acquisition of the raw image data.\\
\hspace*{0.333em}\\
LSST shall not release image or catalog data resulting from a visit,
except for the content of the public alert stream, sooner than time
\textbf{L1PublicTMin} following the acquisition of the raw image data
from that visit.\\
\hspace*{0.333em}\\
For visits resulting in fewer than \textbf{nAlertVisitPeak}, LSST shall
be capable of supporting the distribution of at least \textbf{OTR1} per
cent of alerts via the LSST alert distribution system within time
\textbf{OTT1} from the conclusion of the camera's readout of the raw
exposures used to generate each alert. ~\\
\hspace*{0.333em}\\
Solar System Object orbits will, on average, be calculated before the
following night's observing finishes and the results shall be made
available within time \textbf{L1PublicT} of those calculations being
completed.
\end{minipage}
\\ \cdashline{1-2}
Requirement Parameters & {[}\textbf{OTR1 = 98{{[}percent{]}}} Fraction of detectable alerts for
which an alert is actually transmitted within latency OTT1 (see
LSR-REQ-0101)., \textbf{OTT1 = 1{{[}minute{]}}} The latency of reporting
optical transients following the completion of readout of the last image
of a visit, \textbf{nAlertVisitPeak = 40000{{[}integer{]}}} The
instantaneous peak number of alerts per standard visit.,
\textbf{L1PublicTMin = 6{{[}hour{]}}} Time images and other products
(except alerts) will be embargoed before release to the consortium (or
the public), \textbf{L1PublicT = 24{{[}hour{]}}} Maximum time from the
acquisition of science data to the release of associated Level 1 Data
Products (except alerts){]} \\ \cdashline{1-2}
Requirement Discussion &
\begin{minipage}[]{13cm}
\textbf{Discussion:} Because of the processing flow of SSObject orbit
determination, meeting the base
\textbf{L1PublicT}-after-data-acquisition requirement would be far more
challenging than for the other L1 Data Products, but the system
throughput has to be good enough such that a back log can not build up.
\end{minipage}
\\ \cdashline{1-2}
Requirement Priority & 1b \\ \cdashline{1-2}
Upper Level Requirement &
\begin{tabular}{cl}
DMS-REQ-0003 & Create and Maintain Science Data Archive \\
OSS-REQ-0127 & Level 1 Data Product Availability \\
\end{tabular}
\\ \hline
\end{longtable}
}


\subsubsection{Test Cases Summary}
\begin{longtable}{p{3cm}p{2.5cm}p{2.5cm}p{3cm}p{4cm}}
\toprule
\href{https://jira.lsstcorp.org/secure/Tests.jspa\#/testCase/LVV-T35}{LVV-T35} & \multicolumn{4}{p{12cm}}{ Verify implementation of Nightly Data Accessible Within 24 hrs } \\ \hline
\textbf{Owner} & \textbf{Status} & \textbf{Version} & \textbf{Critical Event} & \textbf{Verification Type} \\ \hline
Eric Bellm & Draft & 1 & false & Test \\ \hline
\end{longtable}
{\scriptsize
\textbf{Objective:}\\
\textbf{Test Items}\\[2\baselineskip]Verify that\\
1. Alerts are available within OTT1\\
2. Level 1 Data Products are available within L1PublicT\\
3. Solar System Object orbits are available within L1PublicT of the
updated calculations completion on the following night.
}
\begin{longtable}{p{3cm}p{2.5cm}p{2.5cm}p{3cm}p{4cm}}
\toprule
\href{https://jira.lsstcorp.org/secure/Tests.jspa\#/testCase/LVV-T95}{LVV-T95} & \multicolumn{4}{p{12cm}}{ Verify implementation of Constraints on Level 1 Special Program Products
Generation } \\ \hline
\textbf{Owner} & \textbf{Status} & \textbf{Version} & \textbf{Critical Event} & \textbf{Verification Type} \\ \hline
Melissa Graham & Draft & 1 & false & Test \\ \hline
\end{longtable}
{\scriptsize
\textbf{Objective:}\\
Execute single-day operations rehearsal. Observe Prompt Processing data
products generated in time. Confirm that data from Special Programs is
processed with the same latency as required for main survey data:
release of public data within L1publicT and Alerts within OTT1.
}
  
 \newpage 
\subsection{[LVV-177] DMS-REQ-0346-V-01: Data Availability }\label{lvv-177}

\begin{longtable}{cccc}
\hline
\textbf{Jira Link} & \textbf{Assignee} & \textbf{Status} & \textbf{Test Cases}\\ \hline
\href{https://jira.lsstcorp.org/browse/LVV-177}{LVV-177} &
Gregory Dubois-Felsmann & Not Covered &
\begin{tabular}{c}
LVV-T27 \\
LVV-T286 \\
\end{tabular}
\\
\hline
\end{longtable}

\textbf{Verification Element Description:} \\
Retrieve a coadd. Query its provenance and retrieve all the information
required to recreate that coadd locally. In theory we could then
rereduce the data and compare it to the original coadd.

{\footnotesize
\begin{longtable}{p{2.5cm}p{13.5cm}}
\hline
\multicolumn{2}{c}{\textbf{Requirement Details}}\\ \hline
Requirement ID & DMS-REQ-0346 \\ \cdashline{1-2}
Requirement Description &
\begin{minipage}[]{13cm}
\textbf{Specification:} All raw data used to generate any public data
product (raw exposures, calibration frames, telemetry, configuration
metadata, etc.) shall be kept and made available for download.
\end{minipage}
\\ \cdashline{1-2}
Requirement Priority & 1b \\ \cdashline{1-2}
Upper Level Requirement &
\begin{tabular}{cl}
OSS-REQ-0004 & The Archive Facility \\
OSS-REQ-0167 & Data Archiving \\
OSS-REQ-0313 & Telemetry Database Retention \\
\end{tabular}
\\ \hline
\end{longtable}
}


\subsubsection{Test Cases Summary}
\begin{longtable}{p{3cm}p{2.5cm}p{2.5cm}p{3cm}p{4cm}}
\toprule
\href{https://jira.lsstcorp.org/secure/Tests.jspa\#/testCase/LVV-T27}{LVV-T27} & \multicolumn{4}{p{12cm}}{ Verify implementation of Data Availability } \\ \hline
\textbf{Owner} & \textbf{Status} & \textbf{Version} & \textbf{Critical Event} & \textbf{Verification Type} \\ \hline
Gregory Dubois-Felsmann & Draft & 1 & false & Test \\ \hline
\end{longtable}
{\scriptsize
\textbf{Objective:}\\
Determine if all required categories of raw data (specifically
enumerated: raw exposures, calibration frames, telemetry, configuration
metadata) can be located through the Science Platform and are available
for download. ~Verify through (1) administrative review; (2) checking
with precursor data; (3) checking on early data feeds from the Summit
such as from AuxTel and ComCam.
}
\begin{longtable}{p{3cm}p{2.5cm}p{2.5cm}p{3cm}p{4cm}}
\toprule
\href{https://jira.lsstcorp.org/secure/Tests.jspa\#/testCase/LVV-T286}{LVV-T286} & \multicolumn{4}{p{12cm}}{ RAS-00-20: Raw image are part of the permanent record of survey via DBB } \\ \hline
\textbf{Owner} & \textbf{Status} & \textbf{Version} & \textbf{Critical Event} & \textbf{Verification Type} \\ \hline
Michelle Butler & Approved & 1 & false & Test \\ \hline
\end{longtable}
{\scriptsize
\textbf{Objective:}\\
This test will check:\\[2\baselineskip]

\begin{itemize}
\tightlist
\item
  That the handoff of a raw image from the Level 1 Archiver Service to
  the DBB buffer manager is successful;
\item
  That the raw image is ingested into the Data Backbone successfully;
\item
  That the monitoring of the above items is successful;
\end{itemize}

This Test Case shall be repeated for each of the different cameras
(ATScam, LSSTCam) and sensors (Science, Wavefront, and Guider)
combination.\\[2\baselineskip]Note: For a complete check of the various
aspects of what it means for a raw image to be in the Data Backbone, see
the tests for the Data Backbone.
}
  
 \newpage 
\subsection{[LVV-184] DMS-REQ-0353-V-01: Publishing predicted visit schedule }\label{lvv-184}

\begin{longtable}{cccc}
\hline
\textbf{Jira Link} & \textbf{Assignee} & \textbf{Status} & \textbf{Test Cases}\\ \hline
\href{https://jira.lsstcorp.org/browse/LVV-184}{LVV-184} &
Colin Slater & Not Covered &
\begin{tabular}{c}
LVV-T60 \\
\end{tabular}
\\
\hline
\end{longtable}

\textbf{Verification Element Description:} \\
Use simulated schedule and test that an external unauthenticated user
can retrieve the information.

{\footnotesize
\begin{longtable}{p{2.5cm}p{13.5cm}}
\hline
\multicolumn{2}{c}{\textbf{Requirement Details}}\\ \hline
Requirement ID & DMS-REQ-0353 \\ \cdashline{1-2}
Requirement Description &
\begin{minipage}[]{13cm}
\textbf{Specification:} A service shall be provided to publish to the
community the next visit location and the predicted visit schedule
provided by the OCS. This service shall consist of both a web page for
human inspection and a web API to allow automated tools to respond
promptly.
\end{minipage}
\\ \cdashline{1-2}
Requirement Discussion &
\begin{minipage}[]{13cm}
\textbf{Discussion:} The next visit and advanced schedule do not need to
be published using the same service or protocol.
\end{minipage}
\\ \cdashline{1-2}
Requirement Priority & 1b \\ \cdashline{1-2}
Upper Level Requirement &
\begin{tabular}{cl}
OSS-REQ-0378 & Advanced Publishing of Scheduler Sequence \\
\end{tabular}
\\ \hline
\end{longtable}
}


\subsubsection{Test Cases Summary}
\begin{longtable}{p{3cm}p{2.5cm}p{2.5cm}p{3cm}p{4cm}}
\toprule
\href{https://jira.lsstcorp.org/secure/Tests.jspa\#/testCase/LVV-T60}{LVV-T60} & \multicolumn{4}{p{12cm}}{ Verify implementation of Publishing predicted visit schedule } \\ \hline
\textbf{Owner} & \textbf{Status} & \textbf{Version} & \textbf{Critical Event} & \textbf{Verification Type} \\ \hline
Eric Bellm & Draft & 1 & false & Test \\ \hline
\end{longtable}
{\scriptsize
\textbf{Objective:}\\
Verify that a predict-visit schedule can be published by the OCS.
}
  
 \newpage 
\subsection{[LVV-186] DMS-REQ-0355-V-01: Max time to retrieve Prompt Products Database query
results }\label{lvv-186}

\begin{longtable}{cccc}
\hline
\textbf{Jira Link} & \textbf{Assignee} & \textbf{Status} & \textbf{Test Cases}\\ \hline
\href{https://jira.lsstcorp.org/browse/LVV-186}{LVV-186} &
Eric Bellm & Not Covered &
\begin{tabular}{c}
\end{tabular}
\\
\hline
\end{longtable}

\textbf{Verification Element Description:} \\
Prompt Products Database query results shall be retrievable in a maximum
time of \textbf{l1QueryTime = 10Â~seconds.}

The associated element DMS-REQ-0355-V-02
(\href{https://jira.lsstcorp.org/browse/LVV-9784}{LVV-9784}) satisfies
the additional constraint on the number of simultaneous users.

\emph{These requirements should be satisfied together.}

{\footnotesize
\begin{longtable}{p{2.5cm}p{13.5cm}}
\hline
\multicolumn{2}{c}{\textbf{Requirement Details}}\\ \hline
Requirement ID & DMS-REQ-0355 \\ \cdashline{1-2}
Requirement Description &
\begin{minipage}[]{13cm}
\textbf{Specification:} The live Prompt Products Database shall support
at least \textbf{l1QueryUsers} simultaneous queries, assuming each query
lasts no more than \textbf{l1QueryTime}.
\end{minipage}
\\ \cdashline{1-2}
Requirement Parameters & {[}\textbf{l1QueryTime = 10{{[}second{]}}} Maximum time allowed for
retrieving results of a query of the Prompt Products Database.,
\textbf{l1QueryUsers = 20{{[}integer{]}}} Minimum number of simultaneous
users querying the Prompt Products Database.{]} \\ \cdashline{1-2}
Requirement Priority & 1b \\ \cdashline{1-2}
Upper Level Requirement &
\begin{tabular}{cl}
OSS-REQ-0181 & Data Products Query and Download Infrastructure \\
\end{tabular}
\\ \hline
\end{longtable}
}


  
 \newpage 
\subsection{[LVV-187] DMS-REQ-0356-V-01: Radius for low-volume query }\label{lvv-187}

\begin{longtable}{cccc}
\hline
\textbf{Jira Link} & \textbf{Assignee} & \textbf{Status} & \textbf{Test Cases}\\ \hline
\href{https://jira.lsstcorp.org/browse/LVV-187}{LVV-187} &
Robert Gruendl & Not Covered &
\begin{tabular}{c}
\end{tabular}
\\
\hline
\end{longtable}

\textbf{Verification Element Description:} \\
Low volume queries shall use a radius of~\textbf{lvSkyRadius =
60~arcseconds}~on the sky.

The associated element
DMS-REQ-0356-V-02~(\href{https://jira.lsstcorp.org/browse/LVV-9785}{LVV-9785})~satisfies
the additional constraint on the maximum size of low volume queries.

The associated element
DMS-REQ-0356-V-03~(\href{https://jira.lsstcorp.org/browse/LVV-9786}{LVV-9786})~satisfies
the additional constraint on the number of simultaneous users.

The associated element DMS-REQ-0356-V-04
\href{https://jira.lsstcorp.org/browse/LVV-9787}{(LVV-9787)} satisfies
the additional constraint on the maximum time to return low volume query
results.

\emph{These requirements should be satisfied together.}

{\footnotesize
\begin{longtable}{p{2.5cm}p{13.5cm}}
\hline
\multicolumn{2}{c}{\textbf{Requirement Details}}\\ \hline
Requirement ID & DMS-REQ-0356 \\ \cdashline{1-2}
Requirement Description &
\begin{minipage}[]{13cm}
\textbf{Specification:} Low volume queries, queries that are spatially
restricted to a circle of radius~\textbf{lvSkyRadius} and return at most
\textbf{lvMaxReturnedResults} of data, shall respond within
\textbf{lvQueryTime} under a load of \textbf{lvQueryUsers} simultaneous
queries.
\end{minipage}
\\ \cdashline{1-2}
Requirement Parameters & {[}\textbf{lvSkyRadius = 60{{[}arcsecond{]}}} Radius to be used for a
low-volume query on the sky., \textbf{lvMaxReturnedResults =
0.5{{[}gigabyte{]}}} Maximum size of a results set for a query to be
defined to be ``low-volume''., \textbf{lvQueryUsers =
100{{[}integer{]}}} Minimum number of simultaneous users performing low
volume queries., \textbf{lvQueryTime = 10{{[}second{]}}} Maximum time
allowed for retrieving results of a low-volume query.{]} \\ \cdashline{1-2}
Requirement Discussion &
\begin{minipage}[]{13cm}
\textbf{Discussion:} We are evaluating whether the latency requirements
of low-volume queries can also be met for certain categories of temporal
queries or queries on indexed attributes which limit the scope of
per-row operations in the query (such as non-indexed WHERE evaluations)
to a comparable fraction of the total dataset. The low-volume query
requirements also apply to queries selecting data by the primary key of
any data product table, or by the associated Object-like primary key for
the ForcedSource and DIASource tables.
\end{minipage}
\\ \cdashline{1-2}
Requirement Priority & 1b \\ \cdashline{1-2}
Upper Level Requirement &
\begin{tabular}{cl}
OSS-REQ-0181 & Data Products Query and Download Infrastructure \\
\end{tabular}
\\ \hline
\end{longtable}
}


  
 \newpage 
\subsection{[LVV-190] DMS-REQ-0364-V-01: Total number of data releases }\label{lvv-190}

\begin{longtable}{cccc}
\hline
\textbf{Jira Link} & \textbf{Assignee} & \textbf{Status} & \textbf{Test Cases}\\ \hline
\href{https://jira.lsstcorp.org/browse/LVV-190}{LVV-190} &
Colin Slater & Not Covered &
\begin{tabular}{c}
LVV-T163 \\
\end{tabular}
\\
\hline
\end{longtable}

\textbf{Verification Element Description:} \\
There shall be at least \textbf{nDRTot = 11{{[}integer{]}}} data
releases over the course of the survey.

Associated element
(\href{https://jira.lsstcorp.org/browse/LVV-9750}{LVV-9750}) addresses
the length of the planned survey.

{\footnotesize
\begin{longtable}{p{2.5cm}p{13.5cm}}
\hline
\multicolumn{2}{c}{\textbf{Requirement Details}}\\ \hline
Requirement ID & DMS-REQ-0364 \\ \cdashline{1-2}
Requirement Description &
\begin{minipage}[]{13cm}
\textbf{Specification:} The data access services shall be designed to
permit, and their software implementation shall support, the service of
at least \textbf{nDRTot} Data Releases accumulated over the (find the
actual survey-length parameter) \textbf{surveyYears}-year planned
survey.
\end{minipage}
\\ \cdashline{1-2}
Requirement Parameters & {[}\textbf{nDRTot = 11{{[}integer{]}}} Total number of data releases
over the survey., \textbf{surveyYears = 10{{[}integer{]}}} Length of the
survey in years{]} \\ \cdashline{1-2}
Requirement Discussion &
\begin{minipage}[]{13cm}
\textbf{Discussion:} It is an operations-era decision to choose the
actual number of releases to be served, and to allocate hardware
resources accordingly. ~The requirement is that the system delivered at
the close of the MREFC construction period be capable of handling ten
years of releases if the operations project chooses to allocate adequate
hardware resources.
\end{minipage}
\\ \cdashline{1-2}
Requirement Priority & 3 \\ \cdashline{1-2}
Upper Level Requirement &
\begin{tabular}{cl}
OSS-REQ-0396 & Data Access Services \\
\end{tabular}
\\ \hline
\end{longtable}
}


\subsubsection{Test Cases Summary}
\begin{longtable}{p{3cm}p{2.5cm}p{2.5cm}p{3cm}p{4cm}}
\toprule
\href{https://jira.lsstcorp.org/secure/Tests.jspa\#/testCase/LVV-T163}{LVV-T163} & \multicolumn{4}{p{12cm}}{ Verify implementation of Data Access Services } \\ \hline
\textbf{Owner} & \textbf{Status} & \textbf{Version} & \textbf{Critical Event} & \textbf{Verification Type} \\ \hline
Robert Gruendl & Draft & 1 & false & Test \\ \hline
\end{longtable}
{\scriptsize
\textbf{Objective:}\\
Demonstrate that Data Access Services are capable of scaling to serve
data from nDRTot (11) data releases over a surveyYears (10) year survey.
}
  
 \newpage 
\subsection{[LVV-191] DMS-REQ-0365-V-01: Operations Subsets }\label{lvv-191}

\begin{longtable}{cccc}
\hline
\textbf{Jira Link} & \textbf{Assignee} & \textbf{Status} & \textbf{Test Cases}\\ \hline
\href{https://jira.lsstcorp.org/browse/LVV-191}{LVV-191} &
Colin Slater & Not Covered &
\begin{tabular}{c}
LVV-T164 \\
\end{tabular}
\\
\hline
\end{longtable}

\textbf{Verification Element Description:} \\
tbc

{\footnotesize
\begin{longtable}{p{2.5cm}p{13.5cm}}
\hline
\multicolumn{2}{c}{\textbf{Requirement Details}}\\ \hline
Requirement ID & DMS-REQ-0365 \\ \cdashline{1-2}
Requirement Description &
\begin{minipage}[]{13cm}
\textbf{Specification:} The data access services shall be designed to
permit the service of operations-designated subsets of the full content
of the ``older Data Releases'' referred to in DMS-REQ-0363.
\end{minipage}
\\ \cdashline{1-2}
Requirement Discussion &
\begin{minipage}[]{13cm}
\textbf{Discussion:} This requirement, and the following one, are
intended to give the operations project flexibility in, for example,
serving only catalogs, and not images, from older releases.
\end{minipage}
\\ \cdashline{1-2}
Requirement Priority & 2 \\ \cdashline{1-2}
Upper Level Requirement &
\begin{tabular}{cl}
OSS-REQ-0398 & Operations Subsets \\
\end{tabular}
\\ \hline
\end{longtable}
}


\subsubsection{Test Cases Summary}
\begin{longtable}{p{3cm}p{2.5cm}p{2.5cm}p{3cm}p{4cm}}
\toprule
\href{https://jira.lsstcorp.org/secure/Tests.jspa\#/testCase/LVV-T164}{LVV-T164} & \multicolumn{4}{p{12cm}}{ Verify implementation of Operations Subsets } \\ \hline
\textbf{Owner} & \textbf{Status} & \textbf{Version} & \textbf{Critical Event} & \textbf{Verification Type} \\ \hline
Robert Gruendl & Draft & 1 & false & Test \\ \hline
\end{longtable}
{\scriptsize
\textbf{Objective:}\\
Demonstrate that Data Access Services are designed such that subsets of
a Data Release may be retained and served (made available) after a Data
Release has been superseded. ~ (Data Backbone, Managed Database, LSP
Portal, LSP JupyterLab, LSP Web APIs, Parallel Distributed Database)
}
  
 \newpage 
\subsection{[LVV-192] DMS-REQ-0366-V-01: Subsets Support }\label{lvv-192}

\begin{longtable}{cccc}
\hline
\textbf{Jira Link} & \textbf{Assignee} & \textbf{Status} & \textbf{Test Cases}\\ \hline
\href{https://jira.lsstcorp.org/browse/LVV-192}{LVV-192} &
Colin Slater & Not Covered &
\begin{tabular}{c}
LVV-T165 \\
\end{tabular}
\\
\hline
\end{longtable}

\textbf{Verification Element Description:} \\
tbc

{\footnotesize
\begin{longtable}{p{2.5cm}p{13.5cm}}
\hline
\multicolumn{2}{c}{\textbf{Requirement Details}}\\ \hline
Requirement ID & DMS-REQ-0366 \\ \cdashline{1-2}
Requirement Description &
\begin{minipage}[]{13cm}
\textbf{Specification:} The data access services shall be designed to
support the service of operations-designated subsets of the content of
the ``older Data Releases'' referred to in requirement DMS-REQ-0363 from
high-latency media.
\end{minipage}
\\ \cdashline{1-2}
Requirement Discussion &
\begin{minipage}[]{13cm}
\textbf{Discussion:} This means that the ``toolkit'' of data access
services should include elements that, for instance, allow users to
understand that certain queries (e.g., for data on tape) may take much
longer than for current data releases, and to monitor the status of such
queries.
\end{minipage}
\\ \cdashline{1-2}
Requirement Priority & 2 \\ \cdashline{1-2}
Upper Level Requirement &
\begin{tabular}{cl}
OSS-REQ-0400 & Subsets Support \\
\end{tabular}
\\ \hline
\end{longtable}
}


\subsubsection{Test Cases Summary}
\begin{longtable}{p{3cm}p{2.5cm}p{2.5cm}p{3cm}p{4cm}}
\toprule
\href{https://jira.lsstcorp.org/secure/Tests.jspa\#/testCase/LVV-T165}{LVV-T165} & \multicolumn{4}{p{12cm}}{ Verify implementation of Subsets Support } \\ \hline
\textbf{Owner} & \textbf{Status} & \textbf{Version} & \textbf{Critical Event} & \textbf{Verification Type} \\ \hline
Robert Lupton & Draft & 1 & false & Test \\ \hline
\end{longtable}
{\scriptsize
\textbf{Objective:}\\
Verify that the DMS can provide designated subsets of previous Data
Releases.
}
  
 \newpage 
\subsection{[LVV-194] DMS-REQ-0368-V-01: Implementation Provisions }\label{lvv-194}

\begin{longtable}{cccc}
\hline
\textbf{Jira Link} & \textbf{Assignee} & \textbf{Status} & \textbf{Test Cases}\\ \hline
\href{https://jira.lsstcorp.org/browse/LVV-194}{LVV-194} &
Robert Gruendl & Not Covered &
\begin{tabular}{c}
LVV-T167 \\
\end{tabular}
\\
\hline
\end{longtable}

\textbf{Verification Element Description:} \\
tbc

{\footnotesize
\begin{longtable}{p{2.5cm}p{13.5cm}}
\hline
\multicolumn{2}{c}{\textbf{Requirement Details}}\\ \hline
Requirement ID & DMS-REQ-0368 \\ \cdashline{1-2}
Requirement Description &
\begin{minipage}[]{13cm}
\textbf{Specification:} Nothing in the design and software
implementation of the data access services shall prevent the performance
requirements set forth in OSS-REQ-0180 and OSS-REQ-0181 from being met
for the ``older Data Releases'' referred to in DMS-REQ-0363, subject to
the provision of sufficient computing and storage resources in the
operations era.
\end{minipage}
\\ \cdashline{1-2}
Requirement Parameters & \textbf{nDRMin = 2{{[}integer{]}}} Minimum number of recent data
releases \\ \cdashline{1-2}
Requirement Discussion &
\begin{minipage}[]{13cm}
\textbf{Discussion:} It is left to the operations project to set
standards for the performance on older releases, but they should not be
limited by design choices made in the construction era.~ That is, the
system must be scalable to handle full-performance service of all Data
Releases, should the operations project so choose.~ This situation does
not arise until, at the release of Data Release (\textbf{nDRMin}+1), the
operations project must decide on the level of service to be provided
for Data Release 1.\\
This requirement may be verified by analysis, e.g., by expert review of
the design of the data access services, as it is recognized that it may
be very difficult to perform live performance measurements relevant to
the scalability of the data access services across a decade.\\
While the system is required to be scalable to full performance, it is
likely that, for the optimal allocation of limited operations-era
resources, performance parameters such as the number of concurrently
running queries or image requests may be reduced for products from older
data releases.
\end{minipage}
\\ \cdashline{1-2}
Requirement Priority & 3 \\ \cdashline{1-2}
Upper Level Requirement &
\begin{tabular}{cl}
OSS-REQ-0399 & Implementation Provisions \\
\end{tabular}
\\ \hline
\end{longtable}
}


\subsubsection{Test Cases Summary}
\begin{longtable}{p{3cm}p{2.5cm}p{2.5cm}p{3cm}p{4cm}}
\toprule
\href{https://jira.lsstcorp.org/secure/Tests.jspa\#/testCase/LVV-T167}{LVV-T167} & \multicolumn{4}{p{12cm}}{ Verify Capability to serve older Data Releases at Full Performance } \\ \hline
\textbf{Owner} & \textbf{Status} & \textbf{Version} & \textbf{Critical Event} & \textbf{Verification Type} \\ \hline
Robert Gruendl & Draft & 1 & false & Test \\ \hline
\end{longtable}
{\scriptsize
\textbf{Objective:}\\
Verify that implementation of the data access services do not preclude
serving all older Data Releases with the same performance requirements
as current Data Releases. ~Note that it is an operational consideration
whether sufficient compute and storage resources would actually be
provisioned to meet those requirements.
}
  
 \newpage 
\subsection{[LVV-195] DMS-REQ-0369-V-01: Evolution }\label{lvv-195}

\begin{longtable}{cccc}
\hline
\textbf{Jira Link} & \textbf{Assignee} & \textbf{Status} & \textbf{Test Cases}\\ \hline
\href{https://jira.lsstcorp.org/browse/LVV-195}{LVV-195} &
Colin Slater & Not Covered &
\begin{tabular}{c}
LVV-T168 \\
\end{tabular}
\\
\hline
\end{longtable}

\textbf{Verification Element Description:} \\
tbc

{\footnotesize
\begin{longtable}{p{2.5cm}p{13.5cm}}
\hline
\multicolumn{2}{c}{\textbf{Requirement Details}}\\ \hline
Requirement ID & DMS-REQ-0369 \\ \cdashline{1-2}
Requirement Description &
\begin{minipage}[]{13cm}
\textbf{Specification:} The data access services shall be designed to
accommodate evolution of the LSST data model from Data Release to Data
Release.
\end{minipage}
\\ \cdashline{1-2}
Requirement Priority & 1b \\ \cdashline{1-2}
Upper Level Requirement &
\begin{tabular}{cl}
OSS-REQ-0395 & Evolution \\
\end{tabular}
\\ \hline
\end{longtable}
}


\subsubsection{Test Cases Summary}
\begin{longtable}{p{3cm}p{2.5cm}p{2.5cm}p{3cm}p{4cm}}
\toprule
\href{https://jira.lsstcorp.org/secure/Tests.jspa\#/testCase/LVV-T168}{LVV-T168} & \multicolumn{4}{p{12cm}}{ Verify design of Data Access Services allows Evolution of the LSST Data
Model } \\ \hline
\textbf{Owner} & \textbf{Status} & \textbf{Version} & \textbf{Critical Event} & \textbf{Verification Type} \\ \hline
Robert Gruendl & Draft & 1 & false & Test \\ \hline
\end{longtable}
{\scriptsize
\textbf{Objective:}\\
Verify that the design of the Data Access Services are able to
accommodate changes/evolution of the LSST data model from one release to
another.
}
  
 \newpage 
\subsection{[LVV-196] DMS-REQ-0370-V-01: Older Release Behavior }\label{lvv-196}

\begin{longtable}{cccc}
\hline
\textbf{Jira Link} & \textbf{Assignee} & \textbf{Status} & \textbf{Test Cases}\\ \hline
\href{https://jira.lsstcorp.org/browse/LVV-196}{LVV-196} &
Colin Slater & Not Covered &
\begin{tabular}{c}
LVV-T169 \\
\end{tabular}
\\
\hline
\end{longtable}

\textbf{Verification Element Description:} \\
tbc

{\footnotesize
\begin{longtable}{p{2.5cm}p{13.5cm}}
\hline
\multicolumn{2}{c}{\textbf{Requirement Details}}\\ \hline
Requirement ID & DMS-REQ-0370 \\ \cdashline{1-2}
Requirement Description &
\begin{minipage}[]{13cm}
\textbf{Specification:} Apart from the flexibility provided by
requirements DMS-REQ-0365, DMS-REQ-0366, DMS-REQ-0368, and DMS-REQ-0369,
the qualitative behavior of the data access services on the ``older Data
Releases'' defined in DMS-REQ-0363 shall match that for the most recent
\textbf{nDRMin} Data Releases.
\end{minipage}
\\ \cdashline{1-2}
Requirement Parameters & \textbf{nDRMin = 2{{[}integer{]}}} Minimum number of recent data
releases \\ \cdashline{1-2}
Requirement Discussion &
\begin{minipage}[]{13cm}
\textbf{Discussion:} Essentially, the data access services should
present the same APIs and user interfaces for all Data Releases except
where a difference is required by a change in the data model or, e.g.,
by changes in UI that may be required to provide an acceptable interface
for high-latency data service.
\end{minipage}
\\ \cdashline{1-2}
Requirement Priority & 3 \\ \cdashline{1-2}
Upper Level Requirement &
\begin{tabular}{cl}
OSS-REQ-0397 & Older Release Behavior \\
\end{tabular}
\\ \hline
\end{longtable}
}


\subsubsection{Test Cases Summary}
\begin{longtable}{p{3cm}p{2.5cm}p{2.5cm}p{3cm}p{4cm}}
\toprule
\href{https://jira.lsstcorp.org/secure/Tests.jspa\#/testCase/LVV-T169}{LVV-T169} & \multicolumn{4}{p{12cm}}{ Verify implementation of Older Release Behavior } \\ \hline
\textbf{Owner} & \textbf{Status} & \textbf{Version} & \textbf{Critical Event} & \textbf{Verification Type} \\ \hline
Gregory Dubois-Felsmann & Draft & 1 & false & Test \\ \hline
\end{longtable}
{\scriptsize
\textbf{Objective:}\\
Verify that the components of the data access system are technically
capable of handling data releases beyond the two for which full services
are required. ~DMS-REQ-0364 requires that up to 11 be supported.
~Verified by inspection, i.e., by determination that the system design
and implementation contain the necessary features to support this number
of releases, and by direct test in a synthetic test environment with
multiple releases.\\
(Involves: Data Backbone, Managed Database, LSP Portal, LSP JupyterLab,
LSP Web APIs, Parallel Distributed Database)
}
  
 \newpage 
\subsection{[LVV-197] DMS-REQ-0371-V-01: Query Availability }\label{lvv-197}

\begin{longtable}{cccc}
\hline
\textbf{Jira Link} & \textbf{Assignee} & \textbf{Status} & \textbf{Test Cases}\\ \hline
\href{https://jira.lsstcorp.org/browse/LVV-197}{LVV-197} &
Colin Slater & Not Covered &
\begin{tabular}{c}
LVV-T170 \\
\end{tabular}
\\
\hline
\end{longtable}

\textbf{Verification Element Description:} \\
tbc

{\footnotesize
\begin{longtable}{p{2.5cm}p{13.5cm}}
\hline
\multicolumn{2}{c}{\textbf{Requirement Details}}\\ \hline
Requirement ID & DMS-REQ-0371 \\ \cdashline{1-2}
Requirement Description &
\begin{minipage}[]{13cm}
(\textbf{Goal}) A query (e.g., in ADQL) written against a particular
Data Release SHOULD continue to be executable against the original Data
Release for as long as it is available in the system, with few, if any,
modifications.
\end{minipage}
\\ \cdashline{1-2}
Requirement Discussion &
\begin{minipage}[]{13cm}
\textbf{Discussion:} This is not a full ``shall'' requirement because
there may be constraints imposed by, e.g., the evolution of security
models, that do not permit all existing services to be retained
unchanged indefinitely. The construction project should attempt to
design interfaces that are resilient to reasonably anticipatable
changes, and the operations project should attempt to preserve backwards
compatibility where feasible.\\
Note that, in comparison, it clearly cannot be guaranteed that queries
developed for earlier Data Releases will be usable unchanged against
newer Data Releases (see also requirement OSS-REQ-0395). Users must
anticipate that the evolution of the LSST pipelines will lead to changes
in the Data Release schemas, though the Project will endeavor to avoid
unnecessary changes.
\end{minipage}
\\ \cdashline{1-2}
Requirement Priority & 3 \\ \cdashline{1-2}
Upper Level Requirement &
\begin{tabular}{cl}
OSS-REQ-0401 & Query Availability \\
\end{tabular}
\\ \hline
\end{longtable}
}


\subsubsection{Test Cases Summary}
\begin{longtable}{p{3cm}p{2.5cm}p{2.5cm}p{3cm}p{4cm}}
\toprule
\href{https://jira.lsstcorp.org/secure/Tests.jspa\#/testCase/LVV-T170}{LVV-T170} & \multicolumn{4}{p{12cm}}{ Verify implementation of Query Availability } \\ \hline
\textbf{Owner} & \textbf{Status} & \textbf{Version} & \textbf{Critical Event} & \textbf{Verification Type} \\ \hline
Colin Slater & Draft & 1 & false & Test \\ \hline
\end{longtable}
{\scriptsize
\textbf{Objective:}\\
Verify that queries continue to be successfully executable over time.
}
  
 \newpage 
\subsection{[LVV-3394] DMS-REQ-0377-V-01: Min number of simultaneous single-CCD coadd cutout
image users }\label{lvv-3394}

\begin{longtable}{cccc}
\hline
\textbf{Jira Link} & \textbf{Assignee} & \textbf{Status} & \textbf{Test Cases}\\ \hline
\href{https://jira.lsstcorp.org/browse/LVV-3394}{LVV-3394} &
Leanne Guy & Not Covered &
\begin{tabular}{c}
LVV-T385 \\
\end{tabular}
\\
\hline
\end{longtable}

\textbf{Verification Element Description:} \\
Minimum number of simultaneous users retrieving a single CCD-sized coadd
cutout must be at least~\textbf{ccdRetrievalUsers = 20.}~The associated
element DMS-REQ-0377-V-02
\href{https://jira.lsstcorp.org/browse/LVV-9797}{(LVV-9797)}~satisfies
the additional time constraint.

\emph{These requirements should be satisfied together.}

{\footnotesize
\begin{longtable}{p{2.5cm}p{13.5cm}}
\hline
\multicolumn{2}{c}{\textbf{Requirement Details}}\\ \hline
Requirement ID & DMS-REQ-0377 \\ \cdashline{1-2}
Requirement Description &
\begin{minipage}[]{13cm}
\textbf{Specification:} A CCD-sized cutout of a coadd, including mask
and variance planes, shall be retrievable using the IVOA SODA protocol
within \textbf{ccdRetrievalTime} with \textbf{ccdRetrievalUsers}
simultaneous requests for distinct areas of the sky.
\end{minipage}
\\ \cdashline{1-2}
Requirement Parameters & {[}\textbf{ccdRetrievalTime = 15{{[}second{]}}} Maximum time allowed for
retrieving a CCD-sized coadd cutout., \textbf{ccdRetrievalUsers =
20{{[}integer{]}}} Minimum number of simultaneous users retrieving a
single CCD-sized coadd cutout.{]} \\ \cdashline{1-2}
Requirement Priority & 1b \\ \cdashline{1-2}
Upper Level Requirement &
\begin{tabular}{cl}
OSS-REQ-0181 & Data Products Query and Download Infrastructure \\
\end{tabular}
\\ \hline
\end{longtable}
}


\subsubsection{Test Cases Summary}
\begin{longtable}{p{3cm}p{2.5cm}p{2.5cm}p{3cm}p{4cm}}
\toprule
\href{https://jira.lsstcorp.org/secure/Tests.jspa\#/testCase/LVV-T385}{LVV-T385} & \multicolumn{4}{p{12cm}}{ Verify implementation of minimum number of simultaneous retrievals of
CCD-sized coadd cutouts } \\ \hline
\textbf{Owner} & \textbf{Status} & \textbf{Version} & \textbf{Critical Event} & \textbf{Verification Type} \\ \hline
Leanne Guy & Defined & 1 & false & Test \\ \hline
\end{longtable}
{\scriptsize
\textbf{Objective:}\\
Verify that at least \textbf{ccdRetrievalUsers = 20~}users can
simultaneously retrieve a single CCD-sized coadd cutout using the IVOA
SODA protocol.~
}
  
 \newpage 
\subsection{[LVV-3395] DMS-REQ-0374-V-01: Max time to retrieve single-CCD, single-visit PVI
image }\label{lvv-3395}

\begin{longtable}{cccc}
\hline
\textbf{Jira Link} & \textbf{Assignee} & \textbf{Status} & \textbf{Test Cases}\\ \hline
\href{https://jira.lsstcorp.org/browse/LVV-3395}{LVV-3395} &
Leanne Guy & Not Covered &
\begin{tabular}{c}
\end{tabular}
\\
\hline
\end{longtable}

\textbf{Verification Element Description:} \\
PVIs of a single CCD image shall be retrievable in a maximum time
of~\textbf{pviRetrievalTime = 10~seconds.}

The associated element DMS-REQ-0374-V-02
(\href{https://jira.lsstcorp.org/browse/LVV-9790}{LVV-9790})~satisfies
the additional constraint on the number of simultaneous users.

Associated element~DMS-REQ-0374-V-03
(\href{https://jira.lsstcorp.org/browse/LVV-9791}{LVV-9791}) satisfies
the expected lifetime of Level-1 data products.

\emph{These requirements should be satisfied together.}

{\footnotesize
\begin{longtable}{p{2.5cm}p{13.5cm}}
\hline
\multicolumn{2}{c}{\textbf{Requirement Details}}\\ \hline
Requirement ID & DMS-REQ-0374 \\ \cdashline{1-2}
Requirement Description &
\begin{minipage}[]{13cm}
\textbf{Specification:} A Processed Visit Image of a single CCD shall be
retrievable using the VO SIAv2 protocol within \textbf{pviRetrievalTime}
with \textbf{pviRetrievalUsers} simultaneous requests for distinct
single-CCD PVIs.
\end{minipage}
\\ \cdashline{1-2}
Requirement Parameters & {[}\textbf{pviRetrievalTime = 10{{[}second{]}}} Maximum time allowed for
retrieving a PVI image of a single CCD from a single visit,
\textbf{l1CacheLifetime = 30{{[}day{]}}} Lifetime in the cache of
un-archived Level-1 data products., \textbf{pviRetrievalUsers =
20{{[}integer{]}}} Minimum number of simultaneous users retrieving a
single PVI image.{]} \\ \cdashline{1-2}
Requirement Discussion &
\begin{minipage}[]{13cm}
\textbf{Discussion:} The performance targets for this requirement assume
the PVIs are available as files on a file system. For example, this
could be those files present in the \textbf{l1CacheLifetime} cache.
\end{minipage}
\\ \cdashline{1-2}
Requirement Priority & 1b \\ \cdashline{1-2}
Upper Level Requirement &
\begin{tabular}{cl}
OSS-REQ-0181 & Data Products Query and Download Infrastructure \\
\end{tabular}
\\ \hline
\end{longtable}
}


  
 \newpage 
\subsection{[LVV-3396] DMS-REQ-0376-V-01: Max time to retrieve all PVI images for single visit }\label{lvv-3396}

\begin{longtable}{cccc}
\hline
\textbf{Jira Link} & \textbf{Assignee} & \textbf{Status} & \textbf{Test Cases}\\ \hline
\href{https://jira.lsstcorp.org/browse/LVV-3396}{LVV-3396} &
Leanne Guy & Not Covered &
\begin{tabular}{c}
\end{tabular}
\\
\hline
\end{longtable}

\textbf{Verification Element Description:} \\
All Processed Visit Images (PVIs) for a single visit that are available
in cache, including mask and variance planes, shall be identifiable with
a single IVOA SIAv2 service query and retrievable, using the link(s)
provided in the response, within~\textbf{allPviRetrievalTime = 60
seconds.}

The associated element DMS-REQ-0376-V-02
(\href{https://jira.lsstcorp.org/browse/LVV-9795}{LVV-9795}) satisfies
the additional constraint on the number of simultaneous users.

Associated element~DMS-REQ-0376-V-03
(\href{https://jira.lsstcorp.org/browse/LVV-9796}{LVV-9796}) satisfies
the expected lifetime of Level-1 data products.

\emph{These requirements should be satisfied both separately and
together.}

{\footnotesize
\begin{longtable}{p{2.5cm}p{13.5cm}}
\hline
\multicolumn{2}{c}{\textbf{Requirement Details}}\\ \hline
Requirement ID & DMS-REQ-0376 \\ \cdashline{1-2}
Requirement Description &
\begin{minipage}[]{13cm}
\textbf{Specification:} All Processed Visit Images for a single visit
that are available in cache, including mask and variance planes, shall
be identifiable with a single IVOA SIAv2 service query and retrievable,
using the link(s) provided in the response, within
\textbf{allPviRetrievalTime.} This requirement shall be met for up to
\textbf{allPviRetrievalUsers} simultaneous requests for distinct
focal-plane PVI sets.
\end{minipage}
\\ \cdashline{1-2}
Requirement Parameters & {[}\textbf{allPviRetrievalUsers = 10{{[}integer{]}}} Minimum number of
simultaneous users retrieving all PVI images for a visit.,
\textbf{allPviRetrievalTime = 60{{[}second{]}}} Maximum time allowed for
retrieving all PVI images of a single visit., \textbf{l1CacheLifetime =
30{{[}day{]}}} Lifetime in the cache of un-archived Level-1 data
products.{]} \\ \cdashline{1-2}
Requirement Discussion &
\begin{minipage}[]{13cm}
\textbf{Discussion:} The performance targets for this requirement assume
the PVIs are available as files on a file system. For example, this
could be those files present in the \textbf{l1CacheLifetime} cache.
\end{minipage}
\\ \cdashline{1-2}
Requirement Priority & 1b \\ \cdashline{1-2}
Upper Level Requirement &
\begin{tabular}{cl}
OSS-REQ-0181 & Data Products Query and Download Infrastructure \\
\end{tabular}
\\ \hline
\end{longtable}
}


  
 \newpage 
\subsection{[LVV-3397] DMS-REQ-0373-V-01: Min number of simultaneous large-area coadd image
users }\label{lvv-3397}

\begin{longtable}{cccc}
\hline
\textbf{Jira Link} & \textbf{Assignee} & \textbf{Status} & \textbf{Test Cases}\\ \hline
\href{https://jira.lsstcorp.org/browse/LVV-3397}{LVV-3397} &
Leanne Guy & Not Covered &
\begin{tabular}{c}
\end{tabular}
\\
\hline
\end{longtable}

\textbf{Verification Element Description:} \\
At least \textbf{fplaneRetrievalUsers = 10}Â~simultaneous users shall be
able to retrieve single, large-area coadd images.

Associated element DMS-REQ-0373-V-02
\href{https://jira.lsstcorp.org/browse/LVV-9789}{(LVV-9789)} satisfies
the constraint on retrieval time for coadd images.

\emph{These requirements should be satisfied together.}

{\footnotesize
\begin{longtable}{p{2.5cm}p{13.5cm}}
\hline
\multicolumn{2}{c}{\textbf{Requirement Details}}\\ \hline
Requirement ID & DMS-REQ-0373 \\ \cdashline{1-2}
Requirement Description &
\begin{minipage}[]{13cm}
\textbf{Specification:} A 10 square degree coadd, including mask and
variance planes, shall be retrievable using the IVOA SODA protocol
within \textbf{fplaneRetrievalTime} with \textbf{fplaneRetrievalUsers}
simultaneous requests for distinct areas of the sky.
\end{minipage}
\\ \cdashline{1-2}
Requirement Parameters & {[}\textbf{fplaneRetrievalTime = 60{{[}second{]}}} Maximum time allowed
for retrieving a focal-plane sized coadd., \textbf{fplaneRetrievalUsers
= 10{{[}integer{]}}} Number of simultaneous users retrieving a single
large area coadd.{]} \\ \cdashline{1-2}
Requirement Priority & 2 \\ \cdashline{1-2}
Upper Level Requirement &
\begin{tabular}{cl}
OSS-REQ-0181 & Data Products Query and Download Infrastructure \\
\end{tabular}
\\ \hline
\end{longtable}
}


  
 \newpage 
\subsection{[LVV-3398] DMS-REQ-0375-V-01: Max time to retrieve single-object postage stamp
images }\label{lvv-3398}

\begin{longtable}{cccc}
\hline
\textbf{Jira Link} & \textbf{Assignee} & \textbf{Status} & \textbf{Test Cases}\\ \hline
\href{https://jira.lsstcorp.org/browse/LVV-3398}{LVV-3398} &
Leanne Guy & Not Covered &
\begin{tabular}{c}
\end{tabular}
\\
\hline
\end{longtable}

\textbf{Verification Element Description:} \\
Users shall be able to retrieve postage stamp images of all observations
of a single Object within~\textbf{postageStampRetrievalTime =
10~seconds.}

The associated element
DMS-REQ-0375-V-02~(\href{https://jira.lsstcorp.org/browse/LVV-9792}{LVV-9792})~satisfies
the additional constraint on the minimum size of a postage stamp cutout.

Associated element~DMS-REQ-0375-V-03
(\href{https://jira.lsstcorp.org/browse/LVV-9793}{LVV-9793}) satisfies
the expected lifetime of Level-1 data products.

The associated element
DMS-REQ-0375-V-04~(\href{https://jira.lsstcorp.org/browse/LVV-9794}{LVV-9794})~satisfies
the additional constraint on the number of simultaneous users retrieving
postage stamp images.

\emph{These requirements should be satisfied together.}

{\footnotesize
\begin{longtable}{p{2.5cm}p{13.5cm}}
\hline
\multicolumn{2}{c}{\textbf{Requirement Details}}\\ \hline
Requirement ID & DMS-REQ-0375 \\ \cdashline{1-2}
Requirement Description &
\begin{minipage}[]{13cm}
\textbf{Specification:} Postage stamp cutouts, of size
\textbf{postageStampSize} square, of all observations of a single Object
shall be retrievable within \textbf{postageStampRetrievalTime}, with
\textbf{postageStampRetrievalUsers} simultaneous requests of distinct
Objects.
\end{minipage}
\\ \cdashline{1-2}
Requirement Parameters & {[}\textbf{postageStampRetrievalUsers = 10{{[}integer{]}}} Minimum
number of simultaneous users retrieving a set of postage stamp images.,
\textbf{postageStampRetrievalTime = 10{{[}second{]}}} Maximum time
allowed for retrieving a set of postage stamp images of a single
Object., \textbf{postageStampSize = 51{{[}pixel{]}}} Minimum square size
of a postage stamp cutout from an image., \textbf{l1CacheLifetime =
30{{[}day{]}}} Lifetime in the cache of un-archived Level-1 data
products.{]} \\ \cdashline{1-2}
Requirement Discussion &
\begin{minipage}[]{13cm}
\textbf{Discussion:} The performance targets for this requirement assume
the PVIs are available as files on a file system. For example, this
could be those files present in the \textbf{l1CacheLifetime} cache.
\end{minipage}
\\ \cdashline{1-2}
Requirement Priority & 2 \\ \cdashline{1-2}
Upper Level Requirement &
\begin{tabular}{cl}
OSS-REQ-0181 & Data Products Query and Download Infrastructure \\
\end{tabular}
\\ \hline
\end{longtable}
}


  
 \newpage 
\subsection{[LVV-3400] DMS-REQ-0358-V-01: Min number of simultaneous DM EFD query users }\label{lvv-3400}

\begin{longtable}{cccc}
\hline
\textbf{Jira Link} & \textbf{Assignee} & \textbf{Status} & \textbf{Test Cases}\\ \hline
\href{https://jira.lsstcorp.org/browse/LVV-3400}{LVV-3400} &
Leanne Guy & Not Covered &
\begin{tabular}{c}
LVV-T1250 \\
\end{tabular}
\\
\hline
\end{longtable}

\textbf{Verification Element Description:} \\
At least \textbf{dmEfdQueryUsers}~\textbf{= 5}~simultaneous users shall
be able to query the EFD.

Associated element DMS-REQ-0358-V-02
(\href{https://jira.lsstcorp.org/browse/LVV-9788}{LVV-9788}) satisfies
the constraint on retrieval time for EFD queries.

\emph{These requirements should be satisfied together.}

{\footnotesize
\begin{longtable}{p{2.5cm}p{13.5cm}}
\hline
\multicolumn{2}{c}{\textbf{Requirement Details}}\\ \hline
Requirement ID & DMS-REQ-0358 \\ \cdashline{1-2}
Requirement Description &
\begin{minipage}[]{13cm}
\textbf{Specification:} The DM copy of the EFD shall support at least
\textbf{dmEfdQueryUsers} simultaneous queries, assuming each query lasts
no more than \textbf{dmEfdQueryTime}.
\end{minipage}
\\ \cdashline{1-2}
Requirement Parameters & {[}\textbf{dmEfdQueryTime = 10{{[}second{]}}} Maximum time allowed for
retrieving results of a DM EFD query., \textbf{dmEfdQueryUsers =
5{{[}integer{]}}} Minimum number of simultaneous users querying the DM
EFD.{]} \\ \cdashline{1-2}
Requirement Priority & 1a \\ \cdashline{1-2}
Upper Level Requirement &
\begin{tabular}{cl}
OSS-REQ-0181 & Data Products Query and Download Infrastructure \\
\end{tabular}
\\ \hline
\end{longtable}
}


\subsubsection{Test Cases Summary}
\begin{longtable}{p{3cm}p{2.5cm}p{2.5cm}p{3cm}p{4cm}}
\toprule
\href{https://jira.lsstcorp.org/secure/Tests.jspa\#/testCase/LVV-T1250}{LVV-T1250} & \multicolumn{4}{p{12cm}}{ Verify implementation of minimum number of simultaneous DM EFD query
users } \\ \hline
\textbf{Owner} & \textbf{Status} & \textbf{Version} & \textbf{Critical Event} & \textbf{Verification Type} \\ \hline
Jeffrey Carlin & Draft & 1 & false & Test \\ \hline
\end{longtable}
{\scriptsize
\textbf{Objective:}\\
Verify that the DM EFD can support \textbf{dmEfdQueryUsers~= 5}
simultaneous queries. The additional requirement that each query must
last no more than \textbf{dmEfdQueryTime = 10 seconds~}will be verified
separately in
\href{https://jira.lsstcorp.org/secure/Tests.jspa\#/testCase/LVV-T1251}{LVV-T1251},
but these must be satisfied together.
}
  
 \newpage 
\subsection{[LVV-3403] DMS-REQ-0361-V-01: Simultaneous users for high-volume queries }\label{lvv-3403}

\begin{longtable}{cccc}
\hline
\textbf{Jira Link} & \textbf{Assignee} & \textbf{Status} & \textbf{Test Cases}\\ \hline
\href{https://jira.lsstcorp.org/browse/LVV-3403}{LVV-3403} &
Leanne Guy & Not Covered &
\begin{tabular}{c}
LVV-T1088 \\
LVV-T1089 \\
LVV-T1090 \\
\end{tabular}
\\
\hline
\end{longtable}

\textbf{Verification Element Description:} \\
Undefined

{\footnotesize
\begin{longtable}{p{2.5cm}p{13.5cm}}
\hline
\multicolumn{2}{c}{\textbf{Requirement Details}}\\ \hline
Requirement ID & DMS-REQ-0361 \\ \cdashline{1-2}
Requirement Description &
\begin{minipage}[]{13cm}
\textbf{Specification:} The system shall support
*hvQueryUsers~*simultaneous high-volume queries running at any given
time.
\end{minipage}
\\ \cdashline{1-2}
Requirement Parameters & \textbf{hvQueryUsers = 50{{[}integer{]}}} Minimum number of simultaneous
users performing high volume queries. \\ \cdashline{1-2}
Requirement Priority & 1b \\ \cdashline{1-2}
Upper Level Requirement &
\begin{tabular}{cl}
OSS-REQ-0181 & Data Products Query and Download Infrastructure \\
\end{tabular}
\\ \hline
\end{longtable}
}


\subsubsection{Test Cases Summary}
\begin{longtable}{p{3cm}p{2.5cm}p{2.5cm}p{3cm}p{4cm}}
\toprule
\href{https://jira.lsstcorp.org/secure/Tests.jspa\#/testCase/LVV-T1088}{LVV-T1088} & \multicolumn{4}{p{12cm}}{ Concurrent Scans Scaling Test } \\ \hline
\textbf{Owner} & \textbf{Status} & \textbf{Version} & \textbf{Critical Event} & \textbf{Verification Type} \\ \hline
Fritz Mueller & Approved & 1 & false & Test \\ \hline
\end{longtable}
{\scriptsize
\textbf{Objective:}\\
This test will show that average completion-time of full-scan queries of
the Object catalog table grows sub-linearly with respect to the number
of simultaneously active full-scan queries, within the limits of machine
resource exhaustion.
}
\begin{longtable}{p{3cm}p{2.5cm}p{2.5cm}p{3cm}p{4cm}}
\toprule
\href{https://jira.lsstcorp.org/secure/Tests.jspa\#/testCase/LVV-T1089}{LVV-T1089} & \multicolumn{4}{p{12cm}}{ Load Test } \\ \hline
\textbf{Owner} & \textbf{Status} & \textbf{Version} & \textbf{Critical Event} & \textbf{Verification Type} \\ \hline
Fritz Mueller & Approved & 1 & false & Test \\ \hline
\end{longtable}
{\scriptsize
\textbf{Objective:}\\
This test will check that Qserv is able to meet average query completion
time targets per query class under a representative load of simultaneous
high and low volume queries while running against an appropriately
scaled test catalog.
}
\begin{longtable}{p{3cm}p{2.5cm}p{2.5cm}p{3cm}p{4cm}}
\toprule
\href{https://jira.lsstcorp.org/secure/Tests.jspa\#/testCase/LVV-T1090}{LVV-T1090} & \multicolumn{4}{p{12cm}}{ Heavy Load Test } \\ \hline
\textbf{Owner} & \textbf{Status} & \textbf{Version} & \textbf{Critical Event} & \textbf{Verification Type} \\ \hline
Fritz Mueller & Approved & 1 & false & Test \\ \hline
\end{longtable}
{\scriptsize
\textbf{Objective:}\\
This test will check that Qserv is able to meet average query completion
time targets per query class under a higher than average load of
simultaneous high and low volume queries while running against an
appropriately scaled test catalog.
}
  
 \newpage 
\subsection{[LVV-4669] CA-DM-DAQ-ICD-0094-V-03: Ability to load data externally\_DM\_3 }\label{lvv-4669}

\begin{longtable}{cccc}
\hline
\textbf{Jira Link} & \textbf{Assignee} & \textbf{Status} & \textbf{Test Cases}\\ \hline
\href{https://jira.lsstcorp.org/browse/LVV-4669}{LVV-4669} &
Leanne Guy & Not Covered &
\begin{tabular}{c}
\end{tabular}
\\
\hline
\end{longtable}

\textbf{Verification Element Description:} \\
Undefined

{\footnotesize
\begin{longtable}{p{2.5cm}p{13.5cm}}
\hline
\multicolumn{2}{c}{\textbf{Requirement Details}}\\ \hline
Requirement ID & CA-DM-DAQ-ICD-0094 \\ \cdashline{1-2}
Requirement Description &
\begin{minipage}[]{13cm}
\textbf{Specification:} The Camera shall provide an interface that
allows the buffer to be loaded with image data from an external source,
and for this data to be retrieved using the interfaces specified in this
section. ~It shall be possible to load image data that can be retrieved
with the image identifier given at load time; it shall also be possible
to load image data that can be retrieved with a new image identifier, as
if it had just been read out from the Camera.
\end{minipage}
\\ \cdashline{1-2}
Requirement Discussion &
\begin{minipage}[]{13cm}
\textbf{Discussion}: This capability is meant to be used to allow the
system to be exercised, and interfaces verified, in advance of the
availability of Camera imaging hardware, as well as for diagnostic
purposes during commissioning and operations.
\end{minipage}
\\ \cdashline{1-2}
Requirement Priority &  \\ \cdashline{1-2}
Upper Level Requirement &
\begin{tabular}{cl}
\end{tabular}
\\ \hline
\end{longtable}
}


  
 \newpage 
\subsection{[LVV-4670] CA-DM-DAQ-ICD-0094-V-04: Ability to load data externally\_DM\_4 }\label{lvv-4670}

\begin{longtable}{cccc}
\hline
\textbf{Jira Link} & \textbf{Assignee} & \textbf{Status} & \textbf{Test Cases}\\ \hline
\href{https://jira.lsstcorp.org/browse/LVV-4670}{LVV-4670} &
Leanne Guy & Not Covered &
\begin{tabular}{c}
\end{tabular}
\\
\hline
\end{longtable}

\textbf{Verification Element Description:} \\
Undefined

{\footnotesize
\begin{longtable}{p{2.5cm}p{13.5cm}}
\hline
\multicolumn{2}{c}{\textbf{Requirement Details}}\\ \hline
Requirement ID & CA-DM-DAQ-ICD-0094 \\ \cdashline{1-2}
Requirement Description &
\begin{minipage}[]{13cm}
\textbf{Specification:} The Camera shall provide an interface that
allows the buffer to be loaded with image data from an external source,
and for this data to be retrieved using the interfaces specified in this
section. ~It shall be possible to load image data that can be retrieved
with the image identifier given at load time; it shall also be possible
to load image data that can be retrieved with a new image identifier, as
if it had just been read out from the Camera.
\end{minipage}
\\ \cdashline{1-2}
Requirement Discussion &
\begin{minipage}[]{13cm}
\textbf{Discussion}: This capability is meant to be used to allow the
system to be exercised, and interfaces verified, in advance of the
availability of Camera imaging hardware, as well as for diagnostic
purposes during commissioning and operations.
\end{minipage}
\\ \cdashline{1-2}
Requirement Priority &  \\ \cdashline{1-2}
Upper Level Requirement &
\begin{tabular}{cl}
\end{tabular}
\\ \hline
\end{longtable}
}


  
 \newpage 
\subsection{[LVV-4675] CA-DM-DAQ-ICD-0082-V-03: Common interface across classes of
sensors\_DM\_3 }\label{lvv-4675}

\begin{longtable}{cccc}
\hline
\textbf{Jira Link} & \textbf{Assignee} & \textbf{Status} & \textbf{Test Cases}\\ \hline
\href{https://jira.lsstcorp.org/browse/LVV-4675}{LVV-4675} &
Leanne Guy & Not Covered &
\begin{tabular}{c}
\end{tabular}
\\
\hline
\end{longtable}

\textbf{Verification Element Description:} \\
Undefined

{\footnotesize
\begin{longtable}{p{2.5cm}p{13.5cm}}
\hline
\multicolumn{2}{c}{\textbf{Requirement Details}}\\ \hline
Requirement ID & CA-DM-DAQ-ICD-0082 \\ \cdashline{1-2}
Requirement Description &
\begin{minipage}[]{13cm}
\textbf{Specification}: A single interface shall support retrieval of
science, wavefront, and full-frame guide sensor images, with API
differences limited to those required for the specification of which
sensor(s) to access in a retrieval, or otherwise explicitly specified
herein.
\end{minipage}
\\ \cdashline{1-2}
Requirement Discussion &
\begin{minipage}[]{13cm}
\textbf{Discussion}: It is expected that during normal 9Hz guider ROI
readout the DAQ system will be configured not to direct the ROI readouts
into the Camera data buffer; therefore they will be invisible to the
pull interface. A subscription to notifications of buffered data
availability for the guide sensors will simply receive no callbacks for
as long as that configuration of the DAQ system is in place.
\end{minipage}
\\ \cdashline{1-2}
Requirement Priority &  \\ \cdashline{1-2}
Upper Level Requirement &
\begin{tabular}{cl}
\end{tabular}
\\ \hline
\end{longtable}
}


  
 \newpage 
\subsection{[LVV-4676] CA-DM-DAQ-ICD-0082-V-04: Common interface across classes of
sensors\_DM\_4 }\label{lvv-4676}

\begin{longtable}{cccc}
\hline
\textbf{Jira Link} & \textbf{Assignee} & \textbf{Status} & \textbf{Test Cases}\\ \hline
\href{https://jira.lsstcorp.org/browse/LVV-4676}{LVV-4676} &
Leanne Guy & Not Covered &
\begin{tabular}{c}
\end{tabular}
\\
\hline
\end{longtable}

\textbf{Verification Element Description:} \\
Undefined

{\footnotesize
\begin{longtable}{p{2.5cm}p{13.5cm}}
\hline
\multicolumn{2}{c}{\textbf{Requirement Details}}\\ \hline
Requirement ID & CA-DM-DAQ-ICD-0082 \\ \cdashline{1-2}
Requirement Description &
\begin{minipage}[]{13cm}
\textbf{Specification}: A single interface shall support retrieval of
science, wavefront, and full-frame guide sensor images, with API
differences limited to those required for the specification of which
sensor(s) to access in a retrieval, or otherwise explicitly specified
herein.
\end{minipage}
\\ \cdashline{1-2}
Requirement Discussion &
\begin{minipage}[]{13cm}
\textbf{Discussion}: It is expected that during normal 9Hz guider ROI
readout the DAQ system will be configured not to direct the ROI readouts
into the Camera data buffer; therefore they will be invisible to the
pull interface. A subscription to notifications of buffered data
availability for the guide sensors will simply receive no callbacks for
as long as that configuration of the DAQ system is in place.
\end{minipage}
\\ \cdashline{1-2}
Requirement Priority &  \\ \cdashline{1-2}
Upper Level Requirement &
\begin{tabular}{cl}
\end{tabular}
\\ \hline
\end{longtable}
}


  
 \newpage 
\subsection{[LVV-4729] CA-DM-DAQ-ICD-0093-V-03: Delivery latency\_DM\_3 }\label{lvv-4729}

\begin{longtable}{cccc}
\hline
\textbf{Jira Link} & \textbf{Assignee} & \textbf{Status} & \textbf{Test Cases}\\ \hline
\href{https://jira.lsstcorp.org/browse/LVV-4729}{LVV-4729} &
Leanne Guy & Not Covered &
\begin{tabular}{c}
\end{tabular}
\\
\hline
\end{longtable}

\textbf{Verification Element Description:} \\
Undefined

{\footnotesize
\begin{longtable}{p{2.5cm}p{13.5cm}}
\hline
\multicolumn{2}{c}{\textbf{Requirement Details}}\\ \hline
Requirement ID & CA-DM-DAQ-ICD-0093 \\ \cdashline{1-2}
Requirement Description &
\begin{minipage}[]{13cm}
\textbf{Specification}: The Camera shall complete the delivery of an
available image within time \textbf{daqLatency} of each request based on
a notification (see CA-DM-DAQ-ICD-0084), starting from the time of the
call to the request interface. This requirement shall apply for
retrievals up to the scale of a full raft from a single consumer. This
requirement shall apply whether or not crosstalk correction is applied.
\end{minipage}
\\ \cdashline{1-2}
Requirement Parameters & \textbf{daqLatency = 1{{[}second{]}}} Data delivery latency from time of
request to time of delivery \\ \cdashline{1-2}
Requirement Priority &  \\ \cdashline{1-2}
Upper Level Requirement &
\begin{tabular}{cl}
\end{tabular}
\\ \hline
\end{longtable}
}


  
 \newpage 
\subsection{[LVV-4730] CA-DM-DAQ-ICD-0093-V-04: Delivery latency\_DM\_4 }\label{lvv-4730}

\begin{longtable}{cccc}
\hline
\textbf{Jira Link} & \textbf{Assignee} & \textbf{Status} & \textbf{Test Cases}\\ \hline
\href{https://jira.lsstcorp.org/browse/LVV-4730}{LVV-4730} &
Leanne Guy & Not Covered &
\begin{tabular}{c}
\end{tabular}
\\
\hline
\end{longtable}

\textbf{Verification Element Description:} \\
Undefined

{\footnotesize
\begin{longtable}{p{2.5cm}p{13.5cm}}
\hline
\multicolumn{2}{c}{\textbf{Requirement Details}}\\ \hline
Requirement ID & CA-DM-DAQ-ICD-0093 \\ \cdashline{1-2}
Requirement Description &
\begin{minipage}[]{13cm}
\textbf{Specification}: The Camera shall complete the delivery of an
available image within time \textbf{daqLatency} of each request based on
a notification (see CA-DM-DAQ-ICD-0084), starting from the time of the
call to the request interface. This requirement shall apply for
retrievals up to the scale of a full raft from a single consumer. This
requirement shall apply whether or not crosstalk correction is applied.
\end{minipage}
\\ \cdashline{1-2}
Requirement Parameters & \textbf{daqLatency = 1{{[}second{]}}} Data delivery latency from time of
request to time of delivery \\ \cdashline{1-2}
Requirement Priority &  \\ \cdashline{1-2}
Upper Level Requirement &
\begin{tabular}{cl}
\end{tabular}
\\ \hline
\end{longtable}
}


  
 \newpage 
\subsection{[LVV-4735] CA-DM-DAQ-ICD-0097-V-03: Error reporting\_DM\_3 }\label{lvv-4735}

\begin{longtable}{cccc}
\hline
\textbf{Jira Link} & \textbf{Assignee} & \textbf{Status} & \textbf{Test Cases}\\ \hline
\href{https://jira.lsstcorp.org/browse/LVV-4735}{LVV-4735} &
Leanne Guy & Not Covered &
\begin{tabular}{c}
\end{tabular}
\\
\hline
\end{longtable}

\textbf{Verification Element Description:} \\
Undefined

{\footnotesize
\begin{longtable}{p{2.5cm}p{13.5cm}}
\hline
\multicolumn{2}{c}{\textbf{Requirement Details}}\\ \hline
Requirement ID & CA-DM-DAQ-ICD-0097 \\ \cdashline{1-2}
Requirement Description &
\begin{minipage}[]{13cm}
\textbf{Specification}: Error reporting from the APIs implementing this
interface shall be by means of return codes for all non-fatal errors.
\end{minipage}
\\ \cdashline{1-2}
Requirement Discussion &
\begin{minipage}[]{13cm}
\textbf{Discussion}: C++ exceptions may be thrown in rare cases for
error conditions from which it is impossible to recover.
\end{minipage}
\\ \cdashline{1-2}
Requirement Priority &  \\ \cdashline{1-2}
Upper Level Requirement &
\begin{tabular}{cl}
\end{tabular}
\\ \hline
\end{longtable}
}


  
 \newpage 
\subsection{[LVV-4736] CA-DM-DAQ-ICD-0097-V-04: Error reporting\_DM\_4 }\label{lvv-4736}

\begin{longtable}{cccc}
\hline
\textbf{Jira Link} & \textbf{Assignee} & \textbf{Status} & \textbf{Test Cases}\\ \hline
\href{https://jira.lsstcorp.org/browse/LVV-4736}{LVV-4736} &
Leanne Guy & Not Covered &
\begin{tabular}{c}
\end{tabular}
\\
\hline
\end{longtable}

\textbf{Verification Element Description:} \\
Undefined

{\footnotesize
\begin{longtable}{p{2.5cm}p{13.5cm}}
\hline
\multicolumn{2}{c}{\textbf{Requirement Details}}\\ \hline
Requirement ID & CA-DM-DAQ-ICD-0097 \\ \cdashline{1-2}
Requirement Description &
\begin{minipage}[]{13cm}
\textbf{Specification}: Error reporting from the APIs implementing this
interface shall be by means of return codes for all non-fatal errors.
\end{minipage}
\\ \cdashline{1-2}
Requirement Discussion &
\begin{minipage}[]{13cm}
\textbf{Discussion}: C++ exceptions may be thrown in rare cases for
error conditions from which it is impossible to recover.
\end{minipage}
\\ \cdashline{1-2}
Requirement Priority &  \\ \cdashline{1-2}
Upper Level Requirement &
\begin{tabular}{cl}
\end{tabular}
\\ \hline
\end{longtable}
}


  
 \newpage 
\subsection{[LVV-4747] CA-DM-DAQ-ICD-0059-V-03: Image identification\_DM\_3 }\label{lvv-4747}

\begin{longtable}{cccc}
\hline
\textbf{Jira Link} & \textbf{Assignee} & \textbf{Status} & \textbf{Test Cases}\\ \hline
\href{https://jira.lsstcorp.org/browse/LVV-4747}{LVV-4747} &
Leanne Guy & Not Covered &
\begin{tabular}{c}
\end{tabular}
\\
\hline
\end{longtable}

\textbf{Verification Element Description:} \\
Undefined

{\footnotesize
\begin{longtable}{p{2.5cm}p{13.5cm}}
\hline
\multicolumn{2}{c}{\textbf{Requirement Details}}\\ \hline
Requirement ID & CA-DM-DAQ-ICD-0059 \\ \cdashline{1-2}
Requirement Description &
\begin{minipage}[]{13cm}
\textbf{Specification:} Image data obtained from the Camera shall
include a unique identifier for all data derived from each amplifier
readout. This identifier shall be capable of being used to associate the
image data with metadata obtained from the Observatory Control System
(OCS) publish-and-subscribe mechanism or from the Engineering and
Facilities Database (EFD), as well as to retrieve image data from the
Camera image data buffer.
\end{minipage}
\\ \cdashline{1-2}
Requirement Discussion &
\begin{minipage}[]{13cm}
\textbf{Discussion}: The rendezvous with other observatory data can be
done either through the above identifier, which is based on the ``image
sequence name'' provided by the OCS to the Camera, or through the
readout timestamp recorded by the DAQ system.
\end{minipage}
\\ \cdashline{1-2}
Requirement Priority &  \\ \cdashline{1-2}
Upper Level Requirement &
\begin{tabular}{cl}
\end{tabular}
\\ \hline
\end{longtable}
}


  
 \newpage 
\subsection{[LVV-4748] CA-DM-DAQ-ICD-0059-V-04: Image identification\_DM\_4 }\label{lvv-4748}

\begin{longtable}{cccc}
\hline
\textbf{Jira Link} & \textbf{Assignee} & \textbf{Status} & \textbf{Test Cases}\\ \hline
\href{https://jira.lsstcorp.org/browse/LVV-4748}{LVV-4748} &
Leanne Guy & Not Covered &
\begin{tabular}{c}
\end{tabular}
\\
\hline
\end{longtable}

\textbf{Verification Element Description:} \\
Undefined

{\footnotesize
\begin{longtable}{p{2.5cm}p{13.5cm}}
\hline
\multicolumn{2}{c}{\textbf{Requirement Details}}\\ \hline
Requirement ID & CA-DM-DAQ-ICD-0059 \\ \cdashline{1-2}
Requirement Description &
\begin{minipage}[]{13cm}
\textbf{Specification:} Image data obtained from the Camera shall
include a unique identifier for all data derived from each amplifier
readout. This identifier shall be capable of being used to associate the
image data with metadata obtained from the Observatory Control System
(OCS) publish-and-subscribe mechanism or from the Engineering and
Facilities Database (EFD), as well as to retrieve image data from the
Camera image data buffer.
\end{minipage}
\\ \cdashline{1-2}
Requirement Discussion &
\begin{minipage}[]{13cm}
\textbf{Discussion}: The rendezvous with other observatory data can be
done either through the above identifier, which is based on the ``image
sequence name'' provided by the OCS to the Camera, or through the
readout timestamp recorded by the DAQ system.
\end{minipage}
\\ \cdashline{1-2}
Requirement Priority &  \\ \cdashline{1-2}
Upper Level Requirement &
\begin{tabular}{cl}
\end{tabular}
\\ \hline
\end{longtable}
}


  
 \newpage 
\subsection{[LVV-4753] CA-DM-DAQ-ICD-0060-V-03: Image identifier characteristics\_DM\_3 }\label{lvv-4753}

\begin{longtable}{cccc}
\hline
\textbf{Jira Link} & \textbf{Assignee} & \textbf{Status} & \textbf{Test Cases}\\ \hline
\href{https://jira.lsstcorp.org/browse/LVV-4753}{LVV-4753} &
Leanne Guy & Not Covered &
\begin{tabular}{c}
\end{tabular}
\\
\hline
\end{longtable}

\textbf{Verification Element Description:} \\
Undefined

{\footnotesize
\begin{longtable}{p{2.5cm}p{13.5cm}}
\hline
\multicolumn{2}{c}{\textbf{Requirement Details}}\\ \hline
Requirement ID & CA-DM-DAQ-ICD-0060 \\ \cdashline{1-2}
Requirement Description &
\begin{minipage}[]{13cm}
\textbf{Specification:} ~The image identifier shall:


\begin{itemize}
\tightlist
\item
  be unique within the entire LSST survey;

\item
  enable differentiation of simulated data and real data;

\item
  enable the determination of which data source originated the data
  (e.g., distinguishing data taken from a spare raft on a test stand
  from data taken in the operational camera);

\item
  have a component that is invariant across the entire FPA for a single
  synchronized readout, including both science and wavefront sensors;

\item
  be invariant no matter how many times this data is delivered to a
  consumer;

\item
  be invariant to whether crosstalk correction has been applied or not.
\end{itemize}
\end{minipage}
\\ \cdashline{1-2}
Requirement Discussion &
\begin{minipage}[]{13cm}
\textbf{Discussion:} An image name satisfying the LSST convention is
sufficient to meet this requirement.
\end{minipage}
\\ \cdashline{1-2}
Requirement Priority &  \\ \cdashline{1-2}
Upper Level Requirement &
\begin{tabular}{cl}
CA-DM-DAQ-ICD-0059 & Image identification \\
\end{tabular}
\\ \hline
\end{longtable}
}


  
 \newpage 
\subsection{[LVV-4754] CA-DM-DAQ-ICD-0060-V-04: Image identifier characteristics\_DM\_4 }\label{lvv-4754}

\begin{longtable}{cccc}
\hline
\textbf{Jira Link} & \textbf{Assignee} & \textbf{Status} & \textbf{Test Cases}\\ \hline
\href{https://jira.lsstcorp.org/browse/LVV-4754}{LVV-4754} &
Leanne Guy & Not Covered &
\begin{tabular}{c}
\end{tabular}
\\
\hline
\end{longtable}

\textbf{Verification Element Description:} \\
Undefined

{\footnotesize
\begin{longtable}{p{2.5cm}p{13.5cm}}
\hline
\multicolumn{2}{c}{\textbf{Requirement Details}}\\ \hline
Requirement ID & CA-DM-DAQ-ICD-0060 \\ \cdashline{1-2}
Requirement Description &
\begin{minipage}[]{13cm}
\textbf{Specification:} ~The image identifier shall:


\begin{itemize}
\tightlist
\item
  be unique within the entire LSST survey;

\item
  enable differentiation of simulated data and real data;

\item
  enable the determination of which data source originated the data
  (e.g., distinguishing data taken from a spare raft on a test stand
  from data taken in the operational camera);

\item
  have a component that is invariant across the entire FPA for a single
  synchronized readout, including both science and wavefront sensors;

\item
  be invariant no matter how many times this data is delivered to a
  consumer;

\item
  be invariant to whether crosstalk correction has been applied or not.
\end{itemize}
\end{minipage}
\\ \cdashline{1-2}
Requirement Discussion &
\begin{minipage}[]{13cm}
\textbf{Discussion:} An image name satisfying the LSST convention is
sufficient to meet this requirement.
\end{minipage}
\\ \cdashline{1-2}
Requirement Priority &  \\ \cdashline{1-2}
Upper Level Requirement &
\begin{tabular}{cl}
CA-DM-DAQ-ICD-0059 & Image identification \\
\end{tabular}
\\ \hline
\end{longtable}
}


  
 \newpage 
\subsection{[LVV-4759] CA-DM-DAQ-ICD-0081-V-03: Image pixel data\_DM\_3 }\label{lvv-4759}

\begin{longtable}{cccc}
\hline
\textbf{Jira Link} & \textbf{Assignee} & \textbf{Status} & \textbf{Test Cases}\\ \hline
\href{https://jira.lsstcorp.org/browse/LVV-4759}{LVV-4759} &
Leanne Guy & Not Covered &
\begin{tabular}{c}
\end{tabular}
\\
\hline
\end{longtable}

\textbf{Verification Element Description:} \\
Undefined

{\footnotesize
\begin{longtable}{p{2.5cm}p{13.5cm}}
\hline
\multicolumn{2}{c}{\textbf{Requirement Details}}\\ \hline
Requirement ID & CA-DM-DAQ-ICD-0081 \\ \cdashline{1-2}
Requirement Description &
\begin{minipage}[]{13cm}
\textbf{Specification}: Image data pixel values shall be delivered as
32-bit signed integers representing ADC counts also known as ADUs.\\
\hspace*{0.333em}\\
\textbf{Specification}: It shall be possible to iterate over the pixel
values from consecutively digitized pixels across all 16 amplifiers
which occupy consecutive groups of 16 consecutive memory locations with
the group memory address increasing in time order of readout. (In other
words, the pixel value from a given row and column from each of the 16
amps in sequence is followed by the pixel value from the next column in
time order from the same row from each amp, and so on.)\\
\hspace*{0.333em}\\
\textbf{Specification}: Pre-scan and post-scan data within a row shall
be delivered contiguously with the physical-pixel data from that row,
unless the interface provides for optional separation of this data and
that option is explicitly exercised.
\end{minipage}
\\ \cdashline{1-2}
Requirement Discussion &
\begin{minipage}[]{13cm}
\textbf{Discussion:} This is based on the understanding that the
underlying raw pixel data is 18-bit, and that the application of
crosstalk correction is mathematically capable of producing negative
pixel values. Note that the requirements that consecutively read-out
pixels be consecutive in memory and that pre/post-scan data be in-line
with physical-pixel data more or less imply that, if the focal plane
region of image data requested is larger than an amplifier segment, the
delivered data will end up being grouped by amplifier rather than
assembled into a CCD-level image. This is acceptable to (and even
desired by) Data Management.
\end{minipage}
\\ \cdashline{1-2}
Requirement Priority &  \\ \cdashline{1-2}
Upper Level Requirement &
\begin{tabular}{cl}
\end{tabular}
\\ \hline
\end{longtable}
}


  
 \newpage 
\subsection{[LVV-4760] CA-DM-DAQ-ICD-0081-V-04: Image pixel data\_DM\_4 }\label{lvv-4760}

\begin{longtable}{cccc}
\hline
\textbf{Jira Link} & \textbf{Assignee} & \textbf{Status} & \textbf{Test Cases}\\ \hline
\href{https://jira.lsstcorp.org/browse/LVV-4760}{LVV-4760} &
Leanne Guy & Not Covered &
\begin{tabular}{c}
\end{tabular}
\\
\hline
\end{longtable}

\textbf{Verification Element Description:} \\
Undefined

{\footnotesize
\begin{longtable}{p{2.5cm}p{13.5cm}}
\hline
\multicolumn{2}{c}{\textbf{Requirement Details}}\\ \hline
Requirement ID & CA-DM-DAQ-ICD-0081 \\ \cdashline{1-2}
Requirement Description &
\begin{minipage}[]{13cm}
\textbf{Specification}: Image data pixel values shall be delivered as
32-bit signed integers representing ADC counts also known as ADUs.\\
\hspace*{0.333em}\\
\textbf{Specification}: It shall be possible to iterate over the pixel
values from consecutively digitized pixels across all 16 amplifiers
which occupy consecutive groups of 16 consecutive memory locations with
the group memory address increasing in time order of readout. (In other
words, the pixel value from a given row and column from each of the 16
amps in sequence is followed by the pixel value from the next column in
time order from the same row from each amp, and so on.)\\
\hspace*{0.333em}\\
\textbf{Specification}: Pre-scan and post-scan data within a row shall
be delivered contiguously with the physical-pixel data from that row,
unless the interface provides for optional separation of this data and
that option is explicitly exercised.
\end{minipage}
\\ \cdashline{1-2}
Requirement Discussion &
\begin{minipage}[]{13cm}
\textbf{Discussion:} This is based on the understanding that the
underlying raw pixel data is 18-bit, and that the application of
crosstalk correction is mathematically capable of producing negative
pixel values. Note that the requirements that consecutively read-out
pixels be consecutive in memory and that pre/post-scan data be in-line
with physical-pixel data more or less imply that, if the focal plane
region of image data requested is larger than an amplifier segment, the
delivered data will end up being grouped by amplifier rather than
assembled into a CCD-level image. This is acceptable to (and even
desired by) Data Management.
\end{minipage}
\\ \cdashline{1-2}
Requirement Priority &  \\ \cdashline{1-2}
Upper Level Requirement &
\begin{tabular}{cl}
\end{tabular}
\\ \hline
\end{longtable}
}


  
 \newpage 
\subsection{[LVV-4765] CA-DM-DAQ-ICD-0047-V-03: Interface for Buffered Data (``pull''
interface)\_DM\_3 }\label{lvv-4765}

\begin{longtable}{cccc}
\hline
\textbf{Jira Link} & \textbf{Assignee} & \textbf{Status} & \textbf{Test Cases}\\ \hline
\href{https://jira.lsstcorp.org/browse/LVV-4765}{LVV-4765} &
Leanne Guy & Not Covered &
\begin{tabular}{c}
\end{tabular}
\\
\hline
\end{longtable}

\textbf{Verification Element Description:} \\
Undefined

{\footnotesize
\begin{longtable}{p{2.5cm}p{13.5cm}}
\hline
\multicolumn{2}{c}{\textbf{Requirement Details}}\\ \hline
Requirement ID & CA-DM-DAQ-ICD-0047 \\ \cdashline{1-2}
Requirement Description &
\begin{minipage}[]{13cm}
\textbf{Specification:} The Camera shall maintain a buffer of recently
acquired image data and shall provide access to images in that buffer to
other LSST subsystems. The interface providing this access will be
referred to as the ``pull'' or ``buffered data'' interface.
\end{minipage}
\\ \cdashline{1-2}
Requirement Discussion &
\begin{minipage}[]{13cm}
\textbf{Discussion}: The buffered data interface is described in
subsidiary requirements below as providing at least the following
capabilities: notification of the availability of new data;
non-destructive read of any data by identifier; non-destructive query
for the identifier of the oldest available data; and discard of the
oldest available data.\\
The buffered data interface largely follows a FIFO queue model, with the
additional ability to read any data in the queue by its identifier.
\end{minipage}
\\ \cdashline{1-2}
Requirement Priority &  \\ \cdashline{1-2}
Upper Level Requirement &
\begin{tabular}{cl}
\end{tabular}
\\ \hline
\end{longtable}
}


  
 \newpage 
\subsection{[LVV-4766] CA-DM-DAQ-ICD-0047-V-04: Interface for Buffered Data (``pull''
interface)\_DM\_4 }\label{lvv-4766}

\begin{longtable}{cccc}
\hline
\textbf{Jira Link} & \textbf{Assignee} & \textbf{Status} & \textbf{Test Cases}\\ \hline
\href{https://jira.lsstcorp.org/browse/LVV-4766}{LVV-4766} &
Leanne Guy & Not Covered &
\begin{tabular}{c}
\end{tabular}
\\
\hline
\end{longtable}

\textbf{Verification Element Description:} \\
Undefined

{\footnotesize
\begin{longtable}{p{2.5cm}p{13.5cm}}
\hline
\multicolumn{2}{c}{\textbf{Requirement Details}}\\ \hline
Requirement ID & CA-DM-DAQ-ICD-0047 \\ \cdashline{1-2}
Requirement Description &
\begin{minipage}[]{13cm}
\textbf{Specification:} The Camera shall maintain a buffer of recently
acquired image data and shall provide access to images in that buffer to
other LSST subsystems. The interface providing this access will be
referred to as the ``pull'' or ``buffered data'' interface.
\end{minipage}
\\ \cdashline{1-2}
Requirement Discussion &
\begin{minipage}[]{13cm}
\textbf{Discussion}: The buffered data interface is described in
subsidiary requirements below as providing at least the following
capabilities: notification of the availability of new data;
non-destructive read of any data by identifier; non-destructive query
for the identifier of the oldest available data; and discard of the
oldest available data.\\
The buffered data interface largely follows a FIFO queue model, with the
additional ability to read any data in the queue by its identifier.
\end{minipage}
\\ \cdashline{1-2}
Requirement Priority &  \\ \cdashline{1-2}
Upper Level Requirement &
\begin{tabular}{cl}
\end{tabular}
\\ \hline
\end{longtable}
}


  
 \newpage 
\subsection{[LVV-4771] CA-DM-DAQ-ICD-0098-V-03: Lookup-by-name interface\_DM\_3 }\label{lvv-4771}

\begin{longtable}{cccc}
\hline
\textbf{Jira Link} & \textbf{Assignee} & \textbf{Status} & \textbf{Test Cases}\\ \hline
\href{https://jira.lsstcorp.org/browse/LVV-4771}{LVV-4771} &
Leanne Guy & Not Covered &
\begin{tabular}{c}
\end{tabular}
\\
\hline
\end{longtable}

\textbf{Verification Element Description:} \\
Undefined

{\footnotesize
\begin{longtable}{p{2.5cm}p{13.5cm}}
\hline
\multicolumn{2}{c}{\textbf{Requirement Details}}\\ \hline
Requirement ID & CA-DM-DAQ-ICD-0098 \\ \cdashline{1-2}
Requirement Description &
\begin{minipage}[]{13cm}
\textbf{Specification}: The Camera shall provide an interface that
permits looking up the Container ID for an image based on its Image
Name. This interface shall ignore the existence of partitions - that is,
it shall return the Container ID for an image regardless of the
partition which may currently contain the image.
\end{minipage}
\\ \cdashline{1-2}
Requirement Priority &  \\ \cdashline{1-2}
Upper Level Requirement &
\begin{tabular}{cl}
\end{tabular}
\\ \hline
\end{longtable}
}


  
 \newpage 
\subsection{[LVV-4772] CA-DM-DAQ-ICD-0098-V-04: Lookup-by-name interface\_DM\_4 }\label{lvv-4772}

\begin{longtable}{cccc}
\hline
\textbf{Jira Link} & \textbf{Assignee} & \textbf{Status} & \textbf{Test Cases}\\ \hline
\href{https://jira.lsstcorp.org/browse/LVV-4772}{LVV-4772} &
Leanne Guy & Not Covered &
\begin{tabular}{c}
\end{tabular}
\\
\hline
\end{longtable}

\textbf{Verification Element Description:} \\
Undefined

{\footnotesize
\begin{longtable}{p{2.5cm}p{13.5cm}}
\hline
\multicolumn{2}{c}{\textbf{Requirement Details}}\\ \hline
Requirement ID & CA-DM-DAQ-ICD-0098 \\ \cdashline{1-2}
Requirement Description &
\begin{minipage}[]{13cm}
\textbf{Specification}: The Camera shall provide an interface that
permits looking up the Container ID for an image based on its Image
Name. This interface shall ignore the existence of partitions - that is,
it shall return the Container ID for an image regardless of the
partition which may currently contain the image.
\end{minipage}
\\ \cdashline{1-2}
Requirement Priority &  \\ \cdashline{1-2}
Upper Level Requirement &
\begin{tabular}{cl}
\end{tabular}
\\ \hline
\end{longtable}
}


  
 \newpage 
\subsection{[LVV-4777] CA-DM-DAQ-ICD-0100-V-03: Safe-to-delete event\_DM\_3 }\label{lvv-4777}

\begin{longtable}{cccc}
\hline
\textbf{Jira Link} & \textbf{Assignee} & \textbf{Status} & \textbf{Test Cases}\\ \hline
\href{https://jira.lsstcorp.org/browse/LVV-4777}{LVV-4777} &
Leanne Guy & Not Covered &
\begin{tabular}{c}
\end{tabular}
\\
\hline
\end{longtable}

\textbf{Verification Element Description:} \\
Undefined

{\footnotesize
\begin{longtable}{p{2.5cm}p{13.5cm}}
\hline
\multicolumn{2}{c}{\textbf{Requirement Details}}\\ \hline
Requirement ID & CA-DM-DAQ-ICD-0100 \\ \cdashline{1-2}
Requirement Description &
\begin{minipage}[]{13cm}
\textbf{Specification:} DM shall publish an event when an image has been
safely stored and the copy in the DAQ is no longer needed.
\end{minipage}
\\ \cdashline{1-2}
Requirement Discussion &
\begin{minipage}[]{13cm}
\textbf{Discussion}: Note that this interface does not take a Spatial ID
argument.
\end{minipage}
\\ \cdashline{1-2}
Requirement Priority &  \\ \cdashline{1-2}
Upper Level Requirement &
\begin{tabular}{cl}
\end{tabular}
\\ \hline
\end{longtable}
}


  
 \newpage 
\subsection{[LVV-4778] CA-DM-DAQ-ICD-0100-V-04: Safe-to-delete event\_DM\_4 }\label{lvv-4778}

\begin{longtable}{cccc}
\hline
\textbf{Jira Link} & \textbf{Assignee} & \textbf{Status} & \textbf{Test Cases}\\ \hline
\href{https://jira.lsstcorp.org/browse/LVV-4778}{LVV-4778} &
Leanne Guy & Not Covered &
\begin{tabular}{c}
\end{tabular}
\\
\hline
\end{longtable}

\textbf{Verification Element Description:} \\
Undefined

{\footnotesize
\begin{longtable}{p{2.5cm}p{13.5cm}}
\hline
\multicolumn{2}{c}{\textbf{Requirement Details}}\\ \hline
Requirement ID & CA-DM-DAQ-ICD-0100 \\ \cdashline{1-2}
Requirement Description &
\begin{minipage}[]{13cm}
\textbf{Specification:} DM shall publish an event when an image has been
safely stored and the copy in the DAQ is no longer needed.
\end{minipage}
\\ \cdashline{1-2}
Requirement Discussion &
\begin{minipage}[]{13cm}
\textbf{Discussion}: Note that this interface does not take a Spatial ID
argument.
\end{minipage}
\\ \cdashline{1-2}
Requirement Priority &  \\ \cdashline{1-2}
Upper Level Requirement &
\begin{tabular}{cl}
\end{tabular}
\\ \hline
\end{longtable}
}


  
 \newpage 
\subsection{[LVV-4784] CA-DM-DAQ-ICD-0092-V-04: Maximum number of simultaneous clients\_DM\_4 }\label{lvv-4784}

\begin{longtable}{cccc}
\hline
\textbf{Jira Link} & \textbf{Assignee} & \textbf{Status} & \textbf{Test Cases}\\ \hline
\href{https://jira.lsstcorp.org/browse/LVV-4784}{LVV-4784} &
Leanne Guy & Not Covered &
\begin{tabular}{c}
\end{tabular}
\\
\hline
\end{longtable}

\textbf{Verification Element Description:} \\
Undefined

{\footnotesize
\begin{longtable}{p{2.5cm}p{13.5cm}}
\hline
\multicolumn{2}{c}{\textbf{Requirement Details}}\\ \hline
Requirement ID & CA-DM-DAQ-ICD-0092 \\ \cdashline{1-2}
Requirement Description &
\begin{minipage}[]{13cm}
\textbf{Specification:} The Camera shall support simultaneous access to
the image store by at least four privileged consumers, each of which may
consist of multiple nodes retrieving, in combination, up to one focal
plane's worth of image data, for which its performance guarantees (for
latency and throughput) are met.
\end{minipage}
\\ \cdashline{1-2}
Requirement Discussion &
\begin{minipage}[]{13cm}
\textbf{Discussion}: The value specified is believed to meet both the
requirements of Data Management and the requirements of the Observatory
for access to images for display and diagnostic purposes. The Camera may
support additional non-privileged consumers, but without any performance
guarantees for them.
\end{minipage}
\\ \cdashline{1-2}
Requirement Priority &  \\ \cdashline{1-2}
Upper Level Requirement &
\begin{tabular}{cl}
\end{tabular}
\\ \hline
\end{longtable}
}


  
 \newpage 
\subsection{[LVV-4789] CA-DM-DAQ-ICD-0084-V-03: Notification interface\_DM\_3 }\label{lvv-4789}

\begin{longtable}{cccc}
\hline
\textbf{Jira Link} & \textbf{Assignee} & \textbf{Status} & \textbf{Test Cases}\\ \hline
\href{https://jira.lsstcorp.org/browse/LVV-4789}{LVV-4789} &
Leanne Guy & Not Covered &
\begin{tabular}{c}
\end{tabular}
\\
\hline
\end{longtable}

\textbf{Verification Element Description:} \\
Undefined

{\footnotesize
\begin{longtable}{p{2.5cm}p{13.5cm}}
\hline
\multicolumn{2}{c}{\textbf{Requirement Details}}\\ \hline
Requirement ID & CA-DM-DAQ-ICD-0084 \\ \cdashline{1-2}
Requirement Description &
\begin{minipage}[]{13cm}
\textbf{Specification}: The Camera shall provide an interface that
allows a client to subscribe to notifications of the availability of
data from a new image in the buffer. A subscription shall normally be
effective, barring unexpected error conditions, for a full night's
observing. A notification shall include the unique identifier of the
image (see CA-DM-DAQ-ICD-0059), as well as the ``container ID'', a key
which permits the retrieval of the image and associated metadata from
the buffer. The delivery of a notification to a client shall be
interpreted as a promise that the read-by-container-ID interface (see
CA-DM-DAQ-ICD-0086) can immediately be used to request retrieval of the
associated image data.
\end{minipage}
\\ \cdashline{1-2}
Requirement Discussion &
\begin{minipage}[]{13cm}
\textbf{Discussion}: It is expected that the image identifier and
container ID will be delivered by value to the callback, obviating the
need for memory management. The Camera expects to be able to offer the
ability to subscribe to time-sliced partial readouts, in integral
numbers of rows, and to support corresponding data requests (see the
note in CA-DM-DAQ-ICD-0091).
\end{minipage}
\\ \cdashline{1-2}
Requirement Priority &  \\ \cdashline{1-2}
Upper Level Requirement &
\begin{tabular}{cl}
\end{tabular}
\\ \hline
\end{longtable}
}


  
 \newpage 
\subsection{[LVV-4790] CA-DM-DAQ-ICD-0084-V-04: Notification interface\_DM\_4 }\label{lvv-4790}

\begin{longtable}{cccc}
\hline
\textbf{Jira Link} & \textbf{Assignee} & \textbf{Status} & \textbf{Test Cases}\\ \hline
\href{https://jira.lsstcorp.org/browse/LVV-4790}{LVV-4790} &
Leanne Guy & Not Covered &
\begin{tabular}{c}
\end{tabular}
\\
\hline
\end{longtable}

\textbf{Verification Element Description:} \\
Undefined

{\footnotesize
\begin{longtable}{p{2.5cm}p{13.5cm}}
\hline
\multicolumn{2}{c}{\textbf{Requirement Details}}\\ \hline
Requirement ID & CA-DM-DAQ-ICD-0084 \\ \cdashline{1-2}
Requirement Description &
\begin{minipage}[]{13cm}
\textbf{Specification}: The Camera shall provide an interface that
allows a client to subscribe to notifications of the availability of
data from a new image in the buffer. A subscription shall normally be
effective, barring unexpected error conditions, for a full night's
observing. A notification shall include the unique identifier of the
image (see CA-DM-DAQ-ICD-0059), as well as the ``container ID'', a key
which permits the retrieval of the image and associated metadata from
the buffer. The delivery of a notification to a client shall be
interpreted as a promise that the read-by-container-ID interface (see
CA-DM-DAQ-ICD-0086) can immediately be used to request retrieval of the
associated image data.
\end{minipage}
\\ \cdashline{1-2}
Requirement Discussion &
\begin{minipage}[]{13cm}
\textbf{Discussion}: It is expected that the image identifier and
container ID will be delivered by value to the callback, obviating the
need for memory management. The Camera expects to be able to offer the
ability to subscribe to time-sliced partial readouts, in integral
numbers of rows, and to support corresponding data requests (see the
note in CA-DM-DAQ-ICD-0091).
\end{minipage}
\\ \cdashline{1-2}
Requirement Priority &  \\ \cdashline{1-2}
Upper Level Requirement &
\begin{tabular}{cl}
\end{tabular}
\\ \hline
\end{longtable}
}


  
 \newpage 
\subsection{[LVV-4795] CA-DM-DAQ-ICD-0099-V-03: Partition catalog query interface\_DM\_3 }\label{lvv-4795}

\begin{longtable}{cccc}
\hline
\textbf{Jira Link} & \textbf{Assignee} & \textbf{Status} & \textbf{Test Cases}\\ \hline
\href{https://jira.lsstcorp.org/browse/LVV-4795}{LVV-4795} &
Leanne Guy & Not Covered &
\begin{tabular}{c}
\end{tabular}
\\
\hline
\end{longtable}

\textbf{Verification Element Description:} \\
Undefined

{\footnotesize
\begin{longtable}{p{2.5cm}p{13.5cm}}
\hline
\multicolumn{2}{c}{\textbf{Requirement Details}}\\ \hline
Requirement ID & CA-DM-DAQ-ICD-0099 \\ \cdashline{1-2}
Requirement Description &
\begin{minipage}[]{13cm}
\textbf{Specification}: The Camera shall provide an interface that can
be used to query the complete list of valid Container IDs for a given
partition. The IDs shall be returned in the order in which they were
inserted into the partition.
\end{minipage}
\\ \cdashline{1-2}
Requirement Priority &  \\ \cdashline{1-2}
Upper Level Requirement &
\begin{tabular}{cl}
\end{tabular}
\\ \hline
\end{longtable}
}


  
 \newpage 
\subsection{[LVV-4796] CA-DM-DAQ-ICD-0099-V-04: Partition catalog query interface\_DM\_4 }\label{lvv-4796}

\begin{longtable}{cccc}
\hline
\textbf{Jira Link} & \textbf{Assignee} & \textbf{Status} & \textbf{Test Cases}\\ \hline
\href{https://jira.lsstcorp.org/browse/LVV-4796}{LVV-4796} &
Leanne Guy & Not Covered &
\begin{tabular}{c}
\end{tabular}
\\
\hline
\end{longtable}

\textbf{Verification Element Description:} \\
Undefined

{\footnotesize
\begin{longtable}{p{2.5cm}p{13.5cm}}
\hline
\multicolumn{2}{c}{\textbf{Requirement Details}}\\ \hline
Requirement ID & CA-DM-DAQ-ICD-0099 \\ \cdashline{1-2}
Requirement Description &
\begin{minipage}[]{13cm}
\textbf{Specification}: The Camera shall provide an interface that can
be used to query the complete list of valid Container IDs for a given
partition. The IDs shall be returned in the order in which they were
inserted into the partition.
\end{minipage}
\\ \cdashline{1-2}
Requirement Priority &  \\ \cdashline{1-2}
Upper Level Requirement &
\begin{tabular}{cl}
\end{tabular}
\\ \hline
\end{longtable}
}


  
 \newpage 
\subsection{[LVV-4801] CA-DM-DAQ-ICD-0085-V-03: Partitioning interfaces\_DM\_3 }\label{lvv-4801}

\begin{longtable}{cccc}
\hline
\textbf{Jira Link} & \textbf{Assignee} & \textbf{Status} & \textbf{Test Cases}\\ \hline
\href{https://jira.lsstcorp.org/browse/LVV-4801}{LVV-4801} &
Leanne Guy & Not Covered &
\begin{tabular}{c}
\end{tabular}
\\
\hline
\end{longtable}

\textbf{Verification Element Description:} \\
Undefined

{\footnotesize
\begin{longtable}{p{2.5cm}p{13.5cm}}
\hline
\multicolumn{2}{c}{\textbf{Requirement Details}}\\ \hline
Requirement ID & CA-DM-DAQ-ICD-0085 \\ \cdashline{1-2}
Requirement Description &
\begin{minipage}[]{13cm}
\textbf{Specification}: The Camera shall provide an interface that
allows the partitioning of the data in the buffer into named sets. The
Camera shall be configurable to direct newly acquired data into a
specified partition or partitions.
\end{minipage}
\\ \cdashline{1-2}
Requirement Discussion &
\begin{minipage}[]{13cm}
\textbf{Discussion}: This facility may be exclusive - such that a given
image may only belong to a single set - or non-exclusive, in which an
image may belong to more than one set.\\
The intent of the partitioning interface is to permit distinguishing
image data that is meant to be visible to, and retrieved by, clients
such as Data Management from image data that is meant for internal
Camera purposes. It is also intended to be used as the mechanism that
supports \citeds{LSE-69} requirement CA-DM-CON-ICD-0019's specification that
there be a means to communicate to DM what Camera engineering images are
being requested to be archived by DM.
\end{minipage}
\\ \cdashline{1-2}
Requirement Priority &  \\ \cdashline{1-2}
Upper Level Requirement &
\begin{tabular}{cl}
\end{tabular}
\\ \hline
\end{longtable}
}


  
 \newpage 
\subsection{[LVV-4802] CA-DM-DAQ-ICD-0085-V-04: Partitioning interfaces\_DM\_4 }\label{lvv-4802}

\begin{longtable}{cccc}
\hline
\textbf{Jira Link} & \textbf{Assignee} & \textbf{Status} & \textbf{Test Cases}\\ \hline
\href{https://jira.lsstcorp.org/browse/LVV-4802}{LVV-4802} &
Leanne Guy & Not Covered &
\begin{tabular}{c}
\end{tabular}
\\
\hline
\end{longtable}

\textbf{Verification Element Description:} \\
Undefined

{\footnotesize
\begin{longtable}{p{2.5cm}p{13.5cm}}
\hline
\multicolumn{2}{c}{\textbf{Requirement Details}}\\ \hline
Requirement ID & CA-DM-DAQ-ICD-0085 \\ \cdashline{1-2}
Requirement Description &
\begin{minipage}[]{13cm}
\textbf{Specification}: The Camera shall provide an interface that
allows the partitioning of the data in the buffer into named sets. The
Camera shall be configurable to direct newly acquired data into a
specified partition or partitions.
\end{minipage}
\\ \cdashline{1-2}
Requirement Discussion &
\begin{minipage}[]{13cm}
\textbf{Discussion}: This facility may be exclusive - such that a given
image may only belong to a single set - or non-exclusive, in which an
image may belong to more than one set.\\
The intent of the partitioning interface is to permit distinguishing
image data that is meant to be visible to, and retrieved by, clients
such as Data Management from image data that is meant for internal
Camera purposes. It is also intended to be used as the mechanism that
supports \citeds{LSE-69} requirement CA-DM-CON-ICD-0019's specification that
there be a means to communicate to DM what Camera engineering images are
being requested to be archived by DM.
\end{minipage}
\\ \cdashline{1-2}
Requirement Priority &  \\ \cdashline{1-2}
Upper Level Requirement &
\begin{tabular}{cl}
\end{tabular}
\\ \hline
\end{longtable}
}


  
 \newpage 
\subsection{[LVV-4807] CA-DM-DAQ-ICD-0086-V-03: Read-by-container-ID interface\_DM\_3 }\label{lvv-4807}

\begin{longtable}{cccc}
\hline
\textbf{Jira Link} & \textbf{Assignee} & \textbf{Status} & \textbf{Test Cases}\\ \hline
\href{https://jira.lsstcorp.org/browse/LVV-4807}{LVV-4807} &
Leanne Guy & Not Covered &
\begin{tabular}{c}
\end{tabular}
\\
\hline
\end{longtable}

\textbf{Verification Element Description:} \\
Undefined

{\footnotesize
\begin{longtable}{p{2.5cm}p{13.5cm}}
\hline
\multicolumn{2}{c}{\textbf{Requirement Details}}\\ \hline
Requirement ID & CA-DM-DAQ-ICD-0086 \\ \cdashline{1-2}
Requirement Description &
\begin{minipage}[]{13cm}
\textbf{Specification}: The Camera shall provide an interface that
allows any image in the buffer, specified by the client providing its
container ID and spatial ID, to be read non-destructively. If the data
are not present in the buffer, the Camera shall respond with an
appropriate error indication. The Camera is not required to distinguish
in such a response whether image data with the requested identifier was
previously present but deleted, or never present.
\end{minipage}
\\ \cdashline{1-2}
Requirement Discussion &
\begin{minipage}[]{13cm}
\textbf{Discussion}: When used for an image or portion of an image for
which readout is in progress. this interface will return an error
condition; it will not block waiting for completion of readout.
\end{minipage}
\\ \cdashline{1-2}
Requirement Priority &  \\ \cdashline{1-2}
Upper Level Requirement &
\begin{tabular}{cl}
\end{tabular}
\\ \hline
\end{longtable}
}


  
 \newpage 
\subsection{[LVV-4808] CA-DM-DAQ-ICD-0086-V-04: Read-by-container-ID interface\_DM\_4 }\label{lvv-4808}

\begin{longtable}{cccc}
\hline
\textbf{Jira Link} & \textbf{Assignee} & \textbf{Status} & \textbf{Test Cases}\\ \hline
\href{https://jira.lsstcorp.org/browse/LVV-4808}{LVV-4808} &
Leanne Guy & Not Covered &
\begin{tabular}{c}
\end{tabular}
\\
\hline
\end{longtable}

\textbf{Verification Element Description:} \\
Undefined

{\footnotesize
\begin{longtable}{p{2.5cm}p{13.5cm}}
\hline
\multicolumn{2}{c}{\textbf{Requirement Details}}\\ \hline
Requirement ID & CA-DM-DAQ-ICD-0086 \\ \cdashline{1-2}
Requirement Description &
\begin{minipage}[]{13cm}
\textbf{Specification}: The Camera shall provide an interface that
allows any image in the buffer, specified by the client providing its
container ID and spatial ID, to be read non-destructively. If the data
are not present in the buffer, the Camera shall respond with an
appropriate error indication. The Camera is not required to distinguish
in such a response whether image data with the requested identifier was
previously present but deleted, or never present.
\end{minipage}
\\ \cdashline{1-2}
Requirement Discussion &
\begin{minipage}[]{13cm}
\textbf{Discussion}: When used for an image or portion of an image for
which readout is in progress. this interface will return an error
condition; it will not block waiting for completion of readout.
\end{minipage}
\\ \cdashline{1-2}
Requirement Priority &  \\ \cdashline{1-2}
Upper Level Requirement &
\begin{tabular}{cl}
\end{tabular}
\\ \hline
\end{longtable}
}


  
 \newpage 
\subsection{[LVV-4819] CA-DM-DAQ-ICD-0091-V-03: Selection of region of focal plane to be
retrieved\_DM\_3 }\label{lvv-4819}

\begin{longtable}{cccc}
\hline
\textbf{Jira Link} & \textbf{Assignee} & \textbf{Status} & \textbf{Test Cases}\\ \hline
\href{https://jira.lsstcorp.org/browse/LVV-4819}{LVV-4819} &
Leanne Guy & Not Covered &
\begin{tabular}{c}
\end{tabular}
\\
\hline
\end{longtable}

\textbf{Verification Element Description:} \\
Undefined

{\footnotesize
\begin{longtable}{p{2.5cm}p{13.5cm}}
\hline
\multicolumn{2}{c}{\textbf{Requirement Details}}\\ \hline
Requirement ID & CA-DM-DAQ-ICD-0091 \\ \cdashline{1-2}
Requirement Description &
\begin{minipage}[]{13cm}
\textbf{Specification}: The Camera shall permit the selection of a
subunit of the focal plane for readout. The largest size that must be
supported without degradation of performance requirements is the raft.
The smallest unit that must be supported is the amplifier. When data
from multiple amplifier segments is requested, the Camera shall by
default organize the delivered data by segment; i.e., it shall not
attempt to stitch together segments into larger units.
\end{minipage}
\\ \cdashline{1-2}
Requirement Discussion &
\begin{minipage}[]{13cm}
\textbf{Discussion}: The specification of the maximum region is not
meant to prohibit the Camera from offering the ability to request larger
regions, at possibly degraded capabilities. There may well be useful
applications for this.\\
It is anticipated that time-sliced sub-amplifier regions, such as sets
of rows, will be supported. The applications for this capability are
still being discussed.
\end{minipage}
\\ \cdashline{1-2}
Requirement Priority &  \\ \cdashline{1-2}
Upper Level Requirement &
\begin{tabular}{cl}
\end{tabular}
\\ \hline
\end{longtable}
}


  
 \newpage 
\subsection{[LVV-4820] CA-DM-DAQ-ICD-0091-V-04: Selection of region of focal plane to be
retrieved\_DM\_4 }\label{lvv-4820}

\begin{longtable}{cccc}
\hline
\textbf{Jira Link} & \textbf{Assignee} & \textbf{Status} & \textbf{Test Cases}\\ \hline
\href{https://jira.lsstcorp.org/browse/LVV-4820}{LVV-4820} &
Leanne Guy & Not Covered &
\begin{tabular}{c}
\end{tabular}
\\
\hline
\end{longtable}

\textbf{Verification Element Description:} \\
Undefined

{\footnotesize
\begin{longtable}{p{2.5cm}p{13.5cm}}
\hline
\multicolumn{2}{c}{\textbf{Requirement Details}}\\ \hline
Requirement ID & CA-DM-DAQ-ICD-0091 \\ \cdashline{1-2}
Requirement Description &
\begin{minipage}[]{13cm}
\textbf{Specification}: The Camera shall permit the selection of a
subunit of the focal plane for readout. The largest size that must be
supported without degradation of performance requirements is the raft.
The smallest unit that must be supported is the amplifier. When data
from multiple amplifier segments is requested, the Camera shall by
default organize the delivered data by segment; i.e., it shall not
attempt to stitch together segments into larger units.
\end{minipage}
\\ \cdashline{1-2}
Requirement Discussion &
\begin{minipage}[]{13cm}
\textbf{Discussion}: The specification of the maximum region is not
meant to prohibit the Camera from offering the ability to request larger
regions, at possibly degraded capabilities. There may well be useful
applications for this.\\
It is anticipated that time-sliced sub-amplifier regions, such as sets
of rows, will be supported. The applications for this capability are
still being discussed.
\end{minipage}
\\ \cdashline{1-2}
Requirement Priority &  \\ \cdashline{1-2}
Upper Level Requirement &
\begin{tabular}{cl}
\end{tabular}
\\ \hline
\end{longtable}
}


  
 \newpage 
\subsection{[LVV-4825] CA-DM-DAQ-ICD-0075-V-03: Software Delivery\_DM\_3 }\label{lvv-4825}

\begin{longtable}{cccc}
\hline
\textbf{Jira Link} & \textbf{Assignee} & \textbf{Status} & \textbf{Test Cases}\\ \hline
\href{https://jira.lsstcorp.org/browse/LVV-4825}{LVV-4825} &
Leanne Guy & Not Covered &
\begin{tabular}{c}
\end{tabular}
\\
\hline
\end{longtable}

\textbf{Verification Element Description:} \\
Undefined

{\footnotesize
\begin{longtable}{p{2.5cm}p{13.5cm}}
\hline
\multicolumn{2}{c}{\textbf{Requirement Details}}\\ \hline
Requirement ID & CA-DM-DAQ-ICD-0075 \\ \cdashline{1-2}
Requirement Description &
\begin{minipage}[]{13cm}
\textbf{Specification:} The camera image data access client software
providing the interfaces in this document shall be delivered as one or
more libraries that can be linked into a C++ application, compiled
against the library headers, with its main program provided by the user.
The libraries will be supplied as pre-compiled shareables in Unix
``.so'' format.
\end{minipage}
\\ \cdashline{1-2}
Requirement Discussion &
\begin{minipage}[]{13cm}
\textbf{Discussion:} The idea here is that the client software is not a
framework into which user code is inserted, but the reverse.\\
The source code for the client libraries will be maintained using common
LSST source control tools.
\end{minipage}
\\ \cdashline{1-2}
Requirement Priority &  \\ \cdashline{1-2}
Upper Level Requirement &
\begin{tabular}{cl}
\end{tabular}
\\ \hline
\end{longtable}
}


  
 \newpage 
\subsection{[LVV-4826] CA-DM-DAQ-ICD-0075-V-04: Software Delivery\_DM\_4 }\label{lvv-4826}

\begin{longtable}{cccc}
\hline
\textbf{Jira Link} & \textbf{Assignee} & \textbf{Status} & \textbf{Test Cases}\\ \hline
\href{https://jira.lsstcorp.org/browse/LVV-4826}{LVV-4826} &
Leanne Guy & Not Covered &
\begin{tabular}{c}
\end{tabular}
\\
\hline
\end{longtable}

\textbf{Verification Element Description:} \\
Undefined

{\footnotesize
\begin{longtable}{p{2.5cm}p{13.5cm}}
\hline
\multicolumn{2}{c}{\textbf{Requirement Details}}\\ \hline
Requirement ID & CA-DM-DAQ-ICD-0075 \\ \cdashline{1-2}
Requirement Description &
\begin{minipage}[]{13cm}
\textbf{Specification:} The camera image data access client software
providing the interfaces in this document shall be delivered as one or
more libraries that can be linked into a C++ application, compiled
against the library headers, with its main program provided by the user.
The libraries will be supplied as pre-compiled shareables in Unix
``.so'' format.
\end{minipage}
\\ \cdashline{1-2}
Requirement Discussion &
\begin{minipage}[]{13cm}
\textbf{Discussion:} The idea here is that the client software is not a
framework into which user code is inserted, but the reverse.\\
The source code for the client libraries will be maintained using common
LSST source control tools.
\end{minipage}
\\ \cdashline{1-2}
Requirement Priority &  \\ \cdashline{1-2}
Upper Level Requirement &
\begin{tabular}{cl}
\end{tabular}
\\ \hline
\end{longtable}
}


  
 \newpage 
\subsection{[LVV-4831] CA-DM-DAQ-ICD-0080-V-03: Structural metadata\_DM\_3 }\label{lvv-4831}

\begin{longtable}{cccc}
\hline
\textbf{Jira Link} & \textbf{Assignee} & \textbf{Status} & \textbf{Test Cases}\\ \hline
\href{https://jira.lsstcorp.org/browse/LVV-4831}{LVV-4831} &
Leanne Guy & Not Covered &
\begin{tabular}{c}
\end{tabular}
\\
\hline
\end{longtable}

\textbf{Verification Element Description:} \\
Undefined

{\footnotesize
\begin{longtable}{p{2.5cm}p{13.5cm}}
\hline
\multicolumn{2}{c}{\textbf{Requirement Details}}\\ \hline
Requirement ID & CA-DM-DAQ-ICD-0080 \\ \cdashline{1-2}
Requirement Description &
\begin{minipage}[]{13cm}
\textbf{Specification:} The image data format and its API shall include
a limited set of ``structural metadata'', sufficient to assemble the
amplifier segments from the readout of the full focal plane into a
representation of the focal plane as a whole. ~This requirement does not
imply the provision of x-y coordinates of the sensors or a sky-to-pixel
mapping. ~It does require that a consumer be able to reconstruct the
correct neighbor relationships between amplifiers and sensors and know
the serial and parallel readout directions for every amplifier segment.
\end{minipage}
\\ \cdashline{1-2}
Requirement Discussion &
\begin{minipage}[]{13cm}
\textbf{Discussion:} The metadata to be provided is under review; it may
include the CCD manufacturer, the number of values read out in the
serial and parallel directions (including image pixels, prescan, and
overscan), the number of prescan values, and the number of overscan
values. ~The metadata items are expected to depend on the sequencer
program. ~The CCS is expected to provide these values to the DAQ, but
the DAQ API must provide a documented interface to retrieve them. ~These
items are thought to be necessary for the DAQ to be able to perform
crosstalk correction.
\end{minipage}
\\ \cdashline{1-2}
Requirement Priority &  \\ \cdashline{1-2}
Upper Level Requirement &
\begin{tabular}{cl}
\end{tabular}
\\ \hline
\end{longtable}
}


  
 \newpage 
\subsection{[LVV-4832] CA-DM-DAQ-ICD-0080-V-04: Structural metadata\_DM\_4 }\label{lvv-4832}

\begin{longtable}{cccc}
\hline
\textbf{Jira Link} & \textbf{Assignee} & \textbf{Status} & \textbf{Test Cases}\\ \hline
\href{https://jira.lsstcorp.org/browse/LVV-4832}{LVV-4832} &
Leanne Guy & Not Covered &
\begin{tabular}{c}
\end{tabular}
\\
\hline
\end{longtable}

\textbf{Verification Element Description:} \\
Undefined

{\footnotesize
\begin{longtable}{p{2.5cm}p{13.5cm}}
\hline
\multicolumn{2}{c}{\textbf{Requirement Details}}\\ \hline
Requirement ID & CA-DM-DAQ-ICD-0080 \\ \cdashline{1-2}
Requirement Description &
\begin{minipage}[]{13cm}
\textbf{Specification:} The image data format and its API shall include
a limited set of ``structural metadata'', sufficient to assemble the
amplifier segments from the readout of the full focal plane into a
representation of the focal plane as a whole. ~This requirement does not
imply the provision of x-y coordinates of the sensors or a sky-to-pixel
mapping. ~It does require that a consumer be able to reconstruct the
correct neighbor relationships between amplifiers and sensors and know
the serial and parallel readout directions for every amplifier segment.
\end{minipage}
\\ \cdashline{1-2}
Requirement Discussion &
\begin{minipage}[]{13cm}
\textbf{Discussion:} The metadata to be provided is under review; it may
include the CCD manufacturer, the number of values read out in the
serial and parallel directions (including image pixels, prescan, and
overscan), the number of prescan values, and the number of overscan
values. ~The metadata items are expected to depend on the sequencer
program. ~The CCS is expected to provide these values to the DAQ, but
the DAQ API must provide a documented interface to retrieve them. ~These
items are thought to be necessary for the DAQ to be able to perform
crosstalk correction.
\end{minipage}
\\ \cdashline{1-2}
Requirement Priority &  \\ \cdashline{1-2}
Upper Level Requirement &
\begin{tabular}{cl}
\end{tabular}
\\ \hline
\end{longtable}
}


  
 \newpage 
\subsection{[LVV-4843] CA-DM-CON-ICD-0003-V-03: Camera Conditions data latency for Alert
Production\_DM\_3 }\label{lvv-4843}

\begin{longtable}{cccc}
\hline
\textbf{Jira Link} & \textbf{Assignee} & \textbf{Status} & \textbf{Test Cases}\\ \hline
\href{https://jira.lsstcorp.org/browse/LVV-4843}{LVV-4843} &
Leanne Guy & Not Covered &
\begin{tabular}{c}
\end{tabular}
\\
\hline
\end{longtable}

\textbf{Verification Element Description:} \\
Undefined

{\footnotesize
\begin{longtable}{p{2.5cm}p{13.5cm}}
\hline
\multicolumn{2}{c}{\textbf{Requirement Details}}\\ \hline
Requirement ID & CA-DM-CON-ICD-0003 \\ \cdashline{1-2}
Requirement Description &
\begin{minipage}[]{13cm}
\textbf{Specification:} Camera Conditions data specified as required for
DM's Alert Production, enumerated in document \citeds{LSE-130}, concerning times
through the end of the readout of an image shall be published via the
OCS middleware within time \textbf{cameraConditionsLatencyDMAP} of the
conclusion of readout. The Camera should generally publish this data
within time \textbf{cameraConditionsLatencyDMAP} of its acquisition.
\end{minipage}
\\ \cdashline{1-2}
Requirement Parameters & \textbf{cameraConditionsLatencyDMAP = 300{{[}millisecond{]}}} Latency
for publication of Camera Conditions data for Alert Production \\ \cdashline{1-2}
Requirement Discussion &
\begin{minipage}[]{13cm}
\textbf{Discussion:} The latter condition expresses the desire that Data
Management not receive all Conditions data as a lump delivery at the end
of readout.
\end{minipage}
\\ \cdashline{1-2}
Requirement Priority &  \\ \cdashline{1-2}
Upper Level Requirement &
\begin{tabular}{cl}
\end{tabular}
\\ \hline
\end{longtable}
}


  
 \newpage 
\subsection{[LVV-4844] CA-DM-CON-ICD-0003-V-04: Camera Conditions data latency for Alert
Production\_DM\_4 }\label{lvv-4844}

\begin{longtable}{cccc}
\hline
\textbf{Jira Link} & \textbf{Assignee} & \textbf{Status} & \textbf{Test Cases}\\ \hline
\href{https://jira.lsstcorp.org/browse/LVV-4844}{LVV-4844} &
Leanne Guy & Not Covered &
\begin{tabular}{c}
\end{tabular}
\\
\hline
\end{longtable}

\textbf{Verification Element Description:} \\
Undefined

{\footnotesize
\begin{longtable}{p{2.5cm}p{13.5cm}}
\hline
\multicolumn{2}{c}{\textbf{Requirement Details}}\\ \hline
Requirement ID & CA-DM-CON-ICD-0003 \\ \cdashline{1-2}
Requirement Description &
\begin{minipage}[]{13cm}
\textbf{Specification:} Camera Conditions data specified as required for
DM's Alert Production, enumerated in document \citeds{LSE-130}, concerning times
through the end of the readout of an image shall be published via the
OCS middleware within time \textbf{cameraConditionsLatencyDMAP} of the
conclusion of readout. The Camera should generally publish this data
within time \textbf{cameraConditionsLatencyDMAP} of its acquisition.
\end{minipage}
\\ \cdashline{1-2}
Requirement Parameters & \textbf{cameraConditionsLatencyDMAP = 300{{[}millisecond{]}}} Latency
for publication of Camera Conditions data for Alert Production \\ \cdashline{1-2}
Requirement Discussion &
\begin{minipage}[]{13cm}
\textbf{Discussion:} The latter condition expresses the desire that Data
Management not receive all Conditions data as a lump delivery at the end
of readout.
\end{minipage}
\\ \cdashline{1-2}
Requirement Priority &  \\ \cdashline{1-2}
Upper Level Requirement &
\begin{tabular}{cl}
\end{tabular}
\\ \hline
\end{longtable}
}


  
 \newpage 
\subsection{[LVV-4849] CA-DM-CON-ICD-0004-V-03: Camera Conditions data latency for all
data\_DM\_3 }\label{lvv-4849}

\begin{longtable}{cccc}
\hline
\textbf{Jira Link} & \textbf{Assignee} & \textbf{Status} & \textbf{Test Cases}\\ \hline
\href{https://jira.lsstcorp.org/browse/LVV-4849}{LVV-4849} &
Leanne Guy & Not Covered &
\begin{tabular}{c}
\end{tabular}
\\
\hline
\end{longtable}

\textbf{Verification Element Description:} \\
Undefined

{\footnotesize
\begin{longtable}{p{2.5cm}p{13.5cm}}
\hline
\multicolumn{2}{c}{\textbf{Requirement Details}}\\ \hline
Requirement ID & CA-DM-CON-ICD-0004 \\ \cdashline{1-2}
Requirement Description &
\begin{minipage}[]{13cm}
\textbf{Specification:} All Camera Conditions data required by DM shall
be published through the OCS middleware within time
\textbf{cameraConditionsLatencyDM} of its measurement time.
\end{minipage}
\\ \cdashline{1-2}
Requirement Parameters & \textbf{cameraConditionsLatencyDM = 10{{[}second{]}}} Latency for
publication of Camera Conditions telemetry data \\ \cdashline{1-2}
Requirement Discussion &
\begin{minipage}[]{13cm}
\textbf{Discussion:} The ``measurement time'' is meant to be a wall
clock time for an underlying physical measurement or equivalent,
determined in a way that is reasonable for the telemetry data in
question. This is the same time that is referenced in \citeds{LSE-71},
requirement OCS-CA-CMD-ICD-0018.\\
The timely publication of telemetry facilitates the storage of copies of
the telemetry database in a time-local organization, and provides
flexibility for DM to manage data continuously or in batches.
\end{minipage}
\\ \cdashline{1-2}
Requirement Priority &  \\ \cdashline{1-2}
Upper Level Requirement &
\begin{tabular}{cl}
\end{tabular}
\\ \hline
\end{longtable}
}


  
 \newpage 
\subsection{[LVV-4850] CA-DM-CON-ICD-0004-V-04: Camera Conditions data latency for all
data\_DM\_4 }\label{lvv-4850}

\begin{longtable}{cccc}
\hline
\textbf{Jira Link} & \textbf{Assignee} & \textbf{Status} & \textbf{Test Cases}\\ \hline
\href{https://jira.lsstcorp.org/browse/LVV-4850}{LVV-4850} &
Leanne Guy & Not Covered &
\begin{tabular}{c}
\end{tabular}
\\
\hline
\end{longtable}

\textbf{Verification Element Description:} \\
Undefined

{\footnotesize
\begin{longtable}{p{2.5cm}p{13.5cm}}
\hline
\multicolumn{2}{c}{\textbf{Requirement Details}}\\ \hline
Requirement ID & CA-DM-CON-ICD-0004 \\ \cdashline{1-2}
Requirement Description &
\begin{minipage}[]{13cm}
\textbf{Specification:} All Camera Conditions data required by DM shall
be published through the OCS middleware within time
\textbf{cameraConditionsLatencyDM} of its measurement time.
\end{minipage}
\\ \cdashline{1-2}
Requirement Parameters & \textbf{cameraConditionsLatencyDM = 10{{[}second{]}}} Latency for
publication of Camera Conditions telemetry data \\ \cdashline{1-2}
Requirement Discussion &
\begin{minipage}[]{13cm}
\textbf{Discussion:} The ``measurement time'' is meant to be a wall
clock time for an underlying physical measurement or equivalent,
determined in a way that is reasonable for the telemetry data in
question. This is the same time that is referenced in \citeds{LSE-71},
requirement OCS-CA-CMD-ICD-0018.\\
The timely publication of telemetry facilitates the storage of copies of
the telemetry database in a time-local organization, and provides
flexibility for DM to manage data continuously or in batches.
\end{minipage}
\\ \cdashline{1-2}
Requirement Priority &  \\ \cdashline{1-2}
Upper Level Requirement &
\begin{tabular}{cl}
\end{tabular}
\\ \hline
\end{longtable}
}


  
 \newpage 
\subsection{[LVV-4855] CA-DM-CON-ICD-0019-V-03: Camera engineering image data archiving\_DM\_3 }\label{lvv-4855}

\begin{longtable}{cccc}
\hline
\textbf{Jira Link} & \textbf{Assignee} & \textbf{Status} & \textbf{Test Cases}\\ \hline
\href{https://jira.lsstcorp.org/browse/LVV-4855}{LVV-4855} &
Leanne Guy & Not Covered &
\begin{tabular}{c}
\end{tabular}
\\
\hline
\end{longtable}

\textbf{Verification Element Description:} \\
Undefined

{\footnotesize
\begin{longtable}{p{2.5cm}p{13.5cm}}
\hline
\multicolumn{2}{c}{\textbf{Requirement Details}}\\ \hline
Requirement ID & CA-DM-CON-ICD-0019 \\ \cdashline{1-2}
Requirement Description &
\begin{minipage}[]{13cm}
\textbf{Specification}: The Data Management subsystem shall provide an
archiving service for engineering image data from the Camera subsystem.
\end{minipage}
\\ \cdashline{1-2}
Requirement Discussion &
\begin{minipage}[]{13cm}
\textbf{Discussion}: Daytime calibration images are considered part of
normal Observatory operations and are not covered under this
requirement.\\
The images acquired by Data Management under this requirement will be
made available through the normal DM image archive programmatic and
graphical user interface tools.
\end{minipage}
\\ \cdashline{1-2}
Requirement Priority &  \\ \cdashline{1-2}
Upper Level Requirement &
\begin{tabular}{cl}
\end{tabular}
\\ \hline
\end{longtable}
}


  
 \newpage 
\subsection{[LVV-4856] CA-DM-CON-ICD-0019-V-04: Camera engineering image data archiving\_DM\_4 }\label{lvv-4856}

\begin{longtable}{cccc}
\hline
\textbf{Jira Link} & \textbf{Assignee} & \textbf{Status} & \textbf{Test Cases}\\ \hline
\href{https://jira.lsstcorp.org/browse/LVV-4856}{LVV-4856} &
Leanne Guy & Not Covered &
\begin{tabular}{c}
\end{tabular}
\\
\hline
\end{longtable}

\textbf{Verification Element Description:} \\
Undefined

{\footnotesize
\begin{longtable}{p{2.5cm}p{13.5cm}}
\hline
\multicolumn{2}{c}{\textbf{Requirement Details}}\\ \hline
Requirement ID & CA-DM-CON-ICD-0019 \\ \cdashline{1-2}
Requirement Description &
\begin{minipage}[]{13cm}
\textbf{Specification}: The Data Management subsystem shall provide an
archiving service for engineering image data from the Camera subsystem.
\end{minipage}
\\ \cdashline{1-2}
Requirement Discussion &
\begin{minipage}[]{13cm}
\textbf{Discussion}: Daytime calibration images are considered part of
normal Observatory operations and are not covered under this
requirement.\\
The images acquired by Data Management under this requirement will be
made available through the normal DM image archive programmatic and
graphical user interface tools.
\end{minipage}
\\ \cdashline{1-2}
Requirement Priority &  \\ \cdashline{1-2}
Upper Level Requirement &
\begin{tabular}{cl}
\end{tabular}
\\ \hline
\end{longtable}
}


  
 \newpage 
\subsection{[LVV-4861] CA-DM-CON-ICD-0008-V-03: Data Management Conditions data latency\_DM\_3 }\label{lvv-4861}

\begin{longtable}{cccc}
\hline
\textbf{Jira Link} & \textbf{Assignee} & \textbf{Status} & \textbf{Test Cases}\\ \hline
\href{https://jira.lsstcorp.org/browse/LVV-4861}{LVV-4861} &
Leanne Guy & Not Covered &
\begin{tabular}{c}
\end{tabular}
\\
\hline
\end{longtable}

\textbf{Verification Element Description:} \\
Undefined

{\footnotesize
\begin{longtable}{p{2.5cm}p{13.5cm}}
\hline
\multicolumn{2}{c}{\textbf{Requirement Details}}\\ \hline
Requirement ID & CA-DM-CON-ICD-0008 \\ \cdashline{1-2}
Requirement Description &
\begin{minipage}[]{13cm}
\textbf{Specification:} DM Conditions data required by the Camera and
derived from individual images in standard visits shall be published
through the OCS middleware no more than time
\textbf{dmConditionsLatencyCam} after the conclusion of the delivery of
all images from the standard visit to DM.
\end{minipage}
\\ \cdashline{1-2}
Requirement Parameters & \textbf{dmConditionsLatencyCam = 60{{[}second{]}}} Latency for
publication of per-visit Conditions data from DM \\ \cdashline{1-2}
Requirement Discussion &
\begin{minipage}[]{13cm}
\textbf{Discussion:} This data should be available by the conclusion of
Alert Production, and the latency is set to match that. It should be
available promptly in order to enhance its usefulness for Observatory
operators. There is, however, no intent to use this data in any
automated feedback loop controlling the Camera.\\
There may be additional types of Conditions data from DM that arise from
the analysis of multiple images, such as synthetic flats. This
requirement does not apply to that type of data. DM will generally make
any such data available to the Camera promptly following its generation.
\end{minipage}
\\ \cdashline{1-2}
Requirement Priority &  \\ \cdashline{1-2}
Upper Level Requirement &
\begin{tabular}{cl}
\end{tabular}
\\ \hline
\end{longtable}
}


  
 \newpage 
\subsection{[LVV-4862] CA-DM-CON-ICD-0008-V-04: Data Management Conditions data latency\_DM\_4 }\label{lvv-4862}

\begin{longtable}{cccc}
\hline
\textbf{Jira Link} & \textbf{Assignee} & \textbf{Status} & \textbf{Test Cases}\\ \hline
\href{https://jira.lsstcorp.org/browse/LVV-4862}{LVV-4862} &
Leanne Guy & Not Covered &
\begin{tabular}{c}
\end{tabular}
\\
\hline
\end{longtable}

\textbf{Verification Element Description:} \\
Undefined

{\footnotesize
\begin{longtable}{p{2.5cm}p{13.5cm}}
\hline
\multicolumn{2}{c}{\textbf{Requirement Details}}\\ \hline
Requirement ID & CA-DM-CON-ICD-0008 \\ \cdashline{1-2}
Requirement Description &
\begin{minipage}[]{13cm}
\textbf{Specification:} DM Conditions data required by the Camera and
derived from individual images in standard visits shall be published
through the OCS middleware no more than time
\textbf{dmConditionsLatencyCam} after the conclusion of the delivery of
all images from the standard visit to DM.
\end{minipage}
\\ \cdashline{1-2}
Requirement Parameters & \textbf{dmConditionsLatencyCam = 60{{[}second{]}}} Latency for
publication of per-visit Conditions data from DM \\ \cdashline{1-2}
Requirement Discussion &
\begin{minipage}[]{13cm}
\textbf{Discussion:} This data should be available by the conclusion of
Alert Production, and the latency is set to match that. It should be
available promptly in order to enhance its usefulness for Observatory
operators. There is, however, no intent to use this data in any
automated feedback loop controlling the Camera.\\
There may be additional types of Conditions data from DM that arise from
the analysis of multiple images, such as synthetic flats. This
requirement does not apply to that type of data. DM will generally make
any such data available to the Camera promptly following its generation.
\end{minipage}
\\ \cdashline{1-2}
Requirement Priority &  \\ \cdashline{1-2}
Upper Level Requirement &
\begin{tabular}{cl}
\end{tabular}
\\ \hline
\end{longtable}
}


  
 \newpage 
\subsection{[LVV-4873] CA-DM-CON-ICD-0002-V-03: Provide Camera Conditions data\_DM\_3 }\label{lvv-4873}

\begin{longtable}{cccc}
\hline
\textbf{Jira Link} & \textbf{Assignee} & \textbf{Status} & \textbf{Test Cases}\\ \hline
\href{https://jira.lsstcorp.org/browse/LVV-4873}{LVV-4873} &
Leanne Guy & Not Covered &
\begin{tabular}{c}
\end{tabular}
\\
\hline
\end{longtable}

\textbf{Verification Element Description:} \\
Undefined

{\footnotesize
\begin{longtable}{p{2.5cm}p{13.5cm}}
\hline
\multicolumn{2}{c}{\textbf{Requirement Details}}\\ \hline
Requirement ID & CA-DM-CON-ICD-0002 \\ \cdashline{1-2}
Requirement Description &
\begin{minipage}[]{13cm}
\textbf{Specification:} The Camera shall provide for the use of DM the
list of Conditions data items, specified in section 1.2 of document
\citeds{LSE-130}, as telemetry via the OCS middleware.
\end{minipage}
\\ \cdashline{1-2}
Requirement Discussion &
\begin{minipage}[]{13cm}
\textbf{Discussion:} This information will include the shutter position
and various conditions of the sensors and electronics. The full list of
telemetry planned to be provided by the Camera is maintained in document
\citeds{LSE-165}; it should be a superset of the telemetry covered by this
requirement.
\end{minipage}
\\ \cdashline{1-2}
Requirement Priority &  \\ \cdashline{1-2}
Upper Level Requirement &
\begin{tabular}{cl}
\end{tabular}
\\ \hline
\end{longtable}
}


  
 \newpage 
\subsection{[LVV-4874] CA-DM-CON-ICD-0002-V-04: Provide Camera Conditions data\_DM\_4 }\label{lvv-4874}

\begin{longtable}{cccc}
\hline
\textbf{Jira Link} & \textbf{Assignee} & \textbf{Status} & \textbf{Test Cases}\\ \hline
\href{https://jira.lsstcorp.org/browse/LVV-4874}{LVV-4874} &
Leanne Guy & Not Covered &
\begin{tabular}{c}
\end{tabular}
\\
\hline
\end{longtable}

\textbf{Verification Element Description:} \\
Undefined

{\footnotesize
\begin{longtable}{p{2.5cm}p{13.5cm}}
\hline
\multicolumn{2}{c}{\textbf{Requirement Details}}\\ \hline
Requirement ID & CA-DM-CON-ICD-0002 \\ \cdashline{1-2}
Requirement Description &
\begin{minipage}[]{13cm}
\textbf{Specification:} The Camera shall provide for the use of DM the
list of Conditions data items, specified in section 1.2 of document
\citeds{LSE-130}, as telemetry via the OCS middleware.
\end{minipage}
\\ \cdashline{1-2}
Requirement Discussion &
\begin{minipage}[]{13cm}
\textbf{Discussion:} This information will include the shutter position
and various conditions of the sensors and electronics. The full list of
telemetry planned to be provided by the Camera is maintained in document
\citeds{LSE-165}; it should be a superset of the telemetry covered by this
requirement.
\end{minipage}
\\ \cdashline{1-2}
Requirement Priority &  \\ \cdashline{1-2}
Upper Level Requirement &
\begin{tabular}{cl}
\end{tabular}
\\ \hline
\end{longtable}
}


  
 \newpage 
\subsection{[LVV-4879] CA-DM-CON-ICD-0005-V-03: Provide Camera Configuration data\_DM\_3 }\label{lvv-4879}

\begin{longtable}{cccc}
\hline
\textbf{Jira Link} & \textbf{Assignee} & \textbf{Status} & \textbf{Test Cases}\\ \hline
\href{https://jira.lsstcorp.org/browse/LVV-4879}{LVV-4879} &
Leanne Guy & Not Covered &
\begin{tabular}{c}
\end{tabular}
\\
\hline
\end{longtable}

\textbf{Verification Element Description:} \\
Undefined

{\footnotesize
\begin{longtable}{p{2.5cm}p{13.5cm}}
\hline
\multicolumn{2}{c}{\textbf{Requirement Details}}\\ \hline
Requirement ID & CA-DM-CON-ICD-0005 \\ \cdashline{1-2}
Requirement Description &
\begin{minipage}[]{13cm}
\textbf{Specification:} The Camera shall provide for the use of DM all
Configuration data enumerated in document \citeds{LSE-130}.
\end{minipage}
\\ \cdashline{1-2}
Requirement Discussion &
\begin{minipage}[]{13cm}
\textbf{Discussion}: Configuration data includes such things as
setpoints for power supplies and temperature regulators. This
requirement will be satisfied by means of the mechanism referred to in
\citeds{LSE-71}, OCS-CA-CMD-ICD-0025. That requirement ensures that the data are
available to DM as soon as Camera configuration is complete, and before
any associated data are taken.
\end{minipage}
\\ \cdashline{1-2}
Requirement Priority &  \\ \cdashline{1-2}
Upper Level Requirement &
\begin{tabular}{cl}
\end{tabular}
\\ \hline
\end{longtable}
}


  
 \newpage 
\subsection{[LVV-4880] CA-DM-CON-ICD-0005-V-04: Provide Camera Configuration data\_DM\_4 }\label{lvv-4880}

\begin{longtable}{cccc}
\hline
\textbf{Jira Link} & \textbf{Assignee} & \textbf{Status} & \textbf{Test Cases}\\ \hline
\href{https://jira.lsstcorp.org/browse/LVV-4880}{LVV-4880} &
Leanne Guy & Not Covered &
\begin{tabular}{c}
\end{tabular}
\\
\hline
\end{longtable}

\textbf{Verification Element Description:} \\
Undefined

{\footnotesize
\begin{longtable}{p{2.5cm}p{13.5cm}}
\hline
\multicolumn{2}{c}{\textbf{Requirement Details}}\\ \hline
Requirement ID & CA-DM-CON-ICD-0005 \\ \cdashline{1-2}
Requirement Description &
\begin{minipage}[]{13cm}
\textbf{Specification:} The Camera shall provide for the use of DM all
Configuration data enumerated in document \citeds{LSE-130}.
\end{minipage}
\\ \cdashline{1-2}
Requirement Discussion &
\begin{minipage}[]{13cm}
\textbf{Discussion}: Configuration data includes such things as
setpoints for power supplies and temperature regulators. This
requirement will be satisfied by means of the mechanism referred to in
\citeds{LSE-71}, OCS-CA-CMD-ICD-0025. That requirement ensures that the data are
available to DM as soon as Camera configuration is complete, and before
any associated data are taken.
\end{minipage}
\\ \cdashline{1-2}
Requirement Priority &  \\ \cdashline{1-2}
Upper Level Requirement &
\begin{tabular}{cl}
\end{tabular}
\\ \hline
\end{longtable}
}


  
 \newpage 
\subsection{[LVV-4885] CA-DM-CON-ICD-0001-V-03: Provide Camera design, assembly, and laboratory
test data\_DM\_3 }\label{lvv-4885}

\begin{longtable}{cccc}
\hline
\textbf{Jira Link} & \textbf{Assignee} & \textbf{Status} & \textbf{Test Cases}\\ \hline
\href{https://jira.lsstcorp.org/browse/LVV-4885}{LVV-4885} &
Leanne Guy & Not Covered &
\begin{tabular}{c}
\end{tabular}
\\
\hline
\end{longtable}

\textbf{Verification Element Description:} \\
Undefined

{\footnotesize
\begin{longtable}{p{2.5cm}p{13.5cm}}
\hline
\multicolumn{2}{c}{\textbf{Requirement Details}}\\ \hline
Requirement ID & CA-DM-CON-ICD-0001 \\ \cdashline{1-2}
Requirement Description &
\begin{minipage}[]{13cm}
\textbf{Specification:} The Camera shall provide to DM design, assembly,
and laboratory test information, as specified in section 1.1 of document
\citeds{LSE-130}.
\end{minipage}
\\ \cdashline{1-2}
Requirement Discussion &
\begin{minipage}[]{13cm}
\textbf{Discussion:} The method(s) of delivery will be specified in
\citeds{LSE-130}.
\end{minipage}
\\ \cdashline{1-2}
Requirement Priority &  \\ \cdashline{1-2}
Upper Level Requirement &
\begin{tabular}{cl}
\end{tabular}
\\ \hline
\end{longtable}
}


  
 \newpage 
\subsection{[LVV-4886] CA-DM-CON-ICD-0001-V-04: Provide Camera design, assembly, and laboratory
test data\_DM\_4 }\label{lvv-4886}

\begin{longtable}{cccc}
\hline
\textbf{Jira Link} & \textbf{Assignee} & \textbf{Status} & \textbf{Test Cases}\\ \hline
\href{https://jira.lsstcorp.org/browse/LVV-4886}{LVV-4886} &
Leanne Guy & Not Covered &
\begin{tabular}{c}
\end{tabular}
\\
\hline
\end{longtable}

\textbf{Verification Element Description:} \\
Undefined

{\footnotesize
\begin{longtable}{p{2.5cm}p{13.5cm}}
\hline
\multicolumn{2}{c}{\textbf{Requirement Details}}\\ \hline
Requirement ID & CA-DM-CON-ICD-0001 \\ \cdashline{1-2}
Requirement Description &
\begin{minipage}[]{13cm}
\textbf{Specification:} The Camera shall provide to DM design, assembly,
and laboratory test information, as specified in section 1.1 of document
\citeds{LSE-130}.
\end{minipage}
\\ \cdashline{1-2}
Requirement Discussion &
\begin{minipage}[]{13cm}
\textbf{Discussion:} The method(s) of delivery will be specified in
\citeds{LSE-130}.
\end{minipage}
\\ \cdashline{1-2}
Requirement Priority &  \\ \cdashline{1-2}
Upper Level Requirement &
\begin{tabular}{cl}
\end{tabular}
\\ \hline
\end{longtable}
}


  
 \newpage 
\subsection{[LVV-4897] CA-DM-CON-ICD-0018-V-03: Provide Camera OCS events needed by Data
Management\_DM\_3 }\label{lvv-4897}

\begin{longtable}{cccc}
\hline
\textbf{Jira Link} & \textbf{Assignee} & \textbf{Status} & \textbf{Test Cases}\\ \hline
\href{https://jira.lsstcorp.org/browse/LVV-4897}{LVV-4897} &
Leanne Guy & Not Covered &
\begin{tabular}{c}
\end{tabular}
\\
\hline
\end{longtable}

\textbf{Verification Element Description:} \\
Undefined

{\footnotesize
\begin{longtable}{p{2.5cm}p{13.5cm}}
\hline
\multicolumn{2}{c}{\textbf{Requirement Details}}\\ \hline
Requirement ID & CA-DM-CON-ICD-0018 \\ \cdashline{1-2}
Requirement Description &
\begin{minipage}[]{13cm}
\textbf{Specification}: The Camera shall publish at least the OCS
middleware ``events'' \textbf{startIntegration} and
\textbf{startReadout}, as defined in \citeds{LSE-71} (requirement
OCS-CA-CMD-ICD-0021), for each image generated in response to a
takeImages() command from the OCS.
\end{minipage}
\\ \cdashline{1-2}
Requirement Discussion &
\begin{minipage}[]{13cm}
\textbf{Discussion}: This is recorded in this ICD to make clear that
these particular events are part of the operational model on which DM
depends, and do not just arise from agreement between Camera and OCS in
their ICD. It does not create an additional substantive requirement on
the Camera.
\end{minipage}
\\ \cdashline{1-2}
Requirement Priority &  \\ \cdashline{1-2}
Upper Level Requirement &
\begin{tabular}{cl}
\end{tabular}
\\ \hline
\end{longtable}
}


  
 \newpage 
\subsection{[LVV-4898] CA-DM-CON-ICD-0018-V-04: Provide Camera OCS events needed by Data
Management\_DM\_4 }\label{lvv-4898}

\begin{longtable}{cccc}
\hline
\textbf{Jira Link} & \textbf{Assignee} & \textbf{Status} & \textbf{Test Cases}\\ \hline
\href{https://jira.lsstcorp.org/browse/LVV-4898}{LVV-4898} &
Leanne Guy & Not Covered &
\begin{tabular}{c}
\end{tabular}
\\
\hline
\end{longtable}

\textbf{Verification Element Description:} \\
Undefined

{\footnotesize
\begin{longtable}{p{2.5cm}p{13.5cm}}
\hline
\multicolumn{2}{c}{\textbf{Requirement Details}}\\ \hline
Requirement ID & CA-DM-CON-ICD-0018 \\ \cdashline{1-2}
Requirement Description &
\begin{minipage}[]{13cm}
\textbf{Specification}: The Camera shall publish at least the OCS
middleware ``events'' \textbf{startIntegration} and
\textbf{startReadout}, as defined in \citeds{LSE-71} (requirement
OCS-CA-CMD-ICD-0021), for each image generated in response to a
takeImages() command from the OCS.
\end{minipage}
\\ \cdashline{1-2}
Requirement Discussion &
\begin{minipage}[]{13cm}
\textbf{Discussion}: This is recorded in this ICD to make clear that
these particular events are part of the operational model on which DM
depends, and do not just arise from agreement between Camera and OCS in
their ICD. It does not create an additional substantive requirement on
the Camera.
\end{minipage}
\\ \cdashline{1-2}
Requirement Priority &  \\ \cdashline{1-2}
Upper Level Requirement &
\begin{tabular}{cl}
\end{tabular}
\\ \hline
\end{longtable}
}


  
 \newpage 
\subsection{[LVV-4903] CA-DM-CON-ICD-0007-V-03: Provide Data Management Conditions data\_DM\_3 }\label{lvv-4903}

\begin{longtable}{cccc}
\hline
\textbf{Jira Link} & \textbf{Assignee} & \textbf{Status} & \textbf{Test Cases}\\ \hline
\href{https://jira.lsstcorp.org/browse/LVV-4903}{LVV-4903} &
Leanne Guy & Not Covered &
\begin{tabular}{c}
\end{tabular}
\\
\hline
\end{longtable}

\textbf{Verification Element Description:} \\
Undefined

{\footnotesize
\begin{longtable}{p{2.5cm}p{13.5cm}}
\hline
\multicolumn{2}{c}{\textbf{Requirement Details}}\\ \hline
Requirement ID & CA-DM-CON-ICD-0007 \\ \cdashline{1-2}
Requirement Description &
\begin{minipage}[]{13cm}
\textbf{Specification:} DM shall generate and make available to the
Camera the Conditions data enumerated in document \citeds{LSE-130}.
\end{minipage}
\\ \cdashline{1-2}
Requirement Discussion &
\begin{minipage}[]{13cm}
\textbf{Discussion:} This is generally data quality information relevant
to evaluating the performance of the Camera (e.g., PSF's, WCS's,
vignetting maps, statistics on the raw data). This information may be
made available either as telemetry or, in certain cases, as data in the
DM science archive (e.g., synthetic calibration flats).\\
The Camera is required not to depend on this data for its day-to-day
operation. In particular, if the Summit-Base link is severed and live DM
analysis of new image data is impossible, the data quality information
will not be available.
\end{minipage}
\\ \cdashline{1-2}
Requirement Priority &  \\ \cdashline{1-2}
Upper Level Requirement &
\begin{tabular}{cl}
\end{tabular}
\\ \hline
\end{longtable}
}


  
 \newpage 
\subsection{[LVV-4904] CA-DM-CON-ICD-0007-V-04: Provide Data Management Conditions data\_DM\_4 }\label{lvv-4904}

\begin{longtable}{cccc}
\hline
\textbf{Jira Link} & \textbf{Assignee} & \textbf{Status} & \textbf{Test Cases}\\ \hline
\href{https://jira.lsstcorp.org/browse/LVV-4904}{LVV-4904} &
Leanne Guy & Not Covered &
\begin{tabular}{c}
\end{tabular}
\\
\hline
\end{longtable}

\textbf{Verification Element Description:} \\
Undefined

{\footnotesize
\begin{longtable}{p{2.5cm}p{13.5cm}}
\hline
\multicolumn{2}{c}{\textbf{Requirement Details}}\\ \hline
Requirement ID & CA-DM-CON-ICD-0007 \\ \cdashline{1-2}
Requirement Description &
\begin{minipage}[]{13cm}
\textbf{Specification:} DM shall generate and make available to the
Camera the Conditions data enumerated in document \citeds{LSE-130}.
\end{minipage}
\\ \cdashline{1-2}
Requirement Discussion &
\begin{minipage}[]{13cm}
\textbf{Discussion:} This is generally data quality information relevant
to evaluating the performance of the Camera (e.g., PSF's, WCS's,
vignetting maps, statistics on the raw data). This information may be
made available either as telemetry or, in certain cases, as data in the
DM science archive (e.g., synthetic calibration flats).\\
The Camera is required not to depend on this data for its day-to-day
operation. In particular, if the Summit-Base link is severed and live DM
analysis of new image data is impossible, the data quality information
will not be available.
\end{minipage}
\\ \cdashline{1-2}
Requirement Priority &  \\ \cdashline{1-2}
Upper Level Requirement &
\begin{tabular}{cl}
\end{tabular}
\\ \hline
\end{longtable}
}


  
 \newpage 
\subsection{[LVV-4909] CA-DM-CON-ICD-0016-V-03: Provide guide sensor data\_DM\_3 }\label{lvv-4909}

\begin{longtable}{cccc}
\hline
\textbf{Jira Link} & \textbf{Assignee} & \textbf{Status} & \textbf{Test Cases}\\ \hline
\href{https://jira.lsstcorp.org/browse/LVV-4909}{LVV-4909} &
Leanne Guy & Not Covered &
\begin{tabular}{c}
\end{tabular}
\\
\hline
\end{longtable}

\textbf{Verification Element Description:} \\
Undefined

{\footnotesize
\begin{longtable}{p{2.5cm}p{13.5cm}}
\hline
\multicolumn{2}{c}{\textbf{Requirement Details}}\\ \hline
Requirement ID & CA-DM-CON-ICD-0016 \\ \cdashline{1-2}
Requirement Description &
\begin{minipage}[]{13cm}
\textbf{Specification}: The Camera shall provide to Data Management the
raw data from the full set of guide sensors during calibration
operations, and any other operational modes that require guide sensor
data to be archived and/or processed by DM.
\end{minipage}
\\ \cdashline{1-2}
Requirement Priority &  \\ \cdashline{1-2}
Upper Level Requirement &
\begin{tabular}{cl}
\end{tabular}
\\ \hline
\end{longtable}
}


  
 \newpage 
\subsection{[LVV-4910] CA-DM-CON-ICD-0016-V-04: Provide guide sensor data\_DM\_4 }\label{lvv-4910}

\begin{longtable}{cccc}
\hline
\textbf{Jira Link} & \textbf{Assignee} & \textbf{Status} & \textbf{Test Cases}\\ \hline
\href{https://jira.lsstcorp.org/browse/LVV-4910}{LVV-4910} &
Leanne Guy & Not Covered &
\begin{tabular}{c}
\end{tabular}
\\
\hline
\end{longtable}

\textbf{Verification Element Description:} \\
Undefined

{\footnotesize
\begin{longtable}{p{2.5cm}p{13.5cm}}
\hline
\multicolumn{2}{c}{\textbf{Requirement Details}}\\ \hline
Requirement ID & CA-DM-CON-ICD-0016 \\ \cdashline{1-2}
Requirement Description &
\begin{minipage}[]{13cm}
\textbf{Specification}: The Camera shall provide to Data Management the
raw data from the full set of guide sensors during calibration
operations, and any other operational modes that require guide sensor
data to be archived and/or processed by DM.
\end{minipage}
\\ \cdashline{1-2}
Requirement Priority &  \\ \cdashline{1-2}
Upper Level Requirement &
\begin{tabular}{cl}
\end{tabular}
\\ \hline
\end{longtable}
}


  
 \newpage 
\subsection{[LVV-4915] CA-DM-CON-ICD-0014-V-03: Provide science sensor data\_DM\_3 }\label{lvv-4915}

\begin{longtable}{cccc}
\hline
\textbf{Jira Link} & \textbf{Assignee} & \textbf{Status} & \textbf{Test Cases}\\ \hline
\href{https://jira.lsstcorp.org/browse/LVV-4915}{LVV-4915} &
Leanne Guy & Not Covered &
\begin{tabular}{c}
\end{tabular}
\\
\hline
\end{longtable}

\textbf{Verification Element Description:} \\
Undefined

{\footnotesize
\begin{longtable}{p{2.5cm}p{13.5cm}}
\hline
\multicolumn{2}{c}{\textbf{Requirement Details}}\\ \hline
Requirement ID & CA-DM-CON-ICD-0014 \\ \cdashline{1-2}
Requirement Description &
\begin{minipage}[]{13cm}
\textbf{Specification:} The Camera shall provide to Data Management raw
data from the full science array during normal science and calibration
operations, and any other operational modes that require science array
data to be archived and/or processed by DM.
\end{minipage}
\\ \cdashline{1-2}
Requirement Priority &  \\ \cdashline{1-2}
Upper Level Requirement &
\begin{tabular}{cl}
\end{tabular}
\\ \hline
\end{longtable}
}


  
 \newpage 
\subsection{[LVV-4916] CA-DM-CON-ICD-0014-V-04: Provide science sensor data\_DM\_4 }\label{lvv-4916}

\begin{longtable}{cccc}
\hline
\textbf{Jira Link} & \textbf{Assignee} & \textbf{Status} & \textbf{Test Cases}\\ \hline
\href{https://jira.lsstcorp.org/browse/LVV-4916}{LVV-4916} &
Leanne Guy & Not Covered &
\begin{tabular}{c}
\end{tabular}
\\
\hline
\end{longtable}

\textbf{Verification Element Description:} \\
Undefined

{\footnotesize
\begin{longtable}{p{2.5cm}p{13.5cm}}
\hline
\multicolumn{2}{c}{\textbf{Requirement Details}}\\ \hline
Requirement ID & CA-DM-CON-ICD-0014 \\ \cdashline{1-2}
Requirement Description &
\begin{minipage}[]{13cm}
\textbf{Specification:} The Camera shall provide to Data Management raw
data from the full science array during normal science and calibration
operations, and any other operational modes that require science array
data to be archived and/or processed by DM.
\end{minipage}
\\ \cdashline{1-2}
Requirement Priority &  \\ \cdashline{1-2}
Upper Level Requirement &
\begin{tabular}{cl}
\end{tabular}
\\ \hline
\end{longtable}
}


  
 \newpage 
\subsection{[LVV-4921] CA-DM-CON-ICD-0015-V-03: Provide wavefront sensor data\_DM\_3 }\label{lvv-4921}

\begin{longtable}{cccc}
\hline
\textbf{Jira Link} & \textbf{Assignee} & \textbf{Status} & \textbf{Test Cases}\\ \hline
\href{https://jira.lsstcorp.org/browse/LVV-4921}{LVV-4921} &
Leanne Guy & Not Covered &
\begin{tabular}{c}
\end{tabular}
\\
\hline
\end{longtable}

\textbf{Verification Element Description:} \\
Undefined

{\footnotesize
\begin{longtable}{p{2.5cm}p{13.5cm}}
\hline
\multicolumn{2}{c}{\textbf{Requirement Details}}\\ \hline
Requirement ID & CA-DM-CON-ICD-0015 \\ \cdashline{1-2}
Requirement Description &
\begin{minipage}[]{13cm}
\textbf{Specification}: The Camera shall provide to Data Management the
raw data from the full set of wavefront sensors during normal science
and calibration operations, and any other operational modes that require
wavefront sensor data to be archived and/or processed by DM.
\end{minipage}
\\ \cdashline{1-2}
Requirement Priority &  \\ \cdashline{1-2}
Upper Level Requirement &
\begin{tabular}{cl}
\end{tabular}
\\ \hline
\end{longtable}
}


  
 \newpage 
\subsection{[LVV-4922] CA-DM-CON-ICD-0015-V-04: Provide wavefront sensor data\_DM\_4 }\label{lvv-4922}

\begin{longtable}{cccc}
\hline
\textbf{Jira Link} & \textbf{Assignee} & \textbf{Status} & \textbf{Test Cases}\\ \hline
\href{https://jira.lsstcorp.org/browse/LVV-4922}{LVV-4922} &
Leanne Guy & Not Covered &
\begin{tabular}{c}
\end{tabular}
\\
\hline
\end{longtable}

\textbf{Verification Element Description:} \\
Undefined

{\footnotesize
\begin{longtable}{p{2.5cm}p{13.5cm}}
\hline
\multicolumn{2}{c}{\textbf{Requirement Details}}\\ \hline
Requirement ID & CA-DM-CON-ICD-0015 \\ \cdashline{1-2}
Requirement Description &
\begin{minipage}[]{13cm}
\textbf{Specification}: The Camera shall provide to Data Management the
raw data from the full set of wavefront sensors during normal science
and calibration operations, and any other operational modes that require
wavefront sensor data to be archived and/or processed by DM.
\end{minipage}
\\ \cdashline{1-2}
Requirement Priority &  \\ \cdashline{1-2}
Upper Level Requirement &
\begin{tabular}{cl}
\end{tabular}
\\ \hline
\end{longtable}
}


  
 \newpage 
\subsection{[LVV-5237] OCS-DM-COM-ICD-0040-V-01: Command Completion Response\_DM\_1 }\label{lvv-5237}

\begin{longtable}{cccc}
\hline
\textbf{Jira Link} & \textbf{Assignee} & \textbf{Status} & \textbf{Test Cases}\\ \hline
\href{https://jira.lsstcorp.org/browse/LVV-5237}{LVV-5237} &
Leanne Guy & Not Covered &
\begin{tabular}{c}
\end{tabular}
\\
\hline
\end{longtable}

\textbf{Verification Element Description:} \\
Undefined

{\footnotesize
\begin{longtable}{p{2.5cm}p{13.5cm}}
\hline
\multicolumn{2}{c}{\textbf{Requirement Details}}\\ \hline
Requirement ID & OCS-DM-COM-ICD-0040 \\ \cdashline{1-2}
Requirement Description &
\begin{minipage}[]{13cm}
\textbf{Specification:} After the successful completion of any of the
above commands, a SummaryState event shall be published for the CSC
indicating which Top-Level subsystem state (as defined in \citeds{LSE-209}) it is
in.
\end{minipage}
\\ \cdashline{1-2}
Requirement Priority &  \\ \cdashline{1-2}
Upper Level Requirement &
\begin{tabular}{cl}
\end{tabular}
\\ \hline
\end{longtable}
}


  
 \newpage 
\subsection{[LVV-5238] OCS-DM-COM-ICD-0040-V-02: Command Completion Response\_DM\_2 }\label{lvv-5238}

\begin{longtable}{cccc}
\hline
\textbf{Jira Link} & \textbf{Assignee} & \textbf{Status} & \textbf{Test Cases}\\ \hline
\href{https://jira.lsstcorp.org/browse/LVV-5238}{LVV-5238} &
Leanne Guy & Not Covered &
\begin{tabular}{c}
\end{tabular}
\\
\hline
\end{longtable}

\textbf{Verification Element Description:} \\
Undefined

{\footnotesize
\begin{longtable}{p{2.5cm}p{13.5cm}}
\hline
\multicolumn{2}{c}{\textbf{Requirement Details}}\\ \hline
Requirement ID & OCS-DM-COM-ICD-0040 \\ \cdashline{1-2}
Requirement Description &
\begin{minipage}[]{13cm}
\textbf{Specification:} After the successful completion of any of the
above commands, a SummaryState event shall be published for the CSC
indicating which Top-Level subsystem state (as defined in \citeds{LSE-209}) it is
in.
\end{minipage}
\\ \cdashline{1-2}
Requirement Priority &  \\ \cdashline{1-2}
Upper Level Requirement &
\begin{tabular}{cl}
\end{tabular}
\\ \hline
\end{longtable}
}


  
 \newpage 
\subsection{[LVV-5243] OCS-DM-COM-ICD-0009-V-01: Command Set Implementation by Data
Management\_DM\_1 }\label{lvv-5243}

\begin{longtable}{cccc}
\hline
\textbf{Jira Link} & \textbf{Assignee} & \textbf{Status} & \textbf{Test Cases}\\ \hline
\href{https://jira.lsstcorp.org/browse/LVV-5243}{LVV-5243} &
Leanne Guy & Not Covered &
\begin{tabular}{c}
\end{tabular}
\\
\hline
\end{longtable}

\textbf{Verification Element Description:} \\
Undefined

{\footnotesize
\begin{longtable}{p{2.5cm}p{13.5cm}}
\hline
\multicolumn{2}{c}{\textbf{Requirement Details}}\\ \hline
Requirement ID & OCS-DM-COM-ICD-0009 \\ \cdashline{1-2}
Requirement Description &
\begin{minipage}[]{13cm}
\textbf{Specification}: Data Management shall provide implementations of
the basic commands required of all devices by the OCS, as defined in
\citeds{LSE-70} and \citeds{LSE-209}.
\end{minipage}
\\ \cdashline{1-2}
Requirement Discussion &
\begin{minipage}[]{13cm}
~
\end{minipage}
\\ \cdashline{1-2}
Requirement Priority &  \\ \cdashline{1-2}
Upper Level Requirement &
\begin{tabular}{cl}
OCS-DM-COM-ICD-0007 & Prompt Processing CSC \\
\end{tabular}
\\ \hline
\end{longtable}
}


  
 \newpage 
\subsection{[LVV-5244] OCS-DM-COM-ICD-0009-V-02: Command Set Implementation by Data
Management\_DM\_2 }\label{lvv-5244}

\begin{longtable}{cccc}
\hline
\textbf{Jira Link} & \textbf{Assignee} & \textbf{Status} & \textbf{Test Cases}\\ \hline
\href{https://jira.lsstcorp.org/browse/LVV-5244}{LVV-5244} &
Leanne Guy & Not Covered &
\begin{tabular}{c}
\end{tabular}
\\
\hline
\end{longtable}

\textbf{Verification Element Description:} \\
Undefined

{\footnotesize
\begin{longtable}{p{2.5cm}p{13.5cm}}
\hline
\multicolumn{2}{c}{\textbf{Requirement Details}}\\ \hline
Requirement ID & OCS-DM-COM-ICD-0009 \\ \cdashline{1-2}
Requirement Description &
\begin{minipage}[]{13cm}
\textbf{Specification}: Data Management shall provide implementations of
the basic commands required of all devices by the OCS, as defined in
\citeds{LSE-70} and \citeds{LSE-209}.
\end{minipage}
\\ \cdashline{1-2}
Requirement Discussion &
\begin{minipage}[]{13cm}
~
\end{minipage}
\\ \cdashline{1-2}
Requirement Priority &  \\ \cdashline{1-2}
Upper Level Requirement &
\begin{tabular}{cl}
OCS-DM-COM-ICD-0007 & Prompt Processing CSC \\
\end{tabular}
\\ \hline
\end{longtable}
}


  
 \newpage 
\subsection{[LVV-5249] OCS-DM-COM-ICD-0013-V-01: configure Successful Completion
Response\_DM\_1 }\label{lvv-5249}

\begin{longtable}{cccc}
\hline
\textbf{Jira Link} & \textbf{Assignee} & \textbf{Status} & \textbf{Test Cases}\\ \hline
\href{https://jira.lsstcorp.org/browse/LVV-5249}{LVV-5249} &
Leanne Guy & Not Covered &
\begin{tabular}{c}
\end{tabular}
\\
\hline
\end{longtable}

\textbf{Verification Element Description:} \\
Undefined

{\footnotesize
\begin{longtable}{p{2.5cm}p{13.5cm}}
\hline
\multicolumn{2}{c}{\textbf{Requirement Details}}\\ \hline
Requirement ID & OCS-DM-COM-ICD-0013 \\ \cdashline{1-2}
Requirement Description &
\begin{minipage}[]{13cm}
\textbf{Specification}: Successful completion of a start command shall
include the publication as a SettingsApplied event of:

\begin{itemize}
\tightlist
\item
  The configuration key (``alias'')
\item
  An immutable name for the configuration set applied (``permanent
  name'')
\item
  The content of the configuration\\
  An AppliedSettingsMatchStart event with parameter True will also be
  sent to indicate that the DM CSC's settings match those requested in
  the \textbf{start} command, as opposed to being manually adjusted to
  be something different.\\
  \hspace*{0.333em}
\end{itemize}
\end{minipage}
\\ \cdashline{1-2}
Requirement Priority &  \\ \cdashline{1-2}
Upper Level Requirement &
\begin{tabular}{cl}
\end{tabular}
\\ \hline
\end{longtable}
}


  
 \newpage 
\subsection{[LVV-5250] OCS-DM-COM-ICD-0013-V-02: configure Successful Completion
Response\_DM\_2 }\label{lvv-5250}

\begin{longtable}{cccc}
\hline
\textbf{Jira Link} & \textbf{Assignee} & \textbf{Status} & \textbf{Test Cases}\\ \hline
\href{https://jira.lsstcorp.org/browse/LVV-5250}{LVV-5250} &
Leanne Guy & Not Covered &
\begin{tabular}{c}
\end{tabular}
\\
\hline
\end{longtable}

\textbf{Verification Element Description:} \\
Undefined

{\footnotesize
\begin{longtable}{p{2.5cm}p{13.5cm}}
\hline
\multicolumn{2}{c}{\textbf{Requirement Details}}\\ \hline
Requirement ID & OCS-DM-COM-ICD-0013 \\ \cdashline{1-2}
Requirement Description &
\begin{minipage}[]{13cm}
\textbf{Specification}: Successful completion of a start command shall
include the publication as a SettingsApplied event of:

\begin{itemize}
\tightlist
\item
  The configuration key (``alias'')
\item
  An immutable name for the configuration set applied (``permanent
  name'')
\item
  The content of the configuration\\
  An AppliedSettingsMatchStart event with parameter True will also be
  sent to indicate that the DM CSC's settings match those requested in
  the \textbf{start} command, as opposed to being manually adjusted to
  be something different.\\
  \hspace*{0.333em}
\end{itemize}
\end{minipage}
\\ \cdashline{1-2}
Requirement Priority &  \\ \cdashline{1-2}
Upper Level Requirement &
\begin{tabular}{cl}
\end{tabular}
\\ \hline
\end{longtable}
}


  
 \newpage 
\subsection{[LVV-5255] OCS-DM-COM-ICD-0015-V-01: disable Command\_DM\_1 }\label{lvv-5255}

\begin{longtable}{cccc}
\hline
\textbf{Jira Link} & \textbf{Assignee} & \textbf{Status} & \textbf{Test Cases}\\ \hline
\href{https://jira.lsstcorp.org/browse/LVV-5255}{LVV-5255} &
Leanne Guy & Not Covered &
\begin{tabular}{c}
\end{tabular}
\\
\hline
\end{longtable}

\textbf{Verification Element Description:} \\
Undefined

{\footnotesize
\begin{longtable}{p{2.5cm}p{13.5cm}}
\hline
\multicolumn{2}{c}{\textbf{Requirement Details}}\\ \hline
Requirement ID & OCS-DM-COM-ICD-0015 \\ \cdashline{1-2}
Requirement Description &
\begin{minipage}[]{13cm}
\textbf{Specification}: Upon completion of the \textbf{disable} command,
a DM device shall cease the initiation of new actions in response to
events or timers.\\
A DM device may report the completion of \textbf{disable} as soon as it
has taken that step. It may still complete actions triggered by events
or timers that were received before \textbf{disable}. Specific devices?
behavior in this respect shall be documented.
\end{minipage}
\\ \cdashline{1-2}
Requirement Discussion &
\begin{minipage}[]{13cm}
\textbf{Parameters:} (none)\\
\textbf{Discussion}: This permits pipelined operation of image
processing to proceed even in a tight sequence of alternating
\textbf{configure} commands and images.
\end{minipage}
\\ \cdashline{1-2}
Requirement Priority &  \\ \cdashline{1-2}
Upper Level Requirement &
\begin{tabular}{cl}
\end{tabular}
\\ \hline
\end{longtable}
}


  
 \newpage 
\subsection{[LVV-5256] OCS-DM-COM-ICD-0015-V-02: disable Command\_DM\_2 }\label{lvv-5256}

\begin{longtable}{cccc}
\hline
\textbf{Jira Link} & \textbf{Assignee} & \textbf{Status} & \textbf{Test Cases}\\ \hline
\href{https://jira.lsstcorp.org/browse/LVV-5256}{LVV-5256} &
Leanne Guy & Not Covered &
\begin{tabular}{c}
\end{tabular}
\\
\hline
\end{longtable}

\textbf{Verification Element Description:} \\
Undefined

{\footnotesize
\begin{longtable}{p{2.5cm}p{13.5cm}}
\hline
\multicolumn{2}{c}{\textbf{Requirement Details}}\\ \hline
Requirement ID & OCS-DM-COM-ICD-0015 \\ \cdashline{1-2}
Requirement Description &
\begin{minipage}[]{13cm}
\textbf{Specification}: Upon completion of the \textbf{disable} command,
a DM device shall cease the initiation of new actions in response to
events or timers.\\
A DM device may report the completion of \textbf{disable} as soon as it
has taken that step. It may still complete actions triggered by events
or timers that were received before \textbf{disable}. Specific devices?
behavior in this respect shall be documented.
\end{minipage}
\\ \cdashline{1-2}
Requirement Discussion &
\begin{minipage}[]{13cm}
\textbf{Parameters:} (none)\\
\textbf{Discussion}: This permits pipelined operation of image
processing to proceed even in a tight sequence of alternating
\textbf{configure} commands and images.
\end{minipage}
\\ \cdashline{1-2}
Requirement Priority &  \\ \cdashline{1-2}
Upper Level Requirement &
\begin{tabular}{cl}
\end{tabular}
\\ \hline
\end{longtable}
}


  
 \newpage 
\subsection{[LVV-5261] OCS-DM-COM-ICD-0014-V-01: enable Command\_DM\_1 }\label{lvv-5261}

\begin{longtable}{cccc}
\hline
\textbf{Jira Link} & \textbf{Assignee} & \textbf{Status} & \textbf{Test Cases}\\ \hline
\href{https://jira.lsstcorp.org/browse/LVV-5261}{LVV-5261} &
Leanne Guy & Not Covered &
\begin{tabular}{c}
\end{tabular}
\\
\hline
\end{longtable}

\textbf{Verification Element Description:} \\
Undefined

{\footnotesize
\begin{longtable}{p{2.5cm}p{13.5cm}}
\hline
\multicolumn{2}{c}{\textbf{Requirement Details}}\\ \hline
Requirement ID & OCS-DM-COM-ICD-0014 \\ \cdashline{1-2}
Requirement Description &
\begin{minipage}[]{13cm}
\textbf{Parameters}: (none)\\
\textbf{Specification}: Upon completion of the \textbf{enable} command,
a DM device shall begin carrying out its configured function as driven
by its monitoring of Observatory events and/or in response to internal
timers and predicates.
\end{minipage}
\\ \cdashline{1-2}
Requirement Priority &  \\ \cdashline{1-2}
Upper Level Requirement &
\begin{tabular}{cl}
\end{tabular}
\\ \hline
\end{longtable}
}


  
 \newpage 
\subsection{[LVV-5262] OCS-DM-COM-ICD-0014-V-02: enable Command\_DM\_2 }\label{lvv-5262}

\begin{longtable}{cccc}
\hline
\textbf{Jira Link} & \textbf{Assignee} & \textbf{Status} & \textbf{Test Cases}\\ \hline
\href{https://jira.lsstcorp.org/browse/LVV-5262}{LVV-5262} &
Leanne Guy & Not Covered &
\begin{tabular}{c}
\end{tabular}
\\
\hline
\end{longtable}

\textbf{Verification Element Description:} \\
Undefined

{\footnotesize
\begin{longtable}{p{2.5cm}p{13.5cm}}
\hline
\multicolumn{2}{c}{\textbf{Requirement Details}}\\ \hline
Requirement ID & OCS-DM-COM-ICD-0014 \\ \cdashline{1-2}
Requirement Description &
\begin{minipage}[]{13cm}
\textbf{Parameters}: (none)\\
\textbf{Specification}: Upon completion of the \textbf{enable} command,
a DM device shall begin carrying out its configured function as driven
by its monitoring of Observatory events and/or in response to internal
timers and predicates.
\end{minipage}
\\ \cdashline{1-2}
Requirement Priority &  \\ \cdashline{1-2}
Upper Level Requirement &
\begin{tabular}{cl}
\end{tabular}
\\ \hline
\end{longtable}
}


  
 \newpage 
\subsection{[LVV-5267] OCS-DM-COM-ICD-0038-V-01: enterControl Command\_DM\_1 }\label{lvv-5267}

\begin{longtable}{cccc}
\hline
\textbf{Jira Link} & \textbf{Assignee} & \textbf{Status} & \textbf{Test Cases}\\ \hline
\href{https://jira.lsstcorp.org/browse/LVV-5267}{LVV-5267} &
Leanne Guy & Not Covered &
\begin{tabular}{c}
\end{tabular}
\\
\hline
\end{longtable}

\textbf{Verification Element Description:} \\
Undefined

{\footnotesize
\begin{longtable}{p{2.5cm}p{13.5cm}}
\hline
\multicolumn{2}{c}{\textbf{Requirement Details}}\\ \hline
Requirement ID & OCS-DM-COM-ICD-0038 \\ \cdashline{1-2}
Requirement Description &
\begin{minipage}[]{13cm}
\textbf{Specification}: Upon completion of the \textbf{enterControl}
command, a DM CSC shall enter the Standby state.
\end{minipage}
\\ \cdashline{1-2}
Requirement Priority &  \\ \cdashline{1-2}
Upper Level Requirement &
\begin{tabular}{cl}
\end{tabular}
\\ \hline
\end{longtable}
}


  
 \newpage 
\subsection{[LVV-5268] OCS-DM-COM-ICD-0038-V-02: enterControl Command\_DM\_2 }\label{lvv-5268}

\begin{longtable}{cccc}
\hline
\textbf{Jira Link} & \textbf{Assignee} & \textbf{Status} & \textbf{Test Cases}\\ \hline
\href{https://jira.lsstcorp.org/browse/LVV-5268}{LVV-5268} &
Leanne Guy & Not Covered &
\begin{tabular}{c}
\end{tabular}
\\
\hline
\end{longtable}

\textbf{Verification Element Description:} \\
Undefined

{\footnotesize
\begin{longtable}{p{2.5cm}p{13.5cm}}
\hline
\multicolumn{2}{c}{\textbf{Requirement Details}}\\ \hline
Requirement ID & OCS-DM-COM-ICD-0038 \\ \cdashline{1-2}
Requirement Description &
\begin{minipage}[]{13cm}
\textbf{Specification}: Upon completion of the \textbf{enterControl}
command, a DM CSC shall enter the Standby state.
\end{minipage}
\\ \cdashline{1-2}
Requirement Priority &  \\ \cdashline{1-2}
Upper Level Requirement &
\begin{tabular}{cl}
\end{tabular}
\\ \hline
\end{longtable}
}


  
 \newpage 
\subsection{[LVV-5273] OCS-DM-COM-ICD-0039-V-01: enterControl Successful Completion
Response\_DM\_1 }\label{lvv-5273}

\begin{longtable}{cccc}
\hline
\textbf{Jira Link} & \textbf{Assignee} & \textbf{Status} & \textbf{Test Cases}\\ \hline
\href{https://jira.lsstcorp.org/browse/LVV-5273}{LVV-5273} &
Leanne Guy & Not Covered &
\begin{tabular}{c}
\end{tabular}
\\
\hline
\end{longtable}

\textbf{Verification Element Description:} \\
Undefined

{\footnotesize
\begin{longtable}{p{2.5cm}p{13.5cm}}
\hline
\multicolumn{2}{c}{\textbf{Requirement Details}}\\ \hline
Requirement ID & OCS-DM-COM-ICD-0039 \\ \cdashline{1-2}
Requirement Description &
\begin{minipage}[]{13cm}
\textbf{Specification}: Successful completion of the
\textbf{enterControl} command shall include the publication of a
RecommendedSettingsVersions event containing a list of available opaque,
unique configuration keys and a list of configuration labels (or
aliases) and their corresponding opaque configuration keys.
\end{minipage}
\\ \cdashline{1-2}
Requirement Priority &  \\ \cdashline{1-2}
Upper Level Requirement &
\begin{tabular}{cl}
\end{tabular}
\\ \hline
\end{longtable}
}


  
 \newpage 
\subsection{[LVV-5274] OCS-DM-COM-ICD-0039-V-02: enterControl Successful Completion
Response\_DM\_2 }\label{lvv-5274}

\begin{longtable}{cccc}
\hline
\textbf{Jira Link} & \textbf{Assignee} & \textbf{Status} & \textbf{Test Cases}\\ \hline
\href{https://jira.lsstcorp.org/browse/LVV-5274}{LVV-5274} &
Leanne Guy & Not Covered &
\begin{tabular}{c}
\end{tabular}
\\
\hline
\end{longtable}

\textbf{Verification Element Description:} \\
Undefined

{\footnotesize
\begin{longtable}{p{2.5cm}p{13.5cm}}
\hline
\multicolumn{2}{c}{\textbf{Requirement Details}}\\ \hline
Requirement ID & OCS-DM-COM-ICD-0039 \\ \cdashline{1-2}
Requirement Description &
\begin{minipage}[]{13cm}
\textbf{Specification}: Successful completion of the
\textbf{enterControl} command shall include the publication of a
RecommendedSettingsVersions event containing a list of available opaque,
unique configuration keys and a list of configuration labels (or
aliases) and their corresponding opaque configuration keys.
\end{minipage}
\\ \cdashline{1-2}
Requirement Priority &  \\ \cdashline{1-2}
Upper Level Requirement &
\begin{tabular}{cl}
\end{tabular}
\\ \hline
\end{longtable}
}


  
 \newpage 
\subsection{[LVV-5279] OCS-DM-COM-ICD-0037-V-01: exit Command\_DM\_1 }\label{lvv-5279}

\begin{longtable}{cccc}
\hline
\textbf{Jira Link} & \textbf{Assignee} & \textbf{Status} & \textbf{Test Cases}\\ \hline
\href{https://jira.lsstcorp.org/browse/LVV-5279}{LVV-5279} &
Leanne Guy & Not Covered &
\begin{tabular}{c}
\end{tabular}
\\
\hline
\end{longtable}

\textbf{Verification Element Description:} \\
Undefined

{\footnotesize
\begin{longtable}{p{2.5cm}p{13.5cm}}
\hline
\multicolumn{2}{c}{\textbf{Requirement Details}}\\ \hline
Requirement ID & OCS-DM-COM-ICD-0037 \\ \cdashline{1-2}
Requirement Description &
\begin{minipage}[]{13cm}
\textbf{Specification}: Upon completion of the \textbf{exit} command, a
DM CSC shall return to the Available substate of the Offline state.
\end{minipage}
\\ \cdashline{1-2}
Requirement Priority &  \\ \cdashline{1-2}
Upper Level Requirement &
\begin{tabular}{cl}
\end{tabular}
\\ \hline
\end{longtable}
}


  
 \newpage 
\subsection{[LVV-5280] OCS-DM-COM-ICD-0037-V-02: exit Command\_DM\_2 }\label{lvv-5280}

\begin{longtable}{cccc}
\hline
\textbf{Jira Link} & \textbf{Assignee} & \textbf{Status} & \textbf{Test Cases}\\ \hline
\href{https://jira.lsstcorp.org/browse/LVV-5280}{LVV-5280} &
Leanne Guy & Not Covered &
\begin{tabular}{c}
\end{tabular}
\\
\hline
\end{longtable}

\textbf{Verification Element Description:} \\
Undefined

{\footnotesize
\begin{longtable}{p{2.5cm}p{13.5cm}}
\hline
\multicolumn{2}{c}{\textbf{Requirement Details}}\\ \hline
Requirement ID & OCS-DM-COM-ICD-0037 \\ \cdashline{1-2}
Requirement Description &
\begin{minipage}[]{13cm}
\textbf{Specification}: Upon completion of the \textbf{exit} command, a
DM CSC shall return to the Available substate of the Offline state.
\end{minipage}
\\ \cdashline{1-2}
Requirement Priority &  \\ \cdashline{1-2}
Upper Level Requirement &
\begin{tabular}{cl}
\end{tabular}
\\ \hline
\end{longtable}
}


  
 \newpage 
\subsection{[LVV-5285] OCS-DM-COM-ICD-0036-V-01: standby Command\_DM\_1 }\label{lvv-5285}

\begin{longtable}{cccc}
\hline
\textbf{Jira Link} & \textbf{Assignee} & \textbf{Status} & \textbf{Test Cases}\\ \hline
\href{https://jira.lsstcorp.org/browse/LVV-5285}{LVV-5285} &
Leanne Guy & Not Covered &
\begin{tabular}{c}
\end{tabular}
\\
\hline
\end{longtable}

\textbf{Verification Element Description:} \\
Undefined

{\footnotesize
\begin{longtable}{p{2.5cm}p{13.5cm}}
\hline
\multicolumn{2}{c}{\textbf{Requirement Details}}\\ \hline
Requirement ID & OCS-DM-COM-ICD-0036 \\ \cdashline{1-2}
Requirement Description &
\begin{minipage}[]{13cm}
\textbf{Specification}: Upon completion of the \textbf{standby} command,
a DM CSC shall return to the unconfigured, Standby state.
\end{minipage}
\\ \cdashline{1-2}
Requirement Priority &  \\ \cdashline{1-2}
Upper Level Requirement &
\begin{tabular}{cl}
\end{tabular}
\\ \hline
\end{longtable}
}


  
 \newpage 
\subsection{[LVV-5286] OCS-DM-COM-ICD-0036-V-02: standby Command\_DM\_2 }\label{lvv-5286}

\begin{longtable}{cccc}
\hline
\textbf{Jira Link} & \textbf{Assignee} & \textbf{Status} & \textbf{Test Cases}\\ \hline
\href{https://jira.lsstcorp.org/browse/LVV-5286}{LVV-5286} &
Leanne Guy & Not Covered &
\begin{tabular}{c}
\end{tabular}
\\
\hline
\end{longtable}

\textbf{Verification Element Description:} \\
Undefined

{\footnotesize
\begin{longtable}{p{2.5cm}p{13.5cm}}
\hline
\multicolumn{2}{c}{\textbf{Requirement Details}}\\ \hline
Requirement ID & OCS-DM-COM-ICD-0036 \\ \cdashline{1-2}
Requirement Description &
\begin{minipage}[]{13cm}
\textbf{Specification}: Upon completion of the \textbf{standby} command,
a DM CSC shall return to the unconfigured, Standby state.
\end{minipage}
\\ \cdashline{1-2}
Requirement Priority &  \\ \cdashline{1-2}
Upper Level Requirement &
\begin{tabular}{cl}
\end{tabular}
\\ \hline
\end{longtable}
}


  
 \newpage 
\subsection{[LVV-5291] OCS-DM-COM-ICD-0012-V-01: Start Command\_DM\_1 }\label{lvv-5291}

\begin{longtable}{cccc}
\hline
\textbf{Jira Link} & \textbf{Assignee} & \textbf{Status} & \textbf{Test Cases}\\ \hline
\href{https://jira.lsstcorp.org/browse/LVV-5291}{LVV-5291} &
Leanne Guy & Not Covered &
\begin{tabular}{c}
\end{tabular}
\\
\hline
\end{longtable}

\textbf{Verification Element Description:} \\
Undefined

{\footnotesize
\begin{longtable}{p{2.5cm}p{13.5cm}}
\hline
\multicolumn{2}{c}{\textbf{Requirement Details}}\\ \hline
Requirement ID & OCS-DM-COM-ICD-0012 \\ \cdashline{1-2}
Requirement Description &
\begin{minipage}[]{13cm}
\textbf{Specification}: The \textbf{Start} command shall cause a Data
Management CSC to set up the details of the behavior it is to perform
upon the receipt of the \textbf{enable} command.\\
Upon completion of \textbf{Start}, a DM CSC shall place itself in the DM
``disabled'' state and shall take no further action until receiving
\textbf{enable}.\\
If at any time following \textbf{Start} a DM CSC can no longer ensure
that its state is consistent with the commanded configuration, it shall
enter the OCS command-model ``ERROR'' state (as defined in \citeds{LSE-209}).\\
The \textbf{Start} command shall only be valid in the DM ``disabled''
state.\\
The translation of the configuration\_key alias to a specific set of
configuration details shall occur only at the time of execution of the
\textbf{Start} command. DM shall not attempt to follow any changes to
the meaning of the alias until the receipt of a subsequent
\textbf{Start} command.\\
\hspace*{0.333em}
\end{minipage}
\\ \cdashline{1-2}
Requirement Discussion &
\begin{minipage}[]{13cm}
\textbf{Parameters}: configuration\_key, type: opaque identifier
(string)\\
\textbf{Discussion}: This command, in effect, establishes the
operational mode that a DM CSC will be in.\\
It is expected that the configuration\_key will be an ``alias''
describing a mode and that the translation of that key to specific
details may evolve over time. The mapping from the ``alias'' to a
concrete set of details is under the control of DM.\\
\textbf{Start} can also take a key (``permanent name'') referring to a
specific, unchangeable, set of details.\\
\hspace*{0.333em}
\end{minipage}
\\ \cdashline{1-2}
Requirement Priority &  \\ \cdashline{1-2}
Upper Level Requirement &
\begin{tabular}{cl}
\end{tabular}
\\ \hline
\end{longtable}
}


  
 \newpage 
\subsection{[LVV-5292] OCS-DM-COM-ICD-0012-V-02: Start Command\_DM\_2 }\label{lvv-5292}

\begin{longtable}{cccc}
\hline
\textbf{Jira Link} & \textbf{Assignee} & \textbf{Status} & \textbf{Test Cases}\\ \hline
\href{https://jira.lsstcorp.org/browse/LVV-5292}{LVV-5292} &
Leanne Guy & Not Covered &
\begin{tabular}{c}
\end{tabular}
\\
\hline
\end{longtable}

\textbf{Verification Element Description:} \\
Undefined

{\footnotesize
\begin{longtable}{p{2.5cm}p{13.5cm}}
\hline
\multicolumn{2}{c}{\textbf{Requirement Details}}\\ \hline
Requirement ID & OCS-DM-COM-ICD-0012 \\ \cdashline{1-2}
Requirement Description &
\begin{minipage}[]{13cm}
\textbf{Specification}: The \textbf{Start} command shall cause a Data
Management CSC to set up the details of the behavior it is to perform
upon the receipt of the \textbf{enable} command.\\
Upon completion of \textbf{Start}, a DM CSC shall place itself in the DM
``disabled'' state and shall take no further action until receiving
\textbf{enable}.\\
If at any time following \textbf{Start} a DM CSC can no longer ensure
that its state is consistent with the commanded configuration, it shall
enter the OCS command-model ``ERROR'' state (as defined in \citeds{LSE-209}).\\
The \textbf{Start} command shall only be valid in the DM ``disabled''
state.\\
The translation of the configuration\_key alias to a specific set of
configuration details shall occur only at the time of execution of the
\textbf{Start} command. DM shall not attempt to follow any changes to
the meaning of the alias until the receipt of a subsequent
\textbf{Start} command.\\
\hspace*{0.333em}
\end{minipage}
\\ \cdashline{1-2}
Requirement Discussion &
\begin{minipage}[]{13cm}
\textbf{Parameters}: configuration\_key, type: opaque identifier
(string)\\
\textbf{Discussion}: This command, in effect, establishes the
operational mode that a DM CSC will be in.\\
It is expected that the configuration\_key will be an ``alias''
describing a mode and that the translation of that key to specific
details may evolve over time. The mapping from the ``alias'' to a
concrete set of details is under the control of DM.\\
\textbf{Start} can also take a key (``permanent name'') referring to a
specific, unchangeable, set of details.\\
\hspace*{0.333em}
\end{minipage}
\\ \cdashline{1-2}
Requirement Priority &  \\ \cdashline{1-2}
Upper Level Requirement &
\begin{tabular}{cl}
\end{tabular}
\\ \hline
\end{longtable}
}


  
 \newpage 
\subsection{[LVV-5297] OCS-DM-COM-ICD-0003-V-01: Data Management CSC Command Response
Model\_DM\_1 }\label{lvv-5297}

\begin{longtable}{cccc}
\hline
\textbf{Jira Link} & \textbf{Assignee} & \textbf{Status} & \textbf{Test Cases}\\ \hline
\href{https://jira.lsstcorp.org/browse/LVV-5297}{LVV-5297} &
Leanne Guy & Not Covered &
\begin{tabular}{c}
\end{tabular}
\\
\hline
\end{longtable}

\textbf{Verification Element Description:} \\
Undefined

{\footnotesize
\begin{longtable}{p{2.5cm}p{13.5cm}}
\hline
\multicolumn{2}{c}{\textbf{Requirement Details}}\\ \hline
Requirement ID & OCS-DM-COM-ICD-0003 \\ \cdashline{1-2}
Requirement Description &
\begin{minipage}[]{13cm}
\textbf{Specification}: Data Management shall receive and respond to
commands issued by the OCS using the Command/Action/Response model of
the SAL software packages, as described in \citeds{LSE-70} ``LSST Observatory
Control Communication Architecture and Protocol'' and \citeds{LSE-209} ``Software
Component to OCS Interface Control Document''.\\
\hspace*{0.333em}
\end{minipage}
\\ \cdashline{1-2}
Requirement Priority &  \\ \cdashline{1-2}
Upper Level Requirement &
\begin{tabular}{cl}
\end{tabular}
\\ \hline
\end{longtable}
}


  
 \newpage 
\subsection{[LVV-5298] OCS-DM-COM-ICD-0003-V-02: Data Management CSC Command Response
Model\_DM\_2 }\label{lvv-5298}

\begin{longtable}{cccc}
\hline
\textbf{Jira Link} & \textbf{Assignee} & \textbf{Status} & \textbf{Test Cases}\\ \hline
\href{https://jira.lsstcorp.org/browse/LVV-5298}{LVV-5298} &
Leanne Guy & Not Covered &
\begin{tabular}{c}
\end{tabular}
\\
\hline
\end{longtable}

\textbf{Verification Element Description:} \\
Undefined

{\footnotesize
\begin{longtable}{p{2.5cm}p{13.5cm}}
\hline
\multicolumn{2}{c}{\textbf{Requirement Details}}\\ \hline
Requirement ID & OCS-DM-COM-ICD-0003 \\ \cdashline{1-2}
Requirement Description &
\begin{minipage}[]{13cm}
\textbf{Specification}: Data Management shall receive and respond to
commands issued by the OCS using the Command/Action/Response model of
the SAL software packages, as described in \citeds{LSE-70} ``LSST Observatory
Control Communication Architecture and Protocol'' and \citeds{LSE-209} ``Software
Component to OCS Interface Control Document''.\\
\hspace*{0.333em}
\end{minipage}
\\ \cdashline{1-2}
Requirement Priority &  \\ \cdashline{1-2}
Upper Level Requirement &
\begin{tabular}{cl}
\end{tabular}
\\ \hline
\end{longtable}
}


  
 \newpage 
\subsection{[LVV-5303] OCS-DM-COM-ICD-0034-V-01: Auxiliary Header Service CSC\_DM\_1 }\label{lvv-5303}

\begin{longtable}{cccc}
\hline
\textbf{Jira Link} & \textbf{Assignee} & \textbf{Status} & \textbf{Test Cases}\\ \hline
\href{https://jira.lsstcorp.org/browse/LVV-5303}{LVV-5303} &
Leanne Guy & Not Covered &
\begin{tabular}{c}
\end{tabular}
\\
\hline
\end{longtable}

\textbf{Verification Element Description:} \\
Undefined

{\footnotesize
\begin{longtable}{p{2.5cm}p{13.5cm}}
\hline
\multicolumn{2}{c}{\textbf{Requirement Details}}\\ \hline
Requirement ID & OCS-DM-COM-ICD-0034 \\ \cdashline{1-2}
Requirement Description &
\begin{minipage}[]{13cm}
\textbf{Specification}: The Auxiliary Header Service CSC shall perform
the same function as the Header Service but for the Auxiliary Telescope
and the Auxiliary Telescope Spectrograph.
\end{minipage}
\\ \cdashline{1-2}
Requirement Priority &  \\ \cdashline{1-2}
Upper Level Requirement &
\begin{tabular}{cl}
\end{tabular}
\\ \hline
\end{longtable}
}


  
 \newpage 
\subsection{[LVV-5304] OCS-DM-COM-ICD-0034-V-02: Auxiliary Header Service CSC\_DM\_2 }\label{lvv-5304}

\begin{longtable}{cccc}
\hline
\textbf{Jira Link} & \textbf{Assignee} & \textbf{Status} & \textbf{Test Cases}\\ \hline
\href{https://jira.lsstcorp.org/browse/LVV-5304}{LVV-5304} &
Leanne Guy & Not Covered &
\begin{tabular}{c}
\end{tabular}
\\
\hline
\end{longtable}

\textbf{Verification Element Description:} \\
Undefined

{\footnotesize
\begin{longtable}{p{2.5cm}p{13.5cm}}
\hline
\multicolumn{2}{c}{\textbf{Requirement Details}}\\ \hline
Requirement ID & OCS-DM-COM-ICD-0034 \\ \cdashline{1-2}
Requirement Description &
\begin{minipage}[]{13cm}
\textbf{Specification}: The Auxiliary Header Service CSC shall perform
the same function as the Header Service but for the Auxiliary Telescope
and the Auxiliary Telescope Spectrograph.
\end{minipage}
\\ \cdashline{1-2}
Requirement Priority &  \\ \cdashline{1-2}
Upper Level Requirement &
\begin{tabular}{cl}
\end{tabular}
\\ \hline
\end{longtable}
}


  
 \newpage 
\subsection{[LVV-5309] OCS-DM-COM-ICD-0032-V-01: Auxiliary Telescope Archiver CSC\_DM\_1 }\label{lvv-5309}

\begin{longtable}{cccc}
\hline
\textbf{Jira Link} & \textbf{Assignee} & \textbf{Status} & \textbf{Test Cases}\\ \hline
\href{https://jira.lsstcorp.org/browse/LVV-5309}{LVV-5309} &
Leanne Guy & Not Covered &
\begin{tabular}{c}
\end{tabular}
\\
\hline
\end{longtable}

\textbf{Verification Element Description:} \\
Undefined

{\footnotesize
\begin{longtable}{p{2.5cm}p{13.5cm}}
\hline
\multicolumn{2}{c}{\textbf{Requirement Details}}\\ \hline
Requirement ID & OCS-DM-COM-ICD-0032 \\ \cdashline{1-2}
Requirement Description &
\begin{minipage}[]{13cm}
\textbf{Specification:} The Auxiliary Telescope Archiver CSC shall
control the ingest and archiving of image data from the Auxiliary
Telescope.~ The data is to be fetched from a separate Camera Data
Acquisition (Camera Data System) unit built specifically for the
Auxiliary Telescope System and is expected to be used as a spectrograph.
\end{minipage}
\\ \cdashline{1-2}
Requirement Discussion &
\begin{minipage}[]{13cm}
\textbf{Discussion}: The Auxiliary Telescope Archiver CSC will provide
the same control interface as the other CSCs listed in this document.~
Its nightly task, however, is much simpler than the Main Telescope CSCs,
as it fetches exactly one CCD of image data.
\end{minipage}
\\ \cdashline{1-2}
Requirement Priority &  \\ \cdashline{1-2}
Upper Level Requirement &
\begin{tabular}{cl}
\end{tabular}
\\ \hline
\end{longtable}
}


  
 \newpage 
\subsection{[LVV-5310] OCS-DM-COM-ICD-0032-V-02: Auxiliary Telescope Archiver CSC\_DM\_2 }\label{lvv-5310}

\begin{longtable}{cccc}
\hline
\textbf{Jira Link} & \textbf{Assignee} & \textbf{Status} & \textbf{Test Cases}\\ \hline
\href{https://jira.lsstcorp.org/browse/LVV-5310}{LVV-5310} &
Leanne Guy & Not Covered &
\begin{tabular}{c}
\end{tabular}
\\
\hline
\end{longtable}

\textbf{Verification Element Description:} \\
Undefined

{\footnotesize
\begin{longtable}{p{2.5cm}p{13.5cm}}
\hline
\multicolumn{2}{c}{\textbf{Requirement Details}}\\ \hline
Requirement ID & OCS-DM-COM-ICD-0032 \\ \cdashline{1-2}
Requirement Description &
\begin{minipage}[]{13cm}
\textbf{Specification:} The Auxiliary Telescope Archiver CSC shall
control the ingest and archiving of image data from the Auxiliary
Telescope.~ The data is to be fetched from a separate Camera Data
Acquisition (Camera Data System) unit built specifically for the
Auxiliary Telescope System and is expected to be used as a spectrograph.
\end{minipage}
\\ \cdashline{1-2}
Requirement Discussion &
\begin{minipage}[]{13cm}
\textbf{Discussion}: The Auxiliary Telescope Archiver CSC will provide
the same control interface as the other CSCs listed in this document.~
Its nightly task, however, is much simpler than the Main Telescope CSCs,
as it fetches exactly one CCD of image data.
\end{minipage}
\\ \cdashline{1-2}
Requirement Priority &  \\ \cdashline{1-2}
Upper Level Requirement &
\begin{tabular}{cl}
\end{tabular}
\\ \hline
\end{longtable}
}


  
 \newpage 
\subsection{[LVV-5315] OCS-DM-COM-ICD-0006-V-01: Catch-up Archiver\_DM\_1 }\label{lvv-5315}

\begin{longtable}{cccc}
\hline
\textbf{Jira Link} & \textbf{Assignee} & \textbf{Status} & \textbf{Test Cases}\\ \hline
\href{https://jira.lsstcorp.org/browse/LVV-5315}{LVV-5315} &
Leanne Guy & Not Covered &
\begin{tabular}{c}
\end{tabular}
\\
\hline
\end{longtable}

\textbf{Verification Element Description:} \\
Undefined

{\footnotesize
\begin{longtable}{p{2.5cm}p{13.5cm}}
\hline
\multicolumn{2}{c}{\textbf{Requirement Details}}\\ \hline
Requirement ID & OCS-DM-COM-ICD-0006 \\ \cdashline{1-2}
Requirement Description &
\begin{minipage}[]{13cm}
\textbf{Specification}: The Catch-up Archiver device shall control the
process of acquisition of backlogs of image data from the Camera data
buffer by Data Management, and its transfer to storage at the Base and
Archive Centers. The configuration mechanism (see requirement
OCS-DM-COM-ICD-0012 below) shall be used to control which components of
the focal plane shall have their data requested and archived by DM.
\end{minipage}
\\ \cdashline{1-2}
Requirement Discussion &
\begin{minipage}[]{13cm}
\textbf{Discussion}: Following an enable command, the Catch-Up Archiver
will interrogate the Camera for the availability of old data in the
Camera buffer, retrieve that data, archive it, and make it available for
processing.\\
\hspace*{0.333em}\\
In general, the Catch-Up Archiver will attempt to follow the
configurations that were -- or would have been -- in force during the
acquisition of the data in order to determine what actions to take. This
information will be retrieved from the EFD and potentially also from the
image data itself.\\
\hspace*{0.333em}\\
DM must, if possible, apply the configuration key interpretation that
would have been in force at the time the start command for the Archiver
CSC was or would have been issued, unless explicitly overridden.\\
\hspace*{0.333em}\\
The configuration of the Catch-up Archiver will include a setting that
allows it to determine what buffered Camera data is actually eligible
for archiving.\\
\hspace*{0.333em}\\
The Catch-Up Archiver will report on the success or failure of the
archiving of each image via an OCS event that includes the Camera image
identifier and, if applicable, the corresponding visit identifier. These
reports will be similar in form to, but on a distinct channel from, the
reports from the Archiver.\\
\hspace*{0.333em}
\end{minipage}
\\ \cdashline{1-2}
Requirement Priority &  \\ \cdashline{1-2}
Upper Level Requirement &
\begin{tabular}{cl}
\end{tabular}
\\ \hline
\end{longtable}
}


  
 \newpage 
\subsection{[LVV-5316] OCS-DM-COM-ICD-0006-V-02: Catch-up Archiver\_DM\_2 }\label{lvv-5316}

\begin{longtable}{cccc}
\hline
\textbf{Jira Link} & \textbf{Assignee} & \textbf{Status} & \textbf{Test Cases}\\ \hline
\href{https://jira.lsstcorp.org/browse/LVV-5316}{LVV-5316} &
Leanne Guy & Not Covered &
\begin{tabular}{c}
\end{tabular}
\\
\hline
\end{longtable}

\textbf{Verification Element Description:} \\
Undefined

{\footnotesize
\begin{longtable}{p{2.5cm}p{13.5cm}}
\hline
\multicolumn{2}{c}{\textbf{Requirement Details}}\\ \hline
Requirement ID & OCS-DM-COM-ICD-0006 \\ \cdashline{1-2}
Requirement Description &
\begin{minipage}[]{13cm}
\textbf{Specification}: The Catch-up Archiver device shall control the
process of acquisition of backlogs of image data from the Camera data
buffer by Data Management, and its transfer to storage at the Base and
Archive Centers. The configuration mechanism (see requirement
OCS-DM-COM-ICD-0012 below) shall be used to control which components of
the focal plane shall have their data requested and archived by DM.
\end{minipage}
\\ \cdashline{1-2}
Requirement Discussion &
\begin{minipage}[]{13cm}
\textbf{Discussion}: Following an enable command, the Catch-Up Archiver
will interrogate the Camera for the availability of old data in the
Camera buffer, retrieve that data, archive it, and make it available for
processing.\\
\hspace*{0.333em}\\
In general, the Catch-Up Archiver will attempt to follow the
configurations that were -- or would have been -- in force during the
acquisition of the data in order to determine what actions to take. This
information will be retrieved from the EFD and potentially also from the
image data itself.\\
\hspace*{0.333em}\\
DM must, if possible, apply the configuration key interpretation that
would have been in force at the time the start command for the Archiver
CSC was or would have been issued, unless explicitly overridden.\\
\hspace*{0.333em}\\
The configuration of the Catch-up Archiver will include a setting that
allows it to determine what buffered Camera data is actually eligible
for archiving.\\
\hspace*{0.333em}\\
The Catch-Up Archiver will report on the success or failure of the
archiving of each image via an OCS event that includes the Camera image
identifier and, if applicable, the corresponding visit identifier. These
reports will be similar in form to, but on a distinct channel from, the
reports from the Archiver.\\
\hspace*{0.333em}
\end{minipage}
\\ \cdashline{1-2}
Requirement Priority &  \\ \cdashline{1-2}
Upper Level Requirement &
\begin{tabular}{cl}
\end{tabular}
\\ \hline
\end{longtable}
}


  
 \newpage 
\subsection{[LVV-5321] OCS-DM-COM-ICD-0004-V-01: Data Management Exposed CSCs\_DM\_1 }\label{lvv-5321}

\begin{longtable}{cccc}
\hline
\textbf{Jira Link} & \textbf{Assignee} & \textbf{Status} & \textbf{Test Cases}\\ \hline
\href{https://jira.lsstcorp.org/browse/LVV-5321}{LVV-5321} &
Leanne Guy & Not Covered &
\begin{tabular}{c}
\end{tabular}
\\
\hline
\end{longtable}

\textbf{Verification Element Description:} \\
Undefined

{\footnotesize
\begin{longtable}{p{2.5cm}p{13.5cm}}
\hline
\multicolumn{2}{c}{\textbf{Requirement Details}}\\ \hline
Requirement ID & OCS-DM-COM-ICD-0004 \\ \cdashline{1-2}
Requirement Description &
\begin{minipage}[]{13cm}
\textbf{Specification}: Data Management shall expose eight CSCs to the
OCS: Archiver, Catch-Up Archiver, Prompt Processing, Auxiliary Telescope
Archiver, EFD Transformation Service, Header Service, Auxiliary Header
Service, and OCS-Driven Batch.\\
\hspace*{0.333em}
\end{minipage}
\\ \cdashline{1-2}
Requirement Discussion &
\begin{minipage}[]{13cm}
\textbf{Discussion}: The behavior of the CSCs in detail is in document
\citeds{LDM-230}. It is briefly summarized in descriptive language in the
Discussion sections below for readability, but does not form a normative
part of this ICD.\\
\hspace*{0.333em}\\
Part of the point of the DM CSC model and the enable/disable protocol is
to allow the OCS, and the Observatory operator, to stop DM from
interacting with the other Observatory components when required for
engineering or diagnostic activities. E.g., it allows DM to be prevented
from attempting to retrieve images from the Camera or interacting with
the EFD query interface.\\
\hspace*{0.333em}
\end{minipage}
\\ \cdashline{1-2}
Requirement Priority &  \\ \cdashline{1-2}
Upper Level Requirement &
\begin{tabular}{cl}
\end{tabular}
\\ \hline
\end{longtable}
}


  
 \newpage 
\subsection{[LVV-5322] OCS-DM-COM-ICD-0004-V-02: Data Management Exposed CSCs\_DM\_2 }\label{lvv-5322}

\begin{longtable}{cccc}
\hline
\textbf{Jira Link} & \textbf{Assignee} & \textbf{Status} & \textbf{Test Cases}\\ \hline
\href{https://jira.lsstcorp.org/browse/LVV-5322}{LVV-5322} &
Leanne Guy & Not Covered &
\begin{tabular}{c}
\end{tabular}
\\
\hline
\end{longtable}

\textbf{Verification Element Description:} \\
Undefined

{\footnotesize
\begin{longtable}{p{2.5cm}p{13.5cm}}
\hline
\multicolumn{2}{c}{\textbf{Requirement Details}}\\ \hline
Requirement ID & OCS-DM-COM-ICD-0004 \\ \cdashline{1-2}
Requirement Description &
\begin{minipage}[]{13cm}
\textbf{Specification}: Data Management shall expose eight CSCs to the
OCS: Archiver, Catch-Up Archiver, Prompt Processing, Auxiliary Telescope
Archiver, EFD Transformation Service, Header Service, Auxiliary Header
Service, and OCS-Driven Batch.\\
\hspace*{0.333em}
\end{minipage}
\\ \cdashline{1-2}
Requirement Discussion &
\begin{minipage}[]{13cm}
\textbf{Discussion}: The behavior of the CSCs in detail is in document
\citeds{LDM-230}. It is briefly summarized in descriptive language in the
Discussion sections below for readability, but does not form a normative
part of this ICD.\\
\hspace*{0.333em}\\
Part of the point of the DM CSC model and the enable/disable protocol is
to allow the OCS, and the Observatory operator, to stop DM from
interacting with the other Observatory components when required for
engineering or diagnostic activities. E.g., it allows DM to be prevented
from attempting to retrieve images from the Camera or interacting with
the EFD query interface.\\
\hspace*{0.333em}
\end{minipage}
\\ \cdashline{1-2}
Requirement Priority &  \\ \cdashline{1-2}
Upper Level Requirement &
\begin{tabular}{cl}
\end{tabular}
\\ \hline
\end{longtable}
}


  
 \newpage 
\subsection{[LVV-5327] OCS-DM-COM-ICD-0008-V-01: EFD Transformation Service CSC\_DM\_1 }\label{lvv-5327}

\begin{longtable}{cccc}
\hline
\textbf{Jira Link} & \textbf{Assignee} & \textbf{Status} & \textbf{Test Cases}\\ \hline
\href{https://jira.lsstcorp.org/browse/LVV-5327}{LVV-5327} &
Leanne Guy & Not Covered &
\begin{tabular}{c}
\end{tabular}
\\
\hline
\end{longtable}

\textbf{Verification Element Description:} \\
Undefined

{\footnotesize
\begin{longtable}{p{2.5cm}p{13.5cm}}
\hline
\multicolumn{2}{c}{\textbf{Requirement Details}}\\ \hline
Requirement ID & OCS-DM-COM-ICD-0008 \\ \cdashline{1-2}
Requirement Description &
\begin{minipage}[]{13cm}
\textbf{Specification}: The EFD Transformation Service CSC shall control
the transformation of the Engineering and Facilities Database to
archival versions, including all content, at the Base and Archive
Centers, as required under the ``Engineering and Facilities Database
Archiving'' section below.\\
\hspace*{0.333em}
\end{minipage}
\\ \cdashline{1-2}
Requirement Discussion &
\begin{minipage}[]{13cm}
\textbf{Discussion}: The expectation is that this CSC will be enabled at
all times when any Observatory component is active and generating
telemetry. The CSC model is provided in order to provide a specific
channel for reporting to the OCS when the replication has failed, as
well as to allow turning it on and off for debugging purposes.\\
\hspace*{0.333em}
\end{minipage}
\\ \cdashline{1-2}
Requirement Priority &  \\ \cdashline{1-2}
Upper Level Requirement &
\begin{tabular}{cl}
\end{tabular}
\\ \hline
\end{longtable}
}


  
 \newpage 
\subsection{[LVV-5328] OCS-DM-COM-ICD-0008-V-02: EFD Transformation Service CSC\_DM\_2 }\label{lvv-5328}

\begin{longtable}{cccc}
\hline
\textbf{Jira Link} & \textbf{Assignee} & \textbf{Status} & \textbf{Test Cases}\\ \hline
\href{https://jira.lsstcorp.org/browse/LVV-5328}{LVV-5328} &
Leanne Guy & Not Covered &
\begin{tabular}{c}
\end{tabular}
\\
\hline
\end{longtable}

\textbf{Verification Element Description:} \\
Undefined

{\footnotesize
\begin{longtable}{p{2.5cm}p{13.5cm}}
\hline
\multicolumn{2}{c}{\textbf{Requirement Details}}\\ \hline
Requirement ID & OCS-DM-COM-ICD-0008 \\ \cdashline{1-2}
Requirement Description &
\begin{minipage}[]{13cm}
\textbf{Specification}: The EFD Transformation Service CSC shall control
the transformation of the Engineering and Facilities Database to
archival versions, including all content, at the Base and Archive
Centers, as required under the ``Engineering and Facilities Database
Archiving'' section below.\\
\hspace*{0.333em}
\end{minipage}
\\ \cdashline{1-2}
Requirement Discussion &
\begin{minipage}[]{13cm}
\textbf{Discussion}: The expectation is that this CSC will be enabled at
all times when any Observatory component is active and generating
telemetry. The CSC model is provided in order to provide a specific
channel for reporting to the OCS when the replication has failed, as
well as to allow turning it on and off for debugging purposes.\\
\hspace*{0.333em}
\end{minipage}
\\ \cdashline{1-2}
Requirement Priority &  \\ \cdashline{1-2}
Upper Level Requirement &
\begin{tabular}{cl}
\end{tabular}
\\ \hline
\end{longtable}
}


  
 \newpage 
\subsection{[LVV-5333] OCS-DM-COM-ICD-0033-V-01: Header Service CSC\_DM\_1 }\label{lvv-5333}

\begin{longtable}{cccc}
\hline
\textbf{Jira Link} & \textbf{Assignee} & \textbf{Status} & \textbf{Test Cases}\\ \hline
\href{https://jira.lsstcorp.org/browse/LVV-5333}{LVV-5333} &
Leanne Guy & Not Covered &
\begin{tabular}{c}
\end{tabular}
\\
\hline
\end{longtable}

\textbf{Verification Element Description:} \\
Undefined

{\footnotesize
\begin{longtable}{p{2.5cm}p{13.5cm}}
\hline
\multicolumn{2}{c}{\textbf{Requirement Details}}\\ \hline
Requirement ID & OCS-DM-COM-ICD-0033 \\ \cdashline{1-2}
Requirement Description &
\begin{minipage}[]{13cm}
\textbf{Specification}: The Header Service CSC will operate within the
Summit instance of the EFD.~ It shall monitor the state of the Main
Telescope system and its cameras via events and telemetry and persist
that state as an EFD Large File Annex entry for each image readout.
\end{minipage}
\\ \cdashline{1-2}
Requirement Priority &  \\ \cdashline{1-2}
Upper Level Requirement &
\begin{tabular}{cl}
\end{tabular}
\\ \hline
\end{longtable}
}


  
 \newpage 
\subsection{[LVV-5334] OCS-DM-COM-ICD-0033-V-02: Header Service CSC\_DM\_2 }\label{lvv-5334}

\begin{longtable}{cccc}
\hline
\textbf{Jira Link} & \textbf{Assignee} & \textbf{Status} & \textbf{Test Cases}\\ \hline
\href{https://jira.lsstcorp.org/browse/LVV-5334}{LVV-5334} &
Leanne Guy & Not Covered &
\begin{tabular}{c}
\end{tabular}
\\
\hline
\end{longtable}

\textbf{Verification Element Description:} \\
Undefined

{\footnotesize
\begin{longtable}{p{2.5cm}p{13.5cm}}
\hline
\multicolumn{2}{c}{\textbf{Requirement Details}}\\ \hline
Requirement ID & OCS-DM-COM-ICD-0033 \\ \cdashline{1-2}
Requirement Description &
\begin{minipage}[]{13cm}
\textbf{Specification}: The Header Service CSC will operate within the
Summit instance of the EFD.~ It shall monitor the state of the Main
Telescope system and its cameras via events and telemetry and persist
that state as an EFD Large File Annex entry for each image readout.
\end{minipage}
\\ \cdashline{1-2}
Requirement Priority &  \\ \cdashline{1-2}
Upper Level Requirement &
\begin{tabular}{cl}
\end{tabular}
\\ \hline
\end{longtable}
}


  
 \newpage 
\subsection{[LVV-5339] OCS-DM-COM-ICD-0005-V-01: Main Camera Archiver\_DM\_1 }\label{lvv-5339}

\begin{longtable}{cccc}
\hline
\textbf{Jira Link} & \textbf{Assignee} & \textbf{Status} & \textbf{Test Cases}\\ \hline
\href{https://jira.lsstcorp.org/browse/LVV-5339}{LVV-5339} &
Leanne Guy & Not Covered &
\begin{tabular}{c}
\end{tabular}
\\
\hline
\end{longtable}

\textbf{Verification Element Description:} \\
Undefined

{\footnotesize
\begin{longtable}{p{2.5cm}p{13.5cm}}
\hline
\multicolumn{2}{c}{\textbf{Requirement Details}}\\ \hline
Requirement ID & OCS-DM-COM-ICD-0005 \\ \cdashline{1-2}
Requirement Description &
\begin{minipage}[]{13cm}
\textbf{Specifications}:\\
The Main Camera Archiving Device (MCAD) shall control the process of
acquisition of raw data from the ComCam, LSST Camera and Auxiliary
Camera by Data Management (DM), as well as header and other relevant
information from OCS. These data are built into files that are ingested
into the Data Backbone.\\
\hspace*{0.333em}\\
The MCAD shall be presented with parameters specified in ID:
OCS-DM-COM-ICD-0031 to classify the files for storage in the Data
Backbone. The normal operation of the MCAD is to acquire and ingest raw
data, metadata and the information necessary for organizing files in the
Data Backbone.\\
\hspace*{0.333em}\\
The MCAD throughput shall be sized to TBD.\\
\hspace*{0.333em}\\
The MCAD shall be able to handle the case where the rate of pixel
generation exceeds the capacity for archive ingest (e.g. the case of
bias generation).\\
\hspace*{0.333em}\\
It shall be possible to determine which exposures have been archived and
which have not.
\end{minipage}
\\ \cdashline{1-2}
Requirement Discussion &
\begin{minipage}[]{13cm}
\textbf{Discussion}: Following an enable command, the Catch-Up Archiver
will interrogate the Camera for the availability of old data in the
Camera buffer, retrieve that data, archive it, and make it available for
processing.\\
\hspace*{0.333em}\\
In general, the Catch-Up Archiver will attempt to follow the
configurations that were -- or would have been -- in force during the
acquisition of the data in order to determine what actions to take. This
information will be retrieved from the EFD and potentially also from the
image data itself.\\
\hspace*{0.333em}\\
DM must, if possible, apply the configuration key interpretation that
would have been in force at the time the start command for the Archiver
CSC was or would have been issued, unless explicitly overridden.\\
\hspace*{0.333em}\\
The configuration of the Catch-up Archiver will include a setting that
allows it to determine what buffered Camera data is actually eligible
for archiving.\\
\hspace*{0.333em}\\
The Catch-Up Archiver will report on the success or failure of the
archiving of each image via an OCS event that includes the Camera image
identifier and, if applicable, the corresponding visit identifier. These
reports will be similar in form to, but on a distinct channel from, the
reports from the Archiver.
\end{minipage}
\\ \cdashline{1-2}
Requirement Priority &  \\ \cdashline{1-2}
Upper Level Requirement &
\begin{tabular}{cl}
\end{tabular}
\\ \hline
\end{longtable}
}


  
 \newpage 
\subsection{[LVV-5340] OCS-DM-COM-ICD-0005-V-02: Main Camera Archiver\_DM\_2 }\label{lvv-5340}

\begin{longtable}{cccc}
\hline
\textbf{Jira Link} & \textbf{Assignee} & \textbf{Status} & \textbf{Test Cases}\\ \hline
\href{https://jira.lsstcorp.org/browse/LVV-5340}{LVV-5340} &
Leanne Guy & Not Covered &
\begin{tabular}{c}
\end{tabular}
\\
\hline
\end{longtable}

\textbf{Verification Element Description:} \\
Undefined

{\footnotesize
\begin{longtable}{p{2.5cm}p{13.5cm}}
\hline
\multicolumn{2}{c}{\textbf{Requirement Details}}\\ \hline
Requirement ID & OCS-DM-COM-ICD-0005 \\ \cdashline{1-2}
Requirement Description &
\begin{minipage}[]{13cm}
\textbf{Specifications}:\\
The Main Camera Archiving Device (MCAD) shall control the process of
acquisition of raw data from the ComCam, LSST Camera and Auxiliary
Camera by Data Management (DM), as well as header and other relevant
information from OCS. These data are built into files that are ingested
into the Data Backbone.\\
\hspace*{0.333em}\\
The MCAD shall be presented with parameters specified in ID:
OCS-DM-COM-ICD-0031 to classify the files for storage in the Data
Backbone. The normal operation of the MCAD is to acquire and ingest raw
data, metadata and the information necessary for organizing files in the
Data Backbone.\\
\hspace*{0.333em}\\
The MCAD throughput shall be sized to TBD.\\
\hspace*{0.333em}\\
The MCAD shall be able to handle the case where the rate of pixel
generation exceeds the capacity for archive ingest (e.g. the case of
bias generation).\\
\hspace*{0.333em}\\
It shall be possible to determine which exposures have been archived and
which have not.
\end{minipage}
\\ \cdashline{1-2}
Requirement Discussion &
\begin{minipage}[]{13cm}
\textbf{Discussion}: Following an enable command, the Catch-Up Archiver
will interrogate the Camera for the availability of old data in the
Camera buffer, retrieve that data, archive it, and make it available for
processing.\\
\hspace*{0.333em}\\
In general, the Catch-Up Archiver will attempt to follow the
configurations that were -- or would have been -- in force during the
acquisition of the data in order to determine what actions to take. This
information will be retrieved from the EFD and potentially also from the
image data itself.\\
\hspace*{0.333em}\\
DM must, if possible, apply the configuration key interpretation that
would have been in force at the time the start command for the Archiver
CSC was or would have been issued, unless explicitly overridden.\\
\hspace*{0.333em}\\
The configuration of the Catch-up Archiver will include a setting that
allows it to determine what buffered Camera data is actually eligible
for archiving.\\
\hspace*{0.333em}\\
The Catch-Up Archiver will report on the success or failure of the
archiving of each image via an OCS event that includes the Camera image
identifier and, if applicable, the corresponding visit identifier. These
reports will be similar in form to, but on a distinct channel from, the
reports from the Archiver.
\end{minipage}
\\ \cdashline{1-2}
Requirement Priority &  \\ \cdashline{1-2}
Upper Level Requirement &
\begin{tabular}{cl}
\end{tabular}
\\ \hline
\end{longtable}
}


  
 \newpage 
\subsection{[LVV-5345] OCS-DM-COM-ICD-0035-V-01: OCS-Driven Batch CSC\_DM\_1 }\label{lvv-5345}

\begin{longtable}{cccc}
\hline
\textbf{Jira Link} & \textbf{Assignee} & \textbf{Status} & \textbf{Test Cases}\\ \hline
\href{https://jira.lsstcorp.org/browse/LVV-5345}{LVV-5345} &
Leanne Guy & Not Covered &
\begin{tabular}{c}
\end{tabular}
\\
\hline
\end{longtable}

\textbf{Verification Element Description:} \\
Undefined

{\footnotesize
\begin{longtable}{p{2.5cm}p{13.5cm}}
\hline
\multicolumn{2}{c}{\textbf{Requirement Details}}\\ \hline
Requirement ID & OCS-DM-COM-ICD-0035 \\ \cdashline{1-2}
Requirement Description &
\begin{minipage}[]{13cm}
\textbf{Specification:} The OCS-Driven Batch CSC is the only DM-provided
CSC that accepts SAL commands beyond the cross-subsystem ones.~ It shall
accept CSC-specific commands to execute batch jobs that process archived
data through pre-defined pipelines.~
\end{minipage}
\\ \cdashline{1-2}
Requirement Discussion &
\begin{minipage}[]{13cm}
\textbf{Discussion}: This CSC permits OCS scripts to perform
calculations, returning results that are visible to the OCS.~ Such
calculations are expected to include daily master calibration
processing, full-focal-plane wavefront processing, and other calibration
and commissioning tasks. The results may be returned in the command
completion acknowledgment message or, more typically, as DM telemetry.\\
\hspace*{0.333em}
\end{minipage}
\\ \cdashline{1-2}
Requirement Priority &  \\ \cdashline{1-2}
Upper Level Requirement &
\begin{tabular}{cl}
\end{tabular}
\\ \hline
\end{longtable}
}


  
 \newpage 
\subsection{[LVV-5346] OCS-DM-COM-ICD-0035-V-02: OCS-Driven Batch CSC\_DM\_2 }\label{lvv-5346}

\begin{longtable}{cccc}
\hline
\textbf{Jira Link} & \textbf{Assignee} & \textbf{Status} & \textbf{Test Cases}\\ \hline
\href{https://jira.lsstcorp.org/browse/LVV-5346}{LVV-5346} &
Leanne Guy & Not Covered &
\begin{tabular}{c}
\end{tabular}
\\
\hline
\end{longtable}

\textbf{Verification Element Description:} \\
Undefined

{\footnotesize
\begin{longtable}{p{2.5cm}p{13.5cm}}
\hline
\multicolumn{2}{c}{\textbf{Requirement Details}}\\ \hline
Requirement ID & OCS-DM-COM-ICD-0035 \\ \cdashline{1-2}
Requirement Description &
\begin{minipage}[]{13cm}
\textbf{Specification:} The OCS-Driven Batch CSC is the only DM-provided
CSC that accepts SAL commands beyond the cross-subsystem ones.~ It shall
accept CSC-specific commands to execute batch jobs that process archived
data through pre-defined pipelines.~
\end{minipage}
\\ \cdashline{1-2}
Requirement Discussion &
\begin{minipage}[]{13cm}
\textbf{Discussion}: This CSC permits OCS scripts to perform
calculations, returning results that are visible to the OCS.~ Such
calculations are expected to include daily master calibration
processing, full-focal-plane wavefront processing, and other calibration
and commissioning tasks. The results may be returned in the command
completion acknowledgment message or, more typically, as DM telemetry.\\
\hspace*{0.333em}
\end{minipage}
\\ \cdashline{1-2}
Requirement Priority &  \\ \cdashline{1-2}
Upper Level Requirement &
\begin{tabular}{cl}
\end{tabular}
\\ \hline
\end{longtable}
}


  
 \newpage 
\subsection{[LVV-5351] OCS-DM-COM-ICD-0007-V-01: Prompt Processing CSC\_DM\_1 }\label{lvv-5351}

\begin{longtable}{cccc}
\hline
\textbf{Jira Link} & \textbf{Assignee} & \textbf{Status} & \textbf{Test Cases}\\ \hline
\href{https://jira.lsstcorp.org/browse/LVV-5351}{LVV-5351} &
Leanne Guy & Not Covered &
\begin{tabular}{c}
\end{tabular}
\\
\hline
\end{longtable}

\textbf{Verification Element Description:} \\
Undefined

{\footnotesize
\begin{longtable}{p{2.5cm}p{13.5cm}}
\hline
\multicolumn{2}{c}{\textbf{Requirement Details}}\\ \hline
Requirement ID & OCS-DM-COM-ICD-0007 \\ \cdashline{1-2}
Requirement Description &
\begin{minipage}[]{13cm}
\textbf{Specification}: The Prompt Processing CSC shall acquire data
from the main imaging camera. The data shall be presented as FITS files
on the computing nodes, carrying out the computation. FITS headers shall
contain all necessary metadata to support processing of the image.\\
In particular, during normal science operations, the Prompt Processing
CSC shall control the operation of the Alert Production pipelines. The
configuration mechanism (see requirement OCS-DM-COM-ICD-0012 below)
shall be used to control what processing is applied.\\
\hspace*{0.333em}
\end{minipage}
\\ \cdashline{1-2}
Requirement Discussion &
\begin{minipage}[]{13cm}
\textbf{Discussion}: Following an enable command, the Prompt Processing
will apply a configuration-controlled processing to each image or visit
(as appropriate to the configuration).\\
\hspace*{0.333em}\\
During calibration operations, the Prompt Processing CSC will evaluate
per-image quality metrics on raw calibration images as they are
acquired, making that assessment available to the OCS.\\
\hspace*{0.333em}\\
The processing of data retrieved by the Catch-Up Archiver is not under
direct OCS control. It is an autonomous function of the DM Archive
Center control system.\\
\hspace*{0.333em}
\end{minipage}
\\ \cdashline{1-2}
Requirement Priority &  \\ \cdashline{1-2}
Upper Level Requirement &
\begin{tabular}{cl}
\end{tabular}
\\ \hline
\end{longtable}
}


  
 \newpage 
\subsection{[LVV-5352] OCS-DM-COM-ICD-0007-V-02: Prompt Processing CSC\_DM\_2 }\label{lvv-5352}

\begin{longtable}{cccc}
\hline
\textbf{Jira Link} & \textbf{Assignee} & \textbf{Status} & \textbf{Test Cases}\\ \hline
\href{https://jira.lsstcorp.org/browse/LVV-5352}{LVV-5352} &
Leanne Guy & Not Covered &
\begin{tabular}{c}
\end{tabular}
\\
\hline
\end{longtable}

\textbf{Verification Element Description:} \\
Undefined

{\footnotesize
\begin{longtable}{p{2.5cm}p{13.5cm}}
\hline
\multicolumn{2}{c}{\textbf{Requirement Details}}\\ \hline
Requirement ID & OCS-DM-COM-ICD-0007 \\ \cdashline{1-2}
Requirement Description &
\begin{minipage}[]{13cm}
\textbf{Specification}: The Prompt Processing CSC shall acquire data
from the main imaging camera. The data shall be presented as FITS files
on the computing nodes, carrying out the computation. FITS headers shall
contain all necessary metadata to support processing of the image.\\
In particular, during normal science operations, the Prompt Processing
CSC shall control the operation of the Alert Production pipelines. The
configuration mechanism (see requirement OCS-DM-COM-ICD-0012 below)
shall be used to control what processing is applied.\\
\hspace*{0.333em}
\end{minipage}
\\ \cdashline{1-2}
Requirement Discussion &
\begin{minipage}[]{13cm}
\textbf{Discussion}: Following an enable command, the Prompt Processing
will apply a configuration-controlled processing to each image or visit
(as appropriate to the configuration).\\
\hspace*{0.333em}\\
During calibration operations, the Prompt Processing CSC will evaluate
per-image quality metrics on raw calibration images as they are
acquired, making that assessment available to the OCS.\\
\hspace*{0.333em}\\
The processing of data retrieved by the Catch-Up Archiver is not under
direct OCS control. It is an autonomous function of the DM Archive
Center control system.\\
\hspace*{0.333em}
\end{minipage}
\\ \cdashline{1-2}
Requirement Priority &  \\ \cdashline{1-2}
Upper Level Requirement &
\begin{tabular}{cl}
\end{tabular}
\\ \hline
\end{longtable}
}


  
 \newpage 
\subsection{[LVV-5357] OCS-DM-COM-ICD-0048-V-01: Alert Production Complete Event\_DM\_1 }\label{lvv-5357}

\begin{longtable}{cccc}
\hline
\textbf{Jira Link} & \textbf{Assignee} & \textbf{Status} & \textbf{Test Cases}\\ \hline
\href{https://jira.lsstcorp.org/browse/LVV-5357}{LVV-5357} &
Leanne Guy & Not Covered &
\begin{tabular}{c}
\end{tabular}
\\
\hline
\end{longtable}

\textbf{Verification Element Description:} \\
Undefined

{\footnotesize
\begin{longtable}{p{2.5cm}p{13.5cm}}
\hline
\multicolumn{2}{c}{\textbf{Requirement Details}}\\ \hline
Requirement ID & OCS-DM-COM-ICD-0048 \\ \cdashline{1-2}
Requirement Description &
\begin{minipage}[]{13cm}
\textbf{Specification}: DM shall publish an event indicating that a
complete visit was successfully processed by the Alert Production
Payload in the Prompt Processing service, including the delivery of
Alerts to the Alert Distribution system.~ This event shall include the
visit name, the image\\
name(s) included in the visit, and the length of time since the last
endReadout event for the visit (float, in units of seconds).
\end{minipage}
\\ \cdashline{1-2}
Requirement Priority &  \\ \cdashline{1-2}
Upper Level Requirement &
\begin{tabular}{cl}
\end{tabular}
\\ \hline
\end{longtable}
}


  
 \newpage 
\subsection{[LVV-5358] OCS-DM-COM-ICD-0048-V-02: Alert Production Complete Event\_DM\_2 }\label{lvv-5358}

\begin{longtable}{cccc}
\hline
\textbf{Jira Link} & \textbf{Assignee} & \textbf{Status} & \textbf{Test Cases}\\ \hline
\href{https://jira.lsstcorp.org/browse/LVV-5358}{LVV-5358} &
Leanne Guy & Not Covered &
\begin{tabular}{c}
\end{tabular}
\\
\hline
\end{longtable}

\textbf{Verification Element Description:} \\
Undefined

{\footnotesize
\begin{longtable}{p{2.5cm}p{13.5cm}}
\hline
\multicolumn{2}{c}{\textbf{Requirement Details}}\\ \hline
Requirement ID & OCS-DM-COM-ICD-0048 \\ \cdashline{1-2}
Requirement Description &
\begin{minipage}[]{13cm}
\textbf{Specification}: DM shall publish an event indicating that a
complete visit was successfully processed by the Alert Production
Payload in the Prompt Processing service, including the delivery of
Alerts to the Alert Distribution system.~ This event shall include the
visit name, the image\\
name(s) included in the visit, and the length of time since the last
endReadout event for the visit (float, in units of seconds).
\end{minipage}
\\ \cdashline{1-2}
Requirement Priority &  \\ \cdashline{1-2}
Upper Level Requirement &
\begin{tabular}{cl}
\end{tabular}
\\ \hline
\end{longtable}
}


  
 \newpage 
\subsection{[LVV-5363] OCS-DM-COM-ICD-0055-V-01: Archiver Resource Availability\_DM\_1 }\label{lvv-5363}

\begin{longtable}{cccc}
\hline
\textbf{Jira Link} & \textbf{Assignee} & \textbf{Status} & \textbf{Test Cases}\\ \hline
\href{https://jira.lsstcorp.org/browse/LVV-5363}{LVV-5363} &
Leanne Guy & Not Covered &
\begin{tabular}{c}
\end{tabular}
\\
\hline
\end{longtable}

\textbf{Verification Element Description:} \\
Undefined

{\footnotesize
\begin{longtable}{p{2.5cm}p{13.5cm}}
\hline
\multicolumn{2}{c}{\textbf{Requirement Details}}\\ \hline
Requirement ID & OCS-DM-COM-ICD-0055 \\ \cdashline{1-2}
Requirement Description &
\begin{minipage}[]{13cm}
\textbf{Specification}: DM shall publish as telemetry, at intervals of
\textbf{dmRsrcInterval}, the number of Archiver Forwarder nodes
available, the load average on each node, the percentage of memory in
use on each node), and the percentage of disk space in use on each local
filesystem on each node.
\end{minipage}
\\ \cdashline{1-2}
Requirement Discussion &
\begin{minipage}[]{13cm}
\textbf{Discussion}: The value type for the number of Archiver Forwarder
nodes available is ``int'', the value type for load average on each node
is ``float'', the percentage of memory in use on each node is ``float'',
and the percentage of disk space in use on each local filesystem on each
node is ``float''.\\
\hspace*{0.333em}
\end{minipage}
\\ \cdashline{1-2}
Requirement Priority &  \\ \cdashline{1-2}
Upper Level Requirement &
\begin{tabular}{cl}
\end{tabular}
\\ \hline
\end{longtable}
}


  
 \newpage 
\subsection{[LVV-5364] OCS-DM-COM-ICD-0055-V-02: Archiver Resource Availability\_DM\_2 }\label{lvv-5364}

\begin{longtable}{cccc}
\hline
\textbf{Jira Link} & \textbf{Assignee} & \textbf{Status} & \textbf{Test Cases}\\ \hline
\href{https://jira.lsstcorp.org/browse/LVV-5364}{LVV-5364} &
Leanne Guy & Not Covered &
\begin{tabular}{c}
\end{tabular}
\\
\hline
\end{longtable}

\textbf{Verification Element Description:} \\
Undefined

{\footnotesize
\begin{longtable}{p{2.5cm}p{13.5cm}}
\hline
\multicolumn{2}{c}{\textbf{Requirement Details}}\\ \hline
Requirement ID & OCS-DM-COM-ICD-0055 \\ \cdashline{1-2}
Requirement Description &
\begin{minipage}[]{13cm}
\textbf{Specification}: DM shall publish as telemetry, at intervals of
\textbf{dmRsrcInterval}, the number of Archiver Forwarder nodes
available, the load average on each node, the percentage of memory in
use on each node), and the percentage of disk space in use on each local
filesystem on each node.
\end{minipage}
\\ \cdashline{1-2}
Requirement Discussion &
\begin{minipage}[]{13cm}
\textbf{Discussion}: The value type for the number of Archiver Forwarder
nodes available is ``int'', the value type for load average on each node
is ``float'', the percentage of memory in use on each node is ``float'',
and the percentage of disk space in use on each local filesystem on each
node is ``float''.\\
\hspace*{0.333em}
\end{minipage}
\\ \cdashline{1-2}
Requirement Priority &  \\ \cdashline{1-2}
Upper Level Requirement &
\begin{tabular}{cl}
\end{tabular}
\\ \hline
\end{longtable}
}


  
 \newpage 
\subsection{[LVV-5369] OCS-DM-COM-ICD-0054-V-01: Base-Archive Network Utilization\_DM\_1 }\label{lvv-5369}

\begin{longtable}{cccc}
\hline
\textbf{Jira Link} & \textbf{Assignee} & \textbf{Status} & \textbf{Test Cases}\\ \hline
\href{https://jira.lsstcorp.org/browse/LVV-5369}{LVV-5369} &
Leanne Guy & Not Covered &
\begin{tabular}{c}
\end{tabular}
\\
\hline
\end{longtable}

\textbf{Verification Element Description:} \\
Undefined

{\footnotesize
\begin{longtable}{p{2.5cm}p{13.5cm}}
\hline
\multicolumn{2}{c}{\textbf{Requirement Details}}\\ \hline
Requirement ID & OCS-DM-COM-ICD-0054 \\ \cdashline{1-2}
Requirement Description &
\begin{minipage}[]{13cm}
\textbf{Specification:} DM shall publish as telemetry, at intervals of
\textbf{netUtilInterval}, the percent utilization of each Base-Archive
network link in each direction. The intervals shall be at 300 seconds,
and the data reported is the utilization over the previous 300 second
interval, as well as a measurement of the round-trip time in each
direction. ~
\end{minipage}
\\ \cdashline{1-2}
Requirement Discussion &
\begin{minipage}[]{13cm}
\textbf{Discussion}: The value type for the percent utilization and the
round trip time is "float' and the unit is in seconds.
\end{minipage}
\\ \cdashline{1-2}
Requirement Priority &  \\ \cdashline{1-2}
Upper Level Requirement &
\begin{tabular}{cl}
\end{tabular}
\\ \hline
\end{longtable}
}


  
 \newpage 
\subsection{[LVV-5370] OCS-DM-COM-ICD-0054-V-02: Base-Archive Network Utilization\_DM\_2 }\label{lvv-5370}

\begin{longtable}{cccc}
\hline
\textbf{Jira Link} & \textbf{Assignee} & \textbf{Status} & \textbf{Test Cases}\\ \hline
\href{https://jira.lsstcorp.org/browse/LVV-5370}{LVV-5370} &
Leanne Guy & Not Covered &
\begin{tabular}{c}
\end{tabular}
\\
\hline
\end{longtable}

\textbf{Verification Element Description:} \\
Undefined

{\footnotesize
\begin{longtable}{p{2.5cm}p{13.5cm}}
\hline
\multicolumn{2}{c}{\textbf{Requirement Details}}\\ \hline
Requirement ID & OCS-DM-COM-ICD-0054 \\ \cdashline{1-2}
Requirement Description &
\begin{minipage}[]{13cm}
\textbf{Specification:} DM shall publish as telemetry, at intervals of
\textbf{netUtilInterval}, the percent utilization of each Base-Archive
network link in each direction. The intervals shall be at 300 seconds,
and the data reported is the utilization over the previous 300 second
interval, as well as a measurement of the round-trip time in each
direction. ~
\end{minipage}
\\ \cdashline{1-2}
Requirement Discussion &
\begin{minipage}[]{13cm}
\textbf{Discussion}: The value type for the percent utilization and the
round trip time is "float' and the unit is in seconds.
\end{minipage}
\\ \cdashline{1-2}
Requirement Priority &  \\ \cdashline{1-2}
Upper Level Requirement &
\begin{tabular}{cl}
\end{tabular}
\\ \hline
\end{longtable}
}


  
 \newpage 
\subsection{[LVV-5375] OCS-DM-COM-ICD-0019-V-01: Data Management Events and Telemetry Required
by the OCS\_DM\_1 }\label{lvv-5375}

\begin{longtable}{cccc}
\hline
\textbf{Jira Link} & \textbf{Assignee} & \textbf{Status} & \textbf{Test Cases}\\ \hline
\href{https://jira.lsstcorp.org/browse/LVV-5375}{LVV-5375} &
Leanne Guy & Not Covered &
\begin{tabular}{c}
\end{tabular}
\\
\hline
\end{longtable}

\textbf{Verification Element Description:} \\
Undefined

{\footnotesize
\begin{longtable}{p{2.5cm}p{13.5cm}}
\hline
\multicolumn{2}{c}{\textbf{Requirement Details}}\\ \hline
Requirement ID & OCS-DM-COM-ICD-0019 \\ \cdashline{1-2}
Requirement Description &
\begin{minipage}[]{13cm}
\textbf{Specification}: Data Management shall publish to the OCS events
and telemetry regarding the progress of Alert Production processing and
archiving of specific images and visits, the observed data quality, and
the general health of the Alert Production system, as defined by the
requirements below.\\
\hspace*{0.333em}
\end{minipage}
\\ \cdashline{1-2}
Requirement Discussion &
\begin{minipage}[]{13cm}
\textbf{Discussion}: The requirements below give the minimum set of
events and telemetry to be published; additional events and telemetry
that give visibility into the health and operation of the DM-constructed
systems will be specified\\
in design documents.\\
\hspace*{0.333em}\\
\hspace*{0.333em}
\end{minipage}
\\ \cdashline{1-2}
Requirement Priority &  \\ \cdashline{1-2}
Upper Level Requirement &
\begin{tabular}{cl}
\end{tabular}
\\ \hline
\end{longtable}
}


  
 \newpage 
\subsection{[LVV-5376] OCS-DM-COM-ICD-0019-V-02: Data Management Events and Telemetry Required
by the OCS\_DM\_2 }\label{lvv-5376}

\begin{longtable}{cccc}
\hline
\textbf{Jira Link} & \textbf{Assignee} & \textbf{Status} & \textbf{Test Cases}\\ \hline
\href{https://jira.lsstcorp.org/browse/LVV-5376}{LVV-5376} &
Leanne Guy & Not Covered &
\begin{tabular}{c}
\end{tabular}
\\
\hline
\end{longtable}

\textbf{Verification Element Description:} \\
Undefined

{\footnotesize
\begin{longtable}{p{2.5cm}p{13.5cm}}
\hline
\multicolumn{2}{c}{\textbf{Requirement Details}}\\ \hline
Requirement ID & OCS-DM-COM-ICD-0019 \\ \cdashline{1-2}
Requirement Description &
\begin{minipage}[]{13cm}
\textbf{Specification}: Data Management shall publish to the OCS events
and telemetry regarding the progress of Alert Production processing and
archiving of specific images and visits, the observed data quality, and
the general health of the Alert Production system, as defined by the
requirements below.\\
\hspace*{0.333em}
\end{minipage}
\\ \cdashline{1-2}
Requirement Discussion &
\begin{minipage}[]{13cm}
\textbf{Discussion}: The requirements below give the minimum set of
events and telemetry to be published; additional events and telemetry
that give visibility into the health and operation of the DM-constructed
systems will be specified\\
in design documents.\\
\hspace*{0.333em}\\
\hspace*{0.333em}
\end{minipage}
\\ \cdashline{1-2}
Requirement Priority &  \\ \cdashline{1-2}
Upper Level Requirement &
\begin{tabular}{cl}
\end{tabular}
\\ \hline
\end{longtable}
}


  
 \newpage 
\subsection{[LVV-5381] OCS-DM-COM-ICD-0017-V-01: Data Management Telemetry Interface
Model\_DM\_1 }\label{lvv-5381}

\begin{longtable}{cccc}
\hline
\textbf{Jira Link} & \textbf{Assignee} & \textbf{Status} & \textbf{Test Cases}\\ \hline
\href{https://jira.lsstcorp.org/browse/LVV-5381}{LVV-5381} &
Leanne Guy & Not Covered &
\begin{tabular}{c}
\end{tabular}
\\
\hline
\end{longtable}

\textbf{Verification Element Description:} \\
Undefined

{\footnotesize
\begin{longtable}{p{2.5cm}p{13.5cm}}
\hline
\multicolumn{2}{c}{\textbf{Requirement Details}}\\ \hline
Requirement ID & OCS-DM-COM-ICD-0017 \\ \cdashline{1-2}
Requirement Description &
\begin{minipage}[]{13cm}
\textbf{Specification}: Data Management shall use the OCS Service
Abstraction Layer (SAL) as defined in \citeds{LSE-70} to present its telemetry
interface to the OCS and, through it, make telemetry available to other
Observatory subsystems and subscribe to telemetry from the OCS and other
subsystems.
\end{minipage}
\\ \cdashline{1-2}
Requirement Discussion &
\begin{minipage}[]{13cm}
\textbf{Discussion}: The agreements between Data Management and the
non-OCS subsystems regarding telemetry exchanges are recorded in the
respective ICDs between DM and those subsystems, and not in this
document.\\
The complete list of telemetry planned to be provided by DM will be
published in a telemetry dictionary, based on the requirements in those
ICDs as well as the present one.
\end{minipage}
\\ \cdashline{1-2}
Requirement Priority &  \\ \cdashline{1-2}
Upper Level Requirement &
\begin{tabular}{cl}
\end{tabular}
\\ \hline
\end{longtable}
}


  
 \newpage 
\subsection{[LVV-5382] OCS-DM-COM-ICD-0017-V-02: Data Management Telemetry Interface
Model\_DM\_2 }\label{lvv-5382}

\begin{longtable}{cccc}
\hline
\textbf{Jira Link} & \textbf{Assignee} & \textbf{Status} & \textbf{Test Cases}\\ \hline
\href{https://jira.lsstcorp.org/browse/LVV-5382}{LVV-5382} &
Leanne Guy & Not Covered &
\begin{tabular}{c}
\end{tabular}
\\
\hline
\end{longtable}

\textbf{Verification Element Description:} \\
Undefined

{\footnotesize
\begin{longtable}{p{2.5cm}p{13.5cm}}
\hline
\multicolumn{2}{c}{\textbf{Requirement Details}}\\ \hline
Requirement ID & OCS-DM-COM-ICD-0017 \\ \cdashline{1-2}
Requirement Description &
\begin{minipage}[]{13cm}
\textbf{Specification}: Data Management shall use the OCS Service
Abstraction Layer (SAL) as defined in \citeds{LSE-70} to present its telemetry
interface to the OCS and, through it, make telemetry available to other
Observatory subsystems and subscribe to telemetry from the OCS and other
subsystems.
\end{minipage}
\\ \cdashline{1-2}
Requirement Discussion &
\begin{minipage}[]{13cm}
\textbf{Discussion}: The agreements between Data Management and the
non-OCS subsystems regarding telemetry exchanges are recorded in the
respective ICDs between DM and those subsystems, and not in this
document.\\
The complete list of telemetry planned to be provided by DM will be
published in a telemetry dictionary, based on the requirements in those
ICDs as well as the present one.
\end{minipage}
\\ \cdashline{1-2}
Requirement Priority &  \\ \cdashline{1-2}
Upper Level Requirement &
\begin{tabular}{cl}
\end{tabular}
\\ \hline
\end{longtable}
}


  
 \newpage 
\subsection{[LVV-5387] OCS-DM-COM-ICD-0018-V-01: Data Management Telemetry Time Stamp\_DM\_1 }\label{lvv-5387}

\begin{longtable}{cccc}
\hline
\textbf{Jira Link} & \textbf{Assignee} & \textbf{Status} & \textbf{Test Cases}\\ \hline
\href{https://jira.lsstcorp.org/browse/LVV-5387}{LVV-5387} &
Leanne Guy & Not Covered &
\begin{tabular}{c}
\end{tabular}
\\
\hline
\end{longtable}

\textbf{Verification Element Description:} \\
Undefined

{\footnotesize
\begin{longtable}{p{2.5cm}p{13.5cm}}
\hline
\multicolumn{2}{c}{\textbf{Requirement Details}}\\ \hline
Requirement ID & OCS-DM-COM-ICD-0018 \\ \cdashline{1-2}
Requirement Description &
\begin{minipage}[]{13cm}
\textbf{Specification:} Data Management shall provide the measurement
time of all published telemetry.
\end{minipage}
\\ \cdashline{1-2}
Requirement Discussion &
\begin{minipage}[]{13cm}
\textbf{Discussion}: The publication mechanism (provided by OCS) along
with the DM-provided time-stamps, are intended to be sufficient to
rendezvous telemetry associated with a particular image. The definition
of the measurement time is made by DM and documented for each telemetry
type. The time-stamp is described in \citeds{LSE-70}.
\end{minipage}
\\ \cdashline{1-2}
Requirement Priority &  \\ \cdashline{1-2}
Upper Level Requirement &
\begin{tabular}{cl}
\end{tabular}
\\ \hline
\end{longtable}
}


  
 \newpage 
\subsection{[LVV-5388] OCS-DM-COM-ICD-0018-V-02: Data Management Telemetry Time Stamp\_DM\_2 }\label{lvv-5388}

\begin{longtable}{cccc}
\hline
\textbf{Jira Link} & \textbf{Assignee} & \textbf{Status} & \textbf{Test Cases}\\ \hline
\href{https://jira.lsstcorp.org/browse/LVV-5388}{LVV-5388} &
Leanne Guy & Not Covered &
\begin{tabular}{c}
\end{tabular}
\\
\hline
\end{longtable}

\textbf{Verification Element Description:} \\
Undefined

{\footnotesize
\begin{longtable}{p{2.5cm}p{13.5cm}}
\hline
\multicolumn{2}{c}{\textbf{Requirement Details}}\\ \hline
Requirement ID & OCS-DM-COM-ICD-0018 \\ \cdashline{1-2}
Requirement Description &
\begin{minipage}[]{13cm}
\textbf{Specification:} Data Management shall provide the measurement
time of all published telemetry.
\end{minipage}
\\ \cdashline{1-2}
Requirement Discussion &
\begin{minipage}[]{13cm}
\textbf{Discussion}: The publication mechanism (provided by OCS) along
with the DM-provided time-stamps, are intended to be sufficient to
rendezvous telemetry associated with a particular image. The definition
of the measurement time is made by DM and documented for each telemetry
type. The time-stamp is described in \citeds{LSE-70}.
\end{minipage}
\\ \cdashline{1-2}
Requirement Priority &  \\ \cdashline{1-2}
Upper Level Requirement &
\begin{tabular}{cl}
\end{tabular}
\\ \hline
\end{longtable}
}


  
 \newpage 
\subsection{[LVV-5393] OCS-DM-COM-ICD-0021-V-01: Data Quality Metrics\_DM\_1 }\label{lvv-5393}

\begin{longtable}{cccc}
\hline
\textbf{Jira Link} & \textbf{Assignee} & \textbf{Status} & \textbf{Test Cases}\\ \hline
\href{https://jira.lsstcorp.org/browse/LVV-5393}{LVV-5393} &
Leanne Guy & Not Covered &
\begin{tabular}{c}
\end{tabular}
\\
\hline
\end{longtable}

\textbf{Verification Element Description:} \\
Undefined

{\footnotesize
\begin{longtable}{p{2.5cm}p{13.5cm}}
\hline
\multicolumn{2}{c}{\textbf{Requirement Details}}\\ \hline
Requirement ID & OCS-DM-COM-ICD-0021 \\ \cdashline{1-2}
Requirement Description &
\begin{minipage}[]{13cm}
\textbf{Specification}: Data Management shall publish as telemetry a set
of data quality metrics that enable the OCS scheduling algorithms for
science operations to assess whether visits acquired should be scored as
successful and to assess the general observing quality -- e.g., weather
and seeing -- across the sky. DM shall also provide metrics that enable
the OCS scheduling algorithm for calibration operations to assess the
progress in collecting usable calibration data.\\
\hspace*{0.333em}
\end{minipage}
\\ \cdashline{1-2}
Requirement Discussion &
\begin{minipage}[]{13cm}
\textbf{Discussion}: DM is not responsible for assessing whether a visit
or image meets the scheduler criteria; this assessment is the
responsibility of the OCS based on lower-level data provided by DM.\\
\hspace*{0.333em}\\
\hspace*{0.333em}
\end{minipage}
\\ \cdashline{1-2}
Requirement Priority &  \\ \cdashline{1-2}
Upper Level Requirement &
\begin{tabular}{cl}
\end{tabular}
\\ \hline
\end{longtable}
}


  
 \newpage 
\subsection{[LVV-5394] OCS-DM-COM-ICD-0021-V-02: Data Quality Metrics\_DM\_2 }\label{lvv-5394}

\begin{longtable}{cccc}
\hline
\textbf{Jira Link} & \textbf{Assignee} & \textbf{Status} & \textbf{Test Cases}\\ \hline
\href{https://jira.lsstcorp.org/browse/LVV-5394}{LVV-5394} &
Leanne Guy & Not Covered &
\begin{tabular}{c}
\end{tabular}
\\
\hline
\end{longtable}

\textbf{Verification Element Description:} \\
Undefined

{\footnotesize
\begin{longtable}{p{2.5cm}p{13.5cm}}
\hline
\multicolumn{2}{c}{\textbf{Requirement Details}}\\ \hline
Requirement ID & OCS-DM-COM-ICD-0021 \\ \cdashline{1-2}
Requirement Description &
\begin{minipage}[]{13cm}
\textbf{Specification}: Data Management shall publish as telemetry a set
of data quality metrics that enable the OCS scheduling algorithms for
science operations to assess whether visits acquired should be scored as
successful and to assess the general observing quality -- e.g., weather
and seeing -- across the sky. DM shall also provide metrics that enable
the OCS scheduling algorithm for calibration operations to assess the
progress in collecting usable calibration data.\\
\hspace*{0.333em}
\end{minipage}
\\ \cdashline{1-2}
Requirement Discussion &
\begin{minipage}[]{13cm}
\textbf{Discussion}: DM is not responsible for assessing whether a visit
or image meets the scheduler criteria; this assessment is the
responsibility of the OCS based on lower-level data provided by DM.\\
\hspace*{0.333em}\\
\hspace*{0.333em}
\end{minipage}
\\ \cdashline{1-2}
Requirement Priority &  \\ \cdashline{1-2}
Upper Level Requirement &
\begin{tabular}{cl}
\end{tabular}
\\ \hline
\end{longtable}
}


  
 \newpage 
\subsection{[LVV-5399] OCS-DM-COM-ICD-0020-V-01: Image and Visit Processing and Archiving
Status\_DM\_1 }\label{lvv-5399}

\begin{longtable}{cccc}
\hline
\textbf{Jira Link} & \textbf{Assignee} & \textbf{Status} & \textbf{Test Cases}\\ \hline
\href{https://jira.lsstcorp.org/browse/LVV-5399}{LVV-5399} &
Leanne Guy & Not Covered &
\begin{tabular}{c}
\end{tabular}
\\
\hline
\end{longtable}

\textbf{Verification Element Description:} \\
Undefined

{\footnotesize
\begin{longtable}{p{2.5cm}p{13.5cm}}
\hline
\multicolumn{2}{c}{\textbf{Requirement Details}}\\ \hline
Requirement ID & OCS-DM-COM-ICD-0020 \\ \cdashline{1-2}
Requirement Description &
\begin{minipage}[]{13cm}
\textbf{Specification}: Data Management shall publish high-level
information concerning the processing and archiving of images.~ All
events listed below shall be published at least once for each successful
completion of the described activity.\\
\hspace*{0.333em}
\end{minipage}
\\ \cdashline{1-2}
Requirement Discussion &
\begin{minipage}[]{13cm}
\textbf{Discussion}: The granularity of the reporting of per-image data,
such as the confirmation of archiving, is to be determined in Phase 3 --
e.g., whether this is reported per CCD, per raft, or for the entire
focal plane. The selection of publication as event or as telemetry will
be revisited at that time, with a view toward the selection of the
appropriate quality of service.\\
\textbf{Note}: Events related to image processing and archiving may
occur a considerable amount of time after the relevant image has been
taken.\\
\hspace*{0.333em}
\end{minipage}
\\ \cdashline{1-2}
Requirement Priority &  \\ \cdashline{1-2}
Upper Level Requirement &
\begin{tabular}{cl}
\end{tabular}
\\ \hline
\end{longtable}
}


  
 \newpage 
\subsection{[LVV-5400] OCS-DM-COM-ICD-0020-V-02: Image and Visit Processing and Archiving
Status\_DM\_2 }\label{lvv-5400}

\begin{longtable}{cccc}
\hline
\textbf{Jira Link} & \textbf{Assignee} & \textbf{Status} & \textbf{Test Cases}\\ \hline
\href{https://jira.lsstcorp.org/browse/LVV-5400}{LVV-5400} &
Leanne Guy & Not Covered &
\begin{tabular}{c}
\end{tabular}
\\
\hline
\end{longtable}

\textbf{Verification Element Description:} \\
Undefined

{\footnotesize
\begin{longtable}{p{2.5cm}p{13.5cm}}
\hline
\multicolumn{2}{c}{\textbf{Requirement Details}}\\ \hline
Requirement ID & OCS-DM-COM-ICD-0020 \\ \cdashline{1-2}
Requirement Description &
\begin{minipage}[]{13cm}
\textbf{Specification}: Data Management shall publish high-level
information concerning the processing and archiving of images.~ All
events listed below shall be published at least once for each successful
completion of the described activity.\\
\hspace*{0.333em}
\end{minipage}
\\ \cdashline{1-2}
Requirement Discussion &
\begin{minipage}[]{13cm}
\textbf{Discussion}: The granularity of the reporting of per-image data,
such as the confirmation of archiving, is to be determined in Phase 3 --
e.g., whether this is reported per CCD, per raft, or for the entire
focal plane. The selection of publication as event or as telemetry will
be revisited at that time, with a view toward the selection of the
appropriate quality of service.\\
\textbf{Note}: Events related to image processing and archiving may
occur a considerable amount of time after the relevant image has been
taken.\\
\hspace*{0.333em}
\end{minipage}
\\ \cdashline{1-2}
Requirement Priority &  \\ \cdashline{1-2}
Upper Level Requirement &
\begin{tabular}{cl}
\end{tabular}
\\ \hline
\end{longtable}
}


  
 \newpage 
\subsection{[LVV-5405] OCS-DM-COM-ICD-0047-V-01: Image Archived Event\_DM\_1 }\label{lvv-5405}

\begin{longtable}{cccc}
\hline
\textbf{Jira Link} & \textbf{Assignee} & \textbf{Status} & \textbf{Test Cases}\\ \hline
\href{https://jira.lsstcorp.org/browse/LVV-5405}{LVV-5405} &
Leanne Guy & Not Covered &
\begin{tabular}{c}
\end{tabular}
\\
\hline
\end{longtable}

\textbf{Verification Element Description:} \\
Undefined

{\footnotesize
\begin{longtable}{p{2.5cm}p{13.5cm}}
\hline
\multicolumn{2}{c}{\textbf{Requirement Details}}\\ \hline
Requirement ID & OCS-DM-COM-ICD-0047 \\ \cdashline{1-2}
Requirement Description &
\begin{minipage}[]{13cm}
\textbf{Specification}: DM shall publish an event indicating that a
complete image, including all configured portions of the focal plane,
was successfully archived, along with its metadata, in the Data Backbone
at both the Base and Archive Facilities.~ This event shall include the
camera (Auxiliary\\
Telescope Spectrograph, ComCam, LSSTCam) and the image name.
\end{minipage}
\\ \cdashline{1-2}
Requirement Priority &  \\ \cdashline{1-2}
Upper Level Requirement &
\begin{tabular}{cl}
\end{tabular}
\\ \hline
\end{longtable}
}


  
 \newpage 
\subsection{[LVV-5406] OCS-DM-COM-ICD-0047-V-02: Image Archived Event\_DM\_2 }\label{lvv-5406}

\begin{longtable}{cccc}
\hline
\textbf{Jira Link} & \textbf{Assignee} & \textbf{Status} & \textbf{Test Cases}\\ \hline
\href{https://jira.lsstcorp.org/browse/LVV-5406}{LVV-5406} &
Leanne Guy & Not Covered &
\begin{tabular}{c}
\end{tabular}
\\
\hline
\end{longtable}

\textbf{Verification Element Description:} \\
Undefined

{\footnotesize
\begin{longtable}{p{2.5cm}p{13.5cm}}
\hline
\multicolumn{2}{c}{\textbf{Requirement Details}}\\ \hline
Requirement ID & OCS-DM-COM-ICD-0047 \\ \cdashline{1-2}
Requirement Description &
\begin{minipage}[]{13cm}
\textbf{Specification}: DM shall publish an event indicating that a
complete image, including all configured portions of the focal plane,
was successfully archived, along with its metadata, in the Data Backbone
at both the Base and Archive Facilities.~ This event shall include the
camera (Auxiliary\\
Telescope Spectrograph, ComCam, LSSTCam) and the image name.
\end{minipage}
\\ \cdashline{1-2}
Requirement Priority &  \\ \cdashline{1-2}
Upper Level Requirement &
\begin{tabular}{cl}
\end{tabular}
\\ \hline
\end{longtable}
}


  
 \newpage 
\subsection{[LVV-5411] OCS-DM-COM-ICD-0046-V-01: Image Forwarded Event\_DM\_1 }\label{lvv-5411}

\begin{longtable}{cccc}
\hline
\textbf{Jira Link} & \textbf{Assignee} & \textbf{Status} & \textbf{Test Cases}\\ \hline
\href{https://jira.lsstcorp.org/browse/LVV-5411}{LVV-5411} &
Leanne Guy & Not Covered &
\begin{tabular}{c}
\end{tabular}
\\
\hline
\end{longtable}

\textbf{Verification Element Description:} \\
Undefined

{\footnotesize
\begin{longtable}{p{2.5cm}p{13.5cm}}
\hline
\multicolumn{2}{c}{\textbf{Requirement Details}}\\ \hline
Requirement ID & OCS-DM-COM-ICD-0046 \\ \cdashline{1-2}
Requirement Description &
\begin{minipage}[]{13cm}
\textbf{Specification}: DM shall publish an event indicating that a
complete image, including all configured portions of the focal plane,
was successfully provided to the NCSA Distributors (for images to be
processed by Prompt Processing). This event shall include the image
name.
\end{minipage}
\\ \cdashline{1-2}
Requirement Priority &  \\ \cdashline{1-2}
Upper Level Requirement &
\begin{tabular}{cl}
\end{tabular}
\\ \hline
\end{longtable}
}


  
 \newpage 
\subsection{[LVV-5412] OCS-DM-COM-ICD-0046-V-02: Image Forwarded Event\_DM\_2 }\label{lvv-5412}

\begin{longtable}{cccc}
\hline
\textbf{Jira Link} & \textbf{Assignee} & \textbf{Status} & \textbf{Test Cases}\\ \hline
\href{https://jira.lsstcorp.org/browse/LVV-5412}{LVV-5412} &
Leanne Guy & Not Covered &
\begin{tabular}{c}
\end{tabular}
\\
\hline
\end{longtable}

\textbf{Verification Element Description:} \\
Undefined

{\footnotesize
\begin{longtable}{p{2.5cm}p{13.5cm}}
\hline
\multicolumn{2}{c}{\textbf{Requirement Details}}\\ \hline
Requirement ID & OCS-DM-COM-ICD-0046 \\ \cdashline{1-2}
Requirement Description &
\begin{minipage}[]{13cm}
\textbf{Specification}: DM shall publish an event indicating that a
complete image, including all configured portions of the focal plane,
was successfully provided to the NCSA Distributors (for images to be
processed by Prompt Processing). This event shall include the image
name.
\end{minipage}
\\ \cdashline{1-2}
Requirement Priority &  \\ \cdashline{1-2}
Upper Level Requirement &
\begin{tabular}{cl}
\end{tabular}
\\ \hline
\end{longtable}
}


  
 \newpage 
\subsection{[LVV-5417] OCS-DM-COM-ICD-0045-V-01: Image in OODS Event\_DM\_1 }\label{lvv-5417}

\begin{longtable}{cccc}
\hline
\textbf{Jira Link} & \textbf{Assignee} & \textbf{Status} & \textbf{Test Cases}\\ \hline
\href{https://jira.lsstcorp.org/browse/LVV-5417}{LVV-5417} &
Leanne Guy & Not Covered &
\begin{tabular}{c}
\end{tabular}
\\
\hline
\end{longtable}

\textbf{Verification Element Description:} \\
Undefined

{\footnotesize
\begin{longtable}{p{2.5cm}p{13.5cm}}
\hline
\multicolumn{2}{c}{\textbf{Requirement Details}}\\ \hline
Requirement ID & OCS-DM-COM-ICD-0045 \\ \cdashline{1-2}
Requirement Description &
\begin{minipage}[]{13cm}
\textbf{Specification}: DM shall publish an event indicating that a
complete image, including all configured portions of the focal plane,
was successfully provided to the Observatory Operations Data Service.~
This event shall include the camera (Auxiliary Telescope Spectrograph,
ComCam, LSSTCam), the image name, and an indication as to whether this
was performed by the normal Archiver or the Catch-Up Archiver.
\end{minipage}
\\ \cdashline{1-2}
Requirement Priority &  \\ \cdashline{1-2}
Upper Level Requirement &
\begin{tabular}{cl}
\end{tabular}
\\ \hline
\end{longtable}
}


  
 \newpage 
\subsection{[LVV-5418] OCS-DM-COM-ICD-0045-V-02: Image in OODS Event\_DM\_2 }\label{lvv-5418}

\begin{longtable}{cccc}
\hline
\textbf{Jira Link} & \textbf{Assignee} & \textbf{Status} & \textbf{Test Cases}\\ \hline
\href{https://jira.lsstcorp.org/browse/LVV-5418}{LVV-5418} &
Leanne Guy & Not Covered &
\begin{tabular}{c}
\end{tabular}
\\
\hline
\end{longtable}

\textbf{Verification Element Description:} \\
Undefined

{\footnotesize
\begin{longtable}{p{2.5cm}p{13.5cm}}
\hline
\multicolumn{2}{c}{\textbf{Requirement Details}}\\ \hline
Requirement ID & OCS-DM-COM-ICD-0045 \\ \cdashline{1-2}
Requirement Description &
\begin{minipage}[]{13cm}
\textbf{Specification}: DM shall publish an event indicating that a
complete image, including all configured portions of the focal plane,
was successfully provided to the Observatory Operations Data Service.~
This event shall include the camera (Auxiliary Telescope Spectrograph,
ComCam, LSSTCam), the image name, and an indication as to whether this
was performed by the normal Archiver or the Catch-Up Archiver.
\end{minipage}
\\ \cdashline{1-2}
Requirement Priority &  \\ \cdashline{1-2}
Upper Level Requirement &
\begin{tabular}{cl}
\end{tabular}
\\ \hline
\end{longtable}
}


  
 \newpage 
\subsection{[LVV-5423] OCS-DM-COM-ICD-0043-V-01: Image Retrieval for Archiving Event\_DM\_1 }\label{lvv-5423}

\begin{longtable}{cccc}
\hline
\textbf{Jira Link} & \textbf{Assignee} & \textbf{Status} & \textbf{Test Cases}\\ \hline
\href{https://jira.lsstcorp.org/browse/LVV-5423}{LVV-5423} &
Leanne Guy & Not Covered &
\begin{tabular}{c}
\end{tabular}
\\
\hline
\end{longtable}

\textbf{Verification Element Description:} \\
Undefined

{\footnotesize
\begin{longtable}{p{2.5cm}p{13.5cm}}
\hline
\multicolumn{2}{c}{\textbf{Requirement Details}}\\ \hline
Requirement ID & OCS-DM-COM-ICD-0043 \\ \cdashline{1-2}
Requirement Description &
\begin{minipage}[]{13cm}
\textbf{Specification:} DM shall publish an event indicating that a
complete image, including all configured portions of the focal plane,
was successfully retrieved from the Camera DAQ or Auxiliary Telescope
Spectrograph DAQ by an Archiver or Catch-Up Archiver Forwarder.~ This
event shall include the camera (Auxiliary Telescope Spectrograph,
ComCam, LSSTCam), the image name, and an indication as to whether this
was performed by the normal Archiver or the Catch-Up Archiver.
\end{minipage}
\\ \cdashline{1-2}
Requirement Priority &  \\ \cdashline{1-2}
Upper Level Requirement &
\begin{tabular}{cl}
\end{tabular}
\\ \hline
\end{longtable}
}


  
 \newpage 
\subsection{[LVV-5424] OCS-DM-COM-ICD-0043-V-02: Image Retrieval for Archiving Event\_DM\_2 }\label{lvv-5424}

\begin{longtable}{cccc}
\hline
\textbf{Jira Link} & \textbf{Assignee} & \textbf{Status} & \textbf{Test Cases}\\ \hline
\href{https://jira.lsstcorp.org/browse/LVV-5424}{LVV-5424} &
Leanne Guy & Not Covered &
\begin{tabular}{c}
\end{tabular}
\\
\hline
\end{longtable}

\textbf{Verification Element Description:} \\
Undefined

{\footnotesize
\begin{longtable}{p{2.5cm}p{13.5cm}}
\hline
\multicolumn{2}{c}{\textbf{Requirement Details}}\\ \hline
Requirement ID & OCS-DM-COM-ICD-0043 \\ \cdashline{1-2}
Requirement Description &
\begin{minipage}[]{13cm}
\textbf{Specification:} DM shall publish an event indicating that a
complete image, including all configured portions of the focal plane,
was successfully retrieved from the Camera DAQ or Auxiliary Telescope
Spectrograph DAQ by an Archiver or Catch-Up Archiver Forwarder.~ This
event shall include the camera (Auxiliary Telescope Spectrograph,
ComCam, LSSTCam), the image name, and an indication as to whether this
was performed by the normal Archiver or the Catch-Up Archiver.
\end{minipage}
\\ \cdashline{1-2}
Requirement Priority &  \\ \cdashline{1-2}
Upper Level Requirement &
\begin{tabular}{cl}
\end{tabular}
\\ \hline
\end{longtable}
}


  
 \newpage 
\subsection{[LVV-5429] OCS-DM-COM-ICD-0044-V-01: Image Retrieval For Processing Event\_DM\_1 }\label{lvv-5429}

\begin{longtable}{cccc}
\hline
\textbf{Jira Link} & \textbf{Assignee} & \textbf{Status} & \textbf{Test Cases}\\ \hline
\href{https://jira.lsstcorp.org/browse/LVV-5429}{LVV-5429} &
Leanne Guy & Not Covered &
\begin{tabular}{c}
\end{tabular}
\\
\hline
\end{longtable}

\textbf{Verification Element Description:} \\
Undefined

{\footnotesize
\begin{longtable}{p{2.5cm}p{13.5cm}}
\hline
\multicolumn{2}{c}{\textbf{Requirement Details}}\\ \hline
Requirement ID & OCS-DM-COM-ICD-0044 \\ \cdashline{1-2}
Requirement Description &
\begin{minipage}[]{13cm}
\textbf{Specification}: DM shall publish an event indicating that a
complete image, including all configured portions of the focal plane,
was successfully retrieved from the Camera DAQ by a Prompt Processing
Forwarder.~ This event shall include the image name.
\end{minipage}
\\ \cdashline{1-2}
Requirement Priority &  \\ \cdashline{1-2}
Upper Level Requirement &
\begin{tabular}{cl}
\end{tabular}
\\ \hline
\end{longtable}
}


  
 \newpage 
\subsection{[LVV-5430] OCS-DM-COM-ICD-0044-V-02: Image Retrieval For Processing Event\_DM\_2 }\label{lvv-5430}

\begin{longtable}{cccc}
\hline
\textbf{Jira Link} & \textbf{Assignee} & \textbf{Status} & \textbf{Test Cases}\\ \hline
\href{https://jira.lsstcorp.org/browse/LVV-5430}{LVV-5430} &
Leanne Guy & Not Covered &
\begin{tabular}{c}
\end{tabular}
\\
\hline
\end{longtable}

\textbf{Verification Element Description:} \\
Undefined

{\footnotesize
\begin{longtable}{p{2.5cm}p{13.5cm}}
\hline
\multicolumn{2}{c}{\textbf{Requirement Details}}\\ \hline
Requirement ID & OCS-DM-COM-ICD-0044 \\ \cdashline{1-2}
Requirement Description &
\begin{minipage}[]{13cm}
\textbf{Specification}: DM shall publish an event indicating that a
complete image, including all configured portions of the focal plane,
was successfully retrieved from the Camera DAQ by a Prompt Processing
Forwarder.~ This event shall include the image name.
\end{minipage}
\\ \cdashline{1-2}
Requirement Priority &  \\ \cdashline{1-2}
Upper Level Requirement &
\begin{tabular}{cl}
\end{tabular}
\\ \hline
\end{longtable}
}


  
 \newpage 
\subsection{[LVV-5435] OCS-DM-COM-ICD-0052-V-01: Number of Alerts Information\_DM\_1 }\label{lvv-5435}

\begin{longtable}{cccc}
\hline
\textbf{Jira Link} & \textbf{Assignee} & \textbf{Status} & \textbf{Test Cases}\\ \hline
\href{https://jira.lsstcorp.org/browse/LVV-5435}{LVV-5435} &
Leanne Guy & Not Covered &
\begin{tabular}{c}
\end{tabular}
\\
\hline
\end{longtable}

\textbf{Verification Element Description:} \\
Undefined

{\footnotesize
\begin{longtable}{p{2.5cm}p{13.5cm}}
\hline
\multicolumn{2}{c}{\textbf{Requirement Details}}\\ \hline
Requirement ID & OCS-DM-COM-ICD-0052 \\ \cdashline{1-2}
Requirement Description &
\begin{minipage}[]{13cm}
\textbf{Specification}: DM shall publish as telemetry, for each detector
in each exposure successfully processed by the Alert Production Payload
in the Prompt Processing service or by any similar offline processing,
the number of Alerts sent due to DiaSources found in that detector.
\end{minipage}
\\ \cdashline{1-2}
Requirement Priority &  \\ \cdashline{1-2}
Upper Level Requirement &
\begin{tabular}{cl}
\end{tabular}
\\ \hline
\end{longtable}
}


  
 \newpage 
\subsection{[LVV-5436] OCS-DM-COM-ICD-0052-V-02: Number of Alerts Information\_DM\_2 }\label{lvv-5436}

\begin{longtable}{cccc}
\hline
\textbf{Jira Link} & \textbf{Assignee} & \textbf{Status} & \textbf{Test Cases}\\ \hline
\href{https://jira.lsstcorp.org/browse/LVV-5436}{LVV-5436} &
Leanne Guy & Not Covered &
\begin{tabular}{c}
\end{tabular}
\\
\hline
\end{longtable}

\textbf{Verification Element Description:} \\
Undefined

{\footnotesize
\begin{longtable}{p{2.5cm}p{13.5cm}}
\hline
\multicolumn{2}{c}{\textbf{Requirement Details}}\\ \hline
Requirement ID & OCS-DM-COM-ICD-0052 \\ \cdashline{1-2}
Requirement Description &
\begin{minipage}[]{13cm}
\textbf{Specification}: DM shall publish as telemetry, for each detector
in each exposure successfully processed by the Alert Production Payload
in the Prompt Processing service or by any similar offline processing,
the number of Alerts sent due to DiaSources found in that detector.
\end{minipage}
\\ \cdashline{1-2}
Requirement Priority &  \\ \cdashline{1-2}
Upper Level Requirement &
\begin{tabular}{cl}
\end{tabular}
\\ \hline
\end{longtable}
}


  
 \newpage 
\subsection{[LVV-5441] OCS-DM-COM-ICD-0051-V-01: Photometric Zeropoint Information\_DM\_1 }\label{lvv-5441}

\begin{longtable}{cccc}
\hline
\textbf{Jira Link} & \textbf{Assignee} & \textbf{Status} & \textbf{Test Cases}\\ \hline
\href{https://jira.lsstcorp.org/browse/LVV-5441}{LVV-5441} &
Leanne Guy & Not Covered &
\begin{tabular}{c}
\end{tabular}
\\
\hline
\end{longtable}

\textbf{Verification Element Description:} \\
Undefined

{\footnotesize
\begin{longtable}{p{2.5cm}p{13.5cm}}
\hline
\multicolumn{2}{c}{\textbf{Requirement Details}}\\ \hline
Requirement ID & OCS-DM-COM-ICD-0051 \\ \cdashline{1-2}
Requirement Description &
\begin{minipage}[]{13cm}
\textbf{Specification}: DM shall publish as telemetry, for each detector
in each exposure successfully processed by the Alert Production Payload
in the Prompt Processing service or by any similar offline processing,
the flux of a zero-magnitude object (double, in units of ADUs) and the
error in the flux (double, in units of ADUs).
\end{minipage}
\\ \cdashline{1-2}
Requirement Priority &  \\ \cdashline{1-2}
Upper Level Requirement &
\begin{tabular}{cl}
\end{tabular}
\\ \hline
\end{longtable}
}


  
 \newpage 
\subsection{[LVV-5442] OCS-DM-COM-ICD-0051-V-02: Photometric Zeropoint Information\_DM\_2 }\label{lvv-5442}

\begin{longtable}{cccc}
\hline
\textbf{Jira Link} & \textbf{Assignee} & \textbf{Status} & \textbf{Test Cases}\\ \hline
\href{https://jira.lsstcorp.org/browse/LVV-5442}{LVV-5442} &
Leanne Guy & Not Covered &
\begin{tabular}{c}
\end{tabular}
\\
\hline
\end{longtable}

\textbf{Verification Element Description:} \\
Undefined

{\footnotesize
\begin{longtable}{p{2.5cm}p{13.5cm}}
\hline
\multicolumn{2}{c}{\textbf{Requirement Details}}\\ \hline
Requirement ID & OCS-DM-COM-ICD-0051 \\ \cdashline{1-2}
Requirement Description &
\begin{minipage}[]{13cm}
\textbf{Specification}: DM shall publish as telemetry, for each detector
in each exposure successfully processed by the Alert Production Payload
in the Prompt Processing service or by any similar offline processing,
the flux of a zero-magnitude object (double, in units of ADUs) and the
error in the flux (double, in units of ADUs).
\end{minipage}
\\ \cdashline{1-2}
Requirement Priority &  \\ \cdashline{1-2}
Upper Level Requirement &
\begin{tabular}{cl}
\end{tabular}
\\ \hline
\end{longtable}
}


  
 \newpage 
\subsection{[LVV-5447] OCS-DM-COM-ICD-0056-V-01: Prompt Processing Resource Availability\_DM\_1 }\label{lvv-5447}

\begin{longtable}{cccc}
\hline
\textbf{Jira Link} & \textbf{Assignee} & \textbf{Status} & \textbf{Test Cases}\\ \hline
\href{https://jira.lsstcorp.org/browse/LVV-5447}{LVV-5447} &
Leanne Guy & Not Covered &
\begin{tabular}{c}
\end{tabular}
\\
\hline
\end{longtable}

\textbf{Verification Element Description:} \\
Undefined

{\footnotesize
\begin{longtable}{p{2.5cm}p{13.5cm}}
\hline
\multicolumn{2}{c}{\textbf{Requirement Details}}\\ \hline
Requirement ID & OCS-DM-COM-ICD-0056 \\ \cdashline{1-2}
Requirement Description &
\begin{minipage}[]{13cm}
\textbf{Specification}: DM shall publish as telemetry, at intervals of
\textbf{dmRsrcInterval}, the number of Prompt Processing Forwarder nodes
available, the number of Prompt Processing Distributor nodes available,
the load average on each node (float), the percentage of memory in use
on each node, and the percentage of disk space in use on each local
filesystem on each node.
\end{minipage}
\\ \cdashline{1-2}
Requirement Discussion &
\begin{minipage}[]{13cm}
\textbf{Discussion}: The value type for the number of Prompt Processing
Forwarder nodes available is ``int'', for the number of Prompt
Processing Distributor nodes available is ``int'', the load average on
each node is ``float', the percentage of memory in use on each node is
''float', and the percentage of disk space in use on each local
filesystem on each node is a ``float''.
\end{minipage}
\\ \cdashline{1-2}
Requirement Priority &  \\ \cdashline{1-2}
Upper Level Requirement &
\begin{tabular}{cl}
\end{tabular}
\\ \hline
\end{longtable}
}


  
 \newpage 
\subsection{[LVV-5448] OCS-DM-COM-ICD-0056-V-02: Prompt Processing Resource Availability\_DM\_2 }\label{lvv-5448}

\begin{longtable}{cccc}
\hline
\textbf{Jira Link} & \textbf{Assignee} & \textbf{Status} & \textbf{Test Cases}\\ \hline
\href{https://jira.lsstcorp.org/browse/LVV-5448}{LVV-5448} &
Leanne Guy & Not Covered &
\begin{tabular}{c}
\end{tabular}
\\
\hline
\end{longtable}

\textbf{Verification Element Description:} \\
Undefined

{\footnotesize
\begin{longtable}{p{2.5cm}p{13.5cm}}
\hline
\multicolumn{2}{c}{\textbf{Requirement Details}}\\ \hline
Requirement ID & OCS-DM-COM-ICD-0056 \\ \cdashline{1-2}
Requirement Description &
\begin{minipage}[]{13cm}
\textbf{Specification}: DM shall publish as telemetry, at intervals of
\textbf{dmRsrcInterval}, the number of Prompt Processing Forwarder nodes
available, the number of Prompt Processing Distributor nodes available,
the load average on each node (float), the percentage of memory in use
on each node, and the percentage of disk space in use on each local
filesystem on each node.
\end{minipage}
\\ \cdashline{1-2}
Requirement Discussion &
\begin{minipage}[]{13cm}
\textbf{Discussion}: The value type for the number of Prompt Processing
Forwarder nodes available is ``int'', for the number of Prompt
Processing Distributor nodes available is ``int'', the load average on
each node is ``float', the percentage of memory in use on each node is
''float', and the percentage of disk space in use on each local
filesystem on each node is a ``float''.
\end{minipage}
\\ \cdashline{1-2}
Requirement Priority &  \\ \cdashline{1-2}
Upper Level Requirement &
\begin{tabular}{cl}
\end{tabular}
\\ \hline
\end{longtable}
}


  
 \newpage 
\subsection{[LVV-5453] OCS-DM-COM-ICD-0050-V-01: PSF Information\_DM\_1 }\label{lvv-5453}

\begin{longtable}{cccc}
\hline
\textbf{Jira Link} & \textbf{Assignee} & \textbf{Status} & \textbf{Test Cases}\\ \hline
\href{https://jira.lsstcorp.org/browse/LVV-5453}{LVV-5453} &
Leanne Guy & Not Covered &
\begin{tabular}{c}
\end{tabular}
\\
\hline
\end{longtable}

\textbf{Verification Element Description:} \\
Undefined

{\footnotesize
\begin{longtable}{p{2.5cm}p{13.5cm}}
\hline
\multicolumn{2}{c}{\textbf{Requirement Details}}\\ \hline
Requirement ID & OCS-DM-COM-ICD-0050 \\ \cdashline{1-2}
Requirement Description &
\begin{minipage}[]{13cm}
\textbf{Specification}: DM shall publish as telemetry, for each detector
in each exposure successfully processed by the Alert Production or Raw
Calibration Validation Payloads in the Prompt Processing service or by
any similar offline processing, the following items derived from a PSF
(Point Spread Function) model: full width at half maximum (double),
Ixx/Iyy/Ixy quadrupole representation of ellipse (three doubles).
\end{minipage}
\\ \cdashline{1-2}
Requirement Discussion &
\begin{minipage}[]{13cm}
\textbf{Discussion}: Note that this telemetry message may be the same as
is provided for TCS in \citeds{LSE-75}.
\end{minipage}
\\ \cdashline{1-2}
Requirement Priority &  \\ \cdashline{1-2}
Upper Level Requirement &
\begin{tabular}{cl}
\end{tabular}
\\ \hline
\end{longtable}
}


  
 \newpage 
\subsection{[LVV-5454] OCS-DM-COM-ICD-0050-V-02: PSF Information\_DM\_2 }\label{lvv-5454}

\begin{longtable}{cccc}
\hline
\textbf{Jira Link} & \textbf{Assignee} & \textbf{Status} & \textbf{Test Cases}\\ \hline
\href{https://jira.lsstcorp.org/browse/LVV-5454}{LVV-5454} &
Leanne Guy & Not Covered &
\begin{tabular}{c}
\end{tabular}
\\
\hline
\end{longtable}

\textbf{Verification Element Description:} \\
Undefined

{\footnotesize
\begin{longtable}{p{2.5cm}p{13.5cm}}
\hline
\multicolumn{2}{c}{\textbf{Requirement Details}}\\ \hline
Requirement ID & OCS-DM-COM-ICD-0050 \\ \cdashline{1-2}
Requirement Description &
\begin{minipage}[]{13cm}
\textbf{Specification}: DM shall publish as telemetry, for each detector
in each exposure successfully processed by the Alert Production or Raw
Calibration Validation Payloads in the Prompt Processing service or by
any similar offline processing, the following items derived from a PSF
(Point Spread Function) model: full width at half maximum (double),
Ixx/Iyy/Ixy quadrupole representation of ellipse (three doubles).
\end{minipage}
\\ \cdashline{1-2}
Requirement Discussion &
\begin{minipage}[]{13cm}
\textbf{Discussion}: Note that this telemetry message may be the same as
is provided for TCS in \citeds{LSE-75}.
\end{minipage}
\\ \cdashline{1-2}
Requirement Priority &  \\ \cdashline{1-2}
Upper Level Requirement &
\begin{tabular}{cl}
\end{tabular}
\\ \hline
\end{longtable}
}


  
 \newpage 
\subsection{[LVV-5459] OCS-DM-COM-ICD-0053-V-01: Summit-Base Network Utilization\_DM\_1 }\label{lvv-5459}

\begin{longtable}{cccc}
\hline
\textbf{Jira Link} & \textbf{Assignee} & \textbf{Status} & \textbf{Test Cases}\\ \hline
\href{https://jira.lsstcorp.org/browse/LVV-5459}{LVV-5459} &
Leanne Guy & Not Covered &
\begin{tabular}{c}
\end{tabular}
\\
\hline
\end{longtable}

\textbf{Verification Element Description:} \\
Undefined

{\footnotesize
\begin{longtable}{p{2.5cm}p{13.5cm}}
\hline
\multicolumn{2}{c}{\textbf{Requirement Details}}\\ \hline
Requirement ID & OCS-DM-COM-ICD-0053 \\ \cdashline{1-2}
Requirement Description &
\begin{minipage}[]{13cm}
\textbf{Specification}: DM shall publish as telemetry, at intervals of
\textbf{netUtilInterval}, the percent utilization of each Summit-Base
network link in each direction. The intervals shall be at 300 seconds,
and the data reported is the utilization over the previous 300 second
interval, as well as a measurement of the round-trip time in each
direction.
\end{minipage}
\\ \cdashline{1-2}
Requirement Discussion &
\begin{minipage}[]{13cm}
\textbf{Discussion}: The value type for the percent utilization and the
round trip time is "float' and the unit is in seconds.
\end{minipage}
\\ \cdashline{1-2}
Requirement Priority &  \\ \cdashline{1-2}
Upper Level Requirement &
\begin{tabular}{cl}
\end{tabular}
\\ \hline
\end{longtable}
}


  
 \newpage 
\subsection{[LVV-5460] OCS-DM-COM-ICD-0053-V-02: Summit-Base Network Utilization\_DM\_2 }\label{lvv-5460}

\begin{longtable}{cccc}
\hline
\textbf{Jira Link} & \textbf{Assignee} & \textbf{Status} & \textbf{Test Cases}\\ \hline
\href{https://jira.lsstcorp.org/browse/LVV-5460}{LVV-5460} &
Leanne Guy & Not Covered &
\begin{tabular}{c}
\end{tabular}
\\
\hline
\end{longtable}

\textbf{Verification Element Description:} \\
Undefined

{\footnotesize
\begin{longtable}{p{2.5cm}p{13.5cm}}
\hline
\multicolumn{2}{c}{\textbf{Requirement Details}}\\ \hline
Requirement ID & OCS-DM-COM-ICD-0053 \\ \cdashline{1-2}
Requirement Description &
\begin{minipage}[]{13cm}
\textbf{Specification}: DM shall publish as telemetry, at intervals of
\textbf{netUtilInterval}, the percent utilization of each Summit-Base
network link in each direction. The intervals shall be at 300 seconds,
and the data reported is the utilization over the previous 300 second
interval, as well as a measurement of the round-trip time in each
direction.
\end{minipage}
\\ \cdashline{1-2}
Requirement Discussion &
\begin{minipage}[]{13cm}
\textbf{Discussion}: The value type for the percent utilization and the
round trip time is "float' and the unit is in seconds.
\end{minipage}
\\ \cdashline{1-2}
Requirement Priority &  \\ \cdashline{1-2}
Upper Level Requirement &
\begin{tabular}{cl}
\end{tabular}
\\ \hline
\end{longtable}
}


  
 \newpage 
\subsection{[LVV-5465] OCS-DM-COM-ICD-0022-V-01: System Health Metrics\_DM\_1 }\label{lvv-5465}

\begin{longtable}{cccc}
\hline
\textbf{Jira Link} & \textbf{Assignee} & \textbf{Status} & \textbf{Test Cases}\\ \hline
\href{https://jira.lsstcorp.org/browse/LVV-5465}{LVV-5465} &
Leanne Guy & Not Covered &
\begin{tabular}{c}
\end{tabular}
\\
\hline
\end{longtable}

\textbf{Verification Element Description:} \\
Undefined

{\footnotesize
\begin{longtable}{p{2.5cm}p{13.5cm}}
\hline
\multicolumn{2}{c}{\textbf{Requirement Details}}\\ \hline
Requirement ID & OCS-DM-COM-ICD-0022 \\ \cdashline{1-2}
Requirement Description &
\begin{minipage}[]{13cm}
\textbf{Specification}: Data Management shall publish a basic set of
metrics on the operational health of the live data processing and
archiving systems. This shall include:

\begin{itemize}
\tightlist
\item
  Utilization statistics on the Summit-Base and Base-Archive network
  links
\item
  Assessment of whether sufficient resources are available to perform
  the configured archiving and processing functions.
\item
  Amount of time taken by data transfers and data processing.
\end{itemize}
\end{minipage}
\\ \cdashline{1-2}
Requirement Discussion &
\begin{minipage}[]{13cm}
\textbf{Discussion}: This is not intended to supplant the health
monitoring and control functions of the Data Management Control System,
but only to permit the central Observatory operator consoles to provide
operators a ``red/green'' indication of DM system health, enabling them
to consult DM operations experts for further information when there are
problems.\\
\hspace*{0.333em}
\end{minipage}
\\ \cdashline{1-2}
Requirement Priority &  \\ \cdashline{1-2}
Upper Level Requirement &
\begin{tabular}{cl}
\end{tabular}
\\ \hline
\end{longtable}
}


  
 \newpage 
\subsection{[LVV-5466] OCS-DM-COM-ICD-0022-V-02: System Health Metrics\_DM\_2 }\label{lvv-5466}

\begin{longtable}{cccc}
\hline
\textbf{Jira Link} & \textbf{Assignee} & \textbf{Status} & \textbf{Test Cases}\\ \hline
\href{https://jira.lsstcorp.org/browse/LVV-5466}{LVV-5466} &
Leanne Guy & Not Covered &
\begin{tabular}{c}
\end{tabular}
\\
\hline
\end{longtable}

\textbf{Verification Element Description:} \\
Undefined

{\footnotesize
\begin{longtable}{p{2.5cm}p{13.5cm}}
\hline
\multicolumn{2}{c}{\textbf{Requirement Details}}\\ \hline
Requirement ID & OCS-DM-COM-ICD-0022 \\ \cdashline{1-2}
Requirement Description &
\begin{minipage}[]{13cm}
\textbf{Specification}: Data Management shall publish a basic set of
metrics on the operational health of the live data processing and
archiving systems. This shall include:

\begin{itemize}
\tightlist
\item
  Utilization statistics on the Summit-Base and Base-Archive network
  links
\item
  Assessment of whether sufficient resources are available to perform
  the configured archiving and processing functions.
\item
  Amount of time taken by data transfers and data processing.
\end{itemize}
\end{minipage}
\\ \cdashline{1-2}
Requirement Discussion &
\begin{minipage}[]{13cm}
\textbf{Discussion}: This is not intended to supplant the health
monitoring and control functions of the Data Management Control System,
but only to permit the central Observatory operator consoles to provide
operators a ``red/green'' indication of DM system health, enabling them
to consult DM operations experts for further information when there are
problems.\\
\hspace*{0.333em}
\end{minipage}
\\ \cdashline{1-2}
Requirement Priority &  \\ \cdashline{1-2}
Upper Level Requirement &
\begin{tabular}{cl}
\end{tabular}
\\ \hline
\end{longtable}
}


  
 \newpage 
\subsection{[LVV-5471] OCS-DM-COM-ICD-0049-V-01: WCS Information\_DM\_1 }\label{lvv-5471}

\begin{longtable}{cccc}
\hline
\textbf{Jira Link} & \textbf{Assignee} & \textbf{Status} & \textbf{Test Cases}\\ \hline
\href{https://jira.lsstcorp.org/browse/LVV-5471}{LVV-5471} &
Leanne Guy & Not Covered &
\begin{tabular}{c}
\end{tabular}
\\
\hline
\end{longtable}

\textbf{Verification Element Description:} \\
Undefined

{\footnotesize
\begin{longtable}{p{2.5cm}p{13.5cm}}
\hline
\multicolumn{2}{c}{\textbf{Requirement Details}}\\ \hline
Requirement ID & OCS-DM-COM-ICD-0049 \\ \cdashline{1-2}
Requirement Description &
\begin{minipage}[]{13cm}
\textbf{Specification}: DM shall publish WCS (World Coordinate System)
telemetry for each visit successfully processed by the Alert Production
Payload in the Prompt Processing service or by any similar offline
processing.~ This telemetry shall contain the equinox (double, currently
2000.0), system (string, currently `ICRS'), unit (string, currently
`deg'), and then, for each sensor, reference pixel x/y coordinates (two
doubles), reference pixel RA/dec coordinates (two doubles), and rotation
and scale matrix (four doubles).
\end{minipage}
\\ \cdashline{1-2}
Requirement Discussion &
\begin{minipage}[]{13cm}
\textbf{Discussion}: Note that this telemetry message may be the same as
is provided for TCS in \citeds{LSE-75}.
\end{minipage}
\\ \cdashline{1-2}
Requirement Priority &  \\ \cdashline{1-2}
Upper Level Requirement &
\begin{tabular}{cl}
\end{tabular}
\\ \hline
\end{longtable}
}


  
 \newpage 
\subsection{[LVV-5472] OCS-DM-COM-ICD-0049-V-02: WCS Information\_DM\_2 }\label{lvv-5472}

\begin{longtable}{cccc}
\hline
\textbf{Jira Link} & \textbf{Assignee} & \textbf{Status} & \textbf{Test Cases}\\ \hline
\href{https://jira.lsstcorp.org/browse/LVV-5472}{LVV-5472} &
Leanne Guy & Not Covered &
\begin{tabular}{c}
\end{tabular}
\\
\hline
\end{longtable}

\textbf{Verification Element Description:} \\
Undefined

{\footnotesize
\begin{longtable}{p{2.5cm}p{13.5cm}}
\hline
\multicolumn{2}{c}{\textbf{Requirement Details}}\\ \hline
Requirement ID & OCS-DM-COM-ICD-0049 \\ \cdashline{1-2}
Requirement Description &
\begin{minipage}[]{13cm}
\textbf{Specification}: DM shall publish WCS (World Coordinate System)
telemetry for each visit successfully processed by the Alert Production
Payload in the Prompt Processing service or by any similar offline
processing.~ This telemetry shall contain the equinox (double, currently
2000.0), system (string, currently `ICRS'), unit (string, currently
`deg'), and then, for each sensor, reference pixel x/y coordinates (two
doubles), reference pixel RA/dec coordinates (two doubles), and rotation
and scale matrix (four doubles).
\end{minipage}
\\ \cdashline{1-2}
Requirement Discussion &
\begin{minipage}[]{13cm}
\textbf{Discussion}: Note that this telemetry message may be the same as
is provided for TCS in \citeds{LSE-75}.
\end{minipage}
\\ \cdashline{1-2}
Requirement Priority &  \\ \cdashline{1-2}
Upper Level Requirement &
\begin{tabular}{cl}
\end{tabular}
\\ \hline
\end{longtable}
}


  
 \newpage 
\subsection{[LVV-5477] OCS-DM-COM-ICD-0023-V-01: Basic Query Functionality Required by
DM\_DM\_1 }\label{lvv-5477}

\begin{longtable}{cccc}
\hline
\textbf{Jira Link} & \textbf{Assignee} & \textbf{Status} & \textbf{Test Cases}\\ \hline
\href{https://jira.lsstcorp.org/browse/LVV-5477}{LVV-5477} &
Leanne Guy & Not Covered &
\begin{tabular}{c}
\end{tabular}
\\
\hline
\end{longtable}

\textbf{Verification Element Description:} \\
Undefined

{\footnotesize
\begin{longtable}{p{2.5cm}p{13.5cm}}
\hline
\multicolumn{2}{c}{\textbf{Requirement Details}}\\ \hline
Requirement ID & OCS-DM-COM-ICD-0023 \\ \cdashline{1-2}
Requirement Description &
\begin{minipage}[]{13cm}
\textbf{Specification}: The OCS shall provide an ``sqlclient'' interface
for querying the temporal data in the Engineering and Facilities
Database. The database shall support temporal queries for commands,
events, and telemetry based on the publication time of the associated
messages, and, for telemetry, based on the measurement time subsystems
are required to provide.
\end{minipage}
\\ \cdashline{1-2}
Requirement Priority &  \\ \cdashline{1-2}
Upper Level Requirement &
\begin{tabular}{cl}
\end{tabular}
\\ \hline
\end{longtable}
}


  
 \newpage 
\subsection{[LVV-5478] OCS-DM-COM-ICD-0023-V-02: Basic Query Functionality Required by
DM\_DM\_2 }\label{lvv-5478}

\begin{longtable}{cccc}
\hline
\textbf{Jira Link} & \textbf{Assignee} & \textbf{Status} & \textbf{Test Cases}\\ \hline
\href{https://jira.lsstcorp.org/browse/LVV-5478}{LVV-5478} &
Leanne Guy & Not Covered &
\begin{tabular}{c}
\end{tabular}
\\
\hline
\end{longtable}

\textbf{Verification Element Description:} \\
Undefined

{\footnotesize
\begin{longtable}{p{2.5cm}p{13.5cm}}
\hline
\multicolumn{2}{c}{\textbf{Requirement Details}}\\ \hline
Requirement ID & OCS-DM-COM-ICD-0023 \\ \cdashline{1-2}
Requirement Description &
\begin{minipage}[]{13cm}
\textbf{Specification}: The OCS shall provide an ``sqlclient'' interface
for querying the temporal data in the Engineering and Facilities
Database. The database shall support temporal queries for commands,
events, and telemetry based on the publication time of the associated
messages, and, for telemetry, based on the measurement time subsystems
are required to provide.
\end{minipage}
\\ \cdashline{1-2}
Requirement Priority &  \\ \cdashline{1-2}
Upper Level Requirement &
\begin{tabular}{cl}
\end{tabular}
\\ \hline
\end{longtable}
}


  
 \newpage 
\subsection{[LVV-5483] OCS-DM-COM-ICD-0025-V-01: Expected Load of Queries from DM\_DM\_1 }\label{lvv-5483}

\begin{longtable}{cccc}
\hline
\textbf{Jira Link} & \textbf{Assignee} & \textbf{Status} & \textbf{Test Cases}\\ \hline
\href{https://jira.lsstcorp.org/browse/LVV-5483}{LVV-5483} &
Leanne Guy & Not Covered &
\begin{tabular}{c}
\end{tabular}
\\
\hline
\end{longtable}

\textbf{Verification Element Description:} \\
Undefined

{\footnotesize
\begin{longtable}{p{2.5cm}p{13.5cm}}
\hline
\multicolumn{2}{c}{\textbf{Requirement Details}}\\ \hline
Requirement ID & OCS-DM-COM-ICD-0025 \\ \cdashline{1-2}
Requirement Description &
\begin{minipage}[]{13cm}
\textbf{Specification:} The OCS shall support, and DM shall not exceed,
\textbf{queryRateDMEFD} level of EFD queries from DM against each table
within the OCS-maintained Base instance(s) of the EFD.\\
\hspace*{0.333em}
\end{minipage}
\\ \cdashline{1-2}
Requirement Parameters & \textbf{queryRateDMEFD = TBD{{[}TBD{]}}} Maximum rate of DM queries
against OCS EFD instance(s). \\ \cdashline{1-2}
Requirement Priority &  \\ \cdashline{1-2}
Upper Level Requirement &
\begin{tabular}{cl}
\end{tabular}
\\ \hline
\end{longtable}
}


  
 \newpage 
\subsection{[LVV-5484] OCS-DM-COM-ICD-0025-V-02: Expected Load of Queries from DM\_DM\_2 }\label{lvv-5484}

\begin{longtable}{cccc}
\hline
\textbf{Jira Link} & \textbf{Assignee} & \textbf{Status} & \textbf{Test Cases}\\ \hline
\href{https://jira.lsstcorp.org/browse/LVV-5484}{LVV-5484} &
Leanne Guy & Not Covered &
\begin{tabular}{c}
\end{tabular}
\\
\hline
\end{longtable}

\textbf{Verification Element Description:} \\
Undefined

{\footnotesize
\begin{longtable}{p{2.5cm}p{13.5cm}}
\hline
\multicolumn{2}{c}{\textbf{Requirement Details}}\\ \hline
Requirement ID & OCS-DM-COM-ICD-0025 \\ \cdashline{1-2}
Requirement Description &
\begin{minipage}[]{13cm}
\textbf{Specification:} The OCS shall support, and DM shall not exceed,
\textbf{queryRateDMEFD} level of EFD queries from DM against each table
within the OCS-maintained Base instance(s) of the EFD.\\
\hspace*{0.333em}
\end{minipage}
\\ \cdashline{1-2}
Requirement Parameters & \textbf{queryRateDMEFD = TBD{{[}TBD{]}}} Maximum rate of DM queries
against OCS EFD instance(s). \\ \cdashline{1-2}
Requirement Priority &  \\ \cdashline{1-2}
Upper Level Requirement &
\begin{tabular}{cl}
\end{tabular}
\\ \hline
\end{longtable}
}


  
 \newpage 
\subsection{[LVV-5489] OCS-DM-COM-ICD-0029-V-01: Archive Latency\_DM\_1 }\label{lvv-5489}

\begin{longtable}{cccc}
\hline
\textbf{Jira Link} & \textbf{Assignee} & \textbf{Status} & \textbf{Test Cases}\\ \hline
\href{https://jira.lsstcorp.org/browse/LVV-5489}{LVV-5489} &
Leanne Guy & Not Covered &
\begin{tabular}{c}
\end{tabular}
\\
\hline
\end{longtable}

\textbf{Verification Element Description:} \\
Undefined

{\footnotesize
\begin{longtable}{p{2.5cm}p{13.5cm}}
\hline
\multicolumn{2}{c}{\textbf{Requirement Details}}\\ \hline
Requirement ID & OCS-DM-COM-ICD-0029 \\ \cdashline{1-2}
Requirement Description &
\begin{minipage}[]{13cm}
\textbf{Specification}: Data Management shall ensure that data are
available for query in the Transformed EFD copies within no more than
time \textbf{efdArchiveLatency} of the storage of new data to the OCS
copy of the EFD.\\
\hspace*{0.333em}
\end{minipage}
\\ \cdashline{1-2}
Requirement Parameters & \textbf{efdArchiveLatency = 3600{{[}second{]}}} Maximum latency time for
the availability of data for query in the DM replicas of the EFD. \\ \cdashline{1-2}
Requirement Priority &  \\ \cdashline{1-2}
Upper Level Requirement &
\begin{tabular}{cl}
\end{tabular}
\\ \hline
\end{longtable}
}


  
 \newpage 
\subsection{[LVV-5490] OCS-DM-COM-ICD-0029-V-02: Archive Latency\_DM\_2 }\label{lvv-5490}

\begin{longtable}{cccc}
\hline
\textbf{Jira Link} & \textbf{Assignee} & \textbf{Status} & \textbf{Test Cases}\\ \hline
\href{https://jira.lsstcorp.org/browse/LVV-5490}{LVV-5490} &
Leanne Guy & Not Covered &
\begin{tabular}{c}
\end{tabular}
\\
\hline
\end{longtable}

\textbf{Verification Element Description:} \\
Undefined

{\footnotesize
\begin{longtable}{p{2.5cm}p{13.5cm}}
\hline
\multicolumn{2}{c}{\textbf{Requirement Details}}\\ \hline
Requirement ID & OCS-DM-COM-ICD-0029 \\ \cdashline{1-2}
Requirement Description &
\begin{minipage}[]{13cm}
\textbf{Specification}: Data Management shall ensure that data are
available for query in the Transformed EFD copies within no more than
time \textbf{efdArchiveLatency} of the storage of new data to the OCS
copy of the EFD.\\
\hspace*{0.333em}
\end{minipage}
\\ \cdashline{1-2}
Requirement Parameters & \textbf{efdArchiveLatency = 3600{{[}second{]}}} Maximum latency time for
the availability of data for query in the DM replicas of the EFD. \\ \cdashline{1-2}
Requirement Priority &  \\ \cdashline{1-2}
Upper Level Requirement &
\begin{tabular}{cl}
\end{tabular}
\\ \hline
\end{longtable}
}


  
 \newpage 
\subsection{[LVV-5495] OCS-DM-COM-ICD-0042-V-01: EFD Disaster Recovery by Data
Management\_DM\_1 }\label{lvv-5495}

\begin{longtable}{cccc}
\hline
\textbf{Jira Link} & \textbf{Assignee} & \textbf{Status} & \textbf{Test Cases}\\ \hline
\href{https://jira.lsstcorp.org/browse/LVV-5495}{LVV-5495} &
Leanne Guy & Not Covered &
\begin{tabular}{c}
\end{tabular}
\\
\hline
\end{longtable}

\textbf{Verification Element Description:} \\
Undefined

{\footnotesize
\begin{longtable}{p{2.5cm}p{13.5cm}}
\hline
\multicolumn{2}{c}{\textbf{Requirement Details}}\\ \hline
Requirement ID & OCS-DM-COM-ICD-0042 \\ \cdashline{1-2}
Requirement Description &
\begin{minipage}[]{13cm}
\textbf{Specification}: DM shall arrange for the preservation of
snapshot backups of the EFD table and Large File Annex contents for
disaster recovery purposes.~ The OCS shall be responsible for creating
these backups and making them available as files.
\end{minipage}
\\ \cdashline{1-2}
Requirement Priority &  \\ \cdashline{1-2}
Upper Level Requirement &
\begin{tabular}{cl}
\end{tabular}
\\ \hline
\end{longtable}
}


  
 \newpage 
\subsection{[LVV-5496] OCS-DM-COM-ICD-0042-V-02: EFD Disaster Recovery by Data
Management\_DM\_2 }\label{lvv-5496}

\begin{longtable}{cccc}
\hline
\textbf{Jira Link} & \textbf{Assignee} & \textbf{Status} & \textbf{Test Cases}\\ \hline
\href{https://jira.lsstcorp.org/browse/LVV-5496}{LVV-5496} &
Leanne Guy & Not Covered &
\begin{tabular}{c}
\end{tabular}
\\
\hline
\end{longtable}

\textbf{Verification Element Description:} \\
Undefined

{\footnotesize
\begin{longtable}{p{2.5cm}p{13.5cm}}
\hline
\multicolumn{2}{c}{\textbf{Requirement Details}}\\ \hline
Requirement ID & OCS-DM-COM-ICD-0042 \\ \cdashline{1-2}
Requirement Description &
\begin{minipage}[]{13cm}
\textbf{Specification}: DM shall arrange for the preservation of
snapshot backups of the EFD table and Large File Annex contents for
disaster recovery purposes.~ The OCS shall be responsible for creating
these backups and making them available as files.
\end{minipage}
\\ \cdashline{1-2}
Requirement Priority &  \\ \cdashline{1-2}
Upper Level Requirement &
\begin{tabular}{cl}
\end{tabular}
\\ \hline
\end{longtable}
}


  
 \newpage 
\subsection{[LVV-5501] OCS-DM-COM-ICD-0030-V-01: EFD Transformation Service Interface\_DM\_1 }\label{lvv-5501}

\begin{longtable}{cccc}
\hline
\textbf{Jira Link} & \textbf{Assignee} & \textbf{Status} & \textbf{Test Cases}\\ \hline
\href{https://jira.lsstcorp.org/browse/LVV-5501}{LVV-5501} &
Leanne Guy & Not Covered &
\begin{tabular}{c}
\end{tabular}
\\
\hline
\end{longtable}

\textbf{Verification Element Description:} \\
Undefined

{\footnotesize
\begin{longtable}{p{2.5cm}p{13.5cm}}
\hline
\multicolumn{2}{c}{\textbf{Requirement Details}}\\ \hline
Requirement ID & OCS-DM-COM-ICD-0030 \\ \cdashline{1-2}
Requirement Description &
\begin{minipage}[]{13cm}
\textbf{Specification}: The archiving of the EFD tables shall be
performed using standard MySQL queries.~ The OCS shall expose this
interface to the DM EFD Transformation Service CSC at the Base
Facility.\\
\hspace*{0.333em}
\end{minipage}
\\ \cdashline{1-2}
Requirement Discussion &
\begin{minipage}[]{13cm}
\textbf{Discussion}: The replication mechanism for the large-file annex
is still to be defined. In particular, it remains to be defined whether
DM will replicate all files in a specified directory or directories, or
whether DM will replicate only files referenced by
large-file-indirection telemetry messages.\\
\hspace*{0.333em}
\end{minipage}
\\ \cdashline{1-2}
Requirement Priority &  \\ \cdashline{1-2}
Upper Level Requirement &
\begin{tabular}{cl}
\end{tabular}
\\ \hline
\end{longtable}
}


  
 \newpage 
\subsection{[LVV-5502] OCS-DM-COM-ICD-0030-V-02: EFD Transformation Service Interface\_DM\_2 }\label{lvv-5502}

\begin{longtable}{cccc}
\hline
\textbf{Jira Link} & \textbf{Assignee} & \textbf{Status} & \textbf{Test Cases}\\ \hline
\href{https://jira.lsstcorp.org/browse/LVV-5502}{LVV-5502} &
Leanne Guy & Not Covered &
\begin{tabular}{c}
\end{tabular}
\\
\hline
\end{longtable}

\textbf{Verification Element Description:} \\
Undefined

{\footnotesize
\begin{longtable}{p{2.5cm}p{13.5cm}}
\hline
\multicolumn{2}{c}{\textbf{Requirement Details}}\\ \hline
Requirement ID & OCS-DM-COM-ICD-0030 \\ \cdashline{1-2}
Requirement Description &
\begin{minipage}[]{13cm}
\textbf{Specification}: The archiving of the EFD tables shall be
performed using standard MySQL queries.~ The OCS shall expose this
interface to the DM EFD Transformation Service CSC at the Base
Facility.\\
\hspace*{0.333em}
\end{minipage}
\\ \cdashline{1-2}
Requirement Discussion &
\begin{minipage}[]{13cm}
\textbf{Discussion}: The replication mechanism for the large-file annex
is still to be defined. In particular, it remains to be defined whether
DM will replicate all files in a specified directory or directories, or
whether DM will replicate only files referenced by
large-file-indirection telemetry messages.\\
\hspace*{0.333em}
\end{minipage}
\\ \cdashline{1-2}
Requirement Priority &  \\ \cdashline{1-2}
Upper Level Requirement &
\begin{tabular}{cl}
\end{tabular}
\\ \hline
\end{longtable}
}


  
 \newpage 
\subsection{[LVV-5513] OCS-DM-COM-ICD-0028-V-01: Expected Data Volume\_DM\_1 }\label{lvv-5513}

\begin{longtable}{cccc}
\hline
\textbf{Jira Link} & \textbf{Assignee} & \textbf{Status} & \textbf{Test Cases}\\ \hline
\href{https://jira.lsstcorp.org/browse/LVV-5513}{LVV-5513} &
Leanne Guy & Not Covered &
\begin{tabular}{c}
\end{tabular}
\\
\hline
\end{longtable}

\textbf{Verification Element Description:} \\
Undefined

{\footnotesize
\begin{longtable}{p{2.5cm}p{13.5cm}}
\hline
\multicolumn{2}{c}{\textbf{Requirement Details}}\\ \hline
Requirement ID & OCS-DM-COM-ICD-0028 \\ \cdashline{1-2}
Requirement Description &
\begin{minipage}[]{13cm}
\textbf{Specification}: Data Management shall support at least
\textbf{efdArchive24hVolume} of new data to be archived per 24-hour
period.\\
\hspace*{0.333em}
\end{minipage}
\\ \cdashline{1-2}
Requirement Parameters & \textbf{efdArchive14hVolume = 300{{[}gigabyte per day{]}}} Minimum
capacity of Data Management to archive EFD data. \\ \cdashline{1-2}
Requirement Discussion &
\begin{minipage}[]{13cm}
\textbf{Discussion:} The DM database server will be at least as capable
as the OCS server, making it relatively easy to keep up.
\end{minipage}
\\ \cdashline{1-2}
Requirement Priority &  \\ \cdashline{1-2}
Upper Level Requirement &
\begin{tabular}{cl}
\end{tabular}
\\ \hline
\end{longtable}
}


  
 \newpage 
\subsection{[LVV-5514] OCS-DM-COM-ICD-0028-V-02: Expected Data Volume\_DM\_2 }\label{lvv-5514}

\begin{longtable}{cccc}
\hline
\textbf{Jira Link} & \textbf{Assignee} & \textbf{Status} & \textbf{Test Cases}\\ \hline
\href{https://jira.lsstcorp.org/browse/LVV-5514}{LVV-5514} &
Leanne Guy & Not Covered &
\begin{tabular}{c}
\end{tabular}
\\
\hline
\end{longtable}

\textbf{Verification Element Description:} \\
Undefined

{\footnotesize
\begin{longtable}{p{2.5cm}p{13.5cm}}
\hline
\multicolumn{2}{c}{\textbf{Requirement Details}}\\ \hline
Requirement ID & OCS-DM-COM-ICD-0028 \\ \cdashline{1-2}
Requirement Description &
\begin{minipage}[]{13cm}
\textbf{Specification}: Data Management shall support at least
\textbf{efdArchive24hVolume} of new data to be archived per 24-hour
period.\\
\hspace*{0.333em}
\end{minipage}
\\ \cdashline{1-2}
Requirement Parameters & \textbf{efdArchive14hVolume = 300{{[}gigabyte per day{]}}} Minimum
capacity of Data Management to archive EFD data. \\ \cdashline{1-2}
Requirement Discussion &
\begin{minipage}[]{13cm}
\textbf{Discussion:} The DM database server will be at least as capable
as the OCS server, making it relatively easy to keep up.
\end{minipage}
\\ \cdashline{1-2}
Requirement Priority &  \\ \cdashline{1-2}
Upper Level Requirement &
\begin{tabular}{cl}
\end{tabular}
\\ \hline
\end{longtable}
}


  
 \newpage 
\subsection{[LVV-5519] OCS-DM-COM-ICD-0041-V-01: Large File Annex Replication Interface\_DM\_1 }\label{lvv-5519}

\begin{longtable}{cccc}
\hline
\textbf{Jira Link} & \textbf{Assignee} & \textbf{Status} & \textbf{Test Cases}\\ \hline
\href{https://jira.lsstcorp.org/browse/LVV-5519}{LVV-5519} &
Leanne Guy & Not Covered &
\begin{tabular}{c}
\end{tabular}
\\
\hline
\end{longtable}

\textbf{Verification Element Description:} \\
Undefined

{\footnotesize
\begin{longtable}{p{2.5cm}p{13.5cm}}
\hline
\multicolumn{2}{c}{\textbf{Requirement Details}}\\ \hline
Requirement ID & OCS-DM-COM-ICD-0041 \\ \cdashline{1-2}
Requirement Description &
\begin{minipage}[]{13cm}
\textbf{Specification}: The archiving of the Large File Annex contents
shall be performed using the standard \emph{rsync} tool.~ The OCS shall
expose a suitable filesystem and server to the DM EFD Transformation
Service CSC at the Base Facility.
\end{minipage}
\\ \cdashline{1-2}
Requirement Priority &  \\ \cdashline{1-2}
Upper Level Requirement &
\begin{tabular}{cl}
\end{tabular}
\\ \hline
\end{longtable}
}


  
 \newpage 
\subsection{[LVV-5520] OCS-DM-COM-ICD-0041-V-02: Large File Annex Replication Interface\_DM\_2 }\label{lvv-5520}

\begin{longtable}{cccc}
\hline
\textbf{Jira Link} & \textbf{Assignee} & \textbf{Status} & \textbf{Test Cases}\\ \hline
\href{https://jira.lsstcorp.org/browse/LVV-5520}{LVV-5520} &
Leanne Guy & Not Covered &
\begin{tabular}{c}
\end{tabular}
\\
\hline
\end{longtable}

\textbf{Verification Element Description:} \\
Undefined

{\footnotesize
\begin{longtable}{p{2.5cm}p{13.5cm}}
\hline
\multicolumn{2}{c}{\textbf{Requirement Details}}\\ \hline
Requirement ID & OCS-DM-COM-ICD-0041 \\ \cdashline{1-2}
Requirement Description &
\begin{minipage}[]{13cm}
\textbf{Specification}: The archiving of the Large File Annex contents
shall be performed using the standard \emph{rsync} tool.~ The OCS shall
expose a suitable filesystem and server to the DM EFD Transformation
Service CSC at the Base Facility.
\end{minipage}
\\ \cdashline{1-2}
Requirement Priority &  \\ \cdashline{1-2}
Upper Level Requirement &
\begin{tabular}{cl}
\end{tabular}
\\ \hline
\end{longtable}
}


  
 \newpage 
\subsection{[LVV-5531] OCS-DM-COM-ICD-0031-V-01: Advance Notice of Pointings\_DM\_1 }\label{lvv-5531}

\begin{longtable}{cccc}
\hline
\textbf{Jira Link} & \textbf{Assignee} & \textbf{Status} & \textbf{Test Cases}\\ \hline
\href{https://jira.lsstcorp.org/browse/LVV-5531}{LVV-5531} &
Leanne Guy & Not Covered &
\begin{tabular}{c}
\end{tabular}
\\
\hline
\end{longtable}

\textbf{Verification Element Description:} \\
Undefined

{\footnotesize
\begin{longtable}{p{2.5cm}p{13.5cm}}
\hline
\multicolumn{2}{c}{\textbf{Requirement Details}}\\ \hline
Requirement ID & OCS-DM-COM-ICD-0031 \\ \cdashline{1-2}
Requirement Description &
\begin{minipage}[]{13cm}
\textbf{Specification}: Advance notice of telescope pointings for
science data acquisition shall be made available to Data Management as
an OCS event, no later than time \textbf{pointingAdvanceNoticeTime}
before the start of the first exposure of a standard visit or the only
exposure of an alternate science visit. The notice shall include the sky
coordinates, rotation angle, the azimuth \& elevation angles at the
start of the first exposure, exposure duration, number of exposures,
estimate of shutter motion start time (at least 1 sec precision), filter
selection, expected air mass, and survey name (e.g., WFD, DDF-1). The
coordinate system for the sky coordinates shall be ICRS, equinox
2000.0.~ The precision and accuracy of all values shall be based on the
capability of the OCS; the precision and accuracy that the OCS is able
to achieve should be published in a separate design document so that DM
can know what they are.
\end{minipage}
\\ \cdashline{1-2}
Requirement Parameters & \textbf{pointingAdvanceNoticeTime = 20{{[}second{]}}} Time before the
first exposure of a standard visit begins by which advance notice of the
pointing must be provided to Data Management. \\ \cdashline{1-2}
Requirement Discussion &
\begin{minipage}[]{13cm}
\textbf{Discussion:} The purpose of this requirement is to permit Data
Management to pipeline the potentially costly operation of preparing
(downsampling and rotation) the subtraction template for the visit with
the processing of the data from previous visits, as well as to retrieve
reference object information from the Science Database. ~This is
essential in order to allow DM to meet the Observatory alert latency
requirement. ~\\
This is not expected to be a difficult constraint to meet during normal
operation, as similar advance knowledge is required to plan dome crawl,
for instance.\\
Note that the pipelined nature of the processing envisioned by DM means
that if the advance notice requirement is not met, and template
preparation is not possible, DM may face the choice of falling behind
for a series of visits, or of deferring the processing of the single
affected visit to avoid disrupting the pipeline. This choice would
depend on the frequency with which the interface requirement is not met,
and whether such failures occurred in bursts.\\
The inclusion of survey name in the visit qualities reported in an
advance notice will enable alternative, near real-time processing
pipelines to be triggered for Special Programs, as required by
OSS-REQ-0384 and DMS-REQ-0320.
\end{minipage}
\\ \cdashline{1-2}
Requirement Priority &  \\ \cdashline{1-2}
Upper Level Requirement &
\begin{tabular}{cl}
\end{tabular}
\\ \hline
\end{longtable}
}


  
 \newpage 
\subsection{[LVV-5532] OCS-DM-COM-ICD-0031-V-02: Advance Notice of Pointings\_DM\_2 }\label{lvv-5532}

\begin{longtable}{cccc}
\hline
\textbf{Jira Link} & \textbf{Assignee} & \textbf{Status} & \textbf{Test Cases}\\ \hline
\href{https://jira.lsstcorp.org/browse/LVV-5532}{LVV-5532} &
Leanne Guy & Not Covered &
\begin{tabular}{c}
\end{tabular}
\\
\hline
\end{longtable}

\textbf{Verification Element Description:} \\
Undefined

{\footnotesize
\begin{longtable}{p{2.5cm}p{13.5cm}}
\hline
\multicolumn{2}{c}{\textbf{Requirement Details}}\\ \hline
Requirement ID & OCS-DM-COM-ICD-0031 \\ \cdashline{1-2}
Requirement Description &
\begin{minipage}[]{13cm}
\textbf{Specification}: Advance notice of telescope pointings for
science data acquisition shall be made available to Data Management as
an OCS event, no later than time \textbf{pointingAdvanceNoticeTime}
before the start of the first exposure of a standard visit or the only
exposure of an alternate science visit. The notice shall include the sky
coordinates, rotation angle, the azimuth \& elevation angles at the
start of the first exposure, exposure duration, number of exposures,
estimate of shutter motion start time (at least 1 sec precision), filter
selection, expected air mass, and survey name (e.g., WFD, DDF-1). The
coordinate system for the sky coordinates shall be ICRS, equinox
2000.0.~ The precision and accuracy of all values shall be based on the
capability of the OCS; the precision and accuracy that the OCS is able
to achieve should be published in a separate design document so that DM
can know what they are.
\end{minipage}
\\ \cdashline{1-2}
Requirement Parameters & \textbf{pointingAdvanceNoticeTime = 20{{[}second{]}}} Time before the
first exposure of a standard visit begins by which advance notice of the
pointing must be provided to Data Management. \\ \cdashline{1-2}
Requirement Discussion &
\begin{minipage}[]{13cm}
\textbf{Discussion:} The purpose of this requirement is to permit Data
Management to pipeline the potentially costly operation of preparing
(downsampling and rotation) the subtraction template for the visit with
the processing of the data from previous visits, as well as to retrieve
reference object information from the Science Database. ~This is
essential in order to allow DM to meet the Observatory alert latency
requirement. ~\\
This is not expected to be a difficult constraint to meet during normal
operation, as similar advance knowledge is required to plan dome crawl,
for instance.\\
Note that the pipelined nature of the processing envisioned by DM means
that if the advance notice requirement is not met, and template
preparation is not possible, DM may face the choice of falling behind
for a series of visits, or of deferring the processing of the single
affected visit to avoid disrupting the pipeline. This choice would
depend on the frequency with which the interface requirement is not met,
and whether such failures occurred in bursts.\\
The inclusion of survey name in the visit qualities reported in an
advance notice will enable alternative, near real-time processing
pipelines to be triggered for Special Programs, as required by
OSS-REQ-0384 and DMS-REQ-0320.
\end{minipage}
\\ \cdashline{1-2}
Requirement Priority &  \\ \cdashline{1-2}
Upper Level Requirement &
\begin{tabular}{cl}
\end{tabular}
\\ \hline
\end{longtable}
}


  
 \newpage 
\subsection{[LVV-5537] OCS-DM-COM-ICD-0002-V-01: OCS SAL Middleware Delivery\_DM\_1 }\label{lvv-5537}

\begin{longtable}{cccc}
\hline
\textbf{Jira Link} & \textbf{Assignee} & \textbf{Status} & \textbf{Test Cases}\\ \hline
\href{https://jira.lsstcorp.org/browse/LVV-5537}{LVV-5537} &
Leanne Guy & Not Covered &
\begin{tabular}{c}
\end{tabular}
\\
\hline
\end{longtable}

\textbf{Verification Element Description:} \\
Undefined

{\footnotesize
\begin{longtable}{p{2.5cm}p{13.5cm}}
\hline
\multicolumn{2}{c}{\textbf{Requirement Details}}\\ \hline
Requirement ID & OCS-DM-COM-ICD-0002 \\ \cdashline{1-2}
Requirement Description &
\begin{minipage}[]{13cm}
\textbf{Specification:} The OCS shall deliver the Service Abstraction
Layer software in a form usable from the C++ and Python languages. The
version(s) of C++ and Python supported and the identities and versions
of additional external libraries required, if any, shall be under
Observatory-level change control. The OCS shall provide the SAL SDK to
generate interface header files and libraries.\\
\hspace*{0.333em}
\end{minipage}
\\ \cdashline{1-2}
Requirement Discussion &
\begin{minipage}[]{13cm}
~
\end{minipage}
\\ \cdashline{1-2}
Requirement Priority &  \\ \cdashline{1-2}
Upper Level Requirement &
\begin{tabular}{cl}
\end{tabular}
\\ \hline
\end{longtable}
}


  
 \newpage 
\subsection{[LVV-5538] OCS-DM-COM-ICD-0002-V-02: OCS SAL Middleware Delivery\_DM\_2 }\label{lvv-5538}

\begin{longtable}{cccc}
\hline
\textbf{Jira Link} & \textbf{Assignee} & \textbf{Status} & \textbf{Test Cases}\\ \hline
\href{https://jira.lsstcorp.org/browse/LVV-5538}{LVV-5538} &
Leanne Guy & Not Covered &
\begin{tabular}{c}
\end{tabular}
\\
\hline
\end{longtable}

\textbf{Verification Element Description:} \\
Undefined

{\footnotesize
\begin{longtable}{p{2.5cm}p{13.5cm}}
\hline
\multicolumn{2}{c}{\textbf{Requirement Details}}\\ \hline
Requirement ID & OCS-DM-COM-ICD-0002 \\ \cdashline{1-2}
Requirement Description &
\begin{minipage}[]{13cm}
\textbf{Specification:} The OCS shall deliver the Service Abstraction
Layer software in a form usable from the C++ and Python languages. The
version(s) of C++ and Python supported and the identities and versions
of additional external libraries required, if any, shall be under
Observatory-level change control. The OCS shall provide the SAL SDK to
generate interface header files and libraries.\\
\hspace*{0.333em}
\end{minipage}
\\ \cdashline{1-2}
Requirement Discussion &
\begin{minipage}[]{13cm}
~
\end{minipage}
\\ \cdashline{1-2}
Requirement Priority &  \\ \cdashline{1-2}
Upper Level Requirement &
\begin{tabular}{cl}
\end{tabular}
\\ \hline
\end{longtable}
}


  
 \newpage 
\subsection{[LVV-5543] OCS-DM-COM-ICD-0001-V-01: OCS Service Abstraction Layer\_DM\_1 }\label{lvv-5543}

\begin{longtable}{cccc}
\hline
\textbf{Jira Link} & \textbf{Assignee} & \textbf{Status} & \textbf{Test Cases}\\ \hline
\href{https://jira.lsstcorp.org/browse/LVV-5543}{LVV-5543} &
Leanne Guy & Not Covered &
\begin{tabular}{c}
\end{tabular}
\\
\hline
\end{longtable}

\textbf{Verification Element Description:} \\
Undefined

{\footnotesize
\begin{longtable}{p{2.5cm}p{13.5cm}}
\hline
\multicolumn{2}{c}{\textbf{Requirement Details}}\\ \hline
Requirement ID & OCS-DM-COM-ICD-0001 \\ \cdashline{1-2}
Requirement Description &
\begin{minipage}[]{13cm}
\textbf{Specification}: The OCS shall provide the Service Abstraction
Layer (SAL) middleware described in Interface Support Document \citeds{LSE-70},
supporting a commandable device abstraction as well as a
publish/subscribe communications protocol for events and telemetry.
\end{minipage}
\\ \cdashline{1-2}
Requirement Priority &  \\ \cdashline{1-2}
Upper Level Requirement &
\begin{tabular}{cl}
\end{tabular}
\\ \hline
\end{longtable}
}


  
 \newpage 
\subsection{[LVV-5544] OCS-DM-COM-ICD-0001-V-02: OCS Service Abstraction Layer\_DM\_2 }\label{lvv-5544}

\begin{longtable}{cccc}
\hline
\textbf{Jira Link} & \textbf{Assignee} & \textbf{Status} & \textbf{Test Cases}\\ \hline
\href{https://jira.lsstcorp.org/browse/LVV-5544}{LVV-5544} &
Leanne Guy & Not Covered &
\begin{tabular}{c}
\end{tabular}
\\
\hline
\end{longtable}

\textbf{Verification Element Description:} \\
Undefined

{\footnotesize
\begin{longtable}{p{2.5cm}p{13.5cm}}
\hline
\multicolumn{2}{c}{\textbf{Requirement Details}}\\ \hline
Requirement ID & OCS-DM-COM-ICD-0001 \\ \cdashline{1-2}
Requirement Description &
\begin{minipage}[]{13cm}
\textbf{Specification}: The OCS shall provide the Service Abstraction
Layer (SAL) middleware described in Interface Support Document \citeds{LSE-70},
supporting a commandable device abstraction as well as a
publish/subscribe communications protocol for events and telemetry.
\end{minipage}
\\ \cdashline{1-2}
Requirement Priority &  \\ \cdashline{1-2}
Upper Level Requirement &
\begin{tabular}{cl}
\end{tabular}
\\ \hline
\end{longtable}
}


  
 \newpage 
\subsection{[LVV-5628] DM-TS-CON-ICD-0003-V-01: Wavefront image archive access\_DM\_1 }\label{lvv-5628}

\begin{longtable}{cccc}
\hline
\textbf{Jira Link} & \textbf{Assignee} & \textbf{Status} & \textbf{Test Cases}\\ \hline
\href{https://jira.lsstcorp.org/browse/LVV-5628}{LVV-5628} &
Leanne Guy & Not Covered &
\begin{tabular}{c}
\end{tabular}
\\
\hline
\end{longtable}

\textbf{Verification Element Description:} \\
Undefined

{\footnotesize
\begin{longtable}{p{2.5cm}p{13.5cm}}
\hline
\multicolumn{2}{c}{\textbf{Requirement Details}}\\ \hline
Requirement ID & DM-TS-CON-ICD-0003 \\ \cdashline{1-2}
Requirement Description &
\begin{minipage}[]{13cm}
\textbf{Specification:} Data Management shall provide access for the
Telescope and Site subsystem, for both personnel and automated
processes, to the archive of Wavefront Sensor images. This shall be the
same interface that is provided for science image archive access within
the project. DM shall support prompt access to the archive from the
Summit, at a service level to be determined, but sufficient to support
any reasonable level of operator-directed access to individual images.
DM shall also support Telescope access to the archive at the Base and
Archive facilities, including support for automated bulk analysis. DM
may restrict bulk access to large quantities of wavefront archive data
at the Summit.
\end{minipage}
\\ \cdashline{1-2}
Requirement Discussion &
\begin{minipage}[]{13cm}
\textbf{Discussion:} This is required in order to support detailed
analysis of the wavefront data, beyond the immediate processing that is
enabled by the Telescope's subscription to a live feed of images.\\
The Telescope group is currently willing to accept that the usual
science image interface with be suitable. The details of this will be
reviewed during Final Design. It is understood that DM will provide
authorization and authentication services to support the access
required.\\
Bulk processing is currently planned to be done using the same toolkit
used for DM's own bulk production, and so DM will support the Telescope
group in its use of these tools.
\end{minipage}
\\ \cdashline{1-2}
Requirement Priority &  \\ \cdashline{1-2}
Upper Level Requirement &
\begin{tabular}{cl}
\end{tabular}
\\ \hline
\end{longtable}
}


  
 \newpage 
\subsection{[LVV-5629] DM-TS-CON-ICD-0003-V-02: Wavefront image archive access\_DM\_2 }\label{lvv-5629}

\begin{longtable}{cccc}
\hline
\textbf{Jira Link} & \textbf{Assignee} & \textbf{Status} & \textbf{Test Cases}\\ \hline
\href{https://jira.lsstcorp.org/browse/LVV-5629}{LVV-5629} &
Leanne Guy & Not Covered &
\begin{tabular}{c}
\end{tabular}
\\
\hline
\end{longtable}

\textbf{Verification Element Description:} \\
Undefined

{\footnotesize
\begin{longtable}{p{2.5cm}p{13.5cm}}
\hline
\multicolumn{2}{c}{\textbf{Requirement Details}}\\ \hline
Requirement ID & DM-TS-CON-ICD-0003 \\ \cdashline{1-2}
Requirement Description &
\begin{minipage}[]{13cm}
\textbf{Specification:} Data Management shall provide access for the
Telescope and Site subsystem, for both personnel and automated
processes, to the archive of Wavefront Sensor images. This shall be the
same interface that is provided for science image archive access within
the project. DM shall support prompt access to the archive from the
Summit, at a service level to be determined, but sufficient to support
any reasonable level of operator-directed access to individual images.
DM shall also support Telescope access to the archive at the Base and
Archive facilities, including support for automated bulk analysis. DM
may restrict bulk access to large quantities of wavefront archive data
at the Summit.
\end{minipage}
\\ \cdashline{1-2}
Requirement Discussion &
\begin{minipage}[]{13cm}
\textbf{Discussion:} This is required in order to support detailed
analysis of the wavefront data, beyond the immediate processing that is
enabled by the Telescope's subscription to a live feed of images.\\
The Telescope group is currently willing to accept that the usual
science image interface with be suitable. The details of this will be
reviewed during Final Design. It is understood that DM will provide
authorization and authentication services to support the access
required.\\
Bulk processing is currently planned to be done using the same toolkit
used for DM's own bulk production, and so DM will support the Telescope
group in its use of these tools.
\end{minipage}
\\ \cdashline{1-2}
Requirement Priority &  \\ \cdashline{1-2}
Upper Level Requirement &
\begin{tabular}{cl}
\end{tabular}
\\ \hline
\end{longtable}
}


  
 \newpage 
\subsection{[LVV-5634] DM-TS-CON-ICD-0010-V-01: Wavefront Processing Pipeline\_DM\_1 }\label{lvv-5634}

\begin{longtable}{cccc}
\hline
\textbf{Jira Link} & \textbf{Assignee} & \textbf{Status} & \textbf{Test Cases}\\ \hline
\href{https://jira.lsstcorp.org/browse/LVV-5634}{LVV-5634} &
Leanne Guy & Not Covered &
\begin{tabular}{c}
\end{tabular}
\\
\hline
\end{longtable}

\textbf{Verification Element Description:} \\
Undefined

{\footnotesize
\begin{longtable}{p{2.5cm}p{13.5cm}}
\hline
\multicolumn{2}{c}{\textbf{Requirement Details}}\\ \hline
Requirement ID & DM-TS-CON-ICD-0010 \\ \cdashline{1-2}
Requirement Description &
\begin{minipage}[]{13cm}
\textbf{Specification}: T\&S shall provide a full-focal-plane wavefront
processing pipeline payload, written using Data Management conventions
and frameworks, to DM for execution in the OCS-Driven Batch Service on
the Commissioning Cluster and/or at NCSA. The input to the pipeline
payload shall be the image names of the intra-focal and extra-focal
images for either ComCam or LSSTCam; its output shall be the Zernike
coefficients describing the wavefront solution for each detector, which
shall be transmitted as telemetry.
\end{minipage}
\\ \cdashline{1-2}
Requirement Discussion &
\begin{minipage}[]{13cm}
\textbf{Discussion}: The AOS will be written as LSST stack tasks and
developed in a way consistent with the DM Developer Guide
(\url{http://developer.lsst.io}) including style guide, code review,
development workflow. It will build under the same build-test-release
harness as the LSST stack. Additionally, T\&S is responsible for
providing any datasets and test scripts required to continuously
integrate and validate the code.\\
\hspace*{0.333em}\\
If the latency of OCS-Driven Batch is insufficient, the pipeline can be
ported to the Prompt Processing service after LSSTCam comes online.
\end{minipage}
\\ \cdashline{1-2}
Requirement Priority &  \\ \cdashline{1-2}
Upper Level Requirement &
\begin{tabular}{cl}
\end{tabular}
\\ \hline
\end{longtable}
}


  
 \newpage 
\subsection{[LVV-5635] DM-TS-CON-ICD-0010-V-02: Wavefront Processing Pipeline\_DM\_2 }\label{lvv-5635}

\begin{longtable}{cccc}
\hline
\textbf{Jira Link} & \textbf{Assignee} & \textbf{Status} & \textbf{Test Cases}\\ \hline
\href{https://jira.lsstcorp.org/browse/LVV-5635}{LVV-5635} &
Leanne Guy & Not Covered &
\begin{tabular}{c}
\end{tabular}
\\
\hline
\end{longtable}

\textbf{Verification Element Description:} \\
Undefined

{\footnotesize
\begin{longtable}{p{2.5cm}p{13.5cm}}
\hline
\multicolumn{2}{c}{\textbf{Requirement Details}}\\ \hline
Requirement ID & DM-TS-CON-ICD-0010 \\ \cdashline{1-2}
Requirement Description &
\begin{minipage}[]{13cm}
\textbf{Specification}: T\&S shall provide a full-focal-plane wavefront
processing pipeline payload, written using Data Management conventions
and frameworks, to DM for execution in the OCS-Driven Batch Service on
the Commissioning Cluster and/or at NCSA. The input to the pipeline
payload shall be the image names of the intra-focal and extra-focal
images for either ComCam or LSSTCam; its output shall be the Zernike
coefficients describing the wavefront solution for each detector, which
shall be transmitted as telemetry.
\end{minipage}
\\ \cdashline{1-2}
Requirement Discussion &
\begin{minipage}[]{13cm}
\textbf{Discussion}: The AOS will be written as LSST stack tasks and
developed in a way consistent with the DM Developer Guide
(\url{http://developer.lsst.io}) including style guide, code review,
development workflow. It will build under the same build-test-release
harness as the LSST stack. Additionally, T\&S is responsible for
providing any datasets and test scripts required to continuously
integrate and validate the code.\\
\hspace*{0.333em}\\
If the latency of OCS-Driven Batch is insufficient, the pipeline can be
ported to the Prompt Processing service after LSSTCam comes online.
\end{minipage}
\\ \cdashline{1-2}
Requirement Priority &  \\ \cdashline{1-2}
Upper Level Requirement &
\begin{tabular}{cl}
\end{tabular}
\\ \hline
\end{longtable}
}


  
 \newpage 
\subsection{[LVV-5652] DM-TS-CON-ICD-0006-V-01: Data\_DM\_1 }\label{lvv-5652}

\begin{longtable}{cccc}
\hline
\textbf{Jira Link} & \textbf{Assignee} & \textbf{Status} & \textbf{Test Cases}\\ \hline
\href{https://jira.lsstcorp.org/browse/LVV-5652}{LVV-5652} &
Leanne Guy & Not Covered &
\begin{tabular}{c}
\end{tabular}
\\
\hline
\end{longtable}

\textbf{Verification Element Description:} \\
Undefined

{\footnotesize
\begin{longtable}{p{2.5cm}p{13.5cm}}
\hline
\multicolumn{2}{c}{\textbf{Requirement Details}}\\ \hline
Requirement ID & DM-TS-CON-ICD-0006 \\ \cdashline{1-2}
Requirement Description &
\begin{minipage}[]{13cm}
\textbf{Specification}: Data Management shall publish as telemetry, for
each detector in each exposure, the following items derived from a PSF
model: full width at half maximum (double), Ixx/Iyy/Ixy quadrupole
representation of ellipse (three doubles).
\end{minipage}
\\ \cdashline{1-2}
Requirement Discussion &
\begin{minipage}[]{13cm}
\textbf{Discussion}: The PSF information will serve as a diagnostic tool
to conduct quality assurance on the active optics solution. For instance
PSF information will be used to determine if there are any focus error
trends.
\end{minipage}
\\ \cdashline{1-2}
Requirement Priority &  \\ \cdashline{1-2}
Upper Level Requirement &
\begin{tabular}{cl}
\end{tabular}
\\ \hline
\end{longtable}
}


  
 \newpage 
\subsection{[LVV-5653] DM-TS-CON-ICD-0006-V-02: Data\_DM\_2 }\label{lvv-5653}

\begin{longtable}{cccc}
\hline
\textbf{Jira Link} & \textbf{Assignee} & \textbf{Status} & \textbf{Test Cases}\\ \hline
\href{https://jira.lsstcorp.org/browse/LVV-5653}{LVV-5653} &
Leanne Guy & Not Covered &
\begin{tabular}{c}
\end{tabular}
\\
\hline
\end{longtable}

\textbf{Verification Element Description:} \\
Undefined

{\footnotesize
\begin{longtable}{p{2.5cm}p{13.5cm}}
\hline
\multicolumn{2}{c}{\textbf{Requirement Details}}\\ \hline
Requirement ID & DM-TS-CON-ICD-0006 \\ \cdashline{1-2}
Requirement Description &
\begin{minipage}[]{13cm}
\textbf{Specification}: Data Management shall publish as telemetry, for
each detector in each exposure, the following items derived from a PSF
model: full width at half maximum (double), Ixx/Iyy/Ixy quadrupole
representation of ellipse (three doubles).
\end{minipage}
\\ \cdashline{1-2}
Requirement Discussion &
\begin{minipage}[]{13cm}
\textbf{Discussion}: The PSF information will serve as a diagnostic tool
to conduct quality assurance on the active optics solution. For instance
PSF information will be used to determine if there are any focus error
trends.
\end{minipage}
\\ \cdashline{1-2}
Requirement Priority &  \\ \cdashline{1-2}
Upper Level Requirement &
\begin{tabular}{cl}
\end{tabular}
\\ \hline
\end{longtable}
}


  
 \newpage 
\subsection{[LVV-5658] DM-TS-CON-ICD-0007-V-01: Timing\_DM\_1 }\label{lvv-5658}

\begin{longtable}{cccc}
\hline
\textbf{Jira Link} & \textbf{Assignee} & \textbf{Status} & \textbf{Test Cases}\\ \hline
\href{https://jira.lsstcorp.org/browse/LVV-5658}{LVV-5658} &
Leanne Guy & Not Covered &
\begin{tabular}{c}
\end{tabular}
\\
\hline
\end{longtable}

\textbf{Verification Element Description:} \\
Undefined

{\footnotesize
\begin{longtable}{p{2.5cm}p{13.5cm}}
\hline
\multicolumn{2}{c}{\textbf{Requirement Details}}\\ \hline
Requirement ID & DM-TS-CON-ICD-0007 \\ \cdashline{1-2}
Requirement Description &
\begin{minipage}[]{13cm}
\textbf{Specification:} The PSF data shall be published as telemetry
within time \textbf{psfSolutionFeedbackTime} of the close of data
acquisition for each exposure.
\end{minipage}
\\ \cdashline{1-2}
Requirement Parameters & \textbf{psfSolutionFeedbackTime = 60{{[}second{]}}} Time following the
conclusion of a readout of an exposure which DM must provide the PSF
information for each detector. \\ \cdashline{1-2}
Requirement Discussion &
\begin{minipage}[]{13cm}
\textbf{Discussion:} The T\&S and commissioning teams express the need
to know about the PSF coordinates on a 60s to decrease the time overhead
of the wait during full array mode and ComCam AOS applications.
\end{minipage}
\\ \cdashline{1-2}
Requirement Priority &  \\ \cdashline{1-2}
Upper Level Requirement &
\begin{tabular}{cl}
\end{tabular}
\\ \hline
\end{longtable}
}


  
 \newpage 
\subsection{[LVV-5659] DM-TS-CON-ICD-0007-V-02: Timing\_DM\_2 }\label{lvv-5659}

\begin{longtable}{cccc}
\hline
\textbf{Jira Link} & \textbf{Assignee} & \textbf{Status} & \textbf{Test Cases}\\ \hline
\href{https://jira.lsstcorp.org/browse/LVV-5659}{LVV-5659} &
Leanne Guy & Not Covered &
\begin{tabular}{c}
\end{tabular}
\\
\hline
\end{longtable}

\textbf{Verification Element Description:} \\
Undefined

{\footnotesize
\begin{longtable}{p{2.5cm}p{13.5cm}}
\hline
\multicolumn{2}{c}{\textbf{Requirement Details}}\\ \hline
Requirement ID & DM-TS-CON-ICD-0007 \\ \cdashline{1-2}
Requirement Description &
\begin{minipage}[]{13cm}
\textbf{Specification:} The PSF data shall be published as telemetry
within time \textbf{psfSolutionFeedbackTime} of the close of data
acquisition for each exposure.
\end{minipage}
\\ \cdashline{1-2}
Requirement Parameters & \textbf{psfSolutionFeedbackTime = 60{{[}second{]}}} Time following the
conclusion of a readout of an exposure which DM must provide the PSF
information for each detector. \\ \cdashline{1-2}
Requirement Discussion &
\begin{minipage}[]{13cm}
\textbf{Discussion:} The T\&S and commissioning teams express the need
to know about the PSF coordinates on a 60s to decrease the time overhead
of the wait during full array mode and ComCam AOS applications.
\end{minipage}
\\ \cdashline{1-2}
Requirement Priority &  \\ \cdashline{1-2}
Upper Level Requirement &
\begin{tabular}{cl}
\end{tabular}
\\ \hline
\end{longtable}
}


  
 \newpage 
\subsection{[LVV-5664] DM-TS-CON-ICD-0009-V-01: Calibration Data Products\_DM\_1 }\label{lvv-5664}

\begin{longtable}{cccc}
\hline
\textbf{Jira Link} & \textbf{Assignee} & \textbf{Status} & \textbf{Test Cases}\\ \hline
\href{https://jira.lsstcorp.org/browse/LVV-5664}{LVV-5664} &
Leanne Guy & Not Covered &
\begin{tabular}{c}
\end{tabular}
\\
\hline
\end{longtable}

\textbf{Verification Element Description:} \\
Undefined

{\footnotesize
\begin{longtable}{p{2.5cm}p{13.5cm}}
\hline
\multicolumn{2}{c}{\textbf{Requirement Details}}\\ \hline
Requirement ID & DM-TS-CON-ICD-0009 \\ \cdashline{1-2}
Requirement Description &
\begin{minipage}[]{13cm}
\textbf{Specification}: Data Management shall provide master calibration
frames for the wavefront and guider sensors via the Observatory
Operations Data Service at the Base Facility. At a minimum, these images
shall be provided in a filesystem accessible via rsync.
\end{minipage}
\\ \cdashline{1-2}
Requirement Discussion &
\begin{minipage}[]{13cm}
\textbf{Discussion}: The AOS will need the Calibration Data Products
related to the WFS to run the ISR. In case of an outage between the base
and the summit, the AOS will use older Calibration Data Products up to
the (8) days specified in \citeds{LSE-161}. The CBP and the flat field screen
will allow to take calibration data of the WFS.
\end{minipage}
\\ \cdashline{1-2}
Requirement Priority &  \\ \cdashline{1-2}
Upper Level Requirement &
\begin{tabular}{cl}
\end{tabular}
\\ \hline
\end{longtable}
}


  
 \newpage 
\subsection{[LVV-5665] DM-TS-CON-ICD-0009-V-02: Calibration Data Products\_DM\_2 }\label{lvv-5665}

\begin{longtable}{cccc}
\hline
\textbf{Jira Link} & \textbf{Assignee} & \textbf{Status} & \textbf{Test Cases}\\ \hline
\href{https://jira.lsstcorp.org/browse/LVV-5665}{LVV-5665} &
Leanne Guy & Not Covered &
\begin{tabular}{c}
\end{tabular}
\\
\hline
\end{longtable}

\textbf{Verification Element Description:} \\
Undefined

{\footnotesize
\begin{longtable}{p{2.5cm}p{13.5cm}}
\hline
\multicolumn{2}{c}{\textbf{Requirement Details}}\\ \hline
Requirement ID & DM-TS-CON-ICD-0009 \\ \cdashline{1-2}
Requirement Description &
\begin{minipage}[]{13cm}
\textbf{Specification}: Data Management shall provide master calibration
frames for the wavefront and guider sensors via the Observatory
Operations Data Service at the Base Facility. At a minimum, these images
shall be provided in a filesystem accessible via rsync.
\end{minipage}
\\ \cdashline{1-2}
Requirement Discussion &
\begin{minipage}[]{13cm}
\textbf{Discussion}: The AOS will need the Calibration Data Products
related to the WFS to run the ISR. In case of an outage between the base
and the summit, the AOS will use older Calibration Data Products up to
the (8) days specified in \citeds{LSE-161}. The CBP and the flat field screen
will allow to take calibration data of the WFS.
\end{minipage}
\\ \cdashline{1-2}
Requirement Priority &  \\ \cdashline{1-2}
Upper Level Requirement &
\begin{tabular}{cl}
\end{tabular}
\\ \hline
\end{longtable}
}


  
 \newpage 
\subsection{[LVV-5670] DM-TS-CON-ICD-0008-V-01: LSST Stack Availability\_DM\_1 }\label{lvv-5670}

\begin{longtable}{cccc}
\hline
\textbf{Jira Link} & \textbf{Assignee} & \textbf{Status} & \textbf{Test Cases}\\ \hline
\href{https://jira.lsstcorp.org/browse/LVV-5670}{LVV-5670} &
Leanne Guy & Not Covered &
\begin{tabular}{c}
\end{tabular}
\\
\hline
\end{longtable}

\textbf{Verification Element Description:} \\
Undefined

{\footnotesize
\begin{longtable}{p{2.5cm}p{13.5cm}}
\hline
\multicolumn{2}{c}{\textbf{Requirement Details}}\\ \hline
Requirement ID & DM-TS-CON-ICD-0008 \\ \cdashline{1-2}
Requirement Description &
\begin{minipage}[]{13cm}
\textbf{Specification}: A container with a T\&S-selected release of the
LSST stack shall be available at the summit.
\end{minipage}
\\ \cdashline{1-2}
Requirement Discussion &
\begin{minipage}[]{13cm}
\textbf{Discussion}: One of the first steps of the Active Optics System
control feedback is to pre-process the wavefront sensor images using
routine developed for the Image Signature Removal. Therefore these
routines shall be physically on the summit for timing purposes.
\end{minipage}
\\ \cdashline{1-2}
Requirement Priority &  \\ \cdashline{1-2}
Upper Level Requirement &
\begin{tabular}{cl}
\end{tabular}
\\ \hline
\end{longtable}
}


  
 \newpage 
\subsection{[LVV-5671] DM-TS-CON-ICD-0008-V-02: LSST Stack Availability\_DM\_2 }\label{lvv-5671}

\begin{longtable}{cccc}
\hline
\textbf{Jira Link} & \textbf{Assignee} & \textbf{Status} & \textbf{Test Cases}\\ \hline
\href{https://jira.lsstcorp.org/browse/LVV-5671}{LVV-5671} &
Leanne Guy & Not Covered &
\begin{tabular}{c}
\end{tabular}
\\
\hline
\end{longtable}

\textbf{Verification Element Description:} \\
Undefined

{\footnotesize
\begin{longtable}{p{2.5cm}p{13.5cm}}
\hline
\multicolumn{2}{c}{\textbf{Requirement Details}}\\ \hline
Requirement ID & DM-TS-CON-ICD-0008 \\ \cdashline{1-2}
Requirement Description &
\begin{minipage}[]{13cm}
\textbf{Specification}: A container with a T\&S-selected release of the
LSST stack shall be available at the summit.
\end{minipage}
\\ \cdashline{1-2}
Requirement Discussion &
\begin{minipage}[]{13cm}
\textbf{Discussion}: One of the first steps of the Active Optics System
control feedback is to pre-process the wavefront sensor images using
routine developed for the Image Signature Removal. Therefore these
routines shall be physically on the summit for timing purposes.
\end{minipage}
\\ \cdashline{1-2}
Requirement Priority &  \\ \cdashline{1-2}
Upper Level Requirement &
\begin{tabular}{cl}
\end{tabular}
\\ \hline
\end{longtable}
}


  
 \newpage 
\subsection{[LVV-5676] DM-TS-CON-ICD-0004-V-01: Use OCS for data transport\_DM\_1 }\label{lvv-5676}

\begin{longtable}{cccc}
\hline
\textbf{Jira Link} & \textbf{Assignee} & \textbf{Status} & \textbf{Test Cases}\\ \hline
\href{https://jira.lsstcorp.org/browse/LVV-5676}{LVV-5676} &
Leanne Guy & Not Covered &
\begin{tabular}{c}
\end{tabular}
\\
\hline
\end{longtable}

\textbf{Verification Element Description:} \\
Undefined

{\footnotesize
\begin{longtable}{p{2.5cm}p{13.5cm}}
\hline
\multicolumn{2}{c}{\textbf{Requirement Details}}\\ \hline
Requirement ID & DM-TS-CON-ICD-0004 \\ \cdashline{1-2}
Requirement Description &
\begin{minipage}[]{13cm}
\textbf{Specification:} All telemetry exchange required by this ICD
shall be mediated by the OCS, following the specification in ICD \citeds{LSE-70}.
\end{minipage}
\\ \cdashline{1-2}
Requirement Discussion &
\begin{minipage}[]{13cm}
\textbf{Discussion:} The intent is that the publish/subscribe mechanism,
the EFD, and the Configuration Database will be the means of
communication.
\end{minipage}
\\ \cdashline{1-2}
Requirement Priority &  \\ \cdashline{1-2}
Upper Level Requirement &
\begin{tabular}{cl}
\end{tabular}
\\ \hline
\end{longtable}
}


  
 \newpage 
\subsection{[LVV-5677] DM-TS-CON-ICD-0004-V-02: Use OCS for data transport\_DM\_2 }\label{lvv-5677}

\begin{longtable}{cccc}
\hline
\textbf{Jira Link} & \textbf{Assignee} & \textbf{Status} & \textbf{Test Cases}\\ \hline
\href{https://jira.lsstcorp.org/browse/LVV-5677}{LVV-5677} &
Leanne Guy & Not Covered &
\begin{tabular}{c}
\end{tabular}
\\
\hline
\end{longtable}

\textbf{Verification Element Description:} \\
Undefined

{\footnotesize
\begin{longtable}{p{2.5cm}p{13.5cm}}
\hline
\multicolumn{2}{c}{\textbf{Requirement Details}}\\ \hline
Requirement ID & DM-TS-CON-ICD-0004 \\ \cdashline{1-2}
Requirement Description &
\begin{minipage}[]{13cm}
\textbf{Specification:} All telemetry exchange required by this ICD
shall be mediated by the OCS, following the specification in ICD \citeds{LSE-70}.
\end{minipage}
\\ \cdashline{1-2}
Requirement Discussion &
\begin{minipage}[]{13cm}
\textbf{Discussion:} The intent is that the publish/subscribe mechanism,
the EFD, and the Configuration Database will be the means of
communication.
\end{minipage}
\\ \cdashline{1-2}
Requirement Priority &  \\ \cdashline{1-2}
Upper Level Requirement &
\begin{tabular}{cl}
\end{tabular}
\\ \hline
\end{longtable}
}


  
 \newpage 
\subsection{[LVV-6140] CA-DM-SUP-ICD-0026-V-03: Analog Electronics Temperature
Measurements\_DM\_3 }\label{lvv-6140}

\begin{longtable}{cccc}
\hline
\textbf{Jira Link} & \textbf{Assignee} & \textbf{Status} & \textbf{Test Cases}\\ \hline
\href{https://jira.lsstcorp.org/browse/LVV-6140}{LVV-6140} &
Leanne Guy & Not Covered &
\begin{tabular}{c}
\end{tabular}
\\
\hline
\end{longtable}

\textbf{Verification Element Description:} \\
Undefined

{\footnotesize
\begin{longtable}{p{2.5cm}p{13.5cm}}
\hline
\multicolumn{2}{c}{\textbf{Requirement Details}}\\ \hline
Requirement ID & CA-DM-SUP-ICD-0026 \\ \cdashline{1-2}
Requirement Description &
\begin{minipage}[]{13cm}
\textbf{Specification:} The Camera shall provide measurements of the
temperatures of all analog electronics in the data acquisition chain.
\end{minipage}
\\ \cdashline{1-2}
Requirement Discussion &
\begin{minipage}[]{13cm}
\textbf{Discussion:} The details are to be determined in Phase 3. It is
TBD whether DM is interested in the temperatures of other parts of the
electronics.
\end{minipage}
\\ \cdashline{1-2}
Requirement Priority &  \\ \cdashline{1-2}
Upper Level Requirement &
\begin{tabular}{cl}
\end{tabular}
\\ \hline
\end{longtable}
}


  
 \newpage 
\subsection{[LVV-6141] CA-DM-SUP-ICD-0026-V-04: Analog Electronics Temperature
Measurements\_DM\_4 }\label{lvv-6141}

\begin{longtable}{cccc}
\hline
\textbf{Jira Link} & \textbf{Assignee} & \textbf{Status} & \textbf{Test Cases}\\ \hline
\href{https://jira.lsstcorp.org/browse/LVV-6141}{LVV-6141} &
Leanne Guy & Not Covered &
\begin{tabular}{c}
\end{tabular}
\\
\hline
\end{longtable}

\textbf{Verification Element Description:} \\
Undefined

{\footnotesize
\begin{longtable}{p{2.5cm}p{13.5cm}}
\hline
\multicolumn{2}{c}{\textbf{Requirement Details}}\\ \hline
Requirement ID & CA-DM-SUP-ICD-0026 \\ \cdashline{1-2}
Requirement Description &
\begin{minipage}[]{13cm}
\textbf{Specification:} The Camera shall provide measurements of the
temperatures of all analog electronics in the data acquisition chain.
\end{minipage}
\\ \cdashline{1-2}
Requirement Discussion &
\begin{minipage}[]{13cm}
\textbf{Discussion:} The details are to be determined in Phase 3. It is
TBD whether DM is interested in the temperatures of other parts of the
electronics.
\end{minipage}
\\ \cdashline{1-2}
Requirement Priority &  \\ \cdashline{1-2}
Upper Level Requirement &
\begin{tabular}{cl}
\end{tabular}
\\ \hline
\end{longtable}
}


  
 \newpage 
\subsection{[LVV-6146] CA-DM-SUP-ICD-0027-V-03: Bias Voltage Measurements\_DM\_3 }\label{lvv-6146}

\begin{longtable}{cccc}
\hline
\textbf{Jira Link} & \textbf{Assignee} & \textbf{Status} & \textbf{Test Cases}\\ \hline
\href{https://jira.lsstcorp.org/browse/LVV-6146}{LVV-6146} &
Leanne Guy & Not Covered &
\begin{tabular}{c}
\end{tabular}
\\
\hline
\end{longtable}

\textbf{Verification Element Description:} \\
Undefined

{\footnotesize
\begin{longtable}{p{2.5cm}p{13.5cm}}
\hline
\multicolumn{2}{c}{\textbf{Requirement Details}}\\ \hline
Requirement ID & CA-DM-SUP-ICD-0027 \\ \cdashline{1-2}
Requirement Description &
\begin{minipage}[]{13cm}
\textbf{Specification:} The Camera shall provide measurements of the
actual bias voltages applied to the sensors.
\end{minipage}
\\ \cdashline{1-2}
Requirement Discussion &
\begin{minipage}[]{13cm}
\textbf{Discussion:} It is assumed that the setpoints will be available
as Configuration data.
\end{minipage}
\\ \cdashline{1-2}
Requirement Priority &  \\ \cdashline{1-2}
Upper Level Requirement &
\begin{tabular}{cl}
\end{tabular}
\\ \hline
\end{longtable}
}


  
 \newpage 
\subsection{[LVV-6147] CA-DM-SUP-ICD-0027-V-04: Bias Voltage Measurements\_DM\_4 }\label{lvv-6147}

\begin{longtable}{cccc}
\hline
\textbf{Jira Link} & \textbf{Assignee} & \textbf{Status} & \textbf{Test Cases}\\ \hline
\href{https://jira.lsstcorp.org/browse/LVV-6147}{LVV-6147} &
Leanne Guy & Not Covered &
\begin{tabular}{c}
\end{tabular}
\\
\hline
\end{longtable}

\textbf{Verification Element Description:} \\
Undefined

{\footnotesize
\begin{longtable}{p{2.5cm}p{13.5cm}}
\hline
\multicolumn{2}{c}{\textbf{Requirement Details}}\\ \hline
Requirement ID & CA-DM-SUP-ICD-0027 \\ \cdashline{1-2}
Requirement Description &
\begin{minipage}[]{13cm}
\textbf{Specification:} The Camera shall provide measurements of the
actual bias voltages applied to the sensors.
\end{minipage}
\\ \cdashline{1-2}
Requirement Discussion &
\begin{minipage}[]{13cm}
\textbf{Discussion:} It is assumed that the setpoints will be available
as Configuration data.
\end{minipage}
\\ \cdashline{1-2}
Requirement Priority &  \\ \cdashline{1-2}
Upper Level Requirement &
\begin{tabular}{cl}
\end{tabular}
\\ \hline
\end{longtable}
}


  
 \newpage 
\subsection{[LVV-6152] CA-DM-SUP-ICD-0024-V-03: Filter Changer Readback Information
Timeliness\_DM\_3 }\label{lvv-6152}

\begin{longtable}{cccc}
\hline
\textbf{Jira Link} & \textbf{Assignee} & \textbf{Status} & \textbf{Test Cases}\\ \hline
\href{https://jira.lsstcorp.org/browse/LVV-6152}{LVV-6152} &
Leanne Guy & Not Covered &
\begin{tabular}{c}
\end{tabular}
\\
\hline
\end{longtable}

\textbf{Verification Element Description:} \\
Undefined

{\footnotesize
\begin{longtable}{p{2.5cm}p{13.5cm}}
\hline
\multicolumn{2}{c}{\textbf{Requirement Details}}\\ \hline
Requirement ID & CA-DM-SUP-ICD-0024 \\ \cdashline{1-2}
Requirement Description &
\begin{minipage}[]{13cm}
\textbf{Specification:} The Camera shall provide the filter identity
information in time for use in Data Management's Alert Production; that
is, with a latency conforming to CA-DM-CON-ICD-0003 in \citeds{LSE-69}.
\end{minipage}
\\ \cdashline{1-2}
Requirement Priority &  \\ \cdashline{1-2}
Upper Level Requirement &
\begin{tabular}{cl}
CA-DM-CON-ICD-0003 & Camera Conditions data latency for Alert Production \\
\end{tabular}
\\ \hline
\end{longtable}
}


  
 \newpage 
\subsection{[LVV-6153] CA-DM-SUP-ICD-0024-V-04: Filter Changer Readback Information
Timeliness\_DM\_4 }\label{lvv-6153}

\begin{longtable}{cccc}
\hline
\textbf{Jira Link} & \textbf{Assignee} & \textbf{Status} & \textbf{Test Cases}\\ \hline
\href{https://jira.lsstcorp.org/browse/LVV-6153}{LVV-6153} &
Leanne Guy & Not Covered &
\begin{tabular}{c}
\end{tabular}
\\
\hline
\end{longtable}

\textbf{Verification Element Description:} \\
Undefined

{\footnotesize
\begin{longtable}{p{2.5cm}p{13.5cm}}
\hline
\multicolumn{2}{c}{\textbf{Requirement Details}}\\ \hline
Requirement ID & CA-DM-SUP-ICD-0024 \\ \cdashline{1-2}
Requirement Description &
\begin{minipage}[]{13cm}
\textbf{Specification:} The Camera shall provide the filter identity
information in time for use in Data Management's Alert Production; that
is, with a latency conforming to CA-DM-CON-ICD-0003 in \citeds{LSE-69}.
\end{minipage}
\\ \cdashline{1-2}
Requirement Priority &  \\ \cdashline{1-2}
Upper Level Requirement &
\begin{tabular}{cl}
CA-DM-CON-ICD-0003 & Camera Conditions data latency for Alert Production \\
\end{tabular}
\\ \hline
\end{longtable}
}


  
 \newpage 
\subsection{[LVV-6158] CA-DM-SUP-ICD-0023-V-03: Filter Changer Readback Information\_DM\_3 }\label{lvv-6158}

\begin{longtable}{cccc}
\hline
\textbf{Jira Link} & \textbf{Assignee} & \textbf{Status} & \textbf{Test Cases}\\ \hline
\href{https://jira.lsstcorp.org/browse/LVV-6158}{LVV-6158} &
Leanne Guy & Not Covered &
\begin{tabular}{c}
\end{tabular}
\\
\hline
\end{longtable}

\textbf{Verification Element Description:} \\
Undefined

{\footnotesize
\begin{longtable}{p{2.5cm}p{13.5cm}}
\hline
\multicolumn{2}{c}{\textbf{Requirement Details}}\\ \hline
Requirement ID & CA-DM-SUP-ICD-0023 \\ \cdashline{1-2}
Requirement Description &
\begin{minipage}[]{13cm}
\textbf{Specification:} The Camera shall provide readback information
from the filter changer, including at a minimum a positive
identification of the specific filter article that is in place.
\end{minipage}
\\ \cdashline{1-2}
Requirement Discussion &
\begin{minipage}[]{13cm}
\textbf{Discussion:} It is understood that the micro-positioning of the
filter (within the range of motion permitted by the clamping fixtures)
will not be measured. The information on the filter identity should be
something of the nature of a serial number and not just a filter
bandpass specifier. This is required to allow the disambiguation of two
filters with the same bandpass, should spare or replacement filters be
used in the system.
\end{minipage}
\\ \cdashline{1-2}
Requirement Priority &  \\ \cdashline{1-2}
Upper Level Requirement &
\begin{tabular}{cl}
\end{tabular}
\\ \hline
\end{longtable}
}


  
 \newpage 
\subsection{[LVV-6159] CA-DM-SUP-ICD-0023-V-04: Filter Changer Readback Information\_DM\_4 }\label{lvv-6159}

\begin{longtable}{cccc}
\hline
\textbf{Jira Link} & \textbf{Assignee} & \textbf{Status} & \textbf{Test Cases}\\ \hline
\href{https://jira.lsstcorp.org/browse/LVV-6159}{LVV-6159} &
Leanne Guy & Not Covered &
\begin{tabular}{c}
\end{tabular}
\\
\hline
\end{longtable}

\textbf{Verification Element Description:} \\
Undefined

{\footnotesize
\begin{longtable}{p{2.5cm}p{13.5cm}}
\hline
\multicolumn{2}{c}{\textbf{Requirement Details}}\\ \hline
Requirement ID & CA-DM-SUP-ICD-0023 \\ \cdashline{1-2}
Requirement Description &
\begin{minipage}[]{13cm}
\textbf{Specification:} The Camera shall provide readback information
from the filter changer, including at a minimum a positive
identification of the specific filter article that is in place.
\end{minipage}
\\ \cdashline{1-2}
Requirement Discussion &
\begin{minipage}[]{13cm}
\textbf{Discussion:} It is understood that the micro-positioning of the
filter (within the range of motion permitted by the clamping fixtures)
will not be measured. The information on the filter identity should be
something of the nature of a serial number and not just a filter
bandpass specifier. This is required to allow the disambiguation of two
filters with the same bandpass, should spare or replacement filters be
used in the system.
\end{minipage}
\\ \cdashline{1-2}
Requirement Priority &  \\ \cdashline{1-2}
Upper Level Requirement &
\begin{tabular}{cl}
\end{tabular}
\\ \hline
\end{longtable}
}


  
 \newpage 
\subsection{[LVV-6164] CA-DM-SUP-ICD-0025-V-03: Focal Plane Temperature Measurements\_DM\_3 }\label{lvv-6164}

\begin{longtable}{cccc}
\hline
\textbf{Jira Link} & \textbf{Assignee} & \textbf{Status} & \textbf{Test Cases}\\ \hline
\href{https://jira.lsstcorp.org/browse/LVV-6164}{LVV-6164} &
Leanne Guy & Not Covered &
\begin{tabular}{c}
\end{tabular}
\\
\hline
\end{longtable}

\textbf{Verification Element Description:} \\
Undefined

{\footnotesize
\begin{longtable}{p{2.5cm}p{13.5cm}}
\hline
\multicolumn{2}{c}{\textbf{Requirement Details}}\\ \hline
Requirement ID & CA-DM-SUP-ICD-0025 \\ \cdashline{1-2}
Requirement Description &
\begin{minipage}[]{13cm}
\textbf{Specification:} The Camera shall provide focal plane temperature
measurements.
\end{minipage}
\\ \cdashline{1-2}
Requirement Discussion &
\begin{minipage}[]{13cm}
\textbf{Discussion:} The frequency of sampling will be determined by the
Camera and will be specified in Phase 3, but should provide several
measurements during each nominal exposure.
\end{minipage}
\\ \cdashline{1-2}
Requirement Priority &  \\ \cdashline{1-2}
Upper Level Requirement &
\begin{tabular}{cl}
\end{tabular}
\\ \hline
\end{longtable}
}


  
 \newpage 
\subsection{[LVV-6165] CA-DM-SUP-ICD-0025-V-04: Focal Plane Temperature Measurements\_DM\_4 }\label{lvv-6165}

\begin{longtable}{cccc}
\hline
\textbf{Jira Link} & \textbf{Assignee} & \textbf{Status} & \textbf{Test Cases}\\ \hline
\href{https://jira.lsstcorp.org/browse/LVV-6165}{LVV-6165} &
Leanne Guy & Not Covered &
\begin{tabular}{c}
\end{tabular}
\\
\hline
\end{longtable}

\textbf{Verification Element Description:} \\
Undefined

{\footnotesize
\begin{longtable}{p{2.5cm}p{13.5cm}}
\hline
\multicolumn{2}{c}{\textbf{Requirement Details}}\\ \hline
Requirement ID & CA-DM-SUP-ICD-0025 \\ \cdashline{1-2}
Requirement Description &
\begin{minipage}[]{13cm}
\textbf{Specification:} The Camera shall provide focal plane temperature
measurements.
\end{minipage}
\\ \cdashline{1-2}
Requirement Discussion &
\begin{minipage}[]{13cm}
\textbf{Discussion:} The frequency of sampling will be determined by the
Camera and will be specified in Phase 3, but should provide several
measurements during each nominal exposure.
\end{minipage}
\\ \cdashline{1-2}
Requirement Priority &  \\ \cdashline{1-2}
Upper Level Requirement &
\begin{tabular}{cl}
\end{tabular}
\\ \hline
\end{longtable}
}


  
 \newpage 
\subsection{[LVV-6170] CA-DM-SUP-ICD-0022-V-03: Shutter Motion Profiles Timeliness\_DM\_3 }\label{lvv-6170}

\begin{longtable}{cccc}
\hline
\textbf{Jira Link} & \textbf{Assignee} & \textbf{Status} & \textbf{Test Cases}\\ \hline
\href{https://jira.lsstcorp.org/browse/LVV-6170}{LVV-6170} &
Leanne Guy & Not Covered &
\begin{tabular}{c}
\end{tabular}
\\
\hline
\end{longtable}

\textbf{Verification Element Description:} \\
Undefined

{\footnotesize
\begin{longtable}{p{2.5cm}p{13.5cm}}
\hline
\multicolumn{2}{c}{\textbf{Requirement Details}}\\ \hline
Requirement ID & CA-DM-SUP-ICD-0022 \\ \cdashline{1-2}
Requirement Description &
\begin{minipage}[]{13cm}
\textbf{Specification:} The Camera shall provide the shutter motion
profiles in time for use in Data Management's Alert Production; that is,
with a latency conforming to CA-DM-CON-ICD-0003 in \citeds{LSE-69}.
\end{minipage}
\\ \cdashline{1-2}
Requirement Priority &  \\ \cdashline{1-2}
Upper Level Requirement &
\begin{tabular}{cl}
CA-DM-CON-ICD-0003 & Camera Conditions data latency for Alert Production \\
\end{tabular}
\\ \hline
\end{longtable}
}


  
 \newpage 
\subsection{[LVV-6171] CA-DM-SUP-ICD-0022-V-04: Shutter Motion Profiles Timeliness\_DM\_4 }\label{lvv-6171}

\begin{longtable}{cccc}
\hline
\textbf{Jira Link} & \textbf{Assignee} & \textbf{Status} & \textbf{Test Cases}\\ \hline
\href{https://jira.lsstcorp.org/browse/LVV-6171}{LVV-6171} &
Leanne Guy & Not Covered &
\begin{tabular}{c}
\end{tabular}
\\
\hline
\end{longtable}

\textbf{Verification Element Description:} \\
Undefined

{\footnotesize
\begin{longtable}{p{2.5cm}p{13.5cm}}
\hline
\multicolumn{2}{c}{\textbf{Requirement Details}}\\ \hline
Requirement ID & CA-DM-SUP-ICD-0022 \\ \cdashline{1-2}
Requirement Description &
\begin{minipage}[]{13cm}
\textbf{Specification:} The Camera shall provide the shutter motion
profiles in time for use in Data Management's Alert Production; that is,
with a latency conforming to CA-DM-CON-ICD-0003 in \citeds{LSE-69}.
\end{minipage}
\\ \cdashline{1-2}
Requirement Priority &  \\ \cdashline{1-2}
Upper Level Requirement &
\begin{tabular}{cl}
CA-DM-CON-ICD-0003 & Camera Conditions data latency for Alert Production \\
\end{tabular}
\\ \hline
\end{longtable}
}


  
 \newpage 
\subsection{[LVV-6176] CA-DM-SUP-ICD-0021-V-03: Shutter Motion Profiles\_DM\_3 }\label{lvv-6176}

\begin{longtable}{cccc}
\hline
\textbf{Jira Link} & \textbf{Assignee} & \textbf{Status} & \textbf{Test Cases}\\ \hline
\href{https://jira.lsstcorp.org/browse/LVV-6176}{LVV-6176} &
Leanne Guy & Not Covered &
\begin{tabular}{c}
\end{tabular}
\\
\hline
\end{longtable}

\textbf{Verification Element Description:} \\
Undefined

{\footnotesize
\begin{longtable}{p{2.5cm}p{13.5cm}}
\hline
\multicolumn{2}{c}{\textbf{Requirement Details}}\\ \hline
Requirement ID & CA-DM-SUP-ICD-0021 \\ \cdashline{1-2}
Requirement Description &
\begin{minipage}[]{13cm}
\textbf{Specification:} The Camera shall provide the shutter motion
profiles (the position of the shutter blade edge as a function of
absolute time, following the observatory time standard OSS-REQ-0086 et
seq.) for each exposure, including the identification of which blades
were used and in which direction they moved.
\end{minipage}
\\ \cdashline{1-2}
Requirement Discussion &
\begin{minipage}[]{13cm}
\textbf{Discussion:} Note that a shadowing model for the shutter is
required to be provided under CA-DM-SUP-ICD-0017, allowing the
reconstruction of the shadow profile for the shutter as a function of
time given the positions recorded in the motion profile.
\end{minipage}
\\ \cdashline{1-2}
Requirement Priority &  \\ \cdashline{1-2}
Upper Level Requirement &
\begin{tabular}{cl}
\end{tabular}
\\ \hline
\end{longtable}
}


  
 \newpage 
\subsection{[LVV-6177] CA-DM-SUP-ICD-0021-V-04: Shutter Motion Profiles\_DM\_4 }\label{lvv-6177}

\begin{longtable}{cccc}
\hline
\textbf{Jira Link} & \textbf{Assignee} & \textbf{Status} & \textbf{Test Cases}\\ \hline
\href{https://jira.lsstcorp.org/browse/LVV-6177}{LVV-6177} &
Leanne Guy & Not Covered &
\begin{tabular}{c}
\end{tabular}
\\
\hline
\end{longtable}

\textbf{Verification Element Description:} \\
Undefined

{\footnotesize
\begin{longtable}{p{2.5cm}p{13.5cm}}
\hline
\multicolumn{2}{c}{\textbf{Requirement Details}}\\ \hline
Requirement ID & CA-DM-SUP-ICD-0021 \\ \cdashline{1-2}
Requirement Description &
\begin{minipage}[]{13cm}
\textbf{Specification:} The Camera shall provide the shutter motion
profiles (the position of the shutter blade edge as a function of
absolute time, following the observatory time standard OSS-REQ-0086 et
seq.) for each exposure, including the identification of which blades
were used and in which direction they moved.
\end{minipage}
\\ \cdashline{1-2}
Requirement Discussion &
\begin{minipage}[]{13cm}
\textbf{Discussion:} Note that a shadowing model for the shutter is
required to be provided under CA-DM-SUP-ICD-0017, allowing the
reconstruction of the shadow profile for the shutter as a function of
time given the positions recorded in the motion profile.
\end{minipage}
\\ \cdashline{1-2}
Requirement Priority &  \\ \cdashline{1-2}
Upper Level Requirement &
\begin{tabular}{cl}
\end{tabular}
\\ \hline
\end{longtable}
}


  
 \newpage 
\subsection{[LVV-6182] CA-DM-SUP-ICD-0028-V-03: Telemetry for Parametric Models\_DM\_3 }\label{lvv-6182}

\begin{longtable}{cccc}
\hline
\textbf{Jira Link} & \textbf{Assignee} & \textbf{Status} & \textbf{Test Cases}\\ \hline
\href{https://jira.lsstcorp.org/browse/LVV-6182}{LVV-6182} &
Leanne Guy & Not Covered &
\begin{tabular}{c}
\end{tabular}
\\
\hline
\end{longtable}

\textbf{Verification Element Description:} \\
Undefined

{\footnotesize
\begin{longtable}{p{2.5cm}p{13.5cm}}
\hline
\multicolumn{2}{c}{\textbf{Requirement Details}}\\ \hline
Requirement ID & CA-DM-SUP-ICD-0028 \\ \cdashline{1-2}
Requirement Description &
\begin{minipage}[]{13cm}
\textbf{Specification:} The Camera shall provide or identify all
telemetry required to support parametric models of the temporal
variation of Camera characteristics otherwise provided under the
``Design, Assembly, and Laboratory Test Data'' section above.
\end{minipage}
\\ \cdashline{1-2}
Requirement Discussion &
\begin{minipage}[]{13cm}
\textbf{Discussion:} This recognizes that if a Camera parameter is found
to have a temperature dependence, for instance, the Camera is permitted
to provide the value of the parameter either as a true measurement or as
a model parameterized by a temperature measurement or measurements.
\end{minipage}
\\ \cdashline{1-2}
Requirement Priority &  \\ \cdashline{1-2}
Upper Level Requirement &
\begin{tabular}{cl}
\end{tabular}
\\ \hline
\end{longtable}
}


  
 \newpage 
\subsection{[LVV-6183] CA-DM-SUP-ICD-0028-V-04: Telemetry for Parametric Models\_DM\_4 }\label{lvv-6183}

\begin{longtable}{cccc}
\hline
\textbf{Jira Link} & \textbf{Assignee} & \textbf{Status} & \textbf{Test Cases}\\ \hline
\href{https://jira.lsstcorp.org/browse/LVV-6183}{LVV-6183} &
Leanne Guy & Not Covered &
\begin{tabular}{c}
\end{tabular}
\\
\hline
\end{longtable}

\textbf{Verification Element Description:} \\
Undefined

{\footnotesize
\begin{longtable}{p{2.5cm}p{13.5cm}}
\hline
\multicolumn{2}{c}{\textbf{Requirement Details}}\\ \hline
Requirement ID & CA-DM-SUP-ICD-0028 \\ \cdashline{1-2}
Requirement Description &
\begin{minipage}[]{13cm}
\textbf{Specification:} The Camera shall provide or identify all
telemetry required to support parametric models of the temporal
variation of Camera characteristics otherwise provided under the
``Design, Assembly, and Laboratory Test Data'' section above.
\end{minipage}
\\ \cdashline{1-2}
Requirement Discussion &
\begin{minipage}[]{13cm}
\textbf{Discussion:} This recognizes that if a Camera parameter is found
to have a temperature dependence, for instance, the Camera is permitted
to provide the value of the parameter either as a true measurement or as
a model parameterized by a temperature measurement or measurements.
\end{minipage}
\\ \cdashline{1-2}
Requirement Priority &  \\ \cdashline{1-2}
Upper Level Requirement &
\begin{tabular}{cl}
\end{tabular}
\\ \hline
\end{longtable}
}


  
 \newpage 
\subsection{[LVV-6188] CA-DM-SUP-ICD-0029-V-03: Association with Camera Images\_DM\_3 }\label{lvv-6188}

\begin{longtable}{cccc}
\hline
\textbf{Jira Link} & \textbf{Assignee} & \textbf{Status} & \textbf{Test Cases}\\ \hline
\href{https://jira.lsstcorp.org/browse/LVV-6188}{LVV-6188} &
Leanne Guy & Not Covered &
\begin{tabular}{c}
\end{tabular}
\\
\hline
\end{longtable}

\textbf{Verification Element Description:} \\
Undefined

{\footnotesize
\begin{longtable}{p{2.5cm}p{13.5cm}}
\hline
\multicolumn{2}{c}{\textbf{Requirement Details}}\\ \hline
Requirement ID & CA-DM-SUP-ICD-0029 \\ \cdashline{1-2}
Requirement Description &
\begin{minipage}[]{13cm}
\textbf{Specification:} Configuration data shall be provided to DM in a
manner that allows the association of this data with the specific camera
images to which it pertains.
\end{minipage}
\\ \cdashline{1-2}
Requirement Discussion &
\begin{minipage}[]{13cm}
\textbf{Discussion:} This will be accomplished by using an
observatory-wide configuration tracking mechanism.
\end{minipage}
\\ \cdashline{1-2}
Requirement Priority &  \\ \cdashline{1-2}
Upper Level Requirement &
\begin{tabular}{cl}
\end{tabular}
\\ \hline
\end{longtable}
}


  
 \newpage 
\subsection{[LVV-6189] CA-DM-SUP-ICD-0029-V-04: Association with Camera Images\_DM\_4 }\label{lvv-6189}

\begin{longtable}{cccc}
\hline
\textbf{Jira Link} & \textbf{Assignee} & \textbf{Status} & \textbf{Test Cases}\\ \hline
\href{https://jira.lsstcorp.org/browse/LVV-6189}{LVV-6189} &
Leanne Guy & Not Covered &
\begin{tabular}{c}
\end{tabular}
\\
\hline
\end{longtable}

\textbf{Verification Element Description:} \\
Undefined

{\footnotesize
\begin{longtable}{p{2.5cm}p{13.5cm}}
\hline
\multicolumn{2}{c}{\textbf{Requirement Details}}\\ \hline
Requirement ID & CA-DM-SUP-ICD-0029 \\ \cdashline{1-2}
Requirement Description &
\begin{minipage}[]{13cm}
\textbf{Specification:} Configuration data shall be provided to DM in a
manner that allows the association of this data with the specific camera
images to which it pertains.
\end{minipage}
\\ \cdashline{1-2}
Requirement Discussion &
\begin{minipage}[]{13cm}
\textbf{Discussion:} This will be accomplished by using an
observatory-wide configuration tracking mechanism.
\end{minipage}
\\ \cdashline{1-2}
Requirement Priority &  \\ \cdashline{1-2}
Upper Level Requirement &
\begin{tabular}{cl}
\end{tabular}
\\ \hline
\end{longtable}
}


  
 \newpage 
\subsection{[LVV-6194] CA-DM-SUP-ICD-0031-V-03: Readout Micro-Program Characteristics\_DM\_3 }\label{lvv-6194}

\begin{longtable}{cccc}
\hline
\textbf{Jira Link} & \textbf{Assignee} & \textbf{Status} & \textbf{Test Cases}\\ \hline
\href{https://jira.lsstcorp.org/browse/LVV-6194}{LVV-6194} &
Leanne Guy & Not Covered &
\begin{tabular}{c}
\end{tabular}
\\
\hline
\end{longtable}

\textbf{Verification Element Description:} \\
Undefined

{\footnotesize
\begin{longtable}{p{2.5cm}p{13.5cm}}
\hline
\multicolumn{2}{c}{\textbf{Requirement Details}}\\ \hline
Requirement ID & CA-DM-SUP-ICD-0031 \\ \cdashline{1-2}
Requirement Description &
\begin{minipage}[]{13cm}
\textbf{Specification:} The camera shall make available the readout
micro-program characteristics, e.g., the readout pixel rate and the
number of overclock pixels, and the readout timing diagram.
\end{minipage}
\\ \cdashline{1-2}
Requirement Priority &  \\ \cdashline{1-2}
Upper Level Requirement &
\begin{tabular}{cl}
\end{tabular}
\\ \hline
\end{longtable}
}


  
 \newpage 
\subsection{[LVV-6195] CA-DM-SUP-ICD-0031-V-04: Readout Micro-Program Characteristics\_DM\_4 }\label{lvv-6195}

\begin{longtable}{cccc}
\hline
\textbf{Jira Link} & \textbf{Assignee} & \textbf{Status} & \textbf{Test Cases}\\ \hline
\href{https://jira.lsstcorp.org/browse/LVV-6195}{LVV-6195} &
Leanne Guy & Not Covered &
\begin{tabular}{c}
\end{tabular}
\\
\hline
\end{longtable}

\textbf{Verification Element Description:} \\
Undefined

{\footnotesize
\begin{longtable}{p{2.5cm}p{13.5cm}}
\hline
\multicolumn{2}{c}{\textbf{Requirement Details}}\\ \hline
Requirement ID & CA-DM-SUP-ICD-0031 \\ \cdashline{1-2}
Requirement Description &
\begin{minipage}[]{13cm}
\textbf{Specification:} The camera shall make available the readout
micro-program characteristics, e.g., the readout pixel rate and the
number of overclock pixels, and the readout timing diagram.
\end{minipage}
\\ \cdashline{1-2}
Requirement Priority &  \\ \cdashline{1-2}
Upper Level Requirement &
\begin{tabular}{cl}
\end{tabular}
\\ \hline
\end{longtable}
}


  
 \newpage 
\subsection{[LVV-6200] CA-DM-SUP-ICD-0030-V-03: Versioning Identifiers for Code \&
Firmware\_DM\_3 }\label{lvv-6200}

\begin{longtable}{cccc}
\hline
\textbf{Jira Link} & \textbf{Assignee} & \textbf{Status} & \textbf{Test Cases}\\ \hline
\href{https://jira.lsstcorp.org/browse/LVV-6200}{LVV-6200} &
Leanne Guy & Not Covered &
\begin{tabular}{c}
\end{tabular}
\\
\hline
\end{longtable}

\textbf{Verification Element Description:} \\
Undefined

{\footnotesize
\begin{longtable}{p{2.5cm}p{13.5cm}}
\hline
\multicolumn{2}{c}{\textbf{Requirement Details}}\\ \hline
Requirement ID & CA-DM-SUP-ICD-0030 \\ \cdashline{1-2}
Requirement Description &
\begin{minipage}[]{13cm}
\textbf{Specification:} The camera shall make available version control
identifiers for all code involved in the data acquisition chain. Code
versions for all firmware that can be updated in place are included in
this configuration data.
\end{minipage}
\\ \cdashline{1-2}
Requirement Discussion &
\begin{minipage}[]{13cm}
\textbf{Discussion:} Code versions for firmware that can only be
modified by physical access to the Camera hardware should be treated as
assembly data, as above.
\end{minipage}
\\ \cdashline{1-2}
Requirement Priority &  \\ \cdashline{1-2}
Upper Level Requirement &
\begin{tabular}{cl}
\end{tabular}
\\ \hline
\end{longtable}
}


  
 \newpage 
\subsection{[LVV-6201] CA-DM-SUP-ICD-0030-V-04: Versioning Identifiers for Code \&
Firmware\_DM\_4 }\label{lvv-6201}

\begin{longtable}{cccc}
\hline
\textbf{Jira Link} & \textbf{Assignee} & \textbf{Status} & \textbf{Test Cases}\\ \hline
\href{https://jira.lsstcorp.org/browse/LVV-6201}{LVV-6201} &
Leanne Guy & Not Covered &
\begin{tabular}{c}
\end{tabular}
\\
\hline
\end{longtable}

\textbf{Verification Element Description:} \\
Undefined

{\footnotesize
\begin{longtable}{p{2.5cm}p{13.5cm}}
\hline
\multicolumn{2}{c}{\textbf{Requirement Details}}\\ \hline
Requirement ID & CA-DM-SUP-ICD-0030 \\ \cdashline{1-2}
Requirement Description &
\begin{minipage}[]{13cm}
\textbf{Specification:} The camera shall make available version control
identifiers for all code involved in the data acquisition chain. Code
versions for all firmware that can be updated in place are included in
this configuration data.
\end{minipage}
\\ \cdashline{1-2}
Requirement Discussion &
\begin{minipage}[]{13cm}
\textbf{Discussion:} Code versions for firmware that can only be
modified by physical access to the Camera hardware should be treated as
assembly data, as above.
\end{minipage}
\\ \cdashline{1-2}
Requirement Priority &  \\ \cdashline{1-2}
Upper Level Requirement &
\begin{tabular}{cl}
\end{tabular}
\\ \hline
\end{longtable}
}


  
 \newpage 
\subsection{[LVV-6206] CA-DM-SUP-ICD-0008-V-03: As-Built Camera Geometry Specifications\_DM\_3 }\label{lvv-6206}

\begin{longtable}{cccc}
\hline
\textbf{Jira Link} & \textbf{Assignee} & \textbf{Status} & \textbf{Test Cases}\\ \hline
\href{https://jira.lsstcorp.org/browse/LVV-6206}{LVV-6206} &
Leanne Guy & Not Covered &
\begin{tabular}{c}
\end{tabular}
\\
\hline
\end{longtable}

\textbf{Verification Element Description:} \\
Undefined

{\footnotesize
\begin{longtable}{p{2.5cm}p{13.5cm}}
\hline
\multicolumn{2}{c}{\textbf{Requirement Details}}\\ \hline
Requirement ID & CA-DM-SUP-ICD-0008 \\ \cdashline{1-2}
Requirement Description &
\begin{minipage}[]{13cm}
\textbf{Specification:} The geometry model shall include at least:

\begin{itemize}
\tightlist
\item
  The spatial position, in the Camera coordinate system, of each sensor;
\item
  A map of the vertical position of the sensor surfaces, z(x,y);
\item
  The location of the intersection of the corrector optical axis with
  the focal plane.
\end{itemize}
\end{minipage}
\\ \cdashline{1-2}
Requirement Discussion &
\begin{minipage}[]{13cm}
\textbf{Discussion:} The required precision is TBD. An initial,
post-assembly version of this information should be provided, as well as
periodic updates (e.g., derived from the CCOB or from focus sweeps) and
updates when there are hardware changes (e.g., when rafts are exchanged
or serviced).
\end{minipage}
\\ \cdashline{1-2}
Requirement Priority &  \\ \cdashline{1-2}
Upper Level Requirement &
\begin{tabular}{cl}
\end{tabular}
\\ \hline
\end{longtable}
}


  
 \newpage 
\subsection{[LVV-6207] CA-DM-SUP-ICD-0008-V-04: As-Built Camera Geometry Specifications\_DM\_4 }\label{lvv-6207}

\begin{longtable}{cccc}
\hline
\textbf{Jira Link} & \textbf{Assignee} & \textbf{Status} & \textbf{Test Cases}\\ \hline
\href{https://jira.lsstcorp.org/browse/LVV-6207}{LVV-6207} &
Leanne Guy & Not Covered &
\begin{tabular}{c}
\end{tabular}
\\
\hline
\end{longtable}

\textbf{Verification Element Description:} \\
Undefined

{\footnotesize
\begin{longtable}{p{2.5cm}p{13.5cm}}
\hline
\multicolumn{2}{c}{\textbf{Requirement Details}}\\ \hline
Requirement ID & CA-DM-SUP-ICD-0008 \\ \cdashline{1-2}
Requirement Description &
\begin{minipage}[]{13cm}
\textbf{Specification:} The geometry model shall include at least:

\begin{itemize}
\tightlist
\item
  The spatial position, in the Camera coordinate system, of each sensor;
\item
  A map of the vertical position of the sensor surfaces, z(x,y);
\item
  The location of the intersection of the corrector optical axis with
  the focal plane.
\end{itemize}
\end{minipage}
\\ \cdashline{1-2}
Requirement Discussion &
\begin{minipage}[]{13cm}
\textbf{Discussion:} The required precision is TBD. An initial,
post-assembly version of this information should be provided, as well as
periodic updates (e.g., derived from the CCOB or from focus sweeps) and
updates when there are hardware changes (e.g., when rafts are exchanged
or serviced).
\end{minipage}
\\ \cdashline{1-2}
Requirement Priority &  \\ \cdashline{1-2}
Upper Level Requirement &
\begin{tabular}{cl}
\end{tabular}
\\ \hline
\end{longtable}
}


  
 \newpage 
\subsection{[LVV-6212] CA-DM-SUP-ICD-0007-V-03: As-Built Camera Geometry\_DM\_3 }\label{lvv-6212}

\begin{longtable}{cccc}
\hline
\textbf{Jira Link} & \textbf{Assignee} & \textbf{Status} & \textbf{Test Cases}\\ \hline
\href{https://jira.lsstcorp.org/browse/LVV-6212}{LVV-6212} &
Leanne Guy & Not Covered &
\begin{tabular}{c}
\end{tabular}
\\
\hline
\end{longtable}

\textbf{Verification Element Description:} \\
Undefined

{\footnotesize
\begin{longtable}{p{2.5cm}p{13.5cm}}
\hline
\multicolumn{2}{c}{\textbf{Requirement Details}}\\ \hline
Requirement ID & CA-DM-SUP-ICD-0007 \\ \cdashline{1-2}
Requirement Description &
\begin{minipage}[]{13cm}
\textbf{Specification:} The Camera shall make available a model of the
as-built geometry of the instrument.
\end{minipage}
\\ \cdashline{1-2}
Requirement Priority &  \\ \cdashline{1-2}
Upper Level Requirement &
\begin{tabular}{cl}
\end{tabular}
\\ \hline
\end{longtable}
}


  
 \newpage 
\subsection{[LVV-6213] CA-DM-SUP-ICD-0007-V-04: As-Built Camera Geometry\_DM\_4 }\label{lvv-6213}

\begin{longtable}{cccc}
\hline
\textbf{Jira Link} & \textbf{Assignee} & \textbf{Status} & \textbf{Test Cases}\\ \hline
\href{https://jira.lsstcorp.org/browse/LVV-6213}{LVV-6213} &
Leanne Guy & Not Covered &
\begin{tabular}{c}
\end{tabular}
\\
\hline
\end{longtable}

\textbf{Verification Element Description:} \\
Undefined

{\footnotesize
\begin{longtable}{p{2.5cm}p{13.5cm}}
\hline
\multicolumn{2}{c}{\textbf{Requirement Details}}\\ \hline
Requirement ID & CA-DM-SUP-ICD-0007 \\ \cdashline{1-2}
Requirement Description &
\begin{minipage}[]{13cm}
\textbf{Specification:} The Camera shall make available a model of the
as-built geometry of the instrument.
\end{minipage}
\\ \cdashline{1-2}
Requirement Priority &  \\ \cdashline{1-2}
Upper Level Requirement &
\begin{tabular}{cl}
\end{tabular}
\\ \hline
\end{longtable}
}


  
 \newpage 
\subsection{[LVV-6218] CA-DM-SUP-ICD-0009-V-03: Coordinate System Conventions\_DM\_3 }\label{lvv-6218}

\begin{longtable}{cccc}
\hline
\textbf{Jira Link} & \textbf{Assignee} & \textbf{Status} & \textbf{Test Cases}\\ \hline
\href{https://jira.lsstcorp.org/browse/LVV-6218}{LVV-6218} &
Leanne Guy & Not Covered &
\begin{tabular}{c}
\end{tabular}
\\
\hline
\end{longtable}

\textbf{Verification Element Description:} \\
Undefined

{\footnotesize
\begin{longtable}{p{2.5cm}p{13.5cm}}
\hline
\multicolumn{2}{c}{\textbf{Requirement Details}}\\ \hline
Requirement ID & CA-DM-SUP-ICD-0009 \\ \cdashline{1-2}
Requirement Description &
\begin{minipage}[]{13cm}
\textbf{Specification:} The geometry model shall obey the LSST
coordinate system conventions.
\end{minipage}
\\ \cdashline{1-2}
Requirement Discussion &
\begin{minipage}[]{13cm}
\textbf{Discussion:} As of this writing the details of the coordinate
system to be used are TBD and expected to be settled in Phase 2. An
LSE-series document is in preparation.
\end{minipage}
\\ \cdashline{1-2}
Requirement Priority &  \\ \cdashline{1-2}
Upper Level Requirement &
\begin{tabular}{cl}
\end{tabular}
\\ \hline
\end{longtable}
}


  
 \newpage 
\subsection{[LVV-6219] CA-DM-SUP-ICD-0009-V-04: Coordinate System Conventions\_DM\_4 }\label{lvv-6219}

\begin{longtable}{cccc}
\hline
\textbf{Jira Link} & \textbf{Assignee} & \textbf{Status} & \textbf{Test Cases}\\ \hline
\href{https://jira.lsstcorp.org/browse/LVV-6219}{LVV-6219} &
Leanne Guy & Not Covered &
\begin{tabular}{c}
\end{tabular}
\\
\hline
\end{longtable}

\textbf{Verification Element Description:} \\
Undefined

{\footnotesize
\begin{longtable}{p{2.5cm}p{13.5cm}}
\hline
\multicolumn{2}{c}{\textbf{Requirement Details}}\\ \hline
Requirement ID & CA-DM-SUP-ICD-0009 \\ \cdashline{1-2}
Requirement Description &
\begin{minipage}[]{13cm}
\textbf{Specification:} The geometry model shall obey the LSST
coordinate system conventions.
\end{minipage}
\\ \cdashline{1-2}
Requirement Discussion &
\begin{minipage}[]{13cm}
\textbf{Discussion:} As of this writing the details of the coordinate
system to be used are TBD and expected to be settled in Phase 2. An
LSE-series document is in preparation.
\end{minipage}
\\ \cdashline{1-2}
Requirement Priority &  \\ \cdashline{1-2}
Upper Level Requirement &
\begin{tabular}{cl}
\end{tabular}
\\ \hline
\end{longtable}
}


  
 \newpage 
\subsection{[LVV-6224] CA-DM-SUP-ICD-0010-V-03: Geometry Distortion Model\_DM\_3 }\label{lvv-6224}

\begin{longtable}{cccc}
\hline
\textbf{Jira Link} & \textbf{Assignee} & \textbf{Status} & \textbf{Test Cases}\\ \hline
\href{https://jira.lsstcorp.org/browse/LVV-6224}{LVV-6224} &
Leanne Guy & Not Covered &
\begin{tabular}{c}
\end{tabular}
\\
\hline
\end{longtable}

\textbf{Verification Element Description:} \\
Undefined

{\footnotesize
\begin{longtable}{p{2.5cm}p{13.5cm}}
\hline
\multicolumn{2}{c}{\textbf{Requirement Details}}\\ \hline
Requirement ID & CA-DM-SUP-ICD-0010 \\ \cdashline{1-2}
Requirement Description &
\begin{minipage}[]{13cm}
\textbf{Specification:} The Camera shall provide a model of the
distortions of the geometry as a function of significant state
variables, such as the spatial orientation of the instrument or its
temperature.
\end{minipage}
\\ \cdashline{1-2}
Requirement Discussion &
\begin{minipage}[]{13cm}
\textbf{Discussion:} The de minimis level below which this is not
required is TBD. When a state variable is needed for the model,
Requirement CA-DM-SUP-ICD-0028 below applies and the Camera must ensure
that that variable is available as telemetry.
\end{minipage}
\\ \cdashline{1-2}
Requirement Priority &  \\ \cdashline{1-2}
Upper Level Requirement &
\begin{tabular}{cl}
CA-DM-SUP-ICD-0028 & Telemetry for Parametric Models \\
\end{tabular}
\\ \hline
\end{longtable}
}


  
 \newpage 
\subsection{[LVV-6225] CA-DM-SUP-ICD-0010-V-04: Geometry Distortion Model\_DM\_4 }\label{lvv-6225}

\begin{longtable}{cccc}
\hline
\textbf{Jira Link} & \textbf{Assignee} & \textbf{Status} & \textbf{Test Cases}\\ \hline
\href{https://jira.lsstcorp.org/browse/LVV-6225}{LVV-6225} &
Leanne Guy & Not Covered &
\begin{tabular}{c}
\end{tabular}
\\
\hline
\end{longtable}

\textbf{Verification Element Description:} \\
Undefined

{\footnotesize
\begin{longtable}{p{2.5cm}p{13.5cm}}
\hline
\multicolumn{2}{c}{\textbf{Requirement Details}}\\ \hline
Requirement ID & CA-DM-SUP-ICD-0010 \\ \cdashline{1-2}
Requirement Description &
\begin{minipage}[]{13cm}
\textbf{Specification:} The Camera shall provide a model of the
distortions of the geometry as a function of significant state
variables, such as the spatial orientation of the instrument or its
temperature.
\end{minipage}
\\ \cdashline{1-2}
Requirement Discussion &
\begin{minipage}[]{13cm}
\textbf{Discussion:} The de minimis level below which this is not
required is TBD. When a state variable is needed for the model,
Requirement CA-DM-SUP-ICD-0028 below applies and the Camera must ensure
that that variable is available as telemetry.
\end{minipage}
\\ \cdashline{1-2}
Requirement Priority &  \\ \cdashline{1-2}
Upper Level Requirement &
\begin{tabular}{cl}
CA-DM-SUP-ICD-0028 & Telemetry for Parametric Models \\
\end{tabular}
\\ \hline
\end{longtable}
}


  
 \newpage 
\subsection{[LVV-6230] CA-DM-SUP-ICD-0020-V-03: Applicable Documentation\_DM\_3 }\label{lvv-6230}

\begin{longtable}{cccc}
\hline
\textbf{Jira Link} & \textbf{Assignee} & \textbf{Status} & \textbf{Test Cases}\\ \hline
\href{https://jira.lsstcorp.org/browse/LVV-6230}{LVV-6230} &
Leanne Guy & Not Covered &
\begin{tabular}{c}
\end{tabular}
\\
\hline
\end{longtable}

\textbf{Verification Element Description:} \\
Undefined

{\footnotesize
\begin{longtable}{p{2.5cm}p{13.5cm}}
\hline
\multicolumn{2}{c}{\textbf{Requirement Details}}\\ \hline
Requirement ID & CA-DM-SUP-ICD-0020 \\ \cdashline{1-2}
Requirement Description &
\begin{minipage}[]{13cm}
\textbf{Specification:} The Camera shall provide the documentation
necessary to understand and apply the data provided under this ICD.
\end{minipage}
\\ \cdashline{1-2}
Requirement Priority &  \\ \cdashline{1-2}
Upper Level Requirement &
\begin{tabular}{cl}
\end{tabular}
\\ \hline
\end{longtable}
}


  
 \newpage 
\subsection{[LVV-6231] CA-DM-SUP-ICD-0020-V-04: Applicable Documentation\_DM\_4 }\label{lvv-6231}

\begin{longtable}{cccc}
\hline
\textbf{Jira Link} & \textbf{Assignee} & \textbf{Status} & \textbf{Test Cases}\\ \hline
\href{https://jira.lsstcorp.org/browse/LVV-6231}{LVV-6231} &
Leanne Guy & Not Covered &
\begin{tabular}{c}
\end{tabular}
\\
\hline
\end{longtable}

\textbf{Verification Element Description:} \\
Undefined

{\footnotesize
\begin{longtable}{p{2.5cm}p{13.5cm}}
\hline
\multicolumn{2}{c}{\textbf{Requirement Details}}\\ \hline
Requirement ID & CA-DM-SUP-ICD-0020 \\ \cdashline{1-2}
Requirement Description &
\begin{minipage}[]{13cm}
\textbf{Specification:} The Camera shall provide the documentation
necessary to understand and apply the data provided under this ICD.
\end{minipage}
\\ \cdashline{1-2}
Requirement Priority &  \\ \cdashline{1-2}
Upper Level Requirement &
\begin{tabular}{cl}
\end{tabular}
\\ \hline
\end{longtable}
}


  
 \newpage 
\subsection{[LVV-6236] CA-DM-SUP-ICD-0019-V-03: Machine Readable Format\_DM\_3 }\label{lvv-6236}

\begin{longtable}{cccc}
\hline
\textbf{Jira Link} & \textbf{Assignee} & \textbf{Status} & \textbf{Test Cases}\\ \hline
\href{https://jira.lsstcorp.org/browse/LVV-6236}{LVV-6236} &
Leanne Guy & Not Covered &
\begin{tabular}{c}
\end{tabular}
\\
\hline
\end{longtable}

\textbf{Verification Element Description:} \\
Undefined

{\footnotesize
\begin{longtable}{p{2.5cm}p{13.5cm}}
\hline
\multicolumn{2}{c}{\textbf{Requirement Details}}\\ \hline
Requirement ID & CA-DM-SUP-ICD-0019 \\ \cdashline{1-2}
Requirement Description &
\begin{minipage}[]{13cm}
\textbf{Specification:} The Camera shall provide all data under this ICD
in a machine-readable form suitable for use as input to automated
processes. The Camera shall provide sufficient metadata to associate the
test results with the identifiers required by CA-DM-SUP-ICD-0002, -0003,
and -0004 above.
\end{minipage}
\\ \cdashline{1-2}
Requirement Priority &  \\ \cdashline{1-2}
Upper Level Requirement &
\begin{tabular}{cl}
CA-DM-SUP-ICD-0002 & Camera Instrument Composition Description \\
CA-DM-SUP-ICD-0003 & Component Geographical and Physical Location Pairing \\
CA-DM-SUP-ICD-0004 & Component Mapping Persistence \\
\end{tabular}
\\ \hline
\end{longtable}
}


  
 \newpage 
\subsection{[LVV-6237] CA-DM-SUP-ICD-0019-V-04: Machine Readable Format\_DM\_4 }\label{lvv-6237}

\begin{longtable}{cccc}
\hline
\textbf{Jira Link} & \textbf{Assignee} & \textbf{Status} & \textbf{Test Cases}\\ \hline
\href{https://jira.lsstcorp.org/browse/LVV-6237}{LVV-6237} &
Leanne Guy & Not Covered &
\begin{tabular}{c}
\end{tabular}
\\
\hline
\end{longtable}

\textbf{Verification Element Description:} \\
Undefined

{\footnotesize
\begin{longtable}{p{2.5cm}p{13.5cm}}
\hline
\multicolumn{2}{c}{\textbf{Requirement Details}}\\ \hline
Requirement ID & CA-DM-SUP-ICD-0019 \\ \cdashline{1-2}
Requirement Description &
\begin{minipage}[]{13cm}
\textbf{Specification:} The Camera shall provide all data under this ICD
in a machine-readable form suitable for use as input to automated
processes. The Camera shall provide sufficient metadata to associate the
test results with the identifiers required by CA-DM-SUP-ICD-0002, -0003,
and -0004 above.
\end{minipage}
\\ \cdashline{1-2}
Requirement Priority &  \\ \cdashline{1-2}
Upper Level Requirement &
\begin{tabular}{cl}
CA-DM-SUP-ICD-0002 & Camera Instrument Composition Description \\
CA-DM-SUP-ICD-0003 & Component Geographical and Physical Location Pairing \\
CA-DM-SUP-ICD-0004 & Component Mapping Persistence \\
\end{tabular}
\\ \hline
\end{longtable}
}


  
 \newpage 
\subsection{[LVV-6242] CA-DM-SUP-ICD-0005-V-03: Focal Plane Electronic Layout
Description\_DM\_3 }\label{lvv-6242}

\begin{longtable}{cccc}
\hline
\textbf{Jira Link} & \textbf{Assignee} & \textbf{Status} & \textbf{Test Cases}\\ \hline
\href{https://jira.lsstcorp.org/browse/LVV-6242}{LVV-6242} &
Leanne Guy & Not Covered &
\begin{tabular}{c}
\end{tabular}
\\
\hline
\end{longtable}

\textbf{Verification Element Description:} \\
Undefined

{\footnotesize
\begin{longtable}{p{2.5cm}p{13.5cm}}
\hline
\multicolumn{2}{c}{\textbf{Requirement Details}}\\ \hline
Requirement ID & CA-DM-SUP-ICD-0005 \\ \cdashline{1-2}
Requirement Description &
\begin{minipage}[]{13cm}
\textbf{Specification:} The Camera shall make available a description of
the electronic layout of the focal plane, including sensor pixel
dimensions, the structure of any pre-scan regions, the readout
directions of the segments within the sensors, the locations of bloom
stops, and the like.
\end{minipage}
\\ \cdashline{1-2}
Requirement Discussion &
\begin{minipage}[]{13cm}
\textbf{Discussion:} Note that the definitions of readout regions and
overclocking are not physical properties of the sensors, but are
programmable, and as such are part of the Configuration Data below.
\end{minipage}
\\ \cdashline{1-2}
Requirement Priority &  \\ \cdashline{1-2}
Upper Level Requirement &
\begin{tabular}{cl}
\end{tabular}
\\ \hline
\end{longtable}
}


  
 \newpage 
\subsection{[LVV-6243] CA-DM-SUP-ICD-0005-V-04: Focal Plane Electronic Layout
Description\_DM\_4 }\label{lvv-6243}

\begin{longtable}{cccc}
\hline
\textbf{Jira Link} & \textbf{Assignee} & \textbf{Status} & \textbf{Test Cases}\\ \hline
\href{https://jira.lsstcorp.org/browse/LVV-6243}{LVV-6243} &
Leanne Guy & Not Covered &
\begin{tabular}{c}
\end{tabular}
\\
\hline
\end{longtable}

\textbf{Verification Element Description:} \\
Undefined

{\footnotesize
\begin{longtable}{p{2.5cm}p{13.5cm}}
\hline
\multicolumn{2}{c}{\textbf{Requirement Details}}\\ \hline
Requirement ID & CA-DM-SUP-ICD-0005 \\ \cdashline{1-2}
Requirement Description &
\begin{minipage}[]{13cm}
\textbf{Specification:} The Camera shall make available a description of
the electronic layout of the focal plane, including sensor pixel
dimensions, the structure of any pre-scan regions, the readout
directions of the segments within the sensors, the locations of bloom
stops, and the like.
\end{minipage}
\\ \cdashline{1-2}
Requirement Discussion &
\begin{minipage}[]{13cm}
\textbf{Discussion:} Note that the definitions of readout regions and
overclocking are not physical properties of the sensors, but are
programmable, and as such are part of the Configuration Data below.
\end{minipage}
\\ \cdashline{1-2}
Requirement Priority &  \\ \cdashline{1-2}
Upper Level Requirement &
\begin{tabular}{cl}
\end{tabular}
\\ \hline
\end{longtable}
}


  
 \newpage 
\subsection{[LVV-6248] CA-DM-SUP-ICD-0006-V-03: Geographical Mapping Between Sensors and
Electronics\_DM\_3 }\label{lvv-6248}

\begin{longtable}{cccc}
\hline
\textbf{Jira Link} & \textbf{Assignee} & \textbf{Status} & \textbf{Test Cases}\\ \hline
\href{https://jira.lsstcorp.org/browse/LVV-6248}{LVV-6248} &
Leanne Guy & Not Covered &
\begin{tabular}{c}
\end{tabular}
\\
\hline
\end{longtable}

\textbf{Verification Element Description:} \\
Undefined

{\footnotesize
\begin{longtable}{p{2.5cm}p{13.5cm}}
\hline
\multicolumn{2}{c}{\textbf{Requirement Details}}\\ \hline
Requirement ID & CA-DM-SUP-ICD-0006 \\ \cdashline{1-2}
Requirement Description &
\begin{minipage}[]{13cm}
\textbf{Specification:} The Camera shall make available the mapping of
geographical identifiers between sensors and their associated
electronics.
\end{minipage}
\\ \cdashline{1-2}
Requirement Priority &  \\ \cdashline{1-2}
Upper Level Requirement &
\begin{tabular}{cl}
\end{tabular}
\\ \hline
\end{longtable}
}


  
 \newpage 
\subsection{[LVV-6249] CA-DM-SUP-ICD-0006-V-04: Geographical Mapping Between Sensors and
Electronics\_DM\_4 }\label{lvv-6249}

\begin{longtable}{cccc}
\hline
\textbf{Jira Link} & \textbf{Assignee} & \textbf{Status} & \textbf{Test Cases}\\ \hline
\href{https://jira.lsstcorp.org/browse/LVV-6249}{LVV-6249} &
Leanne Guy & Not Covered &
\begin{tabular}{c}
\end{tabular}
\\
\hline
\end{longtable}

\textbf{Verification Element Description:} \\
Undefined

{\footnotesize
\begin{longtable}{p{2.5cm}p{13.5cm}}
\hline
\multicolumn{2}{c}{\textbf{Requirement Details}}\\ \hline
Requirement ID & CA-DM-SUP-ICD-0006 \\ \cdashline{1-2}
Requirement Description &
\begin{minipage}[]{13cm}
\textbf{Specification:} The Camera shall make available the mapping of
geographical identifiers between sensors and their associated
electronics.
\end{minipage}
\\ \cdashline{1-2}
Requirement Priority &  \\ \cdashline{1-2}
Upper Level Requirement &
\begin{tabular}{cl}
\end{tabular}
\\ \hline
\end{longtable}
}


  
 \newpage 
\subsection{[LVV-6254] CA-DM-SUP-ICD-0002-V-03: Camera Instrument Composition
Description\_DM\_3 }\label{lvv-6254}

\begin{longtable}{cccc}
\hline
\textbf{Jira Link} & \textbf{Assignee} & \textbf{Status} & \textbf{Test Cases}\\ \hline
\href{https://jira.lsstcorp.org/browse/LVV-6254}{LVV-6254} &
Leanne Guy & Not Covered &
\begin{tabular}{c}
\end{tabular}
\\
\hline
\end{longtable}

\textbf{Verification Element Description:} \\
Undefined

{\footnotesize
\begin{longtable}{p{2.5cm}p{13.5cm}}
\hline
\multicolumn{2}{c}{\textbf{Requirement Details}}\\ \hline
Requirement ID & CA-DM-SUP-ICD-0002 \\ \cdashline{1-2}
Requirement Description &
\begin{minipage}[]{13cm}
\textbf{Specification:} The Camera shall provide data describing the
composition of the instrument. This shall at a minimum include
information on the identity of each sensor, each significant
line-replaceable electronic component in the readout chain, each raft,
and each filter.
\end{minipage}
\\ \cdashline{1-2}
Requirement Discussion &
\begin{minipage}[]{13cm}
\textbf{Discussion:} The full list of components that will be tracked in
this manner is TBD. It is anticipated that the Camera will be likely to
track a superset of the above-required information, simply as a result
of sound engineering practice ? to support diagnostics, preventative
maintenance, spares tracking, and the like.
\end{minipage}
\\ \cdashline{1-2}
Requirement Priority &  \\ \cdashline{1-2}
Upper Level Requirement &
\begin{tabular}{cl}
\end{tabular}
\\ \hline
\end{longtable}
}


  
 \newpage 
\subsection{[LVV-6255] CA-DM-SUP-ICD-0002-V-04: Camera Instrument Composition
Description\_DM\_4 }\label{lvv-6255}

\begin{longtable}{cccc}
\hline
\textbf{Jira Link} & \textbf{Assignee} & \textbf{Status} & \textbf{Test Cases}\\ \hline
\href{https://jira.lsstcorp.org/browse/LVV-6255}{LVV-6255} &
Leanne Guy & Not Covered &
\begin{tabular}{c}
\end{tabular}
\\
\hline
\end{longtable}

\textbf{Verification Element Description:} \\
Undefined

{\footnotesize
\begin{longtable}{p{2.5cm}p{13.5cm}}
\hline
\multicolumn{2}{c}{\textbf{Requirement Details}}\\ \hline
Requirement ID & CA-DM-SUP-ICD-0002 \\ \cdashline{1-2}
Requirement Description &
\begin{minipage}[]{13cm}
\textbf{Specification:} The Camera shall provide data describing the
composition of the instrument. This shall at a minimum include
information on the identity of each sensor, each significant
line-replaceable electronic component in the readout chain, each raft,
and each filter.
\end{minipage}
\\ \cdashline{1-2}
Requirement Discussion &
\begin{minipage}[]{13cm}
\textbf{Discussion:} The full list of components that will be tracked in
this manner is TBD. It is anticipated that the Camera will be likely to
track a superset of the above-required information, simply as a result
of sound engineering practice ? to support diagnostics, preventative
maintenance, spares tracking, and the like.
\end{minipage}
\\ \cdashline{1-2}
Requirement Priority &  \\ \cdashline{1-2}
Upper Level Requirement &
\begin{tabular}{cl}
\end{tabular}
\\ \hline
\end{longtable}
}


  
 \newpage 
\subsection{[LVV-6260] CA-DM-SUP-ICD-0003-V-03: Component Geographical and Physical Location
Pairing\_DM\_3 }\label{lvv-6260}

\begin{longtable}{cccc}
\hline
\textbf{Jira Link} & \textbf{Assignee} & \textbf{Status} & \textbf{Test Cases}\\ \hline
\href{https://jira.lsstcorp.org/browse/LVV-6260}{LVV-6260} &
Leanne Guy & Not Covered &
\begin{tabular}{c}
\end{tabular}
\\
\hline
\end{longtable}

\textbf{Verification Element Description:} \\
Undefined

{\footnotesize
\begin{longtable}{p{2.5cm}p{13.5cm}}
\hline
\multicolumn{2}{c}{\textbf{Requirement Details}}\\ \hline
Requirement ID & CA-DM-SUP-ICD-0003 \\ \cdashline{1-2}
Requirement Description &
\begin{minipage}[]{13cm}
\textbf{Specification:} Each identified component and assembly shall be
tracked by pairing a ``slot'' or geographical identifier of a location
in the camera or in a camera assembly with a physical identity, such as
a serial number, associated with the object in that location.
\end{minipage}
\\ \cdashline{1-2}
Requirement Discussion &
\begin{minipage}[]{13cm}
\textbf{Discussion:} This means, for instance, that it should be
possible to determine that the raft with serial number X is present in
raft bay Y, and that CCD Z is in grid position W on the raft with serial
number X.
\end{minipage}
\\ \cdashline{1-2}
Requirement Priority &  \\ \cdashline{1-2}
Upper Level Requirement &
\begin{tabular}{cl}
\end{tabular}
\\ \hline
\end{longtable}
}


  
 \newpage 
\subsection{[LVV-6261] CA-DM-SUP-ICD-0003-V-04: Component Geographical and Physical Location
Pairing\_DM\_4 }\label{lvv-6261}

\begin{longtable}{cccc}
\hline
\textbf{Jira Link} & \textbf{Assignee} & \textbf{Status} & \textbf{Test Cases}\\ \hline
\href{https://jira.lsstcorp.org/browse/LVV-6261}{LVV-6261} &
Leanne Guy & Not Covered &
\begin{tabular}{c}
\end{tabular}
\\
\hline
\end{longtable}

\textbf{Verification Element Description:} \\
Undefined

{\footnotesize
\begin{longtable}{p{2.5cm}p{13.5cm}}
\hline
\multicolumn{2}{c}{\textbf{Requirement Details}}\\ \hline
Requirement ID & CA-DM-SUP-ICD-0003 \\ \cdashline{1-2}
Requirement Description &
\begin{minipage}[]{13cm}
\textbf{Specification:} Each identified component and assembly shall be
tracked by pairing a ``slot'' or geographical identifier of a location
in the camera or in a camera assembly with a physical identity, such as
a serial number, associated with the object in that location.
\end{minipage}
\\ \cdashline{1-2}
Requirement Discussion &
\begin{minipage}[]{13cm}
\textbf{Discussion:} This means, for instance, that it should be
possible to determine that the raft with serial number X is present in
raft bay Y, and that CCD Z is in grid position W on the raft with serial
number X.
\end{minipage}
\\ \cdashline{1-2}
Requirement Priority &  \\ \cdashline{1-2}
Upper Level Requirement &
\begin{tabular}{cl}
\end{tabular}
\\ \hline
\end{longtable}
}


  
 \newpage 
\subsection{[LVV-6266] CA-DM-SUP-ICD-0004-V-03: Component Mapping Persistence\_DM\_3 }\label{lvv-6266}

\begin{longtable}{cccc}
\hline
\textbf{Jira Link} & \textbf{Assignee} & \textbf{Status} & \textbf{Test Cases}\\ \hline
\href{https://jira.lsstcorp.org/browse/LVV-6266}{LVV-6266} &
Leanne Guy & Not Covered &
\begin{tabular}{c}
\end{tabular}
\\
\hline
\end{longtable}

\textbf{Verification Element Description:} \\
Undefined

{\footnotesize
\begin{longtable}{p{2.5cm}p{13.5cm}}
\hline
\multicolumn{2}{c}{\textbf{Requirement Details}}\\ \hline
Requirement ID & CA-DM-SUP-ICD-0004 \\ \cdashline{1-2}
Requirement Description &
\begin{minipage}[]{13cm}
\textbf{Specification:} The evolution of this mapping shall be made
available for the entire history of integration and test data, and of
commissioning and operations.
\end{minipage}
\\ \cdashline{1-2}
Requirement Discussion &
\begin{minipage}[]{13cm}
\textbf{Discussion:} The spirit of this requirement is that any change
made to the camera assembly hierarchy must be recorded if any archived
data was taken in that configuration.
\end{minipage}
\\ \cdashline{1-2}
Requirement Priority &  \\ \cdashline{1-2}
Upper Level Requirement &
\begin{tabular}{cl}
\end{tabular}
\\ \hline
\end{longtable}
}


  
 \newpage 
\subsection{[LVV-6267] CA-DM-SUP-ICD-0004-V-04: Component Mapping Persistence\_DM\_4 }\label{lvv-6267}

\begin{longtable}{cccc}
\hline
\textbf{Jira Link} & \textbf{Assignee} & \textbf{Status} & \textbf{Test Cases}\\ \hline
\href{https://jira.lsstcorp.org/browse/LVV-6267}{LVV-6267} &
Leanne Guy & Not Covered &
\begin{tabular}{c}
\end{tabular}
\\
\hline
\end{longtable}

\textbf{Verification Element Description:} \\
Undefined

{\footnotesize
\begin{longtable}{p{2.5cm}p{13.5cm}}
\hline
\multicolumn{2}{c}{\textbf{Requirement Details}}\\ \hline
Requirement ID & CA-DM-SUP-ICD-0004 \\ \cdashline{1-2}
Requirement Description &
\begin{minipage}[]{13cm}
\textbf{Specification:} The evolution of this mapping shall be made
available for the entire history of integration and test data, and of
commissioning and operations.
\end{minipage}
\\ \cdashline{1-2}
Requirement Discussion &
\begin{minipage}[]{13cm}
\textbf{Discussion:} The spirit of this requirement is that any change
made to the camera assembly hierarchy must be recorded if any archived
data was taken in that configuration.
\end{minipage}
\\ \cdashline{1-2}
Requirement Priority &  \\ \cdashline{1-2}
Upper Level Requirement &
\begin{tabular}{cl}
\end{tabular}
\\ \hline
\end{longtable}
}


  
 \newpage 
\subsection{[LVV-6272] CA-DM-SUP-ICD-0016-V-03: Optical Distortion Map\_DM\_3 }\label{lvv-6272}

\begin{longtable}{cccc}
\hline
\textbf{Jira Link} & \textbf{Assignee} & \textbf{Status} & \textbf{Test Cases}\\ \hline
\href{https://jira.lsstcorp.org/browse/LVV-6272}{LVV-6272} &
Leanne Guy & Not Covered &
\begin{tabular}{c}
\end{tabular}
\\
\hline
\end{longtable}

\textbf{Verification Element Description:} \\
Undefined

{\footnotesize
\begin{longtable}{p{2.5cm}p{13.5cm}}
\hline
\multicolumn{2}{c}{\textbf{Requirement Details}}\\ \hline
Requirement ID & CA-DM-SUP-ICD-0016 \\ \cdashline{1-2}
Requirement Description &
\begin{minipage}[]{13cm}
\textbf{Specification:} The Camera shall provide the data necessary to
construct an optical distortion map.
\end{minipage}
\\ \cdashline{1-2}
Requirement Discussion &
\begin{minipage}[]{13cm}
\textbf{Discussion:} The Camera is not required to do the end-to-end
modeling; this is expected to be a task for Systems Engineering.
\end{minipage}
\\ \cdashline{1-2}
Requirement Priority &  \\ \cdashline{1-2}
Upper Level Requirement &
\begin{tabular}{cl}
\end{tabular}
\\ \hline
\end{longtable}
}


  
 \newpage 
\subsection{[LVV-6273] CA-DM-SUP-ICD-0016-V-04: Optical Distortion Map\_DM\_4 }\label{lvv-6273}

\begin{longtable}{cccc}
\hline
\textbf{Jira Link} & \textbf{Assignee} & \textbf{Status} & \textbf{Test Cases}\\ \hline
\href{https://jira.lsstcorp.org/browse/LVV-6273}{LVV-6273} &
Leanne Guy & Not Covered &
\begin{tabular}{c}
\end{tabular}
\\
\hline
\end{longtable}

\textbf{Verification Element Description:} \\
Undefined

{\footnotesize
\begin{longtable}{p{2.5cm}p{13.5cm}}
\hline
\multicolumn{2}{c}{\textbf{Requirement Details}}\\ \hline
Requirement ID & CA-DM-SUP-ICD-0016 \\ \cdashline{1-2}
Requirement Description &
\begin{minipage}[]{13cm}
\textbf{Specification:} The Camera shall provide the data necessary to
construct an optical distortion map.
\end{minipage}
\\ \cdashline{1-2}
Requirement Discussion &
\begin{minipage}[]{13cm}
\textbf{Discussion:} The Camera is not required to do the end-to-end
modeling; this is expected to be a task for Systems Engineering.
\end{minipage}
\\ \cdashline{1-2}
Requirement Priority &  \\ \cdashline{1-2}
Upper Level Requirement &
\begin{tabular}{cl}
\end{tabular}
\\ \hline
\end{longtable}
}


  
 \newpage 
\subsection{[LVV-6278] CA-DM-SUP-ICD-0015-V-03: Scattered Light Model\_DM\_3 }\label{lvv-6278}

\begin{longtable}{cccc}
\hline
\textbf{Jira Link} & \textbf{Assignee} & \textbf{Status} & \textbf{Test Cases}\\ \hline
\href{https://jira.lsstcorp.org/browse/LVV-6278}{LVV-6278} &
Leanne Guy & Not Covered &
\begin{tabular}{c}
\end{tabular}
\\
\hline
\end{longtable}

\textbf{Verification Element Description:} \\
Undefined

{\footnotesize
\begin{longtable}{p{2.5cm}p{13.5cm}}
\hline
\multicolumn{2}{c}{\textbf{Requirement Details}}\\ \hline
Requirement ID & CA-DM-SUP-ICD-0015 \\ \cdashline{1-2}
Requirement Description &
\begin{minipage}[]{13cm}
\textbf{Specification:} The Camera shall provide the data necessary to
construct a model of the scattered light and ghosting as a function of
wavelength. The model shall include all relevant optical elements,
including baffles.
\end{minipage}
\\ \cdashline{1-2}
Requirement Discussion &
\begin{minipage}[]{13cm}
\textbf{Discussion:} The Camera is not required to do the end-to-end
modeling; this is expected to be a task for Systems Engineering.
\end{minipage}
\\ \cdashline{1-2}
Requirement Priority &  \\ \cdashline{1-2}
Upper Level Requirement &
\begin{tabular}{cl}
\end{tabular}
\\ \hline
\end{longtable}
}


  
 \newpage 
\subsection{[LVV-6279] CA-DM-SUP-ICD-0015-V-04: Scattered Light Model\_DM\_4 }\label{lvv-6279}

\begin{longtable}{cccc}
\hline
\textbf{Jira Link} & \textbf{Assignee} & \textbf{Status} & \textbf{Test Cases}\\ \hline
\href{https://jira.lsstcorp.org/browse/LVV-6279}{LVV-6279} &
Leanne Guy & Not Covered &
\begin{tabular}{c}
\end{tabular}
\\
\hline
\end{longtable}

\textbf{Verification Element Description:} \\
Undefined

{\footnotesize
\begin{longtable}{p{2.5cm}p{13.5cm}}
\hline
\multicolumn{2}{c}{\textbf{Requirement Details}}\\ \hline
Requirement ID & CA-DM-SUP-ICD-0015 \\ \cdashline{1-2}
Requirement Description &
\begin{minipage}[]{13cm}
\textbf{Specification:} The Camera shall provide the data necessary to
construct a model of the scattered light and ghosting as a function of
wavelength. The model shall include all relevant optical elements,
including baffles.
\end{minipage}
\\ \cdashline{1-2}
Requirement Discussion &
\begin{minipage}[]{13cm}
\textbf{Discussion:} The Camera is not required to do the end-to-end
modeling; this is expected to be a task for Systems Engineering.
\end{minipage}
\\ \cdashline{1-2}
Requirement Priority &  \\ \cdashline{1-2}
Upper Level Requirement &
\begin{tabular}{cl}
\end{tabular}
\\ \hline
\end{longtable}
}


  
 \newpage 
\subsection{[LVV-6284] CA-DM-SUP-ICD-0017-V-03: Shutter Shadowing Model\_DM\_3 }\label{lvv-6284}

\begin{longtable}{cccc}
\hline
\textbf{Jira Link} & \textbf{Assignee} & \textbf{Status} & \textbf{Test Cases}\\ \hline
\href{https://jira.lsstcorp.org/browse/LVV-6284}{LVV-6284} &
Leanne Guy & Not Covered &
\begin{tabular}{c}
\end{tabular}
\\
\hline
\end{longtable}

\textbf{Verification Element Description:} \\
Undefined

{\footnotesize
\begin{longtable}{p{2.5cm}p{13.5cm}}
\hline
\multicolumn{2}{c}{\textbf{Requirement Details}}\\ \hline
Requirement ID & CA-DM-SUP-ICD-0017 \\ \cdashline{1-2}
Requirement Description &
\begin{minipage}[]{13cm}
\textbf{Specification:} The Camera shall provide a model of the
shadowing produced by the shutter as a function of its reported position
(c.f. CA-DM-SUP-ICD-0021 below), as a function of passband.
\end{minipage}
\\ \cdashline{1-2}
Requirement Priority &  \\ \cdashline{1-2}
Upper Level Requirement &
\begin{tabular}{cl}
CA-DM-SUP-ICD-0021 & Shutter Motion Profiles \\
\end{tabular}
\\ \hline
\end{longtable}
}


  
 \newpage 
\subsection{[LVV-6285] CA-DM-SUP-ICD-0017-V-04: Shutter Shadowing Model\_DM\_4 }\label{lvv-6285}

\begin{longtable}{cccc}
\hline
\textbf{Jira Link} & \textbf{Assignee} & \textbf{Status} & \textbf{Test Cases}\\ \hline
\href{https://jira.lsstcorp.org/browse/LVV-6285}{LVV-6285} &
Leanne Guy & Not Covered &
\begin{tabular}{c}
\end{tabular}
\\
\hline
\end{longtable}

\textbf{Verification Element Description:} \\
Undefined

{\footnotesize
\begin{longtable}{p{2.5cm}p{13.5cm}}
\hline
\multicolumn{2}{c}{\textbf{Requirement Details}}\\ \hline
Requirement ID & CA-DM-SUP-ICD-0017 \\ \cdashline{1-2}
Requirement Description &
\begin{minipage}[]{13cm}
\textbf{Specification:} The Camera shall provide a model of the
shadowing produced by the shutter as a function of its reported position
(c.f. CA-DM-SUP-ICD-0021 below), as a function of passband.
\end{minipage}
\\ \cdashline{1-2}
Requirement Priority &  \\ \cdashline{1-2}
Upper Level Requirement &
\begin{tabular}{cl}
CA-DM-SUP-ICD-0021 & Shutter Motion Profiles \\
\end{tabular}
\\ \hline
\end{longtable}
}


  
 \newpage 
\subsection{[LVV-6290] CA-DM-SUP-ICD-0014-V-03: Vignetting Model\_DM\_3 }\label{lvv-6290}

\begin{longtable}{cccc}
\hline
\textbf{Jira Link} & \textbf{Assignee} & \textbf{Status} & \textbf{Test Cases}\\ \hline
\href{https://jira.lsstcorp.org/browse/LVV-6290}{LVV-6290} &
Leanne Guy & Not Covered &
\begin{tabular}{c}
\end{tabular}
\\
\hline
\end{longtable}

\textbf{Verification Element Description:} \\
Undefined

{\footnotesize
\begin{longtable}{p{2.5cm}p{13.5cm}}
\hline
\multicolumn{2}{c}{\textbf{Requirement Details}}\\ \hline
Requirement ID & CA-DM-SUP-ICD-0014 \\ \cdashline{1-2}
Requirement Description &
\begin{minipage}[]{13cm}
\textbf{Specification:} The Camera shall provide the data necessary to
construct a model of the vignetting as a function of wavelength. The
model shall include all relevant optical elements, including baffles.
\end{minipage}
\\ \cdashline{1-2}
Requirement Discussion &
\begin{minipage}[]{13cm}
\textbf{Discussion:} The Camera is not required to do the end-to-end
modeling; this is expected to be a task for Systems Engineering.
\end{minipage}
\\ \cdashline{1-2}
Requirement Priority &  \\ \cdashline{1-2}
Upper Level Requirement &
\begin{tabular}{cl}
\end{tabular}
\\ \hline
\end{longtable}
}


  
 \newpage 
\subsection{[LVV-6291] CA-DM-SUP-ICD-0014-V-04: Vignetting Model\_DM\_4 }\label{lvv-6291}

\begin{longtable}{cccc}
\hline
\textbf{Jira Link} & \textbf{Assignee} & \textbf{Status} & \textbf{Test Cases}\\ \hline
\href{https://jira.lsstcorp.org/browse/LVV-6291}{LVV-6291} &
Leanne Guy & Not Covered &
\begin{tabular}{c}
\end{tabular}
\\
\hline
\end{longtable}

\textbf{Verification Element Description:} \\
Undefined

{\footnotesize
\begin{longtable}{p{2.5cm}p{13.5cm}}
\hline
\multicolumn{2}{c}{\textbf{Requirement Details}}\\ \hline
Requirement ID & CA-DM-SUP-ICD-0014 \\ \cdashline{1-2}
Requirement Description &
\begin{minipage}[]{13cm}
\textbf{Specification:} The Camera shall provide the data necessary to
construct a model of the vignetting as a function of wavelength. The
model shall include all relevant optical elements, including baffles.
\end{minipage}
\\ \cdashline{1-2}
Requirement Discussion &
\begin{minipage}[]{13cm}
\textbf{Discussion:} The Camera is not required to do the end-to-end
modeling; this is expected to be a task for Systems Engineering.
\end{minipage}
\\ \cdashline{1-2}
Requirement Priority &  \\ \cdashline{1-2}
Upper Level Requirement &
\begin{tabular}{cl}
\end{tabular}
\\ \hline
\end{longtable}
}


  
 \newpage 
\subsection{[LVV-6296] CA-DM-SUP-ICD-0013-V-03: Filter and Lens Vendor Test Results\_DM\_3 }\label{lvv-6296}

\begin{longtable}{cccc}
\hline
\textbf{Jira Link} & \textbf{Assignee} & \textbf{Status} & \textbf{Test Cases}\\ \hline
\href{https://jira.lsstcorp.org/browse/LVV-6296}{LVV-6296} &
Leanne Guy & Not Covered &
\begin{tabular}{c}
\end{tabular}
\\
\hline
\end{longtable}

\textbf{Verification Element Description:} \\
Undefined

{\footnotesize
\begin{longtable}{p{2.5cm}p{13.5cm}}
\hline
\multicolumn{2}{c}{\textbf{Requirement Details}}\\ \hline
Requirement ID & CA-DM-SUP-ICD-0013 \\ \cdashline{1-2}
Requirement Description &
\begin{minipage}[]{13cm}
\textbf{Specification}: The Camera shall provide the quantitative
results of optical tests performed by the filter and lens vendors.
\end{minipage}
\\ \cdashline{1-2}
Requirement Discussion &
\begin{minipage}[]{13cm}
\textbf{Discussion}: The precise nature of the vendor tests have not all
been determined at the time of writing of the present version of this
ICD. Measurements of the as-built filter passbands are of particular
interest to Data Management.
\end{minipage}
\\ \cdashline{1-2}
Requirement Priority &  \\ \cdashline{1-2}
Upper Level Requirement &
\begin{tabular}{cl}
\end{tabular}
\\ \hline
\end{longtable}
}


  
 \newpage 
\subsection{[LVV-6297] CA-DM-SUP-ICD-0013-V-04: Filter and Lens Vendor Test Results\_DM\_4 }\label{lvv-6297}

\begin{longtable}{cccc}
\hline
\textbf{Jira Link} & \textbf{Assignee} & \textbf{Status} & \textbf{Test Cases}\\ \hline
\href{https://jira.lsstcorp.org/browse/LVV-6297}{LVV-6297} &
Leanne Guy & Not Covered &
\begin{tabular}{c}
\end{tabular}
\\
\hline
\end{longtable}

\textbf{Verification Element Description:} \\
Undefined

{\footnotesize
\begin{longtable}{p{2.5cm}p{13.5cm}}
\hline
\multicolumn{2}{c}{\textbf{Requirement Details}}\\ \hline
Requirement ID & CA-DM-SUP-ICD-0013 \\ \cdashline{1-2}
Requirement Description &
\begin{minipage}[]{13cm}
\textbf{Specification}: The Camera shall provide the quantitative
results of optical tests performed by the filter and lens vendors.
\end{minipage}
\\ \cdashline{1-2}
Requirement Discussion &
\begin{minipage}[]{13cm}
\textbf{Discussion}: The precise nature of the vendor tests have not all
been determined at the time of writing of the present version of this
ICD. Measurements of the as-built filter passbands are of particular
interest to Data Management.
\end{minipage}
\\ \cdashline{1-2}
Requirement Priority &  \\ \cdashline{1-2}
Upper Level Requirement &
\begin{tabular}{cl}
\end{tabular}
\\ \hline
\end{longtable}
}


  
 \newpage 
\subsection{[LVV-6302] CA-DM-SUP-ICD-0011-V-03: Quantitative Test Results\_DM\_3 }\label{lvv-6302}

\begin{longtable}{cccc}
\hline
\textbf{Jira Link} & \textbf{Assignee} & \textbf{Status} & \textbf{Test Cases}\\ \hline
\href{https://jira.lsstcorp.org/browse/LVV-6302}{LVV-6302} &
Leanne Guy & Not Covered &
\begin{tabular}{c}
\end{tabular}
\\
\hline
\end{longtable}

\textbf{Verification Element Description:} \\
Undefined

{\footnotesize
\begin{longtable}{p{2.5cm}p{13.5cm}}
\hline
\multicolumn{2}{c}{\textbf{Requirement Details}}\\ \hline
Requirement ID & CA-DM-SUP-ICD-0011 \\ \cdashline{1-2}
Requirement Description &
\begin{minipage}[]{13cm}
\textbf{Specification:} The Camera shall provide the quantitative
results of the tests performed under the a) LSST Sensor Electro-Optical
Test Plan, LCA-10103; b) science raft test plan (document TBD); corner
raft test plan (document TBD); and d) Camera Verification Test Plan,
LCA-283.
\end{minipage}
\\ \cdashline{1-2}
Requirement Priority &  \\ \cdashline{1-2}
Upper Level Requirement &
\begin{tabular}{cl}
\end{tabular}
\\ \hline
\end{longtable}
}


  
 \newpage 
\subsection{[LVV-6303] CA-DM-SUP-ICD-0011-V-04: Quantitative Test Results\_DM\_4 }\label{lvv-6303}

\begin{longtable}{cccc}
\hline
\textbf{Jira Link} & \textbf{Assignee} & \textbf{Status} & \textbf{Test Cases}\\ \hline
\href{https://jira.lsstcorp.org/browse/LVV-6303}{LVV-6303} &
Leanne Guy & Not Covered &
\begin{tabular}{c}
\end{tabular}
\\
\hline
\end{longtable}

\textbf{Verification Element Description:} \\
Undefined

{\footnotesize
\begin{longtable}{p{2.5cm}p{13.5cm}}
\hline
\multicolumn{2}{c}{\textbf{Requirement Details}}\\ \hline
Requirement ID & CA-DM-SUP-ICD-0011 \\ \cdashline{1-2}
Requirement Description &
\begin{minipage}[]{13cm}
\textbf{Specification:} The Camera shall provide the quantitative
results of the tests performed under the a) LSST Sensor Electro-Optical
Test Plan, LCA-10103; b) science raft test plan (document TBD); corner
raft test plan (document TBD); and d) Camera Verification Test Plan,
LCA-283.
\end{minipage}
\\ \cdashline{1-2}
Requirement Priority &  \\ \cdashline{1-2}
Upper Level Requirement &
\begin{tabular}{cl}
\end{tabular}
\\ \hline
\end{longtable}
}


  
 \newpage 
\subsection{[LVV-6308] CA-DM-SUP-ICD-0012-V-03: Temperature, Pressure, Physical Orientation
Measurements\_DM\_3 }\label{lvv-6308}

\begin{longtable}{cccc}
\hline
\textbf{Jira Link} & \textbf{Assignee} & \textbf{Status} & \textbf{Test Cases}\\ \hline
\href{https://jira.lsstcorp.org/browse/LVV-6308}{LVV-6308} &
Leanne Guy & Not Covered &
\begin{tabular}{c}
\end{tabular}
\\
\hline
\end{longtable}

\textbf{Verification Element Description:} \\
Undefined

{\footnotesize
\begin{longtable}{p{2.5cm}p{13.5cm}}
\hline
\multicolumn{2}{c}{\textbf{Requirement Details}}\\ \hline
Requirement ID & CA-DM-SUP-ICD-0012 \\ \cdashline{1-2}
Requirement Description &
\begin{minipage}[]{13cm}
\textbf{Specification:} The Camera shall provide the ambient, sensor,
and electronics temperatures and the physical orientation of the test
article measured at the time of all tests whose results are provided
under this section. The Camera shall provide the ambient atmospheric
pressure measured at the time of all optical tests whose results are
provided under this section.
\end{minipage}
\\ \cdashline{1-2}
Requirement Priority &  \\ \cdashline{1-2}
Upper Level Requirement &
\begin{tabular}{cl}
\end{tabular}
\\ \hline
\end{longtable}
}


  
 \newpage 
\subsection{[LVV-6309] CA-DM-SUP-ICD-0012-V-04: Temperature, Pressure, Physical Orientation
Measurements\_DM\_4 }\label{lvv-6309}

\begin{longtable}{cccc}
\hline
\textbf{Jira Link} & \textbf{Assignee} & \textbf{Status} & \textbf{Test Cases}\\ \hline
\href{https://jira.lsstcorp.org/browse/LVV-6309}{LVV-6309} &
Leanne Guy & Not Covered &
\begin{tabular}{c}
\end{tabular}
\\
\hline
\end{longtable}

\textbf{Verification Element Description:} \\
Undefined

{\footnotesize
\begin{longtable}{p{2.5cm}p{13.5cm}}
\hline
\multicolumn{2}{c}{\textbf{Requirement Details}}\\ \hline
Requirement ID & CA-DM-SUP-ICD-0012 \\ \cdashline{1-2}
Requirement Description &
\begin{minipage}[]{13cm}
\textbf{Specification:} The Camera shall provide the ambient, sensor,
and electronics temperatures and the physical orientation of the test
article measured at the time of all tests whose results are provided
under this section. The Camera shall provide the ambient atmospheric
pressure measured at the time of all optical tests whose results are
provided under this section.
\end{minipage}
\\ \cdashline{1-2}
Requirement Priority &  \\ \cdashline{1-2}
Upper Level Requirement &
\begin{tabular}{cl}
\end{tabular}
\\ \hline
\end{longtable}
}


  
 \newpage 
\subsection{[LVV-6314] CA-DM-SUP-ICD-0018-V-03: Thermal Model\_DM\_3 }\label{lvv-6314}

\begin{longtable}{cccc}
\hline
\textbf{Jira Link} & \textbf{Assignee} & \textbf{Status} & \textbf{Test Cases}\\ \hline
\href{https://jira.lsstcorp.org/browse/LVV-6314}{LVV-6314} &
Leanne Guy & Not Covered &
\begin{tabular}{c}
\end{tabular}
\\
\hline
\end{longtable}

\textbf{Verification Element Description:} \\
Undefined

{\footnotesize
\begin{longtable}{p{2.5cm}p{13.5cm}}
\hline
\multicolumn{2}{c}{\textbf{Requirement Details}}\\ \hline
Requirement ID & CA-DM-SUP-ICD-0018 \\ \cdashline{1-2}
Requirement Description &
\begin{minipage}[]{13cm}
\textbf{Specification:} The Camera shall provide a thermal model for
each detector, allowing the estimation of the temperature profile across
the detector as a function of the measurement from the single
temperature sensor per detector.
\end{minipage}
\\ \cdashline{1-2}
Requirement Discussion &
\begin{minipage}[]{13cm}
\textbf{Discussion}: CAM-REQ-0103 specifies that the level of precision
is 0.5K.
\end{minipage}
\\ \cdashline{1-2}
Requirement Priority &  \\ \cdashline{1-2}
Upper Level Requirement &
\begin{tabular}{cl}
\end{tabular}
\\ \hline
\end{longtable}
}


  
 \newpage 
\subsection{[LVV-6315] CA-DM-SUP-ICD-0018-V-04: Thermal Model\_DM\_4 }\label{lvv-6315}

\begin{longtable}{cccc}
\hline
\textbf{Jira Link} & \textbf{Assignee} & \textbf{Status} & \textbf{Test Cases}\\ \hline
\href{https://jira.lsstcorp.org/browse/LVV-6315}{LVV-6315} &
Leanne Guy & Not Covered &
\begin{tabular}{c}
\end{tabular}
\\
\hline
\end{longtable}

\textbf{Verification Element Description:} \\
Undefined

{\footnotesize
\begin{longtable}{p{2.5cm}p{13.5cm}}
\hline
\multicolumn{2}{c}{\textbf{Requirement Details}}\\ \hline
Requirement ID & CA-DM-SUP-ICD-0018 \\ \cdashline{1-2}
Requirement Description &
\begin{minipage}[]{13cm}
\textbf{Specification:} The Camera shall provide a thermal model for
each detector, allowing the estimation of the temperature profile across
the detector as a function of the measurement from the single
temperature sensor per detector.
\end{minipage}
\\ \cdashline{1-2}
Requirement Discussion &
\begin{minipage}[]{13cm}
\textbf{Discussion}: CAM-REQ-0103 specifies that the level of precision
is 0.5K.
\end{minipage}
\\ \cdashline{1-2}
Requirement Priority &  \\ \cdashline{1-2}
Upper Level Requirement &
\begin{tabular}{cl}
\end{tabular}
\\ \hline
\end{longtable}
}


  
 \newpage 
\subsection{[LVV-6320] CA-DM-SUP-ICD-0001-V-03: Version Control\_DM\_3 }\label{lvv-6320}

\begin{longtable}{cccc}
\hline
\textbf{Jira Link} & \textbf{Assignee} & \textbf{Status} & \textbf{Test Cases}\\ \hline
\href{https://jira.lsstcorp.org/browse/LVV-6320}{LVV-6320} &
Leanne Guy & Not Covered &
\begin{tabular}{c}
\end{tabular}
\\
\hline
\end{longtable}

\textbf{Verification Element Description:} \\
Undefined

{\footnotesize
\begin{longtable}{p{2.5cm}p{13.5cm}}
\hline
\multicolumn{2}{c}{\textbf{Requirement Details}}\\ \hline
Requirement ID & CA-DM-SUP-ICD-0001 \\ \cdashline{1-2}
Requirement Description &
\begin{minipage}[]{13cm}
\textbf{Specification:} The camera shall provide version control for the
format of all data to be shared with Data Management.
\end{minipage}
\\ \cdashline{1-2}
Requirement Priority &  \\ \cdashline{1-2}
Upper Level Requirement &
\begin{tabular}{cl}
\end{tabular}
\\ \hline
\end{longtable}
}


  
 \newpage 
\subsection{[LVV-6321] CA-DM-SUP-ICD-0001-V-04: Version Control\_DM\_4 }\label{lvv-6321}

\begin{longtable}{cccc}
\hline
\textbf{Jira Link} & \textbf{Assignee} & \textbf{Status} & \textbf{Test Cases}\\ \hline
\href{https://jira.lsstcorp.org/browse/LVV-6321}{LVV-6321} &
Leanne Guy & Not Covered &
\begin{tabular}{c}
\end{tabular}
\\
\hline
\end{longtable}

\textbf{Verification Element Description:} \\
Undefined

{\footnotesize
\begin{longtable}{p{2.5cm}p{13.5cm}}
\hline
\multicolumn{2}{c}{\textbf{Requirement Details}}\\ \hline
Requirement ID & CA-DM-SUP-ICD-0001 \\ \cdashline{1-2}
Requirement Description &
\begin{minipage}[]{13cm}
\textbf{Specification:} The camera shall provide version control for the
format of all data to be shared with Data Management.
\end{minipage}
\\ \cdashline{1-2}
Requirement Priority &  \\ \cdashline{1-2}
Upper Level Requirement &
\begin{tabular}{cl}
\end{tabular}
\\ \hline
\end{longtable}
}


  
 \newpage 
\subsection{[LVV-6324] EP-DM-CON-ICD-0004-V-01: DM Transfer of Catalog Tabular Data to
EPO\_DM\_1 }\label{lvv-6324}

\begin{longtable}{cccc}
\hline
\textbf{Jira Link} & \textbf{Assignee} & \textbf{Status} & \textbf{Test Cases}\\ \hline
\href{https://jira.lsstcorp.org/browse/LVV-6324}{LVV-6324} &
Leanne Guy & Not Covered &
\begin{tabular}{c}
\end{tabular}
\\
\hline
\end{longtable}

\textbf{Verification Element Description:} \\
Undefined

{\footnotesize
\begin{longtable}{p{2.5cm}p{13.5cm}}
\hline
\multicolumn{2}{c}{\textbf{Requirement Details}}\\ \hline
Requirement ID & EP-DM-CON-ICD-0004 \\ \cdashline{1-2}
Requirement Description &
\begin{minipage}[]{13cm}
\textbf{Specification}: As it becomes available, Data Management shall
transfer to EPO a subset of catalog data as defined in the table below.
\end{minipage}
\\ \cdashline{1-2}
Requirement Discussion &
\begin{minipage}[]{13cm}
\textbf{Discussion}: Definition of the queries to satisfy the subset
requirement will be defined by EPO once ComCam data is available. ~The
historical notion of a single bulk data transfer from DM to EPO just
prior to each data release had a number of disadvantages and risks. ~DM
and EPO were both in favor of changing this to a ``trickle feed'' where
data was sent in small amounts over a longer period of time as part of
the \emph{solar system object processing and data release processing}
pipelines. ~This approach reduces bandwidth spikes, allows for
errors/problems to be identified sooner, and gives DM and EPO greater
flexibility. ~DM is permitted to apply additional restrictions in order
to stay within the mandatory bounds, but should notify EPO in advance.
~The values specified in this requirement represent the maximum size of
the data release products to be transferred from DM to EPO during the
last data transfer that will occur during the ten year survey (DR11),
representing the bounding case.

\begin{tabular}[]{@{}llll@{}}
\toprule
\textbf{Product} & \textbf{Table} & \textbf{Columns} & \textbf{Not to
Exceed (compressed)}\tabularnewline
Prompt & SSObject & \emph{all} & 33 gigabytes\tabularnewline
Data Release & Object & \vtop{\hbox{\strut ·~~~~~~~~
objectId}\hbox{\strut ·~~~~~~~~ bdFluxB}\hbox{\strut ·~~~~~~~~
bdFluxD}\hbox{\strut ·~~~~~~~~ bdEllip}\hbox{\strut ·~~~~~~~~
bdReB}\hbox{\strut ·~~~~~~~~ bdReD}\hbox{\strut ·~~~~~~~~
psChi2}\hbox{\strut ·~~~~~~~~ psCov}\hbox{\strut ·~~~~~~~~
psFlux}\hbox{\strut ·~~~~~~~~ psLnL}\hbox{\strut ·~~~~~~~~
psNdata}\hbox{\strut ·~~~~~~~~ psRadec}\hbox{\strut ·~~~~~~~~
psRadecTai}} & ~10 terabytes\tabularnewline
Data Release & ForcedSource & \emph{all} & 10 terabytes\tabularnewline
\bottomrule
\end{tabular}
\end{minipage}
\\ \cdashline{1-2}
Requirement Priority &  \\ \cdashline{1-2}
Upper Level Requirement &
\begin{tabular}{cl}
\end{tabular}
\\ \hline
\end{longtable}
}


  
 \newpage 
\subsection{[LVV-6325] EP-DM-CON-ICD-0004-V-02: DM Transfer of Catalog Tabular Data to
EPO\_DM\_2 }\label{lvv-6325}

\begin{longtable}{cccc}
\hline
\textbf{Jira Link} & \textbf{Assignee} & \textbf{Status} & \textbf{Test Cases}\\ \hline
\href{https://jira.lsstcorp.org/browse/LVV-6325}{LVV-6325} &
Leanne Guy & Not Covered &
\begin{tabular}{c}
\end{tabular}
\\
\hline
\end{longtable}

\textbf{Verification Element Description:} \\
Undefined

{\footnotesize
\begin{longtable}{p{2.5cm}p{13.5cm}}
\hline
\multicolumn{2}{c}{\textbf{Requirement Details}}\\ \hline
Requirement ID & EP-DM-CON-ICD-0004 \\ \cdashline{1-2}
Requirement Description &
\begin{minipage}[]{13cm}
\textbf{Specification}: As it becomes available, Data Management shall
transfer to EPO a subset of catalog data as defined in the table below.
\end{minipage}
\\ \cdashline{1-2}
Requirement Discussion &
\begin{minipage}[]{13cm}
\textbf{Discussion}: Definition of the queries to satisfy the subset
requirement will be defined by EPO once ComCam data is available. ~The
historical notion of a single bulk data transfer from DM to EPO just
prior to each data release had a number of disadvantages and risks. ~DM
and EPO were both in favor of changing this to a ``trickle feed'' where
data was sent in small amounts over a longer period of time as part of
the \emph{solar system object processing and data release processing}
pipelines. ~This approach reduces bandwidth spikes, allows for
errors/problems to be identified sooner, and gives DM and EPO greater
flexibility. ~DM is permitted to apply additional restrictions in order
to stay within the mandatory bounds, but should notify EPO in advance.
~The values specified in this requirement represent the maximum size of
the data release products to be transferred from DM to EPO during the
last data transfer that will occur during the ten year survey (DR11),
representing the bounding case.

\begin{tabular}[]{@{}llll@{}}
\toprule
\textbf{Product} & \textbf{Table} & \textbf{Columns} & \textbf{Not to
Exceed (compressed)}\tabularnewline
Prompt & SSObject & \emph{all} & 33 gigabytes\tabularnewline
Data Release & Object & \vtop{\hbox{\strut ·~~~~~~~~
objectId}\hbox{\strut ·~~~~~~~~ bdFluxB}\hbox{\strut ·~~~~~~~~
bdFluxD}\hbox{\strut ·~~~~~~~~ bdEllip}\hbox{\strut ·~~~~~~~~
bdReB}\hbox{\strut ·~~~~~~~~ bdReD}\hbox{\strut ·~~~~~~~~
psChi2}\hbox{\strut ·~~~~~~~~ psCov}\hbox{\strut ·~~~~~~~~
psFlux}\hbox{\strut ·~~~~~~~~ psLnL}\hbox{\strut ·~~~~~~~~
psNdata}\hbox{\strut ·~~~~~~~~ psRadec}\hbox{\strut ·~~~~~~~~
psRadecTai}} & ~10 terabytes\tabularnewline
Data Release & ForcedSource & \emph{all} & 10 terabytes\tabularnewline
\bottomrule
\end{tabular}
\end{minipage}
\\ \cdashline{1-2}
Requirement Priority &  \\ \cdashline{1-2}
Upper Level Requirement &
\begin{tabular}{cl}
\end{tabular}
\\ \hline
\end{longtable}
}


  
 \newpage 
\subsection{[LVV-6330] EP-DM-CON-ICD-0021-V-01: DM Generation of a Color Hierarchical
Progressive Survey for EPO\_DM\_1 }\label{lvv-6330}

\begin{longtable}{cccc}
\hline
\textbf{Jira Link} & \textbf{Assignee} & \textbf{Status} & \textbf{Test Cases}\\ \hline
\href{https://jira.lsstcorp.org/browse/LVV-6330}{LVV-6330} &
Leanne Guy & Not Covered &
\begin{tabular}{c}
\end{tabular}
\\
\hline
\end{longtable}

\textbf{Verification Element Description:} \\
Undefined

{\footnotesize
\begin{longtable}{p{2.5cm}p{13.5cm}}
\hline
\multicolumn{2}{c}{\textbf{Requirement Details}}\\ \hline
Requirement ID & EP-DM-CON-ICD-0021 \\ \cdashline{1-2}
Requirement Description &
\begin{minipage}[]{13cm}
\textbf{Specification}: As part of their co-add image processing
pipeline, DM shall create a Hierarchical Progressive Survey (HiPS) for
EPO in the form of color JPEG HEALPix tiles limited to 1 arcsecond
resolution.
\end{minipage}
\\ \cdashline{1-2}
Requirement Discussion &
\begin{minipage}[]{13cm}
\textbf{Discussion}: These tiles will be used in the EPO Portal
Skyviewer (powered by a sky atlas tool, such as Aladin Lite) and our
JupyterLab-based educational investigations (powered by a compatible
astronomy image viewer extension, such as pyaladin). ~The EPO Skyviewer
will have the same sky coverage as the full survey (southern
hemisphere), but the maximum zoom level may be different for different
regions of the sky. ~For some deep drilling fields, we may have
additional zoom levels to see extra detail. ~Definition of EPO's color
scheme, definition of the varying depth coverage, and the method for
transferring the tiles to the EDC will be defined by EPO once ComCam
data is available. ~EPO may later choose PNG if the JPEG user experience
is not satisfactory. ~Fees related to this deliverable will be paid by
EPO but we hope to leverage cost efficiencies by inserting our specific
output as part of existing NCSA data processing workflows. ~Note: these
images do not count toward the EPO world public data subset quota
because the scientific data is scrubbed and the image format is not
FITS-like.
\end{minipage}
\\ \cdashline{1-2}
Requirement Priority &  \\ \cdashline{1-2}
Upper Level Requirement &
\begin{tabular}{cl}
\end{tabular}
\\ \hline
\end{longtable}
}


  
 \newpage 
\subsection{[LVV-6331] EP-DM-CON-ICD-0021-V-02: DM Generation of a Color Hierarchical
Progressive Survey for EPO\_DM\_2 }\label{lvv-6331}

\begin{longtable}{cccc}
\hline
\textbf{Jira Link} & \textbf{Assignee} & \textbf{Status} & \textbf{Test Cases}\\ \hline
\href{https://jira.lsstcorp.org/browse/LVV-6331}{LVV-6331} &
Leanne Guy & Not Covered &
\begin{tabular}{c}
\end{tabular}
\\
\hline
\end{longtable}

\textbf{Verification Element Description:} \\
Undefined

{\footnotesize
\begin{longtable}{p{2.5cm}p{13.5cm}}
\hline
\multicolumn{2}{c}{\textbf{Requirement Details}}\\ \hline
Requirement ID & EP-DM-CON-ICD-0021 \\ \cdashline{1-2}
Requirement Description &
\begin{minipage}[]{13cm}
\textbf{Specification}: As part of their co-add image processing
pipeline, DM shall create a Hierarchical Progressive Survey (HiPS) for
EPO in the form of color JPEG HEALPix tiles limited to 1 arcsecond
resolution.
\end{minipage}
\\ \cdashline{1-2}
Requirement Discussion &
\begin{minipage}[]{13cm}
\textbf{Discussion}: These tiles will be used in the EPO Portal
Skyviewer (powered by a sky atlas tool, such as Aladin Lite) and our
JupyterLab-based educational investigations (powered by a compatible
astronomy image viewer extension, such as pyaladin). ~The EPO Skyviewer
will have the same sky coverage as the full survey (southern
hemisphere), but the maximum zoom level may be different for different
regions of the sky. ~For some deep drilling fields, we may have
additional zoom levels to see extra detail. ~Definition of EPO's color
scheme, definition of the varying depth coverage, and the method for
transferring the tiles to the EDC will be defined by EPO once ComCam
data is available. ~EPO may later choose PNG if the JPEG user experience
is not satisfactory. ~Fees related to this deliverable will be paid by
EPO but we hope to leverage cost efficiencies by inserting our specific
output as part of existing NCSA data processing workflows. ~Note: these
images do not count toward the EPO world public data subset quota
because the scientific data is scrubbed and the image format is not
FITS-like.
\end{minipage}
\\ \cdashline{1-2}
Requirement Priority &  \\ \cdashline{1-2}
Upper Level Requirement &
\begin{tabular}{cl}
\end{tabular}
\\ \hline
\end{longtable}
}


  
 \newpage 
\subsection{[LVV-6342] EP-DM-CON-ICD-0009-V-01: Catalog Format\_DM\_1 }\label{lvv-6342}

\begin{longtable}{cccc}
\hline
\textbf{Jira Link} & \textbf{Assignee} & \textbf{Status} & \textbf{Test Cases}\\ \hline
\href{https://jira.lsstcorp.org/browse/LVV-6342}{LVV-6342} &
Leanne Guy & Not Covered &
\begin{tabular}{c}
\end{tabular}
\\
\hline
\end{longtable}

\textbf{Verification Element Description:} \\
Verified by demonstration of import into EPO system.

Inspection of data content as per future document describing data format
and content.

{\footnotesize
\begin{longtable}{p{2.5cm}p{13.5cm}}
\hline
\multicolumn{2}{c}{\textbf{Requirement Details}}\\ \hline
Requirement ID & EP-DM-CON-ICD-0009 \\ \cdashline{1-2}
Requirement Description &
\begin{minipage}[]{13cm}
\textbf{Specification}: The Data Management System shall deliver catalog
data to EPO preferably in Apache Parquet format.
\end{minipage}
\\ \cdashline{1-2}
Requirement Discussion &
\begin{minipage}[]{13cm}
\textbf{Discussion}: Apache Parquet is a promising, emerging data
format. ~More testing will be required to verify it meets all of EPO's
needs. ~As a fallback option, EPO could accept catalog data in Apache
Avro or the sub-optimal but universally-supported CSV format.
\end{minipage}
\\ \cdashline{1-2}
Requirement Priority &  \\ \cdashline{1-2}
Upper Level Requirement &
\begin{tabular}{cl}
\end{tabular}
\\ \hline
\end{longtable}
}


  
 \newpage 
\subsection{[LVV-6343] EP-DM-CON-ICD-0009-V-02: Catalog Format\_DM\_2 }\label{lvv-6343}

\begin{longtable}{cccc}
\hline
\textbf{Jira Link} & \textbf{Assignee} & \textbf{Status} & \textbf{Test Cases}\\ \hline
\href{https://jira.lsstcorp.org/browse/LVV-6343}{LVV-6343} &
Leanne Guy & Not Covered &
\begin{tabular}{c}
\end{tabular}
\\
\hline
\end{longtable}

\textbf{Verification Element Description:} \\
Undefined

{\footnotesize
\begin{longtable}{p{2.5cm}p{13.5cm}}
\hline
\multicolumn{2}{c}{\textbf{Requirement Details}}\\ \hline
Requirement ID & EP-DM-CON-ICD-0009 \\ \cdashline{1-2}
Requirement Description &
\begin{minipage}[]{13cm}
\textbf{Specification}: The Data Management System shall deliver catalog
data to EPO preferably in Apache Parquet format.
\end{minipage}
\\ \cdashline{1-2}
Requirement Discussion &
\begin{minipage}[]{13cm}
\textbf{Discussion}: Apache Parquet is a promising, emerging data
format. ~More testing will be required to verify it meets all of EPO's
needs. ~As a fallback option, EPO could accept catalog data in Apache
Avro or the sub-optimal but universally-supported CSV format.
\end{minipage}
\\ \cdashline{1-2}
Requirement Priority &  \\ \cdashline{1-2}
Upper Level Requirement &
\begin{tabular}{cl}
\end{tabular}
\\ \hline
\end{longtable}
}


  
 \newpage 
\subsection{[LVV-6348] EP-DM-CON-ICD-0034-V-01: Citizen Science Data\_DM\_1 }\label{lvv-6348}

\begin{longtable}{cccc}
\hline
\textbf{Jira Link} & \textbf{Assignee} & \textbf{Status} & \textbf{Test Cases}\\ \hline
\href{https://jira.lsstcorp.org/browse/LVV-6348}{LVV-6348} &
Leanne Guy & Not Covered &
\begin{tabular}{c}
\end{tabular}
\\
\hline
\end{longtable}

\textbf{Verification Element Description:} \\
Undefined

{\footnotesize
\begin{longtable}{p{2.5cm}p{13.5cm}}
\hline
\multicolumn{2}{c}{\textbf{Requirement Details}}\\ \hline
Requirement ID & EP-DM-CON-ICD-0034 \\ \cdashline{1-2}
Requirement Description &
\begin{minipage}[]{13cm}
\textbf{Specification}: EPO shall lead the development of DM stack
community modules that provide data processing capabilities needed for
citizen science projects as well as a data transfer mechanism and data
rights review workflow.
\end{minipage}
\\ \cdashline{1-2}
Requirement Discussion &
\begin{minipage}[]{13cm}
\textbf{Discussion}: Community modules would include: ``Color Mixer'',
``Metadata Scrubber'', and ``FITS to TIFF/PNG/JPEG Converter''. ~The
data transfer mechanism (notionally referred to as a ``data funnel'')
will transfer data from the DAC to a protected S3 API-compliant object
storage bucket. ~A data rights panel will be established to verify
proper protocol is followed. ~EPO will partner with Zooniverse in this
development effort. ~More detail can be found here:
\url{https://confluence.lsstcorp.org/display/EPO/Citizen+Science}
\end{minipage}
\\ \cdashline{1-2}
Requirement Priority &  \\ \cdashline{1-2}
Upper Level Requirement &
\begin{tabular}{cl}
\end{tabular}
\\ \hline
\end{longtable}
}


  
 \newpage 
\subsection{[LVV-6349] EP-DM-CON-ICD-0034-V-02: Citizen Science Data\_DM\_2 }\label{lvv-6349}

\begin{longtable}{cccc}
\hline
\textbf{Jira Link} & \textbf{Assignee} & \textbf{Status} & \textbf{Test Cases}\\ \hline
\href{https://jira.lsstcorp.org/browse/LVV-6349}{LVV-6349} &
Leanne Guy & Not Covered &
\begin{tabular}{c}
\end{tabular}
\\
\hline
\end{longtable}

\textbf{Verification Element Description:} \\
Undefined

{\footnotesize
\begin{longtable}{p{2.5cm}p{13.5cm}}
\hline
\multicolumn{2}{c}{\textbf{Requirement Details}}\\ \hline
Requirement ID & EP-DM-CON-ICD-0034 \\ \cdashline{1-2}
Requirement Description &
\begin{minipage}[]{13cm}
\textbf{Specification}: EPO shall lead the development of DM stack
community modules that provide data processing capabilities needed for
citizen science projects as well as a data transfer mechanism and data
rights review workflow.
\end{minipage}
\\ \cdashline{1-2}
Requirement Discussion &
\begin{minipage}[]{13cm}
\textbf{Discussion}: Community modules would include: ``Color Mixer'',
``Metadata Scrubber'', and ``FITS to TIFF/PNG/JPEG Converter''. ~The
data transfer mechanism (notionally referred to as a ``data funnel'')
will transfer data from the DAC to a protected S3 API-compliant object
storage bucket. ~A data rights panel will be established to verify
proper protocol is followed. ~EPO will partner with Zooniverse in this
development effort. ~More detail can be found here:
\url{https://confluence.lsstcorp.org/display/EPO/Citizen+Science}
\end{minipage}
\\ \cdashline{1-2}
Requirement Priority &  \\ \cdashline{1-2}
Upper Level Requirement &
\begin{tabular}{cl}
\end{tabular}
\\ \hline
\end{longtable}
}


  
 \newpage 
\subsection{[LVV-6360] EP-DM-CON-ICD-0031-V-01: Data Rights Protection\_DM\_1 }\label{lvv-6360}

\begin{longtable}{cccc}
\hline
\textbf{Jira Link} & \textbf{Assignee} & \textbf{Status} & \textbf{Test Cases}\\ \hline
\href{https://jira.lsstcorp.org/browse/LVV-6360}{LVV-6360} &
Leanne Guy & Not Covered &
\begin{tabular}{c}
\end{tabular}
\\
\hline
\end{longtable}

\textbf{Verification Element Description:} \\
Undefined

{\footnotesize
\begin{longtable}{p{2.5cm}p{13.5cm}}
\hline
\multicolumn{2}{c}{\textbf{Requirement Details}}\\ \hline
Requirement ID & EP-DM-CON-ICD-0031 \\ \cdashline{1-2}
Requirement Description &
\begin{minipage}[]{13cm}
\textbf{Specification}: EPO shall not provide products, interfaces, or
services that could allow users without data rights to query, access, or
otherwise interact with an LSST Data Access Center (DAC).
\end{minipage}
\\ \cdashline{1-2}
Requirement Discussion &
\begin{minipage}[]{13cm}
\textbf{Discussion}: See \url{https://jira.lsstcorp.org/browse/LIT-97}
for further elaboration.
\end{minipage}
\\ \cdashline{1-2}
Requirement Priority &  \\ \cdashline{1-2}
Upper Level Requirement &
\begin{tabular}{cl}
\end{tabular}
\\ \hline
\end{longtable}
}


  
 \newpage 
\subsection{[LVV-6361] EP-DM-CON-ICD-0031-V-02: Data Rights Protection\_DM\_2 }\label{lvv-6361}

\begin{longtable}{cccc}
\hline
\textbf{Jira Link} & \textbf{Assignee} & \textbf{Status} & \textbf{Test Cases}\\ \hline
\href{https://jira.lsstcorp.org/browse/LVV-6361}{LVV-6361} &
Leanne Guy & Not Covered &
\begin{tabular}{c}
\end{tabular}
\\
\hline
\end{longtable}

\textbf{Verification Element Description:} \\
Undefined

{\footnotesize
\begin{longtable}{p{2.5cm}p{13.5cm}}
\hline
\multicolumn{2}{c}{\textbf{Requirement Details}}\\ \hline
Requirement ID & EP-DM-CON-ICD-0031 \\ \cdashline{1-2}
Requirement Description &
\begin{minipage}[]{13cm}
\textbf{Specification}: EPO shall not provide products, interfaces, or
services that could allow users without data rights to query, access, or
otherwise interact with an LSST Data Access Center (DAC).
\end{minipage}
\\ \cdashline{1-2}
Requirement Discussion &
\begin{minipage}[]{13cm}
\textbf{Discussion}: See \url{https://jira.lsstcorp.org/browse/LIT-97}
for further elaboration.
\end{minipage}
\\ \cdashline{1-2}
Requirement Priority &  \\ \cdashline{1-2}
Upper Level Requirement &
\begin{tabular}{cl}
\end{tabular}
\\ \hline
\end{longtable}
}


  
 \newpage 
\subsection{[LVV-6372] EP-DM-CON-ICD-0019-V-01: DM to EPO Data Transfer Cadence\_DM\_1 }\label{lvv-6372}

\begin{longtable}{cccc}
\hline
\textbf{Jira Link} & \textbf{Assignee} & \textbf{Status} & \textbf{Test Cases}\\ \hline
\href{https://jira.lsstcorp.org/browse/LVV-6372}{LVV-6372} &
Leanne Guy & Not Covered &
\begin{tabular}{c}
\end{tabular}
\\
\hline
\end{longtable}

\textbf{Verification Element Description:} \\
Demonstration that data is transferred (DM). DM needs write access to
EPO data storage. Alternatively, EPO must pull data and demonstrate it
is received.\\
Finally inspection of data content to ensure correctness.

verified as complete and accurate by the EPO Scientist within 30 days
following each major LSST data release.

{\footnotesize
\begin{longtable}{p{2.5cm}p{13.5cm}}
\hline
\multicolumn{2}{c}{\textbf{Requirement Details}}\\ \hline
Requirement ID & EP-DM-CON-ICD-0019 \\ \cdashline{1-2}
Requirement Description &
\begin{minipage}[]{13cm}
\textbf{Specification}: The cloud-based EPO Data Center (EDC) shall
receive data products from the U.S. Data Access Center (DAC) at various
frequencies.
\end{minipage}
\\ \cdashline{1-2}
Requirement Discussion &
\begin{minipage}[]{13cm}
\textbf{Discussion}: Such as: a ``trickle feed'' of DM products as
they're processed throughout the year, periodic batches of vetted
citizen science subject sets, and on-demand Science Platform queries.
\end{minipage}
\\ \cdashline{1-2}
Requirement Priority &  \\ \cdashline{1-2}
Upper Level Requirement &
\begin{tabular}{cl}
\end{tabular}
\\ \hline
\end{longtable}
}


  
 \newpage 
\subsection{[LVV-6373] EP-DM-CON-ICD-0019-V-02: DM to EPO Data Transfer Cadence\_DM\_2 }\label{lvv-6373}

\begin{longtable}{cccc}
\hline
\textbf{Jira Link} & \textbf{Assignee} & \textbf{Status} & \textbf{Test Cases}\\ \hline
\href{https://jira.lsstcorp.org/browse/LVV-6373}{LVV-6373} &
Leanne Guy & Not Covered &
\begin{tabular}{c}
\end{tabular}
\\
\hline
\end{longtable}

\textbf{Verification Element Description:} \\
Undefined

{\footnotesize
\begin{longtable}{p{2.5cm}p{13.5cm}}
\hline
\multicolumn{2}{c}{\textbf{Requirement Details}}\\ \hline
Requirement ID & EP-DM-CON-ICD-0019 \\ \cdashline{1-2}
Requirement Description &
\begin{minipage}[]{13cm}
\textbf{Specification}: The cloud-based EPO Data Center (EDC) shall
receive data products from the U.S. Data Access Center (DAC) at various
frequencies.
\end{minipage}
\\ \cdashline{1-2}
Requirement Discussion &
\begin{minipage}[]{13cm}
\textbf{Discussion}: Such as: a ``trickle feed'' of DM products as
they're processed throughout the year, periodic batches of vetted
citizen science subject sets, and on-demand Science Platform queries.
\end{minipage}
\\ \cdashline{1-2}
Requirement Priority &  \\ \cdashline{1-2}
Upper Level Requirement &
\begin{tabular}{cl}
\end{tabular}
\\ \hline
\end{longtable}
}


  
 \newpage 
\subsection{[LVV-6378] EP-DM-CON-ICD-0002-V-03: EPO is an Authorized Science User\_DM\_3 }\label{lvv-6378}

\begin{longtable}{cccc}
\hline
\textbf{Jira Link} & \textbf{Assignee} & \textbf{Status} & \textbf{Test Cases}\\ \hline
\href{https://jira.lsstcorp.org/browse/LVV-6378}{LVV-6378} &
Leanne Guy & Not Covered &
\begin{tabular}{c}
\end{tabular}
\\
\hline
\end{longtable}

\textbf{Verification Element Description:} \\
Undefined

{\footnotesize
\begin{longtable}{p{2.5cm}p{13.5cm}}
\hline
\multicolumn{2}{c}{\textbf{Requirement Details}}\\ \hline
Requirement ID & EP-DM-CON-ICD-0002 \\ \cdashline{1-2}
Requirement Description &
\begin{minipage}[]{13cm}
\textbf{Specification}: DM shall provide to EPO a single DAC account
with the same permissions as an authorized science user.
\end{minipage}
\\ \cdashline{1-2}
Requirement Discussion &
\begin{minipage}[]{13cm}
\textbf{Discussion}: ~This account will enable EPO to access the Science
Platform, make queries, process data within our Kubernetes cluster,
transfer data from the DAC to our EDC, etc. ~Note: data transfer related
to citizen science by a PI will be logged as that individual.
\end{minipage}
\\ \cdashline{1-2}
Requirement Priority &  \\ \cdashline{1-2}
Upper Level Requirement &
\begin{tabular}{cl}
\end{tabular}
\\ \hline
\end{longtable}
}


  
 \newpage 
\subsection{[LVV-6379] EP-DM-CON-ICD-0002-V-04: EPO is an Authorized Science User\_DM\_4 }\label{lvv-6379}

\begin{longtable}{cccc}
\hline
\textbf{Jira Link} & \textbf{Assignee} & \textbf{Status} & \textbf{Test Cases}\\ \hline
\href{https://jira.lsstcorp.org/browse/LVV-6379}{LVV-6379} &
Leanne Guy & Not Covered &
\begin{tabular}{c}
\end{tabular}
\\
\hline
\end{longtable}

\textbf{Verification Element Description:} \\
Undefined

{\footnotesize
\begin{longtable}{p{2.5cm}p{13.5cm}}
\hline
\multicolumn{2}{c}{\textbf{Requirement Details}}\\ \hline
Requirement ID & EP-DM-CON-ICD-0002 \\ \cdashline{1-2}
Requirement Description &
\begin{minipage}[]{13cm}
\textbf{Specification}: DM shall provide to EPO a single DAC account
with the same permissions as an authorized science user.
\end{minipage}
\\ \cdashline{1-2}
Requirement Discussion &
\begin{minipage}[]{13cm}
\textbf{Discussion}: ~This account will enable EPO to access the Science
Platform, make queries, process data within our Kubernetes cluster,
transfer data from the DAC to our EDC, etc. ~Note: data transfer related
to citizen science by a PI will be logged as that individual.
\end{minipage}
\\ \cdashline{1-2}
Requirement Priority &  \\ \cdashline{1-2}
Upper Level Requirement &
\begin{tabular}{cl}
\end{tabular}
\\ \hline
\end{longtable}
}


  
 \newpage 
\subsection{[LVV-6384] EP-DM-CON-ICD-0033-V-01: EPO Quota Management\_DM\_1 }\label{lvv-6384}

\begin{longtable}{cccc}
\hline
\textbf{Jira Link} & \textbf{Assignee} & \textbf{Status} & \textbf{Test Cases}\\ \hline
\href{https://jira.lsstcorp.org/browse/LVV-6384}{LVV-6384} &
Leanne Guy & Not Covered &
\begin{tabular}{c}
\end{tabular}
\\
\hline
\end{longtable}

\textbf{Verification Element Description:} \\
Undefined

{\footnotesize
\begin{longtable}{p{2.5cm}p{13.5cm}}
\hline
\multicolumn{2}{c}{\textbf{Requirement Details}}\\ \hline
Requirement ID & EP-DM-CON-ICD-0033 \\ \cdashline{1-2}
Requirement Description &
\begin{minipage}[]{13cm}
\textbf{Specification}: EPO shall be responsible to ensure EPO data
usage falls within the quota terms outlined in this document.
\end{minipage}
\\ \cdashline{1-2}
Requirement Discussion &
\begin{minipage}[]{13cm}
\textbf{Discussion}: DM/NCSA will not need to programmatically restrict
data usage by the EPO account accessing the DAC but can at their
discretion monitor usage. The onus is on EPO to conform to the quota
agreements stated herein. ~There may be situations where the default
scientist account quota is too low for allowed EPO usage and, upon
mutual agreement, an exception will need to be implemented by DM/NCSA.
\end{minipage}
\\ \cdashline{1-2}
Requirement Priority &  \\ \cdashline{1-2}
Upper Level Requirement &
\begin{tabular}{cl}
\end{tabular}
\\ \hline
\end{longtable}
}


  
 \newpage 
\subsection{[LVV-6385] EP-DM-CON-ICD-0033-V-02: EPO Quota Management\_DM\_2 }\label{lvv-6385}

\begin{longtable}{cccc}
\hline
\textbf{Jira Link} & \textbf{Assignee} & \textbf{Status} & \textbf{Test Cases}\\ \hline
\href{https://jira.lsstcorp.org/browse/LVV-6385}{LVV-6385} &
Leanne Guy & Not Covered &
\begin{tabular}{c}
\end{tabular}
\\
\hline
\end{longtable}

\textbf{Verification Element Description:} \\
Undefined

{\footnotesize
\begin{longtable}{p{2.5cm}p{13.5cm}}
\hline
\multicolumn{2}{c}{\textbf{Requirement Details}}\\ \hline
Requirement ID & EP-DM-CON-ICD-0033 \\ \cdashline{1-2}
Requirement Description &
\begin{minipage}[]{13cm}
\textbf{Specification}: EPO shall be responsible to ensure EPO data
usage falls within the quota terms outlined in this document.
\end{minipage}
\\ \cdashline{1-2}
Requirement Discussion &
\begin{minipage}[]{13cm}
\textbf{Discussion}: DM/NCSA will not need to programmatically restrict
data usage by the EPO account accessing the DAC but can at their
discretion monitor usage. The onus is on EPO to conform to the quota
agreements stated herein. ~There may be situations where the default
scientist account quota is too low for allowed EPO usage and, upon
mutual agreement, an exception will need to be implemented by DM/NCSA.
\end{minipage}
\\ \cdashline{1-2}
Requirement Priority &  \\ \cdashline{1-2}
Upper Level Requirement &
\begin{tabular}{cl}
\end{tabular}
\\ \hline
\end{longtable}
}


  
 \newpage 
\subsection{[LVV-6390] EP-DM-CON-ICD-0032-V-01: EPO World Public Data Subset\_DM\_1 }\label{lvv-6390}

\begin{longtable}{cccc}
\hline
\textbf{Jira Link} & \textbf{Assignee} & \textbf{Status} & \textbf{Test Cases}\\ \hline
\href{https://jira.lsstcorp.org/browse/LVV-6390}{LVV-6390} &
Leanne Guy & Not Covered &
\begin{tabular}{c}
\end{tabular}
\\
\hline
\end{longtable}

\textbf{Verification Element Description:} \\
Undefined

{\footnotesize
\begin{longtable}{p{2.5cm}p{13.5cm}}
\hline
\multicolumn{2}{c}{\textbf{Requirement Details}}\\ \hline
Requirement ID & EP-DM-CON-ICD-0032 \\ \cdashline{1-2}
Requirement Description &
\begin{minipage}[]{13cm}
\textbf{Specification}: EPO shall be able to use and distribute its data
subset publicly, without access restrictions, data rights control, or
tracking required.
\end{minipage}
\\ \cdashline{1-2}
Requirement Discussion &
\begin{minipage}[]{13cm}
\textbf{Discussion}: In short, all EPO data is world public.
\end{minipage}
\\ \cdashline{1-2}
Requirement Priority &  \\ \cdashline{1-2}
Upper Level Requirement &
\begin{tabular}{cl}
\end{tabular}
\\ \hline
\end{longtable}
}


  
 \newpage 
\subsection{[LVV-6391] EP-DM-CON-ICD-0032-V-02: EPO World Public Data Subset\_DM\_2 }\label{lvv-6391}

\begin{longtable}{cccc}
\hline
\textbf{Jira Link} & \textbf{Assignee} & \textbf{Status} & \textbf{Test Cases}\\ \hline
\href{https://jira.lsstcorp.org/browse/LVV-6391}{LVV-6391} &
Leanne Guy & Not Covered &
\begin{tabular}{c}
\end{tabular}
\\
\hline
\end{longtable}

\textbf{Verification Element Description:} \\
Undefined

{\footnotesize
\begin{longtable}{p{2.5cm}p{13.5cm}}
\hline
\multicolumn{2}{c}{\textbf{Requirement Details}}\\ \hline
Requirement ID & EP-DM-CON-ICD-0032 \\ \cdashline{1-2}
Requirement Description &
\begin{minipage}[]{13cm}
\textbf{Specification}: EPO shall be able to use and distribute its data
subset publicly, without access restrictions, data rights control, or
tracking required.
\end{minipage}
\\ \cdashline{1-2}
Requirement Discussion &
\begin{minipage}[]{13cm}
\textbf{Discussion}: In short, all EPO data is world public.
\end{minipage}
\\ \cdashline{1-2}
Requirement Priority &  \\ \cdashline{1-2}
Upper Level Requirement &
\begin{tabular}{cl}
\end{tabular}
\\ \hline
\end{longtable}
}


  
 \newpage 
\subsection{[LVV-6402] EP-DM-CON-ICD-0020-V-03: No Regulatory Issues from EPO\_DM\_3 }\label{lvv-6402}

\begin{longtable}{cccc}
\hline
\textbf{Jira Link} & \textbf{Assignee} & \textbf{Status} & \textbf{Test Cases}\\ \hline
\href{https://jira.lsstcorp.org/browse/LVV-6402}{LVV-6402} &
Leanne Guy & Not Covered &
\begin{tabular}{c}
\end{tabular}
\\
\hline
\end{longtable}

\textbf{Verification Element Description:} \\
Undefined

{\footnotesize
\begin{longtable}{p{2.5cm}p{13.5cm}}
\hline
\multicolumn{2}{c}{\textbf{Requirement Details}}\\ \hline
Requirement ID & EP-DM-CON-ICD-0020 \\ \cdashline{1-2}
Requirement Description &
\begin{minipage}[]{13cm}
\textbf{Specification}: EPO shall ensure that the DM system
(particularly the DAC) will never need to be concerned with any
regulatory issues coming from EPO or its users.
\end{minipage}
\\ \cdashline{1-2}
Requirement Discussion &
\begin{minipage}[]{13cm}
\textbf{Discussion}: ~Possible sources of regulation include the
Children's Online Privacy Protection Act (COPPA) and the Family
Educational Rights and Privacy Act (FERPA), among others. ~It is
expected that EPO will meet this requirement by not passing any
identifying information about its users to DM and by not storing any
state for identifiable users within the DM system. ~Any ``citizen
science'' results incorporated into the DM system (e.g. object
annotations or classifications) will be the responsibility of the
project's Principal Investigator (PI).
\end{minipage}
\\ \cdashline{1-2}
Requirement Priority &  \\ \cdashline{1-2}
Upper Level Requirement &
\begin{tabular}{cl}
\end{tabular}
\\ \hline
\end{longtable}
}


  
 \newpage 
\subsection{[LVV-6403] EP-DM-CON-ICD-0020-V-04: No Regulatory Issues from EPO\_DM\_4 }\label{lvv-6403}

\begin{longtable}{cccc}
\hline
\textbf{Jira Link} & \textbf{Assignee} & \textbf{Status} & \textbf{Test Cases}\\ \hline
\href{https://jira.lsstcorp.org/browse/LVV-6403}{LVV-6403} &
Leanne Guy & Not Covered &
\begin{tabular}{c}
\end{tabular}
\\
\hline
\end{longtable}

\textbf{Verification Element Description:} \\
Undefined

{\footnotesize
\begin{longtable}{p{2.5cm}p{13.5cm}}
\hline
\multicolumn{2}{c}{\textbf{Requirement Details}}\\ \hline
Requirement ID & EP-DM-CON-ICD-0020 \\ \cdashline{1-2}
Requirement Description &
\begin{minipage}[]{13cm}
\textbf{Specification}: EPO shall ensure that the DM system
(particularly the DAC) will never need to be concerned with any
regulatory issues coming from EPO or its users.
\end{minipage}
\\ \cdashline{1-2}
Requirement Discussion &
\begin{minipage}[]{13cm}
\textbf{Discussion}: ~Possible sources of regulation include the
Children's Online Privacy Protection Act (COPPA) and the Family
Educational Rights and Privacy Act (FERPA), among others. ~It is
expected that EPO will meet this requirement by not passing any
identifying information about its users to DM and by not storing any
state for identifiable users within the DM system. ~Any ``citizen
science'' results incorporated into the DM system (e.g. object
annotations or classifications) will be the responsibility of the
project's Principal Investigator (PI).
\end{minipage}
\\ \cdashline{1-2}
Requirement Priority &  \\ \cdashline{1-2}
Upper Level Requirement &
\begin{tabular}{cl}
\end{tabular}
\\ \hline
\end{longtable}
}


  
 \newpage 
\subsection{[LVV-6420] DM-TS-AUX-ICD-0020-V-01: Additional Data - Data Latency\_DM\_1 }\label{lvv-6420}

\begin{longtable}{cccc}
\hline
\textbf{Jira Link} & \textbf{Assignee} & \textbf{Status} & \textbf{Test Cases}\\ \hline
\href{https://jira.lsstcorp.org/browse/LVV-6420}{LVV-6420} &
Leanne Guy & Not Covered &
\begin{tabular}{c}
\end{tabular}
\\
\hline
\end{longtable}

\textbf{Verification Element Description:} \\
Undefined

{\footnotesize
\begin{longtable}{p{2.5cm}p{13.5cm}}
\hline
\multicolumn{2}{c}{\textbf{Requirement Details}}\\ \hline
Requirement ID & DM-TS-AUX-ICD-0020 \\ \cdashline{1-2}
Requirement Description &
\begin{minipage}[]{13cm}
\textbf{Specification:} The data in this section shall be supplied to DM
within time \textbf{additionalDataLatency} of its derivation.
\end{minipage}
\\ \cdashline{1-2}
Requirement Parameters & \textbf{daqLatency = 5{{[}second{]}}} Time to publish all-sky and
weather data. \\ \cdashline{1-2}
Requirement Discussion &
\begin{minipage}[]{13cm}
\textbf{Discussion:} This requirement is driven by the desire to make
data quality assessments available to Observatory operators in a timely
manner. It is currently a soft requirement.
\end{minipage}
\\ \cdashline{1-2}
Requirement Priority &  \\ \cdashline{1-2}
Upper Level Requirement &
\begin{tabular}{cl}
\end{tabular}
\\ \hline
\end{longtable}
}


  
 \newpage 
\subsection{[LVV-6421] DM-TS-AUX-ICD-0020-V-02: Additional Data - Data Latency\_DM\_2 }\label{lvv-6421}

\begin{longtable}{cccc}
\hline
\textbf{Jira Link} & \textbf{Assignee} & \textbf{Status} & \textbf{Test Cases}\\ \hline
\href{https://jira.lsstcorp.org/browse/LVV-6421}{LVV-6421} &
Leanne Guy & Not Covered &
\begin{tabular}{c}
\end{tabular}
\\
\hline
\end{longtable}

\textbf{Verification Element Description:} \\
Undefined

{\footnotesize
\begin{longtable}{p{2.5cm}p{13.5cm}}
\hline
\multicolumn{2}{c}{\textbf{Requirement Details}}\\ \hline
Requirement ID & DM-TS-AUX-ICD-0020 \\ \cdashline{1-2}
Requirement Description &
\begin{minipage}[]{13cm}
\textbf{Specification:} The data in this section shall be supplied to DM
within time \textbf{additionalDataLatency} of its derivation.
\end{minipage}
\\ \cdashline{1-2}
Requirement Parameters & \textbf{daqLatency = 5{{[}second{]}}} Time to publish all-sky and
weather data. \\ \cdashline{1-2}
Requirement Discussion &
\begin{minipage}[]{13cm}
\textbf{Discussion:} This requirement is driven by the desire to make
data quality assessments available to Observatory operators in a timely
manner. It is currently a soft requirement.
\end{minipage}
\\ \cdashline{1-2}
Requirement Priority &  \\ \cdashline{1-2}
Upper Level Requirement &
\begin{tabular}{cl}
\end{tabular}
\\ \hline
\end{longtable}
}


  
 \newpage 
\subsection{[LVV-6426] DM-TS-AUX-ICD-0029-V-01: Cloud Mapping\_DM\_1 }\label{lvv-6426}

\begin{longtable}{cccc}
\hline
\textbf{Jira Link} & \textbf{Assignee} & \textbf{Status} & \textbf{Test Cases}\\ \hline
\href{https://jira.lsstcorp.org/browse/LVV-6426}{LVV-6426} &
Leanne Guy & Not Covered &
\begin{tabular}{c}
\end{tabular}
\\
\hline
\end{longtable}

\textbf{Verification Element Description:} \\
Undefined

{\footnotesize
\begin{longtable}{p{2.5cm}p{13.5cm}}
\hline
\multicolumn{2}{c}{\textbf{Requirement Details}}\\ \hline
Requirement ID & DM-TS-AUX-ICD-0029 \\ \cdashline{1-2}
Requirement Description &
\begin{minipage}[]{13cm}
\textbf{Specification}: The Telescope and Site subsystem shall make
available to Data Management the cloud maps obtained under OSS-REQ-0071.
\end{minipage}
\\ \cdashline{1-2}
Requirement Discussion &
\begin{minipage}[]{13cm}
\textbf{Discussion}: The latency and data format for this information
will be determined during Phase 3 work on this ICD.
\end{minipage}
\\ \cdashline{1-2}
Requirement Priority &  \\ \cdashline{1-2}
Upper Level Requirement &
\begin{tabular}{cl}
\end{tabular}
\\ \hline
\end{longtable}
}


  
 \newpage 
\subsection{[LVV-6427] DM-TS-AUX-ICD-0029-V-02: Cloud Mapping\_DM\_2 }\label{lvv-6427}

\begin{longtable}{cccc}
\hline
\textbf{Jira Link} & \textbf{Assignee} & \textbf{Status} & \textbf{Test Cases}\\ \hline
\href{https://jira.lsstcorp.org/browse/LVV-6427}{LVV-6427} &
Leanne Guy & Not Covered &
\begin{tabular}{c}
\end{tabular}
\\
\hline
\end{longtable}

\textbf{Verification Element Description:} \\
Undefined

{\footnotesize
\begin{longtable}{p{2.5cm}p{13.5cm}}
\hline
\multicolumn{2}{c}{\textbf{Requirement Details}}\\ \hline
Requirement ID & DM-TS-AUX-ICD-0029 \\ \cdashline{1-2}
Requirement Description &
\begin{minipage}[]{13cm}
\textbf{Specification}: The Telescope and Site subsystem shall make
available to Data Management the cloud maps obtained under OSS-REQ-0071.
\end{minipage}
\\ \cdashline{1-2}
Requirement Discussion &
\begin{minipage}[]{13cm}
\textbf{Discussion}: The latency and data format for this information
will be determined during Phase 3 work on this ICD.
\end{minipage}
\\ \cdashline{1-2}
Requirement Priority &  \\ \cdashline{1-2}
Upper Level Requirement &
\begin{tabular}{cl}
\end{tabular}
\\ \hline
\end{longtable}
}


  
 \newpage 
\subsection{[LVV-6432] DM-TS-AUX-ICD-0027-V-01: DIMM Instrument\_DM\_1 }\label{lvv-6432}

\begin{longtable}{cccc}
\hline
\textbf{Jira Link} & \textbf{Assignee} & \textbf{Status} & \textbf{Test Cases}\\ \hline
\href{https://jira.lsstcorp.org/browse/LVV-6432}{LVV-6432} &
Leanne Guy & Not Covered &
\begin{tabular}{c}
\end{tabular}
\\
\hline
\end{longtable}

\textbf{Verification Element Description:} \\
Undefined

{\footnotesize
\begin{longtable}{p{2.5cm}p{13.5cm}}
\hline
\multicolumn{2}{c}{\textbf{Requirement Details}}\\ \hline
Requirement ID & DM-TS-AUX-ICD-0027 \\ \cdashline{1-2}
Requirement Description &
\begin{minipage}[]{13cm}
\textbf{Specification:} The Telescope and Site subsystem shall make
available to DM the seeing data derived from the DIMM instrument.
\end{minipage}
\\ \cdashline{1-2}
Requirement Discussion &
\begin{minipage}[]{13cm}
\textbf{Discussion:} It is anticipated that an existing instrument will
be used, including its associated analysis software. The details of the
data transport required are TBD. DM will compare the seeing data from
the DIMM instrument to the image quality assessment it performs based on
the science image data. The results are of interest to the Observatory
operator.
\end{minipage}
\\ \cdashline{1-2}
Requirement Priority &  \\ \cdashline{1-2}
Upper Level Requirement &
\begin{tabular}{cl}
\end{tabular}
\\ \hline
\end{longtable}
}


  
 \newpage 
\subsection{[LVV-6433] DM-TS-AUX-ICD-0027-V-02: DIMM Instrument\_DM\_2 }\label{lvv-6433}

\begin{longtable}{cccc}
\hline
\textbf{Jira Link} & \textbf{Assignee} & \textbf{Status} & \textbf{Test Cases}\\ \hline
\href{https://jira.lsstcorp.org/browse/LVV-6433}{LVV-6433} &
Leanne Guy & Not Covered &
\begin{tabular}{c}
\end{tabular}
\\
\hline
\end{longtable}

\textbf{Verification Element Description:} \\
Undefined

{\footnotesize
\begin{longtable}{p{2.5cm}p{13.5cm}}
\hline
\multicolumn{2}{c}{\textbf{Requirement Details}}\\ \hline
Requirement ID & DM-TS-AUX-ICD-0027 \\ \cdashline{1-2}
Requirement Description &
\begin{minipage}[]{13cm}
\textbf{Specification:} The Telescope and Site subsystem shall make
available to DM the seeing data derived from the DIMM instrument.
\end{minipage}
\\ \cdashline{1-2}
Requirement Discussion &
\begin{minipage}[]{13cm}
\textbf{Discussion:} It is anticipated that an existing instrument will
be used, including its associated analysis software. The details of the
data transport required are TBD. DM will compare the seeing data from
the DIMM instrument to the image quality assessment it performs based on
the science image data. The results are of interest to the Observatory
operator.
\end{minipage}
\\ \cdashline{1-2}
Requirement Priority &  \\ \cdashline{1-2}
Upper Level Requirement &
\begin{tabular}{cl}
\end{tabular}
\\ \hline
\end{longtable}
}


  
 \newpage 
\subsection{[LVV-6456] DM-TS-AUX-ICD-0025-V-01: Visible-light All-Sky Camera Data
Transport\_DM\_1 }\label{lvv-6456}

\begin{longtable}{cccc}
\hline
\textbf{Jira Link} & \textbf{Assignee} & \textbf{Status} & \textbf{Test Cases}\\ \hline
\href{https://jira.lsstcorp.org/browse/LVV-6456}{LVV-6456} &
Leanne Guy & Not Covered &
\begin{tabular}{c}
\end{tabular}
\\
\hline
\end{longtable}

\textbf{Verification Element Description:} \\
Undefined

{\footnotesize
\begin{longtable}{p{2.5cm}p{13.5cm}}
\hline
\multicolumn{2}{c}{\textbf{Requirement Details}}\\ \hline
Requirement ID & DM-TS-AUX-ICD-0025 \\ \cdashline{1-2}
Requirement Description &
\begin{minipage}[]{13cm}
\textbf{Specification:} The Telescope and Site subsystem shall publish
the visible-light all-sky camera images by means of the
large-binary-data interface of the Engineering and Facilities Database
(EFD).
\end{minipage}
\\ \cdashline{1-2}
Requirement Priority &  \\ \cdashline{1-2}
Upper Level Requirement &
\begin{tabular}{cl}
\end{tabular}
\\ \hline
\end{longtable}
}


  
 \newpage 
\subsection{[LVV-6457] DM-TS-AUX-ICD-0025-V-02: Visible-light All-Sky Camera Data
Transport\_DM\_2 }\label{lvv-6457}

\begin{longtable}{cccc}
\hline
\textbf{Jira Link} & \textbf{Assignee} & \textbf{Status} & \textbf{Test Cases}\\ \hline
\href{https://jira.lsstcorp.org/browse/LVV-6457}{LVV-6457} &
Leanne Guy & Not Covered &
\begin{tabular}{c}
\end{tabular}
\\
\hline
\end{longtable}

\textbf{Verification Element Description:} \\
Undefined

{\footnotesize
\begin{longtable}{p{2.5cm}p{13.5cm}}
\hline
\multicolumn{2}{c}{\textbf{Requirement Details}}\\ \hline
Requirement ID & DM-TS-AUX-ICD-0025 \\ \cdashline{1-2}
Requirement Description &
\begin{minipage}[]{13cm}
\textbf{Specification:} The Telescope and Site subsystem shall publish
the visible-light all-sky camera images by means of the
large-binary-data interface of the Engineering and Facilities Database
(EFD).
\end{minipage}
\\ \cdashline{1-2}
Requirement Priority &  \\ \cdashline{1-2}
Upper Level Requirement &
\begin{tabular}{cl}
\end{tabular}
\\ \hline
\end{longtable}
}


  
 \newpage 
\subsection{[LVV-6462] DM-TS-AUX-ICD-0026-V-01: Visible-Light All-Sky Camera Exposure
Data\_DM\_1 }\label{lvv-6462}

\begin{longtable}{cccc}
\hline
\textbf{Jira Link} & \textbf{Assignee} & \textbf{Status} & \textbf{Test Cases}\\ \hline
\href{https://jira.lsstcorp.org/browse/LVV-6462}{LVV-6462} &
Leanne Guy & Not Covered &
\begin{tabular}{c}
\end{tabular}
\\
\hline
\end{longtable}

\textbf{Verification Element Description:} \\
Undefined

{\footnotesize
\begin{longtable}{p{2.5cm}p{13.5cm}}
\hline
\multicolumn{2}{c}{\textbf{Requirement Details}}\\ \hline
Requirement ID & DM-TS-AUX-ICD-0026 \\ \cdashline{1-2}
Requirement Description &
\begin{minipage}[]{13cm}
\textbf{Specification:} The Telescope and Site subsystem shall publish
as telemetry the absolute time interval over which each exposure was
obtained, as well as any other configurable parameters of each exposure.
\end{minipage}
\\ \cdashline{1-2}
Requirement Priority &  \\ \cdashline{1-2}
Upper Level Requirement &
\begin{tabular}{cl}
\end{tabular}
\\ \hline
\end{longtable}
}


  
 \newpage 
\subsection{[LVV-6463] DM-TS-AUX-ICD-0026-V-02: Visible-Light All-Sky Camera Exposure
Data\_DM\_2 }\label{lvv-6463}

\begin{longtable}{cccc}
\hline
\textbf{Jira Link} & \textbf{Assignee} & \textbf{Status} & \textbf{Test Cases}\\ \hline
\href{https://jira.lsstcorp.org/browse/LVV-6463}{LVV-6463} &
Leanne Guy & Not Covered &
\begin{tabular}{c}
\end{tabular}
\\
\hline
\end{longtable}

\textbf{Verification Element Description:} \\
Undefined

{\footnotesize
\begin{longtable}{p{2.5cm}p{13.5cm}}
\hline
\multicolumn{2}{c}{\textbf{Requirement Details}}\\ \hline
Requirement ID & DM-TS-AUX-ICD-0026 \\ \cdashline{1-2}
Requirement Description &
\begin{minipage}[]{13cm}
\textbf{Specification:} The Telescope and Site subsystem shall publish
as telemetry the absolute time interval over which each exposure was
obtained, as well as any other configurable parameters of each exposure.
\end{minipage}
\\ \cdashline{1-2}
Requirement Priority &  \\ \cdashline{1-2}
Upper Level Requirement &
\begin{tabular}{cl}
\end{tabular}
\\ \hline
\end{longtable}
}


  
 \newpage 
\subsection{[LVV-6468] DM-TS-AUX-ICD-0024-V-01: Visible-light All-Sky Camera\_DM\_1 }\label{lvv-6468}

\begin{longtable}{cccc}
\hline
\textbf{Jira Link} & \textbf{Assignee} & \textbf{Status} & \textbf{Test Cases}\\ \hline
\href{https://jira.lsstcorp.org/browse/LVV-6468}{LVV-6468} &
Leanne Guy & Not Covered &
\begin{tabular}{c}
\end{tabular}
\\
\hline
\end{longtable}

\textbf{Verification Element Description:} \\
Undefined

{\footnotesize
\begin{longtable}{p{2.5cm}p{13.5cm}}
\hline
\multicolumn{2}{c}{\textbf{Requirement Details}}\\ \hline
Requirement ID & DM-TS-AUX-ICD-0024 \\ \cdashline{1-2}
Requirement Description &
\begin{minipage}[]{13cm}
\textbf{Specification:} The Telescope and Site subsystem shall make the
data from the visible-light all-sky camera available to DM.
\end{minipage}
\\ \cdashline{1-2}
Requirement Priority &  \\ \cdashline{1-2}
Upper Level Requirement &
\begin{tabular}{cl}
\end{tabular}
\\ \hline
\end{longtable}
}


  
 \newpage 
\subsection{[LVV-6469] DM-TS-AUX-ICD-0024-V-02: Visible-light All-Sky Camera\_DM\_2 }\label{lvv-6469}

\begin{longtable}{cccc}
\hline
\textbf{Jira Link} & \textbf{Assignee} & \textbf{Status} & \textbf{Test Cases}\\ \hline
\href{https://jira.lsstcorp.org/browse/LVV-6469}{LVV-6469} &
Leanne Guy & Not Covered &
\begin{tabular}{c}
\end{tabular}
\\
\hline
\end{longtable}

\textbf{Verification Element Description:} \\
Undefined

{\footnotesize
\begin{longtable}{p{2.5cm}p{13.5cm}}
\hline
\multicolumn{2}{c}{\textbf{Requirement Details}}\\ \hline
Requirement ID & DM-TS-AUX-ICD-0024 \\ \cdashline{1-2}
Requirement Description &
\begin{minipage}[]{13cm}
\textbf{Specification:} The Telescope and Site subsystem shall make the
data from the visible-light all-sky camera available to DM.
\end{minipage}
\\ \cdashline{1-2}
Requirement Priority &  \\ \cdashline{1-2}
Upper Level Requirement &
\begin{tabular}{cl}
\end{tabular}
\\ \hline
\end{longtable}
}


  
 \newpage 
\subsection{[LVV-6474] DM-TS-AUX-ICD-0037-V-01: Weather Data\_DM\_1 }\label{lvv-6474}

\begin{longtable}{cccc}
\hline
\textbf{Jira Link} & \textbf{Assignee} & \textbf{Status} & \textbf{Test Cases}\\ \hline
\href{https://jira.lsstcorp.org/browse/LVV-6474}{LVV-6474} &
Leanne Guy & Not Covered &
\begin{tabular}{c}
\end{tabular}
\\
\hline
\end{longtable}

\textbf{Verification Element Description:} \\
Undefined

{\footnotesize
\begin{longtable}{p{2.5cm}p{13.5cm}}
\hline
\multicolumn{2}{c}{\textbf{Requirement Details}}\\ \hline
Requirement ID & DM-TS-AUX-ICD-0037 \\ \cdashline{1-2}
Requirement Description &
\begin{minipage}[]{13cm}
\textbf{Specification:} The Telescope and Site subsystem shall make data
acquired by the weather station available to DM.
\end{minipage}
\\ \cdashline{1-2}
Requirement Discussion &
\begin{minipage}[]{13cm}
\textbf{Discussion:} Data will include wind speed, wind direction,
relative humidity, temperature and dew point.
\end{minipage}
\\ \cdashline{1-2}
Requirement Priority &  \\ \cdashline{1-2}
Upper Level Requirement &
\begin{tabular}{cl}
\end{tabular}
\\ \hline
\end{longtable}
}


  
 \newpage 
\subsection{[LVV-6475] DM-TS-AUX-ICD-0037-V-02: Weather Data\_DM\_2 }\label{lvv-6475}

\begin{longtable}{cccc}
\hline
\textbf{Jira Link} & \textbf{Assignee} & \textbf{Status} & \textbf{Test Cases}\\ \hline
\href{https://jira.lsstcorp.org/browse/LVV-6475}{LVV-6475} &
Leanne Guy & Not Covered &
\begin{tabular}{c}
\end{tabular}
\\
\hline
\end{longtable}

\textbf{Verification Element Description:} \\
Undefined

{\footnotesize
\begin{longtable}{p{2.5cm}p{13.5cm}}
\hline
\multicolumn{2}{c}{\textbf{Requirement Details}}\\ \hline
Requirement ID & DM-TS-AUX-ICD-0037 \\ \cdashline{1-2}
Requirement Description &
\begin{minipage}[]{13cm}
\textbf{Specification:} The Telescope and Site subsystem shall make data
acquired by the weather station available to DM.
\end{minipage}
\\ \cdashline{1-2}
Requirement Discussion &
\begin{minipage}[]{13cm}
\textbf{Discussion:} Data will include wind speed, wind direction,
relative humidity, temperature and dew point.
\end{minipage}
\\ \cdashline{1-2}
Requirement Priority &  \\ \cdashline{1-2}
Upper Level Requirement &
\begin{tabular}{cl}
\end{tabular}
\\ \hline
\end{longtable}
}


  
 \newpage 
\subsection{[LVV-6480] DM-TS-AUX-ICD-0002-V-01: Use of OCS Telemetry as Default Data
Transport\_DM\_1 }\label{lvv-6480}

\begin{longtable}{cccc}
\hline
\textbf{Jira Link} & \textbf{Assignee} & \textbf{Status} & \textbf{Test Cases}\\ \hline
\href{https://jira.lsstcorp.org/browse/LVV-6480}{LVV-6480} &
Leanne Guy & Not Covered &
\begin{tabular}{c}
\end{tabular}
\\
\hline
\end{longtable}

\textbf{Verification Element Description:} \\
Undefined

{\footnotesize
\begin{longtable}{p{2.5cm}p{13.5cm}}
\hline
\multicolumn{2}{c}{\textbf{Requirement Details}}\\ \hline
Requirement ID & DM-TS-AUX-ICD-0002 \\ \cdashline{1-2}
Requirement Description &
\begin{minipage}[]{13cm}
\textbf{Specification:} Unless otherwise specified, all operational data
transfers under this ICD shall be represented as OCS telemetry topics,
following the specifications set forth in \citeds{LSE-70} and subordinate
documents.
\end{minipage}
\\ \cdashline{1-2}
Requirement Discussion &
\begin{minipage}[]{13cm}
\textbf{Discussion:} The exceptions, as noted below, are the transfer of
the raw 2D images from the auxiliary telescope spectrograph, from the
all-sky cameras, and from the illumination reference spectrograph, as
well as the reduced spectra from the illumination reference
spectrograph.
\end{minipage}
\\ \cdashline{1-2}
Requirement Priority &  \\ \cdashline{1-2}
Upper Level Requirement &
\begin{tabular}{cl}
\end{tabular}
\\ \hline
\end{longtable}
}


  
 \newpage 
\subsection{[LVV-6481] DM-TS-AUX-ICD-0002-V-02: Use of OCS Telemetry as Default Data
Transport\_DM\_2 }\label{lvv-6481}

\begin{longtable}{cccc}
\hline
\textbf{Jira Link} & \textbf{Assignee} & \textbf{Status} & \textbf{Test Cases}\\ \hline
\href{https://jira.lsstcorp.org/browse/LVV-6481}{LVV-6481} &
Leanne Guy & Not Covered &
\begin{tabular}{c}
\end{tabular}
\\
\hline
\end{longtable}

\textbf{Verification Element Description:} \\
Undefined

{\footnotesize
\begin{longtable}{p{2.5cm}p{13.5cm}}
\hline
\multicolumn{2}{c}{\textbf{Requirement Details}}\\ \hline
Requirement ID & DM-TS-AUX-ICD-0002 \\ \cdashline{1-2}
Requirement Description &
\begin{minipage}[]{13cm}
\textbf{Specification:} Unless otherwise specified, all operational data
transfers under this ICD shall be represented as OCS telemetry topics,
following the specifications set forth in \citeds{LSE-70} and subordinate
documents.
\end{minipage}
\\ \cdashline{1-2}
Requirement Discussion &
\begin{minipage}[]{13cm}
\textbf{Discussion:} The exceptions, as noted below, are the transfer of
the raw 2D images from the auxiliary telescope spectrograph, from the
all-sky cameras, and from the illumination reference spectrograph, as
well as the reduced spectra from the illumination reference
spectrograph.
\end{minipage}
\\ \cdashline{1-2}
Requirement Priority &  \\ \cdashline{1-2}
Upper Level Requirement &
\begin{tabular}{cl}
\end{tabular}
\\ \hline
\end{longtable}
}


  
 \newpage 
\subsection{[LVV-6486] DM-TS-AUX-ICD-0001-V-01: Use of the OCS for Data Transport\_DM\_1 }\label{lvv-6486}

\begin{longtable}{cccc}
\hline
\textbf{Jira Link} & \textbf{Assignee} & \textbf{Status} & \textbf{Test Cases}\\ \hline
\href{https://jira.lsstcorp.org/browse/LVV-6486}{LVV-6486} &
Leanne Guy & Not Covered &
\begin{tabular}{c}
\end{tabular}
\\
\hline
\end{longtable}

\textbf{Verification Element Description:} \\
Undefined

{\footnotesize
\begin{longtable}{p{2.5cm}p{13.5cm}}
\hline
\multicolumn{2}{c}{\textbf{Requirement Details}}\\ \hline
Requirement ID & DM-TS-AUX-ICD-0001 \\ \cdashline{1-2}
Requirement Description &
\begin{minipage}[]{13cm}
\textbf{Specification:} The Observatory Control System middleware, as
defined in document \citeds{LSE-70}, shall be used to mediate all operational
data transfers covered under this ICD except for Auxiliary Telescope
Spectrograph images and associated metadata.
\end{minipage}
\\ \cdashline{1-2}
Requirement Discussion &
\begin{minipage}[]{13cm}
\textbf{Discussion:} No direct data interfaces between DM and the
auxiliary instrumentation are envisioned. The restriction to
``operational'' allows for one-time data transfers, e.g., of fundamental
specifications of the auxiliary instrumentation, to be carried out by
other means.
\end{minipage}
\\ \cdashline{1-2}
Requirement Priority &  \\ \cdashline{1-2}
Upper Level Requirement &
\begin{tabular}{cl}
\end{tabular}
\\ \hline
\end{longtable}
}


  
 \newpage 
\subsection{[LVV-6487] DM-TS-AUX-ICD-0001-V-02: Use of the OCS for Data Transport\_DM\_2 }\label{lvv-6487}

\begin{longtable}{cccc}
\hline
\textbf{Jira Link} & \textbf{Assignee} & \textbf{Status} & \textbf{Test Cases}\\ \hline
\href{https://jira.lsstcorp.org/browse/LVV-6487}{LVV-6487} &
Leanne Guy & Not Covered &
\begin{tabular}{c}
\end{tabular}
\\
\hline
\end{longtable}

\textbf{Verification Element Description:} \\
Undefined

{\footnotesize
\begin{longtable}{p{2.5cm}p{13.5cm}}
\hline
\multicolumn{2}{c}{\textbf{Requirement Details}}\\ \hline
Requirement ID & DM-TS-AUX-ICD-0001 \\ \cdashline{1-2}
Requirement Description &
\begin{minipage}[]{13cm}
\textbf{Specification:} The Observatory Control System middleware, as
defined in document \citeds{LSE-70}, shall be used to mediate all operational
data transfers covered under this ICD except for Auxiliary Telescope
Spectrograph images and associated metadata.
\end{minipage}
\\ \cdashline{1-2}
Requirement Discussion &
\begin{minipage}[]{13cm}
\textbf{Discussion:} No direct data interfaces between DM and the
auxiliary instrumentation are envisioned. The restriction to
``operational'' allows for one-time data transfers, e.g., of fundamental
specifications of the auxiliary instrumentation, to be carried out by
other means.
\end{minipage}
\\ \cdashline{1-2}
Requirement Priority &  \\ \cdashline{1-2}
Upper Level Requirement &
\begin{tabular}{cl}
\end{tabular}
\\ \hline
\end{longtable}
}


  
 \newpage 
\subsection{[LVV-6492] DM-TS-AUX-ICD-0007-V-01: Auxiliary Telescope Exposure Data\_DM\_1 }\label{lvv-6492}

\begin{longtable}{cccc}
\hline
\textbf{Jira Link} & \textbf{Assignee} & \textbf{Status} & \textbf{Test Cases}\\ \hline
\href{https://jira.lsstcorp.org/browse/LVV-6492}{LVV-6492} &
Leanne Guy & Not Covered &
\begin{tabular}{c}
\end{tabular}
\\
\hline
\end{longtable}

\textbf{Verification Element Description:} \\
Undefined

{\footnotesize
\begin{longtable}{p{2.5cm}p{13.5cm}}
\hline
\multicolumn{2}{c}{\textbf{Requirement Details}}\\ \hline
Requirement ID & DM-TS-AUX-ICD-0007 \\ \cdashline{1-2}
Requirement Description &
\begin{minipage}[]{13cm}
\textbf{Specification:} The Telescope and Site subsystem shall publish
as events or telemetry the auxiliary telescope pointing, photodiode
currents, filter wheel settings, temperatures, and pressure, among
others. ~The Auxiliary Telescope Spectrograph camera control system
shall publish the start of integration, shutter open, shutter close,
start of readout, and end of readout events, among others, identical
with those that the Camera CCS publishes according to \citeds{LSE-69}. ~Both
systems shall use the same events as for ComCam or the LSST Camera, but
marked as for the Auxiliary Telescope. ~For each exposure, these
publications shall be made no later than the readout of the raw
spectrograph image data.
\end{minipage}
\\ \cdashline{1-2}
Requirement Discussion &
\begin{minipage}[]{13cm}
\textbf{Discussion:} It is anticipated that there will be additional
telemetry describing the state of the spectrograph, but it is impossible
to enumerate the data items at this time since the basic spectrograph
design has not been chosen.
\end{minipage}
\\ \cdashline{1-2}
Requirement Priority &  \\ \cdashline{1-2}
Upper Level Requirement &
\begin{tabular}{cl}
\end{tabular}
\\ \hline
\end{longtable}
}


  
 \newpage 
\subsection{[LVV-6493] DM-TS-AUX-ICD-0007-V-02: Auxiliary Telescope Exposure Data\_DM\_2 }\label{lvv-6493}

\begin{longtable}{cccc}
\hline
\textbf{Jira Link} & \textbf{Assignee} & \textbf{Status} & \textbf{Test Cases}\\ \hline
\href{https://jira.lsstcorp.org/browse/LVV-6493}{LVV-6493} &
Leanne Guy & Not Covered &
\begin{tabular}{c}
\end{tabular}
\\
\hline
\end{longtable}

\textbf{Verification Element Description:} \\
Undefined

{\footnotesize
\begin{longtable}{p{2.5cm}p{13.5cm}}
\hline
\multicolumn{2}{c}{\textbf{Requirement Details}}\\ \hline
Requirement ID & DM-TS-AUX-ICD-0007 \\ \cdashline{1-2}
Requirement Description &
\begin{minipage}[]{13cm}
\textbf{Specification:} The Telescope and Site subsystem shall publish
as events or telemetry the auxiliary telescope pointing, photodiode
currents, filter wheel settings, temperatures, and pressure, among
others. ~The Auxiliary Telescope Spectrograph camera control system
shall publish the start of integration, shutter open, shutter close,
start of readout, and end of readout events, among others, identical
with those that the Camera CCS publishes according to \citeds{LSE-69}. ~Both
systems shall use the same events as for ComCam or the LSST Camera, but
marked as for the Auxiliary Telescope. ~For each exposure, these
publications shall be made no later than the readout of the raw
spectrograph image data.
\end{minipage}
\\ \cdashline{1-2}
Requirement Discussion &
\begin{minipage}[]{13cm}
\textbf{Discussion:} It is anticipated that there will be additional
telemetry describing the state of the spectrograph, but it is impossible
to enumerate the data items at this time since the basic spectrograph
design has not been chosen.
\end{minipage}
\\ \cdashline{1-2}
Requirement Priority &  \\ \cdashline{1-2}
Upper Level Requirement &
\begin{tabular}{cl}
\end{tabular}
\\ \hline
\end{longtable}
}


  
 \newpage 
\subsection{[LVV-6498] DM-TS-AUX-ICD-0008-V-01: Auxiliary Telescope Spectrograph Calibration
Data\_DM\_1 }\label{lvv-6498}

\begin{longtable}{cccc}
\hline
\textbf{Jira Link} & \textbf{Assignee} & \textbf{Status} & \textbf{Test Cases}\\ \hline
\href{https://jira.lsstcorp.org/browse/LVV-6498}{LVV-6498} &
Leanne Guy & Not Covered &
\begin{tabular}{c}
\end{tabular}
\\
\hline
\end{longtable}

\textbf{Verification Element Description:} \\
Undefined

{\footnotesize
\begin{longtable}{p{2.5cm}p{13.5cm}}
\hline
\multicolumn{2}{c}{\textbf{Requirement Details}}\\ \hline
Requirement ID & DM-TS-AUX-ICD-0008 \\ \cdashline{1-2}
Requirement Description &
\begin{minipage}[]{13cm}
\textbf{Specification:} The Telescope and Site subsystem shall acquire
calibration data for the spectograph, as required and as appropriate to
its design, and shall make these available to DM. ~This data shall
include at a minimum bias frames, dark frames, flat fields, and
calibration lamp exposures.
\end{minipage}
\\ \cdashline{1-2}
Requirement Discussion &
\begin{minipage}[]{13cm}
\textbf{Discussion:} DM will perform instrument signature removal and
calibrations such as bias subtraction, flat-fielding, and wavelength
mapping of the raw image data, and will require appropriate reference
data in order to do so. It is assumed that the OCS will coordinate the
necessary commands to the auxiliary telescope and spectrograph in order
to ensure that the calibration data are collected.
\end{minipage}
\\ \cdashline{1-2}
Requirement Priority &  \\ \cdashline{1-2}
Upper Level Requirement &
\begin{tabular}{cl}
\end{tabular}
\\ \hline
\end{longtable}
}


  
 \newpage 
\subsection{[LVV-6499] DM-TS-AUX-ICD-0008-V-02: Auxiliary Telescope Spectrograph Calibration
Data\_DM\_2 }\label{lvv-6499}

\begin{longtable}{cccc}
\hline
\textbf{Jira Link} & \textbf{Assignee} & \textbf{Status} & \textbf{Test Cases}\\ \hline
\href{https://jira.lsstcorp.org/browse/LVV-6499}{LVV-6499} &
Leanne Guy & Not Covered &
\begin{tabular}{c}
\end{tabular}
\\
\hline
\end{longtable}

\textbf{Verification Element Description:} \\
Undefined

{\footnotesize
\begin{longtable}{p{2.5cm}p{13.5cm}}
\hline
\multicolumn{2}{c}{\textbf{Requirement Details}}\\ \hline
Requirement ID & DM-TS-AUX-ICD-0008 \\ \cdashline{1-2}
Requirement Description &
\begin{minipage}[]{13cm}
\textbf{Specification:} The Telescope and Site subsystem shall acquire
calibration data for the spectograph, as required and as appropriate to
its design, and shall make these available to DM. ~This data shall
include at a minimum bias frames, dark frames, flat fields, and
calibration lamp exposures.
\end{minipage}
\\ \cdashline{1-2}
Requirement Discussion &
\begin{minipage}[]{13cm}
\textbf{Discussion:} DM will perform instrument signature removal and
calibrations such as bias subtraction, flat-fielding, and wavelength
mapping of the raw image data, and will require appropriate reference
data in order to do so. It is assumed that the OCS will coordinate the
necessary commands to the auxiliary telescope and spectrograph in order
to ensure that the calibration data are collected.
\end{minipage}
\\ \cdashline{1-2}
Requirement Priority &  \\ \cdashline{1-2}
Upper Level Requirement &
\begin{tabular}{cl}
\end{tabular}
\\ \hline
\end{longtable}
}


  
 \newpage 
\subsection{[LVV-6528] DM-TS-AUX-ICD-0004-V-01: Auxiliary Telescope Spectrograph Image Data
Transport\_DM\_1 }\label{lvv-6528}

\begin{longtable}{cccc}
\hline
\textbf{Jira Link} & \textbf{Assignee} & \textbf{Status} & \textbf{Test Cases}\\ \hline
\href{https://jira.lsstcorp.org/browse/LVV-6528}{LVV-6528} &
Leanne Guy & Not Covered &
\begin{tabular}{c}
\end{tabular}
\\
\hline
\end{longtable}

\textbf{Verification Element Description:} \\
Undefined

{\footnotesize
\begin{longtable}{p{2.5cm}p{13.5cm}}
\hline
\multicolumn{2}{c}{\textbf{Requirement Details}}\\ \hline
Requirement ID & DM-TS-AUX-ICD-0004 \\ \cdashline{1-2}
Requirement Description &
\begin{minipage}[]{13cm}
\textbf{Specification:} The Telescope and Site subsystem shall publish
the raw two-dimensional image data from the spectrograph by means of the
Camera data acquisition interface specified in \citeds{LSE-68}, including all
relevant timings.
\end{minipage}
\\ \cdashline{1-2}
Requirement Discussion &
\begin{minipage}[]{13cm}
\textbf{Discussion:} See the EFD design document, \citeds{LTS-210}, for a brief
description. The specifics of the large-file interface remain to be
defined. DM will obtain the data via its routine replication of the EFD,
including its large-binary-data sector, to the Data Management Base and
Archive Centers. (The replication is to be defined in the OCS-DM ICD
\citeds{LSE-72}; it is anticipated to occur in close to real time, though it is
not a required step in the Alert Production 60-second pipeline.)
\end{minipage}
\\ \cdashline{1-2}
Requirement Priority &  \\ \cdashline{1-2}
Upper Level Requirement &
\begin{tabular}{cl}
\end{tabular}
\\ \hline
\end{longtable}
}


  
 \newpage 
\subsection{[LVV-6529] DM-TS-AUX-ICD-0004-V-02: Auxiliary Telescope Spectrograph Image Data
Transport\_DM\_2 }\label{lvv-6529}

\begin{longtable}{cccc}
\hline
\textbf{Jira Link} & \textbf{Assignee} & \textbf{Status} & \textbf{Test Cases}\\ \hline
\href{https://jira.lsstcorp.org/browse/LVV-6529}{LVV-6529} &
Leanne Guy & Not Covered &
\begin{tabular}{c}
\end{tabular}
\\
\hline
\end{longtable}

\textbf{Verification Element Description:} \\
Undefined

{\footnotesize
\begin{longtable}{p{2.5cm}p{13.5cm}}
\hline
\multicolumn{2}{c}{\textbf{Requirement Details}}\\ \hline
Requirement ID & DM-TS-AUX-ICD-0004 \\ \cdashline{1-2}
Requirement Description &
\begin{minipage}[]{13cm}
\textbf{Specification:} The Telescope and Site subsystem shall publish
the raw two-dimensional image data from the spectrograph by means of the
Camera data acquisition interface specified in \citeds{LSE-68}, including all
relevant timings.
\end{minipage}
\\ \cdashline{1-2}
Requirement Discussion &
\begin{minipage}[]{13cm}
\textbf{Discussion:} See the EFD design document, \citeds{LTS-210}, for a brief
description. The specifics of the large-file interface remain to be
defined. DM will obtain the data via its routine replication of the EFD,
including its large-binary-data sector, to the Data Management Base and
Archive Centers. (The replication is to be defined in the OCS-DM ICD
\citeds{LSE-72}; it is anticipated to occur in close to real time, though it is
not a required step in the Alert Production 60-second pipeline.)
\end{minipage}
\\ \cdashline{1-2}
Requirement Priority &  \\ \cdashline{1-2}
Upper Level Requirement &
\begin{tabular}{cl}
\end{tabular}
\\ \hline
\end{longtable}
}


  
 \newpage 
\subsection{[LVV-6534] DM-TS-AUX-ICD-0003-V-01: Auxiliary Telescope Spectrograph\_DM\_1 }\label{lvv-6534}

\begin{longtable}{cccc}
\hline
\textbf{Jira Link} & \textbf{Assignee} & \textbf{Status} & \textbf{Test Cases}\\ \hline
\href{https://jira.lsstcorp.org/browse/LVV-6534}{LVV-6534} &
Leanne Guy & Not Covered &
\begin{tabular}{c}
\end{tabular}
\\
\hline
\end{longtable}

\textbf{Verification Element Description:} \\
Undefined

{\footnotesize
\begin{longtable}{p{2.5cm}p{13.5cm}}
\hline
\multicolumn{2}{c}{\textbf{Requirement Details}}\\ \hline
Requirement ID & DM-TS-AUX-ICD-0003 \\ \cdashline{1-2}
Requirement Description &
\begin{minipage}[]{13cm}
\textbf{Specification:} The Telescope and Site subsystem shall make
available to DM the data from the auxiliary telescope spectrograph. DM
shall analyze the data and shall make certain results available to the
Telescope and Site subsystem.
\end{minipage}
\\ \cdashline{1-2}
Requirement Priority &  \\ \cdashline{1-2}
Upper Level Requirement &
\begin{tabular}{cl}
\end{tabular}
\\ \hline
\end{longtable}
}


  
 \newpage 
\subsection{[LVV-6535] DM-TS-AUX-ICD-0003-V-02: Auxiliary Telescope Spectrograph\_DM\_2 }\label{lvv-6535}

\begin{longtable}{cccc}
\hline
\textbf{Jira Link} & \textbf{Assignee} & \textbf{Status} & \textbf{Test Cases}\\ \hline
\href{https://jira.lsstcorp.org/browse/LVV-6535}{LVV-6535} &
Leanne Guy & Not Covered &
\begin{tabular}{c}
\end{tabular}
\\
\hline
\end{longtable}

\textbf{Verification Element Description:} \\
Undefined

{\footnotesize
\begin{longtable}{p{2.5cm}p{13.5cm}}
\hline
\multicolumn{2}{c}{\textbf{Requirement Details}}\\ \hline
Requirement ID & DM-TS-AUX-ICD-0003 \\ \cdashline{1-2}
Requirement Description &
\begin{minipage}[]{13cm}
\textbf{Specification:} The Telescope and Site subsystem shall make
available to DM the data from the auxiliary telescope spectrograph. DM
shall analyze the data and shall make certain results available to the
Telescope and Site subsystem.
\end{minipage}
\\ \cdashline{1-2}
Requirement Priority &  \\ \cdashline{1-2}
Upper Level Requirement &
\begin{tabular}{cl}
\end{tabular}
\\ \hline
\end{longtable}
}


  
 \newpage 
\subsection{[LVV-6540] DM-TS-AUX-ICD-0034-V-01: Calibrated photodiodes\_DM\_1 }\label{lvv-6540}

\begin{longtable}{cccc}
\hline
\textbf{Jira Link} & \textbf{Assignee} & \textbf{Status} & \textbf{Test Cases}\\ \hline
\href{https://jira.lsstcorp.org/browse/LVV-6540}{LVV-6540} &
Leanne Guy & Not Covered &
\begin{tabular}{c}
\end{tabular}
\\
\hline
\end{longtable}

\textbf{Verification Element Description:} \\
Undefined

{\footnotesize
\begin{longtable}{p{2.5cm}p{13.5cm}}
\hline
\multicolumn{2}{c}{\textbf{Requirement Details}}\\ \hline
Requirement ID & DM-TS-AUX-ICD-0034 \\ \cdashline{1-2}
Requirement Description &
\begin{minipage}[]{13cm}
\textbf{Specification:} The Telescope and Site subsystem shall acquire
data from multiple calibrated photodiodes monitoring the calibration
light sources and shall make the data available to DM.
\end{minipage}
\\ \cdashline{1-2}
Requirement Discussion &
\begin{minipage}[]{13cm}
\textbf{Discussion:} The photodiodes are expected to be read out at high
rates (typically 100Hz but up to 1kHz) that are not suitable for
individual readout to be transmitted as telemetry items. The photodiode
DAQ system is expected to aggregate time histories of the photodiode
voltage and send them at a reduced rate. Each photodiode will have a
voltage to flux conversion constant that is expected to be reported as
part of each aggregate.
\end{minipage}
\\ \cdashline{1-2}
Requirement Priority &  \\ \cdashline{1-2}
Upper Level Requirement &
\begin{tabular}{cl}
\end{tabular}
\\ \hline
\end{longtable}
}


  
 \newpage 
\subsection{[LVV-6541] DM-TS-AUX-ICD-0034-V-02: Calibrated photodiodes\_DM\_2 }\label{lvv-6541}

\begin{longtable}{cccc}
\hline
\textbf{Jira Link} & \textbf{Assignee} & \textbf{Status} & \textbf{Test Cases}\\ \hline
\href{https://jira.lsstcorp.org/browse/LVV-6541}{LVV-6541} &
Leanne Guy & Not Covered &
\begin{tabular}{c}
\end{tabular}
\\
\hline
\end{longtable}

\textbf{Verification Element Description:} \\
Undefined

{\footnotesize
\begin{longtable}{p{2.5cm}p{13.5cm}}
\hline
\multicolumn{2}{c}{\textbf{Requirement Details}}\\ \hline
Requirement ID & DM-TS-AUX-ICD-0034 \\ \cdashline{1-2}
Requirement Description &
\begin{minipage}[]{13cm}
\textbf{Specification:} The Telescope and Site subsystem shall acquire
data from multiple calibrated photodiodes monitoring the calibration
light sources and shall make the data available to DM.
\end{minipage}
\\ \cdashline{1-2}
Requirement Discussion &
\begin{minipage}[]{13cm}
\textbf{Discussion:} The photodiodes are expected to be read out at high
rates (typically 100Hz but up to 1kHz) that are not suitable for
individual readout to be transmitted as telemetry items. The photodiode
DAQ system is expected to aggregate time histories of the photodiode
voltage and send them at a reduced rate. Each photodiode will have a
voltage to flux conversion constant that is expected to be reported as
part of each aggregate.
\end{minipage}
\\ \cdashline{1-2}
Requirement Priority &  \\ \cdashline{1-2}
Upper Level Requirement &
\begin{tabular}{cl}
\end{tabular}
\\ \hline
\end{longtable}
}


  
 \newpage 
\subsection{[LVV-6546] DM-TS-AUX-ICD-0036-V-01: Collimated Beam Projector Control System\_DM\_1 }\label{lvv-6546}

\begin{longtable}{cccc}
\hline
\textbf{Jira Link} & \textbf{Assignee} & \textbf{Status} & \textbf{Test Cases}\\ \hline
\href{https://jira.lsstcorp.org/browse/LVV-6546}{LVV-6546} &
Leanne Guy & Not Covered &
\begin{tabular}{c}
\end{tabular}
\\
\hline
\end{longtable}

\textbf{Verification Element Description:} \\
Undefined

{\footnotesize
\begin{longtable}{p{2.5cm}p{13.5cm}}
\hline
\multicolumn{2}{c}{\textbf{Requirement Details}}\\ \hline
Requirement ID & DM-TS-AUX-ICD-0036 \\ \cdashline{1-2}
Requirement Description &
\begin{minipage}[]{13cm}
\textbf{Specification:} The Telescope and Site subsystem shall make
available to DM the collimated beam projector's configuration and
conditions data.
\end{minipage}
\\ \cdashline{1-2}
Requirement Discussion &
\begin{minipage}[]{13cm}
\textbf{Discussion:} The instrument configuration data expected to be
supplied are the altitude and azimuth position, mask wheel position,
mask selection, tip/tilt stage position, and focuser position. The
instrument should report both commanded set points and actual
read-backs.
\end{minipage}
\\ \cdashline{1-2}
Requirement Priority &  \\ \cdashline{1-2}
Upper Level Requirement &
\begin{tabular}{cl}
\end{tabular}
\\ \hline
\end{longtable}
}


  
 \newpage 
\subsection{[LVV-6547] DM-TS-AUX-ICD-0036-V-02: Collimated Beam Projector Control System\_DM\_2 }\label{lvv-6547}

\begin{longtable}{cccc}
\hline
\textbf{Jira Link} & \textbf{Assignee} & \textbf{Status} & \textbf{Test Cases}\\ \hline
\href{https://jira.lsstcorp.org/browse/LVV-6547}{LVV-6547} &
Leanne Guy & Not Covered &
\begin{tabular}{c}
\end{tabular}
\\
\hline
\end{longtable}

\textbf{Verification Element Description:} \\
Undefined

{\footnotesize
\begin{longtable}{p{2.5cm}p{13.5cm}}
\hline
\multicolumn{2}{c}{\textbf{Requirement Details}}\\ \hline
Requirement ID & DM-TS-AUX-ICD-0036 \\ \cdashline{1-2}
Requirement Description &
\begin{minipage}[]{13cm}
\textbf{Specification:} The Telescope and Site subsystem shall make
available to DM the collimated beam projector's configuration and
conditions data.
\end{minipage}
\\ \cdashline{1-2}
Requirement Discussion &
\begin{minipage}[]{13cm}
\textbf{Discussion:} The instrument configuration data expected to be
supplied are the altitude and azimuth position, mask wheel position,
mask selection, tip/tilt stage position, and focuser position. The
instrument should report both commanded set points and actual
read-backs.
\end{minipage}
\\ \cdashline{1-2}
Requirement Priority &  \\ \cdashline{1-2}
Upper Level Requirement &
\begin{tabular}{cl}
\end{tabular}
\\ \hline
\end{longtable}
}


  
 \newpage 
\subsection{[LVV-6552] DM-TS-AUX-ICD-0019-V-01: Dome Screen Illumination Reference System Data
Latency\_DM\_1 }\label{lvv-6552}

\begin{longtable}{cccc}
\hline
\textbf{Jira Link} & \textbf{Assignee} & \textbf{Status} & \textbf{Test Cases}\\ \hline
\href{https://jira.lsstcorp.org/browse/LVV-6552}{LVV-6552} &
Leanne Guy & Not Covered &
\begin{tabular}{c}
\end{tabular}
\\
\hline
\end{longtable}

\textbf{Verification Element Description:} \\
Undefined

{\footnotesize
\begin{longtable}{p{2.5cm}p{13.5cm}}
\hline
\multicolumn{2}{c}{\textbf{Requirement Details}}\\ \hline
Requirement ID & DM-TS-AUX-ICD-0019 \\ \cdashline{1-2}
Requirement Description &
\begin{minipage}[]{13cm}
\textbf{Specification:} Data from the dome screen illumination reference
system shall be published as telemetry as it is acquired, with
\textbf{domeScreenDataLatency} latency.
\end{minipage}
\\ \cdashline{1-2}
Requirement Parameters & \textbf{domeScreenDataLatency = 1{{[}second{]}}} Time to publish
calibration data. \\ \cdashline{1-2}
Requirement Discussion &
\begin{minipage}[]{13cm}
\textbf{Discussion:} DM must be able to incorporate this data into its
analysis of the daytime calibration image data from the Camera in time
to make use of the results in the subsequent night's observing. Since
flats may be acquired immediately preceding the start of science
observations for a night, the latency for the reference system data must
be relatively short.\\
Note that the actual requirement for the total time from the last
calibration image of a day/evening to the first science image of the
evening/night for which the corresponding reduced calibration data are
available is TBD. This will be part of the calibration implementation
plan.
\end{minipage}
\\ \cdashline{1-2}
Requirement Priority &  \\ \cdashline{1-2}
Upper Level Requirement &
\begin{tabular}{cl}
\end{tabular}
\\ \hline
\end{longtable}
}


  
 \newpage 
\subsection{[LVV-6553] DM-TS-AUX-ICD-0019-V-02: Dome Screen Illumination Reference System Data
Latency\_DM\_2 }\label{lvv-6553}

\begin{longtable}{cccc}
\hline
\textbf{Jira Link} & \textbf{Assignee} & \textbf{Status} & \textbf{Test Cases}\\ \hline
\href{https://jira.lsstcorp.org/browse/LVV-6553}{LVV-6553} &
Leanne Guy & Not Covered &
\begin{tabular}{c}
\end{tabular}
\\
\hline
\end{longtable}

\textbf{Verification Element Description:} \\
Undefined

{\footnotesize
\begin{longtable}{p{2.5cm}p{13.5cm}}
\hline
\multicolumn{2}{c}{\textbf{Requirement Details}}\\ \hline
Requirement ID & DM-TS-AUX-ICD-0019 \\ \cdashline{1-2}
Requirement Description &
\begin{minipage}[]{13cm}
\textbf{Specification:} Data from the dome screen illumination reference
system shall be published as telemetry as it is acquired, with
\textbf{domeScreenDataLatency} latency.
\end{minipage}
\\ \cdashline{1-2}
Requirement Parameters & \textbf{domeScreenDataLatency = 1{{[}second{]}}} Time to publish
calibration data. \\ \cdashline{1-2}
Requirement Discussion &
\begin{minipage}[]{13cm}
\textbf{Discussion:} DM must be able to incorporate this data into its
analysis of the daytime calibration image data from the Camera in time
to make use of the results in the subsequent night's observing. Since
flats may be acquired immediately preceding the start of science
observations for a night, the latency for the reference system data must
be relatively short.\\
Note that the actual requirement for the total time from the last
calibration image of a day/evening to the first science image of the
evening/night for which the corresponding reduced calibration data are
available is TBD. This will be part of the calibration implementation
plan.
\end{minipage}
\\ \cdashline{1-2}
Requirement Priority &  \\ \cdashline{1-2}
Upper Level Requirement &
\begin{tabular}{cl}
\end{tabular}
\\ \hline
\end{longtable}
}


  
 \newpage 
\subsection{[LVV-6558] DM-TS-AUX-ICD-0018-V-01: Dome Screen Illumination Reference
System\_DM\_1 }\label{lvv-6558}

\begin{longtable}{cccc}
\hline
\textbf{Jira Link} & \textbf{Assignee} & \textbf{Status} & \textbf{Test Cases}\\ \hline
\href{https://jira.lsstcorp.org/browse/LVV-6558}{LVV-6558} &
Leanne Guy & Not Covered &
\begin{tabular}{c}
\end{tabular}
\\
\hline
\end{longtable}

\textbf{Verification Element Description:} \\
Undefined

{\footnotesize
\begin{longtable}{p{2.5cm}p{13.5cm}}
\hline
\multicolumn{2}{c}{\textbf{Requirement Details}}\\ \hline
Requirement ID & DM-TS-AUX-ICD-0018 \\ \cdashline{1-2}
Requirement Description &
\begin{minipage}[]{13cm}
\textbf{Specification:} The Telescope and Site subsystem shall make
available to DM the data from the illumination reference system for the
dome screen whenever dome screen calibration activities are under way.
\end{minipage}
\\ \cdashline{1-2}
Requirement Priority &  \\ \cdashline{1-2}
Upper Level Requirement &
\begin{tabular}{cl}
\end{tabular}
\\ \hline
\end{longtable}
}


  
 \newpage 
\subsection{[LVV-6559] DM-TS-AUX-ICD-0018-V-02: Dome Screen Illumination Reference
System\_DM\_2 }\label{lvv-6559}

\begin{longtable}{cccc}
\hline
\textbf{Jira Link} & \textbf{Assignee} & \textbf{Status} & \textbf{Test Cases}\\ \hline
\href{https://jira.lsstcorp.org/browse/LVV-6559}{LVV-6559} &
Leanne Guy & Not Covered &
\begin{tabular}{c}
\end{tabular}
\\
\hline
\end{longtable}

\textbf{Verification Element Description:} \\
Undefined

{\footnotesize
\begin{longtable}{p{2.5cm}p{13.5cm}}
\hline
\multicolumn{2}{c}{\textbf{Requirement Details}}\\ \hline
Requirement ID & DM-TS-AUX-ICD-0018 \\ \cdashline{1-2}
Requirement Description &
\begin{minipage}[]{13cm}
\textbf{Specification:} The Telescope and Site subsystem shall make
available to DM the data from the illumination reference system for the
dome screen whenever dome screen calibration activities are under way.
\end{minipage}
\\ \cdashline{1-2}
Requirement Priority &  \\ \cdashline{1-2}
Upper Level Requirement &
\begin{tabular}{cl}
\end{tabular}
\\ \hline
\end{longtable}
}


  
 \newpage 
\subsection{[LVV-6564] DM-TS-AUX-ICD-0014-V-01: GPS Water Vapor Data Quality\_DM\_1 }\label{lvv-6564}

\begin{longtable}{cccc}
\hline
\textbf{Jira Link} & \textbf{Assignee} & \textbf{Status} & \textbf{Test Cases}\\ \hline
\href{https://jira.lsstcorp.org/browse/LVV-6564}{LVV-6564} &
Leanne Guy & Not Covered &
\begin{tabular}{c}
\end{tabular}
\\
\hline
\end{longtable}

\textbf{Verification Element Description:} \\
Undefined

{\footnotesize
\begin{longtable}{p{2.5cm}p{13.5cm}}
\hline
\multicolumn{2}{c}{\textbf{Requirement Details}}\\ \hline
Requirement ID & DM-TS-AUX-ICD-0014 \\ \cdashline{1-2}
Requirement Description &
\begin{minipage}[]{13cm}
\textbf{Specification:} The publication of GPS water vapor data shall
include any data quality or accuracy assessment generated by the data
reduction.
\end{minipage}
\\ \cdashline{1-2}
Requirement Discussion &
\begin{minipage}[]{13cm}
\textbf{Discussion:} Typically GPS instruments are able to estimate the
quality of their data reductions based on data such as the number of GPS
satellites visible and their locations. This and other information can
be used to compute accuracy estimates on the results of the reduction.
\end{minipage}
\\ \cdashline{1-2}
Requirement Priority &  \\ \cdashline{1-2}
Upper Level Requirement &
\begin{tabular}{cl}
\end{tabular}
\\ \hline
\end{longtable}
}


  
 \newpage 
\subsection{[LVV-6565] DM-TS-AUX-ICD-0014-V-02: GPS Water Vapor Data Quality\_DM\_2 }\label{lvv-6565}

\begin{longtable}{cccc}
\hline
\textbf{Jira Link} & \textbf{Assignee} & \textbf{Status} & \textbf{Test Cases}\\ \hline
\href{https://jira.lsstcorp.org/browse/LVV-6565}{LVV-6565} &
Leanne Guy & Not Covered &
\begin{tabular}{c}
\end{tabular}
\\
\hline
\end{longtable}

\textbf{Verification Element Description:} \\
Undefined

{\footnotesize
\begin{longtable}{p{2.5cm}p{13.5cm}}
\hline
\multicolumn{2}{c}{\textbf{Requirement Details}}\\ \hline
Requirement ID & DM-TS-AUX-ICD-0014 \\ \cdashline{1-2}
Requirement Description &
\begin{minipage}[]{13cm}
\textbf{Specification:} The publication of GPS water vapor data shall
include any data quality or accuracy assessment generated by the data
reduction.
\end{minipage}
\\ \cdashline{1-2}
Requirement Discussion &
\begin{minipage}[]{13cm}
\textbf{Discussion:} Typically GPS instruments are able to estimate the
quality of their data reductions based on data such as the number of GPS
satellites visible and their locations. This and other information can
be used to compute accuracy estimates on the results of the reduction.
\end{minipage}
\\ \cdashline{1-2}
Requirement Priority &  \\ \cdashline{1-2}
Upper Level Requirement &
\begin{tabular}{cl}
\end{tabular}
\\ \hline
\end{longtable}
}


  
 \newpage 
\subsection{[LVV-6570] DM-TS-AUX-ICD-0012-V-01: GPS Water Vapor Data\_DM\_1 }\label{lvv-6570}

\begin{longtable}{cccc}
\hline
\textbf{Jira Link} & \textbf{Assignee} & \textbf{Status} & \textbf{Test Cases}\\ \hline
\href{https://jira.lsstcorp.org/browse/LVV-6570}{LVV-6570} &
Leanne Guy & Not Covered &
\begin{tabular}{c}
\end{tabular}
\\
\hline
\end{longtable}

\textbf{Verification Element Description:} \\
Undefined

{\footnotesize
\begin{longtable}{p{2.5cm}p{13.5cm}}
\hline
\multicolumn{2}{c}{\textbf{Requirement Details}}\\ \hline
Requirement ID & DM-TS-AUX-ICD-0012 \\ \cdashline{1-2}
Requirement Description &
\begin{minipage}[]{13cm}
\textbf{Specification:} The Telescope and Site subsystem shall make
available to DM the total column water vapor result derived from the GPS
water vapor instrument.
\end{minipage}
\\ \cdashline{1-2}
Requirement Discussion &
\begin{minipage}[]{13cm}
\textbf{Discussion:} It is assumed that this is a commercially acquired
instrument, delivered by the Telescope \& Site team, with the intrinsic
capability of acquiring multi-channel GPS data. The data will be
analyzed, using algorithms that are in the professional literature, to
extract total column water vapor and, potentially, vertical profile
data. The analysis might be performed on LSST computers or through an
external initiative such as the NOAA ``Suominet'' project.\\
See the LSST Calibration Plan (\citeds{LSE-180}) and the Telescope and Site
requirements document (\citeds{LSE-60}) for details describing the GPS
instrumentation and its measurements. Details of the reduced data
required, including latency, accuracy/precision, and data record fields
are TBD and will be filled in during Phases 2 and 3 of the development
of this ICD. Aspects of this depend on the calibration implementation
plan currently under development.
\end{minipage}
\\ \cdashline{1-2}
Requirement Priority &  \\ \cdashline{1-2}
Upper Level Requirement &
\begin{tabular}{cl}
\end{tabular}
\\ \hline
\end{longtable}
}


  
 \newpage 
\subsection{[LVV-6571] DM-TS-AUX-ICD-0012-V-02: GPS Water Vapor Data\_DM\_2 }\label{lvv-6571}

\begin{longtable}{cccc}
\hline
\textbf{Jira Link} & \textbf{Assignee} & \textbf{Status} & \textbf{Test Cases}\\ \hline
\href{https://jira.lsstcorp.org/browse/LVV-6571}{LVV-6571} &
Leanne Guy & Not Covered &
\begin{tabular}{c}
\end{tabular}
\\
\hline
\end{longtable}

\textbf{Verification Element Description:} \\
Undefined

{\footnotesize
\begin{longtable}{p{2.5cm}p{13.5cm}}
\hline
\multicolumn{2}{c}{\textbf{Requirement Details}}\\ \hline
Requirement ID & DM-TS-AUX-ICD-0012 \\ \cdashline{1-2}
Requirement Description &
\begin{minipage}[]{13cm}
\textbf{Specification:} The Telescope and Site subsystem shall make
available to DM the total column water vapor result derived from the GPS
water vapor instrument.
\end{minipage}
\\ \cdashline{1-2}
Requirement Discussion &
\begin{minipage}[]{13cm}
\textbf{Discussion:} It is assumed that this is a commercially acquired
instrument, delivered by the Telescope \& Site team, with the intrinsic
capability of acquiring multi-channel GPS data. The data will be
analyzed, using algorithms that are in the professional literature, to
extract total column water vapor and, potentially, vertical profile
data. The analysis might be performed on LSST computers or through an
external initiative such as the NOAA ``Suominet'' project.\\
See the LSST Calibration Plan (\citeds{LSE-180}) and the Telescope and Site
requirements document (\citeds{LSE-60}) for details describing the GPS
instrumentation and its measurements. Details of the reduced data
required, including latency, accuracy/precision, and data record fields
are TBD and will be filled in during Phases 2 and 3 of the development
of this ICD. Aspects of this depend on the calibration implementation
plan currently under development.
\end{minipage}
\\ \cdashline{1-2}
Requirement Priority &  \\ \cdashline{1-2}
Upper Level Requirement &
\begin{tabular}{cl}
\end{tabular}
\\ \hline
\end{longtable}
}


  
 \newpage 
\subsection{[LVV-6576] DM-TS-AUX-ICD-0028-V-01: GPS Water Vapor Raw Data Archiving\_DM\_1 }\label{lvv-6576}

\begin{longtable}{cccc}
\hline
\textbf{Jira Link} & \textbf{Assignee} & \textbf{Status} & \textbf{Test Cases}\\ \hline
\href{https://jira.lsstcorp.org/browse/LVV-6576}{LVV-6576} &
Leanne Guy & Not Covered &
\begin{tabular}{c}
\end{tabular}
\\
\hline
\end{longtable}

\textbf{Verification Element Description:} \\
Undefined

{\footnotesize
\begin{longtable}{p{2.5cm}p{13.5cm}}
\hline
\multicolumn{2}{c}{\textbf{Requirement Details}}\\ \hline
Requirement ID & DM-TS-AUX-ICD-0028 \\ \cdashline{1-2}
Requirement Description &
\begin{minipage}[]{13cm}
\textbf{Specification}: The Telescope and Site subsystem shall publish
the raw data from the GPS water vapor instrument as telemetry.
\end{minipage}
\\ \cdashline{1-2}
Requirement Discussion &
\begin{minipage}[]{13cm}
\textbf{Discussion}: The intent of this is to ensure that the raw data
can always be reprocessed if an improved reduction algorithm becomes
available in the future. Publication as telemetry results in the
permanent archiving of the data in the EFD.
\end{minipage}
\\ \cdashline{1-2}
Requirement Priority &  \\ \cdashline{1-2}
Upper Level Requirement &
\begin{tabular}{cl}
\end{tabular}
\\ \hline
\end{longtable}
}


  
 \newpage 
\subsection{[LVV-6577] DM-TS-AUX-ICD-0028-V-02: GPS Water Vapor Raw Data Archiving\_DM\_2 }\label{lvv-6577}

\begin{longtable}{cccc}
\hline
\textbf{Jira Link} & \textbf{Assignee} & \textbf{Status} & \textbf{Test Cases}\\ \hline
\href{https://jira.lsstcorp.org/browse/LVV-6577}{LVV-6577} &
Leanne Guy & Not Covered &
\begin{tabular}{c}
\end{tabular}
\\
\hline
\end{longtable}

\textbf{Verification Element Description:} \\
Undefined

{\footnotesize
\begin{longtable}{p{2.5cm}p{13.5cm}}
\hline
\multicolumn{2}{c}{\textbf{Requirement Details}}\\ \hline
Requirement ID & DM-TS-AUX-ICD-0028 \\ \cdashline{1-2}
Requirement Description &
\begin{minipage}[]{13cm}
\textbf{Specification}: The Telescope and Site subsystem shall publish
the raw data from the GPS water vapor instrument as telemetry.
\end{minipage}
\\ \cdashline{1-2}
Requirement Discussion &
\begin{minipage}[]{13cm}
\textbf{Discussion}: The intent of this is to ensure that the raw data
can always be reprocessed if an improved reduction algorithm becomes
available in the future. Publication as telemetry results in the
permanent archiving of the data in the EFD.
\end{minipage}
\\ \cdashline{1-2}
Requirement Priority &  \\ \cdashline{1-2}
Upper Level Requirement &
\begin{tabular}{cl}
\end{tabular}
\\ \hline
\end{longtable}
}


  
 \newpage 
\subsection{[LVV-6594] DM-TS-AUX-ICD-0035-V-01: SED Spectrograph\_DM\_1 }\label{lvv-6594}

\begin{longtable}{cccc}
\hline
\textbf{Jira Link} & \textbf{Assignee} & \textbf{Status} & \textbf{Test Cases}\\ \hline
\href{https://jira.lsstcorp.org/browse/LVV-6594}{LVV-6594} &
Leanne Guy & Not Covered &
\begin{tabular}{c}
\end{tabular}
\\
\hline
\end{longtable}

\textbf{Verification Element Description:} \\
Undefined

{\footnotesize
\begin{longtable}{p{2.5cm}p{13.5cm}}
\hline
\multicolumn{2}{c}{\textbf{Requirement Details}}\\ \hline
Requirement ID & DM-TS-AUX-ICD-0035 \\ \cdashline{1-2}
Requirement Description &
\begin{minipage}[]{13cm}
\textbf{Specification:} The Telescope and Site subsystem shall acquire
data from the illumination system SED spectrograph and shall make the
data available to DM.
\end{minipage}
\\ \cdashline{1-2}
Requirement Discussion &
\begin{minipage}[]{13cm}
\textbf{Discussion:} The reduction of the raw data from the spectrograph
to spectra will be done by the Telescope and Site subsystem. The reduced
data, expected to be two column files (wavelength and intensity) will be
made available to DM. The Telescope and Site subsystem will also make
the raw data and any self-calibration data from this instrument
available. Most likely all this data will be in files in the large-file
annex of the Engineering and Facilities Database.\\
The reduced spectra will be associated with the concurrent camera
image(s) by Data Management.
\end{minipage}
\\ \cdashline{1-2}
Requirement Priority &  \\ \cdashline{1-2}
Upper Level Requirement &
\begin{tabular}{cl}
\end{tabular}
\\ \hline
\end{longtable}
}


  
 \newpage 
\subsection{[LVV-6595] DM-TS-AUX-ICD-0035-V-02: SED Spectrograph\_DM\_2 }\label{lvv-6595}

\begin{longtable}{cccc}
\hline
\textbf{Jira Link} & \textbf{Assignee} & \textbf{Status} & \textbf{Test Cases}\\ \hline
\href{https://jira.lsstcorp.org/browse/LVV-6595}{LVV-6595} &
Leanne Guy & Not Covered &
\begin{tabular}{c}
\end{tabular}
\\
\hline
\end{longtable}

\textbf{Verification Element Description:} \\
Undefined

{\footnotesize
\begin{longtable}{p{2.5cm}p{13.5cm}}
\hline
\multicolumn{2}{c}{\textbf{Requirement Details}}\\ \hline
Requirement ID & DM-TS-AUX-ICD-0035 \\ \cdashline{1-2}
Requirement Description &
\begin{minipage}[]{13cm}
\textbf{Specification:} The Telescope and Site subsystem shall acquire
data from the illumination system SED spectrograph and shall make the
data available to DM.
\end{minipage}
\\ \cdashline{1-2}
Requirement Discussion &
\begin{minipage}[]{13cm}
\textbf{Discussion:} The reduction of the raw data from the spectrograph
to spectra will be done by the Telescope and Site subsystem. The reduced
data, expected to be two column files (wavelength and intensity) will be
made available to DM. The Telescope and Site subsystem will also make
the raw data and any self-calibration data from this instrument
available. Most likely all this data will be in files in the large-file
annex of the Engineering and Facilities Database.\\
The reduced spectra will be associated with the concurrent camera
image(s) by Data Management.
\end{minipage}
\\ \cdashline{1-2}
Requirement Priority &  \\ \cdashline{1-2}
Upper Level Requirement &
\begin{tabular}{cl}
\end{tabular}
\\ \hline
\end{longtable}
}


  
 \newpage 
\subsection{[LVV-6600] DM-TS-AUX-ICD-0033-V-01: Tunable Laser\_DM\_1 }\label{lvv-6600}

\begin{longtable}{cccc}
\hline
\textbf{Jira Link} & \textbf{Assignee} & \textbf{Status} & \textbf{Test Cases}\\ \hline
\href{https://jira.lsstcorp.org/browse/LVV-6600}{LVV-6600} &
Leanne Guy & Not Covered &
\begin{tabular}{c}
\end{tabular}
\\
\hline
\end{longtable}

\textbf{Verification Element Description:} \\
Undefined

{\footnotesize
\begin{longtable}{p{2.5cm}p{13.5cm}}
\hline
\multicolumn{2}{c}{\textbf{Requirement Details}}\\ \hline
Requirement ID & DM-TS-AUX-ICD-0033 \\ \cdashline{1-2}
Requirement Description &
\begin{minipage}[]{13cm}
*Specification:*The Telescope and Site subsystem shall make the
instrument configuration and conditions data from the tunable laser
light source available to DM.
\end{minipage}
\\ \cdashline{1-2}
Requirement Discussion &
\begin{minipage}[]{13cm}
\textbf{Discussion:} This is expected to include intensity, wavelength
selection, and operating temperature but will be revised upon hardware
selection. We currently plan for both the set points and readback to be
transmitted as telemetry using the OCS middleware (TBR).
\end{minipage}
\\ \cdashline{1-2}
Requirement Priority &  \\ \cdashline{1-2}
Upper Level Requirement &
\begin{tabular}{cl}
\end{tabular}
\\ \hline
\end{longtable}
}


  
 \newpage 
\subsection{[LVV-6601] DM-TS-AUX-ICD-0033-V-02: Tunable Laser\_DM\_2 }\label{lvv-6601}

\begin{longtable}{cccc}
\hline
\textbf{Jira Link} & \textbf{Assignee} & \textbf{Status} & \textbf{Test Cases}\\ \hline
\href{https://jira.lsstcorp.org/browse/LVV-6601}{LVV-6601} &
Leanne Guy & Not Covered &
\begin{tabular}{c}
\end{tabular}
\\
\hline
\end{longtable}

\textbf{Verification Element Description:} \\
Undefined

{\footnotesize
\begin{longtable}{p{2.5cm}p{13.5cm}}
\hline
\multicolumn{2}{c}{\textbf{Requirement Details}}\\ \hline
Requirement ID & DM-TS-AUX-ICD-0033 \\ \cdashline{1-2}
Requirement Description &
\begin{minipage}[]{13cm}
*Specification:*The Telescope and Site subsystem shall make the
instrument configuration and conditions data from the tunable laser
light source available to DM.
\end{minipage}
\\ \cdashline{1-2}
Requirement Discussion &
\begin{minipage}[]{13cm}
\textbf{Discussion:} This is expected to include intensity, wavelength
selection, and operating temperature but will be revised upon hardware
selection. We currently plan for both the set points and readback to be
transmitted as telemetry using the OCS middleware (TBR).
\end{minipage}
\\ \cdashline{1-2}
Requirement Priority &  \\ \cdashline{1-2}
Upper Level Requirement &
\begin{tabular}{cl}
\end{tabular}
\\ \hline
\end{longtable}
}


  
 \newpage 
\subsection{[LVV-6606] DM-TS-AUX-ICD-0032-V-01: White-Light Source\_DM\_1 }\label{lvv-6606}

\begin{longtable}{cccc}
\hline
\textbf{Jira Link} & \textbf{Assignee} & \textbf{Status} & \textbf{Test Cases}\\ \hline
\href{https://jira.lsstcorp.org/browse/LVV-6606}{LVV-6606} &
Leanne Guy & Not Covered &
\begin{tabular}{c}
\end{tabular}
\\
\hline
\end{longtable}

\textbf{Verification Element Description:} \\
Undefined

{\footnotesize
\begin{longtable}{p{2.5cm}p{13.5cm}}
\hline
\multicolumn{2}{c}{\textbf{Requirement Details}}\\ \hline
Requirement ID & DM-TS-AUX-ICD-0032 \\ \cdashline{1-2}
Requirement Description &
\begin{minipage}[]{13cm}
\textbf{Specification:} The Telescope and Site subsystem shall make the
instrument configuration and conditions data from the white light source
available to DM.
\end{minipage}
\\ \cdashline{1-2}
Requirement Discussion &
\begin{minipage}[]{13cm}
\textbf{Discussion:} This is expected to include intensity setting and
operating temperature but will be revised upon hardware selection. We
currently plan for both the set points and readback to be transmitted as
telemetry using the OCS middleware (TBR).
\end{minipage}
\\ \cdashline{1-2}
Requirement Priority &  \\ \cdashline{1-2}
Upper Level Requirement &
\begin{tabular}{cl}
\end{tabular}
\\ \hline
\end{longtable}
}


  
 \newpage 
\subsection{[LVV-6607] DM-TS-AUX-ICD-0032-V-02: White-Light Source\_DM\_2 }\label{lvv-6607}

\begin{longtable}{cccc}
\hline
\textbf{Jira Link} & \textbf{Assignee} & \textbf{Status} & \textbf{Test Cases}\\ \hline
\href{https://jira.lsstcorp.org/browse/LVV-6607}{LVV-6607} &
Leanne Guy & Not Covered &
\begin{tabular}{c}
\end{tabular}
\\
\hline
\end{longtable}

\textbf{Verification Element Description:} \\
Undefined

{\footnotesize
\begin{longtable}{p{2.5cm}p{13.5cm}}
\hline
\multicolumn{2}{c}{\textbf{Requirement Details}}\\ \hline
Requirement ID & DM-TS-AUX-ICD-0032 \\ \cdashline{1-2}
Requirement Description &
\begin{minipage}[]{13cm}
\textbf{Specification:} The Telescope and Site subsystem shall make the
instrument configuration and conditions data from the white light source
available to DM.
\end{minipage}
\\ \cdashline{1-2}
Requirement Discussion &
\begin{minipage}[]{13cm}
\textbf{Discussion:} This is expected to include intensity setting and
operating temperature but will be revised upon hardware selection. We
currently plan for both the set points and readback to be transmitted as
telemetry using the OCS middleware (TBR).
\end{minipage}
\\ \cdashline{1-2}
Requirement Priority &  \\ \cdashline{1-2}
Upper Level Requirement &
\begin{tabular}{cl}
\end{tabular}
\\ \hline
\end{longtable}
}


  
 \newpage 
\subsection{[LVV-6751] EP-DM-CON-ICD-0036-V-01: DM Services\_DM\_1 }\label{lvv-6751}

\begin{longtable}{cccc}
\hline
\textbf{Jira Link} & \textbf{Assignee} & \textbf{Status} & \textbf{Test Cases}\\ \hline
\href{https://jira.lsstcorp.org/browse/LVV-6751}{LVV-6751} &
Leanne Guy & Not Covered &
\begin{tabular}{c}
\end{tabular}
\\
\hline
\end{longtable}

\textbf{Verification Element Description:} \\
Undefined

{\footnotesize
\begin{longtable}{p{2.5cm}p{13.5cm}}
\hline
\multicolumn{2}{c}{\textbf{Requirement Details}}\\ \hline
Requirement ID & EP-DM-CON-ICD-0036 \\ \cdashline{1-2}
Requirement Description &
\begin{minipage}[]{13cm}
\textbf{Specification}: DM shall ensure that the following services are
made available to the EPO systems: Image Cutout Service, Metaserv,
Mini-Broker, TAP, and ObsTAP.
\end{minipage}
\\ \cdashline{1-2}
Requirement Discussion &
\begin{minipage}[]{13cm}
\textbf{Discussion}: The purpose is as follows: Image Cutout Service for
obtaining single-band co-add images, Metaserv for querying the database,
Mini-Broker for applying EPO-provided filters to obtain variable star
classifications, and TAP/ObsTAP for ad-hoc queries. ~In addition to the
EPO-provided filters, we also hope to leverage the simple filters
mentioned in the DPDD: ``we will provide a limited number of pre-defined
filters for a small number of object types of common interest. These
will answer non-exclusive questions such as `is the light curve
consistent with an RR Lyra?'\,''
\end{minipage}
\\ \cdashline{1-2}
Requirement Priority &  \\ \cdashline{1-2}
Upper Level Requirement &
\begin{tabular}{cl}
\end{tabular}
\\ \hline
\end{longtable}
}


  
 \newpage 
\subsection{[LVV-6752] EP-DM-CON-ICD-0036-V-02: DM Services\_DM\_2 }\label{lvv-6752}

\begin{longtable}{cccc}
\hline
\textbf{Jira Link} & \textbf{Assignee} & \textbf{Status} & \textbf{Test Cases}\\ \hline
\href{https://jira.lsstcorp.org/browse/LVV-6752}{LVV-6752} &
Leanne Guy & Not Covered &
\begin{tabular}{c}
\end{tabular}
\\
\hline
\end{longtable}

\textbf{Verification Element Description:} \\
Undefined

{\footnotesize
\begin{longtable}{p{2.5cm}p{13.5cm}}
\hline
\multicolumn{2}{c}{\textbf{Requirement Details}}\\ \hline
Requirement ID & EP-DM-CON-ICD-0036 \\ \cdashline{1-2}
Requirement Description &
\begin{minipage}[]{13cm}
\textbf{Specification}: DM shall ensure that the following services are
made available to the EPO systems: Image Cutout Service, Metaserv,
Mini-Broker, TAP, and ObsTAP.
\end{minipage}
\\ \cdashline{1-2}
Requirement Discussion &
\begin{minipage}[]{13cm}
\textbf{Discussion}: The purpose is as follows: Image Cutout Service for
obtaining single-band co-add images, Metaserv for querying the database,
Mini-Broker for applying EPO-provided filters to obtain variable star
classifications, and TAP/ObsTAP for ad-hoc queries. ~In addition to the
EPO-provided filters, we also hope to leverage the simple filters
mentioned in the DPDD: ``we will provide a limited number of pre-defined
filters for a small number of object types of common interest. These
will answer non-exclusive questions such as `is the light curve
consistent with an RR Lyra?'\,''
\end{minipage}
\\ \cdashline{1-2}
Requirement Priority &  \\ \cdashline{1-2}
Upper Level Requirement &
\begin{tabular}{cl}
\end{tabular}
\\ \hline
\end{longtable}
}


  
 \newpage 
\subsection{[LVV-6757] EP-DM-CON-ICD-0035-V-01: DM Software\_DM\_1 }\label{lvv-6757}

\begin{longtable}{cccc}
\hline
\textbf{Jira Link} & \textbf{Assignee} & \textbf{Status} & \textbf{Test Cases}\\ \hline
\href{https://jira.lsstcorp.org/browse/LVV-6757}{LVV-6757} &
Leanne Guy & Not Covered &
\begin{tabular}{c}
\end{tabular}
\\
\hline
\end{longtable}

\textbf{Verification Element Description:} \\
Undefined

{\footnotesize
\begin{longtable}{p{2.5cm}p{13.5cm}}
\hline
\multicolumn{2}{c}{\textbf{Requirement Details}}\\ \hline
Requirement ID & EP-DM-CON-ICD-0035 \\ \cdashline{1-2}
Requirement Description &
\begin{minipage}[]{13cm}
\textbf{Specification}: ~DM shall provide the following software, which
is expected to be used in development of the EPO systems: Butler,
Supertask.
\end{minipage}
\\ \cdashline{1-2}
Requirement Discussion &
\begin{minipage}[]{13cm}
\textbf{Discussion}: The purpose is as follows: Butler for accessing
formal DM data products, Supertask for grouping and running related
tasks.
\end{minipage}
\\ \cdashline{1-2}
Requirement Priority &  \\ \cdashline{1-2}
Upper Level Requirement &
\begin{tabular}{cl}
\end{tabular}
\\ \hline
\end{longtable}
}


  
 \newpage 
\subsection{[LVV-6758] EP-DM-CON-ICD-0035-V-02: DM Software\_DM\_2 }\label{lvv-6758}

\begin{longtable}{cccc}
\hline
\textbf{Jira Link} & \textbf{Assignee} & \textbf{Status} & \textbf{Test Cases}\\ \hline
\href{https://jira.lsstcorp.org/browse/LVV-6758}{LVV-6758} &
Leanne Guy & Not Covered &
\begin{tabular}{c}
\end{tabular}
\\
\hline
\end{longtable}

\textbf{Verification Element Description:} \\
Undefined

{\footnotesize
\begin{longtable}{p{2.5cm}p{13.5cm}}
\hline
\multicolumn{2}{c}{\textbf{Requirement Details}}\\ \hline
Requirement ID & EP-DM-CON-ICD-0035 \\ \cdashline{1-2}
Requirement Description &
\begin{minipage}[]{13cm}
\textbf{Specification}: ~DM shall provide the following software, which
is expected to be used in development of the EPO systems: Butler,
Supertask.
\end{minipage}
\\ \cdashline{1-2}
Requirement Discussion &
\begin{minipage}[]{13cm}
\textbf{Discussion}: The purpose is as follows: Butler for accessing
formal DM data products, Supertask for grouping and running related
tasks.
\end{minipage}
\\ \cdashline{1-2}
Requirement Priority &  \\ \cdashline{1-2}
Upper Level Requirement &
\begin{tabular}{cl}
\end{tabular}
\\ \hline
\end{longtable}
}


  
 \newpage 
\subsection{[LVV-6763] EP-DM-CON-ICD-0037-V-01: EPO Compute Cluster\_DM\_1 }\label{lvv-6763}

\begin{longtable}{cccc}
\hline
\textbf{Jira Link} & \textbf{Assignee} & \textbf{Status} & \textbf{Test Cases}\\ \hline
\href{https://jira.lsstcorp.org/browse/LVV-6763}{LVV-6763} &
Leanne Guy & Not Covered &
\begin{tabular}{c}
\end{tabular}
\\
\hline
\end{longtable}

\textbf{Verification Element Description:} \\
Undefined

{\footnotesize
\begin{longtable}{p{2.5cm}p{13.5cm}}
\hline
\multicolumn{2}{c}{\textbf{Requirement Details}}\\ \hline
Requirement ID & EP-DM-CON-ICD-0037 \\ \cdashline{1-2}
Requirement Description &
\begin{minipage}[]{13cm}
\textbf{Specification}: ~NCSA shall host a compute cluster for EPO and
allow it to transfer approved public subset data to the EPO Data Center
(EDC).
\end{minipage}
\\ \cdashline{1-2}
Requirement Discussion &
\begin{minipage}[]{13cm}
\textbf{Discussion}: This will be similar to the Kubernetes cluster they
currently host for DM and will be paid for by EPO. ~This EPO cluster
will be used to apply EPO-specific processing (such as converting FITS
to TIFF) close to the data as well as to act as a storage buffer while
transferring data from the DAC to the EDC. ~Special VPN or firewall
configuration will be required to allow the cluster to push data to the
cloud-based EDC.
\end{minipage}
\\ \cdashline{1-2}
Requirement Priority &  \\ \cdashline{1-2}
Upper Level Requirement &
\begin{tabular}{cl}
\end{tabular}
\\ \hline
\end{longtable}
}


  
 \newpage 
\subsection{[LVV-6764] EP-DM-CON-ICD-0037-V-02: EPO Compute Cluster\_DM\_2 }\label{lvv-6764}

\begin{longtable}{cccc}
\hline
\textbf{Jira Link} & \textbf{Assignee} & \textbf{Status} & \textbf{Test Cases}\\ \hline
\href{https://jira.lsstcorp.org/browse/LVV-6764}{LVV-6764} &
Leanne Guy & Not Covered &
\begin{tabular}{c}
\end{tabular}
\\
\hline
\end{longtable}

\textbf{Verification Element Description:} \\
Undefined

{\footnotesize
\begin{longtable}{p{2.5cm}p{13.5cm}}
\hline
\multicolumn{2}{c}{\textbf{Requirement Details}}\\ \hline
Requirement ID & EP-DM-CON-ICD-0037 \\ \cdashline{1-2}
Requirement Description &
\begin{minipage}[]{13cm}
\textbf{Specification}: ~NCSA shall host a compute cluster for EPO and
allow it to transfer approved public subset data to the EPO Data Center
(EDC).
\end{minipage}
\\ \cdashline{1-2}
Requirement Discussion &
\begin{minipage}[]{13cm}
\textbf{Discussion}: This will be similar to the Kubernetes cluster they
currently host for DM and will be paid for by EPO. ~This EPO cluster
will be used to apply EPO-specific processing (such as converting FITS
to TIFF) close to the data as well as to act as a storage buffer while
transferring data from the DAC to the EDC. ~Special VPN or firewall
configuration will be required to allow the cluster to push data to the
cloud-based EDC.
\end{minipage}
\\ \cdashline{1-2}
Requirement Priority &  \\ \cdashline{1-2}
Upper Level Requirement &
\begin{tabular}{cl}
\end{tabular}
\\ \hline
\end{longtable}
}


  
 \newpage 
\subsection{[LVV-6771] SYS-ALL-COM-ICD-0047-V-06: Bulk Data Logging\_DM\_6 }\label{lvv-6771}

\begin{longtable}{cccc}
\hline
\textbf{Jira Link} & \textbf{Assignee} & \textbf{Status} & \textbf{Test Cases}\\ \hline
\href{https://jira.lsstcorp.org/browse/LVV-6771}{LVV-6771} &
Leanne Guy & Not Covered &
\begin{tabular}{c}
\end{tabular}
\\
\hline
\end{longtable}

\textbf{Verification Element Description:} \\
Undefined

{\footnotesize
\begin{longtable}{p{2.5cm}p{13.5cm}}
\hline
\multicolumn{2}{c}{\textbf{Requirement Details}}\\ \hline
Requirement ID & SYS-ALL-COM-ICD-0047 \\ \cdashline{1-2}
Requirement Description &
\begin{minipage}[]{13cm}
\textbf{Specification:} Any bulk data transport mechanism shall log data
to the EFD Cluster.
\end{minipage}
\\ \cdashline{1-2}
Requirement Discussion &
\begin{minipage}[]{13cm}
\textbf{Discussion:}
\end{minipage}
\\ \cdashline{1-2}
Requirement Priority &  \\ \cdashline{1-2}
Upper Level Requirement &
\begin{tabular}{cl}
SYS-ALL-COM-ICD-0046 & Data Logging \\
SYS-ALL-COM-ICD-0048 & Bulk Data Transport \\
\end{tabular}
\\ \hline
\end{longtable}
}


  
 \newpage 
\subsection{[LVV-6772] SYS-ALL-COM-ICD-0047-V-07: Bulk Data Logging\_DM\_7 }\label{lvv-6772}

\begin{longtable}{cccc}
\hline
\textbf{Jira Link} & \textbf{Assignee} & \textbf{Status} & \textbf{Test Cases}\\ \hline
\href{https://jira.lsstcorp.org/browse/LVV-6772}{LVV-6772} &
Leanne Guy & Not Covered &
\begin{tabular}{c}
\end{tabular}
\\
\hline
\end{longtable}

\textbf{Verification Element Description:} \\
Undefined

{\footnotesize
\begin{longtable}{p{2.5cm}p{13.5cm}}
\hline
\multicolumn{2}{c}{\textbf{Requirement Details}}\\ \hline
Requirement ID & SYS-ALL-COM-ICD-0047 \\ \cdashline{1-2}
Requirement Description &
\begin{minipage}[]{13cm}
\textbf{Specification:} Any bulk data transport mechanism shall log data
to the EFD Cluster.
\end{minipage}
\\ \cdashline{1-2}
Requirement Discussion &
\begin{minipage}[]{13cm}
\textbf{Discussion:}
\end{minipage}
\\ \cdashline{1-2}
Requirement Priority &  \\ \cdashline{1-2}
Upper Level Requirement &
\begin{tabular}{cl}
SYS-ALL-COM-ICD-0046 & Data Logging \\
SYS-ALL-COM-ICD-0048 & Bulk Data Transport \\
\end{tabular}
\\ \hline
\end{longtable}
}


  
 \newpage 
\subsection{[LVV-6777] SYS-ALL-COM-ICD-0048-V-06: Bulk Data Transport\_DM\_6 }\label{lvv-6777}

\begin{longtable}{cccc}
\hline
\textbf{Jira Link} & \textbf{Assignee} & \textbf{Status} & \textbf{Test Cases}\\ \hline
\href{https://jira.lsstcorp.org/browse/LVV-6777}{LVV-6777} &
Leanne Guy & Not Covered &
\begin{tabular}{c}
\end{tabular}
\\
\hline
\end{longtable}

\textbf{Verification Element Description:} \\
Undefined

{\footnotesize
\begin{longtable}{p{2.5cm}p{13.5cm}}
\hline
\multicolumn{2}{c}{\textbf{Requirement Details}}\\ \hline
Requirement ID & SYS-ALL-COM-ICD-0048 \\ \cdashline{1-2}
Requirement Description &
\begin{minipage}[]{13cm}
\textbf{Specification:} The principal subsystems shall exchange
point-to-point bulk data (large datasets) through a mutually agreed
mechanism.
\end{minipage}
\\ \cdashline{1-2}
Requirement Discussion &
\begin{minipage}[]{13cm}
\textbf{Discussion:} This allows for some flexibility where teams decide
DDS is not the best option for special cases. DDS, of course, may be the
mutually agreed mechanism.
\end{minipage}
\\ \cdashline{1-2}
Requirement Priority &  \\ \cdashline{1-2}
Upper Level Requirement &
\begin{tabular}{cl}
SYS-ALL-COM-ICD-0042 & Operational Data \\
\end{tabular}
\\ \hline
\end{longtable}
}


  
 \newpage 
\subsection{[LVV-6778] SYS-ALL-COM-ICD-0048-V-07: Bulk Data Transport\_DM\_7 }\label{lvv-6778}

\begin{longtable}{cccc}
\hline
\textbf{Jira Link} & \textbf{Assignee} & \textbf{Status} & \textbf{Test Cases}\\ \hline
\href{https://jira.lsstcorp.org/browse/LVV-6778}{LVV-6778} &
Leanne Guy & Not Covered &
\begin{tabular}{c}
\end{tabular}
\\
\hline
\end{longtable}

\textbf{Verification Element Description:} \\
Undefined

{\footnotesize
\begin{longtable}{p{2.5cm}p{13.5cm}}
\hline
\multicolumn{2}{c}{\textbf{Requirement Details}}\\ \hline
Requirement ID & SYS-ALL-COM-ICD-0048 \\ \cdashline{1-2}
Requirement Description &
\begin{minipage}[]{13cm}
\textbf{Specification:} The principal subsystems shall exchange
point-to-point bulk data (large datasets) through a mutually agreed
mechanism.
\end{minipage}
\\ \cdashline{1-2}
Requirement Discussion &
\begin{minipage}[]{13cm}
\textbf{Discussion:} This allows for some flexibility where teams decide
DDS is not the best option for special cases. DDS, of course, may be the
mutually agreed mechanism.
\end{minipage}
\\ \cdashline{1-2}
Requirement Priority &  \\ \cdashline{1-2}
Upper Level Requirement &
\begin{tabular}{cl}
SYS-ALL-COM-ICD-0042 & Operational Data \\
\end{tabular}
\\ \hline
\end{longtable}
}


  
 \newpage 
\subsection{[LVV-6783] SYS-ALL-COM-ICD-0043-V-06: Common Data Exchange Means\_DM\_6 }\label{lvv-6783}

\begin{longtable}{cccc}
\hline
\textbf{Jira Link} & \textbf{Assignee} & \textbf{Status} & \textbf{Test Cases}\\ \hline
\href{https://jira.lsstcorp.org/browse/LVV-6783}{LVV-6783} &
Leanne Guy & Not Covered &
\begin{tabular}{c}
\end{tabular}
\\
\hline
\end{longtable}

\textbf{Verification Element Description:} \\
Undefined

{\footnotesize
\begin{longtable}{p{2.5cm}p{13.5cm}}
\hline
\multicolumn{2}{c}{\textbf{Requirement Details}}\\ \hline
Requirement ID & SYS-ALL-COM-ICD-0043 \\ \cdashline{1-2}
Requirement Description &
\begin{minipage}[]{13cm}
\textbf{Specification:} The normal means for operational data exchange
between prinicipal systems shall be through a common publish-subscribe
middleware.
\end{minipage}
\\ \cdashline{1-2}
Requirement Discussion &
\begin{minipage}[]{13cm}
\textbf{Discussion:}
\end{minipage}
\\ \cdashline{1-2}
Requirement Priority &  \\ \cdashline{1-2}
Upper Level Requirement &
\begin{tabular}{cl}
SYS-ALL-COM-ICD-0042 & Operational Data \\
\end{tabular}
\\ \hline
\end{longtable}
}


  
 \newpage 
\subsection{[LVV-6784] SYS-ALL-COM-ICD-0043-V-07: Common Data Exchange Means\_DM\_7 }\label{lvv-6784}

\begin{longtable}{cccc}
\hline
\textbf{Jira Link} & \textbf{Assignee} & \textbf{Status} & \textbf{Test Cases}\\ \hline
\href{https://jira.lsstcorp.org/browse/LVV-6784}{LVV-6784} &
Leanne Guy & Not Covered &
\begin{tabular}{c}
\end{tabular}
\\
\hline
\end{longtable}

\textbf{Verification Element Description:} \\
Undefined

{\footnotesize
\begin{longtable}{p{2.5cm}p{13.5cm}}
\hline
\multicolumn{2}{c}{\textbf{Requirement Details}}\\ \hline
Requirement ID & SYS-ALL-COM-ICD-0043 \\ \cdashline{1-2}
Requirement Description &
\begin{minipage}[]{13cm}
\textbf{Specification:} The normal means for operational data exchange
between prinicipal systems shall be through a common publish-subscribe
middleware.
\end{minipage}
\\ \cdashline{1-2}
Requirement Discussion &
\begin{minipage}[]{13cm}
\textbf{Discussion:}
\end{minipage}
\\ \cdashline{1-2}
Requirement Priority &  \\ \cdashline{1-2}
Upper Level Requirement &
\begin{tabular}{cl}
SYS-ALL-COM-ICD-0042 & Operational Data \\
\end{tabular}
\\ \hline
\end{longtable}
}


  
 \newpage 
\subsection{[LVV-6789] SYS-ALL-COM-ICD-0046-V-06: Data Logging\_DM\_6 }\label{lvv-6789}

\begin{longtable}{cccc}
\hline
\textbf{Jira Link} & \textbf{Assignee} & \textbf{Status} & \textbf{Test Cases}\\ \hline
\href{https://jira.lsstcorp.org/browse/LVV-6789}{LVV-6789} &
Leanne Guy & Not Covered &
\begin{tabular}{c}
\end{tabular}
\\
\hline
\end{longtable}

\textbf{Verification Element Description:} \\
Undefined

{\footnotesize
\begin{longtable}{p{2.5cm}p{13.5cm}}
\hline
\multicolumn{2}{c}{\textbf{Requirement Details}}\\ \hline
Requirement ID & SYS-ALL-COM-ICD-0046 \\ \cdashline{1-2}
Requirement Description &
\begin{minipage}[]{13cm}
\textbf{Specification:} The system shall log all operational data to an
Engineering Facilities Database.
\end{minipage}
\\ \cdashline{1-2}
Requirement Discussion &
\begin{minipage}[]{13cm}
\textbf{Discussion:}
\end{minipage}
\\ \cdashline{1-2}
Requirement Priority &  \\ \cdashline{1-2}
Upper Level Requirement &
\begin{tabular}{cl}
SYS-ALL-COM-ICD-0042 & Operational Data \\
\end{tabular}
\\ \hline
\end{longtable}
}


  
 \newpage 
\subsection{[LVV-6790] SYS-ALL-COM-ICD-0046-V-07: Data Logging\_DM\_7 }\label{lvv-6790}

\begin{longtable}{cccc}
\hline
\textbf{Jira Link} & \textbf{Assignee} & \textbf{Status} & \textbf{Test Cases}\\ \hline
\href{https://jira.lsstcorp.org/browse/LVV-6790}{LVV-6790} &
Leanne Guy & Not Covered &
\begin{tabular}{c}
\end{tabular}
\\
\hline
\end{longtable}

\textbf{Verification Element Description:} \\
Undefined

{\footnotesize
\begin{longtable}{p{2.5cm}p{13.5cm}}
\hline
\multicolumn{2}{c}{\textbf{Requirement Details}}\\ \hline
Requirement ID & SYS-ALL-COM-ICD-0046 \\ \cdashline{1-2}
Requirement Description &
\begin{minipage}[]{13cm}
\textbf{Specification:} The system shall log all operational data to an
Engineering Facilities Database.
\end{minipage}
\\ \cdashline{1-2}
Requirement Discussion &
\begin{minipage}[]{13cm}
\textbf{Discussion:}
\end{minipage}
\\ \cdashline{1-2}
Requirement Priority &  \\ \cdashline{1-2}
Upper Level Requirement &
\begin{tabular}{cl}
SYS-ALL-COM-ICD-0042 & Operational Data \\
\end{tabular}
\\ \hline
\end{longtable}
}


  
 \newpage 
\subsection{[LVV-6795] SYS-ALL-COM-ICD-0044-V-06: DDS Standard\_DM\_6 }\label{lvv-6795}

\begin{longtable}{cccc}
\hline
\textbf{Jira Link} & \textbf{Assignee} & \textbf{Status} & \textbf{Test Cases}\\ \hline
\href{https://jira.lsstcorp.org/browse/LVV-6795}{LVV-6795} &
Leanne Guy & Not Covered &
\begin{tabular}{c}
\end{tabular}
\\
\hline
\end{longtable}

\textbf{Verification Element Description:} \\
Undefined

{\footnotesize
\begin{longtable}{p{2.5cm}p{13.5cm}}
\hline
\multicolumn{2}{c}{\textbf{Requirement Details}}\\ \hline
Requirement ID & SYS-ALL-COM-ICD-0044 \\ \cdashline{1-2}
Requirement Description &
\begin{minipage}[]{13cm}
\textbf{Specification:} The system middleware shall comply with the OMG
Data- Distribution Service for Real-Time Systems (DDS) standard.
\end{minipage}
\\ \cdashline{1-2}
Requirement Discussion &
\begin{minipage}[]{13cm}
\textbf{Discussion:}
\end{minipage}
\\ \cdashline{1-2}
Requirement Priority &  \\ \cdashline{1-2}
Upper Level Requirement &
\begin{tabular}{cl}
SYS-ALL-COM-ICD-0046 & Data Logging \\
SYS-ALL-COM-ICD-0043 & Common Data Exchange Means \\
\end{tabular}
\\ \hline
\end{longtable}
}


  
 \newpage 
\subsection{[LVV-6796] SYS-ALL-COM-ICD-0044-V-07: DDS Standard\_DM\_7 }\label{lvv-6796}

\begin{longtable}{cccc}
\hline
\textbf{Jira Link} & \textbf{Assignee} & \textbf{Status} & \textbf{Test Cases}\\ \hline
\href{https://jira.lsstcorp.org/browse/LVV-6796}{LVV-6796} &
Leanne Guy & Not Covered &
\begin{tabular}{c}
\end{tabular}
\\
\hline
\end{longtable}

\textbf{Verification Element Description:} \\
Undefined

{\footnotesize
\begin{longtable}{p{2.5cm}p{13.5cm}}
\hline
\multicolumn{2}{c}{\textbf{Requirement Details}}\\ \hline
Requirement ID & SYS-ALL-COM-ICD-0044 \\ \cdashline{1-2}
Requirement Description &
\begin{minipage}[]{13cm}
\textbf{Specification:} The system middleware shall comply with the OMG
Data- Distribution Service for Real-Time Systems (DDS) standard.
\end{minipage}
\\ \cdashline{1-2}
Requirement Discussion &
\begin{minipage}[]{13cm}
\textbf{Discussion:}
\end{minipage}
\\ \cdashline{1-2}
Requirement Priority &  \\ \cdashline{1-2}
Upper Level Requirement &
\begin{tabular}{cl}
SYS-ALL-COM-ICD-0046 & Data Logging \\
SYS-ALL-COM-ICD-0043 & Common Data Exchange Means \\
\end{tabular}
\\ \hline
\end{longtable}
}


  
 \newpage 
\subsection{[LVV-6801] SYS-ALL-COM-ICD-0045-V-06: DDS Version\_DM\_6 }\label{lvv-6801}

\begin{longtable}{cccc}
\hline
\textbf{Jira Link} & \textbf{Assignee} & \textbf{Status} & \textbf{Test Cases}\\ \hline
\href{https://jira.lsstcorp.org/browse/LVV-6801}{LVV-6801} &
Leanne Guy & Not Covered &
\begin{tabular}{c}
\end{tabular}
\\
\hline
\end{longtable}

\textbf{Verification Element Description:} \\
Undefined

{\footnotesize
\begin{longtable}{p{2.5cm}p{13.5cm}}
\hline
\multicolumn{2}{c}{\textbf{Requirement Details}}\\ \hline
Requirement ID & SYS-ALL-COM-ICD-0045 \\ \cdashline{1-2}
Requirement Description &
\begin{minipage}[]{13cm}
\textbf{Specification:} The system middleware shall comply with a
mutually agreed upon version and implementation of the DDS standard.
\end{minipage}
\\ \cdashline{1-2}
Requirement Discussion &
\begin{minipage}[]{13cm}
\textbf{Discussion:} We want to take advantage of updates, but there may
be reasons not to jump to the absolute latest version immediately.
(Indeed, some time may be required after the release of a version of the
specification before an implementation is available.)
\end{minipage}
\\ \cdashline{1-2}
Requirement Priority &  \\ \cdashline{1-2}
Upper Level Requirement &
\begin{tabular}{cl}
SYS-ALL-COM-ICD-0044 & DDS Standard \\
\end{tabular}
\\ \hline
\end{longtable}
}


  
 \newpage 
\subsection{[LVV-6802] SYS-ALL-COM-ICD-0045-V-07: DDS Version\_DM\_7 }\label{lvv-6802}

\begin{longtable}{cccc}
\hline
\textbf{Jira Link} & \textbf{Assignee} & \textbf{Status} & \textbf{Test Cases}\\ \hline
\href{https://jira.lsstcorp.org/browse/LVV-6802}{LVV-6802} &
Leanne Guy & Not Covered &
\begin{tabular}{c}
\end{tabular}
\\
\hline
\end{longtable}

\textbf{Verification Element Description:} \\
Undefined

{\footnotesize
\begin{longtable}{p{2.5cm}p{13.5cm}}
\hline
\multicolumn{2}{c}{\textbf{Requirement Details}}\\ \hline
Requirement ID & SYS-ALL-COM-ICD-0045 \\ \cdashline{1-2}
Requirement Description &
\begin{minipage}[]{13cm}
\textbf{Specification:} The system middleware shall comply with a
mutually agreed upon version and implementation of the DDS standard.
\end{minipage}
\\ \cdashline{1-2}
Requirement Discussion &
\begin{minipage}[]{13cm}
\textbf{Discussion:} We want to take advantage of updates, but there may
be reasons not to jump to the absolute latest version immediately.
(Indeed, some time may be required after the release of a version of the
specification before an implementation is available.)
\end{minipage}
\\ \cdashline{1-2}
Requirement Priority &  \\ \cdashline{1-2}
Upper Level Requirement &
\begin{tabular}{cl}
SYS-ALL-COM-ICD-0044 & DDS Standard \\
\end{tabular}
\\ \hline
\end{longtable}
}


  
 \newpage 
\subsection{[LVV-6807] SYS-ALL-COM-ICD-0042-V-06: Operational Data\_DM\_6 }\label{lvv-6807}

\begin{longtable}{cccc}
\hline
\textbf{Jira Link} & \textbf{Assignee} & \textbf{Status} & \textbf{Test Cases}\\ \hline
\href{https://jira.lsstcorp.org/browse/LVV-6807}{LVV-6807} &
Leanne Guy & Not Covered &
\begin{tabular}{c}
\end{tabular}
\\
\hline
\end{longtable}

\textbf{Verification Element Description:} \\
Undefined

{\footnotesize
\begin{longtable}{p{2.5cm}p{13.5cm}}
\hline
\multicolumn{2}{c}{\textbf{Requirement Details}}\\ \hline
Requirement ID & SYS-ALL-COM-ICD-0042 \\ \cdashline{1-2}
Requirement Description &
\begin{minipage}[]{13cm}
\textbf{Specification:} The principal systems shall exchange operational
system data.
\end{minipage}
\\ \cdashline{1-2}
Requirement Discussion &
\begin{minipage}[]{13cm}
\textbf{Discussion:} Nonoperational data are not within the scope of
this requirement, to allow for some flexibility, although in most cases
the same data transport mechanism would be appropriate.
\end{minipage}
\\ \cdashline{1-2}
Requirement Priority &  \\ \cdashline{1-2}
Upper Level Requirement &
\begin{tabular}{cl}
\end{tabular}
\\ \hline
\end{longtable}
}


  
 \newpage 
\subsection{[LVV-6808] SYS-ALL-COM-ICD-0042-V-07: Operational Data\_DM\_7 }\label{lvv-6808}

\begin{longtable}{cccc}
\hline
\textbf{Jira Link} & \textbf{Assignee} & \textbf{Status} & \textbf{Test Cases}\\ \hline
\href{https://jira.lsstcorp.org/browse/LVV-6808}{LVV-6808} &
Leanne Guy & Not Covered &
\begin{tabular}{c}
\end{tabular}
\\
\hline
\end{longtable}

\textbf{Verification Element Description:} \\
Undefined

{\footnotesize
\begin{longtable}{p{2.5cm}p{13.5cm}}
\hline
\multicolumn{2}{c}{\textbf{Requirement Details}}\\ \hline
Requirement ID & SYS-ALL-COM-ICD-0042 \\ \cdashline{1-2}
Requirement Description &
\begin{minipage}[]{13cm}
\textbf{Specification:} The principal systems shall exchange operational
system data.
\end{minipage}
\\ \cdashline{1-2}
Requirement Discussion &
\begin{minipage}[]{13cm}
\textbf{Discussion:} Nonoperational data are not within the scope of
this requirement, to allow for some flexibility, although in most cases
the same data transport mechanism would be appropriate.
\end{minipage}
\\ \cdashline{1-2}
Requirement Priority &  \\ \cdashline{1-2}
Upper Level Requirement &
\begin{tabular}{cl}
\end{tabular}
\\ \hline
\end{longtable}
}


  
 \newpage 
\subsection{[LVV-6813] SYS-ALL-COM-ICD-0029-V-06: 64-bit Support\_DM\_6 }\label{lvv-6813}

\begin{longtable}{cccc}
\hline
\textbf{Jira Link} & \textbf{Assignee} & \textbf{Status} & \textbf{Test Cases}\\ \hline
\href{https://jira.lsstcorp.org/browse/LVV-6813}{LVV-6813} &
Leanne Guy & Not Covered &
\begin{tabular}{c}
\end{tabular}
\\
\hline
\end{longtable}

\textbf{Verification Element Description:} \\
Undefined

{\footnotesize
\begin{longtable}{p{2.5cm}p{13.5cm}}
\hline
\multicolumn{2}{c}{\textbf{Requirement Details}}\\ \hline
Requirement ID & SYS-ALL-COM-ICD-0029 \\ \cdashline{1-2}
Requirement Description &
\begin{minipage}[]{13cm}
\textbf{Specification:} The system middleware shall support a 64-bit
version.
\end{minipage}
\\ \cdashline{1-2}
Requirement Discussion &
\begin{minipage}[]{13cm}
\textbf{Discussion:}
\end{minipage}
\\ \cdashline{1-2}
Requirement Priority &  \\ \cdashline{1-2}
Upper Level Requirement &
\begin{tabular}{cl}
SYS-ALL-COM-ICD-0026 & Provide Interface To Middleware \\
SYS-ALL-COM-ICD-0043 & Common Data Exchange Means \\
\end{tabular}
\\ \hline
\end{longtable}
}


  
 \newpage 
\subsection{[LVV-6814] SYS-ALL-COM-ICD-0029-V-07: 64-bit Support\_DM\_7 }\label{lvv-6814}

\begin{longtable}{cccc}
\hline
\textbf{Jira Link} & \textbf{Assignee} & \textbf{Status} & \textbf{Test Cases}\\ \hline
\href{https://jira.lsstcorp.org/browse/LVV-6814}{LVV-6814} &
Leanne Guy & Not Covered &
\begin{tabular}{c}
\end{tabular}
\\
\hline
\end{longtable}

\textbf{Verification Element Description:} \\
Undefined

{\footnotesize
\begin{longtable}{p{2.5cm}p{13.5cm}}
\hline
\multicolumn{2}{c}{\textbf{Requirement Details}}\\ \hline
Requirement ID & SYS-ALL-COM-ICD-0029 \\ \cdashline{1-2}
Requirement Description &
\begin{minipage}[]{13cm}
\textbf{Specification:} The system middleware shall support a 64-bit
version.
\end{minipage}
\\ \cdashline{1-2}
Requirement Discussion &
\begin{minipage}[]{13cm}
\textbf{Discussion:}
\end{minipage}
\\ \cdashline{1-2}
Requirement Priority &  \\ \cdashline{1-2}
Upper Level Requirement &
\begin{tabular}{cl}
SYS-ALL-COM-ICD-0026 & Provide Interface To Middleware \\
SYS-ALL-COM-ICD-0043 & Common Data Exchange Means \\
\end{tabular}
\\ \hline
\end{longtable}
}


  
 \newpage 
\subsection{[LVV-6819] SYS-ALL-COM-ICD-0028-V-06: Configure Quality of Service\_DM\_6 }\label{lvv-6819}

\begin{longtable}{cccc}
\hline
\textbf{Jira Link} & \textbf{Assignee} & \textbf{Status} & \textbf{Test Cases}\\ \hline
\href{https://jira.lsstcorp.org/browse/LVV-6819}{LVV-6819} &
Leanne Guy & Not Covered &
\begin{tabular}{c}
\end{tabular}
\\
\hline
\end{longtable}

\textbf{Verification Element Description:} \\
Undefined

{\footnotesize
\begin{longtable}{p{2.5cm}p{13.5cm}}
\hline
\multicolumn{2}{c}{\textbf{Requirement Details}}\\ \hline
Requirement ID & SYS-ALL-COM-ICD-0028 \\ \cdashline{1-2}
Requirement Description &
\begin{minipage}[]{13cm}
\textbf{Specification:} The API to the middleware shall support
configuring quality of service attributes for DDS topics.
\end{minipage}
\\ \cdashline{1-2}
Requirement Discussion &
\begin{minipage}[]{13cm}
\textbf{Discussion:}
\end{minipage}
\\ \cdashline{1-2}
Requirement Priority &  \\ \cdashline{1-2}
Upper Level Requirement &
\begin{tabular}{cl}
SYS-ALL-COM-ICD-0026 & Provide Interface To Middleware \\
\end{tabular}
\\ \hline
\end{longtable}
}


  
 \newpage 
\subsection{[LVV-6820] SYS-ALL-COM-ICD-0028-V-07: Configure Quality of Service\_DM\_7 }\label{lvv-6820}

\begin{longtable}{cccc}
\hline
\textbf{Jira Link} & \textbf{Assignee} & \textbf{Status} & \textbf{Test Cases}\\ \hline
\href{https://jira.lsstcorp.org/browse/LVV-6820}{LVV-6820} &
Leanne Guy & Not Covered &
\begin{tabular}{c}
\end{tabular}
\\
\hline
\end{longtable}

\textbf{Verification Element Description:} \\
Undefined

{\footnotesize
\begin{longtable}{p{2.5cm}p{13.5cm}}
\hline
\multicolumn{2}{c}{\textbf{Requirement Details}}\\ \hline
Requirement ID & SYS-ALL-COM-ICD-0028 \\ \cdashline{1-2}
Requirement Description &
\begin{minipage}[]{13cm}
\textbf{Specification:} The API to the middleware shall support
configuring quality of service attributes for DDS topics.
\end{minipage}
\\ \cdashline{1-2}
Requirement Discussion &
\begin{minipage}[]{13cm}
\textbf{Discussion:}
\end{minipage}
\\ \cdashline{1-2}
Requirement Priority &  \\ \cdashline{1-2}
Upper Level Requirement &
\begin{tabular}{cl}
SYS-ALL-COM-ICD-0026 & Provide Interface To Middleware \\
\end{tabular}
\\ \hline
\end{longtable}
}


  
 \newpage 
\subsection{[LVV-6825] SYS-ALL-COM-ICD-0005-V-06: Dynamic Data Source Selection\_DM\_6 }\label{lvv-6825}

\begin{longtable}{cccc}
\hline
\textbf{Jira Link} & \textbf{Assignee} & \textbf{Status} & \textbf{Test Cases}\\ \hline
\href{https://jira.lsstcorp.org/browse/LVV-6825}{LVV-6825} &
Leanne Guy & Not Covered &
\begin{tabular}{c}
\end{tabular}
\\
\hline
\end{longtable}

\textbf{Verification Element Description:} \\
Undefined

{\footnotesize
\begin{longtable}{p{2.5cm}p{13.5cm}}
\hline
\multicolumn{2}{c}{\textbf{Requirement Details}}\\ \hline
Requirement ID & SYS-ALL-COM-ICD-0005 \\ \cdashline{1-2}
Requirement Description &
\begin{minipage}[]{13cm}
\textbf{Specification:} The system middleware shall support a dynamic
data source selection mechanism to switch automatically between the
engineering facility databases based on availability,application, and
location.
\end{minipage}
\\ \cdashline{1-2}
Requirement Discussion &
\begin{minipage}[]{13cm}
\textbf{Discussion:}
\end{minipage}
\\ \cdashline{1-2}
Requirement Priority &  \\ \cdashline{1-2}
Upper Level Requirement &
\begin{tabular}{cl}
SYS-ALL-COM-ICD-0026 & Provide Interface To Middleware \\
\end{tabular}
\\ \hline
\end{longtable}
}


  
 \newpage 
\subsection{[LVV-6826] SYS-ALL-COM-ICD-0005-V-07: Dynamic Data Source Selection\_DM\_7 }\label{lvv-6826}

\begin{longtable}{cccc}
\hline
\textbf{Jira Link} & \textbf{Assignee} & \textbf{Status} & \textbf{Test Cases}\\ \hline
\href{https://jira.lsstcorp.org/browse/LVV-6826}{LVV-6826} &
Leanne Guy & Not Covered &
\begin{tabular}{c}
\end{tabular}
\\
\hline
\end{longtable}

\textbf{Verification Element Description:} \\
Undefined

{\footnotesize
\begin{longtable}{p{2.5cm}p{13.5cm}}
\hline
\multicolumn{2}{c}{\textbf{Requirement Details}}\\ \hline
Requirement ID & SYS-ALL-COM-ICD-0005 \\ \cdashline{1-2}
Requirement Description &
\begin{minipage}[]{13cm}
\textbf{Specification:} The system middleware shall support a dynamic
data source selection mechanism to switch automatically between the
engineering facility databases based on availability,application, and
location.
\end{minipage}
\\ \cdashline{1-2}
Requirement Discussion &
\begin{minipage}[]{13cm}
\textbf{Discussion:}
\end{minipage}
\\ \cdashline{1-2}
Requirement Priority &  \\ \cdashline{1-2}
Upper Level Requirement &
\begin{tabular}{cl}
SYS-ALL-COM-ICD-0026 & Provide Interface To Middleware \\
\end{tabular}
\\ \hline
\end{longtable}
}


  
 \newpage 
\subsection{[LVV-6831] SYS-ALL-COM-ICD-0030-V-06: Operating System\_DM\_6 }\label{lvv-6831}

\begin{longtable}{cccc}
\hline
\textbf{Jira Link} & \textbf{Assignee} & \textbf{Status} & \textbf{Test Cases}\\ \hline
\href{https://jira.lsstcorp.org/browse/LVV-6831}{LVV-6831} &
Leanne Guy & Not Covered &
\begin{tabular}{c}
\end{tabular}
\\
\hline
\end{longtable}

\textbf{Verification Element Description:} \\
Undefined

{\footnotesize
\begin{longtable}{p{2.5cm}p{13.5cm}}
\hline
\multicolumn{2}{c}{\textbf{Requirement Details}}\\ \hline
Requirement ID & SYS-ALL-COM-ICD-0030 \\ \cdashline{1-2}
Requirement Description &
\begin{minipage}[]{13cm}
\textbf{Specification:} The system middleware shall support a Linux
operating system.
\end{minipage}
\\ \cdashline{1-2}
Requirement Discussion &
\begin{minipage}[]{13cm}
\textbf{Discussion:}
\end{minipage}
\\ \cdashline{1-2}
Requirement Priority &  \\ \cdashline{1-2}
Upper Level Requirement &
\begin{tabular}{cl}
SYS-ALL-COM-ICD-0026 & Provide Interface To Middleware \\
SYS-ALL-COM-ICD-0043 & Common Data Exchange Means \\
\end{tabular}
\\ \hline
\end{longtable}
}


  
 \newpage 
\subsection{[LVV-6832] SYS-ALL-COM-ICD-0030-V-07: Operating System\_DM\_7 }\label{lvv-6832}

\begin{longtable}{cccc}
\hline
\textbf{Jira Link} & \textbf{Assignee} & \textbf{Status} & \textbf{Test Cases}\\ \hline
\href{https://jira.lsstcorp.org/browse/LVV-6832}{LVV-6832} &
Leanne Guy & Not Covered &
\begin{tabular}{c}
\end{tabular}
\\
\hline
\end{longtable}

\textbf{Verification Element Description:} \\
Undefined

{\footnotesize
\begin{longtable}{p{2.5cm}p{13.5cm}}
\hline
\multicolumn{2}{c}{\textbf{Requirement Details}}\\ \hline
Requirement ID & SYS-ALL-COM-ICD-0030 \\ \cdashline{1-2}
Requirement Description &
\begin{minipage}[]{13cm}
\textbf{Specification:} The system middleware shall support a Linux
operating system.
\end{minipage}
\\ \cdashline{1-2}
Requirement Discussion &
\begin{minipage}[]{13cm}
\textbf{Discussion:}
\end{minipage}
\\ \cdashline{1-2}
Requirement Priority &  \\ \cdashline{1-2}
Upper Level Requirement &
\begin{tabular}{cl}
SYS-ALL-COM-ICD-0026 & Provide Interface To Middleware \\
SYS-ALL-COM-ICD-0043 & Common Data Exchange Means \\
\end{tabular}
\\ \hline
\end{longtable}
}


  
 \newpage 
\subsection{[LVV-6837] SYS-ALL-COM-ICD-0026-V-06: Provide Interface To Middleware\_DM\_6 }\label{lvv-6837}

\begin{longtable}{cccc}
\hline
\textbf{Jira Link} & \textbf{Assignee} & \textbf{Status} & \textbf{Test Cases}\\ \hline
\href{https://jira.lsstcorp.org/browse/LVV-6837}{LVV-6837} &
Leanne Guy & Not Covered &
\begin{tabular}{c}
\end{tabular}
\\
\hline
\end{longtable}

\textbf{Verification Element Description:} \\
Undefined

{\footnotesize
\begin{longtable}{p{2.5cm}p{13.5cm}}
\hline
\multicolumn{2}{c}{\textbf{Requirement Details}}\\ \hline
Requirement ID & SYS-ALL-COM-ICD-0026 \\ \cdashline{1-2}
Requirement Description &
\begin{minipage}[]{13cm}
\textbf{Specification:} The project shall provide a set of APIs to the
middleware for the programming languages:

\begin{itemize}
\tightlist
\item
  C++
\item
  Java
\item
  Python
\item
  LabVIEW
\item
  Tcl
\end{itemize}
\end{minipage}
\\ \cdashline{1-2}
Requirement Discussion &
\begin{minipage}[]{13cm}
\textbf{Discussion:}
\end{minipage}
\\ \cdashline{1-2}
Requirement Priority &  \\ \cdashline{1-2}
Upper Level Requirement &
\begin{tabular}{cl}
\end{tabular}
\\ \hline
\end{longtable}
}


  
 \newpage 
\subsection{[LVV-6838] SYS-ALL-COM-ICD-0026-V-07: Provide Interface To Middleware\_DM\_7 }\label{lvv-6838}

\begin{longtable}{cccc}
\hline
\textbf{Jira Link} & \textbf{Assignee} & \textbf{Status} & \textbf{Test Cases}\\ \hline
\href{https://jira.lsstcorp.org/browse/LVV-6838}{LVV-6838} &
Leanne Guy & Not Covered &
\begin{tabular}{c}
\end{tabular}
\\
\hline
\end{longtable}

\textbf{Verification Element Description:} \\
Undefined

{\footnotesize
\begin{longtable}{p{2.5cm}p{13.5cm}}
\hline
\multicolumn{2}{c}{\textbf{Requirement Details}}\\ \hline
Requirement ID & SYS-ALL-COM-ICD-0026 \\ \cdashline{1-2}
Requirement Description &
\begin{minipage}[]{13cm}
\textbf{Specification:} The project shall provide a set of APIs to the
middleware for the programming languages:

\begin{itemize}
\tightlist
\item
  C++
\item
  Java
\item
  Python
\item
  LabVIEW
\item
  Tcl
\end{itemize}
\end{minipage}
\\ \cdashline{1-2}
Requirement Discussion &
\begin{minipage}[]{13cm}
\textbf{Discussion:}
\end{minipage}
\\ \cdashline{1-2}
Requirement Priority &  \\ \cdashline{1-2}
Upper Level Requirement &
\begin{tabular}{cl}
\end{tabular}
\\ \hline
\end{longtable}
}


  
 \newpage 
\subsection{[LVV-6843] SYS-ALL-COM-ICD-0027-V-06: Support Data Transport Functions\_DM\_6 }\label{lvv-6843}

\begin{longtable}{cccc}
\hline
\textbf{Jira Link} & \textbf{Assignee} & \textbf{Status} & \textbf{Test Cases}\\ \hline
\href{https://jira.lsstcorp.org/browse/LVV-6843}{LVV-6843} &
Leanne Guy & Not Covered &
\begin{tabular}{c}
\end{tabular}
\\
\hline
\end{longtable}

\textbf{Verification Element Description:} \\
Undefined

{\footnotesize
\begin{longtable}{p{2.5cm}p{13.5cm}}
\hline
\multicolumn{2}{c}{\textbf{Requirement Details}}\\ \hline
Requirement ID & SYS-ALL-COM-ICD-0027 \\ \cdashline{1-2}
Requirement Description &
\begin{minipage}[]{13cm}
\textbf{Specification:} The API to the middleware shall support
publishing and subscribing to data on DDS topics.
\end{minipage}
\\ \cdashline{1-2}
Requirement Discussion &
\begin{minipage}[]{13cm}
\textbf{Discussion:}
\end{minipage}
\\ \cdashline{1-2}
Requirement Priority &  \\ \cdashline{1-2}
Upper Level Requirement &
\begin{tabular}{cl}
SYS-ALL-COM-ICD-0026 & Provide Interface To Middleware \\
\end{tabular}
\\ \hline
\end{longtable}
}


  
 \newpage 
\subsection{[LVV-6844] SYS-ALL-COM-ICD-0027-V-07: Support Data Transport Functions\_DM\_7 }\label{lvv-6844}

\begin{longtable}{cccc}
\hline
\textbf{Jira Link} & \textbf{Assignee} & \textbf{Status} & \textbf{Test Cases}\\ \hline
\href{https://jira.lsstcorp.org/browse/LVV-6844}{LVV-6844} &
Leanne Guy & Not Covered &
\begin{tabular}{c}
\end{tabular}
\\
\hline
\end{longtable}

\textbf{Verification Element Description:} \\
Undefined

{\footnotesize
\begin{longtable}{p{2.5cm}p{13.5cm}}
\hline
\multicolumn{2}{c}{\textbf{Requirement Details}}\\ \hline
Requirement ID & SYS-ALL-COM-ICD-0027 \\ \cdashline{1-2}
Requirement Description &
\begin{minipage}[]{13cm}
\textbf{Specification:} The API to the middleware shall support
publishing and subscribing to data on DDS topics.
\end{minipage}
\\ \cdashline{1-2}
Requirement Discussion &
\begin{minipage}[]{13cm}
\textbf{Discussion:}
\end{minipage}
\\ \cdashline{1-2}
Requirement Priority &  \\ \cdashline{1-2}
Upper Level Requirement &
\begin{tabular}{cl}
SYS-ALL-COM-ICD-0026 & Provide Interface To Middleware \\
\end{tabular}
\\ \hline
\end{longtable}
}


  
 \newpage 
\subsection{[LVV-6849] SYS-ALL-COM-ICD-0050-V-06: Event after elapsed time\_DM\_6 }\label{lvv-6849}

\begin{longtable}{cccc}
\hline
\textbf{Jira Link} & \textbf{Assignee} & \textbf{Status} & \textbf{Test Cases}\\ \hline
\href{https://jira.lsstcorp.org/browse/LVV-6849}{LVV-6849} &
Leanne Guy & Not Covered &
\begin{tabular}{c}
\end{tabular}
\\
\hline
\end{longtable}

\textbf{Verification Element Description:} \\
Undefined

{\footnotesize
\begin{longtable}{p{2.5cm}p{13.5cm}}
\hline
\multicolumn{2}{c}{\textbf{Requirement Details}}\\ \hline
Requirement ID & SYS-ALL-COM-ICD-0050 \\ \cdashline{1-2}
Requirement Description &
\begin{minipage}[]{13cm}
\textbf{Specification:} The system shall be able to generate an event to
a client after a prescribed interval has elapsed.
\end{minipage}
\\ \cdashline{1-2}
Requirement Discussion &
\begin{minipage}[]{13cm}
\textbf{Discussion:} If the client process crashes, this will not
return.
\end{minipage}
\\ \cdashline{1-2}
Requirement Priority &  \\ \cdashline{1-2}
Upper Level Requirement &
\begin{tabular}{cl}
\end{tabular}
\\ \hline
\end{longtable}
}


  
 \newpage 
\subsection{[LVV-6850] SYS-ALL-COM-ICD-0050-V-07: Event after elapsed time\_DM\_7 }\label{lvv-6850}

\begin{longtable}{cccc}
\hline
\textbf{Jira Link} & \textbf{Assignee} & \textbf{Status} & \textbf{Test Cases}\\ \hline
\href{https://jira.lsstcorp.org/browse/LVV-6850}{LVV-6850} &
Leanne Guy & Not Covered &
\begin{tabular}{c}
\end{tabular}
\\
\hline
\end{longtable}

\textbf{Verification Element Description:} \\
Undefined

{\footnotesize
\begin{longtable}{p{2.5cm}p{13.5cm}}
\hline
\multicolumn{2}{c}{\textbf{Requirement Details}}\\ \hline
Requirement ID & SYS-ALL-COM-ICD-0050 \\ \cdashline{1-2}
Requirement Description &
\begin{minipage}[]{13cm}
\textbf{Specification:} The system shall be able to generate an event to
a client after a prescribed interval has elapsed.
\end{minipage}
\\ \cdashline{1-2}
Requirement Discussion &
\begin{minipage}[]{13cm}
\textbf{Discussion:} If the client process crashes, this will not
return.
\end{minipage}
\\ \cdashline{1-2}
Requirement Priority &  \\ \cdashline{1-2}
Upper Level Requirement &
\begin{tabular}{cl}
\end{tabular}
\\ \hline
\end{longtable}
}


  
 \newpage 
\subsection{[LVV-6855] SYS-ALL-COM-ICD-0049-V-06: Event at absolute time\_DM\_6 }\label{lvv-6855}

\begin{longtable}{cccc}
\hline
\textbf{Jira Link} & \textbf{Assignee} & \textbf{Status} & \textbf{Test Cases}\\ \hline
\href{https://jira.lsstcorp.org/browse/LVV-6855}{LVV-6855} &
Leanne Guy & Not Covered &
\begin{tabular}{c}
\end{tabular}
\\
\hline
\end{longtable}

\textbf{Verification Element Description:} \\
Undefined

{\footnotesize
\begin{longtable}{p{2.5cm}p{13.5cm}}
\hline
\multicolumn{2}{c}{\textbf{Requirement Details}}\\ \hline
Requirement ID & SYS-ALL-COM-ICD-0049 \\ \cdashline{1-2}
Requirement Description &
\begin{minipage}[]{13cm}
\textbf{Specification:} The system shall be able to generate an event to
a client at an absolute time requested by that client.
\end{minipage}
\\ \cdashline{1-2}
Requirement Discussion &
\begin{minipage}[]{13cm}
\textbf{Discussion:} If the client process crashes, this will not
return.
\end{minipage}
\\ \cdashline{1-2}
Requirement Priority &  \\ \cdashline{1-2}
Upper Level Requirement &
\begin{tabular}{cl}
\end{tabular}
\\ \hline
\end{longtable}
}


  
 \newpage 
\subsection{[LVV-6856] SYS-ALL-COM-ICD-0049-V-07: Event at absolute time\_DM\_7 }\label{lvv-6856}

\begin{longtable}{cccc}
\hline
\textbf{Jira Link} & \textbf{Assignee} & \textbf{Status} & \textbf{Test Cases}\\ \hline
\href{https://jira.lsstcorp.org/browse/LVV-6856}{LVV-6856} &
Leanne Guy & Not Covered &
\begin{tabular}{c}
\end{tabular}
\\
\hline
\end{longtable}

\textbf{Verification Element Description:} \\
Undefined

{\footnotesize
\begin{longtable}{p{2.5cm}p{13.5cm}}
\hline
\multicolumn{2}{c}{\textbf{Requirement Details}}\\ \hline
Requirement ID & SYS-ALL-COM-ICD-0049 \\ \cdashline{1-2}
Requirement Description &
\begin{minipage}[]{13cm}
\textbf{Specification:} The system shall be able to generate an event to
a client at an absolute time requested by that client.
\end{minipage}
\\ \cdashline{1-2}
Requirement Discussion &
\begin{minipage}[]{13cm}
\textbf{Discussion:} If the client process crashes, this will not
return.
\end{minipage}
\\ \cdashline{1-2}
Requirement Priority &  \\ \cdashline{1-2}
Upper Level Requirement &
\begin{tabular}{cl}
\end{tabular}
\\ \hline
\end{longtable}
}


  
 \newpage 
\subsection{[LVV-6861] SYS-ALL-COM-ICD-0031-V-06: Capture event time\_DM\_6 }\label{lvv-6861}

\begin{longtable}{cccc}
\hline
\textbf{Jira Link} & \textbf{Assignee} & \textbf{Status} & \textbf{Test Cases}\\ \hline
\href{https://jira.lsstcorp.org/browse/LVV-6861}{LVV-6861} &
Leanne Guy & Not Covered &
\begin{tabular}{c}
\end{tabular}
\\
\hline
\end{longtable}

\textbf{Verification Element Description:} \\
Undefined

{\footnotesize
\begin{longtable}{p{2.5cm}p{13.5cm}}
\hline
\multicolumn{2}{c}{\textbf{Requirement Details}}\\ \hline
Requirement ID & SYS-ALL-COM-ICD-0031 \\ \cdashline{1-2}
Requirement Description &
\begin{minipage}[]{13cm}
\textbf{Specification:} The system shall allow the sender of a message
to include an event time associated with a message.
\end{minipage}
\\ \cdashline{1-2}
Requirement Discussion &
\begin{minipage}[]{13cm}
\textbf{Discussion:}
\end{minipage}
\\ \cdashline{1-2}
Requirement Priority &  \\ \cdashline{1-2}
Upper Level Requirement &
\begin{tabular}{cl}
\end{tabular}
\\ \hline
\end{longtable}
}


  
 \newpage 
\subsection{[LVV-6862] SYS-ALL-COM-ICD-0031-V-07: Capture event time\_DM\_7 }\label{lvv-6862}

\begin{longtable}{cccc}
\hline
\textbf{Jira Link} & \textbf{Assignee} & \textbf{Status} & \textbf{Test Cases}\\ \hline
\href{https://jira.lsstcorp.org/browse/LVV-6862}{LVV-6862} &
Leanne Guy & Not Covered &
\begin{tabular}{c}
\end{tabular}
\\
\hline
\end{longtable}

\textbf{Verification Element Description:} \\
Undefined

{\footnotesize
\begin{longtable}{p{2.5cm}p{13.5cm}}
\hline
\multicolumn{2}{c}{\textbf{Requirement Details}}\\ \hline
Requirement ID & SYS-ALL-COM-ICD-0031 \\ \cdashline{1-2}
Requirement Description &
\begin{minipage}[]{13cm}
\textbf{Specification:} The system shall allow the sender of a message
to include an event time associated with a message.
\end{minipage}
\\ \cdashline{1-2}
Requirement Discussion &
\begin{minipage}[]{13cm}
\textbf{Discussion:}
\end{minipage}
\\ \cdashline{1-2}
Requirement Priority &  \\ \cdashline{1-2}
Upper Level Requirement &
\begin{tabular}{cl}
\end{tabular}
\\ \hline
\end{longtable}
}


  
 \newpage 
\subsection{[LVV-6867] SYS-ALL-COM-ICD-0033-V-06: Capture message arrival time in message
metadata\_DM\_6 }\label{lvv-6867}

\begin{longtable}{cccc}
\hline
\textbf{Jira Link} & \textbf{Assignee} & \textbf{Status} & \textbf{Test Cases}\\ \hline
\href{https://jira.lsstcorp.org/browse/LVV-6867}{LVV-6867} &
Leanne Guy & Not Covered &
\begin{tabular}{c}
\end{tabular}
\\
\hline
\end{longtable}

\textbf{Verification Element Description:} \\
Undefined

{\footnotesize
\begin{longtable}{p{2.5cm}p{13.5cm}}
\hline
\multicolumn{2}{c}{\textbf{Requirement Details}}\\ \hline
Requirement ID & SYS-ALL-COM-ICD-0033 \\ \cdashline{1-2}
Requirement Description &
\begin{minipage}[]{13cm}
\textbf{Specification:} The system shall capture the time of arrival of
each message at each subscriber in the message metadata.
\end{minipage}
\\ \cdashline{1-2}
Requirement Discussion &
\begin{minipage}[]{13cm}
\textbf{Discussion:} Used for consistency checking and error detection.
\end{minipage}
\\ \cdashline{1-2}
Requirement Priority &  \\ \cdashline{1-2}
Upper Level Requirement &
\begin{tabular}{cl}
\end{tabular}
\\ \hline
\end{longtable}
}


  
 \newpage 
\subsection{[LVV-6868] SYS-ALL-COM-ICD-0033-V-07: Capture message arrival time in message
metadata\_DM\_7 }\label{lvv-6868}

\begin{longtable}{cccc}
\hline
\textbf{Jira Link} & \textbf{Assignee} & \textbf{Status} & \textbf{Test Cases}\\ \hline
\href{https://jira.lsstcorp.org/browse/LVV-6868}{LVV-6868} &
Leanne Guy & Not Covered &
\begin{tabular}{c}
\end{tabular}
\\
\hline
\end{longtable}

\textbf{Verification Element Description:} \\
Undefined

{\footnotesize
\begin{longtable}{p{2.5cm}p{13.5cm}}
\hline
\multicolumn{2}{c}{\textbf{Requirement Details}}\\ \hline
Requirement ID & SYS-ALL-COM-ICD-0033 \\ \cdashline{1-2}
Requirement Description &
\begin{minipage}[]{13cm}
\textbf{Specification:} The system shall capture the time of arrival of
each message at each subscriber in the message metadata.
\end{minipage}
\\ \cdashline{1-2}
Requirement Discussion &
\begin{minipage}[]{13cm}
\textbf{Discussion:} Used for consistency checking and error detection.
\end{minipage}
\\ \cdashline{1-2}
Requirement Priority &  \\ \cdashline{1-2}
Upper Level Requirement &
\begin{tabular}{cl}
\end{tabular}
\\ \hline
\end{longtable}
}


  
 \newpage 
\subsection{[LVV-6873] SYS-ALL-COM-ICD-0035-V-06: Capture message send time in message
metadata\_DM\_6 }\label{lvv-6873}

\begin{longtable}{cccc}
\hline
\textbf{Jira Link} & \textbf{Assignee} & \textbf{Status} & \textbf{Test Cases}\\ \hline
\href{https://jira.lsstcorp.org/browse/LVV-6873}{LVV-6873} &
Leanne Guy & Not Covered &
\begin{tabular}{c}
\end{tabular}
\\
\hline
\end{longtable}

\textbf{Verification Element Description:} \\
Undefined

{\footnotesize
\begin{longtable}{p{2.5cm}p{13.5cm}}
\hline
\multicolumn{2}{c}{\textbf{Requirement Details}}\\ \hline
Requirement ID & SYS-ALL-COM-ICD-0035 \\ \cdashline{1-2}
Requirement Description &
\begin{minipage}[]{13cm}
\textbf{Specification:} The system shall capture the time of egress from
a publisher for a message in the message metadata.
\end{minipage}
\\ \cdashline{1-2}
Requirement Discussion &
\begin{minipage}[]{13cm}
\textbf{Discussion:} Used for consistency checking and error detection.
\end{minipage}
\\ \cdashline{1-2}
Requirement Priority &  \\ \cdashline{1-2}
Upper Level Requirement &
\begin{tabular}{cl}
\end{tabular}
\\ \hline
\end{longtable}
}


  
 \newpage 
\subsection{[LVV-6874] SYS-ALL-COM-ICD-0035-V-07: Capture message send time in message
metadata\_DM\_7 }\label{lvv-6874}

\begin{longtable}{cccc}
\hline
\textbf{Jira Link} & \textbf{Assignee} & \textbf{Status} & \textbf{Test Cases}\\ \hline
\href{https://jira.lsstcorp.org/browse/LVV-6874}{LVV-6874} &
Leanne Guy & Not Covered &
\begin{tabular}{c}
\end{tabular}
\\
\hline
\end{longtable}

\textbf{Verification Element Description:} \\
Undefined

{\footnotesize
\begin{longtable}{p{2.5cm}p{13.5cm}}
\hline
\multicolumn{2}{c}{\textbf{Requirement Details}}\\ \hline
Requirement ID & SYS-ALL-COM-ICD-0035 \\ \cdashline{1-2}
Requirement Description &
\begin{minipage}[]{13cm}
\textbf{Specification:} The system shall capture the time of egress from
a publisher for a message in the message metadata.
\end{minipage}
\\ \cdashline{1-2}
Requirement Discussion &
\begin{minipage}[]{13cm}
\textbf{Discussion:} Used for consistency checking and error detection.
\end{minipage}
\\ \cdashline{1-2}
Requirement Priority &  \\ \cdashline{1-2}
Upper Level Requirement &
\begin{tabular}{cl}
\end{tabular}
\\ \hline
\end{longtable}
}


  
 \newpage 
\subsection{[LVV-6879] SYS-ALL-COM-ICD-0037-V-06: Display times in UTC\_DM\_6 }\label{lvv-6879}

\begin{longtable}{cccc}
\hline
\textbf{Jira Link} & \textbf{Assignee} & \textbf{Status} & \textbf{Test Cases}\\ \hline
\href{https://jira.lsstcorp.org/browse/LVV-6879}{LVV-6879} &
Leanne Guy & Not Covered &
\begin{tabular}{c}
\end{tabular}
\\
\hline
\end{longtable}

\textbf{Verification Element Description:} \\
Undefined

{\footnotesize
\begin{longtable}{p{2.5cm}p{13.5cm}}
\hline
\multicolumn{2}{c}{\textbf{Requirement Details}}\\ \hline
Requirement ID & SYS-ALL-COM-ICD-0037 \\ \cdashline{1-2}
Requirement Description &
\begin{minipage}[]{13cm}
\textbf{Specification:} The system shall display all human-readable
timestamps using Universal Coordinated Time (UTC).
\end{minipage}
\\ \cdashline{1-2}
Requirement Discussion &
\begin{minipage}[]{13cm}
\textbf{Discussion:}
\end{minipage}
\\ \cdashline{1-2}
Requirement Priority &  \\ \cdashline{1-2}
Upper Level Requirement &
\begin{tabular}{cl}
SYS-ALL-COM-ICD-0036 & Human Readable Timestamp Representation \\
\end{tabular}
\\ \hline
\end{longtable}
}


  
 \newpage 
\subsection{[LVV-6880] SYS-ALL-COM-ICD-0037-V-07: Display times in UTC\_DM\_7 }\label{lvv-6880}

\begin{longtable}{cccc}
\hline
\textbf{Jira Link} & \textbf{Assignee} & \textbf{Status} & \textbf{Test Cases}\\ \hline
\href{https://jira.lsstcorp.org/browse/LVV-6880}{LVV-6880} &
Leanne Guy & Not Covered &
\begin{tabular}{c}
\end{tabular}
\\
\hline
\end{longtable}

\textbf{Verification Element Description:} \\
Undefined

{\footnotesize
\begin{longtable}{p{2.5cm}p{13.5cm}}
\hline
\multicolumn{2}{c}{\textbf{Requirement Details}}\\ \hline
Requirement ID & SYS-ALL-COM-ICD-0037 \\ \cdashline{1-2}
Requirement Description &
\begin{minipage}[]{13cm}
\textbf{Specification:} The system shall display all human-readable
timestamps using Universal Coordinated Time (UTC).
\end{minipage}
\\ \cdashline{1-2}
Requirement Discussion &
\begin{minipage}[]{13cm}
\textbf{Discussion:}
\end{minipage}
\\ \cdashline{1-2}
Requirement Priority &  \\ \cdashline{1-2}
Upper Level Requirement &
\begin{tabular}{cl}
SYS-ALL-COM-ICD-0036 & Human Readable Timestamp Representation \\
\end{tabular}
\\ \hline
\end{longtable}
}


  
 \newpage 
\subsection{[LVV-6885] SYS-ALL-COM-ICD-0040-V-06: Follow Clock Synchronization Protocol\_DM\_6 }\label{lvv-6885}

\begin{longtable}{cccc}
\hline
\textbf{Jira Link} & \textbf{Assignee} & \textbf{Status} & \textbf{Test Cases}\\ \hline
\href{https://jira.lsstcorp.org/browse/LVV-6885}{LVV-6885} &
Leanne Guy & Not Covered &
\begin{tabular}{c}
\end{tabular}
\\
\hline
\end{longtable}

\textbf{Verification Element Description:} \\
Undefined

{\footnotesize
\begin{longtable}{p{2.5cm}p{13.5cm}}
\hline
\multicolumn{2}{c}{\textbf{Requirement Details}}\\ \hline
Requirement ID & SYS-ALL-COM-ICD-0040 \\ \cdashline{1-2}
Requirement Description &
\begin{minipage}[]{13cm}
\textbf{Specification:} System components requiring accurate time shall
follow the IEEE 1588-2008 Standard for a Precision Clock Synchronization
Protocol for Networked Measurement and Control Systems, also known as
PTP Version 2.
\end{minipage}
\\ \cdashline{1-2}
Requirement Discussion &
\begin{minipage}[]{13cm}
\textbf{Discussion:} A hardware consequence of this is that the relevant
system components will need firmware support for this protocol in a GigE
Ethernet controller.
\end{minipage}
\\ \cdashline{1-2}
Requirement Priority &  \\ \cdashline{1-2}
Upper Level Requirement &
\begin{tabular}{cl}
\end{tabular}
\\ \hline
\end{longtable}
}


  
 \newpage 
\subsection{[LVV-6886] SYS-ALL-COM-ICD-0040-V-07: Follow Clock Synchronization Protocol\_DM\_7 }\label{lvv-6886}

\begin{longtable}{cccc}
\hline
\textbf{Jira Link} & \textbf{Assignee} & \textbf{Status} & \textbf{Test Cases}\\ \hline
\href{https://jira.lsstcorp.org/browse/LVV-6886}{LVV-6886} &
Leanne Guy & Not Covered &
\begin{tabular}{c}
\end{tabular}
\\
\hline
\end{longtable}

\textbf{Verification Element Description:} \\
Undefined

{\footnotesize
\begin{longtable}{p{2.5cm}p{13.5cm}}
\hline
\multicolumn{2}{c}{\textbf{Requirement Details}}\\ \hline
Requirement ID & SYS-ALL-COM-ICD-0040 \\ \cdashline{1-2}
Requirement Description &
\begin{minipage}[]{13cm}
\textbf{Specification:} System components requiring accurate time shall
follow the IEEE 1588-2008 Standard for a Precision Clock Synchronization
Protocol for Networked Measurement and Control Systems, also known as
PTP Version 2.
\end{minipage}
\\ \cdashline{1-2}
Requirement Discussion &
\begin{minipage}[]{13cm}
\textbf{Discussion:} A hardware consequence of this is that the relevant
system components will need firmware support for this protocol in a GigE
Ethernet controller.
\end{minipage}
\\ \cdashline{1-2}
Requirement Priority &  \\ \cdashline{1-2}
Upper Level Requirement &
\begin{tabular}{cl}
\end{tabular}
\\ \hline
\end{longtable}
}


  
 \newpage 
\subsection{[LVV-6891] SYS-ALL-COM-ICD-0036-V-06: Human Readable Timestamp
Representation\_DM\_6 }\label{lvv-6891}

\begin{longtable}{cccc}
\hline
\textbf{Jira Link} & \textbf{Assignee} & \textbf{Status} & \textbf{Test Cases}\\ \hline
\href{https://jira.lsstcorp.org/browse/LVV-6891}{LVV-6891} &
Leanne Guy & Not Covered &
\begin{tabular}{c}
\end{tabular}
\\
\hline
\end{longtable}

\textbf{Verification Element Description:} \\
Undefined

{\footnotesize
\begin{longtable}{p{2.5cm}p{13.5cm}}
\hline
\multicolumn{2}{c}{\textbf{Requirement Details}}\\ \hline
Requirement ID & SYS-ALL-COM-ICD-0036 \\ \cdashline{1-2}
Requirement Description &
\begin{minipage}[]{13cm}
\textbf{Specification:} The system shall follow ISO 8601 Data elements
and interchange formats -- Information interchange -- Representation of
dates and times when representing human-readable timestamps.
\end{minipage}
\\ \cdashline{1-2}
Requirement Discussion &
\begin{minipage}[]{13cm}
\textbf{Discussion:}
\end{minipage}
\\ \cdashline{1-2}
Requirement Priority &  \\ \cdashline{1-2}
Upper Level Requirement &
\begin{tabular}{cl}
\end{tabular}
\\ \hline
\end{longtable}
}


  
 \newpage 
\subsection{[LVV-6892] SYS-ALL-COM-ICD-0036-V-07: Human Readable Timestamp
Representation\_DM\_7 }\label{lvv-6892}

\begin{longtable}{cccc}
\hline
\textbf{Jira Link} & \textbf{Assignee} & \textbf{Status} & \textbf{Test Cases}\\ \hline
\href{https://jira.lsstcorp.org/browse/LVV-6892}{LVV-6892} &
Leanne Guy & Not Covered &
\begin{tabular}{c}
\end{tabular}
\\
\hline
\end{longtable}

\textbf{Verification Element Description:} \\
Undefined

{\footnotesize
\begin{longtable}{p{2.5cm}p{13.5cm}}
\hline
\multicolumn{2}{c}{\textbf{Requirement Details}}\\ \hline
Requirement ID & SYS-ALL-COM-ICD-0036 \\ \cdashline{1-2}
Requirement Description &
\begin{minipage}[]{13cm}
\textbf{Specification:} The system shall follow ISO 8601 Data elements
and interchange formats -- Information interchange -- Representation of
dates and times when representing human-readable timestamps.
\end{minipage}
\\ \cdashline{1-2}
Requirement Discussion &
\begin{minipage}[]{13cm}
\textbf{Discussion:}
\end{minipage}
\\ \cdashline{1-2}
Requirement Priority &  \\ \cdashline{1-2}
Upper Level Requirement &
\begin{tabular}{cl}
\end{tabular}
\\ \hline
\end{longtable}
}


  
 \newpage 
\subsection{[LVV-6897] SYS-ALL-COM-ICD-0041-V-06: Internal Timestamp Representation\_DM\_6 }\label{lvv-6897}

\begin{longtable}{cccc}
\hline
\textbf{Jira Link} & \textbf{Assignee} & \textbf{Status} & \textbf{Test Cases}\\ \hline
\href{https://jira.lsstcorp.org/browse/LVV-6897}{LVV-6897} &
Leanne Guy & Not Covered &
\begin{tabular}{c}
\end{tabular}
\\
\hline
\end{longtable}

\textbf{Verification Element Description:} \\
Undefined

{\footnotesize
\begin{longtable}{p{2.5cm}p{13.5cm}}
\hline
\multicolumn{2}{c}{\textbf{Requirement Details}}\\ \hline
Requirement ID & SYS-ALL-COM-ICD-0041 \\ \cdashline{1-2}
Requirement Description &
\begin{minipage}[]{13cm}
\textbf{Specification:} Internal to the system (includes SAL or nonSAL
sources), a timestamp shall be in the format specified in the standard,
that is, with a secondsField and a nanosecondsField.
\end{minipage}
\\ \cdashline{1-2}
Requirement Discussion &
\begin{minipage}[]{13cm}
\textbf{Discussion:}
\end{minipage}
\\ \cdashline{1-2}
Requirement Priority &  \\ \cdashline{1-2}
Upper Level Requirement &
\begin{tabular}{cl}
SYS-ALL-COM-ICD-0040 & Follow Clock Synchronization Protocol \\
\end{tabular}
\\ \hline
\end{longtable}
}


  
 \newpage 
\subsection{[LVV-6898] SYS-ALL-COM-ICD-0041-V-07: Internal Timestamp Representation\_DM\_7 }\label{lvv-6898}

\begin{longtable}{cccc}
\hline
\textbf{Jira Link} & \textbf{Assignee} & \textbf{Status} & \textbf{Test Cases}\\ \hline
\href{https://jira.lsstcorp.org/browse/LVV-6898}{LVV-6898} &
Leanne Guy & Not Covered &
\begin{tabular}{c}
\end{tabular}
\\
\hline
\end{longtable}

\textbf{Verification Element Description:} \\
Undefined

{\footnotesize
\begin{longtable}{p{2.5cm}p{13.5cm}}
\hline
\multicolumn{2}{c}{\textbf{Requirement Details}}\\ \hline
Requirement ID & SYS-ALL-COM-ICD-0041 \\ \cdashline{1-2}
Requirement Description &
\begin{minipage}[]{13cm}
\textbf{Specification:} Internal to the system (includes SAL or nonSAL
sources), a timestamp shall be in the format specified in the standard,
that is, with a secondsField and a nanosecondsField.
\end{minipage}
\\ \cdashline{1-2}
Requirement Discussion &
\begin{minipage}[]{13cm}
\textbf{Discussion:}
\end{minipage}
\\ \cdashline{1-2}
Requirement Priority &  \\ \cdashline{1-2}
Upper Level Requirement &
\begin{tabular}{cl}
SYS-ALL-COM-ICD-0040 & Follow Clock Synchronization Protocol \\
\end{tabular}
\\ \hline
\end{longtable}
}


  
 \newpage 
\subsection{[LVV-6903] SYS-ALL-COM-ICD-0038-V-06: Interpret internal time in displayed
timestamp\_DM\_6 }\label{lvv-6903}

\begin{longtable}{cccc}
\hline
\textbf{Jira Link} & \textbf{Assignee} & \textbf{Status} & \textbf{Test Cases}\\ \hline
\href{https://jira.lsstcorp.org/browse/LVV-6903}{LVV-6903} &
Leanne Guy & Not Covered &
\begin{tabular}{c}
\end{tabular}
\\
\hline
\end{longtable}

\textbf{Verification Element Description:} \\
Undefined

{\footnotesize
\begin{longtable}{p{2.5cm}p{13.5cm}}
\hline
\multicolumn{2}{c}{\textbf{Requirement Details}}\\ \hline
Requirement ID & SYS-ALL-COM-ICD-0038 \\ \cdashline{1-2}
Requirement Description &
\begin{minipage}[]{13cm}
\textbf{Specification:} The system shall convert PTP time to UTC upon
request.
\end{minipage}
\\ \cdashline{1-2}
Requirement Discussion &
\begin{minipage}[]{13cm}
\textbf{Discussion:} PTP time (internal representation) uses TAI
(elapsed time from reference date--no leap seconds), but UTC uses leap
seconds.
\end{minipage}
\\ \cdashline{1-2}
Requirement Priority &  \\ \cdashline{1-2}
Upper Level Requirement &
\begin{tabular}{cl}
SYS-ALL-COM-ICD-0041 & Internal Timestamp Representation \\
SYS-ALL-COM-ICD-0037 & Display times in UTC \\
\end{tabular}
\\ \hline
\end{longtable}
}


  
 \newpage 
\subsection{[LVV-6904] SYS-ALL-COM-ICD-0038-V-07: Interpret internal time in displayed
timestamp\_DM\_7 }\label{lvv-6904}

\begin{longtable}{cccc}
\hline
\textbf{Jira Link} & \textbf{Assignee} & \textbf{Status} & \textbf{Test Cases}\\ \hline
\href{https://jira.lsstcorp.org/browse/LVV-6904}{LVV-6904} &
Leanne Guy & Not Covered &
\begin{tabular}{c}
\end{tabular}
\\
\hline
\end{longtable}

\textbf{Verification Element Description:} \\
Undefined

{\footnotesize
\begin{longtable}{p{2.5cm}p{13.5cm}}
\hline
\multicolumn{2}{c}{\textbf{Requirement Details}}\\ \hline
Requirement ID & SYS-ALL-COM-ICD-0038 \\ \cdashline{1-2}
Requirement Description &
\begin{minipage}[]{13cm}
\textbf{Specification:} The system shall convert PTP time to UTC upon
request.
\end{minipage}
\\ \cdashline{1-2}
Requirement Discussion &
\begin{minipage}[]{13cm}
\textbf{Discussion:} PTP time (internal representation) uses TAI
(elapsed time from reference date--no leap seconds), but UTC uses leap
seconds.
\end{minipage}
\\ \cdashline{1-2}
Requirement Priority &  \\ \cdashline{1-2}
Upper Level Requirement &
\begin{tabular}{cl}
SYS-ALL-COM-ICD-0041 & Internal Timestamp Representation \\
SYS-ALL-COM-ICD-0037 & Display times in UTC \\
\end{tabular}
\\ \hline
\end{longtable}
}


  
 \newpage 
\subsection{[LVV-6909] SYS-ALL-COM-ICD-0034-V-06: Log message arrival time at database\_DM\_6 }\label{lvv-6909}

\begin{longtable}{cccc}
\hline
\textbf{Jira Link} & \textbf{Assignee} & \textbf{Status} & \textbf{Test Cases}\\ \hline
\href{https://jira.lsstcorp.org/browse/LVV-6909}{LVV-6909} &
Leanne Guy & Not Covered &
\begin{tabular}{c}
\end{tabular}
\\
\hline
\end{longtable}

\textbf{Verification Element Description:} \\
Undefined

{\footnotesize
\begin{longtable}{p{2.5cm}p{13.5cm}}
\hline
\multicolumn{2}{c}{\textbf{Requirement Details}}\\ \hline
Requirement ID & SYS-ALL-COM-ICD-0034 \\ \cdashline{1-2}
Requirement Description &
\begin{minipage}[]{13cm}
\textbf{Specification:} The system shall log each message and the
associated metadata for the message as it arrives at the logger.
\end{minipage}
\\ \cdashline{1-2}
Requirement Discussion &
\begin{minipage}[]{13cm}
\textbf{Discussion:} This means the system logs only one arrival time
for each message--that associated with arrival at the EFD.
\end{minipage}
\\ \cdashline{1-2}
Requirement Priority &  \\ \cdashline{1-2}
Upper Level Requirement &
\begin{tabular}{cl}
SYS-ALL-COM-ICD-0033 & Capture message arrival time in message metadata \\
\end{tabular}
\\ \hline
\end{longtable}
}


  
 \newpage 
\subsection{[LVV-6910] SYS-ALL-COM-ICD-0034-V-07: Log message arrival time at database\_DM\_7 }\label{lvv-6910}

\begin{longtable}{cccc}
\hline
\textbf{Jira Link} & \textbf{Assignee} & \textbf{Status} & \textbf{Test Cases}\\ \hline
\href{https://jira.lsstcorp.org/browse/LVV-6910}{LVV-6910} &
Leanne Guy & Not Covered &
\begin{tabular}{c}
\end{tabular}
\\
\hline
\end{longtable}

\textbf{Verification Element Description:} \\
Undefined

{\footnotesize
\begin{longtable}{p{2.5cm}p{13.5cm}}
\hline
\multicolumn{2}{c}{\textbf{Requirement Details}}\\ \hline
Requirement ID & SYS-ALL-COM-ICD-0034 \\ \cdashline{1-2}
Requirement Description &
\begin{minipage}[]{13cm}
\textbf{Specification:} The system shall log each message and the
associated metadata for the message as it arrives at the logger.
\end{minipage}
\\ \cdashline{1-2}
Requirement Discussion &
\begin{minipage}[]{13cm}
\textbf{Discussion:} This means the system logs only one arrival time
for each message--that associated with arrival at the EFD.
\end{minipage}
\\ \cdashline{1-2}
Requirement Priority &  \\ \cdashline{1-2}
Upper Level Requirement &
\begin{tabular}{cl}
SYS-ALL-COM-ICD-0033 & Capture message arrival time in message metadata \\
\end{tabular}
\\ \hline
\end{longtable}
}


  
 \newpage 
\subsection{[LVV-6915] SYS-ALL-COM-ICD-0032-V-06: Provide current time to application\_DM\_6 }\label{lvv-6915}

\begin{longtable}{cccc}
\hline
\textbf{Jira Link} & \textbf{Assignee} & \textbf{Status} & \textbf{Test Cases}\\ \hline
\href{https://jira.lsstcorp.org/browse/LVV-6915}{LVV-6915} &
Leanne Guy & Not Covered &
\begin{tabular}{c}
\end{tabular}
\\
\hline
\end{longtable}

\textbf{Verification Element Description:} \\
Undefined

{\footnotesize
\begin{longtable}{p{2.5cm}p{13.5cm}}
\hline
\multicolumn{2}{c}{\textbf{Requirement Details}}\\ \hline
Requirement ID & SYS-ALL-COM-ICD-0032 \\ \cdashline{1-2}
Requirement Description &
\begin{minipage}[]{13cm}
\textbf{Specification:} The system shall provide the currentTime to a
client on demand.
\end{minipage}
\\ \cdashline{1-2}
Requirement Discussion &
\begin{minipage}[]{13cm}
\textbf{Discussion:}
\end{minipage}
\\ \cdashline{1-2}
Requirement Priority &  \\ \cdashline{1-2}
Upper Level Requirement &
\begin{tabular}{cl}
SYS-ALL-COM-ICD-0031 & Capture event time \\
\end{tabular}
\\ \hline
\end{longtable}
}


  
 \newpage 
\subsection{[LVV-6916] SYS-ALL-COM-ICD-0032-V-07: Provide current time to application\_DM\_7 }\label{lvv-6916}

\begin{longtable}{cccc}
\hline
\textbf{Jira Link} & \textbf{Assignee} & \textbf{Status} & \textbf{Test Cases}\\ \hline
\href{https://jira.lsstcorp.org/browse/LVV-6916}{LVV-6916} &
Leanne Guy & Not Covered &
\begin{tabular}{c}
\end{tabular}
\\
\hline
\end{longtable}

\textbf{Verification Element Description:} \\
Undefined

{\footnotesize
\begin{longtable}{p{2.5cm}p{13.5cm}}
\hline
\multicolumn{2}{c}{\textbf{Requirement Details}}\\ \hline
Requirement ID & SYS-ALL-COM-ICD-0032 \\ \cdashline{1-2}
Requirement Description &
\begin{minipage}[]{13cm}
\textbf{Specification:} The system shall provide the currentTime to a
client on demand.
\end{minipage}
\\ \cdashline{1-2}
Requirement Discussion &
\begin{minipage}[]{13cm}
\textbf{Discussion:}
\end{minipage}
\\ \cdashline{1-2}
Requirement Priority &  \\ \cdashline{1-2}
Upper Level Requirement &
\begin{tabular}{cl}
SYS-ALL-COM-ICD-0031 & Capture event time \\
\end{tabular}
\\ \hline
\end{longtable}
}


  
 \newpage 
\subsection{[LVV-6921] SYS-ALL-COM-ICD-0039-V-06: Use standard time conversion library\_DM\_6 }\label{lvv-6921}

\begin{longtable}{cccc}
\hline
\textbf{Jira Link} & \textbf{Assignee} & \textbf{Status} & \textbf{Test Cases}\\ \hline
\href{https://jira.lsstcorp.org/browse/LVV-6921}{LVV-6921} &
Leanne Guy & Not Covered &
\begin{tabular}{c}
\end{tabular}
\\
\hline
\end{longtable}

\textbf{Verification Element Description:} \\
Undefined

{\footnotesize
\begin{longtable}{p{2.5cm}p{13.5cm}}
\hline
\multicolumn{2}{c}{\textbf{Requirement Details}}\\ \hline
Requirement ID & SYS-ALL-COM-ICD-0039 \\ \cdashline{1-2}
Requirement Description &
\begin{minipage}[]{13cm}
\textbf{Specification:} The system shall use the standard time
conversion library for time conversion calculations.
\end{minipage}
\\ \cdashline{1-2}
Requirement Discussion &
\begin{minipage}[]{13cm}
\textbf{Discussion:}
\end{minipage}
\\ \cdashline{1-2}
Requirement Priority &  \\ \cdashline{1-2}
Upper Level Requirement &
\begin{tabular}{cl}
SYS-ALL-COM-ICD-0038 & Interpret internal time in displayed timestamp \\
\end{tabular}
\\ \hline
\end{longtable}
}


  
 \newpage 
\subsection{[LVV-6922] SYS-ALL-COM-ICD-0039-V-07: Use standard time conversion library\_DM\_7 }\label{lvv-6922}

\begin{longtable}{cccc}
\hline
\textbf{Jira Link} & \textbf{Assignee} & \textbf{Status} & \textbf{Test Cases}\\ \hline
\href{https://jira.lsstcorp.org/browse/LVV-6922}{LVV-6922} &
Leanne Guy & Not Covered &
\begin{tabular}{c}
\end{tabular}
\\
\hline
\end{longtable}

\textbf{Verification Element Description:} \\
Undefined

{\footnotesize
\begin{longtable}{p{2.5cm}p{13.5cm}}
\hline
\multicolumn{2}{c}{\textbf{Requirement Details}}\\ \hline
Requirement ID & SYS-ALL-COM-ICD-0039 \\ \cdashline{1-2}
Requirement Description &
\begin{minipage}[]{13cm}
\textbf{Specification:} The system shall use the standard time
conversion library for time conversion calculations.
\end{minipage}
\\ \cdashline{1-2}
Requirement Discussion &
\begin{minipage}[]{13cm}
\textbf{Discussion:}
\end{minipage}
\\ \cdashline{1-2}
Requirement Priority &  \\ \cdashline{1-2}
Upper Level Requirement &
\begin{tabular}{cl}
SYS-ALL-COM-ICD-0038 & Interpret internal time in displayed timestamp \\
\end{tabular}
\\ \hline
\end{longtable}
}


  
 \newpage 
\subsection{[LVV-6927] CPT-OCS-INT-ICD-0001-V-06: Client Interface\_DM\_6 }\label{lvv-6927}

\begin{longtable}{cccc}
\hline
\textbf{Jira Link} & \textbf{Assignee} & \textbf{Status} & \textbf{Test Cases}\\ \hline
\href{https://jira.lsstcorp.org/browse/LVV-6927}{LVV-6927} &
Leanne Guy & Not Covered &
\begin{tabular}{c}
\end{tabular}
\\
\hline
\end{longtable}

\textbf{Verification Element Description:} \\
Undefined

{\footnotesize
\begin{longtable}{p{2.5cm}p{13.5cm}}
\hline
\multicolumn{2}{c}{\textbf{Requirement Details}}\\ \hline
Requirement ID & CPT-OCS-INT-ICD-0001 \\ \cdashline{1-2}
Requirement Description &
\begin{minipage}[]{13cm}
\textbf{Specification:} Each component controller shall define an
external interface that allows an interfacing component (loosely, a
client) to interact with that controller in a precise fashion.
\end{minipage}
\\ \cdashline{1-2}
Requirement Discussion &
\begin{minipage}[]{13cm}
\textbf{Discussion:} A client may be a View associated with the
controller, or another controller.
\end{minipage}
\\ \cdashline{1-2}
Requirement Priority &  \\ \cdashline{1-2}
Upper Level Requirement &
\begin{tabular}{cl}
\end{tabular}
\\ \hline
\end{longtable}
}


  
 \newpage 
\subsection{[LVV-6928] CPT-OCS-INT-ICD-0001-V-07: Client Interface\_DM\_7 }\label{lvv-6928}

\begin{longtable}{cccc}
\hline
\textbf{Jira Link} & \textbf{Assignee} & \textbf{Status} & \textbf{Test Cases}\\ \hline
\href{https://jira.lsstcorp.org/browse/LVV-6928}{LVV-6928} &
Leanne Guy & Not Covered &
\begin{tabular}{c}
\end{tabular}
\\
\hline
\end{longtable}

\textbf{Verification Element Description:} \\
Undefined

{\footnotesize
\begin{longtable}{p{2.5cm}p{13.5cm}}
\hline
\multicolumn{2}{c}{\textbf{Requirement Details}}\\ \hline
Requirement ID & CPT-OCS-INT-ICD-0001 \\ \cdashline{1-2}
Requirement Description &
\begin{minipage}[]{13cm}
\textbf{Specification:} Each component controller shall define an
external interface that allows an interfacing component (loosely, a
client) to interact with that controller in a precise fashion.
\end{minipage}
\\ \cdashline{1-2}
Requirement Discussion &
\begin{minipage}[]{13cm}
\textbf{Discussion:} A client may be a View associated with the
controller, or another controller.
\end{minipage}
\\ \cdashline{1-2}
Requirement Priority &  \\ \cdashline{1-2}
Upper Level Requirement &
\begin{tabular}{cl}
\end{tabular}
\\ \hline
\end{longtable}
}


  
 \newpage 
\subsection{[LVV-6933] CPT-OCS-INT-ICD-0005-V-06: Interface Design\_DM\_6 }\label{lvv-6933}

\begin{longtable}{cccc}
\hline
\textbf{Jira Link} & \textbf{Assignee} & \textbf{Status} & \textbf{Test Cases}\\ \hline
\href{https://jira.lsstcorp.org/browse/LVV-6933}{LVV-6933} &
Leanne Guy & Not Covered &
\begin{tabular}{c}
\end{tabular}
\\
\hline
\end{longtable}

\textbf{Verification Element Description:} \\
Undefined

{\footnotesize
\begin{longtable}{p{2.5cm}p{13.5cm}}
\hline
\multicolumn{2}{c}{\textbf{Requirement Details}}\\ \hline
Requirement ID & CPT-OCS-INT-ICD-0005 \\ \cdashline{1-2}
Requirement Description &
\begin{minipage}[]{13cm}
\textbf{Specification:} This interface information shall be part of the
design definition.
\end{minipage}
\\ \cdashline{1-2}
Requirement Discussion &
\begin{minipage}[]{13cm}
\textbf{Discussion:}
\end{minipage}
\\ \cdashline{1-2}
Requirement Priority &  \\ \cdashline{1-2}
Upper Level Requirement &
\begin{tabular}{cl}
CPT-OCS-INT-ICD-0001 & Client Interface \\
\end{tabular}
\\ \hline
\end{longtable}
}


  
 \newpage 
\subsection{[LVV-6934] CPT-OCS-INT-ICD-0005-V-07: Interface Design\_DM\_7 }\label{lvv-6934}

\begin{longtable}{cccc}
\hline
\textbf{Jira Link} & \textbf{Assignee} & \textbf{Status} & \textbf{Test Cases}\\ \hline
\href{https://jira.lsstcorp.org/browse/LVV-6934}{LVV-6934} &
Leanne Guy & Not Covered &
\begin{tabular}{c}
\end{tabular}
\\
\hline
\end{longtable}

\textbf{Verification Element Description:} \\
Undefined

{\footnotesize
\begin{longtable}{p{2.5cm}p{13.5cm}}
\hline
\multicolumn{2}{c}{\textbf{Requirement Details}}\\ \hline
Requirement ID & CPT-OCS-INT-ICD-0005 \\ \cdashline{1-2}
Requirement Description &
\begin{minipage}[]{13cm}
\textbf{Specification:} This interface information shall be part of the
design definition.
\end{minipage}
\\ \cdashline{1-2}
Requirement Discussion &
\begin{minipage}[]{13cm}
\textbf{Discussion:}
\end{minipage}
\\ \cdashline{1-2}
Requirement Priority &  \\ \cdashline{1-2}
Upper Level Requirement &
\begin{tabular}{cl}
CPT-OCS-INT-ICD-0001 & Client Interface \\
\end{tabular}
\\ \hline
\end{longtable}
}


  
 \newpage 
\subsection{[LVV-6939] CPT-OCS-INT-ICD-0006-V-06: Interface Elements\_DM\_6 }\label{lvv-6939}

\begin{longtable}{cccc}
\hline
\textbf{Jira Link} & \textbf{Assignee} & \textbf{Status} & \textbf{Test Cases}\\ \hline
\href{https://jira.lsstcorp.org/browse/LVV-6939}{LVV-6939} &
Leanne Guy & Not Covered &
\begin{tabular}{c}
\end{tabular}
\\
\hline
\end{longtable}

\textbf{Verification Element Description:} \\
Undefined

{\footnotesize
\begin{longtable}{p{2.5cm}p{13.5cm}}
\hline
\multicolumn{2}{c}{\textbf{Requirement Details}}\\ \hline
Requirement ID & CPT-OCS-INT-ICD-0006 \\ \cdashline{1-2}
Requirement Description &
\begin{minipage}[]{13cm}
\textbf{Specification}: The component interface shall minimally include
the external triggers to which it responds (i.e., data to which it
subscribes), and the consequent behaviors (including data the component
publishes).
\end{minipage}
\\ \cdashline{1-2}
Requirement Priority &  \\ \cdashline{1-2}
Upper Level Requirement &
\begin{tabular}{cl}
CPT-OCS-INT-ICD-0005 & Interface Design \\
\end{tabular}
\\ \hline
\end{longtable}
}


  
 \newpage 
\subsection{[LVV-6940] CPT-OCS-INT-ICD-0006-V-07: Interface Elements\_DM\_7 }\label{lvv-6940}

\begin{longtable}{cccc}
\hline
\textbf{Jira Link} & \textbf{Assignee} & \textbf{Status} & \textbf{Test Cases}\\ \hline
\href{https://jira.lsstcorp.org/browse/LVV-6940}{LVV-6940} &
Leanne Guy & Not Covered &
\begin{tabular}{c}
\end{tabular}
\\
\hline
\end{longtable}

\textbf{Verification Element Description:} \\
Undefined

{\footnotesize
\begin{longtable}{p{2.5cm}p{13.5cm}}
\hline
\multicolumn{2}{c}{\textbf{Requirement Details}}\\ \hline
Requirement ID & CPT-OCS-INT-ICD-0006 \\ \cdashline{1-2}
Requirement Description &
\begin{minipage}[]{13cm}
\textbf{Specification}: The component interface shall minimally include
the external triggers to which it responds (i.e., data to which it
subscribes), and the consequent behaviors (including data the component
publishes).
\end{minipage}
\\ \cdashline{1-2}
Requirement Priority &  \\ \cdashline{1-2}
Upper Level Requirement &
\begin{tabular}{cl}
CPT-OCS-INT-ICD-0005 & Interface Design \\
\end{tabular}
\\ \hline
\end{longtable}
}


  
 \newpage 
\subsection{[LVV-6945] CPT-OCS-INT-ICD-0008-V-06: Real-time Information\_DM\_6 }\label{lvv-6945}

\begin{longtable}{cccc}
\hline
\textbf{Jira Link} & \textbf{Assignee} & \textbf{Status} & \textbf{Test Cases}\\ \hline
\href{https://jira.lsstcorp.org/browse/LVV-6945}{LVV-6945} &
Leanne Guy & Not Covered &
\begin{tabular}{c}
\end{tabular}
\\
\hline
\end{longtable}

\textbf{Verification Element Description:} \\
Undefined

{\footnotesize
\begin{longtable}{p{2.5cm}p{13.5cm}}
\hline
\multicolumn{2}{c}{\textbf{Requirement Details}}\\ \hline
Requirement ID & CPT-OCS-INT-ICD-0008 \\ \cdashline{1-2}
Requirement Description &
\begin{minipage}[]{13cm}
\textbf{Specification:} Each component shall provide real-time
information sufficient for a client to make real-time predictions of
behavior.
\end{minipage}
\\ \cdashline{1-2}
Requirement Priority &  \\ \cdashline{1-2}
Upper Level Requirement &
\begin{tabular}{cl}
CPT-OCS-INT-ICD-0001 & Client Interface \\
\end{tabular}
\\ \hline
\end{longtable}
}


  
 \newpage 
\subsection{[LVV-6946] CPT-OCS-INT-ICD-0008-V-07: Real-time Information\_DM\_7 }\label{lvv-6946}

\begin{longtable}{cccc}
\hline
\textbf{Jira Link} & \textbf{Assignee} & \textbf{Status} & \textbf{Test Cases}\\ \hline
\href{https://jira.lsstcorp.org/browse/LVV-6946}{LVV-6946} &
Leanne Guy & Not Covered &
\begin{tabular}{c}
\end{tabular}
\\
\hline
\end{longtable}

\textbf{Verification Element Description:} \\
Undefined

{\footnotesize
\begin{longtable}{p{2.5cm}p{13.5cm}}
\hline
\multicolumn{2}{c}{\textbf{Requirement Details}}\\ \hline
Requirement ID & CPT-OCS-INT-ICD-0008 \\ \cdashline{1-2}
Requirement Description &
\begin{minipage}[]{13cm}
\textbf{Specification:} Each component shall provide real-time
information sufficient for a client to make real-time predictions of
behavior.
\end{minipage}
\\ \cdashline{1-2}
Requirement Priority &  \\ \cdashline{1-2}
Upper Level Requirement &
\begin{tabular}{cl}
CPT-OCS-INT-ICD-0001 & Client Interface \\
\end{tabular}
\\ \hline
\end{longtable}
}


  
 \newpage 
\subsection{[LVV-6951] CPT-OCS-INT-ICD-0040-V-06: Control Commander Commandee
relationship\_DM\_6 }\label{lvv-6951}

\begin{longtable}{cccc}
\hline
\textbf{Jira Link} & \textbf{Assignee} & \textbf{Status} & \textbf{Test Cases}\\ \hline
\href{https://jira.lsstcorp.org/browse/LVV-6951}{LVV-6951} &
Leanne Guy & Not Covered &
\begin{tabular}{c}
\end{tabular}
\\
\hline
\end{longtable}

\textbf{Verification Element Description:} \\
Undefined

{\footnotesize
\begin{longtable}{p{2.5cm}p{13.5cm}}
\hline
\multicolumn{2}{c}{\textbf{Requirement Details}}\\ \hline
Requirement ID & CPT-OCS-INT-ICD-0040 \\ \cdashline{1-2}
Requirement Description &
\begin{minipage}[]{13cm}
\textbf{Specification:} Requirement states that the system shall allow
commands only accordingly to the Commander-Commandee relationships.
\end{minipage}
\\ \cdashline{1-2}
Requirement Discussion &
\begin{minipage}[]{13cm}
\textbf{Discussion:}
\end{minipage}
\\ \cdashline{1-2}
Requirement Priority &  \\ \cdashline{1-2}
Upper Level Requirement &
\begin{tabular}{cl}
\end{tabular}
\\ \hline
\end{longtable}
}


  
 \newpage 
\subsection{[LVV-6952] CPT-OCS-INT-ICD-0040-V-07: Control Commander Commandee
relationship\_DM\_7 }\label{lvv-6952}

\begin{longtable}{cccc}
\hline
\textbf{Jira Link} & \textbf{Assignee} & \textbf{Status} & \textbf{Test Cases}\\ \hline
\href{https://jira.lsstcorp.org/browse/LVV-6952}{LVV-6952} &
Leanne Guy & Not Covered &
\begin{tabular}{c}
\end{tabular}
\\
\hline
\end{longtable}

\textbf{Verification Element Description:} \\
Undefined

{\footnotesize
\begin{longtable}{p{2.5cm}p{13.5cm}}
\hline
\multicolumn{2}{c}{\textbf{Requirement Details}}\\ \hline
Requirement ID & CPT-OCS-INT-ICD-0040 \\ \cdashline{1-2}
Requirement Description &
\begin{minipage}[]{13cm}
\textbf{Specification:} Requirement states that the system shall allow
commands only accordingly to the Commander-Commandee relationships.
\end{minipage}
\\ \cdashline{1-2}
Requirement Discussion &
\begin{minipage}[]{13cm}
\textbf{Discussion:}
\end{minipage}
\\ \cdashline{1-2}
Requirement Priority &  \\ \cdashline{1-2}
Upper Level Requirement &
\begin{tabular}{cl}
\end{tabular}
\\ \hline
\end{longtable}
}


  
 \newpage 
\subsection{[LVV-6957] CPT-OCS-INT-ICD-0041-V-06: Exclusive Control\_DM\_6 }\label{lvv-6957}

\begin{longtable}{cccc}
\hline
\textbf{Jira Link} & \textbf{Assignee} & \textbf{Status} & \textbf{Test Cases}\\ \hline
\href{https://jira.lsstcorp.org/browse/LVV-6957}{LVV-6957} &
Leanne Guy & Not Covered &
\begin{tabular}{c}
\end{tabular}
\\
\hline
\end{longtable}

\textbf{Verification Element Description:} \\
Undefined

{\footnotesize
\begin{longtable}{p{2.5cm}p{13.5cm}}
\hline
\multicolumn{2}{c}{\textbf{Requirement Details}}\\ \hline
Requirement ID & CPT-OCS-INT-ICD-0041 \\ \cdashline{1-2}
Requirement Description &
\begin{minipage}[]{13cm}
\textbf{Specification:} The system shall allow for a commander to have
exclusive control access of a commandee.
\end{minipage}
\\ \cdashline{1-2}
Requirement Discussion &
\begin{minipage}[]{13cm}
\textbf{Discussion:}
\end{minipage}
\\ \cdashline{1-2}
Requirement Priority &  \\ \cdashline{1-2}
Upper Level Requirement &
\begin{tabular}{cl}
CPT-OCS-INT-ICD-0042 & Update of relationship \\
\end{tabular}
\\ \hline
\end{longtable}
}


  
 \newpage 
\subsection{[LVV-6958] CPT-OCS-INT-ICD-0041-V-07: Exclusive Control\_DM\_7 }\label{lvv-6958}

\begin{longtable}{cccc}
\hline
\textbf{Jira Link} & \textbf{Assignee} & \textbf{Status} & \textbf{Test Cases}\\ \hline
\href{https://jira.lsstcorp.org/browse/LVV-6958}{LVV-6958} &
Leanne Guy & Not Covered &
\begin{tabular}{c}
\end{tabular}
\\
\hline
\end{longtable}

\textbf{Verification Element Description:} \\
Undefined

{\footnotesize
\begin{longtable}{p{2.5cm}p{13.5cm}}
\hline
\multicolumn{2}{c}{\textbf{Requirement Details}}\\ \hline
Requirement ID & CPT-OCS-INT-ICD-0041 \\ \cdashline{1-2}
Requirement Description &
\begin{minipage}[]{13cm}
\textbf{Specification:} The system shall allow for a commander to have
exclusive control access of a commandee.
\end{minipage}
\\ \cdashline{1-2}
Requirement Discussion &
\begin{minipage}[]{13cm}
\textbf{Discussion:}
\end{minipage}
\\ \cdashline{1-2}
Requirement Priority &  \\ \cdashline{1-2}
Upper Level Requirement &
\begin{tabular}{cl}
CPT-OCS-INT-ICD-0042 & Update of relationship \\
\end{tabular}
\\ \hline
\end{longtable}
}


  
 \newpage 
\subsection{[LVV-6963] CPT-OCS-INT-ICD-0042-V-06: Update of relationship\_DM\_6 }\label{lvv-6963}

\begin{longtable}{cccc}
\hline
\textbf{Jira Link} & \textbf{Assignee} & \textbf{Status} & \textbf{Test Cases}\\ \hline
\href{https://jira.lsstcorp.org/browse/LVV-6963}{LVV-6963} &
Leanne Guy & Not Covered &
\begin{tabular}{c}
\end{tabular}
\\
\hline
\end{longtable}

\textbf{Verification Element Description:} \\
Undefined

{\footnotesize
\begin{longtable}{p{2.5cm}p{13.5cm}}
\hline
\multicolumn{2}{c}{\textbf{Requirement Details}}\\ \hline
Requirement ID & CPT-OCS-INT-ICD-0042 \\ \cdashline{1-2}
Requirement Description &
\begin{minipage}[]{13cm}
\textbf{Specification:} The system shall provide an interface to update
the Commander - Commandee relationships.
\end{minipage}
\\ \cdashline{1-2}
Requirement Discussion &
\begin{minipage}[]{13cm}
\textbf{Discussion:}
\end{minipage}
\\ \cdashline{1-2}
Requirement Priority &  \\ \cdashline{1-2}
Upper Level Requirement &
\begin{tabular}{cl}
CPT-OCS-INT-ICD-0040 & Control Commander Commandee relationship \\
\end{tabular}
\\ \hline
\end{longtable}
}


  
 \newpage 
\subsection{[LVV-6964] CPT-OCS-INT-ICD-0042-V-07: Update of relationship\_DM\_7 }\label{lvv-6964}

\begin{longtable}{cccc}
\hline
\textbf{Jira Link} & \textbf{Assignee} & \textbf{Status} & \textbf{Test Cases}\\ \hline
\href{https://jira.lsstcorp.org/browse/LVV-6964}{LVV-6964} &
Leanne Guy & Not Covered &
\begin{tabular}{c}
\end{tabular}
\\
\hline
\end{longtable}

\textbf{Verification Element Description:} \\
Undefined

{\footnotesize
\begin{longtable}{p{2.5cm}p{13.5cm}}
\hline
\multicolumn{2}{c}{\textbf{Requirement Details}}\\ \hline
Requirement ID & CPT-OCS-INT-ICD-0042 \\ \cdashline{1-2}
Requirement Description &
\begin{minipage}[]{13cm}
\textbf{Specification:} The system shall provide an interface to update
the Commander - Commandee relationships.
\end{minipage}
\\ \cdashline{1-2}
Requirement Discussion &
\begin{minipage}[]{13cm}
\textbf{Discussion:}
\end{minipage}
\\ \cdashline{1-2}
Requirement Priority &  \\ \cdashline{1-2}
Upper Level Requirement &
\begin{tabular}{cl}
CPT-OCS-INT-ICD-0040 & Control Commander Commandee relationship \\
\end{tabular}
\\ \hline
\end{longtable}
}


  
 \newpage 
\subsection{[LVV-6969] CPT-OCS-INT-ICD-0002-V-06: Common Summary States\_DM\_6 }\label{lvv-6969}

\begin{longtable}{cccc}
\hline
\textbf{Jira Link} & \textbf{Assignee} & \textbf{Status} & \textbf{Test Cases}\\ \hline
\href{https://jira.lsstcorp.org/browse/LVV-6969}{LVV-6969} &
Leanne Guy & Not Covered &
\begin{tabular}{c}
\end{tabular}
\\
\hline
\end{longtable}

\textbf{Verification Element Description:} \\
Undefined

{\footnotesize
\begin{longtable}{p{2.5cm}p{13.5cm}}
\hline
\multicolumn{2}{c}{\textbf{Requirement Details}}\\ \hline
Requirement ID & CPT-OCS-INT-ICD-0002 \\ \cdashline{1-2}
Requirement Description &
\begin{minipage}[]{13cm}
\textbf{Specification:} The component shall incorporate summary
component states and transitions that conform to the definition in
Perform Top Level Functions. In this way all event-driven components
will present a common base set of behaviors to clients.
\end{minipage}
\\ \cdashline{1-2}
Requirement Priority &  \\ \cdashline{1-2}
Upper Level Requirement &
\begin{tabular}{cl}
CPT-OCS-INT-ICD-0009 & State Machine Description \\
\end{tabular}
\\ \hline
\end{longtable}
}


  
 \newpage 
\subsection{[LVV-6970] CPT-OCS-INT-ICD-0002-V-07: Common Summary States\_DM\_7 }\label{lvv-6970}

\begin{longtable}{cccc}
\hline
\textbf{Jira Link} & \textbf{Assignee} & \textbf{Status} & \textbf{Test Cases}\\ \hline
\href{https://jira.lsstcorp.org/browse/LVV-6970}{LVV-6970} &
Leanne Guy & Not Covered &
\begin{tabular}{c}
\end{tabular}
\\
\hline
\end{longtable}

\textbf{Verification Element Description:} \\
Undefined

{\footnotesize
\begin{longtable}{p{2.5cm}p{13.5cm}}
\hline
\multicolumn{2}{c}{\textbf{Requirement Details}}\\ \hline
Requirement ID & CPT-OCS-INT-ICD-0002 \\ \cdashline{1-2}
Requirement Description &
\begin{minipage}[]{13cm}
\textbf{Specification:} The component shall incorporate summary
component states and transitions that conform to the definition in
Perform Top Level Functions. In this way all event-driven components
will present a common base set of behaviors to clients.
\end{minipage}
\\ \cdashline{1-2}
Requirement Priority &  \\ \cdashline{1-2}
Upper Level Requirement &
\begin{tabular}{cl}
CPT-OCS-INT-ICD-0009 & State Machine Description \\
\end{tabular}
\\ \hline
\end{longtable}
}


  
 \newpage 
\subsection{[LVV-6975] CPT-OCS-INT-ICD-0003-V-06: Component Name in Namespace\_DM\_6 }\label{lvv-6975}

\begin{longtable}{cccc}
\hline
\textbf{Jira Link} & \textbf{Assignee} & \textbf{Status} & \textbf{Test Cases}\\ \hline
\href{https://jira.lsstcorp.org/browse/LVV-6975}{LVV-6975} &
Leanne Guy & Not Covered &
\begin{tabular}{c}
\end{tabular}
\\
\hline
\end{longtable}

\textbf{Verification Element Description:} \\
Undefined

{\footnotesize
\begin{longtable}{p{2.5cm}p{13.5cm}}
\hline
\multicolumn{2}{c}{\textbf{Requirement Details}}\\ \hline
Requirement ID & CPT-OCS-INT-ICD-0003 \\ \cdashline{1-2}
Requirement Description &
\begin{minipage}[]{13cm}
\textbf{Specification:} The component name shall define a namespace.
\end{minipage}
\\ \cdashline{1-2}
Requirement Priority &  \\ \cdashline{1-2}
Upper Level Requirement &
\begin{tabular}{cl}
CPT-OCS-INT-ICD-0012 & Unique Names for States \\
\end{tabular}
\\ \hline
\end{longtable}
}


  
 \newpage 
\subsection{[LVV-6976] CPT-OCS-INT-ICD-0003-V-07: Component Name in Namespace\_DM\_7 }\label{lvv-6976}

\begin{longtable}{cccc}
\hline
\textbf{Jira Link} & \textbf{Assignee} & \textbf{Status} & \textbf{Test Cases}\\ \hline
\href{https://jira.lsstcorp.org/browse/LVV-6976}{LVV-6976} &
Leanne Guy & Not Covered &
\begin{tabular}{c}
\end{tabular}
\\
\hline
\end{longtable}

\textbf{Verification Element Description:} \\
Undefined

{\footnotesize
\begin{longtable}{p{2.5cm}p{13.5cm}}
\hline
\multicolumn{2}{c}{\textbf{Requirement Details}}\\ \hline
Requirement ID & CPT-OCS-INT-ICD-0003 \\ \cdashline{1-2}
Requirement Description &
\begin{minipage}[]{13cm}
\textbf{Specification:} The component name shall define a namespace.
\end{minipage}
\\ \cdashline{1-2}
Requirement Priority &  \\ \cdashline{1-2}
Upper Level Requirement &
\begin{tabular}{cl}
CPT-OCS-INT-ICD-0012 & Unique Names for States \\
\end{tabular}
\\ \hline
\end{longtable}
}


  
 \newpage 
\subsection{[LVV-6981] CPT-OCS-INT-ICD-0009-V-06: State Machine Description\_DM\_6 }\label{lvv-6981}

\begin{longtable}{cccc}
\hline
\textbf{Jira Link} & \textbf{Assignee} & \textbf{Status} & \textbf{Test Cases}\\ \hline
\href{https://jira.lsstcorp.org/browse/LVV-6981}{LVV-6981} &
Leanne Guy & Not Covered &
\begin{tabular}{c}
\end{tabular}
\\
\hline
\end{longtable}

\textbf{Verification Element Description:} \\
Undefined

{\footnotesize
\begin{longtable}{p{2.5cm}p{13.5cm}}
\hline
\multicolumn{2}{c}{\textbf{Requirement Details}}\\ \hline
Requirement ID & CPT-OCS-INT-ICD-0009 \\ \cdashline{1-2}
Requirement Description &
\begin{minipage}[]{13cm}
\textbf{Specification:} The state machine definition shall explicitly
and formally define all transitions (external and internal), States, and
behaviors.\\
{{[}Options: UML State Machine diagram (preferred), table-based states
definition.{]}}
\end{minipage}
\\ \cdashline{1-2}
Requirement Priority &  \\ \cdashline{1-2}
Upper Level Requirement &
\begin{tabular}{cl}
CPT-OCS-INT-ICD-0010 & State-based Behavior \\
\end{tabular}
\\ \hline
\end{longtable}
}


  
 \newpage 
\subsection{[LVV-6982] CPT-OCS-INT-ICD-0009-V-07: State Machine Description\_DM\_7 }\label{lvv-6982}

\begin{longtable}{cccc}
\hline
\textbf{Jira Link} & \textbf{Assignee} & \textbf{Status} & \textbf{Test Cases}\\ \hline
\href{https://jira.lsstcorp.org/browse/LVV-6982}{LVV-6982} &
Leanne Guy & Not Covered &
\begin{tabular}{c}
\end{tabular}
\\
\hline
\end{longtable}

\textbf{Verification Element Description:} \\
Undefined

{\footnotesize
\begin{longtable}{p{2.5cm}p{13.5cm}}
\hline
\multicolumn{2}{c}{\textbf{Requirement Details}}\\ \hline
Requirement ID & CPT-OCS-INT-ICD-0009 \\ \cdashline{1-2}
Requirement Description &
\begin{minipage}[]{13cm}
\textbf{Specification:} The state machine definition shall explicitly
and formally define all transitions (external and internal), States, and
behaviors.\\
{{[}Options: UML State Machine diagram (preferred), table-based states
definition.{]}}
\end{minipage}
\\ \cdashline{1-2}
Requirement Priority &  \\ \cdashline{1-2}
Upper Level Requirement &
\begin{tabular}{cl}
CPT-OCS-INT-ICD-0010 & State-based Behavior \\
\end{tabular}
\\ \hline
\end{longtable}
}


  
 \newpage 
\subsection{[LVV-6987] CPT-OCS-INT-ICD-0072-V-06: State Machine Extension\_DM\_6 }\label{lvv-6987}

\begin{longtable}{cccc}
\hline
\textbf{Jira Link} & \textbf{Assignee} & \textbf{Status} & \textbf{Test Cases}\\ \hline
\href{https://jira.lsstcorp.org/browse/LVV-6987}{LVV-6987} &
Leanne Guy & Not Covered &
\begin{tabular}{c}
\end{tabular}
\\
\hline
\end{longtable}

\textbf{Verification Element Description:} \\
Undefined

{\footnotesize
\begin{longtable}{p{2.5cm}p{13.5cm}}
\hline
\multicolumn{2}{c}{\textbf{Requirement Details}}\\ \hline
Requirement ID & CPT-OCS-INT-ICD-0072 \\ \cdashline{1-2}
Requirement Description &
\begin{minipage}[]{13cm}
\textbf{Specification:} The state machine description shall incorporate
any extensions to the top-level state machine, in terms of additional
triggers and substates, in a detailed state machine that describes
behaviors (and variations to these) in response to external triggers.
\end{minipage}
\\ \cdashline{1-2}
Requirement Discussion &
\begin{minipage}[]{13cm}
\textbf{Discussion:} The Detailed State generally varies between
components. The Detailed State machine may be orthogonal to the
Top-Level State machine.
\end{minipage}
\\ \cdashline{1-2}
Requirement Priority &  \\ \cdashline{1-2}
Upper Level Requirement &
\begin{tabular}{cl}
CPT-OCS-INT-ICD-0002 & Common Summary States \\
\end{tabular}
\\ \hline
\end{longtable}
}


  
 \newpage 
\subsection{[LVV-6988] CPT-OCS-INT-ICD-0072-V-07: State Machine Extension\_DM\_7 }\label{lvv-6988}

\begin{longtable}{cccc}
\hline
\textbf{Jira Link} & \textbf{Assignee} & \textbf{Status} & \textbf{Test Cases}\\ \hline
\href{https://jira.lsstcorp.org/browse/LVV-6988}{LVV-6988} &
Leanne Guy & Not Covered &
\begin{tabular}{c}
\end{tabular}
\\
\hline
\end{longtable}

\textbf{Verification Element Description:} \\
Undefined

{\footnotesize
\begin{longtable}{p{2.5cm}p{13.5cm}}
\hline
\multicolumn{2}{c}{\textbf{Requirement Details}}\\ \hline
Requirement ID & CPT-OCS-INT-ICD-0072 \\ \cdashline{1-2}
Requirement Description &
\begin{minipage}[]{13cm}
\textbf{Specification:} The state machine description shall incorporate
any extensions to the top-level state machine, in terms of additional
triggers and substates, in a detailed state machine that describes
behaviors (and variations to these) in response to external triggers.
\end{minipage}
\\ \cdashline{1-2}
Requirement Discussion &
\begin{minipage}[]{13cm}
\textbf{Discussion:} The Detailed State generally varies between
components. The Detailed State machine may be orthogonal to the
Top-Level State machine.
\end{minipage}
\\ \cdashline{1-2}
Requirement Priority &  \\ \cdashline{1-2}
Upper Level Requirement &
\begin{tabular}{cl}
CPT-OCS-INT-ICD-0002 & Common Summary States \\
\end{tabular}
\\ \hline
\end{longtable}
}


  
 \newpage 
\subsection{[LVV-6993] CPT-OCS-INT-ICD-0010-V-06: State-based Behavior\_DM\_6 }\label{lvv-6993}

\begin{longtable}{cccc}
\hline
\textbf{Jira Link} & \textbf{Assignee} & \textbf{Status} & \textbf{Test Cases}\\ \hline
\href{https://jira.lsstcorp.org/browse/LVV-6993}{LVV-6993} &
Leanne Guy & Not Covered &
\begin{tabular}{c}
\end{tabular}
\\
\hline
\end{longtable}

\textbf{Verification Element Description:} \\
Undefined

{\footnotesize
\begin{longtable}{p{2.5cm}p{13.5cm}}
\hline
\multicolumn{2}{c}{\textbf{Requirement Details}}\\ \hline
Requirement ID & CPT-OCS-INT-ICD-0010 \\ \cdashline{1-2}
Requirement Description &
\begin{minipage}[]{13cm}
\textbf{Specification:} For any component that is, at least in part, a
reactive (state-based) system, the interface shall define the
state-based behaviors of the finite state machine. More formally, the
interface shall include a state machine definition as part of the
interface definition.
\end{minipage}
\\ \cdashline{1-2}
Requirement Priority &  \\ \cdashline{1-2}
Upper Level Requirement &
\begin{tabular}{cl}
CPT-OCS-INT-ICD-0006 & Interface Elements \\
\end{tabular}
\\ \hline
\end{longtable}
}


  
 \newpage 
\subsection{[LVV-6994] CPT-OCS-INT-ICD-0010-V-07: State-based Behavior\_DM\_7 }\label{lvv-6994}

\begin{longtable}{cccc}
\hline
\textbf{Jira Link} & \textbf{Assignee} & \textbf{Status} & \textbf{Test Cases}\\ \hline
\href{https://jira.lsstcorp.org/browse/LVV-6994}{LVV-6994} &
Leanne Guy & Not Covered &
\begin{tabular}{c}
\end{tabular}
\\
\hline
\end{longtable}

\textbf{Verification Element Description:} \\
Undefined

{\footnotesize
\begin{longtable}{p{2.5cm}p{13.5cm}}
\hline
\multicolumn{2}{c}{\textbf{Requirement Details}}\\ \hline
Requirement ID & CPT-OCS-INT-ICD-0010 \\ \cdashline{1-2}
Requirement Description &
\begin{minipage}[]{13cm}
\textbf{Specification:} For any component that is, at least in part, a
reactive (state-based) system, the interface shall define the
state-based behaviors of the finite state machine. More formally, the
interface shall include a state machine definition as part of the
interface definition.
\end{minipage}
\\ \cdashline{1-2}
Requirement Priority &  \\ \cdashline{1-2}
Upper Level Requirement &
\begin{tabular}{cl}
CPT-OCS-INT-ICD-0006 & Interface Elements \\
\end{tabular}
\\ \hline
\end{longtable}
}


  
 \newpage 
\subsection{[LVV-6999] CPT-OCS-INT-ICD-0012-V-06: Unique Names for States\_DM\_6 }\label{lvv-6999}

\begin{longtable}{cccc}
\hline
\textbf{Jira Link} & \textbf{Assignee} & \textbf{Status} & \textbf{Test Cases}\\ \hline
\href{https://jira.lsstcorp.org/browse/LVV-6999}{LVV-6999} &
Leanne Guy & Not Covered &
\begin{tabular}{c}
\end{tabular}
\\
\hline
\end{longtable}

\textbf{Verification Element Description:} \\
Undefined

{\footnotesize
\begin{longtable}{p{2.5cm}p{13.5cm}}
\hline
\multicolumn{2}{c}{\textbf{Requirement Details}}\\ \hline
Requirement ID & CPT-OCS-INT-ICD-0012 \\ \cdashline{1-2}
Requirement Description &
\begin{minipage}[]{13cm}
\textbf{Specification:} The triggers and states shall have names that
are unique within the component.
\end{minipage}
\\ \cdashline{1-2}
Requirement Discussion &
\begin{minipage}[]{13cm}
\textbf{Discussion:} Commonality of design, including names, between
components is desirable.
\end{minipage}
\\ \cdashline{1-2}
Requirement Priority &  \\ \cdashline{1-2}
Upper Level Requirement &
\begin{tabular}{cl}
CPT-OCS-INT-ICD-0009 & State Machine Description \\
\end{tabular}
\\ \hline
\end{longtable}
}


  
 \newpage 
\subsection{[LVV-7000] CPT-OCS-INT-ICD-0012-V-07: Unique Names for States\_DM\_7 }\label{lvv-7000}

\begin{longtable}{cccc}
\hline
\textbf{Jira Link} & \textbf{Assignee} & \textbf{Status} & \textbf{Test Cases}\\ \hline
\href{https://jira.lsstcorp.org/browse/LVV-7000}{LVV-7000} &
Leanne Guy & Not Covered &
\begin{tabular}{c}
\end{tabular}
\\
\hline
\end{longtable}

\textbf{Verification Element Description:} \\
Undefined

{\footnotesize
\begin{longtable}{p{2.5cm}p{13.5cm}}
\hline
\multicolumn{2}{c}{\textbf{Requirement Details}}\\ \hline
Requirement ID & CPT-OCS-INT-ICD-0012 \\ \cdashline{1-2}
Requirement Description &
\begin{minipage}[]{13cm}
\textbf{Specification:} The triggers and states shall have names that
are unique within the component.
\end{minipage}
\\ \cdashline{1-2}
Requirement Discussion &
\begin{minipage}[]{13cm}
\textbf{Discussion:} Commonality of design, including names, between
components is desirable.
\end{minipage}
\\ \cdashline{1-2}
Requirement Priority &  \\ \cdashline{1-2}
Upper Level Requirement &
\begin{tabular}{cl}
CPT-OCS-INT-ICD-0009 & State Machine Description \\
\end{tabular}
\\ \hline
\end{longtable}
}


  
 \newpage 
\subsection{[LVV-7005] CPT-OCS-INT-ICD-0004-V-06: Detailed State Publishing\_DM\_6 }\label{lvv-7005}

\begin{longtable}{cccc}
\hline
\textbf{Jira Link} & \textbf{Assignee} & \textbf{Status} & \textbf{Test Cases}\\ \hline
\href{https://jira.lsstcorp.org/browse/LVV-7005}{LVV-7005} &
Leanne Guy & Not Covered &
\begin{tabular}{c}
\end{tabular}
\\
\hline
\end{longtable}

\textbf{Verification Element Description:} \\
Undefined

{\footnotesize
\begin{longtable}{p{2.5cm}p{13.5cm}}
\hline
\multicolumn{2}{c}{\textbf{Requirement Details}}\\ \hline
Requirement ID & CPT-OCS-INT-ICD-0004 \\ \cdashline{1-2}
Requirement Description &
\begin{minipage}[]{13cm}
\textbf{Specification:} The component state machine definition shall
publish its leaf state (``Detailed State'') upon start-up and when this
value changes.
\end{minipage}
\\ \cdashline{1-2}
Requirement Priority &  \\ \cdashline{1-2}
Upper Level Requirement &
\begin{tabular}{cl}
CPT-OCS-INT-ICD-0007 & Publish State Information \\
\end{tabular}
\\ \hline
\end{longtable}
}


  
 \newpage 
\subsection{[LVV-7006] CPT-OCS-INT-ICD-0004-V-07: Detailed State Publishing\_DM\_7 }\label{lvv-7006}

\begin{longtable}{cccc}
\hline
\textbf{Jira Link} & \textbf{Assignee} & \textbf{Status} & \textbf{Test Cases}\\ \hline
\href{https://jira.lsstcorp.org/browse/LVV-7006}{LVV-7006} &
Leanne Guy & Not Covered &
\begin{tabular}{c}
\end{tabular}
\\
\hline
\end{longtable}

\textbf{Verification Element Description:} \\
Undefined

{\footnotesize
\begin{longtable}{p{2.5cm}p{13.5cm}}
\hline
\multicolumn{2}{c}{\textbf{Requirement Details}}\\ \hline
Requirement ID & CPT-OCS-INT-ICD-0004 \\ \cdashline{1-2}
Requirement Description &
\begin{minipage}[]{13cm}
\textbf{Specification:} The component state machine definition shall
publish its leaf state (``Detailed State'') upon start-up and when this
value changes.
\end{minipage}
\\ \cdashline{1-2}
Requirement Priority &  \\ \cdashline{1-2}
Upper Level Requirement &
\begin{tabular}{cl}
CPT-OCS-INT-ICD-0007 & Publish State Information \\
\end{tabular}
\\ \hline
\end{longtable}
}


  
 \newpage 
\subsection{[LVV-7011] CPT-OCS-INT-ICD-0007-V-06: Publish State Information\_DM\_6 }\label{lvv-7011}

\begin{longtable}{cccc}
\hline
\textbf{Jira Link} & \textbf{Assignee} & \textbf{Status} & \textbf{Test Cases}\\ \hline
\href{https://jira.lsstcorp.org/browse/LVV-7011}{LVV-7011} &
Leanne Guy & Not Covered &
\begin{tabular}{c}
\end{tabular}
\\
\hline
\end{longtable}

\textbf{Verification Element Description:} \\
Undefined

{\footnotesize
\begin{longtable}{p{2.5cm}p{13.5cm}}
\hline
\multicolumn{2}{c}{\textbf{Requirement Details}}\\ \hline
Requirement ID & CPT-OCS-INT-ICD-0007 \\ \cdashline{1-2}
Requirement Description &
\begin{minipage}[]{13cm}
\textbf{Specification:} Each component shall publish its current State
(in terms of a unique operational state) information.
\end{minipage}
\\ \cdashline{1-2}
Requirement Priority &  \\ \cdashline{1-2}
Upper Level Requirement &
\begin{tabular}{cl}
CPT-OCS-INT-ICD-0008 & Real-time Information \\
\end{tabular}
\\ \hline
\end{longtable}
}


  
 \newpage 
\subsection{[LVV-7012] CPT-OCS-INT-ICD-0007-V-07: Publish State Information\_DM\_7 }\label{lvv-7012}

\begin{longtable}{cccc}
\hline
\textbf{Jira Link} & \textbf{Assignee} & \textbf{Status} & \textbf{Test Cases}\\ \hline
\href{https://jira.lsstcorp.org/browse/LVV-7012}{LVV-7012} &
Leanne Guy & Not Covered &
\begin{tabular}{c}
\end{tabular}
\\
\hline
\end{longtable}

\textbf{Verification Element Description:} \\
Undefined

{\footnotesize
\begin{longtable}{p{2.5cm}p{13.5cm}}
\hline
\multicolumn{2}{c}{\textbf{Requirement Details}}\\ \hline
Requirement ID & CPT-OCS-INT-ICD-0007 \\ \cdashline{1-2}
Requirement Description &
\begin{minipage}[]{13cm}
\textbf{Specification:} Each component shall publish its current State
(in terms of a unique operational state) information.
\end{minipage}
\\ \cdashline{1-2}
Requirement Priority &  \\ \cdashline{1-2}
Upper Level Requirement &
\begin{tabular}{cl}
CPT-OCS-INT-ICD-0008 & Real-time Information \\
\end{tabular}
\\ \hline
\end{longtable}
}


  
 \newpage 
\subsection{[LVV-7017] CPT-OCS-INT-ICD-0011-V-06: Summary State Publishing\_DM\_6 }\label{lvv-7017}

\begin{longtable}{cccc}
\hline
\textbf{Jira Link} & \textbf{Assignee} & \textbf{Status} & \textbf{Test Cases}\\ \hline
\href{https://jira.lsstcorp.org/browse/LVV-7017}{LVV-7017} &
Leanne Guy & Not Covered &
\begin{tabular}{c}
\end{tabular}
\\
\hline
\end{longtable}

\textbf{Verification Element Description:} \\
Undefined

{\footnotesize
\begin{longtable}{p{2.5cm}p{13.5cm}}
\hline
\multicolumn{2}{c}{\textbf{Requirement Details}}\\ \hline
Requirement ID & CPT-OCS-INT-ICD-0011 \\ \cdashline{1-2}
Requirement Description &
\begin{minipage}[]{13cm}
\textbf{Specification:} Every interactive component shall publish its
Summary State (major branch on the state hierarchy tree) upon start-up
and when this value changes.
\end{minipage}
\\ \cdashline{1-2}
Requirement Priority &  \\ \cdashline{1-2}
Upper Level Requirement &
\begin{tabular}{cl}
CPT-OCS-INT-ICD-0007 & Publish State Information \\
\end{tabular}
\\ \hline
\end{longtable}
}


  
 \newpage 
\subsection{[LVV-7018] CPT-OCS-INT-ICD-0011-V-07: Summary State Publishing\_DM\_7 }\label{lvv-7018}

\begin{longtable}{cccc}
\hline
\textbf{Jira Link} & \textbf{Assignee} & \textbf{Status} & \textbf{Test Cases}\\ \hline
\href{https://jira.lsstcorp.org/browse/LVV-7018}{LVV-7018} &
Leanne Guy & Not Covered &
\begin{tabular}{c}
\end{tabular}
\\
\hline
\end{longtable}

\textbf{Verification Element Description:} \\
Undefined

{\footnotesize
\begin{longtable}{p{2.5cm}p{13.5cm}}
\hline
\multicolumn{2}{c}{\textbf{Requirement Details}}\\ \hline
Requirement ID & CPT-OCS-INT-ICD-0011 \\ \cdashline{1-2}
Requirement Description &
\begin{minipage}[]{13cm}
\textbf{Specification:} Every interactive component shall publish its
Summary State (major branch on the state hierarchy tree) upon start-up
and when this value changes.
\end{minipage}
\\ \cdashline{1-2}
Requirement Priority &  \\ \cdashline{1-2}
Upper Level Requirement &
\begin{tabular}{cl}
CPT-OCS-INT-ICD-0007 & Publish State Information \\
\end{tabular}
\\ \hline
\end{longtable}
}


  
 \newpage 
\subsection{[LVV-7023] CPT-OCS-INT-ICD-0049-V-06: Apply settings\_DM\_6 }\label{lvv-7023}

\begin{longtable}{cccc}
\hline
\textbf{Jira Link} & \textbf{Assignee} & \textbf{Status} & \textbf{Test Cases}\\ \hline
\href{https://jira.lsstcorp.org/browse/LVV-7023}{LVV-7023} &
Leanne Guy & Not Covered &
\begin{tabular}{c}
\end{tabular}
\\
\hline
\end{longtable}

\textbf{Verification Element Description:} \\
Undefined

{\footnotesize
\begin{longtable}{p{2.5cm}p{13.5cm}}
\hline
\multicolumn{2}{c}{\textbf{Requirement Details}}\\ \hline
Requirement ID & CPT-OCS-INT-ICD-0049 \\ \cdashline{1-2}
Requirement Description &
\begin{minipage}[]{13cm}
\textbf{Specification:} The component shall use a set of settings on
demand.
\end{minipage}
\\ \cdashline{1-2}
Requirement Discussion &
\begin{minipage}[]{13cm}
\textbf{Discussion:}
\end{minipage}
\\ \cdashline{1-2}
Requirement Priority &  \\ \cdashline{1-2}
Upper Level Requirement &
\begin{tabular}{cl}
CPT-OCS-INT-ICD-0071 & Manage settings \\
\end{tabular}
\\ \hline
\end{longtable}
}


  
 \newpage 
\subsection{[LVV-7024] CPT-OCS-INT-ICD-0049-V-07: Apply settings\_DM\_7 }\label{lvv-7024}

\begin{longtable}{cccc}
\hline
\textbf{Jira Link} & \textbf{Assignee} & \textbf{Status} & \textbf{Test Cases}\\ \hline
\href{https://jira.lsstcorp.org/browse/LVV-7024}{LVV-7024} &
Leanne Guy & Not Covered &
\begin{tabular}{c}
\end{tabular}
\\
\hline
\end{longtable}

\textbf{Verification Element Description:} \\
Undefined

{\footnotesize
\begin{longtable}{p{2.5cm}p{13.5cm}}
\hline
\multicolumn{2}{c}{\textbf{Requirement Details}}\\ \hline
Requirement ID & CPT-OCS-INT-ICD-0049 \\ \cdashline{1-2}
Requirement Description &
\begin{minipage}[]{13cm}
\textbf{Specification:} The component shall use a set of settings on
demand.
\end{minipage}
\\ \cdashline{1-2}
Requirement Discussion &
\begin{minipage}[]{13cm}
\textbf{Discussion:}
\end{minipage}
\\ \cdashline{1-2}
Requirement Priority &  \\ \cdashline{1-2}
Upper Level Requirement &
\begin{tabular}{cl}
CPT-OCS-INT-ICD-0071 & Manage settings \\
\end{tabular}
\\ \hline
\end{longtable}
}


  
 \newpage 
\subsection{[LVV-7029] CPT-OCS-INT-ICD-0071-V-06: Manage settings\_DM\_6 }\label{lvv-7029}

\begin{longtable}{cccc}
\hline
\textbf{Jira Link} & \textbf{Assignee} & \textbf{Status} & \textbf{Test Cases}\\ \hline
\href{https://jira.lsstcorp.org/browse/LVV-7029}{LVV-7029} &
Leanne Guy & Not Covered &
\begin{tabular}{c}
\end{tabular}
\\
\hline
\end{longtable}

\textbf{Verification Element Description:} \\
Undefined

{\footnotesize
\begin{longtable}{p{2.5cm}p{13.5cm}}
\hline
\multicolumn{2}{c}{\textbf{Requirement Details}}\\ \hline
Requirement ID & CPT-OCS-INT-ICD-0071 \\ \cdashline{1-2}
Requirement Description &
\begin{minipage}[]{13cm}
\textbf{Specification:} The component shall manage settings.
\end{minipage}
\\ \cdashline{1-2}
Requirement Discussion &
\begin{minipage}[]{13cm}
\textbf{Discussion:}
\end{minipage}
\\ \cdashline{1-2}
Requirement Priority &  \\ \cdashline{1-2}
Upper Level Requirement &
\begin{tabular}{cl}
\end{tabular}
\\ \hline
\end{longtable}
}


  
 \newpage 
\subsection{[LVV-7030] CPT-OCS-INT-ICD-0071-V-07: Manage settings\_DM\_7 }\label{lvv-7030}

\begin{longtable}{cccc}
\hline
\textbf{Jira Link} & \textbf{Assignee} & \textbf{Status} & \textbf{Test Cases}\\ \hline
\href{https://jira.lsstcorp.org/browse/LVV-7030}{LVV-7030} &
Leanne Guy & Not Covered &
\begin{tabular}{c}
\end{tabular}
\\
\hline
\end{longtable}

\textbf{Verification Element Description:} \\
Undefined

{\footnotesize
\begin{longtable}{p{2.5cm}p{13.5cm}}
\hline
\multicolumn{2}{c}{\textbf{Requirement Details}}\\ \hline
Requirement ID & CPT-OCS-INT-ICD-0071 \\ \cdashline{1-2}
Requirement Description &
\begin{minipage}[]{13cm}
\textbf{Specification:} The component shall manage settings.
\end{minipage}
\\ \cdashline{1-2}
Requirement Discussion &
\begin{minipage}[]{13cm}
\textbf{Discussion:}
\end{minipage}
\\ \cdashline{1-2}
Requirement Priority &  \\ \cdashline{1-2}
Upper Level Requirement &
\begin{tabular}{cl}
\end{tabular}
\\ \hline
\end{longtable}
}


  
 \newpage 
\subsection{[LVV-7035] CPT-OCS-INT-ICD-0046-V-06: Notify that settings differ from start
values\_DM\_6 }\label{lvv-7035}

\begin{longtable}{cccc}
\hline
\textbf{Jira Link} & \textbf{Assignee} & \textbf{Status} & \textbf{Test Cases}\\ \hline
\href{https://jira.lsstcorp.org/browse/LVV-7035}{LVV-7035} &
Leanne Guy & Not Covered &
\begin{tabular}{c}
\end{tabular}
\\
\hline
\end{longtable}

\textbf{Verification Element Description:} \\
Undefined

{\footnotesize
\begin{longtable}{p{2.5cm}p{13.5cm}}
\hline
\multicolumn{2}{c}{\textbf{Requirement Details}}\\ \hline
Requirement ID & CPT-OCS-INT-ICD-0046 \\ \cdashline{1-2}
Requirement Description &
\begin{minipage}[]{13cm}
\textbf{Specification:} The component shall publish an event once one or
more settings differ from the settings applied on the Start transition.
\end{minipage}
\\ \cdashline{1-2}
Requirement Discussion &
\begin{minipage}[]{13cm}
\textbf{Discussion:}
\end{minipage}
\\ \cdashline{1-2}
Requirement Priority &  \\ \cdashline{1-2}
Upper Level Requirement &
\begin{tabular}{cl}
CPT-OCS-INT-ICD-0048 & Support limited apply settings while enabled \\
\end{tabular}
\\ \hline
\end{longtable}
}


  
 \newpage 
\subsection{[LVV-7036] CPT-OCS-INT-ICD-0046-V-07: Notify that settings differ from start
values\_DM\_7 }\label{lvv-7036}

\begin{longtable}{cccc}
\hline
\textbf{Jira Link} & \textbf{Assignee} & \textbf{Status} & \textbf{Test Cases}\\ \hline
\href{https://jira.lsstcorp.org/browse/LVV-7036}{LVV-7036} &
Leanne Guy & Not Covered &
\begin{tabular}{c}
\end{tabular}
\\
\hline
\end{longtable}

\textbf{Verification Element Description:} \\
Undefined

{\footnotesize
\begin{longtable}{p{2.5cm}p{13.5cm}}
\hline
\multicolumn{2}{c}{\textbf{Requirement Details}}\\ \hline
Requirement ID & CPT-OCS-INT-ICD-0046 \\ \cdashline{1-2}
Requirement Description &
\begin{minipage}[]{13cm}
\textbf{Specification:} The component shall publish an event once one or
more settings differ from the settings applied on the Start transition.
\end{minipage}
\\ \cdashline{1-2}
Requirement Discussion &
\begin{minipage}[]{13cm}
\textbf{Discussion:}
\end{minipage}
\\ \cdashline{1-2}
Requirement Priority &  \\ \cdashline{1-2}
Upper Level Requirement &
\begin{tabular}{cl}
CPT-OCS-INT-ICD-0048 & Support limited apply settings while enabled \\
\end{tabular}
\\ \hline
\end{longtable}
}


  
 \newpage 
\subsection{[LVV-7041] CPT-OCS-INT-ICD-0045-V-06: Record applied settings\_DM\_6 }\label{lvv-7041}

\begin{longtable}{cccc}
\hline
\textbf{Jira Link} & \textbf{Assignee} & \textbf{Status} & \textbf{Test Cases}\\ \hline
\href{https://jira.lsstcorp.org/browse/LVV-7041}{LVV-7041} &
Leanne Guy & Not Covered &
\begin{tabular}{c}
\end{tabular}
\\
\hline
\end{longtable}

\textbf{Verification Element Description:} \\
Undefined

{\footnotesize
\begin{longtable}{p{2.5cm}p{13.5cm}}
\hline
\multicolumn{2}{c}{\textbf{Requirement Details}}\\ \hline
Requirement ID & CPT-OCS-INT-ICD-0045 \\ \cdashline{1-2}
Requirement Description &
\begin{minipage}[]{13cm}
\textbf{Specification:} When a component applies a set of settings, the
component shall record this event, along with the applied settings set.
\end{minipage}
\\ \cdashline{1-2}
Requirement Discussion &
\begin{minipage}[]{13cm}
\textbf{Discussion:}
\end{minipage}
\\ \cdashline{1-2}
Requirement Priority &  \\ \cdashline{1-2}
Upper Level Requirement &
\begin{tabular}{cl}
\end{tabular}
\\ \hline
\end{longtable}
}


  
 \newpage 
\subsection{[LVV-7042] CPT-OCS-INT-ICD-0045-V-07: Record applied settings\_DM\_7 }\label{lvv-7042}

\begin{longtable}{cccc}
\hline
\textbf{Jira Link} & \textbf{Assignee} & \textbf{Status} & \textbf{Test Cases}\\ \hline
\href{https://jira.lsstcorp.org/browse/LVV-7042}{LVV-7042} &
Leanne Guy & Not Covered &
\begin{tabular}{c}
\end{tabular}
\\
\hline
\end{longtable}

\textbf{Verification Element Description:} \\
Undefined

{\footnotesize
\begin{longtable}{p{2.5cm}p{13.5cm}}
\hline
\multicolumn{2}{c}{\textbf{Requirement Details}}\\ \hline
Requirement ID & CPT-OCS-INT-ICD-0045 \\ \cdashline{1-2}
Requirement Description &
\begin{minipage}[]{13cm}
\textbf{Specification:} When a component applies a set of settings, the
component shall record this event, along with the applied settings set.
\end{minipage}
\\ \cdashline{1-2}
Requirement Discussion &
\begin{minipage}[]{13cm}
\textbf{Discussion:}
\end{minipage}
\\ \cdashline{1-2}
Requirement Priority &  \\ \cdashline{1-2}
Upper Level Requirement &
\begin{tabular}{cl}
\end{tabular}
\\ \hline
\end{longtable}
}


  
 \newpage 
\subsection{[LVV-7047] CPT-OCS-INT-ICD-0048-V-06: Support limited apply settings while
enabled\_DM\_6 }\label{lvv-7047}

\begin{longtable}{cccc}
\hline
\textbf{Jira Link} & \textbf{Assignee} & \textbf{Status} & \textbf{Test Cases}\\ \hline
\href{https://jira.lsstcorp.org/browse/LVV-7047}{LVV-7047} &
Leanne Guy & Not Covered &
\begin{tabular}{c}
\end{tabular}
\\
\hline
\end{longtable}

\textbf{Verification Element Description:} \\
Undefined

{\footnotesize
\begin{longtable}{p{2.5cm}p{13.5cm}}
\hline
\multicolumn{2}{c}{\textbf{Requirement Details}}\\ \hline
Requirement ID & CPT-OCS-INT-ICD-0048 \\ \cdashline{1-2}
Requirement Description &
\begin{minipage}[]{13cm}
\textbf{Specification:} The component shall support applying a subset of
editable settings while in Enabled State.
\end{minipage}
\\ \cdashline{1-2}
Requirement Discussion &
\begin{minipage}[]{13cm}
\textbf{Discussion:}
\end{minipage}
\\ \cdashline{1-2}
Requirement Priority &  \\ \cdashline{1-2}
Upper Level Requirement &
\begin{tabular}{cl}
CPT-OCS-INT-ICD-0049 & Apply settings \\
CPT-OCS-INT-ICD-0060 & Support partial editing while enabled \\
\end{tabular}
\\ \hline
\end{longtable}
}


  
 \newpage 
\subsection{[LVV-7048] CPT-OCS-INT-ICD-0048-V-07: Support limited apply settings while
enabled\_DM\_7 }\label{lvv-7048}

\begin{longtable}{cccc}
\hline
\textbf{Jira Link} & \textbf{Assignee} & \textbf{Status} & \textbf{Test Cases}\\ \hline
\href{https://jira.lsstcorp.org/browse/LVV-7048}{LVV-7048} &
Leanne Guy & Not Covered &
\begin{tabular}{c}
\end{tabular}
\\
\hline
\end{longtable}

\textbf{Verification Element Description:} \\
Undefined

{\footnotesize
\begin{longtable}{p{2.5cm}p{13.5cm}}
\hline
\multicolumn{2}{c}{\textbf{Requirement Details}}\\ \hline
Requirement ID & CPT-OCS-INT-ICD-0048 \\ \cdashline{1-2}
Requirement Description &
\begin{minipage}[]{13cm}
\textbf{Specification:} The component shall support applying a subset of
editable settings while in Enabled State.
\end{minipage}
\\ \cdashline{1-2}
Requirement Discussion &
\begin{minipage}[]{13cm}
\textbf{Discussion:}
\end{minipage}
\\ \cdashline{1-2}
Requirement Priority &  \\ \cdashline{1-2}
Upper Level Requirement &
\begin{tabular}{cl}
CPT-OCS-INT-ICD-0049 & Apply settings \\
CPT-OCS-INT-ICD-0060 & Support partial editing while enabled \\
\end{tabular}
\\ \hline
\end{longtable}
}


  
 \newpage 
\subsection{[LVV-7053] CPT-OCS-INT-ICD-0043-V-06: Publish large file object settings reference
to SAL topic\_DM\_6 }\label{lvv-7053}

\begin{longtable}{cccc}
\hline
\textbf{Jira Link} & \textbf{Assignee} & \textbf{Status} & \textbf{Test Cases}\\ \hline
\href{https://jira.lsstcorp.org/browse/LVV-7053}{LVV-7053} &
Leanne Guy & Not Covered &
\begin{tabular}{c}
\end{tabular}
\\
\hline
\end{longtable}

\textbf{Verification Element Description:} \\
Undefined

{\footnotesize
\begin{longtable}{p{2.5cm}p{13.5cm}}
\hline
\multicolumn{2}{c}{\textbf{Requirement Details}}\\ \hline
Requirement ID & CPT-OCS-INT-ICD-0043 \\ \cdashline{1-2}
Requirement Description &
\begin{minipage}[]{13cm}
\textbf{Specification:} For settings captured in Large File Objects, the
component shall publish the data using the Large File Object mechanism
in \citeds{LTS-210} (``Engineering and Facility Database Design Document'').
\end{minipage}
\\ \cdashline{1-2}
Requirement Discussion &
\begin{minipage}[]{13cm}
\textbf{Discussion:}
\end{minipage}
\\ \cdashline{1-2}
Requirement Priority &  \\ \cdashline{1-2}
Upper Level Requirement &
\begin{tabular}{cl}
CPT-OCS-INT-ICD-0044 & Publish settings to SAL topic \\
\end{tabular}
\\ \hline
\end{longtable}
}


  
 \newpage 
\subsection{[LVV-7054] CPT-OCS-INT-ICD-0043-V-07: Publish large file object settings reference
to SAL topic\_DM\_7 }\label{lvv-7054}

\begin{longtable}{cccc}
\hline
\textbf{Jira Link} & \textbf{Assignee} & \textbf{Status} & \textbf{Test Cases}\\ \hline
\href{https://jira.lsstcorp.org/browse/LVV-7054}{LVV-7054} &
Leanne Guy & Not Covered &
\begin{tabular}{c}
\end{tabular}
\\
\hline
\end{longtable}

\textbf{Verification Element Description:} \\
Undefined

{\footnotesize
\begin{longtable}{p{2.5cm}p{13.5cm}}
\hline
\multicolumn{2}{c}{\textbf{Requirement Details}}\\ \hline
Requirement ID & CPT-OCS-INT-ICD-0043 \\ \cdashline{1-2}
Requirement Description &
\begin{minipage}[]{13cm}
\textbf{Specification:} For settings captured in Large File Objects, the
component shall publish the data using the Large File Object mechanism
in \citeds{LTS-210} (``Engineering and Facility Database Design Document'').
\end{minipage}
\\ \cdashline{1-2}
Requirement Discussion &
\begin{minipage}[]{13cm}
\textbf{Discussion:}
\end{minipage}
\\ \cdashline{1-2}
Requirement Priority &  \\ \cdashline{1-2}
Upper Level Requirement &
\begin{tabular}{cl}
CPT-OCS-INT-ICD-0044 & Publish settings to SAL topic \\
\end{tabular}
\\ \hline
\end{longtable}
}


  
 \newpage 
\subsection{[LVV-7059] CPT-OCS-INT-ICD-0044-V-06: Publish settings to SAL topic\_DM\_6 }\label{lvv-7059}

\begin{longtable}{cccc}
\hline
\textbf{Jira Link} & \textbf{Assignee} & \textbf{Status} & \textbf{Test Cases}\\ \hline
\href{https://jira.lsstcorp.org/browse/LVV-7059}{LVV-7059} &
Leanne Guy & Not Covered &
\begin{tabular}{c}
\end{tabular}
\\
\hline
\end{longtable}

\textbf{Verification Element Description:} \\
Undefined

{\footnotesize
\begin{longtable}{p{2.5cm}p{13.5cm}}
\hline
\multicolumn{2}{c}{\textbf{Requirement Details}}\\ \hline
Requirement ID & CPT-OCS-INT-ICD-0044 \\ \cdashline{1-2}
Requirement Description &
\begin{minipage}[]{13cm}
\textbf{Specification:} The component shall include the full set of
values in the event topic.
\end{minipage}
\\ \cdashline{1-2}
Requirement Discussion &
\begin{minipage}[]{13cm}
\textbf{Discussion:}
\end{minipage}
\\ \cdashline{1-2}
Requirement Priority &  \\ \cdashline{1-2}
Upper Level Requirement &
\begin{tabular}{cl}
CPT-OCS-INT-ICD-0045 & Record applied settings \\
\end{tabular}
\\ \hline
\end{longtable}
}


  
 \newpage 
\subsection{[LVV-7060] CPT-OCS-INT-ICD-0044-V-07: Publish settings to SAL topic\_DM\_7 }\label{lvv-7060}

\begin{longtable}{cccc}
\hline
\textbf{Jira Link} & \textbf{Assignee} & \textbf{Status} & \textbf{Test Cases}\\ \hline
\href{https://jira.lsstcorp.org/browse/LVV-7060}{LVV-7060} &
Leanne Guy & Not Covered &
\begin{tabular}{c}
\end{tabular}
\\
\hline
\end{longtable}

\textbf{Verification Element Description:} \\
Undefined

{\footnotesize
\begin{longtable}{p{2.5cm}p{13.5cm}}
\hline
\multicolumn{2}{c}{\textbf{Requirement Details}}\\ \hline
Requirement ID & CPT-OCS-INT-ICD-0044 \\ \cdashline{1-2}
Requirement Description &
\begin{minipage}[]{13cm}
\textbf{Specification:} The component shall include the full set of
values in the event topic.
\end{minipage}
\\ \cdashline{1-2}
Requirement Discussion &
\begin{minipage}[]{13cm}
\textbf{Discussion:}
\end{minipage}
\\ \cdashline{1-2}
Requirement Priority &  \\ \cdashline{1-2}
Upper Level Requirement &
\begin{tabular}{cl}
CPT-OCS-INT-ICD-0045 & Record applied settings \\
\end{tabular}
\\ \hline
\end{longtable}
}


  
 \newpage 
\subsection{[LVV-7065] CPT-OCS-INT-ICD-0047-V-06: Reset setting difference\_DM\_6 }\label{lvv-7065}

\begin{longtable}{cccc}
\hline
\textbf{Jira Link} & \textbf{Assignee} & \textbf{Status} & \textbf{Test Cases}\\ \hline
\href{https://jira.lsstcorp.org/browse/LVV-7065}{LVV-7065} &
Leanne Guy & Not Covered &
\begin{tabular}{c}
\end{tabular}
\\
\hline
\end{longtable}

\textbf{Verification Element Description:} \\
Undefined

{\footnotesize
\begin{longtable}{p{2.5cm}p{13.5cm}}
\hline
\multicolumn{2}{c}{\textbf{Requirement Details}}\\ \hline
Requirement ID & CPT-OCS-INT-ICD-0047 \\ \cdashline{1-2}
Requirement Description &
\begin{minipage}[]{13cm}
\textbf{Specification:} On the Start transition, the component shall
publish an event marking that the settings are identical to those on the
Start transition.
\end{minipage}
\\ \cdashline{1-2}
Requirement Discussion &
\begin{minipage}[]{13cm}
\textbf{Discussion:}
\end{minipage}
\\ \cdashline{1-2}
Requirement Priority &  \\ \cdashline{1-2}
Upper Level Requirement &
\begin{tabular}{cl}
CPT-OCS-INT-ICD-0046 & Notify that settings differ from start values \\
\end{tabular}
\\ \hline
\end{longtable}
}


  
 \newpage 
\subsection{[LVV-7066] CPT-OCS-INT-ICD-0047-V-07: Reset setting difference\_DM\_7 }\label{lvv-7066}

\begin{longtable}{cccc}
\hline
\textbf{Jira Link} & \textbf{Assignee} & \textbf{Status} & \textbf{Test Cases}\\ \hline
\href{https://jira.lsstcorp.org/browse/LVV-7066}{LVV-7066} &
Leanne Guy & Not Covered &
\begin{tabular}{c}
\end{tabular}
\\
\hline
\end{longtable}

\textbf{Verification Element Description:} \\
Undefined

{\footnotesize
\begin{longtable}{p{2.5cm}p{13.5cm}}
\hline
\multicolumn{2}{c}{\textbf{Requirement Details}}\\ \hline
Requirement ID & CPT-OCS-INT-ICD-0047 \\ \cdashline{1-2}
Requirement Description &
\begin{minipage}[]{13cm}
\textbf{Specification:} On the Start transition, the component shall
publish an event marking that the settings are identical to those on the
Start transition.
\end{minipage}
\\ \cdashline{1-2}
Requirement Discussion &
\begin{minipage}[]{13cm}
\textbf{Discussion:}
\end{minipage}
\\ \cdashline{1-2}
Requirement Priority &  \\ \cdashline{1-2}
Upper Level Requirement &
\begin{tabular}{cl}
CPT-OCS-INT-ICD-0046 & Notify that settings differ from start values \\
\end{tabular}
\\ \hline
\end{longtable}
}


  
 \newpage 
\subsection{[LVV-7071] CPT-OCS-INT-ICD-0061-V-06: Edit editable settings\_DM\_6 }\label{lvv-7071}

\begin{longtable}{cccc}
\hline
\textbf{Jira Link} & \textbf{Assignee} & \textbf{Status} & \textbf{Test Cases}\\ \hline
\href{https://jira.lsstcorp.org/browse/LVV-7071}{LVV-7071} &
Leanne Guy & Not Covered &
\begin{tabular}{c}
\end{tabular}
\\
\hline
\end{longtable}

\textbf{Verification Element Description:} \\
Undefined

{\footnotesize
\begin{longtable}{p{2.5cm}p{13.5cm}}
\hline
\multicolumn{2}{c}{\textbf{Requirement Details}}\\ \hline
Requirement ID & CPT-OCS-INT-ICD-0061 \\ \cdashline{1-2}
Requirement Description &
\begin{minipage}[]{13cm}
\textbf{Specification:} A user shall be able to edit editable settings.
\end{minipage}
\\ \cdashline{1-2}
Requirement Discussion &
\begin{minipage}[]{13cm}
\textbf{Discussion:}
\end{minipage}
\\ \cdashline{1-2}
Requirement Priority &  \\ \cdashline{1-2}
Upper Level Requirement &
\begin{tabular}{cl}
CPT-OCS-INT-ICD-0071 & Manage settings \\
\end{tabular}
\\ \hline
\end{longtable}
}


  
 \newpage 
\subsection{[LVV-7072] CPT-OCS-INT-ICD-0061-V-07: Edit editable settings\_DM\_7 }\label{lvv-7072}

\begin{longtable}{cccc}
\hline
\textbf{Jira Link} & \textbf{Assignee} & \textbf{Status} & \textbf{Test Cases}\\ \hline
\href{https://jira.lsstcorp.org/browse/LVV-7072}{LVV-7072} &
Leanne Guy & Not Covered &
\begin{tabular}{c}
\end{tabular}
\\
\hline
\end{longtable}

\textbf{Verification Element Description:} \\
Undefined

{\footnotesize
\begin{longtable}{p{2.5cm}p{13.5cm}}
\hline
\multicolumn{2}{c}{\textbf{Requirement Details}}\\ \hline
Requirement ID & CPT-OCS-INT-ICD-0061 \\ \cdashline{1-2}
Requirement Description &
\begin{minipage}[]{13cm}
\textbf{Specification:} A user shall be able to edit editable settings.
\end{minipage}
\\ \cdashline{1-2}
Requirement Discussion &
\begin{minipage}[]{13cm}
\textbf{Discussion:}
\end{minipage}
\\ \cdashline{1-2}
Requirement Priority &  \\ \cdashline{1-2}
Upper Level Requirement &
\begin{tabular}{cl}
CPT-OCS-INT-ICD-0071 & Manage settings \\
\end{tabular}
\\ \hline
\end{longtable}
}


  
 \newpage 
\subsection{[LVV-7077] CPT-OCS-INT-ICD-0057-V-06: Editor user interface\_DM\_6 }\label{lvv-7077}

\begin{longtable}{cccc}
\hline
\textbf{Jira Link} & \textbf{Assignee} & \textbf{Status} & \textbf{Test Cases}\\ \hline
\href{https://jira.lsstcorp.org/browse/LVV-7077}{LVV-7077} &
Leanne Guy & Not Covered &
\begin{tabular}{c}
\end{tabular}
\\
\hline
\end{longtable}

\textbf{Verification Element Description:} \\
Undefined

{\footnotesize
\begin{longtable}{p{2.5cm}p{13.5cm}}
\hline
\multicolumn{2}{c}{\textbf{Requirement Details}}\\ \hline
Requirement ID & CPT-OCS-INT-ICD-0057 \\ \cdashline{1-2}
Requirement Description &
\begin{minipage}[]{13cm}
\textbf{Specification:} The component shall provide a user interface to
support editing editable settings.
\end{minipage}
\\ \cdashline{1-2}
Requirement Discussion &
\begin{minipage}[]{13cm}
\textbf{Discussion:}
\end{minipage}
\\ \cdashline{1-2}
Requirement Priority &  \\ \cdashline{1-2}
Upper Level Requirement &
\begin{tabular}{cl}
CPT-OCS-INT-ICD-0061 & Edit editable settings \\
CPT-OCS-INT-ICD-0058 & Support editing settings in quiescent states \\
CPT-OCS-INT-ICD-0060 & Support partial editing while enabled \\
\end{tabular}
\\ \hline
\end{longtable}
}


  
 \newpage 
\subsection{[LVV-7078] CPT-OCS-INT-ICD-0057-V-07: Editor user interface\_DM\_7 }\label{lvv-7078}

\begin{longtable}{cccc}
\hline
\textbf{Jira Link} & \textbf{Assignee} & \textbf{Status} & \textbf{Test Cases}\\ \hline
\href{https://jira.lsstcorp.org/browse/LVV-7078}{LVV-7078} &
Leanne Guy & Not Covered &
\begin{tabular}{c}
\end{tabular}
\\
\hline
\end{longtable}

\textbf{Verification Element Description:} \\
Undefined

{\footnotesize
\begin{longtable}{p{2.5cm}p{13.5cm}}
\hline
\multicolumn{2}{c}{\textbf{Requirement Details}}\\ \hline
Requirement ID & CPT-OCS-INT-ICD-0057 \\ \cdashline{1-2}
Requirement Description &
\begin{minipage}[]{13cm}
\textbf{Specification:} The component shall provide a user interface to
support editing editable settings.
\end{minipage}
\\ \cdashline{1-2}
Requirement Discussion &
\begin{minipage}[]{13cm}
\textbf{Discussion:}
\end{minipage}
\\ \cdashline{1-2}
Requirement Priority &  \\ \cdashline{1-2}
Upper Level Requirement &
\begin{tabular}{cl}
CPT-OCS-INT-ICD-0061 & Edit editable settings \\
CPT-OCS-INT-ICD-0058 & Support editing settings in quiescent states \\
CPT-OCS-INT-ICD-0060 & Support partial editing while enabled \\
\end{tabular}
\\ \hline
\end{longtable}
}


  
 \newpage 
\subsection{[LVV-7083] CPT-OCS-INT-ICD-0052-V-06: Local store\_DM\_6 }\label{lvv-7083}

\begin{longtable}{cccc}
\hline
\textbf{Jira Link} & \textbf{Assignee} & \textbf{Status} & \textbf{Test Cases}\\ \hline
\href{https://jira.lsstcorp.org/browse/LVV-7083}{LVV-7083} &
Leanne Guy & Not Covered &
\begin{tabular}{c}
\end{tabular}
\\
\hline
\end{longtable}

\textbf{Verification Element Description:} \\
Undefined

{\footnotesize
\begin{longtable}{p{2.5cm}p{13.5cm}}
\hline
\multicolumn{2}{c}{\textbf{Requirement Details}}\\ \hline
Requirement ID & CPT-OCS-INT-ICD-0052 \\ \cdashline{1-2}
Requirement Description &
\begin{minipage}[]{13cm}
\textbf{Specification:} Each component shall maintain its own store for
setting values.
\end{minipage}
\\ \cdashline{1-2}
Requirement Priority &  \\ \cdashline{1-2}
Upper Level Requirement &
\begin{tabular}{cl}
\end{tabular}
\\ \hline
\end{longtable}
}


  
 \newpage 
\subsection{[LVV-7084] CPT-OCS-INT-ICD-0052-V-07: Local store\_DM\_7 }\label{lvv-7084}

\begin{longtable}{cccc}
\hline
\textbf{Jira Link} & \textbf{Assignee} & \textbf{Status} & \textbf{Test Cases}\\ \hline
\href{https://jira.lsstcorp.org/browse/LVV-7084}{LVV-7084} &
Leanne Guy & Not Covered &
\begin{tabular}{c}
\end{tabular}
\\
\hline
\end{longtable}

\textbf{Verification Element Description:} \\
Undefined

{\footnotesize
\begin{longtable}{p{2.5cm}p{13.5cm}}
\hline
\multicolumn{2}{c}{\textbf{Requirement Details}}\\ \hline
Requirement ID & CPT-OCS-INT-ICD-0052 \\ \cdashline{1-2}
Requirement Description &
\begin{minipage}[]{13cm}
\textbf{Specification:} Each component shall maintain its own store for
setting values.
\end{minipage}
\\ \cdashline{1-2}
Requirement Priority &  \\ \cdashline{1-2}
Upper Level Requirement &
\begin{tabular}{cl}
\end{tabular}
\\ \hline
\end{longtable}
}


  
 \newpage 
\subsection{[LVV-7089] CPT-OCS-INT-ICD-0050-V-06: Modify setting values\_DM\_6 }\label{lvv-7089}

\begin{longtable}{cccc}
\hline
\textbf{Jira Link} & \textbf{Assignee} & \textbf{Status} & \textbf{Test Cases}\\ \hline
\href{https://jira.lsstcorp.org/browse/LVV-7089}{LVV-7089} &
Leanne Guy & Not Covered &
\begin{tabular}{c}
\end{tabular}
\\
\hline
\end{longtable}

\textbf{Verification Element Description:} \\
Undefined

{\footnotesize
\begin{longtable}{p{2.5cm}p{13.5cm}}
\hline
\multicolumn{2}{c}{\textbf{Requirement Details}}\\ \hline
Requirement ID & CPT-OCS-INT-ICD-0050 \\ \cdashline{1-2}
Requirement Description &
\begin{minipage}[]{13cm}
\textbf{Specification:} The component shall provide a mechanism for a
user to change the values of editable settings.
\end{minipage}
\\ \cdashline{1-2}
Requirement Discussion &
\begin{minipage}[]{13cm}
\textbf{Discussion:}
\end{minipage}
\\ \cdashline{1-2}
Requirement Priority &  \\ \cdashline{1-2}
Upper Level Requirement &
\begin{tabular}{cl}
\end{tabular}
\\ \hline
\end{longtable}
}


  
 \newpage 
\subsection{[LVV-7090] CPT-OCS-INT-ICD-0050-V-07: Modify setting values\_DM\_7 }\label{lvv-7090}

\begin{longtable}{cccc}
\hline
\textbf{Jira Link} & \textbf{Assignee} & \textbf{Status} & \textbf{Test Cases}\\ \hline
\href{https://jira.lsstcorp.org/browse/LVV-7090}{LVV-7090} &
Leanne Guy & Not Covered &
\begin{tabular}{c}
\end{tabular}
\\
\hline
\end{longtable}

\textbf{Verification Element Description:} \\
Undefined

{\footnotesize
\begin{longtable}{p{2.5cm}p{13.5cm}}
\hline
\multicolumn{2}{c}{\textbf{Requirement Details}}\\ \hline
Requirement ID & CPT-OCS-INT-ICD-0050 \\ \cdashline{1-2}
Requirement Description &
\begin{minipage}[]{13cm}
\textbf{Specification:} The component shall provide a mechanism for a
user to change the values of editable settings.
\end{minipage}
\\ \cdashline{1-2}
Requirement Discussion &
\begin{minipage}[]{13cm}
\textbf{Discussion:}
\end{minipage}
\\ \cdashline{1-2}
Requirement Priority &  \\ \cdashline{1-2}
Upper Level Requirement &
\begin{tabular}{cl}
\end{tabular}
\\ \hline
\end{longtable}
}


  
 \newpage 
\subsection{[LVV-7095] CPT-OCS-INT-ICD-0053-V-06: Publish store list on change\_DM\_6 }\label{lvv-7095}

\begin{longtable}{cccc}
\hline
\textbf{Jira Link} & \textbf{Assignee} & \textbf{Status} & \textbf{Test Cases}\\ \hline
\href{https://jira.lsstcorp.org/browse/LVV-7095}{LVV-7095} &
Leanne Guy & Not Covered &
\begin{tabular}{c}
\end{tabular}
\\
\hline
\end{longtable}

\textbf{Verification Element Description:} \\
Undefined

{\footnotesize
\begin{longtable}{p{2.5cm}p{13.5cm}}
\hline
\multicolumn{2}{c}{\textbf{Requirement Details}}\\ \hline
Requirement ID & CPT-OCS-INT-ICD-0053 \\ \cdashline{1-2}
Requirement Description &
\begin{minipage}[]{13cm}
\textbf{Specification:} While the component is connected to SAL/DDS, the
component shall publish a list of recommended setting versions when the
versions in the store change.
\end{minipage}
\\ \cdashline{1-2}
Requirement Priority &  \\ \cdashline{1-2}
Upper Level Requirement &
\begin{tabular}{cl}
CPT-OCS-INT-ICD-0054 & Publish store list on start-up \\
\end{tabular}
\\ \hline
\end{longtable}
}


  
 \newpage 
\subsection{[LVV-7096] CPT-OCS-INT-ICD-0053-V-07: Publish store list on change\_DM\_7 }\label{lvv-7096}

\begin{longtable}{cccc}
\hline
\textbf{Jira Link} & \textbf{Assignee} & \textbf{Status} & \textbf{Test Cases}\\ \hline
\href{https://jira.lsstcorp.org/browse/LVV-7096}{LVV-7096} &
Leanne Guy & Not Covered &
\begin{tabular}{c}
\end{tabular}
\\
\hline
\end{longtable}

\textbf{Verification Element Description:} \\
Undefined

{\footnotesize
\begin{longtable}{p{2.5cm}p{13.5cm}}
\hline
\multicolumn{2}{c}{\textbf{Requirement Details}}\\ \hline
Requirement ID & CPT-OCS-INT-ICD-0053 \\ \cdashline{1-2}
Requirement Description &
\begin{minipage}[]{13cm}
\textbf{Specification:} While the component is connected to SAL/DDS, the
component shall publish a list of recommended setting versions when the
versions in the store change.
\end{minipage}
\\ \cdashline{1-2}
Requirement Priority &  \\ \cdashline{1-2}
Upper Level Requirement &
\begin{tabular}{cl}
CPT-OCS-INT-ICD-0054 & Publish store list on start-up \\
\end{tabular}
\\ \hline
\end{longtable}
}


  
 \newpage 
\subsection{[LVV-7101] CPT-OCS-INT-ICD-0054-V-06: Publish store list on start-up\_DM\_6 }\label{lvv-7101}

\begin{longtable}{cccc}
\hline
\textbf{Jira Link} & \textbf{Assignee} & \textbf{Status} & \textbf{Test Cases}\\ \hline
\href{https://jira.lsstcorp.org/browse/LVV-7101}{LVV-7101} &
Leanne Guy & Not Covered &
\begin{tabular}{c}
\end{tabular}
\\
\hline
\end{longtable}

\textbf{Verification Element Description:} \\
Undefined

{\footnotesize
\begin{longtable}{p{2.5cm}p{13.5cm}}
\hline
\multicolumn{2}{c}{\textbf{Requirement Details}}\\ \hline
Requirement ID & CPT-OCS-INT-ICD-0054 \\ \cdashline{1-2}
Requirement Description &
\begin{minipage}[]{13cm}
\textbf{Specification:} The component shall publish a list of
recommended versions of setting values upon connecting to SAL/DDS.
\end{minipage}
\\ \cdashline{1-2}
Requirement Priority &  \\ \cdashline{1-2}
Upper Level Requirement &
\begin{tabular}{cl}
CPT-OCS-INT-ICD-0055 & Publish version list in store \\
\end{tabular}
\\ \hline
\end{longtable}
}


  
 \newpage 
\subsection{[LVV-7102] CPT-OCS-INT-ICD-0054-V-07: Publish store list on start-up\_DM\_7 }\label{lvv-7102}

\begin{longtable}{cccc}
\hline
\textbf{Jira Link} & \textbf{Assignee} & \textbf{Status} & \textbf{Test Cases}\\ \hline
\href{https://jira.lsstcorp.org/browse/LVV-7102}{LVV-7102} &
Leanne Guy & Not Covered &
\begin{tabular}{c}
\end{tabular}
\\
\hline
\end{longtable}

\textbf{Verification Element Description:} \\
Undefined

{\footnotesize
\begin{longtable}{p{2.5cm}p{13.5cm}}
\hline
\multicolumn{2}{c}{\textbf{Requirement Details}}\\ \hline
Requirement ID & CPT-OCS-INT-ICD-0054 \\ \cdashline{1-2}
Requirement Description &
\begin{minipage}[]{13cm}
\textbf{Specification:} The component shall publish a list of
recommended versions of setting values upon connecting to SAL/DDS.
\end{minipage}
\\ \cdashline{1-2}
Requirement Priority &  \\ \cdashline{1-2}
Upper Level Requirement &
\begin{tabular}{cl}
CPT-OCS-INT-ICD-0055 & Publish version list in store \\
\end{tabular}
\\ \hline
\end{longtable}
}


  
 \newpage 
\subsection{[LVV-7107] CPT-OCS-INT-ICD-0055-V-06: Publish version list in store\_DM\_6 }\label{lvv-7107}

\begin{longtable}{cccc}
\hline
\textbf{Jira Link} & \textbf{Assignee} & \textbf{Status} & \textbf{Test Cases}\\ \hline
\href{https://jira.lsstcorp.org/browse/LVV-7107}{LVV-7107} &
Leanne Guy & Not Covered &
\begin{tabular}{c}
\end{tabular}
\\
\hline
\end{longtable}

\textbf{Verification Element Description:} \\
Undefined

{\footnotesize
\begin{longtable}{p{2.5cm}p{13.5cm}}
\hline
\multicolumn{2}{c}{\textbf{Requirement Details}}\\ \hline
Requirement ID & CPT-OCS-INT-ICD-0055 \\ \cdashline{1-2}
Requirement Description &
\begin{minipage}[]{13cm}
\textbf{Specification:} The component shall publish a list of
recommended versions of setting sets.
\end{minipage}
\\ \cdashline{1-2}
Requirement Priority &  \\ \cdashline{1-2}
Upper Level Requirement &
\begin{tabular}{cl}
CPT-OCS-INT-ICD-0052 & Local store \\
\end{tabular}
\\ \hline
\end{longtable}
}


  
 \newpage 
\subsection{[LVV-7108] CPT-OCS-INT-ICD-0055-V-07: Publish version list in store\_DM\_7 }\label{lvv-7108}

\begin{longtable}{cccc}
\hline
\textbf{Jira Link} & \textbf{Assignee} & \textbf{Status} & \textbf{Test Cases}\\ \hline
\href{https://jira.lsstcorp.org/browse/LVV-7108}{LVV-7108} &
Leanne Guy & Not Covered &
\begin{tabular}{c}
\end{tabular}
\\
\hline
\end{longtable}

\textbf{Verification Element Description:} \\
Undefined

{\footnotesize
\begin{longtable}{p{2.5cm}p{13.5cm}}
\hline
\multicolumn{2}{c}{\textbf{Requirement Details}}\\ \hline
Requirement ID & CPT-OCS-INT-ICD-0055 \\ \cdashline{1-2}
Requirement Description &
\begin{minipage}[]{13cm}
\textbf{Specification:} The component shall publish a list of
recommended versions of setting sets.
\end{minipage}
\\ \cdashline{1-2}
Requirement Priority &  \\ \cdashline{1-2}
Upper Level Requirement &
\begin{tabular}{cl}
CPT-OCS-INT-ICD-0052 & Local store \\
\end{tabular}
\\ \hline
\end{longtable}
}


  
 \newpage 
\subsection{[LVV-7113] CPT-OCS-INT-ICD-0051-V-06: Store settings\_DM\_6 }\label{lvv-7113}

\begin{longtable}{cccc}
\hline
\textbf{Jira Link} & \textbf{Assignee} & \textbf{Status} & \textbf{Test Cases}\\ \hline
\href{https://jira.lsstcorp.org/browse/LVV-7113}{LVV-7113} &
Leanne Guy & Not Covered &
\begin{tabular}{c}
\end{tabular}
\\
\hline
\end{longtable}

\textbf{Verification Element Description:} \\
Undefined

{\footnotesize
\begin{longtable}{p{2.5cm}p{13.5cm}}
\hline
\multicolumn{2}{c}{\textbf{Requirement Details}}\\ \hline
Requirement ID & CPT-OCS-INT-ICD-0051 \\ \cdashline{1-2}
Requirement Description &
\begin{minipage}[]{13cm}
\textbf{Specification:} The component shall store setting values.
\end{minipage}
\\ \cdashline{1-2}
Requirement Discussion &
\begin{minipage}[]{13cm}
\textbf{Discussion:}
\end{minipage}
\\ \cdashline{1-2}
Requirement Priority &  \\ \cdashline{1-2}
Upper Level Requirement &
\begin{tabular}{cl}
CPT-OCS-INT-ICD-0050 & Modify setting values \\
\end{tabular}
\\ \hline
\end{longtable}
}


  
 \newpage 
\subsection{[LVV-7114] CPT-OCS-INT-ICD-0051-V-07: Store settings\_DM\_7 }\label{lvv-7114}

\begin{longtable}{cccc}
\hline
\textbf{Jira Link} & \textbf{Assignee} & \textbf{Status} & \textbf{Test Cases}\\ \hline
\href{https://jira.lsstcorp.org/browse/LVV-7114}{LVV-7114} &
Leanne Guy & Not Covered &
\begin{tabular}{c}
\end{tabular}
\\
\hline
\end{longtable}

\textbf{Verification Element Description:} \\
Undefined

{\footnotesize
\begin{longtable}{p{2.5cm}p{13.5cm}}
\hline
\multicolumn{2}{c}{\textbf{Requirement Details}}\\ \hline
Requirement ID & CPT-OCS-INT-ICD-0051 \\ \cdashline{1-2}
Requirement Description &
\begin{minipage}[]{13cm}
\textbf{Specification:} The component shall store setting values.
\end{minipage}
\\ \cdashline{1-2}
Requirement Discussion &
\begin{minipage}[]{13cm}
\textbf{Discussion:}
\end{minipage}
\\ \cdashline{1-2}
Requirement Priority &  \\ \cdashline{1-2}
Upper Level Requirement &
\begin{tabular}{cl}
CPT-OCS-INT-ICD-0050 & Modify setting values \\
\end{tabular}
\\ \hline
\end{longtable}
}


  
 \newpage 
\subsection{[LVV-7119] CPT-OCS-INT-ICD-0073-V-06: Support editing labels\_DM\_6 }\label{lvv-7119}

\begin{longtable}{cccc}
\hline
\textbf{Jira Link} & \textbf{Assignee} & \textbf{Status} & \textbf{Test Cases}\\ \hline
\href{https://jira.lsstcorp.org/browse/LVV-7119}{LVV-7119} &
Leanne Guy & Not Covered &
\begin{tabular}{c}
\end{tabular}
\\
\hline
\end{longtable}

\textbf{Verification Element Description:} \\
Undefined

{\footnotesize
\begin{longtable}{p{2.5cm}p{13.5cm}}
\hline
\multicolumn{2}{c}{\textbf{Requirement Details}}\\ \hline
Requirement ID & CPT-OCS-INT-ICD-0073 \\ \cdashline{1-2}
Requirement Description &
\begin{minipage}[]{13cm}
\textbf{Specification:} The component shall support editing of settings
labels (rather than fixing these in a release).
\end{minipage}
\\ \cdashline{1-2}
Requirement Discussion &
\begin{minipage}[]{13cm}
\textbf{Discussion:}
\end{minipage}
\\ \cdashline{1-2}
Requirement Priority &  \\ \cdashline{1-2}
Upper Level Requirement &
\begin{tabular}{cl}
CPT-OCS-INT-ICD-0067 & Recall using label \\
\end{tabular}
\\ \hline
\end{longtable}
}


  
 \newpage 
\subsection{[LVV-7120] CPT-OCS-INT-ICD-0073-V-07: Support editing labels\_DM\_7 }\label{lvv-7120}

\begin{longtable}{cccc}
\hline
\textbf{Jira Link} & \textbf{Assignee} & \textbf{Status} & \textbf{Test Cases}\\ \hline
\href{https://jira.lsstcorp.org/browse/LVV-7120}{LVV-7120} &
Leanne Guy & Not Covered &
\begin{tabular}{c}
\end{tabular}
\\
\hline
\end{longtable}

\textbf{Verification Element Description:} \\
Undefined

{\footnotesize
\begin{longtable}{p{2.5cm}p{13.5cm}}
\hline
\multicolumn{2}{c}{\textbf{Requirement Details}}\\ \hline
Requirement ID & CPT-OCS-INT-ICD-0073 \\ \cdashline{1-2}
Requirement Description &
\begin{minipage}[]{13cm}
\textbf{Specification:} The component shall support editing of settings
labels (rather than fixing these in a release).
\end{minipage}
\\ \cdashline{1-2}
Requirement Discussion &
\begin{minipage}[]{13cm}
\textbf{Discussion:}
\end{minipage}
\\ \cdashline{1-2}
Requirement Priority &  \\ \cdashline{1-2}
Upper Level Requirement &
\begin{tabular}{cl}
CPT-OCS-INT-ICD-0067 & Recall using label \\
\end{tabular}
\\ \hline
\end{longtable}
}


  
 \newpage 
\subsection{[LVV-7125] CPT-OCS-INT-ICD-0058-V-06: Support editing settings in quiescent
states\_DM\_6 }\label{lvv-7125}

\begin{longtable}{cccc}
\hline
\textbf{Jira Link} & \textbf{Assignee} & \textbf{Status} & \textbf{Test Cases}\\ \hline
\href{https://jira.lsstcorp.org/browse/LVV-7125}{LVV-7125} &
Leanne Guy & Not Covered &
\begin{tabular}{c}
\end{tabular}
\\
\hline
\end{longtable}

\textbf{Verification Element Description:} \\
Undefined

{\footnotesize
\begin{longtable}{p{2.5cm}p{13.5cm}}
\hline
\multicolumn{2}{c}{\textbf{Requirement Details}}\\ \hline
Requirement ID & CPT-OCS-INT-ICD-0058 \\ \cdashline{1-2}
Requirement Description &
\begin{minipage}[]{13cm}
\textbf{Specification:} The component shall support editing all editable
settings while in Standby or Offline States.
\end{minipage}
\\ \cdashline{1-2}
Requirement Discussion &
\begin{minipage}[]{13cm}
\textbf{Discussion:}
\end{minipage}
\\ \cdashline{1-2}
Requirement Priority &  \\ \cdashline{1-2}
Upper Level Requirement &
\begin{tabular}{cl}
CPT-OCS-INT-ICD-0061 & Edit editable settings \\
\end{tabular}
\\ \hline
\end{longtable}
}


  
 \newpage 
\subsection{[LVV-7126] CPT-OCS-INT-ICD-0058-V-07: Support editing settings in quiescent
states\_DM\_7 }\label{lvv-7126}

\begin{longtable}{cccc}
\hline
\textbf{Jira Link} & \textbf{Assignee} & \textbf{Status} & \textbf{Test Cases}\\ \hline
\href{https://jira.lsstcorp.org/browse/LVV-7126}{LVV-7126} &
Leanne Guy & Not Covered &
\begin{tabular}{c}
\end{tabular}
\\
\hline
\end{longtable}

\textbf{Verification Element Description:} \\
Undefined

{\footnotesize
\begin{longtable}{p{2.5cm}p{13.5cm}}
\hline
\multicolumn{2}{c}{\textbf{Requirement Details}}\\ \hline
Requirement ID & CPT-OCS-INT-ICD-0058 \\ \cdashline{1-2}
Requirement Description &
\begin{minipage}[]{13cm}
\textbf{Specification:} The component shall support editing all editable
settings while in Standby or Offline States.
\end{minipage}
\\ \cdashline{1-2}
Requirement Discussion &
\begin{minipage}[]{13cm}
\textbf{Discussion:}
\end{minipage}
\\ \cdashline{1-2}
Requirement Priority &  \\ \cdashline{1-2}
Upper Level Requirement &
\begin{tabular}{cl}
CPT-OCS-INT-ICD-0061 & Edit editable settings \\
\end{tabular}
\\ \hline
\end{longtable}
}


  
 \newpage 
\subsection{[LVV-7131] CPT-OCS-INT-ICD-0059-V-06: Support partial editing by external
client\_DM\_6 }\label{lvv-7131}

\begin{longtable}{cccc}
\hline
\textbf{Jira Link} & \textbf{Assignee} & \textbf{Status} & \textbf{Test Cases}\\ \hline
\href{https://jira.lsstcorp.org/browse/LVV-7131}{LVV-7131} &
Leanne Guy & Not Covered &
\begin{tabular}{c}
\end{tabular}
\\
\hline
\end{longtable}

\textbf{Verification Element Description:} \\
Undefined

{\footnotesize
\begin{longtable}{p{2.5cm}p{13.5cm}}
\hline
\multicolumn{2}{c}{\textbf{Requirement Details}}\\ \hline
Requirement ID & CPT-OCS-INT-ICD-0059 \\ \cdashline{1-2}
Requirement Description &
\begin{minipage}[]{13cm}
\textbf{Specification:} The component shall support editing a subset of
editable settings while in EnabledState or DisabledState via commands
from an external client.
\end{minipage}
\\ \cdashline{1-2}
Requirement Discussion &
\begin{minipage}[]{13cm}
\textbf{Discussion:}
\end{minipage}
\\ \cdashline{1-2}
Requirement Priority &  \\ \cdashline{1-2}
Upper Level Requirement &
\begin{tabular}{cl}
CPT-OCS-INT-ICD-0060 & Support partial editing while enabled \\
\end{tabular}
\\ \hline
\end{longtable}
}


  
 \newpage 
\subsection{[LVV-7132] CPT-OCS-INT-ICD-0059-V-07: Support partial editing by external
client\_DM\_7 }\label{lvv-7132}

\begin{longtable}{cccc}
\hline
\textbf{Jira Link} & \textbf{Assignee} & \textbf{Status} & \textbf{Test Cases}\\ \hline
\href{https://jira.lsstcorp.org/browse/LVV-7132}{LVV-7132} &
Leanne Guy & Not Covered &
\begin{tabular}{c}
\end{tabular}
\\
\hline
\end{longtable}

\textbf{Verification Element Description:} \\
Undefined

{\footnotesize
\begin{longtable}{p{2.5cm}p{13.5cm}}
\hline
\multicolumn{2}{c}{\textbf{Requirement Details}}\\ \hline
Requirement ID & CPT-OCS-INT-ICD-0059 \\ \cdashline{1-2}
Requirement Description &
\begin{minipage}[]{13cm}
\textbf{Specification:} The component shall support editing a subset of
editable settings while in EnabledState or DisabledState via commands
from an external client.
\end{minipage}
\\ \cdashline{1-2}
Requirement Discussion &
\begin{minipage}[]{13cm}
\textbf{Discussion:}
\end{minipage}
\\ \cdashline{1-2}
Requirement Priority &  \\ \cdashline{1-2}
Upper Level Requirement &
\begin{tabular}{cl}
CPT-OCS-INT-ICD-0060 & Support partial editing while enabled \\
\end{tabular}
\\ \hline
\end{longtable}
}


  
 \newpage 
\subsection{[LVV-7137] CPT-OCS-INT-ICD-0060-V-06: Support partial editing while enabled\_DM\_6 }\label{lvv-7137}

\begin{longtable}{cccc}
\hline
\textbf{Jira Link} & \textbf{Assignee} & \textbf{Status} & \textbf{Test Cases}\\ \hline
\href{https://jira.lsstcorp.org/browse/LVV-7137}{LVV-7137} &
Leanne Guy & Not Covered &
\begin{tabular}{c}
\end{tabular}
\\
\hline
\end{longtable}

\textbf{Verification Element Description:} \\
Undefined

{\footnotesize
\begin{longtable}{p{2.5cm}p{13.5cm}}
\hline
\multicolumn{2}{c}{\textbf{Requirement Details}}\\ \hline
Requirement ID & CPT-OCS-INT-ICD-0060 \\ \cdashline{1-2}
Requirement Description &
\begin{minipage}[]{13cm}
\textbf{Specification:} The component shall support editing a subset of
editable settings while in Enabled State.
\end{minipage}
\\ \cdashline{1-2}
Requirement Discussion &
\begin{minipage}[]{13cm}
\textbf{Discussion:}
\end{minipage}
\\ \cdashline{1-2}
Requirement Priority &  \\ \cdashline{1-2}
Upper Level Requirement &
\begin{tabular}{cl}
CPT-OCS-INT-ICD-0061 & Edit editable settings \\
\end{tabular}
\\ \hline
\end{longtable}
}


  
 \newpage 
\subsection{[LVV-7138] CPT-OCS-INT-ICD-0060-V-07: Support partial editing while enabled\_DM\_7 }\label{lvv-7138}

\begin{longtable}{cccc}
\hline
\textbf{Jira Link} & \textbf{Assignee} & \textbf{Status} & \textbf{Test Cases}\\ \hline
\href{https://jira.lsstcorp.org/browse/LVV-7138}{LVV-7138} &
Leanne Guy & Not Covered &
\begin{tabular}{c}
\end{tabular}
\\
\hline
\end{longtable}

\textbf{Verification Element Description:} \\
Undefined

{\footnotesize
\begin{longtable}{p{2.5cm}p{13.5cm}}
\hline
\multicolumn{2}{c}{\textbf{Requirement Details}}\\ \hline
Requirement ID & CPT-OCS-INT-ICD-0060 \\ \cdashline{1-2}
Requirement Description &
\begin{minipage}[]{13cm}
\textbf{Specification:} The component shall support editing a subset of
editable settings while in Enabled State.
\end{minipage}
\\ \cdashline{1-2}
Requirement Discussion &
\begin{minipage}[]{13cm}
\textbf{Discussion:}
\end{minipage}
\\ \cdashline{1-2}
Requirement Priority &  \\ \cdashline{1-2}
Upper Level Requirement &
\begin{tabular}{cl}
CPT-OCS-INT-ICD-0061 & Edit editable settings \\
\end{tabular}
\\ \hline
\end{longtable}
}


  
 \newpage 
\subsection{[LVV-7143] CPT-OCS-INT-ICD-0056-V-06: Validate editable settings\_DM\_6 }\label{lvv-7143}

\begin{longtable}{cccc}
\hline
\textbf{Jira Link} & \textbf{Assignee} & \textbf{Status} & \textbf{Test Cases}\\ \hline
\href{https://jira.lsstcorp.org/browse/LVV-7143}{LVV-7143} &
Leanne Guy & Not Covered &
\begin{tabular}{c}
\end{tabular}
\\
\hline
\end{longtable}

\textbf{Verification Element Description:} \\
Undefined

{\footnotesize
\begin{longtable}{p{2.5cm}p{13.5cm}}
\hline
\multicolumn{2}{c}{\textbf{Requirement Details}}\\ \hline
Requirement ID & CPT-OCS-INT-ICD-0056 \\ \cdashline{1-2}
Requirement Description &
\begin{minipage}[]{13cm}
\textbf{Specification:} The component shall validate editable setting
values.
\end{minipage}
\\ \cdashline{1-2}
Requirement Discussion &
\begin{minipage}[]{13cm}
\textbf{Discussion:}
\end{minipage}
\\ \cdashline{1-2}
Requirement Priority &  \\ \cdashline{1-2}
Upper Level Requirement &
\begin{tabular}{cl}
CPT-OCS-INT-ICD-0050 & Modify setting values \\
\end{tabular}
\\ \hline
\end{longtable}
}


  
 \newpage 
\subsection{[LVV-7144] CPT-OCS-INT-ICD-0056-V-07: Validate editable settings\_DM\_7 }\label{lvv-7144}

\begin{longtable}{cccc}
\hline
\textbf{Jira Link} & \textbf{Assignee} & \textbf{Status} & \textbf{Test Cases}\\ \hline
\href{https://jira.lsstcorp.org/browse/LVV-7144}{LVV-7144} &
Leanne Guy & Not Covered &
\begin{tabular}{c}
\end{tabular}
\\
\hline
\end{longtable}

\textbf{Verification Element Description:} \\
Undefined

{\footnotesize
\begin{longtable}{p{2.5cm}p{13.5cm}}
\hline
\multicolumn{2}{c}{\textbf{Requirement Details}}\\ \hline
Requirement ID & CPT-OCS-INT-ICD-0056 \\ \cdashline{1-2}
Requirement Description &
\begin{minipage}[]{13cm}
\textbf{Specification:} The component shall validate editable setting
values.
\end{minipage}
\\ \cdashline{1-2}
Requirement Discussion &
\begin{minipage}[]{13cm}
\textbf{Discussion:}
\end{minipage}
\\ \cdashline{1-2}
Requirement Priority &  \\ \cdashline{1-2}
Upper Level Requirement &
\begin{tabular}{cl}
CPT-OCS-INT-ICD-0050 & Modify setting values \\
\end{tabular}
\\ \hline
\end{longtable}
}


  
 \newpage 
\subsection{[LVV-7149] CPT-OCS-INT-ICD-0063-V-06: Publish labels and version mapping\_DM\_6 }\label{lvv-7149}

\begin{longtable}{cccc}
\hline
\textbf{Jira Link} & \textbf{Assignee} & \textbf{Status} & \textbf{Test Cases}\\ \hline
\href{https://jira.lsstcorp.org/browse/LVV-7149}{LVV-7149} &
Leanne Guy & Not Covered &
\begin{tabular}{c}
\end{tabular}
\\
\hline
\end{longtable}

\textbf{Verification Element Description:} \\
Undefined

{\footnotesize
\begin{longtable}{p{2.5cm}p{13.5cm}}
\hline
\multicolumn{2}{c}{\textbf{Requirement Details}}\\ \hline
Requirement ID & CPT-OCS-INT-ICD-0063 \\ \cdashline{1-2}
Requirement Description &
\begin{minipage}[]{13cm}
\textbf{Specification:} The component shall publish a mapping of the
presently active labels and the setting set versions associated with
each of these.
\end{minipage}
\\ \cdashline{1-2}
Requirement Discussion &
\begin{minipage}[]{13cm}
\textbf{Discussion:}
\end{minipage}
\\ \cdashline{1-2}
Requirement Priority &  \\ \cdashline{1-2}
Upper Level Requirement &
\begin{tabular}{cl}
CPT-OCS-INT-ICD-0067 & Recall using label \\
\end{tabular}
\\ \hline
\end{longtable}
}


  
 \newpage 
\subsection{[LVV-7150] CPT-OCS-INT-ICD-0063-V-07: Publish labels and version mapping\_DM\_7 }\label{lvv-7150}

\begin{longtable}{cccc}
\hline
\textbf{Jira Link} & \textbf{Assignee} & \textbf{Status} & \textbf{Test Cases}\\ \hline
\href{https://jira.lsstcorp.org/browse/LVV-7150}{LVV-7150} &
Leanne Guy & Not Covered &
\begin{tabular}{c}
\end{tabular}
\\
\hline
\end{longtable}

\textbf{Verification Element Description:} \\
Undefined

{\footnotesize
\begin{longtable}{p{2.5cm}p{13.5cm}}
\hline
\multicolumn{2}{c}{\textbf{Requirement Details}}\\ \hline
Requirement ID & CPT-OCS-INT-ICD-0063 \\ \cdashline{1-2}
Requirement Description &
\begin{minipage}[]{13cm}
\textbf{Specification:} The component shall publish a mapping of the
presently active labels and the setting set versions associated with
each of these.
\end{minipage}
\\ \cdashline{1-2}
Requirement Discussion &
\begin{minipage}[]{13cm}
\textbf{Discussion:}
\end{minipage}
\\ \cdashline{1-2}
Requirement Priority &  \\ \cdashline{1-2}
Upper Level Requirement &
\begin{tabular}{cl}
CPT-OCS-INT-ICD-0067 & Recall using label \\
\end{tabular}
\\ \hline
\end{longtable}
}


  
 \newpage 
\subsection{[LVV-7155] CPT-OCS-INT-ICD-0064-V-06: Publish mapping on change\_DM\_6 }\label{lvv-7155}

\begin{longtable}{cccc}
\hline
\textbf{Jira Link} & \textbf{Assignee} & \textbf{Status} & \textbf{Test Cases}\\ \hline
\href{https://jira.lsstcorp.org/browse/LVV-7155}{LVV-7155} &
Leanne Guy & Not Covered &
\begin{tabular}{c}
\end{tabular}
\\
\hline
\end{longtable}

\textbf{Verification Element Description:} \\
Undefined

{\footnotesize
\begin{longtable}{p{2.5cm}p{13.5cm}}
\hline
\multicolumn{2}{c}{\textbf{Requirement Details}}\\ \hline
Requirement ID & CPT-OCS-INT-ICD-0064 \\ \cdashline{1-2}
Requirement Description &
\begin{minipage}[]{13cm}
\textbf{Specification:} While the component is connected to SAL/DDS, the
component shall publish a mapping of the presently active labels and the
setting set versions associated with each of these when the mapping
changes.
\end{minipage}
\\ \cdashline{1-2}
Requirement Discussion &
\begin{minipage}[]{13cm}
\textbf{Discussion:}
\end{minipage}
\\ \cdashline{1-2}
Requirement Priority &  \\ \cdashline{1-2}
Upper Level Requirement &
\begin{tabular}{cl}
CPT-OCS-INT-ICD-0065 & Publish mapping on start-up \\
\end{tabular}
\\ \hline
\end{longtable}
}


  
 \newpage 
\subsection{[LVV-7156] CPT-OCS-INT-ICD-0064-V-07: Publish mapping on change\_DM\_7 }\label{lvv-7156}

\begin{longtable}{cccc}
\hline
\textbf{Jira Link} & \textbf{Assignee} & \textbf{Status} & \textbf{Test Cases}\\ \hline
\href{https://jira.lsstcorp.org/browse/LVV-7156}{LVV-7156} &
Leanne Guy & Not Covered &
\begin{tabular}{c}
\end{tabular}
\\
\hline
\end{longtable}

\textbf{Verification Element Description:} \\
Undefined

{\footnotesize
\begin{longtable}{p{2.5cm}p{13.5cm}}
\hline
\multicolumn{2}{c}{\textbf{Requirement Details}}\\ \hline
Requirement ID & CPT-OCS-INT-ICD-0064 \\ \cdashline{1-2}
Requirement Description &
\begin{minipage}[]{13cm}
\textbf{Specification:} While the component is connected to SAL/DDS, the
component shall publish a mapping of the presently active labels and the
setting set versions associated with each of these when the mapping
changes.
\end{minipage}
\\ \cdashline{1-2}
Requirement Discussion &
\begin{minipage}[]{13cm}
\textbf{Discussion:}
\end{minipage}
\\ \cdashline{1-2}
Requirement Priority &  \\ \cdashline{1-2}
Upper Level Requirement &
\begin{tabular}{cl}
CPT-OCS-INT-ICD-0065 & Publish mapping on start-up \\
\end{tabular}
\\ \hline
\end{longtable}
}


  
 \newpage 
\subsection{[LVV-7161] CPT-OCS-INT-ICD-0065-V-06: Publish mapping on start-up\_DM\_6 }\label{lvv-7161}

\begin{longtable}{cccc}
\hline
\textbf{Jira Link} & \textbf{Assignee} & \textbf{Status} & \textbf{Test Cases}\\ \hline
\href{https://jira.lsstcorp.org/browse/LVV-7161}{LVV-7161} &
Leanne Guy & Not Covered &
\begin{tabular}{c}
\end{tabular}
\\
\hline
\end{longtable}

\textbf{Verification Element Description:} \\
Undefined

{\footnotesize
\begin{longtable}{p{2.5cm}p{13.5cm}}
\hline
\multicolumn{2}{c}{\textbf{Requirement Details}}\\ \hline
Requirement ID & CPT-OCS-INT-ICD-0065 \\ \cdashline{1-2}
Requirement Description &
\begin{minipage}[]{13cm}
\textbf{Specification:} The component shall publish a mapping of the
presently active labels and the setting set versions associated with
each of these upon connecting to SAL/DDS.
\end{minipage}
\\ \cdashline{1-2}
Requirement Discussion &
\begin{minipage}[]{13cm}
\textbf{Discussion:}
\end{minipage}
\\ \cdashline{1-2}
Requirement Priority &  \\ \cdashline{1-2}
Upper Level Requirement &
\begin{tabular}{cl}
CPT-OCS-INT-ICD-0063 & Publish labels and version mapping \\
\end{tabular}
\\ \hline
\end{longtable}
}


  
 \newpage 
\subsection{[LVV-7162] CPT-OCS-INT-ICD-0065-V-07: Publish mapping on start-up\_DM\_7 }\label{lvv-7162}

\begin{longtable}{cccc}
\hline
\textbf{Jira Link} & \textbf{Assignee} & \textbf{Status} & \textbf{Test Cases}\\ \hline
\href{https://jira.lsstcorp.org/browse/LVV-7162}{LVV-7162} &
Leanne Guy & Not Covered &
\begin{tabular}{c}
\end{tabular}
\\
\hline
\end{longtable}

\textbf{Verification Element Description:} \\
Undefined

{\footnotesize
\begin{longtable}{p{2.5cm}p{13.5cm}}
\hline
\multicolumn{2}{c}{\textbf{Requirement Details}}\\ \hline
Requirement ID & CPT-OCS-INT-ICD-0065 \\ \cdashline{1-2}
Requirement Description &
\begin{minipage}[]{13cm}
\textbf{Specification:} The component shall publish a mapping of the
presently active labels and the setting set versions associated with
each of these upon connecting to SAL/DDS.
\end{minipage}
\\ \cdashline{1-2}
Requirement Discussion &
\begin{minipage}[]{13cm}
\textbf{Discussion:}
\end{minipage}
\\ \cdashline{1-2}
Requirement Priority &  \\ \cdashline{1-2}
Upper Level Requirement &
\begin{tabular}{cl}
CPT-OCS-INT-ICD-0063 & Publish labels and version mapping \\
\end{tabular}
\\ \hline
\end{longtable}
}


  
 \newpage 
\subsection{[LVV-7167] CPT-OCS-INT-ICD-0066-V-06: Recall preset editable settings\_DM\_6 }\label{lvv-7167}

\begin{longtable}{cccc}
\hline
\textbf{Jira Link} & \textbf{Assignee} & \textbf{Status} & \textbf{Test Cases}\\ \hline
\href{https://jira.lsstcorp.org/browse/LVV-7167}{LVV-7167} &
Leanne Guy & Not Covered &
\begin{tabular}{c}
\end{tabular}
\\
\hline
\end{longtable}

\textbf{Verification Element Description:} \\
Undefined

{\footnotesize
\begin{longtable}{p{2.5cm}p{13.5cm}}
\hline
\multicolumn{2}{c}{\textbf{Requirement Details}}\\ \hline
Requirement ID & CPT-OCS-INT-ICD-0066 \\ \cdashline{1-2}
Requirement Description &
\begin{minipage}[]{13cm}
\textbf{Specification:} The component shall recall preset editable
settings values on demand.
\end{minipage}
\\ \cdashline{1-2}
Requirement Discussion &
\begin{minipage}[]{13cm}
\textbf{Discussion:}
\end{minipage}
\\ \cdashline{1-2}
Requirement Priority &  \\ \cdashline{1-2}
Upper Level Requirement &
\begin{tabular}{cl}
CPT-OCS-INT-ICD-0069 & Recall values \\
\end{tabular}
\\ \hline
\end{longtable}
}


  
 \newpage 
\subsection{[LVV-7168] CPT-OCS-INT-ICD-0066-V-07: Recall preset editable settings\_DM\_7 }\label{lvv-7168}

\begin{longtable}{cccc}
\hline
\textbf{Jira Link} & \textbf{Assignee} & \textbf{Status} & \textbf{Test Cases}\\ \hline
\href{https://jira.lsstcorp.org/browse/LVV-7168}{LVV-7168} &
Leanne Guy & Not Covered &
\begin{tabular}{c}
\end{tabular}
\\
\hline
\end{longtable}

\textbf{Verification Element Description:} \\
Undefined

{\footnotesize
\begin{longtable}{p{2.5cm}p{13.5cm}}
\hline
\multicolumn{2}{c}{\textbf{Requirement Details}}\\ \hline
Requirement ID & CPT-OCS-INT-ICD-0066 \\ \cdashline{1-2}
Requirement Description &
\begin{minipage}[]{13cm}
\textbf{Specification:} The component shall recall preset editable
settings values on demand.
\end{minipage}
\\ \cdashline{1-2}
Requirement Discussion &
\begin{minipage}[]{13cm}
\textbf{Discussion:}
\end{minipage}
\\ \cdashline{1-2}
Requirement Priority &  \\ \cdashline{1-2}
Upper Level Requirement &
\begin{tabular}{cl}
CPT-OCS-INT-ICD-0069 & Recall values \\
\end{tabular}
\\ \hline
\end{longtable}
}


  
 \newpage 
\subsection{[LVV-7173] CPT-OCS-INT-ICD-0062-V-06: Recall settings\_DM\_6 }\label{lvv-7173}

\begin{longtable}{cccc}
\hline
\textbf{Jira Link} & \textbf{Assignee} & \textbf{Status} & \textbf{Test Cases}\\ \hline
\href{https://jira.lsstcorp.org/browse/LVV-7173}{LVV-7173} &
Leanne Guy & Not Covered &
\begin{tabular}{c}
\end{tabular}
\\
\hline
\end{longtable}

\textbf{Verification Element Description:} \\
Undefined

{\footnotesize
\begin{longtable}{p{2.5cm}p{13.5cm}}
\hline
\multicolumn{2}{c}{\textbf{Requirement Details}}\\ \hline
Requirement ID & CPT-OCS-INT-ICD-0062 \\ \cdashline{1-2}
Requirement Description &
\begin{minipage}[]{13cm}
\textbf{Specification:} The component shall recall stored setting values
on demand.
\end{minipage}
\\ \cdashline{1-2}
Requirement Discussion &
\begin{minipage}[]{13cm}
\textbf{Discussion:}
\end{minipage}
\\ \cdashline{1-2}
Requirement Priority &  \\ \cdashline{1-2}
Upper Level Requirement &
\begin{tabular}{cl}
CPT-OCS-INT-ICD-0061 & Edit editable settings \\
CPT-OCS-INT-ICD-0049 & Apply settings \\
\end{tabular}
\\ \hline
\end{longtable}
}


  
 \newpage 
\subsection{[LVV-7174] CPT-OCS-INT-ICD-0062-V-07: Recall settings\_DM\_7 }\label{lvv-7174}

\begin{longtable}{cccc}
\hline
\textbf{Jira Link} & \textbf{Assignee} & \textbf{Status} & \textbf{Test Cases}\\ \hline
\href{https://jira.lsstcorp.org/browse/LVV-7174}{LVV-7174} &
Leanne Guy & Not Covered &
\begin{tabular}{c}
\end{tabular}
\\
\hline
\end{longtable}

\textbf{Verification Element Description:} \\
Undefined

{\footnotesize
\begin{longtable}{p{2.5cm}p{13.5cm}}
\hline
\multicolumn{2}{c}{\textbf{Requirement Details}}\\ \hline
Requirement ID & CPT-OCS-INT-ICD-0062 \\ \cdashline{1-2}
Requirement Description &
\begin{minipage}[]{13cm}
\textbf{Specification:} The component shall recall stored setting values
on demand.
\end{minipage}
\\ \cdashline{1-2}
Requirement Discussion &
\begin{minipage}[]{13cm}
\textbf{Discussion:}
\end{minipage}
\\ \cdashline{1-2}
Requirement Priority &  \\ \cdashline{1-2}
Upper Level Requirement &
\begin{tabular}{cl}
CPT-OCS-INT-ICD-0061 & Edit editable settings \\
CPT-OCS-INT-ICD-0049 & Apply settings \\
\end{tabular}
\\ \hline
\end{longtable}
}


  
 \newpage 
\subsection{[LVV-7179] CPT-OCS-INT-ICD-0067-V-06: Recall using label\_DM\_6 }\label{lvv-7179}

\begin{longtable}{cccc}
\hline
\textbf{Jira Link} & \textbf{Assignee} & \textbf{Status} & \textbf{Test Cases}\\ \hline
\href{https://jira.lsstcorp.org/browse/LVV-7179}{LVV-7179} &
Leanne Guy & Not Covered &
\begin{tabular}{c}
\end{tabular}
\\
\hline
\end{longtable}

\textbf{Verification Element Description:} \\
Undefined

{\footnotesize
\begin{longtable}{p{2.5cm}p{13.5cm}}
\hline
\multicolumn{2}{c}{\textbf{Requirement Details}}\\ \hline
Requirement ID & CPT-OCS-INT-ICD-0067 \\ \cdashline{1-2}
Requirement Description &
\begin{minipage}[]{13cm}
\textbf{Specification:} The component shall support recalling stored
setting values associated with a label. Note that the specific setting
values associated with a particular label may vary over time.
\end{minipage}
\\ \cdashline{1-2}
Requirement Discussion &
\begin{minipage}[]{13cm}
\textbf{Discussion:}
\end{minipage}
\\ \cdashline{1-2}
Requirement Priority &  \\ \cdashline{1-2}
Upper Level Requirement &
\begin{tabular}{cl}
CPT-OCS-INT-ICD-0069 & Recall values \\
\end{tabular}
\\ \hline
\end{longtable}
}


  
 \newpage 
\subsection{[LVV-7180] CPT-OCS-INT-ICD-0067-V-07: Recall using label\_DM\_7 }\label{lvv-7180}

\begin{longtable}{cccc}
\hline
\textbf{Jira Link} & \textbf{Assignee} & \textbf{Status} & \textbf{Test Cases}\\ \hline
\href{https://jira.lsstcorp.org/browse/LVV-7180}{LVV-7180} &
Leanne Guy & Not Covered &
\begin{tabular}{c}
\end{tabular}
\\
\hline
\end{longtable}

\textbf{Verification Element Description:} \\
Undefined

{\footnotesize
\begin{longtable}{p{2.5cm}p{13.5cm}}
\hline
\multicolumn{2}{c}{\textbf{Requirement Details}}\\ \hline
Requirement ID & CPT-OCS-INT-ICD-0067 \\ \cdashline{1-2}
Requirement Description &
\begin{minipage}[]{13cm}
\textbf{Specification:} The component shall support recalling stored
setting values associated with a label. Note that the specific setting
values associated with a particular label may vary over time.
\end{minipage}
\\ \cdashline{1-2}
Requirement Discussion &
\begin{minipage}[]{13cm}
\textbf{Discussion:}
\end{minipage}
\\ \cdashline{1-2}
Requirement Priority &  \\ \cdashline{1-2}
Upper Level Requirement &
\begin{tabular}{cl}
CPT-OCS-INT-ICD-0069 & Recall values \\
\end{tabular}
\\ \hline
\end{longtable}
}


  
 \newpage 
\subsection{[LVV-7185] CPT-OCS-INT-ICD-0068-V-06: Recall using version identifier\_DM\_6 }\label{lvv-7185}

\begin{longtable}{cccc}
\hline
\textbf{Jira Link} & \textbf{Assignee} & \textbf{Status} & \textbf{Test Cases}\\ \hline
\href{https://jira.lsstcorp.org/browse/LVV-7185}{LVV-7185} &
Leanne Guy & Not Covered &
\begin{tabular}{c}
\end{tabular}
\\
\hline
\end{longtable}

\textbf{Verification Element Description:} \\
Undefined

{\footnotesize
\begin{longtable}{p{2.5cm}p{13.5cm}}
\hline
\multicolumn{2}{c}{\textbf{Requirement Details}}\\ \hline
Requirement ID & CPT-OCS-INT-ICD-0068 \\ \cdashline{1-2}
Requirement Description &
\begin{minipage}[]{13cm}
\textbf{Specification:} The component shall recall a specific version of
settings values when provided a unique identifier for the version.
\end{minipage}
\\ \cdashline{1-2}
Requirement Discussion &
\begin{minipage}[]{13cm}
\textbf{Discussion:}
\end{minipage}
\\ \cdashline{1-2}
Requirement Priority &  \\ \cdashline{1-2}
Upper Level Requirement &
\begin{tabular}{cl}
CPT-OCS-INT-ICD-0069 & Recall values \\
\end{tabular}
\\ \hline
\end{longtable}
}


  
 \newpage 
\subsection{[LVV-7186] CPT-OCS-INT-ICD-0068-V-07: Recall using version identifier\_DM\_7 }\label{lvv-7186}

\begin{longtable}{cccc}
\hline
\textbf{Jira Link} & \textbf{Assignee} & \textbf{Status} & \textbf{Test Cases}\\ \hline
\href{https://jira.lsstcorp.org/browse/LVV-7186}{LVV-7186} &
Leanne Guy & Not Covered &
\begin{tabular}{c}
\end{tabular}
\\
\hline
\end{longtable}

\textbf{Verification Element Description:} \\
Undefined

{\footnotesize
\begin{longtable}{p{2.5cm}p{13.5cm}}
\hline
\multicolumn{2}{c}{\textbf{Requirement Details}}\\ \hline
Requirement ID & CPT-OCS-INT-ICD-0068 \\ \cdashline{1-2}
Requirement Description &
\begin{minipage}[]{13cm}
\textbf{Specification:} The component shall recall a specific version of
settings values when provided a unique identifier for the version.
\end{minipage}
\\ \cdashline{1-2}
Requirement Discussion &
\begin{minipage}[]{13cm}
\textbf{Discussion:}
\end{minipage}
\\ \cdashline{1-2}
Requirement Priority &  \\ \cdashline{1-2}
Upper Level Requirement &
\begin{tabular}{cl}
CPT-OCS-INT-ICD-0069 & Recall values \\
\end{tabular}
\\ \hline
\end{longtable}
}


  
 \newpage 
\subsection{[LVV-7191] CPT-OCS-INT-ICD-0069-V-06: Recall values\_DM\_6 }\label{lvv-7191}

\begin{longtable}{cccc}
\hline
\textbf{Jira Link} & \textbf{Assignee} & \textbf{Status} & \textbf{Test Cases}\\ \hline
\href{https://jira.lsstcorp.org/browse/LVV-7191}{LVV-7191} &
Leanne Guy & Not Covered &
\begin{tabular}{c}
\end{tabular}
\\
\hline
\end{longtable}

\textbf{Verification Element Description:} \\
Undefined

{\footnotesize
\begin{longtable}{p{2.5cm}p{13.5cm}}
\hline
\multicolumn{2}{c}{\textbf{Requirement Details}}\\ \hline
Requirement ID & CPT-OCS-INT-ICD-0069 \\ \cdashline{1-2}
Requirement Description &
\begin{minipage}[]{13cm}
\textbf{Specification:} The component shall recall stored settings
values on demand.
\end{minipage}
\\ \cdashline{1-2}
Requirement Discussion &
\begin{minipage}[]{13cm}
\textbf{Discussion:}
\end{minipage}
\\ \cdashline{1-2}
Requirement Priority &  \\ \cdashline{1-2}
Upper Level Requirement &
\begin{tabular}{cl}
\end{tabular}
\\ \hline
\end{longtable}
}


  
 \newpage 
\subsection{[LVV-7192] CPT-OCS-INT-ICD-0069-V-07: Recall values\_DM\_7 }\label{lvv-7192}

\begin{longtable}{cccc}
\hline
\textbf{Jira Link} & \textbf{Assignee} & \textbf{Status} & \textbf{Test Cases}\\ \hline
\href{https://jira.lsstcorp.org/browse/LVV-7192}{LVV-7192} &
Leanne Guy & Not Covered &
\begin{tabular}{c}
\end{tabular}
\\
\hline
\end{longtable}

\textbf{Verification Element Description:} \\
Undefined

{\footnotesize
\begin{longtable}{p{2.5cm}p{13.5cm}}
\hline
\multicolumn{2}{c}{\textbf{Requirement Details}}\\ \hline
Requirement ID & CPT-OCS-INT-ICD-0069 \\ \cdashline{1-2}
Requirement Description &
\begin{minipage}[]{13cm}
\textbf{Specification:} The component shall recall stored settings
values on demand.
\end{minipage}
\\ \cdashline{1-2}
Requirement Discussion &
\begin{minipage}[]{13cm}
\textbf{Discussion:}
\end{minipage}
\\ \cdashline{1-2}
Requirement Priority &  \\ \cdashline{1-2}
Upper Level Requirement &
\begin{tabular}{cl}
\end{tabular}
\\ \hline
\end{longtable}
}


  
 \newpage 
\subsection{[LVV-7197] CPT-OCS-INT-ICD-0070-V-06: Support recall by external client\_DM\_6 }\label{lvv-7197}

\begin{longtable}{cccc}
\hline
\textbf{Jira Link} & \textbf{Assignee} & \textbf{Status} & \textbf{Test Cases}\\ \hline
\href{https://jira.lsstcorp.org/browse/LVV-7197}{LVV-7197} &
Leanne Guy & Not Covered &
\begin{tabular}{c}
\end{tabular}
\\
\hline
\end{longtable}

\textbf{Verification Element Description:} \\
Undefined

{\footnotesize
\begin{longtable}{p{2.5cm}p{13.5cm}}
\hline
\multicolumn{2}{c}{\textbf{Requirement Details}}\\ \hline
Requirement ID & CPT-OCS-INT-ICD-0070 \\ \cdashline{1-2}
Requirement Description &
\begin{minipage}[]{13cm}
\textbf{Specification:} The system shall support recall and apply
functionality via a command from an external client.
\end{minipage}
\\ \cdashline{1-2}
Requirement Discussion &
\begin{minipage}[]{13cm}
\textbf{Discussion:}
\end{minipage}
\\ \cdashline{1-2}
Requirement Priority &  \\ \cdashline{1-2}
Upper Level Requirement &
\begin{tabular}{cl}
CPT-OCS-INT-ICD-0062 & Recall settings \\
\end{tabular}
\\ \hline
\end{longtable}
}


  
 \newpage 
\subsection{[LVV-7198] CPT-OCS-INT-ICD-0070-V-07: Support recall by external client\_DM\_7 }\label{lvv-7198}

\begin{longtable}{cccc}
\hline
\textbf{Jira Link} & \textbf{Assignee} & \textbf{Status} & \textbf{Test Cases}\\ \hline
\href{https://jira.lsstcorp.org/browse/LVV-7198}{LVV-7198} &
Leanne Guy & Not Covered &
\begin{tabular}{c}
\end{tabular}
\\
\hline
\end{longtable}

\textbf{Verification Element Description:} \\
Undefined

{\footnotesize
\begin{longtable}{p{2.5cm}p{13.5cm}}
\hline
\multicolumn{2}{c}{\textbf{Requirement Details}}\\ \hline
Requirement ID & CPT-OCS-INT-ICD-0070 \\ \cdashline{1-2}
Requirement Description &
\begin{minipage}[]{13cm}
\textbf{Specification:} The system shall support recall and apply
functionality via a command from an external client.
\end{minipage}
\\ \cdashline{1-2}
Requirement Discussion &
\begin{minipage}[]{13cm}
\textbf{Discussion:}
\end{minipage}
\\ \cdashline{1-2}
Requirement Priority &  \\ \cdashline{1-2}
Upper Level Requirement &
\begin{tabular}{cl}
CPT-OCS-INT-ICD-0062 & Recall settings \\
\end{tabular}
\\ \hline
\end{longtable}
}


  
 \newpage 
\subsection{[LVV-9637] DMS-REQ-0372-V-01: Archiving Camera Test Data }\label{lvv-9637}

\begin{longtable}{cccc}
\hline
\textbf{Jira Link} & \textbf{Assignee} & \textbf{Status} & \textbf{Test Cases}\\ \hline
\href{https://jira.lsstcorp.org/browse/LVV-9637}{LVV-9637} &
Leanne Guy & Not Covered &
\begin{tabular}{c}
LVV-T1264 \\
\end{tabular}
\\
\hline
\end{longtable}

\textbf{Verification Element Description:} \\
Demonstrate that a subset of camera test data is available via Butler
repos.

{\footnotesize
\begin{longtable}{p{2.5cm}p{13.5cm}}
\hline
\multicolumn{2}{c}{\textbf{Requirement Details}}\\ \hline
Requirement ID & DMS-REQ-0372 \\ \cdashline{1-2}
Requirement Description &
\begin{minipage}[]{13cm}
\textbf{Specification:}~The DMS~shall be able to archive a designated
subset of Camera test data and make it available in an environment
matching the data backbone interfaces.
\end{minipage}
\\ \cdashline{1-2}
Requirement Discussion &
\begin{minipage}[]{13cm}
\textbf{Discussion:} This requirement describes a capability needed
primarily in construction.
\end{minipage}
\\ \cdashline{1-2}
Requirement Priority & 1a \\ \cdashline{1-2}
Upper Level Requirement &
\begin{tabular}{cl}
\end{tabular}
\\ \hline
\end{longtable}
}


\subsubsection{Test Cases Summary}
\begin{longtable}{p{3cm}p{2.5cm}p{2.5cm}p{3cm}p{4cm}}
\toprule
\href{https://jira.lsstcorp.org/secure/Tests.jspa\#/testCase/LVV-T1264}{LVV-T1264} & \multicolumn{4}{p{12cm}}{ Verify implementation of archiving camera test data } \\ \hline
\textbf{Owner} & \textbf{Status} & \textbf{Version} & \textbf{Critical Event} & \textbf{Verification Type} \\ \hline
Robert Gruendl & Defined & 1 & false & Test \\ \hline
\end{longtable}
{\scriptsize
\textbf{Objective:}\\
Verify that a subset of camera test data has been ingested into Butler
repos and is available through standard data access tools.
}
  
 \newpage 
\subsection{[LVV-9740] DMS-REQ-0004-V-02: Latency of reporting optical transients }\label{lvv-9740}

\begin{longtable}{cccc}
\hline
\textbf{Jira Link} & \textbf{Assignee} & \textbf{Status} & \textbf{Test Cases}\\ \hline
\href{https://jira.lsstcorp.org/browse/LVV-9740}{LVV-9740} &
Leanne Guy & Not Covered &
\begin{tabular}{c}
LVV-T1276 \\
\end{tabular}
\\
\hline
\end{longtable}

\textbf{Verification Element Description:} \\
Verify that optical transients are reported withinÂ~\textbf{OTT1 =
1{{[}minute{]}}} of readout of the last visit image. **

Associated element
(\href{https://jira.lsstcorp.org/browse/LVV-175}{LVV-175}) satisfies the
maximum time allotted for public release of L1 Data Products.

Associated element
(\href{https://jira.lsstcorp.org/browse/LVV-9803}{LVV-9803}) satisfies
the availability of Solar System Object orbits.

{\footnotesize
\begin{longtable}{p{2.5cm}p{13.5cm}}
\hline
\multicolumn{2}{c}{\textbf{Requirement Details}}\\ \hline
Requirement ID & DMS-REQ-0004 \\ \cdashline{1-2}
Requirement Description &
\begin{minipage}[]{13cm}
\textbf{Specification:} With the exception of alerts and Solar System
Objects, all Level 1 Data Products shall be made public within time
\textbf{L1PublicT} of the acquisition of the raw image data.\\
\hspace*{0.333em}\\
LSST shall not release image or catalog data resulting from a visit,
except for the content of the public alert stream, sooner than time
\textbf{L1PublicTMin} following the acquisition of the raw image data
from that visit.\\
\hspace*{0.333em}\\
For visits resulting in fewer than \textbf{nAlertVisitPeak}, LSST shall
be capable of supporting the distribution of at least \textbf{OTR1} per
cent of alerts via the LSST alert distribution system within time
\textbf{OTT1} from the conclusion of the camera's readout of the raw
exposures used to generate each alert. ~\\
\hspace*{0.333em}\\
Solar System Object orbits will, on average, be calculated before the
following night's observing finishes and the results shall be made
available within time \textbf{L1PublicT} of those calculations being
completed.
\end{minipage}
\\ \cdashline{1-2}
Requirement Parameters & {[}\textbf{OTR1 = 98{{[}percent{]}}} Fraction of detectable alerts for
which an alert is actually transmitted within latency OTT1 (see
LSR-REQ-0101)., \textbf{OTT1 = 1{{[}minute{]}}} The latency of reporting
optical transients following the completion of readout of the last image
of a visit, \textbf{nAlertVisitPeak = 40000{{[}integer{]}}} The
instantaneous peak number of alerts per standard visit.,
\textbf{L1PublicTMin = 6{{[}hour{]}}} Time images and other products
(except alerts) will be embargoed before release to the consortium (or
the public), \textbf{L1PublicT = 24{{[}hour{]}}} Maximum time from the
acquisition of science data to the release of associated Level 1 Data
Products (except alerts){]} \\ \cdashline{1-2}
Requirement Discussion &
\begin{minipage}[]{13cm}
\textbf{Discussion:} Because of the processing flow of SSObject orbit
determination, meeting the base
\textbf{L1PublicT}-after-data-acquisition requirement would be far more
challenging than for the other L1 Data Products, but the system
throughput has to be good enough such that a back log can not build up.
\end{minipage}
\\ \cdashline{1-2}
Requirement Priority & 1b \\ \cdashline{1-2}
Upper Level Requirement &
\begin{tabular}{cl}
DMS-REQ-0003 & Create and Maintain Science Data Archive \\
OSS-REQ-0127 & Level 1 Data Product Availability \\
\end{tabular}
\\ \hline
\end{longtable}
}


\subsubsection{Test Cases Summary}
\begin{longtable}{p{3cm}p{2.5cm}p{2.5cm}p{3cm}p{4cm}}
\toprule
\href{https://jira.lsstcorp.org/secure/Tests.jspa\#/testCase/LVV-T1276}{LVV-T1276} & \multicolumn{4}{p{12cm}}{ Verify implementation of latency of reporting optical transients } \\ \hline
\textbf{Owner} & \textbf{Status} & \textbf{Version} & \textbf{Critical Event} & \textbf{Verification Type} \\ \hline
Eric Bellm & Draft & 1 & false & Test \\ \hline
\end{longtable}
{\scriptsize
\textbf{Objective:}\\
Verify that alerts are generated for optical transients
within~\textbf{OTT1 = 1 minute~}of the completion of the readout of the
last image.
}
  
 \newpage 
\subsection{[LVV-9742] DMS-REQ-0271-V-02: Max nearby stars associated with DIASource }\label{lvv-9742}

\begin{longtable}{cccc}
\hline
\textbf{Jira Link} & \textbf{Assignee} & \textbf{Status} & \textbf{Test Cases}\\ \hline
\href{https://jira.lsstcorp.org/browse/LVV-9742}{LVV-9742} &
Leanne Guy & Not Covered &
\begin{tabular}{c}
\end{tabular}
\\
\hline
\end{longtable}

\textbf{Verification Element Description:} \\
Verify that no more than \textbf{diaNearbyObjMaxStar = 3} stars are
associated with each DIASource.

Associated element
(\href{https://jira.lsstcorp.org/browse/LVV-9743}{LVV-9743}) satisfies
the radius within which an Object is considered coincident with a
DIASource.

Associated element
(\href{https://jira.lsstcorp.org/browse/LVV-102}{LVV-102}) satisfies the
maximum number of galaxies that can be associated with a DIASource.

Â~

{\footnotesize
\begin{longtable}{p{2.5cm}p{13.5cm}}
\hline
\multicolumn{2}{c}{\textbf{Requirement Details}}\\ \hline
Requirement ID & DMS-REQ-0271 \\ \cdashline{1-2}
Requirement Description &
\begin{minipage}[]{13cm}
\textbf{Specification:} The DMS shall construct a catalog of all
astrophysical objects identified through difference image analysis
(DIAObjects). The DIAObject entries shall include metadata attributes
including at least: a unique identifier; the identifiers of the
\textbf{diaNearbyObjMaxStar} nearest stars and
\textbf{diaNearbyObjMaxGalaxy} nearest galaxies in the Object catalog
lying within \textbf{diaNearbyObjRadius}, the probability that the
DIAObject is the same as the nearby Object; and a set of DIAObject
properties.
\end{minipage}
\\ \cdashline{1-2}
Requirement Parameters & {[}\textbf{diaNearbyObjMaxGalaxy = 3{{[}integer{]}}} Maximum number of
nearby galaxies that can be associated with a DIASource.,
\textbf{diaNearbyObjRadius = 60{{[}arcsecond{]}}} Radius within which an
Object is considered to be near, and possibly coincident with, the
DIASource., \textbf{diaNearbyObjMaxStar = 3{{[}integer{]}}} Maximum
number of stars that can be associated with a DIASource.{]} \\ \cdashline{1-2}
Requirement Priority & 1b \\ \cdashline{1-2}
Upper Level Requirement &
\begin{tabular}{cl}
OSS-REQ-0130 & Catalogs (Level 1) \\
\end{tabular}
\\ \hline
\end{longtable}
}


  
 \newpage 
\subsection{[LVV-9744] DMS-REQ-0344-V-02: Latency of reporting optical transients }\label{lvv-9744}

\begin{longtable}{cccc}
\hline
\textbf{Jira Link} & \textbf{Assignee} & \textbf{Status} & \textbf{Test Cases}\\ \hline
\href{https://jira.lsstcorp.org/browse/LVV-9744}{LVV-9744} &
Leanne Guy & Not Covered &
\begin{tabular}{c}
LVV-T1866 \\
\end{tabular}
\\
\hline
\end{longtable}

\textbf{Verification Element Description:} \\
Verify that optical transients (Level 1 data products) are reported
within OTT1 = 1 minute of last image readout.

Associated element
(\href{https://jira.lsstcorp.org/browse/LVV-175}{LVV-175}) satisfies the
maximum time allotted for public release of L1 Data Products.

Â~

{\footnotesize
\begin{longtable}{p{2.5cm}p{13.5cm}}
\hline
\multicolumn{2}{c}{\textbf{Requirement Details}}\\ \hline
Requirement ID & DMS-REQ-0344 \\ \cdashline{1-2}
Requirement Description &
\begin{minipage}[]{13cm}
\textbf{Specification:} The publishing of Level 1 data products from
Special Programs shall be subject to the same performance requirements
of the standard Level 1 system. In particular \textbf{L1PublicT} and
\textbf{OTT1}.
\end{minipage}
\\ \cdashline{1-2}
Requirement Parameters & {[}\textbf{OTT1 = 1{{[}minute{]}}} The latency of reporting optical
transients following the completion of readout of the last image of a
visit, \textbf{L1PublicT = 24{{[}hour{]}}} Maximum time from the
acquisition of science data to the release of associated Level 1 Data
Products (except alerts){]} \\ \cdashline{1-2}
Requirement Priority & 2 \\ \cdashline{1-2}
Upper Level Requirement &
\begin{tabular}{cl}
OSS-REQ-0392 & Data Products Handling for Special Programs \\
\end{tabular}
\\ \hline
\end{longtable}
}


\subsubsection{Test Cases Summary}
\begin{longtable}{p{3cm}p{2.5cm}p{2.5cm}p{3cm}p{4cm}}
\toprule
\href{https://jira.lsstcorp.org/secure/Tests.jspa\#/testCase/LVV-T1866}{LVV-T1866} & \multicolumn{4}{p{12cm}}{ Verify latency of reporting optical transients from Special Programs } \\ \hline
\textbf{Owner} & \textbf{Status} & \textbf{Version} & \textbf{Critical Event} & \textbf{Verification Type} \\ \hline
Jeffrey Carlin & Draft & 1 & false & Test \\ \hline
\end{longtable}
{\scriptsize
\textbf{Objective:}\\
Verify that optical transients (Level 1 data products) are reported
within OTT1 = 1 minute of last image readout for Special Programs.
}
  
 \newpage 
\subsection{[LVV-9748] DMS-REQ-0343-V-02: Number of simultaneous users }\label{lvv-9748}

\begin{longtable}{cccc}
\hline
\textbf{Jira Link} & \textbf{Assignee} & \textbf{Status} & \textbf{Test Cases}\\ \hline
\href{https://jira.lsstcorp.org/browse/LVV-9748}{LVV-9748} &
Leanne Guy & Not Covered &
\begin{tabular}{c}
LVV-T1252 \\
\end{tabular}
\\
\hline
\end{longtable}

\textbf{Verification Element Description:} \\
Verify that the LSST alert filtering system supports at least 100
simultaneous users.

Additional element
(\href{https://jira.lsstcorp.org/browse/LVV-174}{LVV-174}) satisfies the
constraint on the number of alerts received per user.

{\footnotesize
\begin{longtable}{p{2.5cm}p{13.5cm}}
\hline
\multicolumn{2}{c}{\textbf{Requirement Details}}\\ \hline
Requirement ID & DMS-REQ-0343 \\ \cdashline{1-2}
Requirement Description &
\begin{minipage}[]{13cm}
\textbf{Specification:} The LSST alert filtering service shall support
\textbf{numBrokerUsers} simultaneous users with each user allocated a
bandwidth capable of receiving the equivalent of
\textbf{numBrokerAlerts} alerts per visit.
\end{minipage}
\\ \cdashline{1-2}
Requirement Parameters & {[}\textbf{numBrokerUsers = 100{{[}integer{]}}} Supported number of
simultaneous users connected to the LSST alert filtering system.,
\textbf{numBrokerAlerts = 20{{[}integer{]}}} Number of full-sized alerts
that can be received per visit per user.{]} \\ \cdashline{1-2}
Requirement Discussion &
\begin{minipage}[]{13cm}
\textbf{Discussion:} The constraint on number of alerts is specified for
the full VOEvent alert content, but could also be satisfied by all
alerts being received with minimal alert content.
\end{minipage}
\\ \cdashline{1-2}
Requirement Priority & 2 \\ \cdashline{1-2}
Upper Level Requirement &
\begin{tabular}{cl}
OSS-REQ-0193 & Alerts per Visit \\
OSS-REQ-0184 & Transient Alert Publication \\
\end{tabular}
\\ \hline
\end{longtable}
}


\subsubsection{Test Cases Summary}
\begin{longtable}{p{3cm}p{2.5cm}p{2.5cm}p{3cm}p{4cm}}
\toprule
\href{https://jira.lsstcorp.org/secure/Tests.jspa\#/testCase/LVV-T1252}{LVV-T1252} & \multicolumn{4}{p{12cm}}{ Verify number of simultaneous alert filter users } \\ \hline
\textbf{Owner} & \textbf{Status} & \textbf{Version} & \textbf{Critical Event} & \textbf{Verification Type} \\ \hline
Eric Bellm & Defined & 1 & false & Test \\ \hline
\end{longtable}
{\scriptsize
\textbf{Objective:}\\
Verify that the DMS alert filter service supports \textbf{numBrokerUsers
= 100~}simultaneous brokers.
}
  
 \newpage 
\subsection{[LVV-9749] DMS-REQ-0341-V-02: Min number of precovery service connections }\label{lvv-9749}

\begin{longtable}{cccc}
\hline
\textbf{Jira Link} & \textbf{Assignee} & \textbf{Status} & \textbf{Test Cases}\\ \hline
\href{https://jira.lsstcorp.org/browse/LVV-9749}{LVV-9749} &
Leanne Guy & Not Covered &
\begin{tabular}{c}
\end{tabular}
\\
\hline
\end{longtable}

\textbf{Verification Element Description:} \\
Submit multiple precovery requests and verify that at least the minimum
number of connections is supported.

Associated element
(\href{https://jira.lsstcorp.org/browse/LVV-172}{LVV-172}) satisfies the
maximum elapsed time for availability of precovery service results.

{\footnotesize
\begin{longtable}{p{2.5cm}p{13.5cm}}
\hline
\multicolumn{2}{c}{\textbf{Requirement Details}}\\ \hline
Requirement ID & DMS-REQ-0341 \\ \cdashline{1-2}
Requirement Description &
\begin{minipage}[]{13cm}
\textbf{Specification:} A ``precovery service'' shall be available to
end-users to request precovery for a provided sky location across all
previous visits, making the results available within
\textbf{precoveryServiceElapsed} hours of the request and supporting at
least \textbf{precoveryServicePeakUsers} submissions per hour.
\end{minipage}
\\ \cdashline{1-2}
Requirement Parameters & {[}\textbf{precoveryServiceElapsed = 24{{[}hour{]}}} Maximum time
between submitting a request and receiving the results.,
\textbf{precoveryServicePeakUsers = 10{{[}integer{]}}} Minimum number of
precovery service connections to be supported per hour.{]} \\ \cdashline{1-2}
Requirement Discussion &
\begin{minipage}[]{13cm}
\textbf{Discussion:} This is forced photometry on difference images from
each visit. This will include a web interface and scriptable APIs.
\end{minipage}
\\ \cdashline{1-2}
Requirement Priority & 1b \\ \cdashline{1-2}
Upper Level Requirement &
\begin{tabular}{cl}
OSS-REQ-0126 & Level 1 Data Products \\
\end{tabular}
\\ \hline
\end{longtable}
}


  
 \newpage 
\subsection{[LVV-9750] DMS-REQ-0364-V-02: Length of survey }\label{lvv-9750}

\begin{longtable}{cccc}
\hline
\textbf{Jira Link} & \textbf{Assignee} & \textbf{Status} & \textbf{Test Cases}\\ \hline
\href{https://jira.lsstcorp.org/browse/LVV-9750}{LVV-9750} &
Leanne Guy & Not Covered &
\begin{tabular}{c}
\end{tabular}
\\
\hline
\end{longtable}

\textbf{Verification Element Description:} \\
The survey length shall beÂ~\textbf{surveyYears = 10} years.

Associated element
(\href{https://jira.lsstcorp.org/browse/LVV-190}{LVV-190}) satisfies the
requirement on number of data releases over the survey.

{\footnotesize
\begin{longtable}{p{2.5cm}p{13.5cm}}
\hline
\multicolumn{2}{c}{\textbf{Requirement Details}}\\ \hline
Requirement ID & DMS-REQ-0364 \\ \cdashline{1-2}
Requirement Description &
\begin{minipage}[]{13cm}
\textbf{Specification:} The data access services shall be designed to
permit, and their software implementation shall support, the service of
at least \textbf{nDRTot} Data Releases accumulated over the (find the
actual survey-length parameter) \textbf{surveyYears}-year planned
survey.
\end{minipage}
\\ \cdashline{1-2}
Requirement Parameters & {[}\textbf{nDRTot = 11{{[}integer{]}}} Total number of data releases
over the survey., \textbf{surveyYears = 10{{[}integer{]}}} Length of the
survey in years{]} \\ \cdashline{1-2}
Requirement Discussion &
\begin{minipage}[]{13cm}
\textbf{Discussion:} It is an operations-era decision to choose the
actual number of releases to be served, and to allocate hardware
resources accordingly. ~The requirement is that the system delivered at
the close of the MREFC construction period be capable of handling ten
years of releases if the operations project chooses to allocate adequate
hardware resources.
\end{minipage}
\\ \cdashline{1-2}
Requirement Priority & 3 \\ \cdashline{1-2}
Upper Level Requirement &
\begin{tabular}{cl}
OSS-REQ-0396 & Data Access Services \\
\end{tabular}
\\ \hline
\end{longtable}
}


  
 \newpage 
\subsection{[LVV-9784] DMS-REQ-0355-V-02: Min number of simultaneous Prompt Products query
users }\label{lvv-9784}

\begin{longtable}{cccc}
\hline
\textbf{Jira Link} & \textbf{Assignee} & \textbf{Status} & \textbf{Test Cases}\\ \hline
\href{https://jira.lsstcorp.org/browse/LVV-9784}{LVV-9784} &
Leanne Guy & Not Covered &
\begin{tabular}{c}
\end{tabular}
\\
\hline
\end{longtable}

\textbf{Verification Element Description:} \\
A minimum of~\textbf{l1QueryUsers = 20}~users must be able to
simultaneously execute Prompt Products Database queries.

The associated element DMS-REQ-0355-V-01
\href{https://jira.lsstcorp.org/browse/LVV-186}{(LVV-186)} satisfies the
additional constraint on the maximum time to return Prompt Products
Database query results.

\emph{These requirements should be satisfied together.}

{\footnotesize
\begin{longtable}{p{2.5cm}p{13.5cm}}
\hline
\multicolumn{2}{c}{\textbf{Requirement Details}}\\ \hline
Requirement ID & DMS-REQ-0355 \\ \cdashline{1-2}
Requirement Description &
\begin{minipage}[]{13cm}
\textbf{Specification:} The live Prompt Products Database shall support
at least \textbf{l1QueryUsers} simultaneous queries, assuming each query
lasts no more than \textbf{l1QueryTime}.
\end{minipage}
\\ \cdashline{1-2}
Requirement Parameters & {[}\textbf{l1QueryTime = 10{{[}second{]}}} Maximum time allowed for
retrieving results of a query of the Prompt Products Database.,
\textbf{l1QueryUsers = 20{{[}integer{]}}} Minimum number of simultaneous
users querying the Prompt Products Database.{]} \\ \cdashline{1-2}
Requirement Priority & 1b \\ \cdashline{1-2}
Upper Level Requirement &
\begin{tabular}{cl}
OSS-REQ-0181 & Data Products Query and Download Infrastructure \\
\end{tabular}
\\ \hline
\end{longtable}
}


  
 \newpage 
\subsection{[LVV-9785] DMS-REQ-0356-V-02: Max size of low-volume query results }\label{lvv-9785}

\begin{longtable}{cccc}
\hline
\textbf{Jira Link} & \textbf{Assignee} & \textbf{Status} & \textbf{Test Cases}\\ \hline
\href{https://jira.lsstcorp.org/browse/LVV-9785}{LVV-9785} &
Leanne Guy & Not Covered &
\begin{tabular}{c}
\end{tabular}
\\
\hline
\end{longtable}

\textbf{Verification Element Description:} \\
For a query to be defined as ``low-volume,'' the maximum size of its
results must be no more than~\textbf{lvMaxReturnedResults =
0.5~gigabytes.}

The associated element
DMS-REQ-0356-V-01~(\href{https://jira.lsstcorp.org/browse/LVV-187}{LVV-187})~satisfies
the additional constraint on the radius of low volume queries.

The associated element
DMS-REQ-0356-V-03~(\href{https://jira.lsstcorp.org/browse/LVV-9786}{LVV-9786})~satisfies
the additional constraint on the number of simultaneous users.

The associated element
DMS-REQ-0356-V-04~(\href{https://jira.lsstcorp.org/browse/LVV-9787}{LVV-9787})~satisfies
the additional constraint on the maximum time to return low volume query
results.

\emph{These requirements should be satisfied together.}

{\footnotesize
\begin{longtable}{p{2.5cm}p{13.5cm}}
\hline
\multicolumn{2}{c}{\textbf{Requirement Details}}\\ \hline
Requirement ID & DMS-REQ-0356 \\ \cdashline{1-2}
Requirement Description &
\begin{minipage}[]{13cm}
\textbf{Specification:} Low volume queries, queries that are spatially
restricted to a circle of radius~\textbf{lvSkyRadius} and return at most
\textbf{lvMaxReturnedResults} of data, shall respond within
\textbf{lvQueryTime} under a load of \textbf{lvQueryUsers} simultaneous
queries.
\end{minipage}
\\ \cdashline{1-2}
Requirement Parameters & {[}\textbf{lvSkyRadius = 60{{[}arcsecond{]}}} Radius to be used for a
low-volume query on the sky., \textbf{lvMaxReturnedResults =
0.5{{[}gigabyte{]}}} Maximum size of a results set for a query to be
defined to be ``low-volume''., \textbf{lvQueryUsers =
100{{[}integer{]}}} Minimum number of simultaneous users performing low
volume queries., \textbf{lvQueryTime = 10{{[}second{]}}} Maximum time
allowed for retrieving results of a low-volume query.{]} \\ \cdashline{1-2}
Requirement Discussion &
\begin{minipage}[]{13cm}
\textbf{Discussion:} We are evaluating whether the latency requirements
of low-volume queries can also be met for certain categories of temporal
queries or queries on indexed attributes which limit the scope of
per-row operations in the query (such as non-indexed WHERE evaluations)
to a comparable fraction of the total dataset. The low-volume query
requirements also apply to queries selecting data by the primary key of
any data product table, or by the associated Object-like primary key for
the ForcedSource and DIASource tables.
\end{minipage}
\\ \cdashline{1-2}
Requirement Priority & 1b \\ \cdashline{1-2}
Upper Level Requirement &
\begin{tabular}{cl}
OSS-REQ-0181 & Data Products Query and Download Infrastructure \\
\end{tabular}
\\ \hline
\end{longtable}
}


  
 \newpage 
\subsection{[LVV-9786] DMS-REQ-0356-V-03: Min number of simultaneous low-volume query users }\label{lvv-9786}

\begin{longtable}{cccc}
\hline
\textbf{Jira Link} & \textbf{Assignee} & \textbf{Status} & \textbf{Test Cases}\\ \hline
\href{https://jira.lsstcorp.org/browse/LVV-9786}{LVV-9786} &
Leanne Guy & Not Covered &
\begin{tabular}{c}
LVV-T1089 \\
LVV-T1090 \\
\end{tabular}
\\
\hline
\end{longtable}

\textbf{Verification Element Description:} \\
A minimum of~\textbf{lvQueryUsers = 100}~users must be able to
simultaneously execute low volume queries.

The associated element
DMS-REQ-0356-V-01~(\href{https://jira.lsstcorp.org/browse/LVV-187}{LVV-187})~satisfies
the additional constraint on the radius of low volume queries.

The associated element
DMS-REQ-0356-V-02~(\href{https://jira.lsstcorp.org/browse/LVV-9785}{LVV-9785})~satisfies
the additional constraint on the maximum size of low volume queries.

The associated element
DMS-REQ-0356-V-04~(\href{https://jira.lsstcorp.org/browse/LVV-9787}{LVV-9787})~satisfies
the additional constraint on the maximum time to return low volume query
results.

\emph{These requirements should be satisfied together.}

{\footnotesize
\begin{longtable}{p{2.5cm}p{13.5cm}}
\hline
\multicolumn{2}{c}{\textbf{Requirement Details}}\\ \hline
Requirement ID & DMS-REQ-0356 \\ \cdashline{1-2}
Requirement Description &
\begin{minipage}[]{13cm}
\textbf{Specification:} Low volume queries, queries that are spatially
restricted to a circle of radius~\textbf{lvSkyRadius} and return at most
\textbf{lvMaxReturnedResults} of data, shall respond within
\textbf{lvQueryTime} under a load of \textbf{lvQueryUsers} simultaneous
queries.
\end{minipage}
\\ \cdashline{1-2}
Requirement Parameters & {[}\textbf{lvSkyRadius = 60{{[}arcsecond{]}}} Radius to be used for a
low-volume query on the sky., \textbf{lvMaxReturnedResults =
0.5{{[}gigabyte{]}}} Maximum size of a results set for a query to be
defined to be ``low-volume''., \textbf{lvQueryUsers =
100{{[}integer{]}}} Minimum number of simultaneous users performing low
volume queries., \textbf{lvQueryTime = 10{{[}second{]}}} Maximum time
allowed for retrieving results of a low-volume query.{]} \\ \cdashline{1-2}
Requirement Discussion &
\begin{minipage}[]{13cm}
\textbf{Discussion:} We are evaluating whether the latency requirements
of low-volume queries can also be met for certain categories of temporal
queries or queries on indexed attributes which limit the scope of
per-row operations in the query (such as non-indexed WHERE evaluations)
to a comparable fraction of the total dataset. The low-volume query
requirements also apply to queries selecting data by the primary key of
any data product table, or by the associated Object-like primary key for
the ForcedSource and DIASource tables.
\end{minipage}
\\ \cdashline{1-2}
Requirement Priority & 1b \\ \cdashline{1-2}
Upper Level Requirement &
\begin{tabular}{cl}
OSS-REQ-0181 & Data Products Query and Download Infrastructure \\
\end{tabular}
\\ \hline
\end{longtable}
}


\subsubsection{Test Cases Summary}
\begin{longtable}{p{3cm}p{2.5cm}p{2.5cm}p{3cm}p{4cm}}
\toprule
\href{https://jira.lsstcorp.org/secure/Tests.jspa\#/testCase/LVV-T1089}{LVV-T1089} & \multicolumn{4}{p{12cm}}{ Load Test } \\ \hline
\textbf{Owner} & \textbf{Status} & \textbf{Version} & \textbf{Critical Event} & \textbf{Verification Type} \\ \hline
Fritz Mueller & Approved & 1 & false & Test \\ \hline
\end{longtable}
{\scriptsize
\textbf{Objective:}\\
This test will check that Qserv is able to meet average query completion
time targets per query class under a representative load of simultaneous
high and low volume queries while running against an appropriately
scaled test catalog.
}
\begin{longtable}{p{3cm}p{2.5cm}p{2.5cm}p{3cm}p{4cm}}
\toprule
\href{https://jira.lsstcorp.org/secure/Tests.jspa\#/testCase/LVV-T1090}{LVV-T1090} & \multicolumn{4}{p{12cm}}{ Heavy Load Test } \\ \hline
\textbf{Owner} & \textbf{Status} & \textbf{Version} & \textbf{Critical Event} & \textbf{Verification Type} \\ \hline
Fritz Mueller & Approved & 1 & false & Test \\ \hline
\end{longtable}
{\scriptsize
\textbf{Objective:}\\
This test will check that Qserv is able to meet average query completion
time targets per query class under a higher than average load of
simultaneous high and low volume queries while running against an
appropriately scaled test catalog.
}
  
 \newpage 
\subsection{[LVV-9787] DMS-REQ-0356-V-04: Max time to retrieve low-volume query results }\label{lvv-9787}

\begin{longtable}{cccc}
\hline
\textbf{Jira Link} & \textbf{Assignee} & \textbf{Status} & \textbf{Test Cases}\\ \hline
\href{https://jira.lsstcorp.org/browse/LVV-9787}{LVV-9787} &
Leanne Guy & Not Covered &
\begin{tabular}{c}
LVV-T1085 \\
LVV-T1089 \\
LVV-T1090 \\
\end{tabular}
\\
\hline
\end{longtable}

\textbf{Verification Element Description:} \\
Low volume query results shall be retrievable in a maximum time
of~\textbf{lvQueryTime = 10~seconds.}

The associated element
DMS-REQ-0356-V-01~(\href{https://jira.lsstcorp.org/browse/LVV-187}{LVV-187})~satisfies
the additional constraint on the radius of low volume queries.

The associated element
DMS-REQ-0356-V-02~(\href{https://jira.lsstcorp.org/browse/LVV-9785}{LVV-9785})~satisfies
the additional constraint on the maximum size of low volume queries.

The associated element
DMS-REQ-0356-V-03~(\href{https://jira.lsstcorp.org/browse/LVV-9786}{LVV-9786})~satisfies
the additional constraint on the number of simultaneous users.

\emph{These requirements should be satisfied together.}

{\footnotesize
\begin{longtable}{p{2.5cm}p{13.5cm}}
\hline
\multicolumn{2}{c}{\textbf{Requirement Details}}\\ \hline
Requirement ID & DMS-REQ-0356 \\ \cdashline{1-2}
Requirement Description &
\begin{minipage}[]{13cm}
\textbf{Specification:} Low volume queries, queries that are spatially
restricted to a circle of radius~\textbf{lvSkyRadius} and return at most
\textbf{lvMaxReturnedResults} of data, shall respond within
\textbf{lvQueryTime} under a load of \textbf{lvQueryUsers} simultaneous
queries.
\end{minipage}
\\ \cdashline{1-2}
Requirement Parameters & {[}\textbf{lvSkyRadius = 60{{[}arcsecond{]}}} Radius to be used for a
low-volume query on the sky., \textbf{lvMaxReturnedResults =
0.5{{[}gigabyte{]}}} Maximum size of a results set for a query to be
defined to be ``low-volume''., \textbf{lvQueryUsers =
100{{[}integer{]}}} Minimum number of simultaneous users performing low
volume queries., \textbf{lvQueryTime = 10{{[}second{]}}} Maximum time
allowed for retrieving results of a low-volume query.{]} \\ \cdashline{1-2}
Requirement Discussion &
\begin{minipage}[]{13cm}
\textbf{Discussion:} We are evaluating whether the latency requirements
of low-volume queries can also be met for certain categories of temporal
queries or queries on indexed attributes which limit the scope of
per-row operations in the query (such as non-indexed WHERE evaluations)
to a comparable fraction of the total dataset. The low-volume query
requirements also apply to queries selecting data by the primary key of
any data product table, or by the associated Object-like primary key for
the ForcedSource and DIASource tables.
\end{minipage}
\\ \cdashline{1-2}
Requirement Priority & 1b \\ \cdashline{1-2}
Upper Level Requirement &
\begin{tabular}{cl}
OSS-REQ-0181 & Data Products Query and Download Infrastructure \\
\end{tabular}
\\ \hline
\end{longtable}
}


\subsubsection{Test Cases Summary}
\begin{longtable}{p{3cm}p{2.5cm}p{2.5cm}p{3cm}p{4cm}}
\toprule
\href{https://jira.lsstcorp.org/secure/Tests.jspa\#/testCase/LVV-T1085}{LVV-T1085} & \multicolumn{4}{p{12cm}}{ Short Queries Functional Test } \\ \hline
\textbf{Owner} & \textbf{Status} & \textbf{Version} & \textbf{Critical Event} & \textbf{Verification Type} \\ \hline
Fritz Mueller & Approved & 1 & false & Test \\ \hline
\end{longtable}
{\scriptsize
\textbf{Objective:}\\
The objective of this test is to ensure that the short queries are
performing as expected and establish a timing baseline benchmark for
these types of queries.
}
\begin{longtable}{p{3cm}p{2.5cm}p{2.5cm}p{3cm}p{4cm}}
\toprule
\href{https://jira.lsstcorp.org/secure/Tests.jspa\#/testCase/LVV-T1089}{LVV-T1089} & \multicolumn{4}{p{12cm}}{ Load Test } \\ \hline
\textbf{Owner} & \textbf{Status} & \textbf{Version} & \textbf{Critical Event} & \textbf{Verification Type} \\ \hline
Fritz Mueller & Approved & 1 & false & Test \\ \hline
\end{longtable}
{\scriptsize
\textbf{Objective:}\\
This test will check that Qserv is able to meet average query completion
time targets per query class under a representative load of simultaneous
high and low volume queries while running against an appropriately
scaled test catalog.
}
\begin{longtable}{p{3cm}p{2.5cm}p{2.5cm}p{3cm}p{4cm}}
\toprule
\href{https://jira.lsstcorp.org/secure/Tests.jspa\#/testCase/LVV-T1090}{LVV-T1090} & \multicolumn{4}{p{12cm}}{ Heavy Load Test } \\ \hline
\textbf{Owner} & \textbf{Status} & \textbf{Version} & \textbf{Critical Event} & \textbf{Verification Type} \\ \hline
Fritz Mueller & Approved & 1 & false & Test \\ \hline
\end{longtable}
{\scriptsize
\textbf{Objective:}\\
This test will check that Qserv is able to meet average query completion
time targets per query class under a higher than average load of
simultaneous high and low volume queries while running against an
appropriately scaled test catalog.
}
  
 \newpage 
\subsection{[LVV-9788] DMS-REQ-0358-V-02: Max time to retrieve DM EFD query results }\label{lvv-9788}

\begin{longtable}{cccc}
\hline
\textbf{Jira Link} & \textbf{Assignee} & \textbf{Status} & \textbf{Test Cases}\\ \hline
\href{https://jira.lsstcorp.org/browse/LVV-9788}{LVV-9788} &
Leanne Guy & Not Covered &
\begin{tabular}{c}
LVV-T1251 \\
\end{tabular}
\\
\hline
\end{longtable}

\textbf{Verification Element Description:} \\
DM EFD query results shall be retrievable in a maximum time
of~\textbf{dmEfdQueryTime = 10~seconds.}

The associated element
DMS-REQ-0358-V-01~\href{https://jira.lsstcorp.org/browse/LVV-3400}{(LVV-3400)}~satisfies
the additional constraint on the number of simultaneous users.

\emph{These requirements should be satisfied together.}

{\footnotesize
\begin{longtable}{p{2.5cm}p{13.5cm}}
\hline
\multicolumn{2}{c}{\textbf{Requirement Details}}\\ \hline
Requirement ID & DMS-REQ-0358 \\ \cdashline{1-2}
Requirement Description &
\begin{minipage}[]{13cm}
\textbf{Specification:} The DM copy of the EFD shall support at least
\textbf{dmEfdQueryUsers} simultaneous queries, assuming each query lasts
no more than \textbf{dmEfdQueryTime}.
\end{minipage}
\\ \cdashline{1-2}
Requirement Parameters & {[}\textbf{dmEfdQueryTime = 10{{[}second{]}}} Maximum time allowed for
retrieving results of a DM EFD query., \textbf{dmEfdQueryUsers =
5{{[}integer{]}}} Minimum number of simultaneous users querying the DM
EFD.{]} \\ \cdashline{1-2}
Requirement Priority & 1a \\ \cdashline{1-2}
Upper Level Requirement &
\begin{tabular}{cl}
OSS-REQ-0181 & Data Products Query and Download Infrastructure \\
\end{tabular}
\\ \hline
\end{longtable}
}


\subsubsection{Test Cases Summary}
\begin{longtable}{p{3cm}p{2.5cm}p{2.5cm}p{3cm}p{4cm}}
\toprule
\href{https://jira.lsstcorp.org/secure/Tests.jspa\#/testCase/LVV-T1251}{LVV-T1251} & \multicolumn{4}{p{12cm}}{ Verify implementation of maximum time to retrieve DM EFD query results } \\ \hline
\textbf{Owner} & \textbf{Status} & \textbf{Version} & \textbf{Critical Event} & \textbf{Verification Type} \\ \hline
Jeffrey Carlin & Draft & 1 & false & Test \\ \hline
\end{longtable}
{\scriptsize
\textbf{Objective:}\\
Verify that the DM EFD can support \textbf{dmEfdQueryUsers~= 5}
simultaneous queries, with each query must executing in no more than
\textbf{dmEfdQueryTime = 10 seconds.~}The requirement on at least 5
simultaneous queries will be verified separately in
\href{https://jira.lsstcorp.org/secure/Tests.jspa\#/testCase/LVV-T1250}{LVV-T1250},\href{https://jira.lsstcorp.org/secure/Tests.jspa\#/testCase/LVV-T1251}{}
but these must be satisfied together.
}
  
 \newpage 
\subsection{[LVV-9789] DMS-REQ-0373-V-02: Max time to retrieve large-area coadd image }\label{lvv-9789}

\begin{longtable}{cccc}
\hline
\textbf{Jira Link} & \textbf{Assignee} & \textbf{Status} & \textbf{Test Cases}\\ \hline
\href{https://jira.lsstcorp.org/browse/LVV-9789}{LVV-9789} &
Leanne Guy & Not Covered &
\begin{tabular}{c}
\end{tabular}
\\
\hline
\end{longtable}

\textbf{Verification Element Description:} \\
Large-area coadds shall be retrievable in a maximum time
of~\textbf{fplaneRetrievalTime = 60~seconds.}

The associated element DMS-REQ-0373-V-01
\href{https://jira.lsstcorp.org/browse/LVV-3397}{(LVV-3397)} satisfies
the additional constraint on the number of simultaneous users.

\emph{These requirements should be satisfied together.}

{\footnotesize
\begin{longtable}{p{2.5cm}p{13.5cm}}
\hline
\multicolumn{2}{c}{\textbf{Requirement Details}}\\ \hline
Requirement ID & DMS-REQ-0373 \\ \cdashline{1-2}
Requirement Description &
\begin{minipage}[]{13cm}
\textbf{Specification:} A 10 square degree coadd, including mask and
variance planes, shall be retrievable using the IVOA SODA protocol
within \textbf{fplaneRetrievalTime} with \textbf{fplaneRetrievalUsers}
simultaneous requests for distinct areas of the sky.
\end{minipage}
\\ \cdashline{1-2}
Requirement Parameters & {[}\textbf{fplaneRetrievalTime = 60{{[}second{]}}} Maximum time allowed
for retrieving a focal-plane sized coadd., \textbf{fplaneRetrievalUsers
= 10{{[}integer{]}}} Number of simultaneous users retrieving a single
large area coadd.{]} \\ \cdashline{1-2}
Requirement Priority & 2 \\ \cdashline{1-2}
Upper Level Requirement &
\begin{tabular}{cl}
OSS-REQ-0181 & Data Products Query and Download Infrastructure \\
\end{tabular}
\\ \hline
\end{longtable}
}


  
 \newpage 
\subsection{[LVV-9790] DMS-REQ-0374-V-02: Min number of simultaneous PVI image users }\label{lvv-9790}

\begin{longtable}{cccc}
\hline
\textbf{Jira Link} & \textbf{Assignee} & \textbf{Status} & \textbf{Test Cases}\\ \hline
\href{https://jira.lsstcorp.org/browse/LVV-9790}{LVV-9790} &
Leanne Guy & Not Covered &
\begin{tabular}{c}
\end{tabular}
\\
\hline
\end{longtable}

\textbf{Verification Element Description:} \\
At least~\textbf{pviRetrievalUsers = 20}~simultaneous users shall be
able to retrieve single, distinct PVI images.

Associated element DMS-REQ-0374-V-01
(\href{https://jira.lsstcorp.org/browse/LVV-3395}{LVV-3395})~satisfies
the constraint on retrieval time for PVI images.

Associated element~DMS-REQ-0374-V-03
(\href{https://jira.lsstcorp.org/browse/LVV-9791}{LVV-9791}) satisfies
the expected lifetime of Level-1 data products.

\emph{These requirements should be satisfied together.}

{\footnotesize
\begin{longtable}{p{2.5cm}p{13.5cm}}
\hline
\multicolumn{2}{c}{\textbf{Requirement Details}}\\ \hline
Requirement ID & DMS-REQ-0374 \\ \cdashline{1-2}
Requirement Description &
\begin{minipage}[]{13cm}
\textbf{Specification:} A Processed Visit Image of a single CCD shall be
retrievable using the VO SIAv2 protocol within \textbf{pviRetrievalTime}
with \textbf{pviRetrievalUsers} simultaneous requests for distinct
single-CCD PVIs.
\end{minipage}
\\ \cdashline{1-2}
Requirement Parameters & {[}\textbf{pviRetrievalTime = 10{{[}second{]}}} Maximum time allowed for
retrieving a PVI image of a single CCD from a single visit,
\textbf{l1CacheLifetime = 30{{[}day{]}}} Lifetime in the cache of
un-archived Level-1 data products., \textbf{pviRetrievalUsers =
20{{[}integer{]}}} Minimum number of simultaneous users retrieving a
single PVI image.{]} \\ \cdashline{1-2}
Requirement Discussion &
\begin{minipage}[]{13cm}
\textbf{Discussion:} The performance targets for this requirement assume
the PVIs are available as files on a file system. For example, this
could be those files present in the \textbf{l1CacheLifetime} cache.
\end{minipage}
\\ \cdashline{1-2}
Requirement Priority & 1b \\ \cdashline{1-2}
Upper Level Requirement &
\begin{tabular}{cl}
OSS-REQ-0181 & Data Products Query and Download Infrastructure \\
\end{tabular}
\\ \hline
\end{longtable}
}


  
 \newpage 
\subsection{[LVV-9791] DMS-REQ-0374-V-03: Uncached L1 data product lifetime - single-CCD }\label{lvv-9791}

\begin{longtable}{cccc}
\hline
\textbf{Jira Link} & \textbf{Assignee} & \textbf{Status} & \textbf{Test Cases}\\ \hline
\href{https://jira.lsstcorp.org/browse/LVV-9791}{LVV-9791} &
Leanne Guy & Not Covered &
\begin{tabular}{c}
\end{tabular}
\\
\hline
\end{longtable}

\textbf{Verification Element Description:} \\
The PVIs must be available as files on the files system
for~\textbf{l1CacheLifetime = 30~days.}

Associated element DMS-REQ-0374-V-01
(\href{https://jira.lsstcorp.org/browse/LVV-3395}{LVV-3395})~satisfies
the constraint on retrieval time for PVI images.

Associated element~DMS-REQ-0374-V-02
\href{https://jira.lsstcorp.org/browse/LVV-9790}{(LVV-9790)} satisfies
the additional constraint on the number of simultaneous users.

\emph{These requirements should be satisfied together.}

{\footnotesize
\begin{longtable}{p{2.5cm}p{13.5cm}}
\hline
\multicolumn{2}{c}{\textbf{Requirement Details}}\\ \hline
Requirement ID & DMS-REQ-0374 \\ \cdashline{1-2}
Requirement Description &
\begin{minipage}[]{13cm}
\textbf{Specification:} A Processed Visit Image of a single CCD shall be
retrievable using the VO SIAv2 protocol within \textbf{pviRetrievalTime}
with \textbf{pviRetrievalUsers} simultaneous requests for distinct
single-CCD PVIs.
\end{minipage}
\\ \cdashline{1-2}
Requirement Parameters & {[}\textbf{pviRetrievalTime = 10{{[}second{]}}} Maximum time allowed for
retrieving a PVI image of a single CCD from a single visit,
\textbf{l1CacheLifetime = 30{{[}day{]}}} Lifetime in the cache of
un-archived Level-1 data products., \textbf{pviRetrievalUsers =
20{{[}integer{]}}} Minimum number of simultaneous users retrieving a
single PVI image.{]} \\ \cdashline{1-2}
Requirement Discussion &
\begin{minipage}[]{13cm}
\textbf{Discussion:} The performance targets for this requirement assume
the PVIs are available as files on a file system. For example, this
could be those files present in the \textbf{l1CacheLifetime} cache.
\end{minipage}
\\ \cdashline{1-2}
Requirement Priority & 1b \\ \cdashline{1-2}
Upper Level Requirement &
\begin{tabular}{cl}
OSS-REQ-0181 & Data Products Query and Download Infrastructure \\
\end{tabular}
\\ \hline
\end{longtable}
}


  
 \newpage 
\subsection{[LVV-9792] DMS-REQ-0375-V-02: Min size of postage stamp cutout }\label{lvv-9792}

\begin{longtable}{cccc}
\hline
\textbf{Jira Link} & \textbf{Assignee} & \textbf{Status} & \textbf{Test Cases}\\ \hline
\href{https://jira.lsstcorp.org/browse/LVV-9792}{LVV-9792} &
Leanne Guy & Not Covered &
\begin{tabular}{c}
\end{tabular}
\\
\hline
\end{longtable}

\textbf{Verification Element Description:} \\
Postage stamp cutouts from images must be at
least~\textbf{postageStampSize = 51~pixels} in size.

Associated element DMS-REQ-0375-V-01
(\href{https://jira.lsstcorp.org/browse/LVV-3398}{LVV-3398}) satisfies
the maximum retrieval time for postage stamp images.

Associated element~DMS-REQ-0375-V-03
(\href{https://jira.lsstcorp.org/browse/LVV-9793}{LVV-9793}) satisfies
the expected lifetime of Level-1 data products.

The associated element DMS-REQ-0375-V-04
(\href{https://jira.lsstcorp.org/browse/LVV-9794}{LVV-9794}) satisfies
the additional constraint on the number of simultaneous users retrieving
postage stamp images.

\emph{These requirements should be satisfied together.}

{\footnotesize
\begin{longtable}{p{2.5cm}p{13.5cm}}
\hline
\multicolumn{2}{c}{\textbf{Requirement Details}}\\ \hline
Requirement ID & DMS-REQ-0375 \\ \cdashline{1-2}
Requirement Description &
\begin{minipage}[]{13cm}
\textbf{Specification:} Postage stamp cutouts, of size
\textbf{postageStampSize} square, of all observations of a single Object
shall be retrievable within \textbf{postageStampRetrievalTime}, with
\textbf{postageStampRetrievalUsers} simultaneous requests of distinct
Objects.
\end{minipage}
\\ \cdashline{1-2}
Requirement Parameters & {[}\textbf{postageStampRetrievalUsers = 10{{[}integer{]}}} Minimum
number of simultaneous users retrieving a set of postage stamp images.,
\textbf{postageStampRetrievalTime = 10{{[}second{]}}} Maximum time
allowed for retrieving a set of postage stamp images of a single
Object., \textbf{postageStampSize = 51{{[}pixel{]}}} Minimum square size
of a postage stamp cutout from an image., \textbf{l1CacheLifetime =
30{{[}day{]}}} Lifetime in the cache of un-archived Level-1 data
products.{]} \\ \cdashline{1-2}
Requirement Discussion &
\begin{minipage}[]{13cm}
\textbf{Discussion:} The performance targets for this requirement assume
the PVIs are available as files on a file system. For example, this
could be those files present in the \textbf{l1CacheLifetime} cache.
\end{minipage}
\\ \cdashline{1-2}
Requirement Priority & 2 \\ \cdashline{1-2}
Upper Level Requirement &
\begin{tabular}{cl}
OSS-REQ-0181 & Data Products Query and Download Infrastructure \\
\end{tabular}
\\ \hline
\end{longtable}
}


  
 \newpage 
\subsection{[LVV-9793] DMS-REQ-0375-V-03: Uncached L1 data product lifetime - postage stamp }\label{lvv-9793}

\begin{longtable}{cccc}
\hline
\textbf{Jira Link} & \textbf{Assignee} & \textbf{Status} & \textbf{Test Cases}\\ \hline
\href{https://jira.lsstcorp.org/browse/LVV-9793}{LVV-9793} &
Leanne Guy & Not Covered &
\begin{tabular}{c}
\end{tabular}
\\
\hline
\end{longtable}

\textbf{Verification Element Description:} \\
The PVIs must be available as files on the files system
for~\textbf{l1CacheLifetime = 30~days.}

Associated element DMS-REQ-0375-V-01
\href{https://jira.lsstcorp.org/browse/LVV-3398}{(LVV-3398)} satisfies
the maximum retrieval time for postage stamp images.

The associated element DMS-REQ-0375-V-02
\href{https://jira.lsstcorp.org/browse/LVV-9792}{(LVV-9792)} satisfies
the additional constraint on the minimum size of a postage stamp cutout.

The associated element DMS-REQ-0375-V-04
\href{https://jira.lsstcorp.org/browse/LVV-9794}{(LVV-9794)} satisfies
the additional constraint on the number of simultaneous users retrieving
postage stamp images.

\emph{These requirements should be satisfied together.}

{\footnotesize
\begin{longtable}{p{2.5cm}p{13.5cm}}
\hline
\multicolumn{2}{c}{\textbf{Requirement Details}}\\ \hline
Requirement ID & DMS-REQ-0375 \\ \cdashline{1-2}
Requirement Description &
\begin{minipage}[]{13cm}
\textbf{Specification:} Postage stamp cutouts, of size
\textbf{postageStampSize} square, of all observations of a single Object
shall be retrievable within \textbf{postageStampRetrievalTime}, with
\textbf{postageStampRetrievalUsers} simultaneous requests of distinct
Objects.
\end{minipage}
\\ \cdashline{1-2}
Requirement Parameters & {[}\textbf{postageStampRetrievalUsers = 10{{[}integer{]}}} Minimum
number of simultaneous users retrieving a set of postage stamp images.,
\textbf{postageStampRetrievalTime = 10{{[}second{]}}} Maximum time
allowed for retrieving a set of postage stamp images of a single
Object., \textbf{postageStampSize = 51{{[}pixel{]}}} Minimum square size
of a postage stamp cutout from an image., \textbf{l1CacheLifetime =
30{{[}day{]}}} Lifetime in the cache of un-archived Level-1 data
products.{]} \\ \cdashline{1-2}
Requirement Discussion &
\begin{minipage}[]{13cm}
\textbf{Discussion:} The performance targets for this requirement assume
the PVIs are available as files on a file system. For example, this
could be those files present in the \textbf{l1CacheLifetime} cache.
\end{minipage}
\\ \cdashline{1-2}
Requirement Priority & 2 \\ \cdashline{1-2}
Upper Level Requirement &
\begin{tabular}{cl}
OSS-REQ-0181 & Data Products Query and Download Infrastructure \\
\end{tabular}
\\ \hline
\end{longtable}
}


  
 \newpage 
\subsection{[LVV-9794] DMS-REQ-0375-V-04: Min number of simultaneous postage stamp users }\label{lvv-9794}

\begin{longtable}{cccc}
\hline
\textbf{Jira Link} & \textbf{Assignee} & \textbf{Status} & \textbf{Test Cases}\\ \hline
\href{https://jira.lsstcorp.org/browse/LVV-9794}{LVV-9794} &
Leanne Guy & Not Covered &
\begin{tabular}{c}
\end{tabular}
\\
\hline
\end{longtable}

\textbf{Verification Element Description:} \\
A minimum of \textbf{postageStampRetrievalUsers = 10} users must be able
to simultaneously retrieve distinct sets of postage stamp cutouts.

Associated element DMS-REQ-0375-V-01
(\href{https://jira.lsstcorp.org/browse/LVV-3398}{LVV-3398}) satisfies
the maximum retrieval time for postage stamp images.

The associated element
DMS-REQ-0375-V-02~(\href{https://jira.lsstcorp.org/browse/LVV-9792}{LVV-9792})~satisfies
the additional constraint on the minimum size of a postage stamp cutout.

Associated element~DMS-REQ-0375-V-03
(\href{https://jira.lsstcorp.org/browse/LVV-9793}{LVV-9793}) satisfies
the expected lifetime of Level-1 data products.

\emph{These requirements should be satisfied together.}

{\footnotesize
\begin{longtable}{p{2.5cm}p{13.5cm}}
\hline
\multicolumn{2}{c}{\textbf{Requirement Details}}\\ \hline
Requirement ID & DMS-REQ-0375 \\ \cdashline{1-2}
Requirement Description &
\begin{minipage}[]{13cm}
\textbf{Specification:} Postage stamp cutouts, of size
\textbf{postageStampSize} square, of all observations of a single Object
shall be retrievable within \textbf{postageStampRetrievalTime}, with
\textbf{postageStampRetrievalUsers} simultaneous requests of distinct
Objects.
\end{minipage}
\\ \cdashline{1-2}
Requirement Parameters & {[}\textbf{postageStampRetrievalUsers = 10{{[}integer{]}}} Minimum
number of simultaneous users retrieving a set of postage stamp images.,
\textbf{postageStampRetrievalTime = 10{{[}second{]}}} Maximum time
allowed for retrieving a set of postage stamp images of a single
Object., \textbf{postageStampSize = 51{{[}pixel{]}}} Minimum square size
of a postage stamp cutout from an image., \textbf{l1CacheLifetime =
30{{[}day{]}}} Lifetime in the cache of un-archived Level-1 data
products.{]} \\ \cdashline{1-2}
Requirement Discussion &
\begin{minipage}[]{13cm}
\textbf{Discussion:} The performance targets for this requirement assume
the PVIs are available as files on a file system. For example, this
could be those files present in the \textbf{l1CacheLifetime} cache.
\end{minipage}
\\ \cdashline{1-2}
Requirement Priority & 2 \\ \cdashline{1-2}
Upper Level Requirement &
\begin{tabular}{cl}
OSS-REQ-0181 & Data Products Query and Download Infrastructure \\
\end{tabular}
\\ \hline
\end{longtable}
}


  
 \newpage 
\subsection{[LVV-9795] DMS-REQ-0376-V-02: Min number of simultaneous users retrieving all PVI
images }\label{lvv-9795}

\begin{longtable}{cccc}
\hline
\textbf{Jira Link} & \textbf{Assignee} & \textbf{Status} & \textbf{Test Cases}\\ \hline
\href{https://jira.lsstcorp.org/browse/LVV-9795}{LVV-9795} &
Leanne Guy & Not Covered &
\begin{tabular}{c}
\end{tabular}
\\
\hline
\end{longtable}

\textbf{Verification Element Description:} \\
The minimum number of simultaneous users retrieving distinct focal-plane
PVI sets shall be~\textbf{allPviRetrievalUsers = 10.}

Associated element DMS-REQ-0376-V-01
(\href{https://jira.lsstcorp.org/browse/LVV-3396}{LVV-3396})~satisfies
the maximum retrieval time.

Associated element~DMS-REQ-0376-V-03
(\href{https://jira.lsstcorp.org/browse/LVV-9796}{LVV-9796}) satisfies
the expected lifetime of Level-1 data products.

\emph{These requirements should be satisfied both separately and
together.}

{\footnotesize
\begin{longtable}{p{2.5cm}p{13.5cm}}
\hline
\multicolumn{2}{c}{\textbf{Requirement Details}}\\ \hline
Requirement ID & DMS-REQ-0376 \\ \cdashline{1-2}
Requirement Description &
\begin{minipage}[]{13cm}
\textbf{Specification:} All Processed Visit Images for a single visit
that are available in cache, including mask and variance planes, shall
be identifiable with a single IVOA SIAv2 service query and retrievable,
using the link(s) provided in the response, within
\textbf{allPviRetrievalTime.} This requirement shall be met for up to
\textbf{allPviRetrievalUsers} simultaneous requests for distinct
focal-plane PVI sets.
\end{minipage}
\\ \cdashline{1-2}
Requirement Parameters & {[}\textbf{allPviRetrievalUsers = 10{{[}integer{]}}} Minimum number of
simultaneous users retrieving all PVI images for a visit.,
\textbf{allPviRetrievalTime = 60{{[}second{]}}} Maximum time allowed for
retrieving all PVI images of a single visit., \textbf{l1CacheLifetime =
30{{[}day{]}}} Lifetime in the cache of un-archived Level-1 data
products.{]} \\ \cdashline{1-2}
Requirement Discussion &
\begin{minipage}[]{13cm}
\textbf{Discussion:} The performance targets for this requirement assume
the PVIs are available as files on a file system. For example, this
could be those files present in the \textbf{l1CacheLifetime} cache.
\end{minipage}
\\ \cdashline{1-2}
Requirement Priority & 1b \\ \cdashline{1-2}
Upper Level Requirement &
\begin{tabular}{cl}
OSS-REQ-0181 & Data Products Query and Download Infrastructure \\
\end{tabular}
\\ \hline
\end{longtable}
}


  
 \newpage 
\subsection{[LVV-9796] DMS-REQ-0376-V-03: Uncached L1 data product lifetime - focal-plane }\label{lvv-9796}

\begin{longtable}{cccc}
\hline
\textbf{Jira Link} & \textbf{Assignee} & \textbf{Status} & \textbf{Test Cases}\\ \hline
\href{https://jira.lsstcorp.org/browse/LVV-9796}{LVV-9796} &
Leanne Guy & Not Covered &
\begin{tabular}{c}
\end{tabular}
\\
\hline
\end{longtable}

\textbf{Verification Element Description:} \\
The PVIs must be available as files on the files system
for~\textbf{l1CacheLifetime = 30~days.}

Associated element DMS-REQ-0376-V-01
(\href{https://jira.lsstcorp.org/browse/LVV-3396}{LVV-3396})~satisfies
the maximum retrieval time.

The associated element DMS-REQ-0376-V-02
(\href{https://jira.lsstcorp.org/browse/LVV-9795}{LVV-9795}) satisfies
the additional constraint on the number of simultaneous users.

\emph{These requirements should be satisfied both separately and
together.}

{\footnotesize
\begin{longtable}{p{2.5cm}p{13.5cm}}
\hline
\multicolumn{2}{c}{\textbf{Requirement Details}}\\ \hline
Requirement ID & DMS-REQ-0376 \\ \cdashline{1-2}
Requirement Description &
\begin{minipage}[]{13cm}
\textbf{Specification:} All Processed Visit Images for a single visit
that are available in cache, including mask and variance planes, shall
be identifiable with a single IVOA SIAv2 service query and retrievable,
using the link(s) provided in the response, within
\textbf{allPviRetrievalTime.} This requirement shall be met for up to
\textbf{allPviRetrievalUsers} simultaneous requests for distinct
focal-plane PVI sets.
\end{minipage}
\\ \cdashline{1-2}
Requirement Parameters & {[}\textbf{allPviRetrievalUsers = 10{{[}integer{]}}} Minimum number of
simultaneous users retrieving all PVI images for a visit.,
\textbf{allPviRetrievalTime = 60{{[}second{]}}} Maximum time allowed for
retrieving all PVI images of a single visit., \textbf{l1CacheLifetime =
30{{[}day{]}}} Lifetime in the cache of un-archived Level-1 data
products.{]} \\ \cdashline{1-2}
Requirement Discussion &
\begin{minipage}[]{13cm}
\textbf{Discussion:} The performance targets for this requirement assume
the PVIs are available as files on a file system. For example, this
could be those files present in the \textbf{l1CacheLifetime} cache.
\end{minipage}
\\ \cdashline{1-2}
Requirement Priority & 1b \\ \cdashline{1-2}
Upper Level Requirement &
\begin{tabular}{cl}
OSS-REQ-0181 & Data Products Query and Download Infrastructure \\
\end{tabular}
\\ \hline
\end{longtable}
}


  
 \newpage 
\subsection{[LVV-9797] DMS-REQ-0377-V-02: Max time to retrieve single-CCD coadd cutout image }\label{lvv-9797}

\begin{longtable}{cccc}
\hline
\textbf{Jira Link} & \textbf{Assignee} & \textbf{Status} & \textbf{Test Cases}\\ \hline
\href{https://jira.lsstcorp.org/browse/LVV-9797}{LVV-9797} &
Leanne Guy & Not Covered &
\begin{tabular}{c}
LVV-T1332 \\
\end{tabular}
\\
\hline
\end{longtable}

\textbf{Verification Element Description:} \\
Maximum time allowed for retrieving a CCD-sized coadd cutout using the
IVOA SODA protocol must be~\textbf{ccdRetrievalTime = 15~seconds}.~

The associated element
DMS-REQ-0377-V-01~(\href{https://jira.lsstcorp.org/browse/LVV-3394}{LVV-3394})~satisfies
the additional simultaneous users constraint.

\emph{These requirements should be satisfied together.}

{\footnotesize
\begin{longtable}{p{2.5cm}p{13.5cm}}
\hline
\multicolumn{2}{c}{\textbf{Requirement Details}}\\ \hline
Requirement ID & DMS-REQ-0377 \\ \cdashline{1-2}
Requirement Description &
\begin{minipage}[]{13cm}
\textbf{Specification:} A CCD-sized cutout of a coadd, including mask
and variance planes, shall be retrievable using the IVOA SODA protocol
within \textbf{ccdRetrievalTime} with \textbf{ccdRetrievalUsers}
simultaneous requests for distinct areas of the sky.
\end{minipage}
\\ \cdashline{1-2}
Requirement Parameters & {[}\textbf{ccdRetrievalTime = 15{{[}second{]}}} Maximum time allowed for
retrieving a CCD-sized coadd cutout., \textbf{ccdRetrievalUsers =
20{{[}integer{]}}} Minimum number of simultaneous users retrieving a
single CCD-sized coadd cutout.{]} \\ \cdashline{1-2}
Requirement Priority & 1b \\ \cdashline{1-2}
Upper Level Requirement &
\begin{tabular}{cl}
OSS-REQ-0181 & Data Products Query and Download Infrastructure \\
\end{tabular}
\\ \hline
\end{longtable}
}


\subsubsection{Test Cases Summary}
\begin{longtable}{p{3cm}p{2.5cm}p{2.5cm}p{3cm}p{4cm}}
\toprule
\href{https://jira.lsstcorp.org/secure/Tests.jspa\#/testCase/LVV-T1332}{LVV-T1332} & \multicolumn{4}{p{12cm}}{ Verify implementation of maximum time for retrieval of CCD-sized coadd
cutouts } \\ \hline
\textbf{Owner} & \textbf{Status} & \textbf{Version} & \textbf{Critical Event} & \textbf{Verification Type} \\ \hline
Leanne Guy & Defined & 1 & false & Test \\ \hline
\end{longtable}
{\scriptsize
\textbf{Objective:}\\
Verify that at least \textbf{ccdRetrievalUsers = 20~}users can retrieve
CCD-sized coadd cutouts using the IVOA SODA protocol within a maximum
retrieval time of~\textbf{ccdRetrievalTime = 15 seconds}.
}
  
 \newpage 
\subsection{[LVV-9803] DMS-REQ-0004-V-03: Time to availability of Solar System Object orbits }\label{lvv-9803}

\begin{longtable}{cccc}
\hline
\textbf{Jira Link} & \textbf{Assignee} & \textbf{Status} & \textbf{Test Cases}\\ \hline
\href{https://jira.lsstcorp.org/browse/LVV-9803}{LVV-9803} &
Leanne Guy & Not Covered &
\begin{tabular}{c}
LVV-T102 \\
\end{tabular}
\\
\hline
\end{longtable}

\textbf{Verification Element Description:} \\
Verify that Solar System Object orbits are calculated and made available
within \textbf{L1PublicT = 24 hours.}

Associated element
(\href{https://jira.lsstcorp.org/browse/LVV-175}{LVV-175}) satisfies the
maximum time allotted for public release of L1 Data Products.

Associated element
(\href{https://jira.lsstcorp.org/browse/LVV-9740}{LVV-9740}) satisfies
the latency of reporting transients.

{\footnotesize
\begin{longtable}{p{2.5cm}p{13.5cm}}
\hline
\multicolumn{2}{c}{\textbf{Requirement Details}}\\ \hline
Requirement ID & DMS-REQ-0004 \\ \cdashline{1-2}
Requirement Description &
\begin{minipage}[]{13cm}
\textbf{Specification:} With the exception of alerts and Solar System
Objects, all Level 1 Data Products shall be made public within time
\textbf{L1PublicT} of the acquisition of the raw image data.\\
\hspace*{0.333em}\\
LSST shall not release image or catalog data resulting from a visit,
except for the content of the public alert stream, sooner than time
\textbf{L1PublicTMin} following the acquisition of the raw image data
from that visit.\\
\hspace*{0.333em}\\
For visits resulting in fewer than \textbf{nAlertVisitPeak}, LSST shall
be capable of supporting the distribution of at least \textbf{OTR1} per
cent of alerts via the LSST alert distribution system within time
\textbf{OTT1} from the conclusion of the camera's readout of the raw
exposures used to generate each alert. ~\\
\hspace*{0.333em}\\
Solar System Object orbits will, on average, be calculated before the
following night's observing finishes and the results shall be made
available within time \textbf{L1PublicT} of those calculations being
completed.
\end{minipage}
\\ \cdashline{1-2}
Requirement Parameters & {[}\textbf{OTR1 = 98{{[}percent{]}}} Fraction of detectable alerts for
which an alert is actually transmitted within latency OTT1 (see
LSR-REQ-0101)., \textbf{OTT1 = 1{{[}minute{]}}} The latency of reporting
optical transients following the completion of readout of the last image
of a visit, \textbf{nAlertVisitPeak = 40000{{[}integer{]}}} The
instantaneous peak number of alerts per standard visit.,
\textbf{L1PublicTMin = 6{{[}hour{]}}} Time images and other products
(except alerts) will be embargoed before release to the consortium (or
the public), \textbf{L1PublicT = 24{{[}hour{]}}} Maximum time from the
acquisition of science data to the release of associated Level 1 Data
Products (except alerts){]} \\ \cdashline{1-2}
Requirement Discussion &
\begin{minipage}[]{13cm}
\textbf{Discussion:} Because of the processing flow of SSObject orbit
determination, meeting the base
\textbf{L1PublicT}-after-data-acquisition requirement would be far more
challenging than for the other L1 Data Products, but the system
throughput has to be good enough such that a back log can not build up.
\end{minipage}
\\ \cdashline{1-2}
Requirement Priority & 1b \\ \cdashline{1-2}
Upper Level Requirement &
\begin{tabular}{cl}
DMS-REQ-0003 & Create and Maintain Science Data Archive \\
OSS-REQ-0127 & Level 1 Data Product Availability \\
\end{tabular}
\\ \hline
\end{longtable}
}


\subsubsection{Test Cases Summary}
\begin{longtable}{p{3cm}p{2.5cm}p{2.5cm}p{3cm}p{4cm}}
\toprule
\href{https://jira.lsstcorp.org/secure/Tests.jspa\#/testCase/LVV-T102}{LVV-T102} & \multicolumn{4}{p{12cm}}{ Verify implementation of Solar System Objects Available Within Specified
Time } \\ \hline
\textbf{Owner} & \textbf{Status} & \textbf{Version} & \textbf{Critical Event} & \textbf{Verification Type} \\ \hline
Kian-Tat Lim & Draft & 1 & false & Test \\ \hline
\end{longtable}
{\scriptsize
\textbf{Objective:}\\
Execute single-day operations rehearsal, observe data products generated
in time
}
  
 \newpage 
\subsection{[LVV-9806] DMS-LSP-REQ-0007-V-01: Abide by the Data Access Policies\_1 }\label{lvv-9806}

\begin{longtable}{cccc}
\hline
\textbf{Jira Link} & \textbf{Assignee} & \textbf{Status} & \textbf{Test Cases}\\ \hline
\href{https://jira.lsstcorp.org/browse/LVV-9806}{LVV-9806} &
Gregory Dubois-Felsmann & Not Covered &
\begin{tabular}{c}
LVV-T605 \\
\end{tabular}
\\
\hline
\end{longtable}

\textbf{Verification Element Description:} \\
Undefined

{\footnotesize
\begin{longtable}{p{2.5cm}p{13.5cm}}
\hline
\multicolumn{2}{c}{\textbf{Requirement Details}}\\ \hline
Requirement ID & DMS-LSP-REQ-0007 \\ \cdashline{1-2}
Requirement Description &
\begin{minipage}[]{13cm}
The LSP shall comply with the public data access policy and access
restrictions defined by the LSST Project and operations organization.
\end{minipage}
\\ \cdashline{1-2}
Requirement Discussion &
\begin{minipage}[]{13cm}
This includes both the enforcement of Project-level data rights (e.g.,
to the Level 1 and Level 2 data) and user-provided access controls to
user-created data (e.g., Level 3 data products).
\end{minipage}
\\ \cdashline{1-2}
Requirement Priority &  \\ \cdashline{1-2}
Upper Level Requirement &
\begin{tabular}{cl}
\end{tabular}
\\ \hline
\end{longtable}
}


\subsubsection{Test Cases Summary}
\begin{longtable}{p{3cm}p{2.5cm}p{2.5cm}p{3cm}p{4cm}}
\toprule
\href{https://jira.lsstcorp.org/secure/Tests.jspa\#/testCase/LVV-T605}{LVV-T605} & \multicolumn{4}{p{12cm}}{ Verify that LSP complies with LSST data access policies } \\ \hline
\textbf{Owner} & \textbf{Status} & \textbf{Version} & \textbf{Critical Event} & \textbf{Verification Type} \\ \hline
Jeffrey Carlin & Draft & 1 & false & Test \\ \hline
\end{longtable}
{\scriptsize
\textbf{Objective:}\\
Verify that the LSP complies with the public data access policy and
access restrictions defined by the LSST Project.
}
  
 \newpage 
\subsection{[LVV-9807] DMS-LSP-REQ-0001-V-01: Access to All Released or Authorized Data
Products\_1 }\label{lvv-9807}

\begin{longtable}{cccc}
\hline
\textbf{Jira Link} & \textbf{Assignee} & \textbf{Status} & \textbf{Test Cases}\\ \hline
\href{https://jira.lsstcorp.org/browse/LVV-9807}{LVV-9807} &
Gregory Dubois-Felsmann & Not Covered &
\begin{tabular}{c}
LVV-T2 \\
LVV-T598 \\
\end{tabular}
\\
\hline
\end{longtable}

\textbf{Verification Element Description:} \\
Undefined

{\footnotesize
\begin{longtable}{p{2.5cm}p{13.5cm}}
\hline
\multicolumn{2}{c}{\textbf{Requirement Details}}\\ \hline
Requirement ID & DMS-LSP-REQ-0001 \\ \cdashline{1-2}
Requirement Description &
\begin{minipage}[]{13cm}
The LSP shall provide the capability to access all the Project's
released data products, including, but not limited to, the data products
enumerated in the DPDD (\citeds{LSE-163}), as well as all user data products to
which a user has access.
\end{minipage}
\\ \cdashline{1-2}
Requirement Priority &  \\ \cdashline{1-2}
Upper Level Requirement &
\begin{tabular}{cl}
\end{tabular}
\\ \hline
\end{longtable}
}


\subsubsection{Test Cases Summary}
\begin{longtable}{p{3cm}p{2.5cm}p{2.5cm}p{3cm}p{4cm}}
\toprule
\href{https://jira.lsstcorp.org/secure/Tests.jspa\#/testCase/LVV-T2}{LVV-T2} & \multicolumn{4}{p{12cm}}{ LSP-00-00: Verification of the presence of the expected WISE data } \\ \hline
\textbf{Owner} & \textbf{Status} & \textbf{Version} & \textbf{Critical Event} & \textbf{Verification Type} \\ \hline
Gregory Dubois-Felsmann & Deprecated & 1 & false & Test \\ \hline
\end{longtable}
{\scriptsize
\textbf{Objective:}\\
This test will check:

\begin{itemize}
\tightlist
\item
  That the expected tables are present in the database and accessible
  via the API Aspect and the Portal Aspect;
\item
  That the tables are present with the expected schema as documented in
  the IPAC- provided WISE documentation;
\item
  That the row counts in the tables are as expected;
\item
  That the tables cover essentially the entire sky, as expected from the
  characteristics of the WISE mission.
\end{itemize}

\textbf{Requirements (to be removed when Reqs are synchronized from
magic draw)}

\begin{itemize}
\tightlist
\item
  DMS-LSP-REQ-0001
\item
  DMS-LSP-REQ-0005
\end{itemize}
}
\begin{longtable}{p{3cm}p{2.5cm}p{2.5cm}p{3cm}p{4cm}}
\toprule
\href{https://jira.lsstcorp.org/secure/Tests.jspa\#/testCase/LVV-T598}{LVV-T598} & \multicolumn{4}{p{12cm}}{ Verify access to All Released or Authorized Data Products } \\ \hline
\textbf{Owner} & \textbf{Status} & \textbf{Version} & \textbf{Critical Event} & \textbf{Verification Type} \\ \hline
Jeffrey Carlin & Draft & 1 & false & Inspection \\ \hline
\end{longtable}
{\scriptsize
\textbf{Objective:}\\
Verify that the LSP can access all data products defined in the DPDD,
and additional data products.
}
  
 \newpage 
\subsection{[LVV-9808] DMS-LSP-REQ-0004-V-01: API (Data Access) Aspect\_1 }\label{lvv-9808}

\begin{longtable}{cccc}
\hline
\textbf{Jira Link} & \textbf{Assignee} & \textbf{Status} & \textbf{Test Cases}\\ \hline
\href{https://jira.lsstcorp.org/browse/LVV-9808}{LVV-9808} &
Gregory Dubois-Felsmann & Not Covered &
\begin{tabular}{c}
LVV-T3 \\
LVV-T602 \\
LVV-T1437 \\
\end{tabular}
\\
\hline
\end{longtable}

\textbf{Verification Element Description:} \\
Undefined

{\footnotesize
\begin{longtable}{p{2.5cm}p{13.5cm}}
\hline
\multicolumn{2}{c}{\textbf{Requirement Details}}\\ \hline
Requirement ID & DMS-LSP-REQ-0004 \\ \cdashline{1-2}
Requirement Description &
\begin{minipage}[]{13cm}
The LSP shall provide a Web API for access to all the LSST data products
and the user storage resources.
\end{minipage}
\\ \cdashline{1-2}
Requirement Discussion &
\begin{minipage}[]{13cm}
The Web API will use VO standards as much as is practical and supported
by community expectations; see DMS-LSP-REQ-0006 below. It may also
include additional APIs to provide services unique to LSST.
\end{minipage}
\\ \cdashline{1-2}
Requirement Priority &  \\ \cdashline{1-2}
Upper Level Requirement &
\begin{tabular}{cl}
\end{tabular}
\\ \hline
\end{longtable}
}


\subsubsection{Test Cases Summary}
\begin{longtable}{p{3cm}p{2.5cm}p{2.5cm}p{3cm}p{4cm}}
\toprule
\href{https://jira.lsstcorp.org/secure/Tests.jspa\#/testCase/LVV-T3}{LVV-T3} & \multicolumn{4}{p{12cm}}{ LSP-00-05: Demonstration of low-volume and/or indexed queries against
the WISE data via API } \\ \hline
\textbf{Owner} & \textbf{Status} & \textbf{Version} & \textbf{Critical Event} & \textbf{Verification Type} \\ \hline
Gregory Dubois-Felsmann & Deprecated & 1 & false & Test \\ \hline
\end{longtable}
{\scriptsize
\textbf{Objective:}\\
This test will check that the following low-volume queries can be
performed against the WISE catalogs via the API Aspect.

\begin{itemize}
\tightlist
\item
  Small cone searches against the Object-like, ForcedSource-like, and
  Source-like tables; and
\item
  Searches by exact ID matching against the Object-like,
  ForcedSource-like, and Source- like tables
\end{itemize}

~\\
The tests will record their performance for comparison against similar
queries in the produc- tion WISE archive at IRSA, and the returned data
will be compared to that for similar queries against the API services
provided by IRSA.\\[2\baselineskip]\textbf{Requirement (to remove once
requirements are synchronized from magic draw)}\\
DMS-LSP-REQ-004
}
\begin{longtable}{p{3cm}p{2.5cm}p{2.5cm}p{3cm}p{4cm}}
\toprule
\href{https://jira.lsstcorp.org/secure/Tests.jspa\#/testCase/LVV-T602}{LVV-T602} & \multicolumn{4}{p{12cm}}{ Verify LSP provides web API } \\ \hline
\textbf{Owner} & \textbf{Status} & \textbf{Version} & \textbf{Critical Event} & \textbf{Verification Type} \\ \hline
Michael Wood-Vasey & Draft & 1 & false & Test \\ \hline
\end{longtable}
{\scriptsize
\textbf{Objective:}\\
Verify that the LSP provides a web API for access to LSST data products
and user storage resources.~
}
\begin{longtable}{p{3cm}p{2.5cm}p{2.5cm}p{3cm}p{4cm}}
\toprule
\href{https://jira.lsstcorp.org/secure/Tests.jspa\#/testCase/LVV-T1437}{LVV-T1437} & \multicolumn{4}{p{12cm}}{ LDM-503-10a: API Aspect tests for LSP with Authentication and TAP
milestone } \\ \hline
\textbf{Owner} & \textbf{Status} & \textbf{Version} & \textbf{Critical Event} & \textbf{Verification Type} \\ \hline
Gregory Dubois-Felsmann & Defined & 1 & false & Test \\ \hline
\end{longtable}
{\scriptsize
\textbf{Objective:}\\
This test case verifies that the TAP service in the API Aspect of the
Science Platform is accessible to authorized users through a login
process, and that TAP searches can be performed using the IVOA TAP
protocol from remote sites.\\[2\baselineskip]In so doing and in
conjunction with the other LDM-503-10a test cases collected under
LVV-P48, it addresses all or part of the following requirements:

\begin{itemize}
\tightlist
\item
  DMS-LSP-REQ-0004, DMS-LSP-REQ-0005, DMS-LSP-REQ-0006,
  DMS-LSP-REQ-0020, DMS-LSP-REQ-0022, DMS-LSP-REQ-0023, DMS-LSP-REQ-0024
\item
  DMS-API-REQ-0003, DMS-API-REQ-0004, DMS-API-REQ-0006,
  DMS-API-REQ-0007, DMS-API-REQ-0008, DMS-API-REQ-0009,
  DMS-API-REQ-0023, and DMS-API-REQ-0039, primarily
\end{itemize}

Note this test was not designed to perform a full verification of the
above requirements, but rather to demonstrate having reached a certain
level of partial capability during construction.
}
  
 \newpage 
\subsection{[LVV-9809] DMS-LSP-REQ-0005-V-01: Linkage of Aspects\_1 }\label{lvv-9809}

\begin{longtable}{cccc}
\hline
\textbf{Jira Link} & \textbf{Assignee} & \textbf{Status} & \textbf{Test Cases}\\ \hline
\href{https://jira.lsstcorp.org/browse/LVV-9809}{LVV-9809} &
Gregory Dubois-Felsmann & Not Covered &
\begin{tabular}{c}
LVV-T2 \\
LVV-T603 \\
LVV-T1334 \\
LVV-T1436 \\
LVV-T1437 \\
\end{tabular}
\\
\hline
\end{longtable}

\textbf{Verification Element Description:} \\
Undefined

{\footnotesize
\begin{longtable}{p{2.5cm}p{13.5cm}}
\hline
\multicolumn{2}{c}{\textbf{Requirement Details}}\\ \hline
Requirement ID & DMS-LSP-REQ-0005 \\ \cdashline{1-2}
Requirement Description &
\begin{minipage}[]{13cm}
The LSP shall facilitate access to the same LSST and user data through
multiple aspects.
\end{minipage}
\\ \cdashline{1-2}
Requirement Discussion &
\begin{minipage}[]{13cm}
It should be possible to identify or create data in one aspect and
retrieve it in another. This requirement is satisfied principally by
building the various aspects over the same underlying data services.
\end{minipage}
\\ \cdashline{1-2}
Requirement Priority &  \\ \cdashline{1-2}
Upper Level Requirement &
\begin{tabular}{cl}
\end{tabular}
\\ \hline
\end{longtable}
}


\subsubsection{Test Cases Summary}
\begin{longtable}{p{3cm}p{2.5cm}p{2.5cm}p{3cm}p{4cm}}
\toprule
\href{https://jira.lsstcorp.org/secure/Tests.jspa\#/testCase/LVV-T2}{LVV-T2} & \multicolumn{4}{p{12cm}}{ LSP-00-00: Verification of the presence of the expected WISE data } \\ \hline
\textbf{Owner} & \textbf{Status} & \textbf{Version} & \textbf{Critical Event} & \textbf{Verification Type} \\ \hline
Gregory Dubois-Felsmann & Deprecated & 1 & false & Test \\ \hline
\end{longtable}
{\scriptsize
\textbf{Objective:}\\
This test will check:

\begin{itemize}
\tightlist
\item
  That the expected tables are present in the database and accessible
  via the API Aspect and the Portal Aspect;
\item
  That the tables are present with the expected schema as documented in
  the IPAC- provided WISE documentation;
\item
  That the row counts in the tables are as expected;
\item
  That the tables cover essentially the entire sky, as expected from the
  characteristics of the WISE mission.
\end{itemize}

\textbf{Requirements (to be removed when Reqs are synchronized from
magic draw)}

\begin{itemize}
\tightlist
\item
  DMS-LSP-REQ-0001
\item
  DMS-LSP-REQ-0005
\end{itemize}
}
\begin{longtable}{p{3cm}p{2.5cm}p{2.5cm}p{3cm}p{4cm}}
\toprule
\href{https://jira.lsstcorp.org/secure/Tests.jspa\#/testCase/LVV-T603}{LVV-T603} & \multicolumn{4}{p{12cm}}{ Verify data access through multiple linked aspects } \\ \hline
\textbf{Owner} & \textbf{Status} & \textbf{Version} & \textbf{Critical Event} & \textbf{Verification Type} \\ \hline
Jeffrey Carlin & Draft & 1 & false & Inspection \\ \hline
\end{longtable}
{\scriptsize
\textbf{Objective:}\\
Verify that the LSP facilitates access of the same LSST or user data
through multiple aspects.
}
\begin{longtable}{p{3cm}p{2.5cm}p{2.5cm}p{3cm}p{4cm}}
\toprule
\href{https://jira.lsstcorp.org/secure/Tests.jspa\#/testCase/LVV-T1334}{LVV-T1334} & \multicolumn{4}{p{12cm}}{ LDM-503-10a: Portal Aspect tests for LSP with Authentication and TAP
milestone } \\ \hline
\textbf{Owner} & \textbf{Status} & \textbf{Version} & \textbf{Critical Event} & \textbf{Verification Type} \\ \hline
Gregory Dubois-Felsmann & Defined & 1 & false & Test \\ \hline
\end{longtable}
{\scriptsize
\textbf{Objective:}\\
This test case verifies that the Portal Aspect of the Science Platform
is accessible to authorized users through a login process, and that TAP
searches can be performed from the Portal Aspect UI.\\[2\baselineskip]In
so doing and in conjunction with the other LDM-503-10a test cases
collected under LVV-P48, it addresses all or part of the following
requirements:

\begin{itemize}
\tightlist
\item
  DMS-LSP-REQ-0002, DMS-LSP-REQ-0005, DMS-LSP-REQ-0006,
  DMS-LSP-REQ-0020, DMS-LSP-REQ-0022, DMS-LSP-REQ-0023, DMS-LSP-REQ-0024
\item
  DMS-PRTL-REQ-0001, DMS-PRTL-REQ-0015, DMS-PRTL-REQ-0016,
  DMS-PRTL-REQ-0017, DMS-PRTL-REQ-0020, DMS-PRTL-REQ-0026,
  DMS-PRTL-REQ-0049, and DMS-PRTL-REQ-0095, primarily
\end{itemize}

Note this test was not designed to perform a full verification of the
above requirements, but rather to demonstrate having reached a certain
level of partial capability during construction.
}
\begin{longtable}{p{3cm}p{2.5cm}p{2.5cm}p{3cm}p{4cm}}
\toprule
\href{https://jira.lsstcorp.org/secure/Tests.jspa\#/testCase/LVV-T1436}{LVV-T1436} & \multicolumn{4}{p{12cm}}{ LDM-503-10a: Notebook Aspect tests for LSP with Authentication and TAP
milestone } \\ \hline
\textbf{Owner} & \textbf{Status} & \textbf{Version} & \textbf{Critical Event} & \textbf{Verification Type} \\ \hline
Gregory Dubois-Felsmann & Defined & 1 & false & Test \\ \hline
\end{longtable}
{\scriptsize
\textbf{Objective:}\\
This test case verifies that the Notebook Aspect of the Science Platform
is accessible to authorized users through a login process, and that TAP
searches can be performed from Python code in the Notebook
Aspect.\\[2\baselineskip]In so doing and in conjunction with the other
LDM-503-10a test cases collected under LVV-P48, it addresses all or part
of the following requirements:

\begin{itemize}
\tightlist
\item
  DMS-LSP-REQ-0003, DMS-LSP-REQ-0005, DMS-LSP-REQ-0006,
  DMS-LSP-REQ-0020, DMS-LSP-REQ-0022, DMS-LSP-REQ-0023, DMS-LSP-REQ-0024
\item
  DMS-NB-REQ-0001, DMS-NB-REQ-0002, DMS-NB-REQ-0005, DMS-NB-REQ-0006,
  DMS-NB-REQ-0013, DMS-NB-REQ-0017, and DMS-NB-REQ-0029, primarily
\end{itemize}

Note this test was not designed to perform a full verification of the
above requirements, but rather to demonstrate having reached a certain
level of partial capability during construction.
}
\begin{longtable}{p{3cm}p{2.5cm}p{2.5cm}p{3cm}p{4cm}}
\toprule
\href{https://jira.lsstcorp.org/secure/Tests.jspa\#/testCase/LVV-T1437}{LVV-T1437} & \multicolumn{4}{p{12cm}}{ LDM-503-10a: API Aspect tests for LSP with Authentication and TAP
milestone } \\ \hline
\textbf{Owner} & \textbf{Status} & \textbf{Version} & \textbf{Critical Event} & \textbf{Verification Type} \\ \hline
Gregory Dubois-Felsmann & Defined & 1 & false & Test \\ \hline
\end{longtable}
{\scriptsize
\textbf{Objective:}\\
This test case verifies that the TAP service in the API Aspect of the
Science Platform is accessible to authorized users through a login
process, and that TAP searches can be performed using the IVOA TAP
protocol from remote sites.\\[2\baselineskip]In so doing and in
conjunction with the other LDM-503-10a test cases collected under
LVV-P48, it addresses all or part of the following requirements:

\begin{itemize}
\tightlist
\item
  DMS-LSP-REQ-0004, DMS-LSP-REQ-0005, DMS-LSP-REQ-0006,
  DMS-LSP-REQ-0020, DMS-LSP-REQ-0022, DMS-LSP-REQ-0023, DMS-LSP-REQ-0024
\item
  DMS-API-REQ-0003, DMS-API-REQ-0004, DMS-API-REQ-0006,
  DMS-API-REQ-0007, DMS-API-REQ-0008, DMS-API-REQ-0009,
  DMS-API-REQ-0023, and DMS-API-REQ-0039, primarily
\end{itemize}

Note this test was not designed to perform a full verification of the
above requirements, but rather to demonstrate having reached a certain
level of partial capability during construction.
}
  
 \newpage 
\subsection{[LVV-9810] DMS-LSP-REQ-0003-V-01: Notebook Aspect\_1 }\label{lvv-9810}

\begin{longtable}{cccc}
\hline
\textbf{Jira Link} & \textbf{Assignee} & \textbf{Status} & \textbf{Test Cases}\\ \hline
\href{https://jira.lsstcorp.org/browse/LVV-9810}{LVV-9810} &
Gregory Dubois-Felsmann & Not Covered &
\begin{tabular}{c}
LVV-T601 \\
LVV-T1436 \\
\end{tabular}
\\
\hline
\end{longtable}

\textbf{Verification Element Description:} \\
Undefined

{\footnotesize
\begin{longtable}{p{2.5cm}p{13.5cm}}
\hline
\multicolumn{2}{c}{\textbf{Requirement Details}}\\ \hline
Requirement ID & DMS-LSP-REQ-0003 \\ \cdashline{1-2}
Requirement Description &
\begin{minipage}[]{13cm}
The LSP shall provide an interactive Python computing environment,
accessible through a Web browser, with access to all the LSST data
products and to user computing and storage resources.
\end{minipage}
\\ \cdashline{1-2}
Requirement Discussion &
\begin{minipage}[]{13cm}
The term ``Notebook aspect'' (and, a fortiori, "Jupyter{{[}Lab{]}}
aspect", which has also been used in DM discussions) refers to the
reference implementation of this aspect of the LSP in terms of Jupyter
notebooks and the forthcoming JupyterLab successor to that technology,
with an IPython kernel back end. For the purposes of this requirements
document, however, it seems preferable to avoid using an external brand
name in the normative text.
\end{minipage}
\\ \cdashline{1-2}
Requirement Priority &  \\ \cdashline{1-2}
Upper Level Requirement &
\begin{tabular}{cl}
\end{tabular}
\\ \hline
\end{longtable}
}


\subsubsection{Test Cases Summary}
\begin{longtable}{p{3cm}p{2.5cm}p{2.5cm}p{3cm}p{4cm}}
\toprule
\href{https://jira.lsstcorp.org/secure/Tests.jspa\#/testCase/LVV-T601}{LVV-T601} & \multicolumn{4}{p{12cm}}{ Verify LSP provides a notebook aspect } \\ \hline
\textbf{Owner} & \textbf{Status} & \textbf{Version} & \textbf{Critical Event} & \textbf{Verification Type} \\ \hline
Michael Wood-Vasey & Draft & 1 & false & Inspection \\ \hline
\end{longtable}
{\scriptsize
\textbf{Objective:}\\
Verify that the LSP provides an interactive Python computing
environment, accessible via web browser, with access to LSST data
products and user storage resources.
}
\begin{longtable}{p{3cm}p{2.5cm}p{2.5cm}p{3cm}p{4cm}}
\toprule
\href{https://jira.lsstcorp.org/secure/Tests.jspa\#/testCase/LVV-T1436}{LVV-T1436} & \multicolumn{4}{p{12cm}}{ LDM-503-10a: Notebook Aspect tests for LSP with Authentication and TAP
milestone } \\ \hline
\textbf{Owner} & \textbf{Status} & \textbf{Version} & \textbf{Critical Event} & \textbf{Verification Type} \\ \hline
Gregory Dubois-Felsmann & Defined & 1 & false & Test \\ \hline
\end{longtable}
{\scriptsize
\textbf{Objective:}\\
This test case verifies that the Notebook Aspect of the Science Platform
is accessible to authorized users through a login process, and that TAP
searches can be performed from Python code in the Notebook
Aspect.\\[2\baselineskip]In so doing and in conjunction with the other
LDM-503-10a test cases collected under LVV-P48, it addresses all or part
of the following requirements:

\begin{itemize}
\tightlist
\item
  DMS-LSP-REQ-0003, DMS-LSP-REQ-0005, DMS-LSP-REQ-0006,
  DMS-LSP-REQ-0020, DMS-LSP-REQ-0022, DMS-LSP-REQ-0023, DMS-LSP-REQ-0024
\item
  DMS-NB-REQ-0001, DMS-NB-REQ-0002, DMS-NB-REQ-0005, DMS-NB-REQ-0006,
  DMS-NB-REQ-0013, DMS-NB-REQ-0017, and DMS-NB-REQ-0029, primarily
\end{itemize}

Note this test was not designed to perform a full verification of the
above requirements, but rather to demonstrate having reached a certain
level of partial capability during construction.
}
  
 \newpage 
\subsection{[LVV-9811] DMS-LSP-REQ-0002-V-01: Portal Aspect\_1 }\label{lvv-9811}

\begin{longtable}{cccc}
\hline
\textbf{Jira Link} & \textbf{Assignee} & \textbf{Status} & \textbf{Test Cases}\\ \hline
\href{https://jira.lsstcorp.org/browse/LVV-9811}{LVV-9811} &
Gregory Dubois-Felsmann & Not Covered &
\begin{tabular}{c}
LVV-T5 \\
LVV-T600 \\
LVV-T1334 \\
\end{tabular}
\\
\hline
\end{longtable}

\textbf{Verification Element Description:} \\
Undefined

{\footnotesize
\begin{longtable}{p{2.5cm}p{13.5cm}}
\hline
\multicolumn{2}{c}{\textbf{Requirement Details}}\\ \hline
Requirement ID & DMS-LSP-REQ-0002 \\ \cdashline{1-2}
Requirement Description &
\begin{minipage}[]{13cm}
The LSP shall provide a Web-based ``Portal'' means of access to all the
LSST data products, and to user storage resources.
\end{minipage}
\\ \cdashline{1-2}
Requirement Discussion &
\begin{minipage}[]{13cm}
The Portal is defined by further requirements.
\end{minipage}
\\ \cdashline{1-2}
Requirement Priority &  \\ \cdashline{1-2}
Upper Level Requirement &
\begin{tabular}{cl}
\end{tabular}
\\ \hline
\end{longtable}
}


\subsubsection{Test Cases Summary}
\begin{longtable}{p{3cm}p{2.5cm}p{2.5cm}p{3cm}p{4cm}}
\toprule
\href{https://jira.lsstcorp.org/secure/Tests.jspa\#/testCase/LVV-T5}{LVV-T5} & \multicolumn{4}{p{12cm}}{ LSP-00-15: Execution of basic catalog queries in the Portal } \\ \hline
\textbf{Owner} & \textbf{Status} & \textbf{Version} & \textbf{Critical Event} & \textbf{Verification Type} \\ \hline
Gregory Dubois-Felsmann & Deprecated & 1 & false & Test \\ \hline
\end{longtable}
{\scriptsize
\textbf{Objective:}\\
This test will test the functional requirements to be able to perform a
range of basic queries through the Portal Aspect of the LSP:

\begin{itemize}
\tightlist
\item
  Cone searches on the Object-like, ForcedSource-like, and Source-like
  WISE tables;~
\item
  Multi-target cone searches;
\item
  Form-based searches for exact equality, e.g., for row IDs; and
\item
  Form-based searches for sets of object attributes.
\end{itemize}

In addition, it tests the ability to download tabular query results from
the Portal Aspect.
}
\begin{longtable}{p{3cm}p{2.5cm}p{2.5cm}p{3cm}p{4cm}}
\toprule
\href{https://jira.lsstcorp.org/secure/Tests.jspa\#/testCase/LVV-T600}{LVV-T600} & \multicolumn{4}{p{12cm}}{ Verify LSP provides a portal aspect } \\ \hline
\textbf{Owner} & \textbf{Status} & \textbf{Version} & \textbf{Critical Event} & \textbf{Verification Type} \\ \hline
Michael Wood-Vasey & Draft & 1 & false & Inspection \\ \hline
\end{longtable}
{\scriptsize
\textbf{Objective:}\\
Verify that the LSP provides a web-based ``Portal'' to access LSST data
products and user storage resources.

The Portal is defined by further requirements.
}
\begin{longtable}{p{3cm}p{2.5cm}p{2.5cm}p{3cm}p{4cm}}
\toprule
\href{https://jira.lsstcorp.org/secure/Tests.jspa\#/testCase/LVV-T1334}{LVV-T1334} & \multicolumn{4}{p{12cm}}{ LDM-503-10a: Portal Aspect tests for LSP with Authentication and TAP
milestone } \\ \hline
\textbf{Owner} & \textbf{Status} & \textbf{Version} & \textbf{Critical Event} & \textbf{Verification Type} \\ \hline
Gregory Dubois-Felsmann & Defined & 1 & false & Test \\ \hline
\end{longtable}
{\scriptsize
\textbf{Objective:}\\
This test case verifies that the Portal Aspect of the Science Platform
is accessible to authorized users through a login process, and that TAP
searches can be performed from the Portal Aspect UI.\\[2\baselineskip]In
so doing and in conjunction with the other LDM-503-10a test cases
collected under LVV-P48, it addresses all or part of the following
requirements:

\begin{itemize}
\tightlist
\item
  DMS-LSP-REQ-0002, DMS-LSP-REQ-0005, DMS-LSP-REQ-0006,
  DMS-LSP-REQ-0020, DMS-LSP-REQ-0022, DMS-LSP-REQ-0023, DMS-LSP-REQ-0024
\item
  DMS-PRTL-REQ-0001, DMS-PRTL-REQ-0015, DMS-PRTL-REQ-0016,
  DMS-PRTL-REQ-0017, DMS-PRTL-REQ-0020, DMS-PRTL-REQ-0026,
  DMS-PRTL-REQ-0049, and DMS-PRTL-REQ-0095, primarily
\end{itemize}

Note this test was not designed to perform a full verification of the
above requirements, but rather to demonstrate having reached a certain
level of partial capability during construction.
}
  
 \newpage 
\subsection{[LVV-9812] DMS-LSP-REQ-0006-V-01: Use of VO Standards\_1 }\label{lvv-9812}

\begin{longtable}{cccc}
\hline
\textbf{Jira Link} & \textbf{Assignee} & \textbf{Status} & \textbf{Test Cases}\\ \hline
\href{https://jira.lsstcorp.org/browse/LVV-9812}{LVV-9812} &
Gregory Dubois-Felsmann & Not Covered &
\begin{tabular}{c}
LVV-T604 \\
LVV-T1334 \\
LVV-T1436 \\
LVV-T1437 \\
\end{tabular}
\\
\hline
\end{longtable}

\textbf{Verification Element Description:} \\
Undefined

{\footnotesize
\begin{longtable}{p{2.5cm}p{13.5cm}}
\hline
\multicolumn{2}{c}{\textbf{Requirement Details}}\\ \hline
Requirement ID & DMS-LSP-REQ-0006 \\ \cdashline{1-2}
Requirement Description &
\begin{minipage}[]{13cm}
The LSP shall utilize stable and accepted Virtual Observatory standards
for publically offered APIs wherever feasible.
\end{minipage}
\\ \cdashline{1-2}
Requirement Discussion &
\begin{minipage}[]{13cm}
The intent is to establish a ``VO First'' posture, and require explicit
exception be requested for any APIs needing to deviate from this
requirement.\\
This requirement applies most importantly to the API Aspect's
externally-facing data access APIs, but also in other areas.\\
E.g., from this requirement we may derive that the Portal must support
access to externally-source reference catalog via VO-compliant queries,
or that a Portal service for pushing data to user applications should
support SAMP.
\end{minipage}
\\ \cdashline{1-2}
Requirement Priority &  \\ \cdashline{1-2}
Upper Level Requirement &
\begin{tabular}{cl}
\end{tabular}
\\ \hline
\end{longtable}
}


\subsubsection{Test Cases Summary}
\begin{longtable}{p{3cm}p{2.5cm}p{2.5cm}p{3cm}p{4cm}}
\toprule
\href{https://jira.lsstcorp.org/secure/Tests.jspa\#/testCase/LVV-T604}{LVV-T604} & \multicolumn{4}{p{12cm}}{ Verify use of VO standards } \\ \hline
\textbf{Owner} & \textbf{Status} & \textbf{Version} & \textbf{Critical Event} & \textbf{Verification Type} \\ \hline
Jeffrey Carlin & Draft & 1 & false & Inspection \\ \hline
\end{longtable}
{\scriptsize
\textbf{Objective:}\\
Verify that the LSP utilizes stable and accepted Virtual Observatory
standards for public APIs.
}
\begin{longtable}{p{3cm}p{2.5cm}p{2.5cm}p{3cm}p{4cm}}
\toprule
\href{https://jira.lsstcorp.org/secure/Tests.jspa\#/testCase/LVV-T1334}{LVV-T1334} & \multicolumn{4}{p{12cm}}{ LDM-503-10a: Portal Aspect tests for LSP with Authentication and TAP
milestone } \\ \hline
\textbf{Owner} & \textbf{Status} & \textbf{Version} & \textbf{Critical Event} & \textbf{Verification Type} \\ \hline
Gregory Dubois-Felsmann & Defined & 1 & false & Test \\ \hline
\end{longtable}
{\scriptsize
\textbf{Objective:}\\
This test case verifies that the Portal Aspect of the Science Platform
is accessible to authorized users through a login process, and that TAP
searches can be performed from the Portal Aspect UI.\\[2\baselineskip]In
so doing and in conjunction with the other LDM-503-10a test cases
collected under LVV-P48, it addresses all or part of the following
requirements:

\begin{itemize}
\tightlist
\item
  DMS-LSP-REQ-0002, DMS-LSP-REQ-0005, DMS-LSP-REQ-0006,
  DMS-LSP-REQ-0020, DMS-LSP-REQ-0022, DMS-LSP-REQ-0023, DMS-LSP-REQ-0024
\item
  DMS-PRTL-REQ-0001, DMS-PRTL-REQ-0015, DMS-PRTL-REQ-0016,
  DMS-PRTL-REQ-0017, DMS-PRTL-REQ-0020, DMS-PRTL-REQ-0026,
  DMS-PRTL-REQ-0049, and DMS-PRTL-REQ-0095, primarily
\end{itemize}

Note this test was not designed to perform a full verification of the
above requirements, but rather to demonstrate having reached a certain
level of partial capability during construction.
}
\begin{longtable}{p{3cm}p{2.5cm}p{2.5cm}p{3cm}p{4cm}}
\toprule
\href{https://jira.lsstcorp.org/secure/Tests.jspa\#/testCase/LVV-T1436}{LVV-T1436} & \multicolumn{4}{p{12cm}}{ LDM-503-10a: Notebook Aspect tests for LSP with Authentication and TAP
milestone } \\ \hline
\textbf{Owner} & \textbf{Status} & \textbf{Version} & \textbf{Critical Event} & \textbf{Verification Type} \\ \hline
Gregory Dubois-Felsmann & Defined & 1 & false & Test \\ \hline
\end{longtable}
{\scriptsize
\textbf{Objective:}\\
This test case verifies that the Notebook Aspect of the Science Platform
is accessible to authorized users through a login process, and that TAP
searches can be performed from Python code in the Notebook
Aspect.\\[2\baselineskip]In so doing and in conjunction with the other
LDM-503-10a test cases collected under LVV-P48, it addresses all or part
of the following requirements:

\begin{itemize}
\tightlist
\item
  DMS-LSP-REQ-0003, DMS-LSP-REQ-0005, DMS-LSP-REQ-0006,
  DMS-LSP-REQ-0020, DMS-LSP-REQ-0022, DMS-LSP-REQ-0023, DMS-LSP-REQ-0024
\item
  DMS-NB-REQ-0001, DMS-NB-REQ-0002, DMS-NB-REQ-0005, DMS-NB-REQ-0006,
  DMS-NB-REQ-0013, DMS-NB-REQ-0017, and DMS-NB-REQ-0029, primarily
\end{itemize}

Note this test was not designed to perform a full verification of the
above requirements, but rather to demonstrate having reached a certain
level of partial capability during construction.
}
\begin{longtable}{p{3cm}p{2.5cm}p{2.5cm}p{3cm}p{4cm}}
\toprule
\href{https://jira.lsstcorp.org/secure/Tests.jspa\#/testCase/LVV-T1437}{LVV-T1437} & \multicolumn{4}{p{12cm}}{ LDM-503-10a: API Aspect tests for LSP with Authentication and TAP
milestone } \\ \hline
\textbf{Owner} & \textbf{Status} & \textbf{Version} & \textbf{Critical Event} & \textbf{Verification Type} \\ \hline
Gregory Dubois-Felsmann & Defined & 1 & false & Test \\ \hline
\end{longtable}
{\scriptsize
\textbf{Objective:}\\
This test case verifies that the TAP service in the API Aspect of the
Science Platform is accessible to authorized users through a login
process, and that TAP searches can be performed using the IVOA TAP
protocol from remote sites.\\[2\baselineskip]In so doing and in
conjunction with the other LDM-503-10a test cases collected under
LVV-P48, it addresses all or part of the following requirements:

\begin{itemize}
\tightlist
\item
  DMS-LSP-REQ-0004, DMS-LSP-REQ-0005, DMS-LSP-REQ-0006,
  DMS-LSP-REQ-0020, DMS-LSP-REQ-0022, DMS-LSP-REQ-0023, DMS-LSP-REQ-0024
\item
  DMS-API-REQ-0003, DMS-API-REQ-0004, DMS-API-REQ-0006,
  DMS-API-REQ-0007, DMS-API-REQ-0008, DMS-API-REQ-0009,
  DMS-API-REQ-0023, and DMS-API-REQ-0039, primarily
\end{itemize}

Note this test was not designed to perform a full verification of the
above requirements, but rather to demonstrate having reached a certain
level of partial capability during construction.
}
  
 \newpage 
\subsection{[LVV-9813] DMS-LSP-REQ-0009-V-01: Semantic Linkage: Uncertainties\_1 }\label{lvv-9813}

\begin{longtable}{cccc}
\hline
\textbf{Jira Link} & \textbf{Assignee} & \textbf{Status} & \textbf{Test Cases}\\ \hline
\href{https://jira.lsstcorp.org/browse/LVV-9813}{LVV-9813} &
Gregory Dubois-Felsmann & Not Covered &
\begin{tabular}{c}
LVV-T607 \\
\end{tabular}
\\
\hline
\end{longtable}

\textbf{Verification Element Description:} \\
Undefined

{\footnotesize
\begin{longtable}{p{2.5cm}p{13.5cm}}
\hline
\multicolumn{2}{c}{\textbf{Requirement Details}}\\ \hline
Requirement ID & DMS-LSP-REQ-0009 \\ \cdashline{1-2}
Requirement Description &
\begin{minipage}[]{13cm}
The LSP shall support the identification of relationships between data
items (notably database columns) that represent a quantity and its
uncertainty(ies).
\end{minipage}
\\ \cdashline{1-2}
Requirement Discussion &
\begin{minipage}[]{13cm}
This should extend to relationships between groups of quantities and
other quantities that represent their covariances, but this is not an
open-ended requirement to support all such cases.\\
These relationships should be expressed as appropriate to the LSP
aspect, and require ``upstream'' support at the point of definition or
creation of the data.
\end{minipage}
\\ \cdashline{1-2}
Requirement Priority &  \\ \cdashline{1-2}
Upper Level Requirement &
\begin{tabular}{cl}
\end{tabular}
\\ \hline
\end{longtable}
}


\subsubsection{Test Cases Summary}
\begin{longtable}{p{3cm}p{2.5cm}p{2.5cm}p{3cm}p{4cm}}
\toprule
\href{https://jira.lsstcorp.org/secure/Tests.jspa\#/testCase/LVV-T607}{LVV-T607} & \multicolumn{4}{p{12cm}}{ Verify semantic linkages between data items and uncertainties } \\ \hline
\textbf{Owner} & \textbf{Status} & \textbf{Version} & \textbf{Critical Event} & \textbf{Verification Type} \\ \hline
Jeffrey Carlin & Draft & 1 & false & Test \\ \hline
\end{longtable}
{\scriptsize
\textbf{Objective:}\\
Verify that the LSP provides methods to identify uncertainties
associated with a given quantity.
}
  
 \newpage 
\subsection{[LVV-9814] DMS-LSP-REQ-0008-V-01: Semantic Linkage\_1 }\label{lvv-9814}

\begin{longtable}{cccc}
\hline
\textbf{Jira Link} & \textbf{Assignee} & \textbf{Status} & \textbf{Test Cases}\\ \hline
\href{https://jira.lsstcorp.org/browse/LVV-9814}{LVV-9814} &
Gregory Dubois-Felsmann & Not Covered &
\begin{tabular}{c}
LVV-T8 \\
LVV-T9 \\
LVV-T606 \\
\end{tabular}
\\
\hline
\end{longtable}

\textbf{Verification Element Description:} \\
Undefined

{\footnotesize
\begin{longtable}{p{2.5cm}p{13.5cm}}
\hline
\multicolumn{2}{c}{\textbf{Requirement Details}}\\ \hline
Requirement ID & DMS-LSP-REQ-0008 \\ \cdashline{1-2}
Requirement Description &
\begin{minipage}[]{13cm}
The LSP shall support the identification of linkages between data items
that reflect their provenance and data dependencies.
\end{minipage}
\\ \cdashline{1-2}
Requirement Discussion &
\begin{minipage}[]{13cm}
For instance, from a calibrated image it should be possible to identify
the raw image from which it was generated, and the calibration data used
in its processing; from a catalog entry it should be possible to
identify the image(s) on which the measurement was made.\\
In practice this has DM system-level implications. For the LSP it relies
on upstream components recording the appropriate provenance metadata. At
the DAX level it mainly just requires exposing the tables and columns
containing this metadata, and identifying them as such (e.g., by
reporting foreign-key relationships) in the Reflection APIs. At the
Notebook level this is largely satisfied by providing Butler-level
access to this information.
\end{minipage}
\\ \cdashline{1-2}
Requirement Priority &  \\ \cdashline{1-2}
Upper Level Requirement &
\begin{tabular}{cl}
\end{tabular}
\\ \hline
\end{longtable}
}


\subsubsection{Test Cases Summary}
\begin{longtable}{p{3cm}p{2.5cm}p{2.5cm}p{3cm}p{4cm}}
\toprule
\href{https://jira.lsstcorp.org/secure/Tests.jspa\#/testCase/LVV-T8}{LVV-T8} & \multicolumn{4}{p{12cm}}{ LSP-00-30: Linkage of catalog query results with associated images } \\ \hline
\textbf{Owner} & \textbf{Status} & \textbf{Version} & \textbf{Critical Event} & \textbf{Verification Type} \\ \hline
Gregory Dubois-Felsmann & Deprecated & 1 & false & Test \\ \hline
\end{longtable}
{\scriptsize
\textbf{Objective:}\\
This test will check for the ability, in the Portal Aspect of the LSST
Science Platform, to match catalog data with the image data on which the
measurements were performed, specifically:

\begin{itemize}
\tightlist
\item
  Navigating from a catalog query result to the associated images; and~
\item
  Overlaying catalog query results on associated images.
\end{itemize}

Because of limited staff resources, these tests will be based on the
original PDAC dataset, the LSST Summer 2013 processing of the SDSS
Stripe 82 data. The image data for the WISE and NEOWISE missions have
not been loaded into PDAC.
}
\begin{longtable}{p{3cm}p{2.5cm}p{2.5cm}p{3cm}p{4cm}}
\toprule
\href{https://jira.lsstcorp.org/secure/Tests.jspa\#/testCase/LVV-T9}{LVV-T9} & \multicolumn{4}{p{12cm}}{ LSP-00-35: Linkage of catalog query results to related catalog data } \\ \hline
\textbf{Owner} & \textbf{Status} & \textbf{Version} & \textbf{Critical Event} & \textbf{Verification Type} \\ \hline
Gregory Dubois-Felsmann & Deprecated & 1 & false & Test \\ \hline
\end{longtable}
{\scriptsize
\textbf{Objective:}\\
This test will check for the ability, in the Portal Aspect of the LSST
Science Platform, to match catalog data with related catalog data.
Specifically, the test verifies the ability to navigate from a coadded
source catalog entry to the associated forced
photometry.\\[2\baselineskip]\textbf{Requirements (to be removed when
Reqs are synchronized from magic draw)}

\begin{itemize}
\tightlist
\item
  DMS-LSP-REQ-0008
\end{itemize}
}
\begin{longtable}{p{3cm}p{2.5cm}p{2.5cm}p{3cm}p{4cm}}
\toprule
\href{https://jira.lsstcorp.org/secure/Tests.jspa\#/testCase/LVV-T606}{LVV-T606} & \multicolumn{4}{p{12cm}}{ Verify semantic linkages between data items } \\ \hline
\textbf{Owner} & \textbf{Status} & \textbf{Version} & \textbf{Critical Event} & \textbf{Verification Type} \\ \hline
Jeffrey Carlin & Draft & 1 & false & Test \\ \hline
\end{longtable}
{\scriptsize
\textbf{Objective:}\\
Verify that the LSP provides access to linkages between data items that
reflect their provenance and data dependencies.
}
  
 \newpage 
\subsection{[LVV-9815] DMS-LSP-REQ-0010-V-01: Transfer of Portal Data References to Notebook\_1 }\label{lvv-9815}

\begin{longtable}{cccc}
\hline
\textbf{Jira Link} & \textbf{Assignee} & \textbf{Status} & \textbf{Test Cases}\\ \hline
\href{https://jira.lsstcorp.org/browse/LVV-9815}{LVV-9815} &
Gregory Dubois-Felsmann & Not Covered &
\begin{tabular}{c}
LVV-T608 \\
\end{tabular}
\\
\hline
\end{longtable}

\textbf{Verification Element Description:} \\
Undefined

{\footnotesize
\begin{longtable}{p{2.5cm}p{13.5cm}}
\hline
\multicolumn{2}{c}{\textbf{Requirement Details}}\\ \hline
Requirement ID & DMS-LSP-REQ-0010 \\ \cdashline{1-2}
Requirement Description &
\begin{minipage}[]{13cm}
The LSP shall facilitate the transfer to the Notebook aspect of
references allowing retrieval in a notebook of the data explored in the
Portal session.
\end{minipage}
\\ \cdashline{1-2}
Requirement Discussion &
\begin{minipage}[]{13cm}
This allows a user to locate and preview data in the Portal environment
and then readily transfer their work to the Notebook aspect for detailed
analysis.
\end{minipage}
\\ \cdashline{1-2}
Requirement Priority &  \\ \cdashline{1-2}
Upper Level Requirement &
\begin{tabular}{cl}
\end{tabular}
\\ \hline
\end{longtable}
}


\subsubsection{Test Cases Summary}
\begin{longtable}{p{3cm}p{2.5cm}p{2.5cm}p{3cm}p{4cm}}
\toprule
\href{https://jira.lsstcorp.org/secure/Tests.jspa\#/testCase/LVV-T608}{LVV-T608} & \multicolumn{4}{p{12cm}}{ Verify transfer of Portal data references to Notebook aspect } \\ \hline
\textbf{Owner} & \textbf{Status} & \textbf{Version} & \textbf{Critical Event} & \textbf{Verification Type} \\ \hline
Jeffrey Carlin & Draft & 1 & false & Test \\ \hline
\end{longtable}
{\scriptsize
\textbf{Objective:}\\
Verify that data references derived from Portal exploration can be
transferred and used in to retrieve the same data in the Notebook
aspect.
}
  
 \newpage 
\subsection{[LVV-9816] DMS-LSP-REQ-0012-V-01: User Database Workspace\_1 }\label{lvv-9816}

\begin{longtable}{cccc}
\hline
\textbf{Jira Link} & \textbf{Assignee} & \textbf{Status} & \textbf{Test Cases}\\ \hline
\href{https://jira.lsstcorp.org/browse/LVV-9816}{LVV-9816} &
Gregory Dubois-Felsmann & Not Covered &
\begin{tabular}{c}
LVV-T610 \\
\end{tabular}
\\
\hline
\end{longtable}

\textbf{Verification Element Description:} \\
Undefined

{\footnotesize
\begin{longtable}{p{2.5cm}p{13.5cm}}
\hline
\multicolumn{2}{c}{\textbf{Requirement Details}}\\ \hline
Requirement ID & DMS-LSP-REQ-0012 \\ \cdashline{1-2}
Requirement Description &
\begin{minipage}[]{13cm}
The LSP shall provide for the creation, use, and management of user
databases (User Generated tabular data products), and shall enable
interaction with user databases with the same facilities as for
Project-created database to the extent feasible.
\end{minipage}
\\ \cdashline{1-2}
Requirement Discussion &
\begin{minipage}[]{13cm}
Some database-related capabilities of the LSP rely on the availability
of detailed metadata on the Project-created databases that goes beyond
the normal content of a database schema (e.g., IVOA UCDs for table
columns). Users will be enabled, but not required, to supply such
metadata for their own databases (and they may do so incorrectly), so
LSP functionality that depends on it may not be available for user
databases.
\end{minipage}
\\ \cdashline{1-2}
Requirement Priority &  \\ \cdashline{1-2}
Upper Level Requirement &
\begin{tabular}{cl}
\end{tabular}
\\ \hline
\end{longtable}
}


\subsubsection{Test Cases Summary}
\begin{longtable}{p{3cm}p{2.5cm}p{2.5cm}p{3cm}p{4cm}}
\toprule
\href{https://jira.lsstcorp.org/secure/Tests.jspa\#/testCase/LVV-T610}{LVV-T610} & \multicolumn{4}{p{12cm}}{ Verify providing user generated database in LSP } \\ \hline
\textbf{Owner} & \textbf{Status} & \textbf{Version} & \textbf{Critical Event} & \textbf{Verification Type} \\ \hline
Jeffrey Carlin & Draft & 1 & false & Test \\ \hline
\end{longtable}
{\scriptsize
\textbf{Objective:}\\
Verify that the LSP allows for creation, use, and management of User
Generated databases, and interaction with user databases by the same
facilities as Project databases, where feasible.
}
  
 \newpage 
\subsection{[LVV-9817] DMS-LSP-REQ-0011-V-01: User File Workspace\_1 }\label{lvv-9817}

\begin{longtable}{cccc}
\hline
\textbf{Jira Link} & \textbf{Assignee} & \textbf{Status} & \textbf{Test Cases}\\ \hline
\href{https://jira.lsstcorp.org/browse/LVV-9817}{LVV-9817} &
Gregory Dubois-Felsmann & Not Covered &
\begin{tabular}{c}
LVV-T609 \\
\end{tabular}
\\
\hline
\end{longtable}

\textbf{Verification Element Description:} \\
Undefined

{\footnotesize
\begin{longtable}{p{2.5cm}p{13.5cm}}
\hline
\multicolumn{2}{c}{\textbf{Requirement Details}}\\ \hline
Requirement ID & DMS-LSP-REQ-0011 \\ \cdashline{1-2}
Requirement Description &
\begin{minipage}[]{13cm}
The LSP shall provide a ``User File Workspace'': resources for the
storage of file oriented User Generated data, which shall be accessible
from all three aspects.
\end{minipage}
\\ \cdashline{1-2}
Requirement Discussion &
\begin{minipage}[]{13cm}
All users with data rights will have a ``birthright'' quota of User File
Workspace storage. A proposal-based process will be available for
requests for additional storage. The Workspace will not, in general,
place constraints on the format of the data stored. It may be used for
image data as well as for tabular data in file-oriented storage forms.
User databases are a separate mechanism.\\
The Workspace is expected to be made available through an implementation
of the IVOA VOSpace standard.
\end{minipage}
\\ \cdashline{1-2}
Requirement Priority &  \\ \cdashline{1-2}
Upper Level Requirement &
\begin{tabular}{cl}
\end{tabular}
\\ \hline
\end{longtable}
}


\subsubsection{Test Cases Summary}
\begin{longtable}{p{3cm}p{2.5cm}p{2.5cm}p{3cm}p{4cm}}
\toprule
\href{https://jira.lsstcorp.org/secure/Tests.jspa\#/testCase/LVV-T609}{LVV-T609} & \multicolumn{4}{p{12cm}}{ Verify providing user file storage in LSP } \\ \hline
\textbf{Owner} & \textbf{Status} & \textbf{Version} & \textbf{Critical Event} & \textbf{Verification Type} \\ \hline
Jeffrey Carlin & Draft & 1 & false & Test \\ \hline
\end{longtable}
{\scriptsize
\textbf{Objective:}\\
Verify that the LSP provides a user file workspace for storage of user
generated data files. These shall be accessible from all three aspects.
}
  
 \newpage 
\subsection{[LVV-9818] DMS-LSP-REQ-0013-V-01: User Workspace Access Controls\_1 }\label{lvv-9818}

\begin{longtable}{cccc}
\hline
\textbf{Jira Link} & \textbf{Assignee} & \textbf{Status} & \textbf{Test Cases}\\ \hline
\href{https://jira.lsstcorp.org/browse/LVV-9818}{LVV-9818} &
Gregory Dubois-Felsmann & Not Covered &
\begin{tabular}{c}
LVV-T611 \\
\end{tabular}
\\
\hline
\end{longtable}

\textbf{Verification Element Description:} \\
Undefined

{\footnotesize
\begin{longtable}{p{2.5cm}p{13.5cm}}
\hline
\multicolumn{2}{c}{\textbf{Requirement Details}}\\ \hline
Requirement ID & DMS-LSP-REQ-0013 \\ \cdashline{1-2}
Requirement Description &
\begin{minipage}[]{13cm}
The LSP shall permit users to place access restrictions on data in the
User File and Database Workspaces, based on both user and user-group
identities, and shall enforce these restrictions in all its aspects.
\end{minipage}
\\ \cdashline{1-2}
Requirement Priority &  \\ \cdashline{1-2}
Upper Level Requirement &
\begin{tabular}{cl}
\end{tabular}
\\ \hline
\end{longtable}
}


\subsubsection{Test Cases Summary}
\begin{longtable}{p{3cm}p{2.5cm}p{2.5cm}p{3cm}p{4cm}}
\toprule
\href{https://jira.lsstcorp.org/secure/Tests.jspa\#/testCase/LVV-T611}{LVV-T611} & \multicolumn{4}{p{12cm}}{ Verify access controls in user workspace } \\ \hline
\textbf{Owner} & \textbf{Status} & \textbf{Version} & \textbf{Critical Event} & \textbf{Verification Type} \\ \hline
Jeffrey Carlin & Draft & 1 & false & Test \\ \hline
\end{longtable}
{\scriptsize
\textbf{Objective:}\\
Verify that LSP users can place access restrictions on data in the User
File and Database workspaces, and that these restrictions are enforced
across all aspects.
}
  
 \newpage 
\subsection{[LVV-9819] DMS-LSP-REQ-0014-V-01: Download Data\_1 }\label{lvv-9819}

\begin{longtable}{cccc}
\hline
\textbf{Jira Link} & \textbf{Assignee} & \textbf{Status} & \textbf{Test Cases}\\ \hline
\href{https://jira.lsstcorp.org/browse/LVV-9819}{LVV-9819} &
Gregory Dubois-Felsmann & Not Covered &
\begin{tabular}{c}
LVV-T5 \\
LVV-T6 \\
LVV-T7 \\
LVV-T612 \\
\end{tabular}
\\
\hline
\end{longtable}

\textbf{Verification Element Description:} \\
Undefined

{\footnotesize
\begin{longtable}{p{2.5cm}p{13.5cm}}
\hline
\multicolumn{2}{c}{\textbf{Requirement Details}}\\ \hline
Requirement ID & DMS-LSP-REQ-0014 \\ \cdashline{1-2}
Requirement Description &
\begin{minipage}[]{13cm}
The LSP shall provide means for downloading data resulting from queries
or other operations, or from the Workspace, to the user's system.
\end{minipage}
\\ \cdashline{1-2}
Requirement Discussion &
\begin{minipage}[]{13cm}
In general the API aspect can be used to retrieve data to remote sites.
Additional mechanisms will be provided as appropriate to the various
aspects, and are described in lower-level requirements. For the DAC
instances of the LSP, the ``user's system'' will generally be on the
public Internet; for other instances it may be on project-internal
systems.
\end{minipage}
\\ \cdashline{1-2}
Requirement Priority &  \\ \cdashline{1-2}
Upper Level Requirement &
\begin{tabular}{cl}
\end{tabular}
\\ \hline
\end{longtable}
}


\subsubsection{Test Cases Summary}
\begin{longtable}{p{3cm}p{2.5cm}p{2.5cm}p{3cm}p{4cm}}
\toprule
\href{https://jira.lsstcorp.org/secure/Tests.jspa\#/testCase/LVV-T5}{LVV-T5} & \multicolumn{4}{p{12cm}}{ LSP-00-15: Execution of basic catalog queries in the Portal } \\ \hline
\textbf{Owner} & \textbf{Status} & \textbf{Version} & \textbf{Critical Event} & \textbf{Verification Type} \\ \hline
Gregory Dubois-Felsmann & Deprecated & 1 & false & Test \\ \hline
\end{longtable}
{\scriptsize
\textbf{Objective:}\\
This test will test the functional requirements to be able to perform a
range of basic queries through the Portal Aspect of the LSP:

\begin{itemize}
\tightlist
\item
  Cone searches on the Object-like, ForcedSource-like, and Source-like
  WISE tables;~
\item
  Multi-target cone searches;
\item
  Form-based searches for exact equality, e.g., for row IDs; and
\item
  Form-based searches for sets of object attributes.
\end{itemize}

In addition, it tests the ability to download tabular query results from
the Portal Aspect.
}
\begin{longtable}{p{3cm}p{2.5cm}p{2.5cm}p{3cm}p{4cm}}
\toprule
\href{https://jira.lsstcorp.org/secure/Tests.jspa\#/testCase/LVV-T6}{LVV-T6} & \multicolumn{4}{p{12cm}}{ LSP-00-20: Operation of the UI for interaction with tabular data results } \\ \hline
\textbf{Owner} & \textbf{Status} & \textbf{Version} & \textbf{Critical Event} & \textbf{Verification Type} \\ \hline
Gregory Dubois-Felsmann & Deprecated & 1 & false & Test \\ \hline
\end{longtable}
{\scriptsize
\textbf{Objective:}\\
This test will test the functional requirements to be able to perform
certain basic exploratory data analysis functions on tabular data
results in the Portal Aspect UI:

\begin{itemize}
\tightlist
\item
  Sort tabular results;
\item
  Filter tabular results based on the contents of columns;~
\item
  Perform per-row selections from a table;
\item
  Display 1D histograms of selected attributes;
\item
  Display 2D scatter plots of selected attributes;
\item
  Perform graphical selections of rows from plots; and
\item
  Download tabular query results reflecting sorting and selection.
\end{itemize}

This test does not address the limits of scaling of these capabilities
to large query results. That will be addressed in future test
specifications. The test report should include notes on the sizes of
results that were used.
}
\begin{longtable}{p{3cm}p{2.5cm}p{2.5cm}p{3cm}p{4cm}}
\toprule
\href{https://jira.lsstcorp.org/secure/Tests.jspa\#/testCase/LVV-T7}{LVV-T7} & \multicolumn{4}{p{12cm}}{ LSP-00-25: Image metadata, image, and image cutout queries } \\ \hline
\textbf{Owner} & \textbf{Status} & \textbf{Version} & \textbf{Critical Event} & \textbf{Verification Type} \\ \hline
Gregory Dubois-Felsmann & Deprecated & 1 & false & Test \\ \hline
\end{longtable}
{\scriptsize
\textbf{Objective:}\\
This test will check basic functionality related to image search and
retrieval, via both the API Aspect and the Portal Aspect of the LSST
Science Platform:

\begin{itemize}
\tightlist
\item
  Searching for images containing a specified point;
\item
  Displaying selected images;
\item
  Obtaining and displaying image cutouts at a specified point; and
\item
  Downloading selected images and image cutouts.
\end{itemize}

Because of limited staff resources, these tests will be based on the
original PDAC dataset, the LSST Summer 2013 processing of the SDSS
Stripe 82 data. The image data for the WISE and NEOWISE missions have
not been loaded into PDAC.\\[2\baselineskip]
}
\begin{longtable}{p{3cm}p{2.5cm}p{2.5cm}p{3cm}p{4cm}}
\toprule
\href{https://jira.lsstcorp.org/secure/Tests.jspa\#/testCase/LVV-T612}{LVV-T612} & \multicolumn{4}{p{12cm}}{ Verify ability to download data from LSP } \\ \hline
\textbf{Owner} & \textbf{Status} & \textbf{Version} & \textbf{Critical Event} & \textbf{Verification Type} \\ \hline
Jeffrey Carlin & Draft & 1 & false & Test \\ \hline
\end{longtable}
{\scriptsize
\textbf{Objective:}\\
Verify that the LSP provides a means to download data from queries, user
workspaces, or other operations, to the user's system.
}
  
 \newpage 
\subsection{[LVV-9820] DMS-LSP-REQ-0018-V-01: Image Data Download File Format\_1 }\label{lvv-9820}

\begin{longtable}{cccc}
\hline
\textbf{Jira Link} & \textbf{Assignee} & \textbf{Status} & \textbf{Test Cases}\\ \hline
\href{https://jira.lsstcorp.org/browse/LVV-9820}{LVV-9820} &
Gregory Dubois-Felsmann & Not Covered &
\begin{tabular}{c}
LVV-T7 \\
LVV-T616 \\
\end{tabular}
\\
\hline
\end{longtable}

\textbf{Verification Element Description:} \\
Undefined

{\footnotesize
\begin{longtable}{p{2.5cm}p{13.5cm}}
\hline
\multicolumn{2}{c}{\textbf{Requirement Details}}\\ \hline
Requirement ID & DMS-LSP-REQ-0018 \\ \cdashline{1-2}
Requirement Description &
\begin{minipage}[]{13cm}
The LSP shall allow LSST image data products to be downloaded or saved
to the workspace as FITS files including the appropriate metadata.
\end{minipage}
\\ \cdashline{1-2}
Requirement Discussion &
\begin{minipage}[]{13cm}
This needs some discussion as to what other formats we may want and must
be readable by future releases.
\end{minipage}
\\ \cdashline{1-2}
Requirement Priority &  \\ \cdashline{1-2}
Upper Level Requirement &
\begin{tabular}{cl}
\end{tabular}
\\ \hline
\end{longtable}
}


\subsubsection{Test Cases Summary}
\begin{longtable}{p{3cm}p{2.5cm}p{2.5cm}p{3cm}p{4cm}}
\toprule
\href{https://jira.lsstcorp.org/secure/Tests.jspa\#/testCase/LVV-T7}{LVV-T7} & \multicolumn{4}{p{12cm}}{ LSP-00-25: Image metadata, image, and image cutout queries } \\ \hline
\textbf{Owner} & \textbf{Status} & \textbf{Version} & \textbf{Critical Event} & \textbf{Verification Type} \\ \hline
Gregory Dubois-Felsmann & Deprecated & 1 & false & Test \\ \hline
\end{longtable}
{\scriptsize
\textbf{Objective:}\\
This test will check basic functionality related to image search and
retrieval, via both the API Aspect and the Portal Aspect of the LSST
Science Platform:

\begin{itemize}
\tightlist
\item
  Searching for images containing a specified point;
\item
  Displaying selected images;
\item
  Obtaining and displaying image cutouts at a specified point; and
\item
  Downloading selected images and image cutouts.
\end{itemize}

Because of limited staff resources, these tests will be based on the
original PDAC dataset, the LSST Summer 2013 processing of the SDSS
Stripe 82 data. The image data for the WISE and NEOWISE missions have
not been loaded into PDAC.\\[2\baselineskip]
}
\begin{longtable}{p{3cm}p{2.5cm}p{2.5cm}p{3cm}p{4cm}}
\toprule
\href{https://jira.lsstcorp.org/secure/Tests.jspa\#/testCase/LVV-T616}{LVV-T616} & \multicolumn{4}{p{12cm}}{ Verify file formats provided for image data download } \\ \hline
\textbf{Owner} & \textbf{Status} & \textbf{Version} & \textbf{Critical Event} & \textbf{Verification Type} \\ \hline
Jeffrey Carlin & Draft & 1 & false & Test \\ \hline
\end{longtable}
{\scriptsize
\textbf{Objective:}\\
Verify that LSST image data products can be downloaded via the LSP in
FITS format, with appropriate metadata included.
}
  
 \newpage 
\subsection{[LVV-9821] DMS-LSP-REQ-0017-V-01: Tabular Data Download File Formats\_1 }\label{lvv-9821}

\begin{longtable}{cccc}
\hline
\textbf{Jira Link} & \textbf{Assignee} & \textbf{Status} & \textbf{Test Cases}\\ \hline
\href{https://jira.lsstcorp.org/browse/LVV-9821}{LVV-9821} &
Gregory Dubois-Felsmann & Not Covered &
\begin{tabular}{c}
LVV-T6 \\
LVV-T615 \\
\end{tabular}
\\
\hline
\end{longtable}

\textbf{Verification Element Description:} \\
Undefined

{\footnotesize
\begin{longtable}{p{2.5cm}p{13.5cm}}
\hline
\multicolumn{2}{c}{\textbf{Requirement Details}}\\ \hline
Requirement ID & DMS-LSP-REQ-0017 \\ \cdashline{1-2}
Requirement Description &
\begin{minipage}[]{13cm}
The LSP shall allow tabular search results, including but not limited to
data from the source and object tables and the image metadata tables, to
be downloaded or saved to the workspace in at least the following
formats: FITS table, VOTable, and ASCII delimiter-separated table (e.g.,
CSV).
\end{minipage}
\\ \cdashline{1-2}
Requirement Discussion &
\begin{minipage}[]{13cm}
A final set of formats needs to be discussed and approved by the
Project.\\
This is primarily a requirement on the API aspect and the DAX services.
\end{minipage}
\\ \cdashline{1-2}
Requirement Priority &  \\ \cdashline{1-2}
Upper Level Requirement &
\begin{tabular}{cl}
\end{tabular}
\\ \hline
\end{longtable}
}


\subsubsection{Test Cases Summary}
\begin{longtable}{p{3cm}p{2.5cm}p{2.5cm}p{3cm}p{4cm}}
\toprule
\href{https://jira.lsstcorp.org/secure/Tests.jspa\#/testCase/LVV-T6}{LVV-T6} & \multicolumn{4}{p{12cm}}{ LSP-00-20: Operation of the UI for interaction with tabular data results } \\ \hline
\textbf{Owner} & \textbf{Status} & \textbf{Version} & \textbf{Critical Event} & \textbf{Verification Type} \\ \hline
Gregory Dubois-Felsmann & Deprecated & 1 & false & Test \\ \hline
\end{longtable}
{\scriptsize
\textbf{Objective:}\\
This test will test the functional requirements to be able to perform
certain basic exploratory data analysis functions on tabular data
results in the Portal Aspect UI:

\begin{itemize}
\tightlist
\item
  Sort tabular results;
\item
  Filter tabular results based on the contents of columns;~
\item
  Perform per-row selections from a table;
\item
  Display 1D histograms of selected attributes;
\item
  Display 2D scatter plots of selected attributes;
\item
  Perform graphical selections of rows from plots; and
\item
  Download tabular query results reflecting sorting and selection.
\end{itemize}

This test does not address the limits of scaling of these capabilities
to large query results. That will be addressed in future test
specifications. The test report should include notes on the sizes of
results that were used.
}
\begin{longtable}{p{3cm}p{2.5cm}p{2.5cm}p{3cm}p{4cm}}
\toprule
\href{https://jira.lsstcorp.org/secure/Tests.jspa\#/testCase/LVV-T615}{LVV-T615} & \multicolumn{4}{p{12cm}}{ Verify file formats provided for tabular data download } \\ \hline
\textbf{Owner} & \textbf{Status} & \textbf{Version} & \textbf{Critical Event} & \textbf{Verification Type} \\ \hline
Jeffrey Carlin & Draft & 1 & false & Test \\ \hline
\end{longtable}
{\scriptsize
\textbf{Objective:}\\
Verify that the LSP allows tabular data from search results to be
downloaded in FITS, VOTable, and ASCII delimiter-separated tables (e.g.,
CSV).
}
  
 \newpage 
\subsection{[LVV-9822] DMS-LSP-REQ-0016-V-01: Transfer Data to Workspace\_1 }\label{lvv-9822}

\begin{longtable}{cccc}
\hline
\textbf{Jira Link} & \textbf{Assignee} & \textbf{Status} & \textbf{Test Cases}\\ \hline
\href{https://jira.lsstcorp.org/browse/LVV-9822}{LVV-9822} &
Gregory Dubois-Felsmann & Not Covered &
\begin{tabular}{c}
LVV-T614 \\
\end{tabular}
\\
\hline
\end{longtable}

\textbf{Verification Element Description:} \\
Undefined

{\footnotesize
\begin{longtable}{p{2.5cm}p{13.5cm}}
\hline
\multicolumn{2}{c}{\textbf{Requirement Details}}\\ \hline
Requirement ID & DMS-LSP-REQ-0016 \\ \cdashline{1-2}
Requirement Description &
\begin{minipage}[]{13cm}
Transfer of data to and from the Workspace shall be usable as an
alternative in all features of the LSP where download or upload,
respectively, are available.
\end{minipage}
\\ \cdashline{1-2}
Requirement Discussion &
\begin{minipage}[]{13cm}
The user can ``download'' data, e.g., a query result, to the Workspace,
and can ``upload'' data from the Workspace into other functions of the
LSP, e.g., to use as a target list in a multi-object search.
\end{minipage}
\\ \cdashline{1-2}
Requirement Priority &  \\ \cdashline{1-2}
Upper Level Requirement &
\begin{tabular}{cl}
\end{tabular}
\\ \hline
\end{longtable}
}


\subsubsection{Test Cases Summary}
\begin{longtable}{p{3cm}p{2.5cm}p{2.5cm}p{3cm}p{4cm}}
\toprule
\href{https://jira.lsstcorp.org/secure/Tests.jspa\#/testCase/LVV-T614}{LVV-T614} & \multicolumn{4}{p{12cm}}{ Verify ability to transfer data to and from the Workspace } \\ \hline
\textbf{Owner} & \textbf{Status} & \textbf{Version} & \textbf{Critical Event} & \textbf{Verification Type} \\ \hline
Jeffrey Carlin & Draft & 1 & false & Test \\ \hline
\end{longtable}
{\scriptsize
\textbf{Objective:}\\
Verify that users can transfer data between all features of the LSP that
allow for upload and download of data.
}
  
 \newpage 
\subsection{[LVV-9823] DMS-LSP-REQ-0015-V-01: Upload Data\_1 }\label{lvv-9823}

\begin{longtable}{cccc}
\hline
\textbf{Jira Link} & \textbf{Assignee} & \textbf{Status} & \textbf{Test Cases}\\ \hline
\href{https://jira.lsstcorp.org/browse/LVV-9823}{LVV-9823} &
Gregory Dubois-Felsmann & Not Covered &
\begin{tabular}{c}
LVV-T613 \\
\end{tabular}
\\
\hline
\end{longtable}

\textbf{Verification Element Description:} \\
Undefined

{\footnotesize
\begin{longtable}{p{2.5cm}p{13.5cm}}
\hline
\multicolumn{2}{c}{\textbf{Requirement Details}}\\ \hline
Requirement ID & DMS-LSP-REQ-0015 \\ \cdashline{1-2}
Requirement Description &
\begin{minipage}[]{13cm}
The LSP shall provide means for uploading data from the user's system
for use in the LSP aspects, including for storage in the Workspace.
\end{minipage}
\\ \cdashline{1-2}
Requirement Priority &  \\ \cdashline{1-2}
Upper Level Requirement &
\begin{tabular}{cl}
\end{tabular}
\\ \hline
\end{longtable}
}


\subsubsection{Test Cases Summary}
\begin{longtable}{p{3cm}p{2.5cm}p{2.5cm}p{3cm}p{4cm}}
\toprule
\href{https://jira.lsstcorp.org/secure/Tests.jspa\#/testCase/LVV-T613}{LVV-T613} & \multicolumn{4}{p{12cm}}{ Verify ability to upload data to LSP } \\ \hline
\textbf{Owner} & \textbf{Status} & \textbf{Version} & \textbf{Critical Event} & \textbf{Verification Type} \\ \hline
Jeffrey Carlin & Draft & 1 & false & Test \\ \hline
\end{longtable}
{\scriptsize
\textbf{Objective:}\\
Verify that LSP users can upload data from their system for use in the
LSP aspects and storage in their user workspace.
}
  
 \newpage 
\subsection{[LVV-9824] DMS-LSP-REQ-0028-V-01: Peak Volume for Moderate-Sized Queries\_1 }\label{lvv-9824}

\begin{longtable}{cccc}
\hline
\textbf{Jira Link} & \textbf{Assignee} & \textbf{Status} & \textbf{Test Cases}\\ \hline
\href{https://jira.lsstcorp.org/browse/LVV-9824}{LVV-9824} &
Gregory Dubois-Felsmann & Not Covered &
\begin{tabular}{c}
LVV-T4 \\
LVV-T617 \\
\end{tabular}
\\
\hline
\end{longtable}

\textbf{Verification Element Description:} \\
Undefined

{\footnotesize
\begin{longtable}{p{2.5cm}p{13.5cm}}
\hline
\multicolumn{2}{c}{\textbf{Requirement Details}}\\ \hline
Requirement ID & DMS-LSP-REQ-0028 \\ \cdashline{1-2}
Requirement Description &
\begin{minipage}[]{13cm}
The LSP shall handle at peak usage 50 simultaneous queries without
degradation, with the following properties: input selection of up to 1E7
objects in the catalog, result data set of up to 0.1GB, and a response
time of 10 seconds.
\end{minipage}
\\ \cdashline{1-2}
Requirement Discussion &
\begin{minipage}[]{13cm}
This requirement flows down from several requirements in the DMSR
(\citeds{LSE-61}) which constrain both the performance of the database systems
(via \citeds{LDM-555}) and the Science Platform. In the Science Platform context,
this requirement implies that the LSP will not degrade the performance
supported by the underlying database systems. This applies to queries
originating from any of the Aspects, and covers all of the stages of the
query: query generation, query running, results generation, display of
results, and downloading and saving of results.
\end{minipage}
\\ \cdashline{1-2}
Requirement Priority &  \\ \cdashline{1-2}
Upper Level Requirement &
\begin{tabular}{cl}
\end{tabular}
\\ \hline
\end{longtable}
}


\subsubsection{Test Cases Summary}
\begin{longtable}{p{3cm}p{2.5cm}p{2.5cm}p{3cm}p{4cm}}
\toprule
\href{https://jira.lsstcorp.org/secure/Tests.jspa\#/testCase/LVV-T4}{LVV-T4} & \multicolumn{4}{p{12cm}}{ LSP-00-10: Demonstration of table-scan queries against the WISE data via
API } \\ \hline
\textbf{Owner} & \textbf{Status} & \textbf{Version} & \textbf{Critical Event} & \textbf{Verification Type} \\ \hline
Gregory Dubois-Felsmann & Deprecated & 1 & false & Test \\ \hline
\end{longtable}
{\scriptsize
\textbf{Objective:}\\
This test exercises a range of table-scan-type queries against the WISE
data. Queries shall be performed against the Object-like table, the
Forced-Source-like table, and against at least one of the Source-like
tables. A range of query result sizes should be exercised, and shall
include at least:

\begin{itemize}
\tightlist
\item
  Queries returning a very small amount of data, fewer than 100 rows,
  and a small subset of columns;
\item
  Queries matching a scaled version of the ``low volume'' query
  definition from the Data Access White Paper; and
\item
  Queries matching a scaled version of the ``high volume'' query
  definition from the Data Access White Paper.
\end{itemize}

The scaling of the ``low volume'' query definition (``50 simultaneous
queries against 10 million objects in the catalog, response 10 sec,
result data set: 0.1 GB'') is based on a assumption that the ``against
10 million objects'' is applied against the O(20 billion) rows
anticipated in the Object table, and that it contemplates reducing the
scope of any non-indexed portion of the WHERE clause of the query to
that fraction of one in ∼ 2000 of the rows in the table. Scaled to the ∼
750 million rows in the WISE Object-like (AllWISE ``Source Catalog'')
table, this would be ∼ 375,000 rows. Similarly scaling the result set
size suggests a result set of ∼ 3.7 MB.\\
Successful completion will be evaluated based on the system's ability to
perform the query at all and to return a result with characteristics
corresponding to plausible estimates or extrap- olations from
scaled-down queries against the IRSA WISE archive. Exact verification
may not be realistic because of the lack of a system capable of
performing the equivalent queries in the production WISE archive.\\
At a later date it may be possible to attempt equivalent queries using a
non-database sys- tem and verify the exact correspondence of results,
but the non-database system does not presently
exist1.\\[3\baselineskip]\textbf{Requirements (to be removed when Reqs
are synchronized from magic draw)}

\begin{itemize}
\tightlist
\item
  DMS-LSP-REQ-0028~
\item
  DMS-LSP-REQ-0029
\end{itemize}
}
\begin{longtable}{p{3cm}p{2.5cm}p{2.5cm}p{3cm}p{4cm}}
\toprule
\href{https://jira.lsstcorp.org/secure/Tests.jspa\#/testCase/LVV-T617}{LVV-T617} & \multicolumn{4}{p{12cm}}{ Verify support for peak volume of moderate-sized queries } \\ \hline
\textbf{Owner} & \textbf{Status} & \textbf{Version} & \textbf{Critical Event} & \textbf{Verification Type} \\ \hline
Jeffrey Carlin & Draft & 1 & false & Test \\ \hline
\end{longtable}
{\scriptsize
\textbf{Objective:}\\
Verify that the LSP can handle a peak usage of 50 simultaneous queries
without degradation, where the queries include input selection of up to
1E7 objects in the catalog, result data set of up to 0.1GB, and a
response time of 10 seconds.
}
  
 \newpage 
\subsection{[LVV-9825] DMS-LSP-REQ-0029-V-01: Peak Volume for Queries on all Objects\_1 }\label{lvv-9825}

\begin{longtable}{cccc}
\hline
\textbf{Jira Link} & \textbf{Assignee} & \textbf{Status} & \textbf{Test Cases}\\ \hline
\href{https://jira.lsstcorp.org/browse/LVV-9825}{LVV-9825} &
Gregory Dubois-Felsmann & Not Covered &
\begin{tabular}{c}
LVV-T4 \\
LVV-T618 \\
\end{tabular}
\\
\hline
\end{longtable}

\textbf{Verification Element Description:} \\
Undefined

{\footnotesize
\begin{longtable}{p{2.5cm}p{13.5cm}}
\hline
\multicolumn{2}{c}{\textbf{Requirement Details}}\\ \hline
Requirement ID & DMS-LSP-REQ-0029 \\ \cdashline{1-2}
Requirement Description &
\begin{minipage}[]{13cm}
The LSP shall handle at peak usage 20 simultaneous queries without
degradation, with the following properties: input selection of up the
entire object database, result data set of up to 6 GB, and a response
time of one hour.
\end{minipage}
\\ \cdashline{1-2}
Requirement Discussion &
\begin{minipage}[]{13cm}
This requirement flows down from several requirements in the DMSR
(\citeds{LSE-61}) which constrain both the performance of the database systems
(via \citeds{LDM-555}) and the Science Platform. In the Science Platform context,
this requirement implies that the LSP will not degrade the performance
supported by the underlying database systems. This applies to queries
originating from any of the Aspects, and covers all of the stages of the
query: query generation, query running, results generation, display of
results, and downloading and saving of results.
\end{minipage}
\\ \cdashline{1-2}
Requirement Priority &  \\ \cdashline{1-2}
Upper Level Requirement &
\begin{tabular}{cl}
\end{tabular}
\\ \hline
\end{longtable}
}


\subsubsection{Test Cases Summary}
\begin{longtable}{p{3cm}p{2.5cm}p{2.5cm}p{3cm}p{4cm}}
\toprule
\href{https://jira.lsstcorp.org/secure/Tests.jspa\#/testCase/LVV-T4}{LVV-T4} & \multicolumn{4}{p{12cm}}{ LSP-00-10: Demonstration of table-scan queries against the WISE data via
API } \\ \hline
\textbf{Owner} & \textbf{Status} & \textbf{Version} & \textbf{Critical Event} & \textbf{Verification Type} \\ \hline
Gregory Dubois-Felsmann & Deprecated & 1 & false & Test \\ \hline
\end{longtable}
{\scriptsize
\textbf{Objective:}\\
This test exercises a range of table-scan-type queries against the WISE
data. Queries shall be performed against the Object-like table, the
Forced-Source-like table, and against at least one of the Source-like
tables. A range of query result sizes should be exercised, and shall
include at least:

\begin{itemize}
\tightlist
\item
  Queries returning a very small amount of data, fewer than 100 rows,
  and a small subset of columns;
\item
  Queries matching a scaled version of the ``low volume'' query
  definition from the Data Access White Paper; and
\item
  Queries matching a scaled version of the ``high volume'' query
  definition from the Data Access White Paper.
\end{itemize}

The scaling of the ``low volume'' query definition (``50 simultaneous
queries against 10 million objects in the catalog, response 10 sec,
result data set: 0.1 GB'') is based on a assumption that the ``against
10 million objects'' is applied against the O(20 billion) rows
anticipated in the Object table, and that it contemplates reducing the
scope of any non-indexed portion of the WHERE clause of the query to
that fraction of one in ∼ 2000 of the rows in the table. Scaled to the ∼
750 million rows in the WISE Object-like (AllWISE ``Source Catalog'')
table, this would be ∼ 375,000 rows. Similarly scaling the result set
size suggests a result set of ∼ 3.7 MB.\\
Successful completion will be evaluated based on the system's ability to
perform the query at all and to return a result with characteristics
corresponding to plausible estimates or extrap- olations from
scaled-down queries against the IRSA WISE archive. Exact verification
may not be realistic because of the lack of a system capable of
performing the equivalent queries in the production WISE archive.\\
At a later date it may be possible to attempt equivalent queries using a
non-database sys- tem and verify the exact correspondence of results,
but the non-database system does not presently
exist1.\\[3\baselineskip]\textbf{Requirements (to be removed when Reqs
are synchronized from magic draw)}

\begin{itemize}
\tightlist
\item
  DMS-LSP-REQ-0028~
\item
  DMS-LSP-REQ-0029
\end{itemize}
}
\begin{longtable}{p{3cm}p{2.5cm}p{2.5cm}p{3cm}p{4cm}}
\toprule
\href{https://jira.lsstcorp.org/secure/Tests.jspa\#/testCase/LVV-T618}{LVV-T618} & \multicolumn{4}{p{12cm}}{ Verify support for peak volume of queries on all Objects } \\ \hline
\textbf{Owner} & \textbf{Status} & \textbf{Version} & \textbf{Critical Event} & \textbf{Verification Type} \\ \hline
Jeffrey Carlin & Draft & 1 & false & Test \\ \hline
\end{longtable}
{\scriptsize
\textbf{Objective:}\\
Verify that the LSP can handle a peak usage of 20 simultaneous queries
without degradation, where the queries include input selection of up to
the entire object database, result data set of up to 6 GB, and a
response time of 1 hour.
}
  
 \newpage 
\subsection{[LVV-9826] DMS-LSP-REQ-0030-V-01: Peak Volume of In-process Queries\_1 }\label{lvv-9826}

\begin{longtable}{cccc}
\hline
\textbf{Jira Link} & \textbf{Assignee} & \textbf{Status} & \textbf{Test Cases}\\ \hline
\href{https://jira.lsstcorp.org/browse/LVV-9826}{LVV-9826} &
Gregory Dubois-Felsmann & Not Covered &
\begin{tabular}{c}
LVV-T619 \\
\end{tabular}
\\
\hline
\end{longtable}

\textbf{Verification Element Description:} \\
Undefined

{\footnotesize
\begin{longtable}{p{2.5cm}p{13.5cm}}
\hline
\multicolumn{2}{c}{\textbf{Requirement Details}}\\ \hline
Requirement ID & DMS-LSP-REQ-0030 \\ \cdashline{1-2}
Requirement Description &
\begin{minipage}[]{13cm}
The LSP shall simultaneously handle at peak usage 20 * 6 GB = 120 GB
downloads
\end{minipage}
\\ \cdashline{1-2}
Requirement Discussion &
\begin{minipage}[]{13cm}
This requirement flows down from several requirements in the DMSR
(\citeds{LSE-61}) which constrain both the performance of the database systems
(via \citeds{LDM-555}) and the Science Platform. In the Science Platform context,
this requirement implies that the LSP will not degrade the performance
supported by the underlying database systems. This applies to queries
originating from any of the Aspects, and covers all of the stages of the
query: query generation, query running, results generation, display of
results, and downloading and saving of results.
\end{minipage}
\\ \cdashline{1-2}
Requirement Priority &  \\ \cdashline{1-2}
Upper Level Requirement &
\begin{tabular}{cl}
\end{tabular}
\\ \hline
\end{longtable}
}


\subsubsection{Test Cases Summary}
\begin{longtable}{p{3cm}p{2.5cm}p{2.5cm}p{3cm}p{4cm}}
\toprule
\href{https://jira.lsstcorp.org/secure/Tests.jspa\#/testCase/LVV-T619}{LVV-T619} & \multicolumn{4}{p{12cm}}{ Verify LSP handles peak volume of queries } \\ \hline
\textbf{Owner} & \textbf{Status} & \textbf{Version} & \textbf{Critical Event} & \textbf{Verification Type} \\ \hline
Jeffrey Carlin & Draft & 1 & false & Test \\ \hline
\end{longtable}
{\scriptsize
\textbf{Objective:}\\
Verify that the LSP can simultaneously handle peak usage of 20*6 GB =
120 GB of downloads.
}
  
 \newpage 
\subsection{[LVV-9827] DMS-LSP-REQ-0031-V-01: Query Result Download Bandwidth\_1 }\label{lvv-9827}

\begin{longtable}{cccc}
\hline
\textbf{Jira Link} & \textbf{Assignee} & \textbf{Status} & \textbf{Test Cases}\\ \hline
\href{https://jira.lsstcorp.org/browse/LVV-9827}{LVV-9827} &
Gregory Dubois-Felsmann & Not Covered &
\begin{tabular}{c}
LVV-T620 \\
\end{tabular}
\\
\hline
\end{longtable}

\textbf{Verification Element Description:} \\
Undefined

{\footnotesize
\begin{longtable}{p{2.5cm}p{13.5cm}}
\hline
\multicolumn{2}{c}{\textbf{Requirement Details}}\\ \hline
Requirement ID & DMS-LSP-REQ-0031 \\ \cdashline{1-2}
Requirement Description &
\begin{minipage}[]{13cm}
The LSP shall support a download rate of 6 Gbps for query results
including results tables and images.
\end{minipage}
\\ \cdashline{1-2}
Requirement Discussion &
\begin{minipage}[]{13cm}
This requirement flows down from several requirements in the DMSR
(\citeds{LSE-61}) which constrain both the performance of the database systems
(via \citeds{LDM-555}) and the Science Platform. In the Science Platform context,
this requirement implies that the LSP will not degrade the performance
supported by the underlying database systems. This applies to queries
originating from any of the Aspects, and covers all of the stages of the
query: query generation, query running, results generation, display of
results, and downloading and saving of results.
\end{minipage}
\\ \cdashline{1-2}
Requirement Priority &  \\ \cdashline{1-2}
Upper Level Requirement &
\begin{tabular}{cl}
\end{tabular}
\\ \hline
\end{longtable}
}


\subsubsection{Test Cases Summary}
\begin{longtable}{p{3cm}p{2.5cm}p{2.5cm}p{3cm}p{4cm}}
\toprule
\href{https://jira.lsstcorp.org/secure/Tests.jspa\#/testCase/LVV-T620}{LVV-T620} & \multicolumn{4}{p{12cm}}{ Verify LSP supports required download bandwidth } \\ \hline
\textbf{Owner} & \textbf{Status} & \textbf{Version} & \textbf{Critical Event} & \textbf{Verification Type} \\ \hline
Jeffrey Carlin & Draft & 1 & false & Test \\ \hline
\end{longtable}
{\scriptsize
\textbf{Objective:}\\
Verify that the LSP supports a download rate of at least 6 Gbps for
query results including tables and images.
}
  
 \newpage 
\subsection{[LVV-9828] DMS-LSP-REQ-0019-V-01: Documentation\_1 }\label{lvv-9828}

\begin{longtable}{cccc}
\hline
\textbf{Jira Link} & \textbf{Assignee} & \textbf{Status} & \textbf{Test Cases}\\ \hline
\href{https://jira.lsstcorp.org/browse/LVV-9828}{LVV-9828} &
Gregory Dubois-Felsmann & Not Covered &
\begin{tabular}{c}
LVV-T621 \\
\end{tabular}
\\
\hline
\end{longtable}

\textbf{Verification Element Description:} \\
Undefined

{\footnotesize
\begin{longtable}{p{2.5cm}p{13.5cm}}
\hline
\multicolumn{2}{c}{\textbf{Requirement Details}}\\ \hline
Requirement ID & DMS-LSP-REQ-0019 \\ \cdashline{1-2}
Requirement Description &
\begin{minipage}[]{13cm}
The LSP shall provide user and reference documentation for all its
aspects.
\end{minipage}
\\ \cdashline{1-2}
Requirement Priority &  \\ \cdashline{1-2}
Upper Level Requirement &
\begin{tabular}{cl}
\end{tabular}
\\ \hline
\end{longtable}
}


\subsubsection{Test Cases Summary}
\begin{longtable}{p{3cm}p{2.5cm}p{2.5cm}p{3cm}p{4cm}}
\toprule
\href{https://jira.lsstcorp.org/secure/Tests.jspa\#/testCase/LVV-T621}{LVV-T621} & \multicolumn{4}{p{12cm}}{ Verify LSP user reference and documentation } \\ \hline
\textbf{Owner} & \textbf{Status} & \textbf{Version} & \textbf{Critical Event} & \textbf{Verification Type} \\ \hline
Jeffrey Carlin & Draft & 1 & false & Inspection \\ \hline
\end{longtable}
{\scriptsize
\textbf{Objective:}\\
Verify that the LSP provides user reference and documentation for all of
its aspects.
}
  
 \newpage 
\subsection{[LVV-9829] DMS-LSP-REQ-0025-V-01: Acceptable Use Policy\_1 }\label{lvv-9829}

\begin{longtable}{cccc}
\hline
\textbf{Jira Link} & \textbf{Assignee} & \textbf{Status} & \textbf{Test Cases}\\ \hline
\href{https://jira.lsstcorp.org/browse/LVV-9829}{LVV-9829} &
Gregory Dubois-Felsmann & Not Covered &
\begin{tabular}{c}
LVV-T627 \\
\end{tabular}
\\
\hline
\end{longtable}

\textbf{Verification Element Description:} \\
Undefined

{\footnotesize
\begin{longtable}{p{2.5cm}p{13.5cm}}
\hline
\multicolumn{2}{c}{\textbf{Requirement Details}}\\ \hline
Requirement ID & DMS-LSP-REQ-0025 \\ \cdashline{1-2}
Requirement Description &
\begin{minipage}[]{13cm}
Non-project-staff users of the LSP shall be required to agree to and
abide by an Acceptable Use Policy, to be determined by the LSST project
or its operations organization, as a condition of access to any Project
instance of the LSP.
\end{minipage}
\\ \cdashline{1-2}
Requirement Discussion &
\begin{minipage}[]{13cm}
The policies for project staff are set in other documents.
\end{minipage}
\\ \cdashline{1-2}
Requirement Priority &  \\ \cdashline{1-2}
Upper Level Requirement &
\begin{tabular}{cl}
\end{tabular}
\\ \hline
\end{longtable}
}


\subsubsection{Test Cases Summary}
\begin{longtable}{p{3cm}p{2.5cm}p{2.5cm}p{3cm}p{4cm}}
\toprule
\href{https://jira.lsstcorp.org/secure/Tests.jspa\#/testCase/LVV-T627}{LVV-T627} & \multicolumn{4}{p{12cm}}{ Verify implementation of Acceptable Use Policy } \\ \hline
\textbf{Owner} & \textbf{Status} & \textbf{Version} & \textbf{Critical Event} & \textbf{Verification Type} \\ \hline
Jeffrey Carlin & Draft & 1 & false & Inspection \\ \hline
\end{longtable}
{\scriptsize
\textbf{Objective:}\\
Verify that non-Project users of the LSP are required to agree to and
abide by an Acceptable Use Policy.
}
  
 \newpage 
\subsection{[LVV-9830] DMS-LSP-REQ-0020-V-01: Authenticated User Access\_1 }\label{lvv-9830}

\begin{longtable}{cccc}
\hline
\textbf{Jira Link} & \textbf{Assignee} & \textbf{Status} & \textbf{Test Cases}\\ \hline
\href{https://jira.lsstcorp.org/browse/LVV-9830}{LVV-9830} &
Gregory Dubois-Felsmann & Not Covered &
\begin{tabular}{c}
LVV-T622 \\
LVV-T1334 \\
LVV-T1436 \\
LVV-T1437 \\
\end{tabular}
\\
\hline
\end{longtable}

\textbf{Verification Element Description:} \\
Undefined

{\footnotesize
\begin{longtable}{p{2.5cm}p{13.5cm}}
\hline
\multicolumn{2}{c}{\textbf{Requirement Details}}\\ \hline
Requirement ID & DMS-LSP-REQ-0020 \\ \cdashline{1-2}
Requirement Description &
\begin{minipage}[]{13cm}
The functions and services of the LSP, including all three aspects,
shall be available only to authenticated users, except where other
requirements or other change-controlled specifications authorize or
mandate otherwise.
\end{minipage}
\\ \cdashline{1-2}
Requirement Priority &  \\ \cdashline{1-2}
Upper Level Requirement &
\begin{tabular}{cl}
\end{tabular}
\\ \hline
\end{longtable}
}


\subsubsection{Test Cases Summary}
\begin{longtable}{p{3cm}p{2.5cm}p{2.5cm}p{3cm}p{4cm}}
\toprule
\href{https://jira.lsstcorp.org/secure/Tests.jspa\#/testCase/LVV-T622}{LVV-T622} & \multicolumn{4}{p{12cm}}{ Verify LSP only available to authenticated users } \\ \hline
\textbf{Owner} & \textbf{Status} & \textbf{Version} & \textbf{Critical Event} & \textbf{Verification Type} \\ \hline
Jeffrey Carlin & Defined & 1 & false & Inspection \\ \hline
\end{longtable}
{\scriptsize
\textbf{Objective:}\\
Verify that the functions and services of all three aspects of the LSP
are accessible only to authenticated users.
}
\begin{longtable}{p{3cm}p{2.5cm}p{2.5cm}p{3cm}p{4cm}}
\toprule
\href{https://jira.lsstcorp.org/secure/Tests.jspa\#/testCase/LVV-T1334}{LVV-T1334} & \multicolumn{4}{p{12cm}}{ LDM-503-10a: Portal Aspect tests for LSP with Authentication and TAP
milestone } \\ \hline
\textbf{Owner} & \textbf{Status} & \textbf{Version} & \textbf{Critical Event} & \textbf{Verification Type} \\ \hline
Gregory Dubois-Felsmann & Defined & 1 & false & Test \\ \hline
\end{longtable}
{\scriptsize
\textbf{Objective:}\\
This test case verifies that the Portal Aspect of the Science Platform
is accessible to authorized users through a login process, and that TAP
searches can be performed from the Portal Aspect UI.\\[2\baselineskip]In
so doing and in conjunction with the other LDM-503-10a test cases
collected under LVV-P48, it addresses all or part of the following
requirements:

\begin{itemize}
\tightlist
\item
  DMS-LSP-REQ-0002, DMS-LSP-REQ-0005, DMS-LSP-REQ-0006,
  DMS-LSP-REQ-0020, DMS-LSP-REQ-0022, DMS-LSP-REQ-0023, DMS-LSP-REQ-0024
\item
  DMS-PRTL-REQ-0001, DMS-PRTL-REQ-0015, DMS-PRTL-REQ-0016,
  DMS-PRTL-REQ-0017, DMS-PRTL-REQ-0020, DMS-PRTL-REQ-0026,
  DMS-PRTL-REQ-0049, and DMS-PRTL-REQ-0095, primarily
\end{itemize}

Note this test was not designed to perform a full verification of the
above requirements, but rather to demonstrate having reached a certain
level of partial capability during construction.
}
\begin{longtable}{p{3cm}p{2.5cm}p{2.5cm}p{3cm}p{4cm}}
\toprule
\href{https://jira.lsstcorp.org/secure/Tests.jspa\#/testCase/LVV-T1436}{LVV-T1436} & \multicolumn{4}{p{12cm}}{ LDM-503-10a: Notebook Aspect tests for LSP with Authentication and TAP
milestone } \\ \hline
\textbf{Owner} & \textbf{Status} & \textbf{Version} & \textbf{Critical Event} & \textbf{Verification Type} \\ \hline
Gregory Dubois-Felsmann & Defined & 1 & false & Test \\ \hline
\end{longtable}
{\scriptsize
\textbf{Objective:}\\
This test case verifies that the Notebook Aspect of the Science Platform
is accessible to authorized users through a login process, and that TAP
searches can be performed from Python code in the Notebook
Aspect.\\[2\baselineskip]In so doing and in conjunction with the other
LDM-503-10a test cases collected under LVV-P48, it addresses all or part
of the following requirements:

\begin{itemize}
\tightlist
\item
  DMS-LSP-REQ-0003, DMS-LSP-REQ-0005, DMS-LSP-REQ-0006,
  DMS-LSP-REQ-0020, DMS-LSP-REQ-0022, DMS-LSP-REQ-0023, DMS-LSP-REQ-0024
\item
  DMS-NB-REQ-0001, DMS-NB-REQ-0002, DMS-NB-REQ-0005, DMS-NB-REQ-0006,
  DMS-NB-REQ-0013, DMS-NB-REQ-0017, and DMS-NB-REQ-0029, primarily
\end{itemize}

Note this test was not designed to perform a full verification of the
above requirements, but rather to demonstrate having reached a certain
level of partial capability during construction.
}
\begin{longtable}{p{3cm}p{2.5cm}p{2.5cm}p{3cm}p{4cm}}
\toprule
\href{https://jira.lsstcorp.org/secure/Tests.jspa\#/testCase/LVV-T1437}{LVV-T1437} & \multicolumn{4}{p{12cm}}{ LDM-503-10a: API Aspect tests for LSP with Authentication and TAP
milestone } \\ \hline
\textbf{Owner} & \textbf{Status} & \textbf{Version} & \textbf{Critical Event} & \textbf{Verification Type} \\ \hline
Gregory Dubois-Felsmann & Defined & 1 & false & Test \\ \hline
\end{longtable}
{\scriptsize
\textbf{Objective:}\\
This test case verifies that the TAP service in the API Aspect of the
Science Platform is accessible to authorized users through a login
process, and that TAP searches can be performed using the IVOA TAP
protocol from remote sites.\\[2\baselineskip]In so doing and in
conjunction with the other LDM-503-10a test cases collected under
LVV-P48, it addresses all or part of the following requirements:

\begin{itemize}
\tightlist
\item
  DMS-LSP-REQ-0004, DMS-LSP-REQ-0005, DMS-LSP-REQ-0006,
  DMS-LSP-REQ-0020, DMS-LSP-REQ-0022, DMS-LSP-REQ-0023, DMS-LSP-REQ-0024
\item
  DMS-API-REQ-0003, DMS-API-REQ-0004, DMS-API-REQ-0006,
  DMS-API-REQ-0007, DMS-API-REQ-0008, DMS-API-REQ-0009,
  DMS-API-REQ-0023, and DMS-API-REQ-0039, primarily
\end{itemize}

Note this test was not designed to perform a full verification of the
above requirements, but rather to demonstrate having reached a certain
level of partial capability during construction.
}
  
 \newpage 
\subsection{[LVV-9831] DMS-LSP-REQ-0022-V-01: Common Identity\_1 }\label{lvv-9831}

\begin{longtable}{cccc}
\hline
\textbf{Jira Link} & \textbf{Assignee} & \textbf{Status} & \textbf{Test Cases}\\ \hline
\href{https://jira.lsstcorp.org/browse/LVV-9831}{LVV-9831} &
Gregory Dubois-Felsmann & Not Covered &
\begin{tabular}{c}
LVV-T624 \\
LVV-T1334 \\
LVV-T1436 \\
LVV-T1437 \\
\end{tabular}
\\
\hline
\end{longtable}

\textbf{Verification Element Description:} \\
Undefined

{\footnotesize
\begin{longtable}{p{2.5cm}p{13.5cm}}
\hline
\multicolumn{2}{c}{\textbf{Requirement Details}}\\ \hline
Requirement ID & DMS-LSP-REQ-0022 \\ \cdashline{1-2}
Requirement Description &
\begin{minipage}[]{13cm}
A user shall be able to use the same credentials to authenticate to all
aspects of the LSP, and to receive access to any personal data or other
state that is available cross-aspects.
\end{minipage}
\\ \cdashline{1-2}
Requirement Discussion &
\begin{minipage}[]{13cm}
This does not explicitly mandate ``strong single-sign-on'' in the sense
that someone who has logged in to the Portal aspect can then proceed to
the Notebook aspect without a separate login. (TBR: This behavior would
be highly desirable and we may wish to adopt this as a requirement after
all.) It does require ``weak single-sign-on'' - the same credentials
work everywhere.
\end{minipage}
\\ \cdashline{1-2}
Requirement Priority &  \\ \cdashline{1-2}
Upper Level Requirement &
\begin{tabular}{cl}
\end{tabular}
\\ \hline
\end{longtable}
}


\subsubsection{Test Cases Summary}
\begin{longtable}{p{3cm}p{2.5cm}p{2.5cm}p{3cm}p{4cm}}
\toprule
\href{https://jira.lsstcorp.org/secure/Tests.jspa\#/testCase/LVV-T624}{LVV-T624} & \multicolumn{4}{p{12cm}}{ Verify implementation of common identity across LSP aspects } \\ \hline
\textbf{Owner} & \textbf{Status} & \textbf{Version} & \textbf{Critical Event} & \textbf{Verification Type} \\ \hline
Leanne Guy & Draft & 1 & false & Inspection \\ \hline
\end{longtable}
{\scriptsize
\textbf{Objective:}\\
Verify that users can authenticate and access all three aspects of the
LSP using the same credentials.
}
\begin{longtable}{p{3cm}p{2.5cm}p{2.5cm}p{3cm}p{4cm}}
\toprule
\href{https://jira.lsstcorp.org/secure/Tests.jspa\#/testCase/LVV-T1334}{LVV-T1334} & \multicolumn{4}{p{12cm}}{ LDM-503-10a: Portal Aspect tests for LSP with Authentication and TAP
milestone } \\ \hline
\textbf{Owner} & \textbf{Status} & \textbf{Version} & \textbf{Critical Event} & \textbf{Verification Type} \\ \hline
Gregory Dubois-Felsmann & Defined & 1 & false & Test \\ \hline
\end{longtable}
{\scriptsize
\textbf{Objective:}\\
This test case verifies that the Portal Aspect of the Science Platform
is accessible to authorized users through a login process, and that TAP
searches can be performed from the Portal Aspect UI.\\[2\baselineskip]In
so doing and in conjunction with the other LDM-503-10a test cases
collected under LVV-P48, it addresses all or part of the following
requirements:

\begin{itemize}
\tightlist
\item
  DMS-LSP-REQ-0002, DMS-LSP-REQ-0005, DMS-LSP-REQ-0006,
  DMS-LSP-REQ-0020, DMS-LSP-REQ-0022, DMS-LSP-REQ-0023, DMS-LSP-REQ-0024
\item
  DMS-PRTL-REQ-0001, DMS-PRTL-REQ-0015, DMS-PRTL-REQ-0016,
  DMS-PRTL-REQ-0017, DMS-PRTL-REQ-0020, DMS-PRTL-REQ-0026,
  DMS-PRTL-REQ-0049, and DMS-PRTL-REQ-0095, primarily
\end{itemize}

Note this test was not designed to perform a full verification of the
above requirements, but rather to demonstrate having reached a certain
level of partial capability during construction.
}
\begin{longtable}{p{3cm}p{2.5cm}p{2.5cm}p{3cm}p{4cm}}
\toprule
\href{https://jira.lsstcorp.org/secure/Tests.jspa\#/testCase/LVV-T1436}{LVV-T1436} & \multicolumn{4}{p{12cm}}{ LDM-503-10a: Notebook Aspect tests for LSP with Authentication and TAP
milestone } \\ \hline
\textbf{Owner} & \textbf{Status} & \textbf{Version} & \textbf{Critical Event} & \textbf{Verification Type} \\ \hline
Gregory Dubois-Felsmann & Defined & 1 & false & Test \\ \hline
\end{longtable}
{\scriptsize
\textbf{Objective:}\\
This test case verifies that the Notebook Aspect of the Science Platform
is accessible to authorized users through a login process, and that TAP
searches can be performed from Python code in the Notebook
Aspect.\\[2\baselineskip]In so doing and in conjunction with the other
LDM-503-10a test cases collected under LVV-P48, it addresses all or part
of the following requirements:

\begin{itemize}
\tightlist
\item
  DMS-LSP-REQ-0003, DMS-LSP-REQ-0005, DMS-LSP-REQ-0006,
  DMS-LSP-REQ-0020, DMS-LSP-REQ-0022, DMS-LSP-REQ-0023, DMS-LSP-REQ-0024
\item
  DMS-NB-REQ-0001, DMS-NB-REQ-0002, DMS-NB-REQ-0005, DMS-NB-REQ-0006,
  DMS-NB-REQ-0013, DMS-NB-REQ-0017, and DMS-NB-REQ-0029, primarily
\end{itemize}

Note this test was not designed to perform a full verification of the
above requirements, but rather to demonstrate having reached a certain
level of partial capability during construction.
}
\begin{longtable}{p{3cm}p{2.5cm}p{2.5cm}p{3cm}p{4cm}}
\toprule
\href{https://jira.lsstcorp.org/secure/Tests.jspa\#/testCase/LVV-T1437}{LVV-T1437} & \multicolumn{4}{p{12cm}}{ LDM-503-10a: API Aspect tests for LSP with Authentication and TAP
milestone } \\ \hline
\textbf{Owner} & \textbf{Status} & \textbf{Version} & \textbf{Critical Event} & \textbf{Verification Type} \\ \hline
Gregory Dubois-Felsmann & Defined & 1 & false & Test \\ \hline
\end{longtable}
{\scriptsize
\textbf{Objective:}\\
This test case verifies that the TAP service in the API Aspect of the
Science Platform is accessible to authorized users through a login
process, and that TAP searches can be performed using the IVOA TAP
protocol from remote sites.\\[2\baselineskip]In so doing and in
conjunction with the other LDM-503-10a test cases collected under
LVV-P48, it addresses all or part of the following requirements:

\begin{itemize}
\tightlist
\item
  DMS-LSP-REQ-0004, DMS-LSP-REQ-0005, DMS-LSP-REQ-0006,
  DMS-LSP-REQ-0020, DMS-LSP-REQ-0022, DMS-LSP-REQ-0023, DMS-LSP-REQ-0024
\item
  DMS-API-REQ-0003, DMS-API-REQ-0004, DMS-API-REQ-0006,
  DMS-API-REQ-0007, DMS-API-REQ-0008, DMS-API-REQ-0009,
  DMS-API-REQ-0023, and DMS-API-REQ-0039, primarily
\end{itemize}

Note this test was not designed to perform a full verification of the
above requirements, but rather to demonstrate having reached a certain
level of partial capability during construction.
}
  
 \newpage 
\subsection{[LVV-9832] DMS-LSP-REQ-0021-V-01: New-user Support\_1 }\label{lvv-9832}

\begin{longtable}{cccc}
\hline
\textbf{Jira Link} & \textbf{Assignee} & \textbf{Status} & \textbf{Test Cases}\\ \hline
\href{https://jira.lsstcorp.org/browse/LVV-9832}{LVV-9832} &
Gregory Dubois-Felsmann & Not Covered &
\begin{tabular}{c}
LVV-T623 \\
\end{tabular}
\\
\hline
\end{longtable}

\textbf{Verification Element Description:} \\
Undefined

{\footnotesize
\begin{longtable}{p{2.5cm}p{13.5cm}}
\hline
\multicolumn{2}{c}{\textbf{Requirement Details}}\\ \hline
Requirement ID & DMS-LSP-REQ-0021 \\ \cdashline{1-2}
Requirement Description &
\begin{minipage}[]{13cm}
The Portal and Notebook aspects shall provide guidance to
unauthenticated users as to how to establish an identity as usable for
authentication to the LSP.
\end{minipage}
\\ \cdashline{1-2}
Requirement Discussion &
\begin{minipage}[]{13cm}
This could be as simple as a link to a ``register for LSST access'' page
from the login screens of the Portal and Notebook.
\end{minipage}
\\ \cdashline{1-2}
Requirement Priority &  \\ \cdashline{1-2}
Upper Level Requirement &
\begin{tabular}{cl}
\end{tabular}
\\ \hline
\end{longtable}
}


\subsubsection{Test Cases Summary}
\begin{longtable}{p{3cm}p{2.5cm}p{2.5cm}p{3cm}p{4cm}}
\toprule
\href{https://jira.lsstcorp.org/secure/Tests.jspa\#/testCase/LVV-T623}{LVV-T623} & \multicolumn{4}{p{12cm}}{ Verify support for new LSP users } \\ \hline
\textbf{Owner} & \textbf{Status} & \textbf{Version} & \textbf{Critical Event} & \textbf{Verification Type} \\ \hline
Jeffrey Carlin & Draft & 1 & false & Inspection \\ \hline
\end{longtable}
{\scriptsize
\textbf{Objective:}\\
Verify that guidance is provided to new users about how to become
authenticated users of the LSP.
}
  
 \newpage 
\subsection{[LVV-9833] DMS-LSP-REQ-0027-V-01: Privacy of User Activities\_1 }\label{lvv-9833}

\begin{longtable}{cccc}
\hline
\textbf{Jira Link} & \textbf{Assignee} & \textbf{Status} & \textbf{Test Cases}\\ \hline
\href{https://jira.lsstcorp.org/browse/LVV-9833}{LVV-9833} &
Gregory Dubois-Felsmann & Not Covered &
\begin{tabular}{c}
LVV-T629 \\
\end{tabular}
\\
\hline
\end{longtable}

\textbf{Verification Element Description:} \\
Undefined

{\footnotesize
\begin{longtable}{p{2.5cm}p{13.5cm}}
\hline
\multicolumn{2}{c}{\textbf{Requirement Details}}\\ \hline
Requirement ID & DMS-LSP-REQ-0027 \\ \cdashline{1-2}
Requirement Description &
\begin{minipage}[]{13cm}
The LSP shall ensure that a user's activities on the LSP are not visible
to other users without the originating user's explicit authorization.
\end{minipage}
\\ \cdashline{1-2}
Requirement Discussion &
\begin{minipage}[]{13cm}
Users will expect, for instance, that the queries they perform are not
revealed to other users without their consent.
\end{minipage}
\\ \cdashline{1-2}
Requirement Priority &  \\ \cdashline{1-2}
Upper Level Requirement &
\begin{tabular}{cl}
\end{tabular}
\\ \hline
\end{longtable}
}


\subsubsection{Test Cases Summary}
\begin{longtable}{p{3cm}p{2.5cm}p{2.5cm}p{3cm}p{4cm}}
\toprule
\href{https://jira.lsstcorp.org/secure/Tests.jspa\#/testCase/LVV-T629}{LVV-T629} & \multicolumn{4}{p{12cm}}{ Verify privacy of users' activities } \\ \hline
\textbf{Owner} & \textbf{Status} & \textbf{Version} & \textbf{Critical Event} & \textbf{Verification Type} \\ \hline
Jeffrey Carlin & Draft & 1 & false & Inspection \\ \hline
\end{longtable}
{\scriptsize
\textbf{Objective:}\\
Verify that users' activities on the LSP are not visible to other users
without the originating user's explicit permission.
}
  
 \newpage 
\subsection{[LVV-9834] DMS-LSP-REQ-0023-V-01: Use of External Identity Providers\_1 }\label{lvv-9834}

\begin{longtable}{cccc}
\hline
\textbf{Jira Link} & \textbf{Assignee} & \textbf{Status} & \textbf{Test Cases}\\ \hline
\href{https://jira.lsstcorp.org/browse/LVV-9834}{LVV-9834} &
Gregory Dubois-Felsmann & Not Covered &
\begin{tabular}{c}
LVV-T625 \\
LVV-T1334 \\
LVV-T1436 \\
LVV-T1437 \\
\end{tabular}
\\
\hline
\end{longtable}

\textbf{Verification Element Description:} \\
Undefined

{\footnotesize
\begin{longtable}{p{2.5cm}p{13.5cm}}
\hline
\multicolumn{2}{c}{\textbf{Requirement Details}}\\ \hline
Requirement ID & DMS-LSP-REQ-0023 \\ \cdashline{1-2}
Requirement Description &
\begin{minipage}[]{13cm}
The LSP shall permit users to authenticate to the system using external
credentials, from identity providers determined to be trusted by the
LSST project or its operations organization.
\end{minipage}
\\ \cdashline{1-2}
Requirement Discussion &
\begin{minipage}[]{13cm}
This means that a user should be able to authenticate to an instance of
the LSP using, for example, Github credentials, or credentials from a
home institution.\\
The policies for how LSST determines that an external user, with
external credentials, has data rights and may establish an identity in
LSST systems are set forth in other documents.
\end{minipage}
\\ \cdashline{1-2}
Requirement Priority &  \\ \cdashline{1-2}
Upper Level Requirement &
\begin{tabular}{cl}
\end{tabular}
\\ \hline
\end{longtable}
}


\subsubsection{Test Cases Summary}
\begin{longtable}{p{3cm}p{2.5cm}p{2.5cm}p{3cm}p{4cm}}
\toprule
\href{https://jira.lsstcorp.org/secure/Tests.jspa\#/testCase/LVV-T625}{LVV-T625} & \multicolumn{4}{p{12cm}}{ Verify authentication via external identity providers } \\ \hline
\textbf{Owner} & \textbf{Status} & \textbf{Version} & \textbf{Critical Event} & \textbf{Verification Type} \\ \hline
Leanne Guy & Draft & 1 & false & Inspection \\ \hline
\end{longtable}
{\scriptsize
\textbf{Objective:}\\
Verify that LSP users can be authenticated using external credentials
from trusted identity providers.
}
\begin{longtable}{p{3cm}p{2.5cm}p{2.5cm}p{3cm}p{4cm}}
\toprule
\href{https://jira.lsstcorp.org/secure/Tests.jspa\#/testCase/LVV-T1334}{LVV-T1334} & \multicolumn{4}{p{12cm}}{ LDM-503-10a: Portal Aspect tests for LSP with Authentication and TAP
milestone } \\ \hline
\textbf{Owner} & \textbf{Status} & \textbf{Version} & \textbf{Critical Event} & \textbf{Verification Type} \\ \hline
Gregory Dubois-Felsmann & Defined & 1 & false & Test \\ \hline
\end{longtable}
{\scriptsize
\textbf{Objective:}\\
This test case verifies that the Portal Aspect of the Science Platform
is accessible to authorized users through a login process, and that TAP
searches can be performed from the Portal Aspect UI.\\[2\baselineskip]In
so doing and in conjunction with the other LDM-503-10a test cases
collected under LVV-P48, it addresses all or part of the following
requirements:

\begin{itemize}
\tightlist
\item
  DMS-LSP-REQ-0002, DMS-LSP-REQ-0005, DMS-LSP-REQ-0006,
  DMS-LSP-REQ-0020, DMS-LSP-REQ-0022, DMS-LSP-REQ-0023, DMS-LSP-REQ-0024
\item
  DMS-PRTL-REQ-0001, DMS-PRTL-REQ-0015, DMS-PRTL-REQ-0016,
  DMS-PRTL-REQ-0017, DMS-PRTL-REQ-0020, DMS-PRTL-REQ-0026,
  DMS-PRTL-REQ-0049, and DMS-PRTL-REQ-0095, primarily
\end{itemize}

Note this test was not designed to perform a full verification of the
above requirements, but rather to demonstrate having reached a certain
level of partial capability during construction.
}
\begin{longtable}{p{3cm}p{2.5cm}p{2.5cm}p{3cm}p{4cm}}
\toprule
\href{https://jira.lsstcorp.org/secure/Tests.jspa\#/testCase/LVV-T1436}{LVV-T1436} & \multicolumn{4}{p{12cm}}{ LDM-503-10a: Notebook Aspect tests for LSP with Authentication and TAP
milestone } \\ \hline
\textbf{Owner} & \textbf{Status} & \textbf{Version} & \textbf{Critical Event} & \textbf{Verification Type} \\ \hline
Gregory Dubois-Felsmann & Defined & 1 & false & Test \\ \hline
\end{longtable}
{\scriptsize
\textbf{Objective:}\\
This test case verifies that the Notebook Aspect of the Science Platform
is accessible to authorized users through a login process, and that TAP
searches can be performed from Python code in the Notebook
Aspect.\\[2\baselineskip]In so doing and in conjunction with the other
LDM-503-10a test cases collected under LVV-P48, it addresses all or part
of the following requirements:

\begin{itemize}
\tightlist
\item
  DMS-LSP-REQ-0003, DMS-LSP-REQ-0005, DMS-LSP-REQ-0006,
  DMS-LSP-REQ-0020, DMS-LSP-REQ-0022, DMS-LSP-REQ-0023, DMS-LSP-REQ-0024
\item
  DMS-NB-REQ-0001, DMS-NB-REQ-0002, DMS-NB-REQ-0005, DMS-NB-REQ-0006,
  DMS-NB-REQ-0013, DMS-NB-REQ-0017, and DMS-NB-REQ-0029, primarily
\end{itemize}

Note this test was not designed to perform a full verification of the
above requirements, but rather to demonstrate having reached a certain
level of partial capability during construction.
}
\begin{longtable}{p{3cm}p{2.5cm}p{2.5cm}p{3cm}p{4cm}}
\toprule
\href{https://jira.lsstcorp.org/secure/Tests.jspa\#/testCase/LVV-T1437}{LVV-T1437} & \multicolumn{4}{p{12cm}}{ LDM-503-10a: API Aspect tests for LSP with Authentication and TAP
milestone } \\ \hline
\textbf{Owner} & \textbf{Status} & \textbf{Version} & \textbf{Critical Event} & \textbf{Verification Type} \\ \hline
Gregory Dubois-Felsmann & Defined & 1 & false & Test \\ \hline
\end{longtable}
{\scriptsize
\textbf{Objective:}\\
This test case verifies that the TAP service in the API Aspect of the
Science Platform is accessible to authorized users through a login
process, and that TAP searches can be performed using the IVOA TAP
protocol from remote sites.\\[2\baselineskip]In so doing and in
conjunction with the other LDM-503-10a test cases collected under
LVV-P48, it addresses all or part of the following requirements:

\begin{itemize}
\tightlist
\item
  DMS-LSP-REQ-0004, DMS-LSP-REQ-0005, DMS-LSP-REQ-0006,
  DMS-LSP-REQ-0020, DMS-LSP-REQ-0022, DMS-LSP-REQ-0023, DMS-LSP-REQ-0024
\item
  DMS-API-REQ-0003, DMS-API-REQ-0004, DMS-API-REQ-0006,
  DMS-API-REQ-0007, DMS-API-REQ-0008, DMS-API-REQ-0009,
  DMS-API-REQ-0023, and DMS-API-REQ-0039, primarily
\end{itemize}

Note this test was not designed to perform a full verification of the
above requirements, but rather to demonstrate having reached a certain
level of partial capability during construction.
}
  
 \newpage 
\subsection{[LVV-9835] DMS-LSP-REQ-0024-V-01: Use of Multiple Sets of Credentials\_1 }\label{lvv-9835}

\begin{longtable}{cccc}
\hline
\textbf{Jira Link} & \textbf{Assignee} & \textbf{Status} & \textbf{Test Cases}\\ \hline
\href{https://jira.lsstcorp.org/browse/LVV-9835}{LVV-9835} &
Gregory Dubois-Felsmann & Not Covered &
\begin{tabular}{c}
LVV-T626 \\
LVV-T1334 \\
LVV-T1436 \\
LVV-T1437 \\
\end{tabular}
\\
\hline
\end{longtable}

\textbf{Verification Element Description:} \\
Undefined

{\footnotesize
\begin{longtable}{p{2.5cm}p{13.5cm}}
\hline
\multicolumn{2}{c}{\textbf{Requirement Details}}\\ \hline
Requirement ID & DMS-LSP-REQ-0024 \\ \cdashline{1-2}
Requirement Description &
\begin{minipage}[]{13cm}
The LSP shall permit users to associate multiple sets of credentials,
from different providers, with the same identity within the LSP.
\end{minipage}
\\ \cdashline{1-2}
Requirement Priority &  \\ \cdashline{1-2}
Upper Level Requirement &
\begin{tabular}{cl}
\end{tabular}
\\ \hline
\end{longtable}
}


\subsubsection{Test Cases Summary}
\begin{longtable}{p{3cm}p{2.5cm}p{2.5cm}p{3cm}p{4cm}}
\toprule
\href{https://jira.lsstcorp.org/secure/Tests.jspa\#/testCase/LVV-T626}{LVV-T626} & \multicolumn{4}{p{12cm}}{ Verify LSP identity can have multiple associated credentials } \\ \hline
\textbf{Owner} & \textbf{Status} & \textbf{Version} & \textbf{Critical Event} & \textbf{Verification Type} \\ \hline
Jeffrey Carlin & Draft & 1 & false & Inspection \\ \hline
\end{longtable}
{\scriptsize
\textbf{Objective:}\\
Verify that an LSP user can have multiple credentials, from different
providers, associated with the same identity within the LSP.
}
\begin{longtable}{p{3cm}p{2.5cm}p{2.5cm}p{3cm}p{4cm}}
\toprule
\href{https://jira.lsstcorp.org/secure/Tests.jspa\#/testCase/LVV-T1334}{LVV-T1334} & \multicolumn{4}{p{12cm}}{ LDM-503-10a: Portal Aspect tests for LSP with Authentication and TAP
milestone } \\ \hline
\textbf{Owner} & \textbf{Status} & \textbf{Version} & \textbf{Critical Event} & \textbf{Verification Type} \\ \hline
Gregory Dubois-Felsmann & Defined & 1 & false & Test \\ \hline
\end{longtable}
{\scriptsize
\textbf{Objective:}\\
This test case verifies that the Portal Aspect of the Science Platform
is accessible to authorized users through a login process, and that TAP
searches can be performed from the Portal Aspect UI.\\[2\baselineskip]In
so doing and in conjunction with the other LDM-503-10a test cases
collected under LVV-P48, it addresses all or part of the following
requirements:

\begin{itemize}
\tightlist
\item
  DMS-LSP-REQ-0002, DMS-LSP-REQ-0005, DMS-LSP-REQ-0006,
  DMS-LSP-REQ-0020, DMS-LSP-REQ-0022, DMS-LSP-REQ-0023, DMS-LSP-REQ-0024
\item
  DMS-PRTL-REQ-0001, DMS-PRTL-REQ-0015, DMS-PRTL-REQ-0016,
  DMS-PRTL-REQ-0017, DMS-PRTL-REQ-0020, DMS-PRTL-REQ-0026,
  DMS-PRTL-REQ-0049, and DMS-PRTL-REQ-0095, primarily
\end{itemize}

Note this test was not designed to perform a full verification of the
above requirements, but rather to demonstrate having reached a certain
level of partial capability during construction.
}
\begin{longtable}{p{3cm}p{2.5cm}p{2.5cm}p{3cm}p{4cm}}
\toprule
\href{https://jira.lsstcorp.org/secure/Tests.jspa\#/testCase/LVV-T1436}{LVV-T1436} & \multicolumn{4}{p{12cm}}{ LDM-503-10a: Notebook Aspect tests for LSP with Authentication and TAP
milestone } \\ \hline
\textbf{Owner} & \textbf{Status} & \textbf{Version} & \textbf{Critical Event} & \textbf{Verification Type} \\ \hline
Gregory Dubois-Felsmann & Defined & 1 & false & Test \\ \hline
\end{longtable}
{\scriptsize
\textbf{Objective:}\\
This test case verifies that the Notebook Aspect of the Science Platform
is accessible to authorized users through a login process, and that TAP
searches can be performed from Python code in the Notebook
Aspect.\\[2\baselineskip]In so doing and in conjunction with the other
LDM-503-10a test cases collected under LVV-P48, it addresses all or part
of the following requirements:

\begin{itemize}
\tightlist
\item
  DMS-LSP-REQ-0003, DMS-LSP-REQ-0005, DMS-LSP-REQ-0006,
  DMS-LSP-REQ-0020, DMS-LSP-REQ-0022, DMS-LSP-REQ-0023, DMS-LSP-REQ-0024
\item
  DMS-NB-REQ-0001, DMS-NB-REQ-0002, DMS-NB-REQ-0005, DMS-NB-REQ-0006,
  DMS-NB-REQ-0013, DMS-NB-REQ-0017, and DMS-NB-REQ-0029, primarily
\end{itemize}

Note this test was not designed to perform a full verification of the
above requirements, but rather to demonstrate having reached a certain
level of partial capability during construction.
}
\begin{longtable}{p{3cm}p{2.5cm}p{2.5cm}p{3cm}p{4cm}}
\toprule
\href{https://jira.lsstcorp.org/secure/Tests.jspa\#/testCase/LVV-T1437}{LVV-T1437} & \multicolumn{4}{p{12cm}}{ LDM-503-10a: API Aspect tests for LSP with Authentication and TAP
milestone } \\ \hline
\textbf{Owner} & \textbf{Status} & \textbf{Version} & \textbf{Critical Event} & \textbf{Verification Type} \\ \hline
Gregory Dubois-Felsmann & Defined & 1 & false & Test \\ \hline
\end{longtable}
{\scriptsize
\textbf{Objective:}\\
This test case verifies that the TAP service in the API Aspect of the
Science Platform is accessible to authorized users through a login
process, and that TAP searches can be performed using the IVOA TAP
protocol from remote sites.\\[2\baselineskip]In so doing and in
conjunction with the other LDM-503-10a test cases collected under
LVV-P48, it addresses all or part of the following requirements:

\begin{itemize}
\tightlist
\item
  DMS-LSP-REQ-0004, DMS-LSP-REQ-0005, DMS-LSP-REQ-0006,
  DMS-LSP-REQ-0020, DMS-LSP-REQ-0022, DMS-LSP-REQ-0023, DMS-LSP-REQ-0024
\item
  DMS-API-REQ-0003, DMS-API-REQ-0004, DMS-API-REQ-0006,
  DMS-API-REQ-0007, DMS-API-REQ-0008, DMS-API-REQ-0009,
  DMS-API-REQ-0023, and DMS-API-REQ-0039, primarily
\end{itemize}

Note this test was not designed to perform a full verification of the
above requirements, but rather to demonstrate having reached a certain
level of partial capability during construction.
}
  
 \newpage 
\subsection{[LVV-9836] DMS-LSP-REQ-0026-V-01: Using secure protocols\_1 }\label{lvv-9836}

\begin{longtable}{cccc}
\hline
\textbf{Jira Link} & \textbf{Assignee} & \textbf{Status} & \textbf{Test Cases}\\ \hline
\href{https://jira.lsstcorp.org/browse/LVV-9836}{LVV-9836} &
Gregory Dubois-Felsmann & Not Covered &
\begin{tabular}{c}
LVV-T628 \\
LVV-T1436 \\
\end{tabular}
\\
\hline
\end{longtable}

\textbf{Verification Element Description:} \\
Undefined

{\footnotesize
\begin{longtable}{p{2.5cm}p{13.5cm}}
\hline
\multicolumn{2}{c}{\textbf{Requirement Details}}\\ \hline
Requirement ID & DMS-LSP-REQ-0026 \\ \cdashline{1-2}
Requirement Description &
\begin{minipage}[]{13cm}
All external connections to the LSP shall be encrypted using protocols
and cipher suites compliant with LSST cybersecurity policy.
\end{minipage}
\\ \cdashline{1-2}
Requirement Discussion &
\begin{minipage}[]{13cm}
All connections from remote clients to LSST servers (including any web
sockets to enable AJAX-like functionality) will be encrypted. It is
expected that this will mean general use of ``https:'' protocols.\\
Connections back out to external archives that do not support secure
communications may be left unencrypted.
\end{minipage}
\\ \cdashline{1-2}
Requirement Priority &  \\ \cdashline{1-2}
Upper Level Requirement &
\begin{tabular}{cl}
\end{tabular}
\\ \hline
\end{longtable}
}


\subsubsection{Test Cases Summary}
\begin{longtable}{p{3cm}p{2.5cm}p{2.5cm}p{3cm}p{4cm}}
\toprule
\href{https://jira.lsstcorp.org/secure/Tests.jspa\#/testCase/LVV-T628}{LVV-T628} & \multicolumn{4}{p{12cm}}{ Verify LSP connections encrypted } \\ \hline
\textbf{Owner} & \textbf{Status} & \textbf{Version} & \textbf{Critical Event} & \textbf{Verification Type} \\ \hline
Jeffrey Carlin & Draft & 1 & false & Inspection \\ \hline
\end{longtable}
{\scriptsize
\textbf{Objective:}\\
Verify that all external connections to the LSP are encrypted in
accordance with LSST cybersecurity policy.
}
\begin{longtable}{p{3cm}p{2.5cm}p{2.5cm}p{3cm}p{4cm}}
\toprule
\href{https://jira.lsstcorp.org/secure/Tests.jspa\#/testCase/LVV-T1436}{LVV-T1436} & \multicolumn{4}{p{12cm}}{ LDM-503-10a: Notebook Aspect tests for LSP with Authentication and TAP
milestone } \\ \hline
\textbf{Owner} & \textbf{Status} & \textbf{Version} & \textbf{Critical Event} & \textbf{Verification Type} \\ \hline
Gregory Dubois-Felsmann & Defined & 1 & false & Test \\ \hline
\end{longtable}
{\scriptsize
\textbf{Objective:}\\
This test case verifies that the Notebook Aspect of the Science Platform
is accessible to authorized users through a login process, and that TAP
searches can be performed from Python code in the Notebook
Aspect.\\[2\baselineskip]In so doing and in conjunction with the other
LDM-503-10a test cases collected under LVV-P48, it addresses all or part
of the following requirements:

\begin{itemize}
\tightlist
\item
  DMS-LSP-REQ-0003, DMS-LSP-REQ-0005, DMS-LSP-REQ-0006,
  DMS-LSP-REQ-0020, DMS-LSP-REQ-0022, DMS-LSP-REQ-0023, DMS-LSP-REQ-0024
\item
  DMS-NB-REQ-0001, DMS-NB-REQ-0002, DMS-NB-REQ-0005, DMS-NB-REQ-0006,
  DMS-NB-REQ-0013, DMS-NB-REQ-0017, and DMS-NB-REQ-0029, primarily
\end{itemize}

Note this test was not designed to perform a full verification of the
above requirements, but rather to demonstrate having reached a certain
level of partial capability during construction.
}
  
 \newpage 
\subsection{[LVV-9837] DMS-LSP-REQ-0033-V-01: Internet-Accessible (IPv4)\_1 }\label{lvv-9837}

\begin{longtable}{cccc}
\hline
\textbf{Jira Link} & \textbf{Assignee} & \textbf{Status} & \textbf{Test Cases}\\ \hline
\href{https://jira.lsstcorp.org/browse/LVV-9837}{LVV-9837} &
Gregory Dubois-Felsmann & Not Covered &
\begin{tabular}{c}
LVV-T631 \\
\end{tabular}
\\
\hline
\end{longtable}

\textbf{Verification Element Description:} \\
Undefined

{\footnotesize
\begin{longtable}{p{2.5cm}p{13.5cm}}
\hline
\multicolumn{2}{c}{\textbf{Requirement Details}}\\ \hline
Requirement ID & DMS-LSP-REQ-0033 \\ \cdashline{1-2}
Requirement Description &
\begin{minipage}[]{13cm}
The LSP shall support access from the public Internet using IPv4
protocols.
\end{minipage}
\\ \cdashline{1-2}
Requirement Discussion &
\begin{minipage}[]{13cm}
This covers all three aspects of the LSP and thus both browser and other
Web-API access.
\end{minipage}
\\ \cdashline{1-2}
Requirement Priority &  \\ \cdashline{1-2}
Upper Level Requirement &
\begin{tabular}{cl}
\end{tabular}
\\ \hline
\end{longtable}
}


\subsubsection{Test Cases Summary}
\begin{longtable}{p{3cm}p{2.5cm}p{2.5cm}p{3cm}p{4cm}}
\toprule
\href{https://jira.lsstcorp.org/secure/Tests.jspa\#/testCase/LVV-T631}{LVV-T631} & \multicolumn{4}{p{12cm}}{ Verify LSP access from the public Internet (IPv4) } \\ \hline
\textbf{Owner} & \textbf{Status} & \textbf{Version} & \textbf{Critical Event} & \textbf{Verification Type} \\ \hline
Jeffrey Carlin & Draft & 1 & false & Inspection \\ \hline
\end{longtable}
{\scriptsize
\textbf{Objective:}\\
Verify that the LSP is accessible from the public Internet using IPv4
protocols.
}
  
 \newpage 
\subsection{[LVV-9838] DMS-LSP-REQ-0034-V-01: Internet-Accessible (IPv6)\_1 }\label{lvv-9838}

\begin{longtable}{cccc}
\hline
\textbf{Jira Link} & \textbf{Assignee} & \textbf{Status} & \textbf{Test Cases}\\ \hline
\href{https://jira.lsstcorp.org/browse/LVV-9838}{LVV-9838} &
Gregory Dubois-Felsmann & Not Covered &
\begin{tabular}{c}
LVV-T632 \\
\end{tabular}
\\
\hline
\end{longtable}

\textbf{Verification Element Description:} \\
Undefined

{\footnotesize
\begin{longtable}{p{2.5cm}p{13.5cm}}
\hline
\multicolumn{2}{c}{\textbf{Requirement Details}}\\ \hline
Requirement ID & DMS-LSP-REQ-0034 \\ \cdashline{1-2}
Requirement Description &
\begin{minipage}[]{13cm}
The LSP shall support access from the public Internet using IPv6
protocols.
\end{minipage}
\\ \cdashline{1-2}
Requirement Discussion &
\begin{minipage}[]{13cm}
This covers all three aspects of the LSP and thus both browser and other
Web-API access.
\end{minipage}
\\ \cdashline{1-2}
Requirement Priority &  \\ \cdashline{1-2}
Upper Level Requirement &
\begin{tabular}{cl}
\end{tabular}
\\ \hline
\end{longtable}
}


\subsubsection{Test Cases Summary}
\begin{longtable}{p{3cm}p{2.5cm}p{2.5cm}p{3cm}p{4cm}}
\toprule
\href{https://jira.lsstcorp.org/secure/Tests.jspa\#/testCase/LVV-T632}{LVV-T632} & \multicolumn{4}{p{12cm}}{ Verify LSP access from the public Internet (IPv6) } \\ \hline
\textbf{Owner} & \textbf{Status} & \textbf{Version} & \textbf{Critical Event} & \textbf{Verification Type} \\ \hline
Jeffrey Carlin & Draft & 1 & false & Inspection \\ \hline
\end{longtable}
{\scriptsize
\textbf{Objective:}\\
Verify that the LSP is accessible from the public Internet using IPv6
protocols.
}
  
 \newpage 
\subsection{[LVV-9839] DMS-LSP-REQ-0032-V-01: Multiple installations\_1 }\label{lvv-9839}

\begin{longtable}{cccc}
\hline
\textbf{Jira Link} & \textbf{Assignee} & \textbf{Status} & \textbf{Test Cases}\\ \hline
\href{https://jira.lsstcorp.org/browse/LVV-9839}{LVV-9839} &
Gregory Dubois-Felsmann & Not Covered &
\begin{tabular}{c}
LVV-T630 \\
\end{tabular}
\\
\hline
\end{longtable}

\textbf{Verification Element Description:} \\
Undefined

{\footnotesize
\begin{longtable}{p{2.5cm}p{13.5cm}}
\hline
\multicolumn{2}{c}{\textbf{Requirement Details}}\\ \hline
Requirement ID & DMS-LSP-REQ-0032 \\ \cdashline{1-2}
Requirement Description &
\begin{minipage}[]{13cm}
The LSP design shall facilitate the installation and maintenance of
multiple instances of the LSP and shall support both instances that are
accessible from the public Internet and instances that are accessible
only within the LSST Project.
\end{minipage}
\\ \cdashline{1-2}
Requirement Discussion &
\begin{minipage}[]{13cm}
In addition to the public instances at the Project-provided Data Access
Centers, it is expected that additional instances will be used to
support internal processes such as commissioning.\\
It is also desirable that the LSP be designed and implemented so that
its components can readily be installed outside the Project-provided
facilities (e.g., at IN2P3), but (TBR) this is not a formal requirement.
\end{minipage}
\\ \cdashline{1-2}
Requirement Priority &  \\ \cdashline{1-2}
Upper Level Requirement &
\begin{tabular}{cl}
\end{tabular}
\\ \hline
\end{longtable}
}


\subsubsection{Test Cases Summary}
\begin{longtable}{p{3cm}p{2.5cm}p{2.5cm}p{3cm}p{4cm}}
\toprule
\href{https://jira.lsstcorp.org/secure/Tests.jspa\#/testCase/LVV-T630}{LVV-T630} & \multicolumn{4}{p{12cm}}{ Verify multiple LSP instances } \\ \hline
\textbf{Owner} & \textbf{Status} & \textbf{Version} & \textbf{Critical Event} & \textbf{Verification Type} \\ \hline
Jeffrey Carlin & Draft & 1 & false & Inspection \\ \hline
\end{longtable}
{\scriptsize
\textbf{Objective:}\\
Verify that separate instances of the LSP accessible to the public, and
only within the LSST Project, are available and maintained.
}
  
 \newpage 
\subsection{[LVV-9840] DMS-LSP-REQ-0035-V-01: System-Availability Indication\_1 }\label{lvv-9840}

\begin{longtable}{cccc}
\hline
\textbf{Jira Link} & \textbf{Assignee} & \textbf{Status} & \textbf{Test Cases}\\ \hline
\href{https://jira.lsstcorp.org/browse/LVV-9840}{LVV-9840} &
Gregory Dubois-Felsmann & Not Covered &
\begin{tabular}{c}
LVV-T633 \\
\end{tabular}
\\
\hline
\end{longtable}

\textbf{Verification Element Description:} \\
Undefined

{\footnotesize
\begin{longtable}{p{2.5cm}p{13.5cm}}
\hline
\multicolumn{2}{c}{\textbf{Requirement Details}}\\ \hline
Requirement ID & DMS-LSP-REQ-0035 \\ \cdashline{1-2}
Requirement Description &
\begin{minipage}[]{13cm}
The LSP aspects shall provide means to inform users when their services
are unavailable, including for reasons of maintenance or excessive load.
\end{minipage}
\\ \cdashline{1-2}
Requirement Discussion &
\begin{minipage}[]{13cm}
This is a DM requirement, not just a Science Platform requirement, as it
interacts with lower-level system-availability issues.\\
Even at LSP level this is a complicated problem, as availability of
query services through the Portal, for instance, involves time budgeting
and resource management for the all of the stages of the query: query
generation, query running, results generation, display of results, and
downloading and saving of results.
\end{minipage}
\\ \cdashline{1-2}
Requirement Priority &  \\ \cdashline{1-2}
Upper Level Requirement &
\begin{tabular}{cl}
\end{tabular}
\\ \hline
\end{longtable}
}


\subsubsection{Test Cases Summary}
\begin{longtable}{p{3cm}p{2.5cm}p{2.5cm}p{3cm}p{4cm}}
\toprule
\href{https://jira.lsstcorp.org/secure/Tests.jspa\#/testCase/LVV-T633}{LVV-T633} & \multicolumn{4}{p{12cm}}{ Verify indication of system availability } \\ \hline
\textbf{Owner} & \textbf{Status} & \textbf{Version} & \textbf{Critical Event} & \textbf{Verification Type} \\ \hline
Jeffrey Carlin & Draft & 1 & false & Inspection \\ \hline
\end{longtable}
{\scriptsize
\textbf{Objective:}\\
Verify that the LSP informs users when services are unavailable due to
maintenance or excessive load.
}
  
 \newpage 
\subsection{[LVV-9841] DMS-PRTL-REQ-0001-V-01: Portal is a Web Application\_1 }\label{lvv-9841}

\begin{longtable}{cccc}
\hline
\textbf{Jira Link} & \textbf{Assignee} & \textbf{Status} & \textbf{Test Cases}\\ \hline
\href{https://jira.lsstcorp.org/browse/LVV-9841}{LVV-9841} &
Gregory Dubois-Felsmann & Not Covered &
\begin{tabular}{c}
LVV-T634 \\
LVV-T1334 \\
\end{tabular}
\\
\hline
\end{longtable}

\textbf{Verification Element Description:} \\
Undefined

{\footnotesize
\begin{longtable}{p{2.5cm}p{13.5cm}}
\hline
\multicolumn{2}{c}{\textbf{Requirement Details}}\\ \hline
Requirement ID & DMS-PRTL-REQ-0001 \\ \cdashline{1-2}
Requirement Description &
\begin{minipage}[]{13cm}
The Portal aspect shall be accessible through commonly used desktop web
browsers without requiring users to download and install local software
packages.
\end{minipage}
\\ \cdashline{1-2}
Requirement Discussion &
\begin{minipage}[]{13cm}
The supported browsers will be documented by the Project.
\end{minipage}
\\ \cdashline{1-2}
Requirement Priority &  \\ \cdashline{1-2}
Upper Level Requirement &
\begin{tabular}{cl}
\end{tabular}
\\ \hline
\end{longtable}
}


\subsubsection{Test Cases Summary}
\begin{longtable}{p{3cm}p{2.5cm}p{2.5cm}p{3cm}p{4cm}}
\toprule
\href{https://jira.lsstcorp.org/secure/Tests.jspa\#/testCase/LVV-T634}{LVV-T634} & \multicolumn{4}{p{12cm}}{ Verify Portal is a web application } \\ \hline
\textbf{Owner} & \textbf{Status} & \textbf{Version} & \textbf{Critical Event} & \textbf{Verification Type} \\ \hline
Jeffrey Carlin & Draft & 1 & false & Inspection \\ \hline
\end{longtable}
{\scriptsize
\textbf{Objective:}\\
Verify that the Portal is a web application that is accessible to users
via common web browsers and without downloading and installing local
software.
}
\begin{longtable}{p{3cm}p{2.5cm}p{2.5cm}p{3cm}p{4cm}}
\toprule
\href{https://jira.lsstcorp.org/secure/Tests.jspa\#/testCase/LVV-T1334}{LVV-T1334} & \multicolumn{4}{p{12cm}}{ LDM-503-10a: Portal Aspect tests for LSP with Authentication and TAP
milestone } \\ \hline
\textbf{Owner} & \textbf{Status} & \textbf{Version} & \textbf{Critical Event} & \textbf{Verification Type} \\ \hline
Gregory Dubois-Felsmann & Defined & 1 & false & Test \\ \hline
\end{longtable}
{\scriptsize
\textbf{Objective:}\\
This test case verifies that the Portal Aspect of the Science Platform
is accessible to authorized users through a login process, and that TAP
searches can be performed from the Portal Aspect UI.\\[2\baselineskip]In
so doing and in conjunction with the other LDM-503-10a test cases
collected under LVV-P48, it addresses all or part of the following
requirements:

\begin{itemize}
\tightlist
\item
  DMS-LSP-REQ-0002, DMS-LSP-REQ-0005, DMS-LSP-REQ-0006,
  DMS-LSP-REQ-0020, DMS-LSP-REQ-0022, DMS-LSP-REQ-0023, DMS-LSP-REQ-0024
\item
  DMS-PRTL-REQ-0001, DMS-PRTL-REQ-0015, DMS-PRTL-REQ-0016,
  DMS-PRTL-REQ-0017, DMS-PRTL-REQ-0020, DMS-PRTL-REQ-0026,
  DMS-PRTL-REQ-0049, and DMS-PRTL-REQ-0095, primarily
\end{itemize}

Note this test was not designed to perform a full verification of the
above requirements, but rather to demonstrate having reached a certain
level of partial capability during construction.
}
  
 \newpage 
\subsection{[LVV-9842] DMS-PRTL-REQ-0005-V-01: Access to Calibration Products\_1 }\label{lvv-9842}

\begin{longtable}{cccc}
\hline
\textbf{Jira Link} & \textbf{Assignee} & \textbf{Status} & \textbf{Test Cases}\\ \hline
\href{https://jira.lsstcorp.org/browse/LVV-9842}{LVV-9842} &
Gregory Dubois-Felsmann & Not Covered &
\begin{tabular}{c}
LVV-T638 \\
\end{tabular}
\\
\hline
\end{longtable}

\textbf{Verification Element Description:} \\
Undefined

{\footnotesize
\begin{longtable}{p{2.5cm}p{13.5cm}}
\hline
\multicolumn{2}{c}{\textbf{Requirement Details}}\\ \hline
Requirement ID & DMS-PRTL-REQ-0005 \\ \cdashline{1-2}
Requirement Description &
\begin{minipage}[]{13cm}
The Portal aspect shall enable access to Project calibration data
products, both directly and via linkages from science data products
generated using them.
\end{minipage}
\\ \cdashline{1-2}
Requirement Discussion &
\begin{minipage}[]{13cm}
This is a sub-requirement of ``Semantic Linkage: Portal'' that carries
forward a specific requirement originating from early SUIT
specifications.
\end{minipage}
\\ \cdashline{1-2}
Requirement Priority &  \\ \cdashline{1-2}
Upper Level Requirement &
\begin{tabular}{cl}
\end{tabular}
\\ \hline
\end{longtable}
}


\subsubsection{Test Cases Summary}
\begin{longtable}{p{3cm}p{2.5cm}p{2.5cm}p{3cm}p{4cm}}
\toprule
\href{https://jira.lsstcorp.org/secure/Tests.jspa\#/testCase/LVV-T638}{LVV-T638} & \multicolumn{4}{p{12cm}}{ Verify access to calibration products via Portal } \\ \hline
\textbf{Owner} & \textbf{Status} & \textbf{Version} & \textbf{Critical Event} & \textbf{Verification Type} \\ \hline
Jeffrey Carlin & Draft & 1 & false & Inspection \\ \hline
\end{longtable}
{\scriptsize
\textbf{Objective:}\\
Verify that calibration products are accessible from the Portal aspect,
both directly and via linkages from science data products that use them.
This is a sub-requirement of DMS-PRTL-REQ-0004 (associated test case:
\href{https://jira.lsstcorp.org/secure/Tests.jspa\#/testCase/LVV-T637}{LVV-T637}).
}
  
 \newpage 
\subsection{[LVV-9843] DMS-PRTL-REQ-0007-V-01: Access to External Archives\_1 }\label{lvv-9843}

\begin{longtable}{cccc}
\hline
\textbf{Jira Link} & \textbf{Assignee} & \textbf{Status} & \textbf{Test Cases}\\ \hline
\href{https://jira.lsstcorp.org/browse/LVV-9843}{LVV-9843} &
Gregory Dubois-Felsmann & Not Covered &
\begin{tabular}{c}
LVV-T640 \\
\end{tabular}
\\
\hline
\end{longtable}

\textbf{Verification Element Description:} \\
Undefined

{\footnotesize
\begin{longtable}{p{2.5cm}p{13.5cm}}
\hline
\multicolumn{2}{c}{\textbf{Requirement Details}}\\ \hline
Requirement ID & DMS-PRTL-REQ-0007 \\ \cdashline{1-2}
Requirement Description &
\begin{minipage}[]{13cm}
The Portal aspect shall provide an interface to outside catalog and
image data services that are available via standard astronomical VO
interfaces to enable a user to determine what external astronomical data
are associated with a given location on the sky and return that data for
use within the Portal.
\end{minipage}
\\ \cdashline{1-2}
Requirement Discussion &
\begin{minipage}[]{13cm}
The purpose of this requirement is to help users connect LSST data with
other data already available in community archives (e.g., IRSA, NED,
MAST, HEASARC), and should support any VO-compliant archive. If access
to a non-VO compliant archive is critical for the needs of the LSST
project that could be considered here.
\end{minipage}
\\ \cdashline{1-2}
Requirement Priority &  \\ \cdashline{1-2}
Upper Level Requirement &
\begin{tabular}{cl}
\end{tabular}
\\ \hline
\end{longtable}
}


\subsubsection{Test Cases Summary}
\begin{longtable}{p{3cm}p{2.5cm}p{2.5cm}p{3cm}p{4cm}}
\toprule
\href{https://jira.lsstcorp.org/secure/Tests.jspa\#/testCase/LVV-T640}{LVV-T640} & \multicolumn{4}{p{12cm}}{ Verify access to external archives from Portal } \\ \hline
\textbf{Owner} & \textbf{Status} & \textbf{Version} & \textbf{Critical Event} & \textbf{Verification Type} \\ \hline
Jeffrey Carlin & Draft & 1 & false & Inspection \\ \hline
\end{longtable}
{\scriptsize
\textbf{Objective:}\\
Verify that an interface to outside catalog and image data is available,
that allows a user to determine what external astronomical data are
associated with a given location on the sky and return those data for
use within the Portal.
}
  
 \newpage 
\subsection{[LVV-9844] DMS-PRTL-REQ-0008-V-01: API for Access to Portal Session State\_1 }\label{lvv-9844}

\begin{longtable}{cccc}
\hline
\textbf{Jira Link} & \textbf{Assignee} & \textbf{Status} & \textbf{Test Cases}\\ \hline
\href{https://jira.lsstcorp.org/browse/LVV-9844}{LVV-9844} &
Gregory Dubois-Felsmann & Not Covered &
\begin{tabular}{c}
LVV-T641 \\
\end{tabular}
\\
\hline
\end{longtable}

\textbf{Verification Element Description:} \\
Undefined

{\footnotesize
\begin{longtable}{p{2.5cm}p{13.5cm}}
\hline
\multicolumn{2}{c}{\textbf{Requirement Details}}\\ \hline
Requirement ID & DMS-PRTL-REQ-0008 \\ \cdashline{1-2}
Requirement Description &
\begin{minipage}[]{13cm}
The Portal aspect shall provide a network API that allows authenticated
remote access by a user to aspects of their session state in the Portal.
The minimal requirement is for access to the list of queries performed
in that session.
\end{minipage}
\\ \cdashline{1-2}
Requirement Discussion &
\begin{minipage}[]{13cm}
Access to the list of queries allows user code in the Notebook aspect to
use those query references to retrieve the data which was being explored
in the Portal.
\end{minipage}
\\ \cdashline{1-2}
Requirement Priority &  \\ \cdashline{1-2}
Upper Level Requirement &
\begin{tabular}{cl}
\end{tabular}
\\ \hline
\end{longtable}
}


\subsubsection{Test Cases Summary}
\begin{longtable}{p{3cm}p{2.5cm}p{2.5cm}p{3cm}p{4cm}}
\toprule
\href{https://jira.lsstcorp.org/secure/Tests.jspa\#/testCase/LVV-T641}{LVV-T641} & \multicolumn{4}{p{12cm}}{ Verify API for Access to Portal Session State } \\ \hline
\textbf{Owner} & \textbf{Status} & \textbf{Version} & \textbf{Critical Event} & \textbf{Verification Type} \\ \hline
Jeffrey Carlin & Draft & 1 & false & Inspection \\ \hline
\end{longtable}
{\scriptsize
\textbf{Objective:}\\
Verify that the Portal aspect provides a network API that allows
authenticated remote access by a user to aspects of their session state
in the Portal. The minimal requirement is for access to the list of
queries performed in that session.
}
  
 \newpage 
\subsection{[LVV-9845] DMS-PRTL-REQ-0006-V-01: Coadded Image to Single-Epoch Image
Associations\_1 }\label{lvv-9845}

\begin{longtable}{cccc}
\hline
\textbf{Jira Link} & \textbf{Assignee} & \textbf{Status} & \textbf{Test Cases}\\ \hline
\href{https://jira.lsstcorp.org/browse/LVV-9845}{LVV-9845} &
Gregory Dubois-Felsmann & Not Covered &
\begin{tabular}{c}
LVV-T639 \\
\end{tabular}
\\
\hline
\end{longtable}

\textbf{Verification Element Description:} \\
Undefined

{\footnotesize
\begin{longtable}{p{2.5cm}p{13.5cm}}
\hline
\multicolumn{2}{c}{\textbf{Requirement Details}}\\ \hline
Requirement ID & DMS-PRTL-REQ-0006 \\ \cdashline{1-2}
Requirement Description &
\begin{minipage}[]{13cm}
The Portal aspect shall facilitate users following the associations
between coadded images and the single epoch images that were used to
generate them.
\end{minipage}
\\ \cdashline{1-2}
Requirement Discussion &
\begin{minipage}[]{13cm}
This is a sub-requirement of ``Semantic Linkage: Portal'' that carries
forward a specific requirement originating from early SUIT
specifications.
\end{minipage}
\\ \cdashline{1-2}
Requirement Priority &  \\ \cdashline{1-2}
Upper Level Requirement &
\begin{tabular}{cl}
\end{tabular}
\\ \hline
\end{longtable}
}


\subsubsection{Test Cases Summary}
\begin{longtable}{p{3cm}p{2.5cm}p{2.5cm}p{3cm}p{4cm}}
\toprule
\href{https://jira.lsstcorp.org/secure/Tests.jspa\#/testCase/LVV-T639}{LVV-T639} & \multicolumn{4}{p{12cm}}{ Verify associations between single images and coadds } \\ \hline
\textbf{Owner} & \textbf{Status} & \textbf{Version} & \textbf{Critical Event} & \textbf{Verification Type} \\ \hline
Jeffrey Carlin & Draft & 1 & false & Inspection \\ \hline
\end{longtable}
{\scriptsize
\textbf{Objective:}\\
Verify that users can discover the associations between coadded images
and the single-epoch images that contributed to the coadds. This is a
sub-requirement of DMS-PRTL-REQ-0004 (associated test case:
\href{https://jira.lsstcorp.org/secure/Tests.jspa\#/testCase/LVV-T637}{LVV-T637}).
}
  
 \newpage 
\subsection{[LVV-9846] DMS-PRTL-REQ-0003-V-01: Portal Access to Workspace\_1 }\label{lvv-9846}

\begin{longtable}{cccc}
\hline
\textbf{Jira Link} & \textbf{Assignee} & \textbf{Status} & \textbf{Test Cases}\\ \hline
\href{https://jira.lsstcorp.org/browse/LVV-9846}{LVV-9846} &
Gregory Dubois-Felsmann & Not Covered &
\begin{tabular}{c}
LVV-T636 \\
LVV-T1818 \\
\end{tabular}
\\
\hline
\end{longtable}

\textbf{Verification Element Description:} \\
Undefined

{\footnotesize
\begin{longtable}{p{2.5cm}p{13.5cm}}
\hline
\multicolumn{2}{c}{\textbf{Requirement Details}}\\ \hline
Requirement ID & DMS-PRTL-REQ-0003 \\ \cdashline{1-2}
Requirement Description &
\begin{minipage}[]{13cm}
The Portal aspect shall have the capability to discover all data in the
user's Workspace.
\end{minipage}
\\ \cdashline{1-2}
Requirement Discussion &
\begin{minipage}[]{13cm}
This will allow for data and images to be retrieved from the environment
for use within the portal.
\end{minipage}
\\ \cdashline{1-2}
Requirement Priority &  \\ \cdashline{1-2}
Upper Level Requirement &
\begin{tabular}{cl}
\end{tabular}
\\ \hline
\end{longtable}
}


\subsubsection{Test Cases Summary}
\begin{longtable}{p{3cm}p{2.5cm}p{2.5cm}p{3cm}p{4cm}}
\toprule
\href{https://jira.lsstcorp.org/secure/Tests.jspa\#/testCase/LVV-T636}{LVV-T636} & \multicolumn{4}{p{12cm}}{ Verify Portal access to Workspace } \\ \hline
\textbf{Owner} & \textbf{Status} & \textbf{Version} & \textbf{Critical Event} & \textbf{Verification Type} \\ \hline
Jeffrey Carlin & Draft & 1 & false & Inspection \\ \hline
\end{longtable}
{\scriptsize
\textbf{Objective:}\\
Verify that users can discover and retrieve data and images within their
Workspace.
}
\begin{longtable}{p{3cm}p{2.5cm}p{2.5cm}p{3cm}p{4cm}}
\toprule
\href{https://jira.lsstcorp.org/secure/Tests.jspa\#/testCase/LVV-T1818}{LVV-T1818} & \multicolumn{4}{p{12cm}}{ DM-SUIT-8: Verify Portal integration with workspace (via WebDAV) } \\ \hline
\textbf{Owner} & \textbf{Status} & \textbf{Version} & \textbf{Critical Event} & \textbf{Verification Type} \\ \hline
Gregory Dubois-Felsmann & Defined & 1 & false & Demonstration \\ \hline
\end{longtable}
{\scriptsize
\textbf{Objective:}\\
This test case verifies that the Portal Aspect software is capable of
accessing a file-oriented workspace via the WebDAV
protocol.\\[2\baselineskip]In so doing, it partially verifies several
Portal Aspect requirements that relate to this capability -
``partially'' because some of these requirements depend on workspace
capabilities which were not present in the prototype WebDAV service
delivered by the DAX group, because some of the requirements also cover
the User Database Workspace (not relevant to this milestone, and not yet
available), and also because the milestone was not envisioned as an
exhaustive test covering edge cases:

\begin{itemize}
\tightlist
\item
  DMS-PRTL-REQ-0003 (LVV-9846, Portal access to workspace) is covered at
  ``demonstration'' level, with basic tests of saving image and tabular
  data to the workspace, and only for the User File Workspace ~(there is
  currently no User Database Workspace prototype available);
\item
  DMS-PRTL-REQ-0046 (LVV-9886, Visualization of workspace data) is
  covered at ``demonstration'' level for a couple of FITS image and
  table files, and only for the User File Workspace;
\item
  DMS-PRTL-REQ-0110 (LVV-9954, Tabular data download) is covered at
  ``demonstration'' level, only for catalog data (there was no image
  metadata in the LSP deployment at the time of test), and only for the
  User File Workspace;~
\item
  DMS-PRTL-REQ-0095 (LVV-9932, Saving Displayed Tabular Data) is covered
  at ``demonstration'' level for a simple subset operation in the table
  browser; and
\item
  DMS-PRTL-REQ-0111 (LVV-9951, Image data download) is covered at
  ``demonstration'' level, and only for download from an image display
  screen itself (as LSST-style image metadata services, e.g., ObsTAP,
  were not available in the LSP at the time of testing).
\end{itemize}
}
  
 \newpage 
\subsection{[LVV-9847] DMS-PRTL-REQ-0002-V-01: Portal Discovery of all Data Products\_1 }\label{lvv-9847}

\begin{longtable}{cccc}
\hline
\textbf{Jira Link} & \textbf{Assignee} & \textbf{Status} & \textbf{Test Cases}\\ \hline
\href{https://jira.lsstcorp.org/browse/LVV-9847}{LVV-9847} &
Gregory Dubois-Felsmann & Not Covered &
\begin{tabular}{c}
LVV-T635 \\
\end{tabular}
\\
\hline
\end{longtable}

\textbf{Verification Element Description:} \\
Undefined

{\footnotesize
\begin{longtable}{p{2.5cm}p{13.5cm}}
\hline
\multicolumn{2}{c}{\textbf{Requirement Details}}\\ \hline
Requirement ID & DMS-PRTL-REQ-0002 \\ \cdashline{1-2}
Requirement Description &
\begin{minipage}[]{13cm}
The Portal aspect shall provide the capability to discover and access
all the Project's released data products, including, but not limited to,
the data products enumerated in the DPDD (\citeds{LSE-163}), the calibration
database, and the Reformatted EFD, as well as all user data products to
which a user has access.
\end{minipage}
\\ \cdashline{1-2}
Requirement Discussion &
\begin{minipage}[]{13cm}
The Portal's workflows should allow a user to learn what data exist:
what data releases are available, what image and catalog data they
contain, the names of all databases, tables, and columns, etc.\\
For all tabular data products the Generic Query requirements below cover
the basic level of access provided.
\end{minipage}
\\ \cdashline{1-2}
Requirement Priority &  \\ \cdashline{1-2}
Upper Level Requirement &
\begin{tabular}{cl}
\end{tabular}
\\ \hline
\end{longtable}
}


\subsubsection{Test Cases Summary}
\begin{longtable}{p{3cm}p{2.5cm}p{2.5cm}p{3cm}p{4cm}}
\toprule
\href{https://jira.lsstcorp.org/secure/Tests.jspa\#/testCase/LVV-T635}{LVV-T635} & \multicolumn{4}{p{12cm}}{ Verify Portal discovery of all data products } \\ \hline
\textbf{Owner} & \textbf{Status} & \textbf{Version} & \textbf{Critical Event} & \textbf{Verification Type} \\ \hline
Jeffrey Carlin & Draft & 1 & false & Inspection \\ \hline
\end{longtable}
{\scriptsize
\textbf{Objective:}\\
Verify that the Portal enables discovery of all data products released
by the Project, including all products enumerated in the DPDD, the
calibration database, and the reformatted EFD, as well as user data
products to which the user has access.
}
  
 \newpage 
\subsection{[LVV-9848] DMS-PRTL-REQ-0004-V-01: Semantic Linkage: Portal Workflows\_1 }\label{lvv-9848}

\begin{longtable}{cccc}
\hline
\textbf{Jira Link} & \textbf{Assignee} & \textbf{Status} & \textbf{Test Cases}\\ \hline
\href{https://jira.lsstcorp.org/browse/LVV-9848}{LVV-9848} &
Gregory Dubois-Felsmann & Not Covered &
\begin{tabular}{c}
LVV-T8 \\
LVV-T637 \\
\end{tabular}
\\
\hline
\end{longtable}

\textbf{Verification Element Description:} \\
Undefined

{\footnotesize
\begin{longtable}{p{2.5cm}p{13.5cm}}
\hline
\multicolumn{2}{c}{\textbf{Requirement Details}}\\ \hline
Requirement ID & DMS-PRTL-REQ-0004 \\ \cdashline{1-2}
Requirement Description &
\begin{minipage}[]{13cm}
The Portal aspect shall provide for the identification and retrieval of
semantically linked data.
\end{minipage}
\\ \cdashline{1-2}
Requirement Discussion &
\begin{minipage}[]{13cm}
The Portal should provide straightforward UI workflows for starting from
a selected data item (image or catalog entry) and identifying related
data, including both direct data-dependency and provenance linkages and
more scientifically oriented linkages such as the ability to navigate
from an Object to its associated ForcedSources.\\
Some of these linkages will be supported by the automatic exploitation
of metadata exposed through the underlying data access APIs; others will
be specifically designed workflows reflecting scientific understanding
of the relationships among LSST data products and their processing.
\end{minipage}
\\ \cdashline{1-2}
Requirement Priority &  \\ \cdashline{1-2}
Upper Level Requirement &
\begin{tabular}{cl}
\end{tabular}
\\ \hline
\end{longtable}
}


\subsubsection{Test Cases Summary}
\begin{longtable}{p{3cm}p{2.5cm}p{2.5cm}p{3cm}p{4cm}}
\toprule
\href{https://jira.lsstcorp.org/secure/Tests.jspa\#/testCase/LVV-T8}{LVV-T8} & \multicolumn{4}{p{12cm}}{ LSP-00-30: Linkage of catalog query results with associated images } \\ \hline
\textbf{Owner} & \textbf{Status} & \textbf{Version} & \textbf{Critical Event} & \textbf{Verification Type} \\ \hline
Gregory Dubois-Felsmann & Deprecated & 1 & false & Test \\ \hline
\end{longtable}
{\scriptsize
\textbf{Objective:}\\
This test will check for the ability, in the Portal Aspect of the LSST
Science Platform, to match catalog data with the image data on which the
measurements were performed, specifically:

\begin{itemize}
\tightlist
\item
  Navigating from a catalog query result to the associated images; and~
\item
  Overlaying catalog query results on associated images.
\end{itemize}

Because of limited staff resources, these tests will be based on the
original PDAC dataset, the LSST Summer 2013 processing of the SDSS
Stripe 82 data. The image data for the WISE and NEOWISE missions have
not been loaded into PDAC.
}
\begin{longtable}{p{3cm}p{2.5cm}p{2.5cm}p{3cm}p{4cm}}
\toprule
\href{https://jira.lsstcorp.org/secure/Tests.jspa\#/testCase/LVV-T637}{LVV-T637} & \multicolumn{4}{p{12cm}}{ Verify Portal provides semantic linkages between data products } \\ \hline
\textbf{Owner} & \textbf{Status} & \textbf{Version} & \textbf{Critical Event} & \textbf{Verification Type} \\ \hline
Jeffrey Carlin & Draft & 1 & false & Test \\ \hline
\end{longtable}
{\scriptsize
\textbf{Objective:}\\
Verify that the Portal aspect provides users the means to identify and
retrieve semantically linked data. The Portal should provide
straightforward UI workflows for starting from a selected data item
(image or catalog entry) and identifying related data, including both
direct data-dependency and provenance linkages and more scientifically
oriented linkages such as the ability to navigate from an Object to its
associated ForcedSources.
}
  
 \newpage 
\subsection{[LVV-9849] DMS-PRTL-REQ-0010-V-01: Long Query Backgrounding\_1 }\label{lvv-9849}

\begin{longtable}{cccc}
\hline
\textbf{Jira Link} & \textbf{Assignee} & \textbf{Status} & \textbf{Test Cases}\\ \hline
\href{https://jira.lsstcorp.org/browse/LVV-9849}{LVV-9849} &
Gregory Dubois-Felsmann & Not Covered &
\begin{tabular}{c}
LVV-T643 \\
\end{tabular}
\\
\hline
\end{longtable}

\textbf{Verification Element Description:} \\
Undefined

{\footnotesize
\begin{longtable}{p{2.5cm}p{13.5cm}}
\hline
\multicolumn{2}{c}{\textbf{Requirement Details}}\\ \hline
Requirement ID & DMS-PRTL-REQ-0010 \\ \cdashline{1-2}
Requirement Description &
\begin{minipage}[]{13cm}
The Portal aspect shall notify the user if a query is estimated to take
longer than 60 seconds and will allow the user to put the query in
background if desired.
\end{minipage}
\\ \cdashline{1-2}
Requirement Discussion &
\begin{minipage}[]{13cm}
This requirement arose from user panel feedback and the original SUIT
requirements review. It requires support from DAX/database for query
estimation and that is still TBD. Having the query ``in the background''
refers to the user's perspective of being able to proceed with other
work while waiting for it to complete, and may or may not involve a
truly asynchronous DAX query.
\end{minipage}
\\ \cdashline{1-2}
Requirement Priority &  \\ \cdashline{1-2}
Upper Level Requirement &
\begin{tabular}{cl}
\end{tabular}
\\ \hline
\end{longtable}
}


\subsubsection{Test Cases Summary}
\begin{longtable}{p{3cm}p{2.5cm}p{2.5cm}p{3cm}p{4cm}}
\toprule
\href{https://jira.lsstcorp.org/secure/Tests.jspa\#/testCase/LVV-T643}{LVV-T643} & \multicolumn{4}{p{12cm}}{ Verify capability to run long queries in the background } \\ \hline
\textbf{Owner} & \textbf{Status} & \textbf{Version} & \textbf{Critical Event} & \textbf{Verification Type} \\ \hline
Jeffrey Carlin & Draft & 1 & false & Inspection \\ \hline
\end{longtable}
{\scriptsize
\textbf{Objective:}\\
Verify that the Portal aspect will notify the user if a query is
estimated to take longer than 60 seconds, and will allow the user to put
the query in background if desired.
}
  
 \newpage 
\subsection{[LVV-9850] DMS-PRTL-REQ-0013-V-01: Query History Inspection\_1 }\label{lvv-9850}

\begin{longtable}{cccc}
\hline
\textbf{Jira Link} & \textbf{Assignee} & \textbf{Status} & \textbf{Test Cases}\\ \hline
\href{https://jira.lsstcorp.org/browse/LVV-9850}{LVV-9850} &
Gregory Dubois-Felsmann & Not Covered &
\begin{tabular}{c}
LVV-T646 \\
\end{tabular}
\\
\hline
\end{longtable}

\textbf{Verification Element Description:} \\
Undefined

{\footnotesize
\begin{longtable}{p{2.5cm}p{13.5cm}}
\hline
\multicolumn{2}{c}{\textbf{Requirement Details}}\\ \hline
Requirement ID & DMS-PRTL-REQ-0013 \\ \cdashline{1-2}
Requirement Description &
\begin{minipage}[]{13cm}
The Portal aspect shall provide a user interface for browsing the
history of LSST project and user database queries performed by the user,
for re-executing a selected query on demand, and, for recent queries,
re-retrieving their results.
\end{minipage}
\\ \cdashline{1-2}
Requirement Discussion &
\begin{minipage}[]{13cm}
The implementation of this capability is assumed to rest on the API
aspect, and the underlying database systems, supporting this
functionality. The Portal interface to this is a thin UI. This should be
imagined as a CASJobs-like query history.\\
Note that because the API aspect is shared between the Portal and
Notebook, this capability can be used to retrieve in the Portal the
results of queries recently executed in the Notebook aspect; this allows
complex ADQL queries to be formulated programmatically in the Notebook
while still permitting their results to be inspected in the Portal.\\
Because of the reliance on the API aspect, the Portal is not required to
provide this capability for external archives.
\end{minipage}
\\ \cdashline{1-2}
Requirement Priority &  \\ \cdashline{1-2}
Upper Level Requirement &
\begin{tabular}{cl}
\end{tabular}
\\ \hline
\end{longtable}
}


\subsubsection{Test Cases Summary}
\begin{longtable}{p{3cm}p{2.5cm}p{2.5cm}p{3cm}p{4cm}}
\toprule
\href{https://jira.lsstcorp.org/secure/Tests.jspa\#/testCase/LVV-T646}{LVV-T646} & \multicolumn{4}{p{12cm}}{ Verify ability to browse query history } \\ \hline
\textbf{Owner} & \textbf{Status} & \textbf{Version} & \textbf{Critical Event} & \textbf{Verification Type} \\ \hline
Jeffrey Carlin & Draft & 1 & false & Inspection \\ \hline
\end{longtable}
{\scriptsize
\textbf{Objective:}\\
Verify that a user interface exists where users can browse the history
of queries they have performed, and subsequently re-execute them if
desired.
}
  
 \newpage 
\subsection{[LVV-9851] DMS-PRTL-REQ-0012-V-01: Query Results Size Limitation\_1 }\label{lvv-9851}

\begin{longtable}{cccc}
\hline
\textbf{Jira Link} & \textbf{Assignee} & \textbf{Status} & \textbf{Test Cases}\\ \hline
\href{https://jira.lsstcorp.org/browse/LVV-9851}{LVV-9851} &
Gregory Dubois-Felsmann & Not Covered &
\begin{tabular}{c}
LVV-T645 \\
\end{tabular}
\\
\hline
\end{longtable}

\textbf{Verification Element Description:} \\
Undefined

{\footnotesize
\begin{longtable}{p{2.5cm}p{13.5cm}}
\hline
\multicolumn{2}{c}{\textbf{Requirement Details}}\\ \hline
Requirement ID & DMS-PRTL-REQ-0012 \\ \cdashline{1-2}
Requirement Description &
\begin{minipage}[]{13cm}
The Portal aspect shall have a mechanism to notify a user that a query
result is predicted to exceed, or has exceeded, threshold(s) for the
maximum results size allowed and that the query has been disallowed or
terminated as a result.
\end{minipage}
\\ \cdashline{1-2}
Requirement Discussion &
\begin{minipage}[]{13cm}
If the size limitation is applied at run time as query results
accumulate, this is an important concrete sub-case of the
DMS-PRTL-REQ-0011 requirement; if it is applied earlier, it is a
separate mechanism.\\
The underlying DAX capabilities required to support this remain to be
specified in detail.
\end{minipage}
\\ \cdashline{1-2}
Requirement Priority &  \\ \cdashline{1-2}
Upper Level Requirement &
\begin{tabular}{cl}
\end{tabular}
\\ \hline
\end{longtable}
}


\subsubsection{Test Cases Summary}
\begin{longtable}{p{3cm}p{2.5cm}p{2.5cm}p{3cm}p{4cm}}
\toprule
\href{https://jira.lsstcorp.org/secure/Tests.jspa\#/testCase/LVV-T645}{LVV-T645} & \multicolumn{4}{p{12cm}}{ Verify limitation of query results size } \\ \hline
\textbf{Owner} & \textbf{Status} & \textbf{Version} & \textbf{Critical Event} & \textbf{Verification Type} \\ \hline
Jeffrey Carlin & Draft & 1 & false & Test \\ \hline
\end{longtable}
{\scriptsize
\textbf{Objective:}\\
Verify that the Portal aspect estimates query results size, and notifies
user if the query result exceeds thresholds and has been disallowed or
terminated as a result.
}
  
 \newpage 
\subsection{[LVV-9852] DMS-PRTL-REQ-0014-V-01: Query Saving - Portal\_1 }\label{lvv-9852}

\begin{longtable}{cccc}
\hline
\textbf{Jira Link} & \textbf{Assignee} & \textbf{Status} & \textbf{Test Cases}\\ \hline
\href{https://jira.lsstcorp.org/browse/LVV-9852}{LVV-9852} &
Gregory Dubois-Felsmann & Not Covered &
\begin{tabular}{c}
LVV-T647 \\
\end{tabular}
\\
\hline
\end{longtable}

\textbf{Verification Element Description:} \\
Undefined

{\footnotesize
\begin{longtable}{p{2.5cm}p{13.5cm}}
\hline
\multicolumn{2}{c}{\textbf{Requirement Details}}\\ \hline
Requirement ID & DMS-PRTL-REQ-0014 \\ \cdashline{1-2}
Requirement Description &
\begin{minipage}[]{13cm}
The Portal aspect shall provide a UI for the saving of a specification
artifact for a user-performed query, either for downloading or for
saving to the Workspace, and a UI for re-executing a saved query found
in the Workspace or uploaded remotely.
\end{minipage}
\\ \cdashline{1-2}
Requirement Priority &  \\ \cdashline{1-2}
Upper Level Requirement &
\begin{tabular}{cl}
\end{tabular}
\\ \hline
\end{longtable}
}


\subsubsection{Test Cases Summary}
\begin{longtable}{p{3cm}p{2.5cm}p{2.5cm}p{3cm}p{4cm}}
\toprule
\href{https://jira.lsstcorp.org/secure/Tests.jspa\#/testCase/LVV-T647}{LVV-T647} & \multicolumn{4}{p{12cm}}{ Verify implementation of saving of queries } \\ \hline
\textbf{Owner} & \textbf{Status} & \textbf{Version} & \textbf{Critical Event} & \textbf{Verification Type} \\ \hline
Jeffrey Carlin & Draft & 1 & false & Inspection \\ \hline
\end{longtable}
{\scriptsize
\textbf{Objective:}\\
The Portal aspect shall provide a UI for the saving of a specification
artifact for a user-performed query, either for downloading or for
saving to the Workspace, and a UI for re-executing a saved query found
in the Workspace or uploaded remotely.~
}
  
 \newpage 
\subsection{[LVV-9853] DMS-PRTL-REQ-0011-V-01: Query Status and Termination Notification\_1 }\label{lvv-9853}

\begin{longtable}{cccc}
\hline
\textbf{Jira Link} & \textbf{Assignee} & \textbf{Status} & \textbf{Test Cases}\\ \hline
\href{https://jira.lsstcorp.org/browse/LVV-9853}{LVV-9853} &
Gregory Dubois-Felsmann & Not Covered &
\begin{tabular}{c}
LVV-T644 \\
\end{tabular}
\\
\hline
\end{longtable}

\textbf{Verification Element Description:} \\
Undefined

{\footnotesize
\begin{longtable}{p{2.5cm}p{13.5cm}}
\hline
\multicolumn{2}{c}{\textbf{Requirement Details}}\\ \hline
Requirement ID & DMS-PRTL-REQ-0011 \\ \cdashline{1-2}
Requirement Description &
\begin{minipage}[]{13cm}
The Portal aspect shall notify the user of the status of user-initiated
database queries, including whether the query has been terminated for
any reason.
\end{minipage}
\\ \cdashline{1-2}
Requirement Discussion &
\begin{minipage}[]{13cm}
There needs to be an operations discussion about when and why database
queries are terminated, and what sorts of notifications will be
generated by DAX in such cases.\\
How the Portal Aspect will notify the user is not specified by this
requirement; it might be by email to a registered address, or it might
be through a notification mechanism in the Portal UI, or something else.
\end{minipage}
\\ \cdashline{1-2}
Requirement Priority &  \\ \cdashline{1-2}
Upper Level Requirement &
\begin{tabular}{cl}
\end{tabular}
\\ \hline
\end{longtable}
}


\subsubsection{Test Cases Summary}
\begin{longtable}{p{3cm}p{2.5cm}p{2.5cm}p{3cm}p{4cm}}
\toprule
\href{https://jira.lsstcorp.org/secure/Tests.jspa\#/testCase/LVV-T644}{LVV-T644} & \multicolumn{4}{p{12cm}}{ Verify user notification of query status } \\ \hline
\textbf{Owner} & \textbf{Status} & \textbf{Version} & \textbf{Critical Event} & \textbf{Verification Type} \\ \hline
Jeffrey Carlin & Draft & 1 & false & Inspection \\ \hline
\end{longtable}
{\scriptsize
\textbf{Objective:}\\
Verify that the Portal notifies the user of the status of user-initiated
queries, including whether the query has been terminated for any reason.
}
  
 \newpage 
\subsection{[LVV-9854] DMS-PRTL-REQ-0009-V-01: Support Synchronous and Asynchronous Queries\_1 }\label{lvv-9854}

\begin{longtable}{cccc}
\hline
\textbf{Jira Link} & \textbf{Assignee} & \textbf{Status} & \textbf{Test Cases}\\ \hline
\href{https://jira.lsstcorp.org/browse/LVV-9854}{LVV-9854} &
Gregory Dubois-Felsmann & Not Covered &
\begin{tabular}{c}
LVV-T642 \\
\end{tabular}
\\
\hline
\end{longtable}

\textbf{Verification Element Description:} \\
Undefined

{\footnotesize
\begin{longtable}{p{2.5cm}p{13.5cm}}
\hline
\multicolumn{2}{c}{\textbf{Requirement Details}}\\ \hline
Requirement ID & DMS-PRTL-REQ-0009 \\ \cdashline{1-2}
Requirement Description &
\begin{minipage}[]{13cm}
The Portal aspect shall provide UI models for both synchronous and
asynchronous queries, based on user preference, loading, and resource
capabilities.
\end{minipage}
\\ \cdashline{1-2}
Requirement Discussion &
\begin{minipage}[]{13cm}
This Portal capability should include an interface to initiate, monitor,
and control the execution of both sync and async queries, as well as
browse their results. Long running queries may be forced to be
asynchronous.\\
There is not a 1:1 relationship between this Portal capability and the
analogous capability in the API aspect; for instance, the Portal may
provide UI support for placing a long-running, technically synchronous
(from the DAX perspective), query into the background from the user's
perspective.
\end{minipage}
\\ \cdashline{1-2}
Requirement Priority &  \\ \cdashline{1-2}
Upper Level Requirement &
\begin{tabular}{cl}
\end{tabular}
\\ \hline
\end{longtable}
}


\subsubsection{Test Cases Summary}
\begin{longtable}{p{3cm}p{2.5cm}p{2.5cm}p{3cm}p{4cm}}
\toprule
\href{https://jira.lsstcorp.org/secure/Tests.jspa\#/testCase/LVV-T642}{LVV-T642} & \multicolumn{4}{p{12cm}}{ Verify Portal supports both synchronous and asynchronous queries } \\ \hline
\textbf{Owner} & \textbf{Status} & \textbf{Version} & \textbf{Critical Event} & \textbf{Verification Type} \\ \hline
Jeffrey Carlin & Draft & 1 & false & Inspection \\ \hline
\end{longtable}
{\scriptsize
\textbf{Objective:}\\
Verify that the Portal aspect provides UI models for both synchronous
and asynchronous queries. This Portal capability should include an
interface to initiate, monitor, and control the execution of both sync
and async queries, as well as browse their results. Long running queries
may be forced to be asynchronous.
}
  
 \newpage 
\subsection{[LVV-9855] DMS-PRTL-REQ-0017-V-01: Generic Query - ADQL-based\_1 }\label{lvv-9855}

\begin{longtable}{cccc}
\hline
\textbf{Jira Link} & \textbf{Assignee} & \textbf{Status} & \textbf{Test Cases}\\ \hline
\href{https://jira.lsstcorp.org/browse/LVV-9855}{LVV-9855} &
Gregory Dubois-Felsmann & Not Covered &
\begin{tabular}{c}
LVV-T650 \\
LVV-T1334 \\
\end{tabular}
\\
\hline
\end{longtable}

\textbf{Verification Element Description:} \\
Undefined

{\footnotesize
\begin{longtable}{p{2.5cm}p{13.5cm}}
\hline
\multicolumn{2}{c}{\textbf{Requirement Details}}\\ \hline
Requirement ID & DMS-PRTL-REQ-0017 \\ \cdashline{1-2}
Requirement Description &
\begin{minipage}[]{13cm}
The Portal aspect shall provide a means for entering a query against any
table directly in ADQL. This facility shall be available for every
table, including user-supplied tables.
\end{minipage}
\\ \cdashline{1-2}
Requirement Discussion &
\begin{minipage}[]{13cm}
This is essentially a pass-through to the underlying API aspect TAP
service.\\
UI support for selecting from available identifiers (e.g. table or
column names) for use in the ADQL, based on the Discovery and Reflection
API, would be highly desirable.\\
UI support for the specification of coordinate values for use in the
ADQL, based on the Spatial Query Parameters requirements below, would
also be desirable.
\end{minipage}
\\ \cdashline{1-2}
Requirement Priority &  \\ \cdashline{1-2}
Upper Level Requirement &
\begin{tabular}{cl}
\end{tabular}
\\ \hline
\end{longtable}
}


\subsubsection{Test Cases Summary}
\begin{longtable}{p{3cm}p{2.5cm}p{2.5cm}p{3cm}p{4cm}}
\toprule
\href{https://jira.lsstcorp.org/secure/Tests.jspa\#/testCase/LVV-T650}{LVV-T650} & \multicolumn{4}{p{12cm}}{ Verify implementation of ADQL-based generic query in API aspect } \\ \hline
\textbf{Owner} & \textbf{Status} & \textbf{Version} & \textbf{Critical Event} & \textbf{Verification Type} \\ \hline
Jeffrey Carlin & Draft & 1 & false & Inspection \\ \hline
\end{longtable}
{\scriptsize
\textbf{Objective:}\\
The Portal aspect shall provide a means for entering a query against any
table directly in ADQL. This facility shall be available for every
table, including user-supplied tables.
}
\begin{longtable}{p{3cm}p{2.5cm}p{2.5cm}p{3cm}p{4cm}}
\toprule
\href{https://jira.lsstcorp.org/secure/Tests.jspa\#/testCase/LVV-T1334}{LVV-T1334} & \multicolumn{4}{p{12cm}}{ LDM-503-10a: Portal Aspect tests for LSP with Authentication and TAP
milestone } \\ \hline
\textbf{Owner} & \textbf{Status} & \textbf{Version} & \textbf{Critical Event} & \textbf{Verification Type} \\ \hline
Gregory Dubois-Felsmann & Defined & 1 & false & Test \\ \hline
\end{longtable}
{\scriptsize
\textbf{Objective:}\\
This test case verifies that the Portal Aspect of the Science Platform
is accessible to authorized users through a login process, and that TAP
searches can be performed from the Portal Aspect UI.\\[2\baselineskip]In
so doing and in conjunction with the other LDM-503-10a test cases
collected under LVV-P48, it addresses all or part of the following
requirements:

\begin{itemize}
\tightlist
\item
  DMS-LSP-REQ-0002, DMS-LSP-REQ-0005, DMS-LSP-REQ-0006,
  DMS-LSP-REQ-0020, DMS-LSP-REQ-0022, DMS-LSP-REQ-0023, DMS-LSP-REQ-0024
\item
  DMS-PRTL-REQ-0001, DMS-PRTL-REQ-0015, DMS-PRTL-REQ-0016,
  DMS-PRTL-REQ-0017, DMS-PRTL-REQ-0020, DMS-PRTL-REQ-0026,
  DMS-PRTL-REQ-0049, and DMS-PRTL-REQ-0095, primarily
\end{itemize}

Note this test was not designed to perform a full verification of the
above requirements, but rather to demonstrate having reached a certain
level of partial capability during construction.
}
  
 \newpage 
\subsection{[LVV-9856] DMS-PRTL-REQ-0016-V-01: Generic Query - Form-based\_1 }\label{lvv-9856}

\begin{longtable}{cccc}
\hline
\textbf{Jira Link} & \textbf{Assignee} & \textbf{Status} & \textbf{Test Cases}\\ \hline
\href{https://jira.lsstcorp.org/browse/LVV-9856}{LVV-9856} &
Gregory Dubois-Felsmann & Not Covered &
\begin{tabular}{c}
LVV-T5 \\
LVV-T649 \\
LVV-T1334 \\
\end{tabular}
\\
\hline
\end{longtable}

\textbf{Verification Element Description:} \\
Undefined

{\footnotesize
\begin{longtable}{p{2.5cm}p{13.5cm}}
\hline
\multicolumn{2}{c}{\textbf{Requirement Details}}\\ \hline
Requirement ID & DMS-PRTL-REQ-0016 \\ \cdashline{1-2}
Requirement Description &
\begin{minipage}[]{13cm}
The Portal aspect shall provide a search-builder form-based interface
for generic table queries. This facility may have reduced functionality
for user tables for which the user has not provided full, or accurate,
metadata.
\end{minipage}
\\ \cdashline{1-2}
Requirement Discussion &
\begin{minipage}[]{13cm}
The Discovery and Reflection API will be used to construct a form
allowing query against any attribute of a table, taking the column
metadata provided by the Reflection API into account to display units
and other information that will help the user construct a meaningful
query.\\
We wish to allow and facilitate, but not require, users to provide
detailed metadata for the tables they create; when they do, the system
will take advantage of it.
\end{minipage}
\\ \cdashline{1-2}
Requirement Priority &  \\ \cdashline{1-2}
Upper Level Requirement &
\begin{tabular}{cl}
\end{tabular}
\\ \hline
\end{longtable}
}


\subsubsection{Test Cases Summary}
\begin{longtable}{p{3cm}p{2.5cm}p{2.5cm}p{3cm}p{4cm}}
\toprule
\href{https://jira.lsstcorp.org/secure/Tests.jspa\#/testCase/LVV-T5}{LVV-T5} & \multicolumn{4}{p{12cm}}{ LSP-00-15: Execution of basic catalog queries in the Portal } \\ \hline
\textbf{Owner} & \textbf{Status} & \textbf{Version} & \textbf{Critical Event} & \textbf{Verification Type} \\ \hline
Gregory Dubois-Felsmann & Deprecated & 1 & false & Test \\ \hline
\end{longtable}
{\scriptsize
\textbf{Objective:}\\
This test will test the functional requirements to be able to perform a
range of basic queries through the Portal Aspect of the LSP:

\begin{itemize}
\tightlist
\item
  Cone searches on the Object-like, ForcedSource-like, and Source-like
  WISE tables;~
\item
  Multi-target cone searches;
\item
  Form-based searches for exact equality, e.g., for row IDs; and
\item
  Form-based searches for sets of object attributes.
\end{itemize}

In addition, it tests the ability to download tabular query results from
the Portal Aspect.
}
\begin{longtable}{p{3cm}p{2.5cm}p{2.5cm}p{3cm}p{4cm}}
\toprule
\href{https://jira.lsstcorp.org/secure/Tests.jspa\#/testCase/LVV-T649}{LVV-T649} & \multicolumn{4}{p{12cm}}{ Verify implementation of form-based generic query in API aspect } \\ \hline
\textbf{Owner} & \textbf{Status} & \textbf{Version} & \textbf{Critical Event} & \textbf{Verification Type} \\ \hline
Jeffrey Carlin & Draft & 1 & false & Inspection \\ \hline
\end{longtable}
{\scriptsize
\textbf{Objective:}\\
The Portal aspect shall provide a search-builder form-based interface
for generic table queries. This facility may have reduced functionality
for user tables for which the user has not provided full, or accurate,
metadata.
}
\begin{longtable}{p{3cm}p{2.5cm}p{2.5cm}p{3cm}p{4cm}}
\toprule
\href{https://jira.lsstcorp.org/secure/Tests.jspa\#/testCase/LVV-T1334}{LVV-T1334} & \multicolumn{4}{p{12cm}}{ LDM-503-10a: Portal Aspect tests for LSP with Authentication and TAP
milestone } \\ \hline
\textbf{Owner} & \textbf{Status} & \textbf{Version} & \textbf{Critical Event} & \textbf{Verification Type} \\ \hline
Gregory Dubois-Felsmann & Defined & 1 & false & Test \\ \hline
\end{longtable}
{\scriptsize
\textbf{Objective:}\\
This test case verifies that the Portal Aspect of the Science Platform
is accessible to authorized users through a login process, and that TAP
searches can be performed from the Portal Aspect UI.\\[2\baselineskip]In
so doing and in conjunction with the other LDM-503-10a test cases
collected under LVV-P48, it addresses all or part of the following
requirements:

\begin{itemize}
\tightlist
\item
  DMS-LSP-REQ-0002, DMS-LSP-REQ-0005, DMS-LSP-REQ-0006,
  DMS-LSP-REQ-0020, DMS-LSP-REQ-0022, DMS-LSP-REQ-0023, DMS-LSP-REQ-0024
\item
  DMS-PRTL-REQ-0001, DMS-PRTL-REQ-0015, DMS-PRTL-REQ-0016,
  DMS-PRTL-REQ-0017, DMS-PRTL-REQ-0020, DMS-PRTL-REQ-0026,
  DMS-PRTL-REQ-0049, and DMS-PRTL-REQ-0095, primarily
\end{itemize}

Note this test was not designed to perform a full verification of the
above requirements, but rather to demonstrate having reached a certain
level of partial capability during construction.
}
  
 \newpage 
\subsection{[LVV-9857] DMS-PRTL-REQ-0015-V-01: Generic Query\_1 }\label{lvv-9857}

\begin{longtable}{cccc}
\hline
\textbf{Jira Link} & \textbf{Assignee} & \textbf{Status} & \textbf{Test Cases}\\ \hline
\href{https://jira.lsstcorp.org/browse/LVV-9857}{LVV-9857} &
Gregory Dubois-Felsmann & Not Covered &
\begin{tabular}{c}
LVV-T648 \\
LVV-T1334 \\
\end{tabular}
\\
\hline
\end{longtable}

\textbf{Verification Element Description:} \\
Undefined

{\footnotesize
\begin{longtable}{p{2.5cm}p{13.5cm}}
\hline
\multicolumn{2}{c}{\textbf{Requirement Details}}\\ \hline
Requirement ID & DMS-PRTL-REQ-0015 \\ \cdashline{1-2}
Requirement Description &
\begin{minipage}[]{13cm}
The Portal aspect shall enable the generation of queries against any
tabular data exposed in the API aspect.
\end{minipage}
\\ \cdashline{1-2}
Requirement Discussion &
\begin{minipage}[]{13cm}
Provision of this relies on the Discovery and Reflection API
requirement. The idea is that some level of query will automatically be
provided for every table even for tables for which no
deliberately-designed custom search screens are available.\\
Note that image metadata is tabular data, so a basic level of ability to
query for lists of images is available through ``generic queries'',
though more image-specific workflows will be provided.
\end{minipage}
\\ \cdashline{1-2}
Requirement Priority &  \\ \cdashline{1-2}
Upper Level Requirement &
\begin{tabular}{cl}
\end{tabular}
\\ \hline
\end{longtable}
}


\subsubsection{Test Cases Summary}
\begin{longtable}{p{3cm}p{2.5cm}p{2.5cm}p{3cm}p{4cm}}
\toprule
\href{https://jira.lsstcorp.org/secure/Tests.jspa\#/testCase/LVV-T648}{LVV-T648} & \multicolumn{4}{p{12cm}}{ Verify implementation of generic queries in API aspect } \\ \hline
\textbf{Owner} & \textbf{Status} & \textbf{Version} & \textbf{Critical Event} & \textbf{Verification Type} \\ \hline
Jeffrey Carlin & Draft & 1 & false & Test \\ \hline
\end{longtable}
{\scriptsize
\textbf{Objective:}\\
The Portal aspect shall enable the generation of queries against any
tabular data exposed in the API aspect.
}
\begin{longtable}{p{3cm}p{2.5cm}p{2.5cm}p{3cm}p{4cm}}
\toprule
\href{https://jira.lsstcorp.org/secure/Tests.jspa\#/testCase/LVV-T1334}{LVV-T1334} & \multicolumn{4}{p{12cm}}{ LDM-503-10a: Portal Aspect tests for LSP with Authentication and TAP
milestone } \\ \hline
\textbf{Owner} & \textbf{Status} & \textbf{Version} & \textbf{Critical Event} & \textbf{Verification Type} \\ \hline
Gregory Dubois-Felsmann & Defined & 1 & false & Test \\ \hline
\end{longtable}
{\scriptsize
\textbf{Objective:}\\
This test case verifies that the Portal Aspect of the Science Platform
is accessible to authorized users through a login process, and that TAP
searches can be performed from the Portal Aspect UI.\\[2\baselineskip]In
so doing and in conjunction with the other LDM-503-10a test cases
collected under LVV-P48, it addresses all or part of the following
requirements:

\begin{itemize}
\tightlist
\item
  DMS-LSP-REQ-0002, DMS-LSP-REQ-0005, DMS-LSP-REQ-0006,
  DMS-LSP-REQ-0020, DMS-LSP-REQ-0022, DMS-LSP-REQ-0023, DMS-LSP-REQ-0024
\item
  DMS-PRTL-REQ-0001, DMS-PRTL-REQ-0015, DMS-PRTL-REQ-0016,
  DMS-PRTL-REQ-0017, DMS-PRTL-REQ-0020, DMS-PRTL-REQ-0026,
  DMS-PRTL-REQ-0049, and DMS-PRTL-REQ-0095, primarily
\end{itemize}

Note this test was not designed to perform a full verification of the
above requirements, but rather to demonstrate having reached a certain
level of partial capability during construction.
}
  
 \newpage 
\subsection{[LVV-9858] DMS-PRTL-REQ-0018-V-01: Query Result Size\_1 }\label{lvv-9858}

\begin{longtable}{cccc}
\hline
\textbf{Jira Link} & \textbf{Assignee} & \textbf{Status} & \textbf{Test Cases}\\ \hline
\href{https://jira.lsstcorp.org/browse/LVV-9858}{LVV-9858} &
Gregory Dubois-Felsmann & Not Covered &
\begin{tabular}{c}
LVV-T651 \\
\end{tabular}
\\
\hline
\end{longtable}

\textbf{Verification Element Description:} \\
Undefined

{\footnotesize
\begin{longtable}{p{2.5cm}p{13.5cm}}
\hline
\multicolumn{2}{c}{\textbf{Requirement Details}}\\ \hline
Requirement ID & DMS-PRTL-REQ-0018 \\ \cdashline{1-2}
Requirement Description &
\begin{minipage}[]{13cm}
The Portal aspect shall provide UI support for any mechanism provided by
the API Aspect for determining or estimating the number of rows matching
the query criteria without generating a full return set.
\end{minipage}
\\ \cdashline{1-2}
Requirement Discussion &
\begin{minipage}[]{13cm}
What level of support for this will be provided by the DAX services
remains to be determined, so the requirement has been written to
accommodate that uncertainty.\\
A proper implementation would rely on row-counting in the ADQL submitted
to the DAX services.
\end{minipage}
\\ \cdashline{1-2}
Requirement Priority &  \\ \cdashline{1-2}
Upper Level Requirement &
\begin{tabular}{cl}
\end{tabular}
\\ \hline
\end{longtable}
}


\subsubsection{Test Cases Summary}
\begin{longtable}{p{3cm}p{2.5cm}p{2.5cm}p{3cm}p{4cm}}
\toprule
\href{https://jira.lsstcorp.org/secure/Tests.jspa\#/testCase/LVV-T651}{LVV-T651} & \multicolumn{4}{p{12cm}}{ Verify estimation of query result size } \\ \hline
\textbf{Owner} & \textbf{Status} & \textbf{Version} & \textbf{Critical Event} & \textbf{Verification Type} \\ \hline
Jeffrey Carlin & Draft & 1 & false & Inspection \\ \hline
\end{longtable}
{\scriptsize
\textbf{Objective:}\\
Verify that UI support exists to estimate (or determine exactly) the
size of results that would be returned by a query without returning the
full set of results.
}
  
 \newpage 
\subsection{[LVV-9859] DMS-PRTL-REQ-0028-V-01: Query by Identifier\_1 }\label{lvv-9859}

\begin{longtable}{cccc}
\hline
\textbf{Jira Link} & \textbf{Assignee} & \textbf{Status} & \textbf{Test Cases}\\ \hline
\href{https://jira.lsstcorp.org/browse/LVV-9859}{LVV-9859} &
Gregory Dubois-Felsmann & Not Covered &
\begin{tabular}{c}
LVV-T5 \\
LVV-T652 \\
\end{tabular}
\\
\hline
\end{longtable}

\textbf{Verification Element Description:} \\
Undefined

{\footnotesize
\begin{longtable}{p{2.5cm}p{13.5cm}}
\hline
\multicolumn{2}{c}{\textbf{Requirement Details}}\\ \hline
Requirement ID & DMS-PRTL-REQ-0028 \\ \cdashline{1-2}
Requirement Description &
\begin{minipage}[]{13cm}
The Portal aspect shall provide UI support for performing queries for
data on any LSST data product entity with a unique ID by that ID.
\end{minipage}
\\ \cdashline{1-2}
Requirement Discussion &
\begin{minipage}[]{13cm}
Given, e.g., an Object ID or a Visit ID, it should be possible to
perform a query by that ID and retrieve all, or a selection of, the data
for the corresponding Object or Visit. This requirement is limited to
the direct content of the corresponding table row; the following
requirements add semantic workflows returning additional data.\\
It is desirable for this capability to be available wherever such a
unique ID is displayed in the UI, though the mandatory requirement is
only for a dedicated form for such a query.
\end{minipage}
\\ \cdashline{1-2}
Requirement Priority &  \\ \cdashline{1-2}
Upper Level Requirement &
\begin{tabular}{cl}
\end{tabular}
\\ \hline
\end{longtable}
}


\subsubsection{Test Cases Summary}
\begin{longtable}{p{3cm}p{2.5cm}p{2.5cm}p{3cm}p{4cm}}
\toprule
\href{https://jira.lsstcorp.org/secure/Tests.jspa\#/testCase/LVV-T5}{LVV-T5} & \multicolumn{4}{p{12cm}}{ LSP-00-15: Execution of basic catalog queries in the Portal } \\ \hline
\textbf{Owner} & \textbf{Status} & \textbf{Version} & \textbf{Critical Event} & \textbf{Verification Type} \\ \hline
Gregory Dubois-Felsmann & Deprecated & 1 & false & Test \\ \hline
\end{longtable}
{\scriptsize
\textbf{Objective:}\\
This test will test the functional requirements to be able to perform a
range of basic queries through the Portal Aspect of the LSP:

\begin{itemize}
\tightlist
\item
  Cone searches on the Object-like, ForcedSource-like, and Source-like
  WISE tables;~
\item
  Multi-target cone searches;
\item
  Form-based searches for exact equality, e.g., for row IDs; and
\item
  Form-based searches for sets of object attributes.
\end{itemize}

In addition, it tests the ability to download tabular query results from
the Portal Aspect.
}
\begin{longtable}{p{3cm}p{2.5cm}p{2.5cm}p{3cm}p{4cm}}
\toprule
\href{https://jira.lsstcorp.org/secure/Tests.jspa\#/testCase/LVV-T652}{LVV-T652} & \multicolumn{4}{p{12cm}}{ Verify query by unique identifier } \\ \hline
\textbf{Owner} & \textbf{Status} & \textbf{Version} & \textbf{Critical Event} & \textbf{Verification Type} \\ \hline
Jeffrey Carlin & Draft & 1 & false & Test \\ \hline
\end{longtable}
{\scriptsize
\textbf{Objective:}\\
Verify that queries can be performed to find data on any LSST data
product with a unique ID by that ID.
}
  
 \newpage 
\subsection{[LVV-9860] DMS-PRTL-REQ-0029-V-01: Query by LSST Object and Source Identifiers:
Specific Match to Identifier\_1 }\label{lvv-9860}

\begin{longtable}{cccc}
\hline
\textbf{Jira Link} & \textbf{Assignee} & \textbf{Status} & \textbf{Test Cases}\\ \hline
\href{https://jira.lsstcorp.org/browse/LVV-9860}{LVV-9860} &
Gregory Dubois-Felsmann & Not Covered &
\begin{tabular}{c}
LVV-T653 \\
\end{tabular}
\\
\hline
\end{longtable}

\textbf{Verification Element Description:} \\
Undefined

{\footnotesize
\begin{longtable}{p{2.5cm}p{13.5cm}}
\hline
\multicolumn{2}{c}{\textbf{Requirement Details}}\\ \hline
Requirement ID & DMS-PRTL-REQ-0029 \\ \cdashline{1-2}
Requirement Description &
\begin{minipage}[]{13cm}
The Portal aspect shall provide UI support to query and return data
associated with a specific LSST (DIA)Object, (DIA)Source, or
ForcedSource identifier, including catalog data associated with the
entity as well as the image data and metadata directly associated with
the measurement.
\end{minipage}
\\ \cdashline{1-2}
Requirement Discussion &
\begin{minipage}[]{13cm}
For example, for an Object ID query, the Portal will, upon request,
return information on the coadded images on which the Object was
observed. For a ForcedSource ID, the Portal should return information on
the associated Object and on the single-epoch image on which the
ForcedSource measurement was made.
\end{minipage}
\\ \cdashline{1-2}
Requirement Priority &  \\ \cdashline{1-2}
Upper Level Requirement &
\begin{tabular}{cl}
\end{tabular}
\\ \hline
\end{longtable}
}


\subsubsection{Test Cases Summary}
\begin{longtable}{p{3cm}p{2.5cm}p{2.5cm}p{3cm}p{4cm}}
\toprule
\href{https://jira.lsstcorp.org/secure/Tests.jspa\#/testCase/LVV-T653}{LVV-T653} & \multicolumn{4}{p{12cm}}{ Verify query by object or source identifier } \\ \hline
\textbf{Owner} & \textbf{Status} & \textbf{Version} & \textbf{Critical Event} & \textbf{Verification Type} \\ \hline
Jeffrey Carlin & Draft & 1 & false & Inspection \\ \hline
\end{longtable}
{\scriptsize
\textbf{Objective:}\\
Verify that queries can be performed for a given object or source ID
(e.g., (DIA)Object, (DIA)Source, ForcedSource), and return catalog,
image, and metadata associated with measurements of the object/source.
}
  
 \newpage 
\subsection{[LVV-9861] DMS-PRTL-REQ-0030-V-01: Query by Solar System Objects: Specific Match to
Identifier\_1 }\label{lvv-9861}

\begin{longtable}{cccc}
\hline
\textbf{Jira Link} & \textbf{Assignee} & \textbf{Status} & \textbf{Test Cases}\\ \hline
\href{https://jira.lsstcorp.org/browse/LVV-9861}{LVV-9861} &
Gregory Dubois-Felsmann & Not Covered &
\begin{tabular}{c}
LVV-T654 \\
\end{tabular}
\\
\hline
\end{longtable}

\textbf{Verification Element Description:} \\
Undefined

{\footnotesize
\begin{longtable}{p{2.5cm}p{13.5cm}}
\hline
\multicolumn{2}{c}{\textbf{Requirement Details}}\\ \hline
Requirement ID & DMS-PRTL-REQ-0030 \\ \cdashline{1-2}
Requirement Description &
\begin{minipage}[]{13cm}
The Portal aspect shall provide UI support to query and return data
specifically associated with a Solar System Object.
\end{minipage}
\\ \cdashline{1-2}
Requirement Discussion &
\begin{minipage}[]{13cm}
The Portal will need to be able to return the data (catalog and images)
associated with a specific solar system identifier and that cover the
time range wanted. This requires that the SSO database will have
identified known solar objects that have been observed.
\end{minipage}
\\ \cdashline{1-2}
Requirement Priority &  \\ \cdashline{1-2}
Upper Level Requirement &
\begin{tabular}{cl}
\end{tabular}
\\ \hline
\end{longtable}
}


\subsubsection{Test Cases Summary}
\begin{longtable}{p{3cm}p{2.5cm}p{2.5cm}p{3cm}p{4cm}}
\toprule
\href{https://jira.lsstcorp.org/secure/Tests.jspa\#/testCase/LVV-T654}{LVV-T654} & \multicolumn{4}{p{12cm}}{ Verify query by Solar System object identifier } \\ \hline
\textbf{Owner} & \textbf{Status} & \textbf{Version} & \textbf{Critical Event} & \textbf{Verification Type} \\ \hline
Jeffrey Carlin & Draft & 1 & false & Inspection \\ \hline
\end{longtable}
{\scriptsize
\textbf{Objective:}\\
Verify that the UI supports queries and returns data associated with a
specific Solar System Object.
}
  
 \newpage 
\subsection{[LVV-9862] DMS-PRTL-REQ-0022-V-01: Positional Query: Astrophysical Coordinate
Systems\_1 }\label{lvv-9862}

\begin{longtable}{cccc}
\hline
\textbf{Jira Link} & \textbf{Assignee} & \textbf{Status} & \textbf{Test Cases}\\ \hline
\href{https://jira.lsstcorp.org/browse/LVV-9862}{LVV-9862} &
Gregory Dubois-Felsmann & Not Covered &
\begin{tabular}{c}
LVV-T5 \\
LVV-T657 \\
\end{tabular}
\\
\hline
\end{longtable}

\textbf{Verification Element Description:} \\
Undefined

{\footnotesize
\begin{longtable}{p{2.5cm}p{13.5cm}}
\hline
\multicolumn{2}{c}{\textbf{Requirement Details}}\\ \hline
Requirement ID & DMS-PRTL-REQ-0022 \\ \cdashline{1-2}
Requirement Description &
\begin{minipage}[]{13cm}
The Portal aspect shall support positional queries based on the
following astrophysical coordinate systems: equatorial, ecliptic, and
galactic.
\end{minipage}
\\ \cdashline{1-2}
Requirement Discussion &
\begin{minipage}[]{13cm}
The UI should include information on (and possibly selection of) the
particular definition of each available coordinate system.
\end{minipage}
\\ \cdashline{1-2}
Requirement Priority &  \\ \cdashline{1-2}
Upper Level Requirement &
\begin{tabular}{cl}
\end{tabular}
\\ \hline
\end{longtable}
}


\subsubsection{Test Cases Summary}
\begin{longtable}{p{3cm}p{2.5cm}p{2.5cm}p{3cm}p{4cm}}
\toprule
\href{https://jira.lsstcorp.org/secure/Tests.jspa\#/testCase/LVV-T5}{LVV-T5} & \multicolumn{4}{p{12cm}}{ LSP-00-15: Execution of basic catalog queries in the Portal } \\ \hline
\textbf{Owner} & \textbf{Status} & \textbf{Version} & \textbf{Critical Event} & \textbf{Verification Type} \\ \hline
Gregory Dubois-Felsmann & Deprecated & 1 & false & Test \\ \hline
\end{longtable}
{\scriptsize
\textbf{Objective:}\\
This test will test the functional requirements to be able to perform a
range of basic queries through the Portal Aspect of the LSP:

\begin{itemize}
\tightlist
\item
  Cone searches on the Object-like, ForcedSource-like, and Source-like
  WISE tables;~
\item
  Multi-target cone searches;
\item
  Form-based searches for exact equality, e.g., for row IDs; and
\item
  Form-based searches for sets of object attributes.
\end{itemize}

In addition, it tests the ability to download tabular query results from
the Portal Aspect.
}
\begin{longtable}{p{3cm}p{2.5cm}p{2.5cm}p{3cm}p{4cm}}
\toprule
\href{https://jira.lsstcorp.org/secure/Tests.jspa\#/testCase/LVV-T657}{LVV-T657} & \multicolumn{4}{p{12cm}}{ Verify implementation of astrophysical coordinate systems } \\ \hline
\textbf{Owner} & \textbf{Status} & \textbf{Version} & \textbf{Critical Event} & \textbf{Verification Type} \\ \hline
Jeffrey Carlin & Draft & 1 & false & Inspection \\ \hline
\end{longtable}
{\scriptsize
\textbf{Objective:}\\
Verify that the Portal aspect supports positional queries based on
equatorial, ecliptic, and Galactic astrophysical coordinate systems.
}
  
 \newpage 
\subsection{[LVV-9863] DMS-PRTL-REQ-0023-V-01: Positional Query: Astrophysical Source Name
Lookup\_1 }\label{lvv-9863}

\begin{longtable}{cccc}
\hline
\textbf{Jira Link} & \textbf{Assignee} & \textbf{Status} & \textbf{Test Cases}\\ \hline
\href{https://jira.lsstcorp.org/browse/LVV-9863}{LVV-9863} &
Gregory Dubois-Felsmann & Not Covered &
\begin{tabular}{c}
LVV-T658 \\
\end{tabular}
\\
\hline
\end{longtable}

\textbf{Verification Element Description:} \\
Undefined

{\footnotesize
\begin{longtable}{p{2.5cm}p{13.5cm}}
\hline
\multicolumn{2}{c}{\textbf{Requirement Details}}\\ \hline
Requirement ID & DMS-PRTL-REQ-0023 \\ \cdashline{1-2}
Requirement Description &
\begin{minipage}[]{13cm}
The Portal aspect shall support the specification of coordinates for use
within all positional queries by the use of source names in common
community-established astrophysical source name lookup services.
\end{minipage}
\\ \cdashline{1-2}
Requirement Discussion &
\begin{minipage}[]{13cm}
Services include, but are not limited to, NED, SIMBAD, and Horizons
\end{minipage}
\\ \cdashline{1-2}
Requirement Priority &  \\ \cdashline{1-2}
Upper Level Requirement &
\begin{tabular}{cl}
\end{tabular}
\\ \hline
\end{longtable}
}


\subsubsection{Test Cases Summary}
\begin{longtable}{p{3cm}p{2.5cm}p{2.5cm}p{3cm}p{4cm}}
\toprule
\href{https://jira.lsstcorp.org/secure/Tests.jspa\#/testCase/LVV-T658}{LVV-T658} & \multicolumn{4}{p{12cm}}{ Verify positional query by astrophysical source name } \\ \hline
\textbf{Owner} & \textbf{Status} & \textbf{Version} & \textbf{Critical Event} & \textbf{Verification Type} \\ \hline
Jeffrey Carlin & Draft & 1 & false & Inspection \\ \hline
\end{longtable}
{\scriptsize
\textbf{Objective:}\\
Verify that the Portal aspect supports queries based on the use of
source names in commonly-used astrophysical source name lookup services
(e.g., NED, Simbad, Horizons).
}
  
 \newpage 
\subsection{[LVV-9864] DMS-PRTL-REQ-0024-V-01: Positional Query: LSST Object and Source
Identifiers\_1 }\label{lvv-9864}

\begin{longtable}{cccc}
\hline
\textbf{Jira Link} & \textbf{Assignee} & \textbf{Status} & \textbf{Test Cases}\\ \hline
\href{https://jira.lsstcorp.org/browse/LVV-9864}{LVV-9864} &
Gregory Dubois-Felsmann & Not Covered &
\begin{tabular}{c}
LVV-T659 \\
\end{tabular}
\\
\hline
\end{longtable}

\textbf{Verification Element Description:} \\
Undefined

{\footnotesize
\begin{longtable}{p{2.5cm}p{13.5cm}}
\hline
\multicolumn{2}{c}{\textbf{Requirement Details}}\\ \hline
Requirement ID & DMS-PRTL-REQ-0024 \\ \cdashline{1-2}
Requirement Description &
\begin{minipage}[]{13cm}
The Portal aspect shall support the specification of coordinates for use
within all positional queries by the use of specific LSST catalog entry
identifiers, including those for the Object, DIAObject, Source, and
DIASource tables. The default choice of coordinate columns within these
tables to use for the ID-to-coordinate translation shall be documented
and shall be able to be determined from the UI.
\end{minipage}
\\ \cdashline{1-2}
Requirement Discussion &
\begin{minipage}[]{13cm}
LSST object and source identifiers are not necessarily associated with
previously known astrophysical sources. The Portal will need to be able
to interpret source and object names and return positions that can be
used in positional-based searches.
\end{minipage}
\\ \cdashline{1-2}
Requirement Priority &  \\ \cdashline{1-2}
Upper Level Requirement &
\begin{tabular}{cl}
\end{tabular}
\\ \hline
\end{longtable}
}


\subsubsection{Test Cases Summary}
\begin{longtable}{p{3cm}p{2.5cm}p{2.5cm}p{3cm}p{4cm}}
\toprule
\href{https://jira.lsstcorp.org/secure/Tests.jspa\#/testCase/LVV-T659}{LVV-T659} & \multicolumn{4}{p{12cm}}{ Verify positional query by Source or Object name } \\ \hline
\textbf{Owner} & \textbf{Status} & \textbf{Version} & \textbf{Critical Event} & \textbf{Verification Type} \\ \hline
Jeffrey Carlin & Draft & 1 & false & Test \\ \hline
\end{longtable}
{\scriptsize
\textbf{Objective:}\\
Verify that positional queries can be performed for coordinates based on
a given object or source ID (e.g., (DIA)Object, (DIA)Source,
ForcedSource).
}
  
 \newpage 
\subsection{[LVV-9865] DMS-PRTL-REQ-0021-V-01: Positional Query: Multiple Positions/Objects\_1 }\label{lvv-9865}

\begin{longtable}{cccc}
\hline
\textbf{Jira Link} & \textbf{Assignee} & \textbf{Status} & \textbf{Test Cases}\\ \hline
\href{https://jira.lsstcorp.org/browse/LVV-9865}{LVV-9865} &
Gregory Dubois-Felsmann & Not Covered &
\begin{tabular}{c}
LVV-T5 \\
LVV-T656 \\
\end{tabular}
\\
\hline
\end{longtable}

\textbf{Verification Element Description:} \\
Undefined

{\footnotesize
\begin{longtable}{p{2.5cm}p{13.5cm}}
\hline
\multicolumn{2}{c}{\textbf{Requirement Details}}\\ \hline
Requirement ID & DMS-PRTL-REQ-0021 \\ \cdashline{1-2}
Requirement Description &
\begin{minipage}[]{13cm}
The Portal aspect shall support list-based positional queries, with the
coordinates used specified by any of the means of specifying positions
required elsewhere herein.
\end{minipage}
\\ \cdashline{1-2}
Requirement Discussion &
\begin{minipage}[]{13cm}
Lists of positions may be uploaded or taken from the Workspace.
Efficient implementation of list-based queries requires a corresponding
API aspect / DAX service, to avoid the submission of large numbers of
separate queries.\\
The system is not required to, but may, accept lists in which different
elements are specified in more than one of the supported means of
specifying positions; e.g., it is not required to support a list that is
partially in equatorial coordinates and partially in LSST object IDs,
because of the parsing and interpretation complexities involved. The
implementation should not preclude adding the ability to mix
specification types later one, guided by feedback from the user
community as to what is most useful.
\end{minipage}
\\ \cdashline{1-2}
Requirement Priority &  \\ \cdashline{1-2}
Upper Level Requirement &
\begin{tabular}{cl}
\end{tabular}
\\ \hline
\end{longtable}
}


\subsubsection{Test Cases Summary}
\begin{longtable}{p{3cm}p{2.5cm}p{2.5cm}p{3cm}p{4cm}}
\toprule
\href{https://jira.lsstcorp.org/secure/Tests.jspa\#/testCase/LVV-T5}{LVV-T5} & \multicolumn{4}{p{12cm}}{ LSP-00-15: Execution of basic catalog queries in the Portal } \\ \hline
\textbf{Owner} & \textbf{Status} & \textbf{Version} & \textbf{Critical Event} & \textbf{Verification Type} \\ \hline
Gregory Dubois-Felsmann & Deprecated & 1 & false & Test \\ \hline
\end{longtable}
{\scriptsize
\textbf{Objective:}\\
This test will test the functional requirements to be able to perform a
range of basic queries through the Portal Aspect of the LSP:

\begin{itemize}
\tightlist
\item
  Cone searches on the Object-like, ForcedSource-like, and Source-like
  WISE tables;~
\item
  Multi-target cone searches;
\item
  Form-based searches for exact equality, e.g., for row IDs; and
\item
  Form-based searches for sets of object attributes.
\end{itemize}

In addition, it tests the ability to download tabular query results from
the Portal Aspect.
}
\begin{longtable}{p{3cm}p{2.5cm}p{2.5cm}p{3cm}p{4cm}}
\toprule
\href{https://jira.lsstcorp.org/secure/Tests.jspa\#/testCase/LVV-T656}{LVV-T656} & \multicolumn{4}{p{12cm}}{ Verify query by list of positions } \\ \hline
\textbf{Owner} & \textbf{Status} & \textbf{Version} & \textbf{Critical Event} & \textbf{Verification Type} \\ \hline
Jeffrey Carlin & Draft & 1 & false & Test \\ \hline
\end{longtable}
{\scriptsize
\textbf{Objective:}\\
Verify that the Portal supports queries based on a list of object
positions. The coordinates may be specified by any of the supported
means of specifying positions.
}
  
 \newpage 
\subsection{[LVV-9866] DMS-PRTL-REQ-0020-V-01: Positional Query: Position on the Sky\_1 }\label{lvv-9866}

\begin{longtable}{cccc}
\hline
\textbf{Jira Link} & \textbf{Assignee} & \textbf{Status} & \textbf{Test Cases}\\ \hline
\href{https://jira.lsstcorp.org/browse/LVV-9866}{LVV-9866} &
Gregory Dubois-Felsmann & Not Covered &
\begin{tabular}{c}
LVV-T655 \\
LVV-T1334 \\
\end{tabular}
\\
\hline
\end{longtable}

\textbf{Verification Element Description:} \\
Undefined

{\footnotesize
\begin{longtable}{p{2.5cm}p{13.5cm}}
\hline
\multicolumn{2}{c}{\textbf{Requirement Details}}\\ \hline
Requirement ID & DMS-PRTL-REQ-0020 \\ \cdashline{1-2}
Requirement Description &
\begin{minipage}[]{13cm}
The Portal aspect shall support queries based on an astrophysical
position (i.e., coordinates) on the sky.
\end{minipage}
\\ \cdashline{1-2}
Requirement Discussion &
\begin{minipage}[]{13cm}
The intent of this requirement is to enable users to search for data at
or near locations in equatorial, ecliptic or galactic coordinates and
not a specific astrophysical object.\\
The Portal is responsible for any conversion required to put the
coordinate query parameters into the forms accepted by the underlying
API aspect / DAX services.
\end{minipage}
\\ \cdashline{1-2}
Requirement Priority &  \\ \cdashline{1-2}
Upper Level Requirement &
\begin{tabular}{cl}
\end{tabular}
\\ \hline
\end{longtable}
}


\subsubsection{Test Cases Summary}
\begin{longtable}{p{3cm}p{2.5cm}p{2.5cm}p{3cm}p{4cm}}
\toprule
\href{https://jira.lsstcorp.org/secure/Tests.jspa\#/testCase/LVV-T655}{LVV-T655} & \multicolumn{4}{p{12cm}}{ Verify query by position on the sky } \\ \hline
\textbf{Owner} & \textbf{Status} & \textbf{Version} & \textbf{Critical Event} & \textbf{Verification Type} \\ \hline
Jeffrey Carlin & Draft & 1 & false & Inspection \\ \hline
\end{longtable}
{\scriptsize
\textbf{Objective:}\\
Verify that the Portal aspect supports queries based on astrophysical
coordinates on the sky.
}
\begin{longtable}{p{3cm}p{2.5cm}p{2.5cm}p{3cm}p{4cm}}
\toprule
\href{https://jira.lsstcorp.org/secure/Tests.jspa\#/testCase/LVV-T1334}{LVV-T1334} & \multicolumn{4}{p{12cm}}{ LDM-503-10a: Portal Aspect tests for LSP with Authentication and TAP
milestone } \\ \hline
\textbf{Owner} & \textbf{Status} & \textbf{Version} & \textbf{Critical Event} & \textbf{Verification Type} \\ \hline
Gregory Dubois-Felsmann & Defined & 1 & false & Test \\ \hline
\end{longtable}
{\scriptsize
\textbf{Objective:}\\
This test case verifies that the Portal Aspect of the Science Platform
is accessible to authorized users through a login process, and that TAP
searches can be performed from the Portal Aspect UI.\\[2\baselineskip]In
so doing and in conjunction with the other LDM-503-10a test cases
collected under LVV-P48, it addresses all or part of the following
requirements:

\begin{itemize}
\tightlist
\item
  DMS-LSP-REQ-0002, DMS-LSP-REQ-0005, DMS-LSP-REQ-0006,
  DMS-LSP-REQ-0020, DMS-LSP-REQ-0022, DMS-LSP-REQ-0023, DMS-LSP-REQ-0024
\item
  DMS-PRTL-REQ-0001, DMS-PRTL-REQ-0015, DMS-PRTL-REQ-0016,
  DMS-PRTL-REQ-0017, DMS-PRTL-REQ-0020, DMS-PRTL-REQ-0026,
  DMS-PRTL-REQ-0049, and DMS-PRTL-REQ-0095, primarily
\end{itemize}

Note this test was not designed to perform a full verification of the
above requirements, but rather to demonstrate having reached a certain
level of partial capability during construction.
}
  
 \newpage 
\subsection{[LVV-9868] DMS-PRTL-REQ-0027-V-01: Positional Query by Region: Box-Search\_1 }\label{lvv-9868}

\begin{longtable}{cccc}
\hline
\textbf{Jira Link} & \textbf{Assignee} & \textbf{Status} & \textbf{Test Cases}\\ \hline
\href{https://jira.lsstcorp.org/browse/LVV-9868}{LVV-9868} &
Gregory Dubois-Felsmann & Not Covered &
\begin{tabular}{c}
LVV-T5 \\
LVV-T662 \\
\end{tabular}
\\
\hline
\end{longtable}

\textbf{Verification Element Description:} \\
Undefined

{\footnotesize
\begin{longtable}{p{2.5cm}p{13.5cm}}
\hline
\multicolumn{2}{c}{\textbf{Requirement Details}}\\ \hline
Requirement ID & DMS-PRTL-REQ-0027 \\ \cdashline{1-2}
Requirement Description &
\begin{minipage}[]{13cm}
The Portal aspect shall support position-based queries based on a
coordinate-system box search.
\end{minipage}
\\ \cdashline{1-2}
Requirement Priority &  \\ \cdashline{1-2}
Upper Level Requirement &
\begin{tabular}{cl}
\end{tabular}
\\ \hline
\end{longtable}
}


\subsubsection{Test Cases Summary}
\begin{longtable}{p{3cm}p{2.5cm}p{2.5cm}p{3cm}p{4cm}}
\toprule
\href{https://jira.lsstcorp.org/secure/Tests.jspa\#/testCase/LVV-T5}{LVV-T5} & \multicolumn{4}{p{12cm}}{ LSP-00-15: Execution of basic catalog queries in the Portal } \\ \hline
\textbf{Owner} & \textbf{Status} & \textbf{Version} & \textbf{Critical Event} & \textbf{Verification Type} \\ \hline
Gregory Dubois-Felsmann & Deprecated & 1 & false & Test \\ \hline
\end{longtable}
{\scriptsize
\textbf{Objective:}\\
This test will test the functional requirements to be able to perform a
range of basic queries through the Portal Aspect of the LSP:

\begin{itemize}
\tightlist
\item
  Cone searches on the Object-like, ForcedSource-like, and Source-like
  WISE tables;~
\item
  Multi-target cone searches;
\item
  Form-based searches for exact equality, e.g., for row IDs; and
\item
  Form-based searches for sets of object attributes.
\end{itemize}

In addition, it tests the ability to download tabular query results from
the Portal Aspect.
}
\begin{longtable}{p{3cm}p{2.5cm}p{2.5cm}p{3cm}p{4cm}}
\toprule
\href{https://jira.lsstcorp.org/secure/Tests.jspa\#/testCase/LVV-T662}{LVV-T662} & \multicolumn{4}{p{12cm}}{ Verify query by box search } \\ \hline
\textbf{Owner} & \textbf{Status} & \textbf{Version} & \textbf{Critical Event} & \textbf{Verification Type} \\ \hline
Jeffrey Carlin & Draft & 1 & false & Test \\ \hline
\end{longtable}
{\scriptsize
\textbf{Objective:}\\
Verify that the Portal supports positional queries based on a coordinate
system box search.
}
  
 \newpage 
\subsection{[LVV-9870] DMS-PRTL-REQ-0019-V-01: Query by Date and Time: Time Range of
Observation\_1 }\label{lvv-9870}

\begin{longtable}{cccc}
\hline
\textbf{Jira Link} & \textbf{Assignee} & \textbf{Status} & \textbf{Test Cases}\\ \hline
\href{https://jira.lsstcorp.org/browse/LVV-9870}{LVV-9870} &
Gregory Dubois-Felsmann & Not Covered &
\begin{tabular}{c}
LVV-T663 \\
\end{tabular}
\\
\hline
\end{longtable}

\textbf{Verification Element Description:} \\
Undefined

{\footnotesize
\begin{longtable}{p{2.5cm}p{13.5cm}}
\hline
\multicolumn{2}{c}{\textbf{Requirement Details}}\\ \hline
Requirement ID & DMS-PRTL-REQ-0019 \\ \cdashline{1-2}
Requirement Description &
\begin{minipage}[]{13cm}
The Portal aspect shall support queries based on times and ranges of
date/time values in both UT and (barycentric) Julian date.
\end{minipage}
\\ \cdashline{1-2}
Requirement Discussion &
\begin{minipage}[]{13cm}
The intent of this requirement is to enable users to search for an event
within a range of times/dates and is akin to a positional box search of
images that overlap that time range or are fully enveloped in that time
range.\\
This type of query should be available for all time-point based tables
in the LSST data products (e.g., Visit, Source, ForcedSource, DIASource,
and EFD), and more generally for all tables with columns indicated as
times by their metadata (e.g., by UCDs).\\
The Portal is responsible for conversion of user-supplied times to the
appropriate form for use in the underlying API aspect and database
queries.
\end{minipage}
\\ \cdashline{1-2}
Requirement Priority &  \\ \cdashline{1-2}
Upper Level Requirement &
\begin{tabular}{cl}
\end{tabular}
\\ \hline
\end{longtable}
}


\subsubsection{Test Cases Summary}
\begin{longtable}{p{3cm}p{2.5cm}p{2.5cm}p{3cm}p{4cm}}
\toprule
\href{https://jira.lsstcorp.org/secure/Tests.jspa\#/testCase/LVV-T663}{LVV-T663} & \multicolumn{4}{p{12cm}}{ Verify query by time of observation } \\ \hline
\textbf{Owner} & \textbf{Status} & \textbf{Version} & \textbf{Critical Event} & \textbf{Verification Type} \\ \hline
Jeffrey Carlin & Draft & 1 & false & Test \\ \hline
\end{longtable}
{\scriptsize
\textbf{Objective:}\\
Verify that the Portal supports queries based on time or ranges of
date/time values in both UT and (barycentric) Julian date.
}
  
 \newpage 
\subsection{[LVV-9871] DMS-PRTL-REQ-0034-V-01: Access to Original Alert State\_1 }\label{lvv-9871}

\begin{longtable}{cccc}
\hline
\textbf{Jira Link} & \textbf{Assignee} & \textbf{Status} & \textbf{Test Cases}\\ \hline
\href{https://jira.lsstcorp.org/browse/LVV-9871}{LVV-9871} &
Gregory Dubois-Felsmann & Not Covered &
\begin{tabular}{c}
LVV-T668 \\
\end{tabular}
\\
\hline
\end{longtable}

\textbf{Verification Element Description:} \\
Undefined

{\footnotesize
\begin{longtable}{p{2.5cm}p{13.5cm}}
\hline
\multicolumn{2}{c}{\textbf{Requirement Details}}\\ \hline
Requirement ID & DMS-PRTL-REQ-0034 \\ \cdashline{1-2}
Requirement Description &
\begin{minipage}[]{13cm}
The Portal aspect shall provide access to the alerts as they were
originally raised.
\end{minipage}
\\ \cdashline{1-2}
Requirement Discussion &
\begin{minipage}[]{13cm}
This is in the context of features elsewhere in the system that may
allow for a chain of annotations of alerts.
\end{minipage}
\\ \cdashline{1-2}
Requirement Priority &  \\ \cdashline{1-2}
Upper Level Requirement &
\begin{tabular}{cl}
\end{tabular}
\\ \hline
\end{longtable}
}


\subsubsection{Test Cases Summary}
\begin{longtable}{p{3cm}p{2.5cm}p{2.5cm}p{3cm}p{4cm}}
\toprule
\href{https://jira.lsstcorp.org/secure/Tests.jspa\#/testCase/LVV-T668}{LVV-T668} & \multicolumn{4}{p{12cm}}{ Verify access to original alert state } \\ \hline
\textbf{Owner} & \textbf{Status} & \textbf{Version} & \textbf{Critical Event} & \textbf{Verification Type} \\ \hline
Jeffrey Carlin & Draft & 1 & false & Inspection \\ \hline
\end{longtable}
{\scriptsize
\textbf{Objective:}\\
Verify that alerts as they were originally raised are accessible via the
Portal.
}
  
 \newpage 
\subsection{[LVV-9872] DMS-PRTL-REQ-0033-V-01: Queries on the Alerts Database\_1 }\label{lvv-9872}

\begin{longtable}{cccc}
\hline
\textbf{Jira Link} & \textbf{Assignee} & \textbf{Status} & \textbf{Test Cases}\\ \hline
\href{https://jira.lsstcorp.org/browse/LVV-9872}{LVV-9872} &
Gregory Dubois-Felsmann & Not Covered &
\begin{tabular}{c}
LVV-T667 \\
\end{tabular}
\\
\hline
\end{longtable}

\textbf{Verification Element Description:} \\
Undefined

{\footnotesize
\begin{longtable}{p{2.5cm}p{13.5cm}}
\hline
\multicolumn{2}{c}{\textbf{Requirement Details}}\\ \hline
Requirement ID & DMS-PRTL-REQ-0033 \\ \cdashline{1-2}
Requirement Description &
\begin{minipage}[]{13cm}
The Portal aspect shall provide a query interface to the Alert Database,
allowing searches based on parameters which shall include, but may not
be limited to, Alert ID, time of alert, position on the sky, filter, and
alert characteristics.
\end{minipage}
\\ \cdashline{1-2}
Requirement Discussion &
\begin{minipage}[]{13cm}
This capability, as all others in this section, is limited to data
rights holders. Non-data-rights holders have access to alerts only
through the alerts stream(s) sent to public brokers.
\end{minipage}
\\ \cdashline{1-2}
Requirement Priority &  \\ \cdashline{1-2}
Upper Level Requirement &
\begin{tabular}{cl}
\end{tabular}
\\ \hline
\end{longtable}
}


\subsubsection{Test Cases Summary}
\begin{longtable}{p{3cm}p{2.5cm}p{2.5cm}p{3cm}p{4cm}}
\toprule
\href{https://jira.lsstcorp.org/secure/Tests.jspa\#/testCase/LVV-T667}{LVV-T667} & \multicolumn{4}{p{12cm}}{ Verify queries on the alerts database } \\ \hline
\textbf{Owner} & \textbf{Status} & \textbf{Version} & \textbf{Critical Event} & \textbf{Verification Type} \\ \hline
Jeffrey Carlin & Draft & 1 & false & Inspection \\ \hline
\end{longtable}
{\scriptsize
\textbf{Objective:}\\
Verify that the Portal supports queries on parameters in the Alerts
Database.
}
  
 \newpage 
\subsection{[LVV-9873] DMS-PRTL-REQ-0032-V-01: Query Tabular Data based upon Image MetaData\_1 }\label{lvv-9873}

\begin{longtable}{cccc}
\hline
\textbf{Jira Link} & \textbf{Assignee} & \textbf{Status} & \textbf{Test Cases}\\ \hline
\href{https://jira.lsstcorp.org/browse/LVV-9873}{LVV-9873} &
Gregory Dubois-Felsmann & Not Covered &
\begin{tabular}{c}
LVV-T666 \\
\end{tabular}
\\
\hline
\end{longtable}

\textbf{Verification Element Description:} \\
Undefined

{\footnotesize
\begin{longtable}{p{2.5cm}p{13.5cm}}
\hline
\multicolumn{2}{c}{\textbf{Requirement Details}}\\ \hline
Requirement ID & DMS-PRTL-REQ-0032 \\ \cdashline{1-2}
Requirement Description &
\begin{minipage}[]{13cm}
The Portal aspect shall be able to support queries of catalog data that
include constraints on the properties of the images on which the catalog
measurements were made.
\end{minipage}
\\ \cdashline{1-2}
Requirement Discussion &
\begin{minipage}[]{13cm}
This allows, for instance, limiting the return of Source catalog entries
to measurements made on images taken with constraints on airmass, moon
angle, etc.
\end{minipage}
\\ \cdashline{1-2}
Requirement Priority &  \\ \cdashline{1-2}
Upper Level Requirement &
\begin{tabular}{cl}
\end{tabular}
\\ \hline
\end{longtable}
}


\subsubsection{Test Cases Summary}
\begin{longtable}{p{3cm}p{2.5cm}p{2.5cm}p{3cm}p{4cm}}
\toprule
\href{https://jira.lsstcorp.org/secure/Tests.jspa\#/testCase/LVV-T666}{LVV-T666} & \multicolumn{4}{p{12cm}}{ Verify query by image metadata } \\ \hline
\textbf{Owner} & \textbf{Status} & \textbf{Version} & \textbf{Critical Event} & \textbf{Verification Type} \\ \hline
Jeffrey Carlin & Draft & 1 & false & Test \\ \hline
\end{longtable}
{\scriptsize
\textbf{Objective:}\\
Verify that the Portal supports queries on image metadata (e.g.,
airmass, moon angle, etc.) from the images the catalog measurements were
made from.
}
  
 \newpage 
\subsection{[LVV-9874] DMS-PRTL-REQ-0031-V-01: Tabular Data Query Specifications\_1 }\label{lvv-9874}

\begin{longtable}{cccc}
\hline
\textbf{Jira Link} & \textbf{Assignee} & \textbf{Status} & \textbf{Test Cases}\\ \hline
\href{https://jira.lsstcorp.org/browse/LVV-9874}{LVV-9874} &
Gregory Dubois-Felsmann & Not Covered &
\begin{tabular}{c}
LVV-T664 \\
\end{tabular}
\\
\hline
\end{longtable}

\textbf{Verification Element Description:} \\
Undefined

{\footnotesize
\begin{longtable}{p{2.5cm}p{13.5cm}}
\hline
\multicolumn{2}{c}{\textbf{Requirement Details}}\\ \hline
Requirement ID & DMS-PRTL-REQ-0031 \\ \cdashline{1-2}
Requirement Description &
\begin{minipage}[]{13cm}
The Portal aspect shall provide a user interface to execute queries of
the (DIA)Object and (DIA)Source tables, driven by the data dictionary
associated with the tables.
\end{minipage}
\\ \cdashline{1-2}
Requirement Discussion &
\begin{minipage}[]{13cm}
This should be satisfied almost completely by the ``Generic Query -
Form-Based'' requirement above, but with some additional work on the UI
to produce a more friendly workflow.
\end{minipage}
\\ \cdashline{1-2}
Requirement Priority &  \\ \cdashline{1-2}
Upper Level Requirement &
\begin{tabular}{cl}
\end{tabular}
\\ \hline
\end{longtable}
}


\subsubsection{Test Cases Summary}
\begin{longtable}{p{3cm}p{2.5cm}p{2.5cm}p{3cm}p{4cm}}
\toprule
\href{https://jira.lsstcorp.org/secure/Tests.jspa\#/testCase/LVV-T664}{LVV-T664} & \multicolumn{4}{p{12cm}}{ Verify implementation of user-friendly tabular query } \\ \hline
\textbf{Owner} & \textbf{Status} & \textbf{Version} & \textbf{Critical Event} & \textbf{Verification Type} \\ \hline
Jeffrey Carlin & Draft & 1 & false & Inspection \\ \hline
\end{longtable}
{\scriptsize
\textbf{Objective:}\\
The Portal aspect shall provide a user interface to execute queries of
the (DIA)Object and (DIA)Source tables, driven by the data dictionary
associated with the tables.
}
  
 \newpage 
\subsection{[LVV-9875] DMS-PRTL-REQ-0039-V-01: Coadded Image Query Specifications\_1 }\label{lvv-9875}

\begin{longtable}{cccc}
\hline
\textbf{Jira Link} & \textbf{Assignee} & \textbf{Status} & \textbf{Test Cases}\\ \hline
\href{https://jira.lsstcorp.org/browse/LVV-9875}{LVV-9875} &
Gregory Dubois-Felsmann & Not Covered &
\begin{tabular}{c}
LVV-T673 \\
\end{tabular}
\\
\hline
\end{longtable}

\textbf{Verification Element Description:} \\
Undefined

{\footnotesize
\begin{longtable}{p{2.5cm}p{13.5cm}}
\hline
\multicolumn{2}{c}{\textbf{Requirement Details}}\\ \hline
Requirement ID & DMS-PRTL-REQ-0039 \\ \cdashline{1-2}
Requirement Description &
\begin{minipage}[]{13cm}
The Portal aspect shall provide UI support for queries for coadded
images based on the image metadata that describe the provenance of the
images (e.g., filters, position on the sky, date, number of single-epoch
images, coverage, survey depth).
\end{minipage}
\\ \cdashline{1-2}
Requirement Priority &  \\ \cdashline{1-2}
Upper Level Requirement &
\begin{tabular}{cl}
\end{tabular}
\\ \hline
\end{longtable}
}


\subsubsection{Test Cases Summary}
\begin{longtable}{p{3cm}p{2.5cm}p{2.5cm}p{3cm}p{4cm}}
\toprule
\href{https://jira.lsstcorp.org/secure/Tests.jspa\#/testCase/LVV-T673}{LVV-T673} & \multicolumn{4}{p{12cm}}{ Verify query for coadds by image metadata } \\ \hline
\textbf{Owner} & \textbf{Status} & \textbf{Version} & \textbf{Critical Event} & \textbf{Verification Type} \\ \hline
Jeffrey Carlin & Draft & 1 & false & Inspection \\ \hline
\end{longtable}
{\scriptsize
\textbf{Objective:}\\
Verify that the Portal aspect supports queries based on image metadata
describing the provenance of the contributing images, that return the
corresponding coadd image(s).
}
  
 \newpage 
\subsection{[LVV-9876] DMS-PRTL-REQ-0037-V-01: Query for Single Epoch CCD Image\_1 }\label{lvv-9876}

\begin{longtable}{cccc}
\hline
\textbf{Jira Link} & \textbf{Assignee} & \textbf{Status} & \textbf{Test Cases}\\ \hline
\href{https://jira.lsstcorp.org/browse/LVV-9876}{LVV-9876} &
Gregory Dubois-Felsmann & Not Covered &
\begin{tabular}{c}
LVV-T671 \\
\end{tabular}
\\
\hline
\end{longtable}

\textbf{Verification Element Description:} \\
Undefined

{\footnotesize
\begin{longtable}{p{2.5cm}p{13.5cm}}
\hline
\multicolumn{2}{c}{\textbf{Requirement Details}}\\ \hline
Requirement ID & DMS-PRTL-REQ-0037 \\ \cdashline{1-2}
Requirement Description &
\begin{minipage}[]{13cm}
The Portal aspect shall enable a user to limit the list of images
selected by a single-epoch visit image query to those from a specified
CCD.
\end{minipage}
\\ \cdashline{1-2}
Requirement Discussion &
\begin{minipage}[]{13cm}
The intent is to enable the return of every image from CCD X as a
function of time or filter to enable viewing just that CCD.
\end{minipage}
\\ \cdashline{1-2}
Requirement Priority &  \\ \cdashline{1-2}
Upper Level Requirement &
\begin{tabular}{cl}
\end{tabular}
\\ \hline
\end{longtable}
}


\subsubsection{Test Cases Summary}
\begin{longtable}{p{3cm}p{2.5cm}p{2.5cm}p{3cm}p{4cm}}
\toprule
\href{https://jira.lsstcorp.org/secure/Tests.jspa\#/testCase/LVV-T671}{LVV-T671} & \multicolumn{4}{p{12cm}}{ Verify query for single-epoch CCD images } \\ \hline
\textbf{Owner} & \textbf{Status} & \textbf{Version} & \textbf{Critical Event} & \textbf{Verification Type} \\ \hline
Jeffrey Carlin & Draft & 1 & false & Inspection \\ \hline
\end{longtable}
{\scriptsize
\textbf{Objective:}\\
Verify that users of the single-epoch query service
(\href{https://jira.lsstcorp.org/browse/LVV-9878}{LVV-9878}) can limit
the returned visit images to only a specified CCD.
}
  
 \newpage 
\subsection{[LVV-9877] DMS-PRTL-REQ-0036-V-01: Query for Single Epoch Raft Images\_1 }\label{lvv-9877}

\begin{longtable}{cccc}
\hline
\textbf{Jira Link} & \textbf{Assignee} & \textbf{Status} & \textbf{Test Cases}\\ \hline
\href{https://jira.lsstcorp.org/browse/LVV-9877}{LVV-9877} &
Gregory Dubois-Felsmann & Not Covered &
\begin{tabular}{c}
LVV-T670 \\
\end{tabular}
\\
\hline
\end{longtable}

\textbf{Verification Element Description:} \\
Undefined

{\footnotesize
\begin{longtable}{p{2.5cm}p{13.5cm}}
\hline
\multicolumn{2}{c}{\textbf{Requirement Details}}\\ \hline
Requirement ID & DMS-PRTL-REQ-0036 \\ \cdashline{1-2}
Requirement Description &
\begin{minipage}[]{13cm}
The Portal aspect shall enable a user to limit the list of images
selected by a single-epoch visit image query to those from a specified
raft.
\end{minipage}
\\ \cdashline{1-2}
Requirement Discussion &
\begin{minipage}[]{13cm}
The intent is to enable the return of every image from RAFT X as a
function of time or filter to enable viewing just that raft.
\end{minipage}
\\ \cdashline{1-2}
Requirement Priority &  \\ \cdashline{1-2}
Upper Level Requirement &
\begin{tabular}{cl}
\end{tabular}
\\ \hline
\end{longtable}
}


\subsubsection{Test Cases Summary}
\begin{longtable}{p{3cm}p{2.5cm}p{2.5cm}p{3cm}p{4cm}}
\toprule
\href{https://jira.lsstcorp.org/secure/Tests.jspa\#/testCase/LVV-T670}{LVV-T670} & \multicolumn{4}{p{12cm}}{ Verify query for single-epoch raft images } \\ \hline
\textbf{Owner} & \textbf{Status} & \textbf{Version} & \textbf{Critical Event} & \textbf{Verification Type} \\ \hline
Jeffrey Carlin & Draft & 1 & false & Inspection \\ \hline
\end{longtable}
{\scriptsize
\textbf{Objective:}\\
Verify that users of the single-epoch query service
(\href{https://jira.lsstcorp.org/browse/LVV-9878}{LVV-9878}) can limit
the returned visit images to only a specified raft.
}
  
 \newpage 
\subsection{[LVV-9878] DMS-PRTL-REQ-0035-V-01: Query for Single Epoch Visit Images\_1 }\label{lvv-9878}

\begin{longtable}{cccc}
\hline
\textbf{Jira Link} & \textbf{Assignee} & \textbf{Status} & \textbf{Test Cases}\\ \hline
\href{https://jira.lsstcorp.org/browse/LVV-9878}{LVV-9878} &
Gregory Dubois-Felsmann & Not Covered &
\begin{tabular}{c}
LVV-T669 \\
\end{tabular}
\\
\hline
\end{longtable}

\textbf{Verification Element Description:} \\
Undefined

{\footnotesize
\begin{longtable}{p{2.5cm}p{13.5cm}}
\hline
\multicolumn{2}{c}{\textbf{Requirement Details}}\\ \hline
Requirement ID & DMS-PRTL-REQ-0035 \\ \cdashline{1-2}
Requirement Description &
\begin{minipage}[]{13cm}
The Portal aspect shall enable a user to proceed from a visit-selection
query or a list of visits and return a list of all single-epoch images
of a specified type corresponding to those visits.
\end{minipage}
\\ \cdashline{1-2}
Requirement Discussion &
\begin{minipage}[]{13cm}
The common image types will be raw, PVI (processed, i.e., calibrated,
visit image), and difference image.
\end{minipage}
\\ \cdashline{1-2}
Requirement Priority &  \\ \cdashline{1-2}
Upper Level Requirement &
\begin{tabular}{cl}
\end{tabular}
\\ \hline
\end{longtable}
}


\subsubsection{Test Cases Summary}
\begin{longtable}{p{3cm}p{2.5cm}p{2.5cm}p{3cm}p{4cm}}
\toprule
\href{https://jira.lsstcorp.org/secure/Tests.jspa\#/testCase/LVV-T669}{LVV-T669} & \multicolumn{4}{p{12cm}}{ Verify query for single-epoch visit images } \\ \hline
\textbf{Owner} & \textbf{Status} & \textbf{Version} & \textbf{Critical Event} & \textbf{Verification Type} \\ \hline
Jeffrey Carlin & Draft & 1 & false & Inspection \\ \hline
\end{longtable}
{\scriptsize
\textbf{Objective:}\\
Verify that users with a list of visits (either directly, or from a
visit-selection query) can query for single-epoch images corresponding
to those visits.
}
  
 \newpage 
\subsection{[LVV-9879] DMS-PRTL-REQ-0038-V-01: Single-Epoch Image Query Specifications\_1 }\label{lvv-9879}

\begin{longtable}{cccc}
\hline
\textbf{Jira Link} & \textbf{Assignee} & \textbf{Status} & \textbf{Test Cases}\\ \hline
\href{https://jira.lsstcorp.org/browse/LVV-9879}{LVV-9879} &
Gregory Dubois-Felsmann & Not Covered &
\begin{tabular}{c}
LVV-T672 \\
\end{tabular}
\\
\hline
\end{longtable}

\textbf{Verification Element Description:} \\
Undefined

{\footnotesize
\begin{longtable}{p{2.5cm}p{13.5cm}}
\hline
\multicolumn{2}{c}{\textbf{Requirement Details}}\\ \hline
Requirement ID & DMS-PRTL-REQ-0038 \\ \cdashline{1-2}
Requirement Description &
\begin{minipage}[]{13cm}
The Portal aspect shall provide UI support for queries for visits and
their single-epoch images of specified type, based on image metadata
parameters including pointing, time and date, and filter selection, as
well as on parameters from the Reformatted EFD.
\end{minipage}
\\ \cdashline{1-2}
Requirement Discussion &
\begin{minipage}[]{13cm}
The parameters specifically named are expected to be highlighted in the
UI, rather than requiring the user to scroll through a long
generic-table-query form to find the appropriate fields. The UI will
provide support for generating a join query including tables from the
R-EFD, and for selecting the R-EFD tables and columns to use.
\end{minipage}
\\ \cdashline{1-2}
Requirement Priority &  \\ \cdashline{1-2}
Upper Level Requirement &
\begin{tabular}{cl}
\end{tabular}
\\ \hline
\end{longtable}
}


\subsubsection{Test Cases Summary}
\begin{longtable}{p{3cm}p{2.5cm}p{2.5cm}p{3cm}p{4cm}}
\toprule
\href{https://jira.lsstcorp.org/secure/Tests.jspa\#/testCase/LVV-T672}{LVV-T672} & \multicolumn{4}{p{12cm}}{ Verify metadata query for single-epoch images } \\ \hline
\textbf{Owner} & \textbf{Status} & \textbf{Version} & \textbf{Critical Event} & \textbf{Verification Type} \\ \hline
Jeffrey Carlin & Draft & 1 & false & Inspection \\ \hline
\end{longtable}
{\scriptsize
\textbf{Objective:}\\
Verify that the Portal provides an option to query for visits and
single-epoch images of a certain type based on image metadata or
parameters from the reformatted EFD.
}
  
 \newpage 
\subsection{[LVV-9880] DMS-PRTL-REQ-0041-V-01: Query for Coadded Image Cutouts\_1 }\label{lvv-9880}

\begin{longtable}{cccc}
\hline
\textbf{Jira Link} & \textbf{Assignee} & \textbf{Status} & \textbf{Test Cases}\\ \hline
\href{https://jira.lsstcorp.org/browse/LVV-9880}{LVV-9880} &
Gregory Dubois-Felsmann & Not Covered &
\begin{tabular}{c}
LVV-T7 \\
LVV-T674 \\
\end{tabular}
\\
\hline
\end{longtable}

\textbf{Verification Element Description:} \\
Undefined

{\footnotesize
\begin{longtable}{p{2.5cm}p{13.5cm}}
\hline
\multicolumn{2}{c}{\textbf{Requirement Details}}\\ \hline
Requirement ID & DMS-PRTL-REQ-0041 \\ \cdashline{1-2}
Requirement Description &
\begin{minipage}[]{13cm}
The Portal aspect shall enable a user to perform a coadded image query,
as above, and additionally return a list of sub-images (i.e., cutouts)
from the all-sky co-added images based upon user-specified center
position and image size, including the appropriate metadata for
describing the image cut-outs.
\end{minipage}
\\ \cdashline{1-2}
Requirement Discussion &
\begin{minipage}[]{13cm}
This is a front end to a cutout capability in the API aspect.
\end{minipage}
\\ \cdashline{1-2}
Requirement Priority &  \\ \cdashline{1-2}
Upper Level Requirement &
\begin{tabular}{cl}
\end{tabular}
\\ \hline
\end{longtable}
}


\subsubsection{Test Cases Summary}
\begin{longtable}{p{3cm}p{2.5cm}p{2.5cm}p{3cm}p{4cm}}
\toprule
\href{https://jira.lsstcorp.org/secure/Tests.jspa\#/testCase/LVV-T7}{LVV-T7} & \multicolumn{4}{p{12cm}}{ LSP-00-25: Image metadata, image, and image cutout queries } \\ \hline
\textbf{Owner} & \textbf{Status} & \textbf{Version} & \textbf{Critical Event} & \textbf{Verification Type} \\ \hline
Gregory Dubois-Felsmann & Deprecated & 1 & false & Test \\ \hline
\end{longtable}
{\scriptsize
\textbf{Objective:}\\
This test will check basic functionality related to image search and
retrieval, via both the API Aspect and the Portal Aspect of the LSST
Science Platform:

\begin{itemize}
\tightlist
\item
  Searching for images containing a specified point;
\item
  Displaying selected images;
\item
  Obtaining and displaying image cutouts at a specified point; and
\item
  Downloading selected images and image cutouts.
\end{itemize}

Because of limited staff resources, these tests will be based on the
original PDAC dataset, the LSST Summer 2013 processing of the SDSS
Stripe 82 data. The image data for the WISE and NEOWISE missions have
not been loaded into PDAC.\\[2\baselineskip]
}
\begin{longtable}{p{3cm}p{2.5cm}p{2.5cm}p{3cm}p{4cm}}
\toprule
\href{https://jira.lsstcorp.org/secure/Tests.jspa\#/testCase/LVV-T674}{LVV-T674} & \multicolumn{4}{p{12cm}}{ Verify query for coadd image cutouts } \\ \hline
\textbf{Owner} & \textbf{Status} & \textbf{Version} & \textbf{Critical Event} & \textbf{Verification Type} \\ \hline
Jeffrey Carlin & Draft & 1 & false & Inspection \\ \hline
\end{longtable}
{\scriptsize
\textbf{Objective:}\\
Verify that Portal users can query based on image metadata for coadds,
then obtain a list of sub-images (cutouts) with a specified center
position and size.
}
  
 \newpage 
\subsection{[LVV-9881] DMS-PRTL-REQ-0040-V-01: Query for Single Epoch Image Cutouts\_1 }\label{lvv-9881}

\begin{longtable}{cccc}
\hline
\textbf{Jira Link} & \textbf{Assignee} & \textbf{Status} & \textbf{Test Cases}\\ \hline
\href{https://jira.lsstcorp.org/browse/LVV-9881}{LVV-9881} &
Gregory Dubois-Felsmann & Not Covered &
\begin{tabular}{c}
LVV-T7 \\
LVV-T675 \\
\end{tabular}
\\
\hline
\end{longtable}

\textbf{Verification Element Description:} \\
Undefined

{\footnotesize
\begin{longtable}{p{2.5cm}p{13.5cm}}
\hline
\multicolumn{2}{c}{\textbf{Requirement Details}}\\ \hline
Requirement ID & DMS-PRTL-REQ-0040 \\ \cdashline{1-2}
Requirement Description &
\begin{minipage}[]{13cm}
The Portal aspect shall enable a user to perform a single-epoch image
query, as above, and additionally return a list of sub-images (i.e.,
cutouts) from them based upon a specified center position, time range,
and image size, including the appropriate metadata for describing the
image cut-outs.
\end{minipage}
\\ \cdashline{1-2}
Requirement Discussion &
\begin{minipage}[]{13cm}
This is a front end to a cutout capability in the API aspect.
\end{minipage}
\\ \cdashline{1-2}
Requirement Priority &  \\ \cdashline{1-2}
Upper Level Requirement &
\begin{tabular}{cl}
\end{tabular}
\\ \hline
\end{longtable}
}


\subsubsection{Test Cases Summary}
\begin{longtable}{p{3cm}p{2.5cm}p{2.5cm}p{3cm}p{4cm}}
\toprule
\href{https://jira.lsstcorp.org/secure/Tests.jspa\#/testCase/LVV-T7}{LVV-T7} & \multicolumn{4}{p{12cm}}{ LSP-00-25: Image metadata, image, and image cutout queries } \\ \hline
\textbf{Owner} & \textbf{Status} & \textbf{Version} & \textbf{Critical Event} & \textbf{Verification Type} \\ \hline
Gregory Dubois-Felsmann & Deprecated & 1 & false & Test \\ \hline
\end{longtable}
{\scriptsize
\textbf{Objective:}\\
This test will check basic functionality related to image search and
retrieval, via both the API Aspect and the Portal Aspect of the LSST
Science Platform:

\begin{itemize}
\tightlist
\item
  Searching for images containing a specified point;
\item
  Displaying selected images;
\item
  Obtaining and displaying image cutouts at a specified point; and
\item
  Downloading selected images and image cutouts.
\end{itemize}

Because of limited staff resources, these tests will be based on the
original PDAC dataset, the LSST Summer 2013 processing of the SDSS
Stripe 82 data. The image data for the WISE and NEOWISE missions have
not been loaded into PDAC.\\[2\baselineskip]
}
\begin{longtable}{p{3cm}p{2.5cm}p{2.5cm}p{3cm}p{4cm}}
\toprule
\href{https://jira.lsstcorp.org/secure/Tests.jspa\#/testCase/LVV-T675}{LVV-T675} & \multicolumn{4}{p{12cm}}{ Verify query for single-epoch image cutouts } \\ \hline
\textbf{Owner} & \textbf{Status} & \textbf{Version} & \textbf{Critical Event} & \textbf{Verification Type} \\ \hline
Jeffrey Carlin & Draft & 1 & false & Inspection \\ \hline
\end{longtable}
{\scriptsize
\textbf{Objective:}\\
Verify that Portal users can query based on image metadata for
single-epoch images, then obtain a list of sub-images (cutouts) with a
specified center position and size.
}
  
 \newpage 
\subsection{[LVV-9882] DMS-PRTL-REQ-0044-V-01: Linking Visualization of Image Data to Tabular
Data\_1 }\label{lvv-9882}

\begin{longtable}{cccc}
\hline
\textbf{Jira Link} & \textbf{Assignee} & \textbf{Status} & \textbf{Test Cases}\\ \hline
\href{https://jira.lsstcorp.org/browse/LVV-9882}{LVV-9882} &
Gregory Dubois-Felsmann & Not Covered &
\begin{tabular}{c}
LVV-T679 \\
\end{tabular}
\\
\hline
\end{longtable}

\textbf{Verification Element Description:} \\
Undefined

{\footnotesize
\begin{longtable}{p{2.5cm}p{13.5cm}}
\hline
\multicolumn{2}{c}{\textbf{Requirement Details}}\\ \hline
Requirement ID & DMS-PRTL-REQ-0044 \\ \cdashline{1-2}
Requirement Description &
\begin{minipage}[]{13cm}
The Portal aspect shall provide the capability for the user to navigate
between visualized tabular data and visualized image data.
\end{minipage}
\\ \cdashline{1-2}
Requirement Discussion &
\begin{minipage}[]{13cm}
For instance, there should be very simple UI support to display an image
based on a row in an image metadata table, or to navigate from a
selected source overplotted on an image to further information about
that source.
\end{minipage}
\\ \cdashline{1-2}
Requirement Priority &  \\ \cdashline{1-2}
Upper Level Requirement &
\begin{tabular}{cl}
\end{tabular}
\\ \hline
\end{longtable}
}


\subsubsection{Test Cases Summary}
\begin{longtable}{p{3cm}p{2.5cm}p{2.5cm}p{3cm}p{4cm}}
\toprule
\href{https://jira.lsstcorp.org/secure/Tests.jspa\#/testCase/LVV-T679}{LVV-T679} & \multicolumn{4}{p{12cm}}{ Verify visualization linking image and tabular data } \\ \hline
\textbf{Owner} & \textbf{Status} & \textbf{Version} & \textbf{Critical Event} & \textbf{Verification Type} \\ \hline
Jeffrey Carlin & Draft & 1 & false & Inspection \\ \hline
\end{longtable}
{\scriptsize
\textbf{Objective:}\\
Verify that the Portal aspect provides a capability for users to
navigate between visualization and tabular data for a given tabular
entry.
}
  
 \newpage 
\subsection{[LVV-9883] DMS-PRTL-REQ-0043-V-01: Visualization of Ancillary Information\_1 }\label{lvv-9883}

\begin{longtable}{cccc}
\hline
\textbf{Jira Link} & \textbf{Assignee} & \textbf{Status} & \textbf{Test Cases}\\ \hline
\href{https://jira.lsstcorp.org/browse/LVV-9883}{LVV-9883} &
Gregory Dubois-Felsmann & Not Covered &
\begin{tabular}{c}
LVV-T678 \\
\end{tabular}
\\
\hline
\end{longtable}

\textbf{Verification Element Description:} \\
Undefined

{\footnotesize
\begin{longtable}{p{2.5cm}p{13.5cm}}
\hline
\multicolumn{2}{c}{\textbf{Requirement Details}}\\ \hline
Requirement ID & DMS-PRTL-REQ-0043 \\ \cdashline{1-2}
Requirement Description &
\begin{minipage}[]{13cm}
The Portal aspect shall include the ability to visualize selected
ancillary information produced by the LSST pipeline including, but not
limited to, image regions, image bit-planes, survey footprints,
focal-plane footprints and PSF representations.
\end{minipage}
\\ \cdashline{1-2}
Requirement Discussion &
\begin{minipage}[]{13cm}
The intent here is to call attention to the fact there is more than just
the survey images and coadds that are have a ?2-dimension? form that
need to be visualized and presented to the user in the interface.\\
The specific ancillary data products to visualize will be determined
during construction, based in part on feedback received during PDAC
operation and the use of the Portal tools by developers.\\
It is desirable that custom visualizations be available for important
and frequently used ones such as Footprints (which can readily be
displayed as pixel overlays). Where dedicated Portal visualizations are
not available, however, users should be able to use either LSST-provided
or community libraries in the Notebook aspect to create custom
visualizations.
\end{minipage}
\\ \cdashline{1-2}
Requirement Priority &  \\ \cdashline{1-2}
Upper Level Requirement &
\begin{tabular}{cl}
\end{tabular}
\\ \hline
\end{longtable}
}


\subsubsection{Test Cases Summary}
\begin{longtable}{p{3cm}p{2.5cm}p{2.5cm}p{3cm}p{4cm}}
\toprule
\href{https://jira.lsstcorp.org/secure/Tests.jspa\#/testCase/LVV-T678}{LVV-T678} & \multicolumn{4}{p{12cm}}{ Verify visualization of ancillary information } \\ \hline
\textbf{Owner} & \textbf{Status} & \textbf{Version} & \textbf{Critical Event} & \textbf{Verification Type} \\ \hline
Jeffrey Carlin & Draft & 1 & false & Inspection \\ \hline
\end{longtable}
{\scriptsize
\textbf{Objective:}\\
Verify that the Portal provides the ability to visualize certain
ancillary information produced by the LSST pipeline, including, but not
limited to, image regions, image bit-planes, survey footprints,
focal-plane footprints and PSF representations.
}
  
 \newpage 
\subsection{[LVV-9884] DMS-PRTL-REQ-0042-V-01: Visualization of Tabular and Image Data\_1 }\label{lvv-9884}

\begin{longtable}{cccc}
\hline
\textbf{Jira Link} & \textbf{Assignee} & \textbf{Status} & \textbf{Test Cases}\\ \hline
\href{https://jira.lsstcorp.org/browse/LVV-9884}{LVV-9884} &
Gregory Dubois-Felsmann & Not Covered &
\begin{tabular}{c}
LVV-T677 \\
\end{tabular}
\\
\hline
\end{longtable}

\textbf{Verification Element Description:} \\
Undefined

{\footnotesize
\begin{longtable}{p{2.5cm}p{13.5cm}}
\hline
\multicolumn{2}{c}{\textbf{Requirement Details}}\\ \hline
Requirement ID & DMS-PRTL-REQ-0042 \\ \cdashline{1-2}
Requirement Description &
\begin{minipage}[]{13cm}
The Portal aspect shall provide the capability to visualize all tabular
and image data products in the DPDD, as well as user data products.
\end{minipage}
\\ \cdashline{1-2}
Requirement Discussion &
\begin{minipage}[]{13cm}
The products in the DPDD are the primary data products for use by the
LSST users. The ``tabular and image'' qualification indicates that the
Portal is not required to provide a dedicated visualization for all data
products that do not naturally fall into one of those categories.\\
For user data products, the amount of detail and labeling, and the
amount of UI support, will be less if they lack the full level of
metadata that comes with the Project's own data products.
\end{minipage}
\\ \cdashline{1-2}
Requirement Priority &  \\ \cdashline{1-2}
Upper Level Requirement &
\begin{tabular}{cl}
\end{tabular}
\\ \hline
\end{longtable}
}


\subsubsection{Test Cases Summary}
\begin{longtable}{p{3cm}p{2.5cm}p{2.5cm}p{3cm}p{4cm}}
\toprule
\href{https://jira.lsstcorp.org/secure/Tests.jspa\#/testCase/LVV-T677}{LVV-T677} & \multicolumn{4}{p{12cm}}{ Verify Portal provides visualization of tabular and image data } \\ \hline
\textbf{Owner} & \textbf{Status} & \textbf{Version} & \textbf{Critical Event} & \textbf{Verification Type} \\ \hline
Jeffrey Carlin & Draft & 1 & false & Inspection \\ \hline
\end{longtable}
{\scriptsize
\textbf{Objective:}\\
Verify that the Portal aspect provides the capability to visualize all
tabular and image data defined in the DPDD, as well as user data
products.
}
  
 \newpage 
\subsection{[LVV-9885] DMS-PRTL-REQ-0045-V-01: Visualization of Uploaded Tabular and Image
Data\_1 }\label{lvv-9885}

\begin{longtable}{cccc}
\hline
\textbf{Jira Link} & \textbf{Assignee} & \textbf{Status} & \textbf{Test Cases}\\ \hline
\href{https://jira.lsstcorp.org/browse/LVV-9885}{LVV-9885} &
Gregory Dubois-Felsmann & Not Covered &
\begin{tabular}{c}
LVV-T680 \\
\end{tabular}
\\
\hline
\end{longtable}

\textbf{Verification Element Description:} \\
Undefined

{\footnotesize
\begin{longtable}{p{2.5cm}p{13.5cm}}
\hline
\multicolumn{2}{c}{\textbf{Requirement Details}}\\ \hline
Requirement ID & DMS-PRTL-REQ-0045 \\ \cdashline{1-2}
Requirement Description &
\begin{minipage}[]{13cm}
The Portal aspect shall support a convenient workflow for the
visualization of uploaded tabular and image data products.
\end{minipage}
\\ \cdashline{1-2}
Requirement Discussion &
\begin{minipage}[]{13cm}
The idea is to provide something close to a one-button ``show me this''
workflow.
\end{minipage}
\\ \cdashline{1-2}
Requirement Priority &  \\ \cdashline{1-2}
Upper Level Requirement &
\begin{tabular}{cl}
\end{tabular}
\\ \hline
\end{longtable}
}


\subsubsection{Test Cases Summary}
\begin{longtable}{p{3cm}p{2.5cm}p{2.5cm}p{3cm}p{4cm}}
\toprule
\href{https://jira.lsstcorp.org/secure/Tests.jspa\#/testCase/LVV-T680}{LVV-T680} & \multicolumn{4}{p{12cm}}{ Verify visualization tool for uploaded tabular or image data } \\ \hline
\textbf{Owner} & \textbf{Status} & \textbf{Version} & \textbf{Critical Event} & \textbf{Verification Type} \\ \hline
Jeffrey Carlin & Draft & 1 & false & Inspection \\ \hline
\end{longtable}
{\scriptsize
\textbf{Objective:}\\
Verify that the Portal provides a means of visualizing uploaded tables
or images.
}
  
 \newpage 
\subsection{[LVV-9886] DMS-PRTL-REQ-0046-V-01: Visualization of Workspace Data\_1 }\label{lvv-9886}

\begin{longtable}{cccc}
\hline
\textbf{Jira Link} & \textbf{Assignee} & \textbf{Status} & \textbf{Test Cases}\\ \hline
\href{https://jira.lsstcorp.org/browse/LVV-9886}{LVV-9886} &
Gregory Dubois-Felsmann & Not Covered &
\begin{tabular}{c}
LVV-T681 \\
LVV-T1818 \\
\end{tabular}
\\
\hline
\end{longtable}

\textbf{Verification Element Description:} \\
Undefined

{\footnotesize
\begin{longtable}{p{2.5cm}p{13.5cm}}
\hline
\multicolumn{2}{c}{\textbf{Requirement Details}}\\ \hline
Requirement ID & DMS-PRTL-REQ-0046 \\ \cdashline{1-2}
Requirement Description &
\begin{minipage}[]{13cm}
The Portal aspect shall support a convenient workflow for the
visualization of data selected in a workspace browser.
\end{minipage}
\\ \cdashline{1-2}
Requirement Discussion &
\begin{minipage}[]{13cm}
This should appear as a standard ``select and open'' workflow, with the
Portal determining a reasonable action to take based on its
determination of the type of Workspace data selected.
\end{minipage}
\\ \cdashline{1-2}
Requirement Priority &  \\ \cdashline{1-2}
Upper Level Requirement &
\begin{tabular}{cl}
\end{tabular}
\\ \hline
\end{longtable}
}


\subsubsection{Test Cases Summary}
\begin{longtable}{p{3cm}p{2.5cm}p{2.5cm}p{3cm}p{4cm}}
\toprule
\href{https://jira.lsstcorp.org/secure/Tests.jspa\#/testCase/LVV-T681}{LVV-T681} & \multicolumn{4}{p{12cm}}{ Verify visualization of workspace data } \\ \hline
\textbf{Owner} & \textbf{Status} & \textbf{Version} & \textbf{Critical Event} & \textbf{Verification Type} \\ \hline
Jeffrey Carlin & Draft & 1 & false & Inspection \\ \hline
\end{longtable}
{\scriptsize
\textbf{Objective:}\\
Verify that data selected in a workspace browser can be conveniently
visualized.
}
\begin{longtable}{p{3cm}p{2.5cm}p{2.5cm}p{3cm}p{4cm}}
\toprule
\href{https://jira.lsstcorp.org/secure/Tests.jspa\#/testCase/LVV-T1818}{LVV-T1818} & \multicolumn{4}{p{12cm}}{ DM-SUIT-8: Verify Portal integration with workspace (via WebDAV) } \\ \hline
\textbf{Owner} & \textbf{Status} & \textbf{Version} & \textbf{Critical Event} & \textbf{Verification Type} \\ \hline
Gregory Dubois-Felsmann & Defined & 1 & false & Demonstration \\ \hline
\end{longtable}
{\scriptsize
\textbf{Objective:}\\
This test case verifies that the Portal Aspect software is capable of
accessing a file-oriented workspace via the WebDAV
protocol.\\[2\baselineskip]In so doing, it partially verifies several
Portal Aspect requirements that relate to this capability -
``partially'' because some of these requirements depend on workspace
capabilities which were not present in the prototype WebDAV service
delivered by the DAX group, because some of the requirements also cover
the User Database Workspace (not relevant to this milestone, and not yet
available), and also because the milestone was not envisioned as an
exhaustive test covering edge cases:

\begin{itemize}
\tightlist
\item
  DMS-PRTL-REQ-0003 (LVV-9846, Portal access to workspace) is covered at
  ``demonstration'' level, with basic tests of saving image and tabular
  data to the workspace, and only for the User File Workspace ~(there is
  currently no User Database Workspace prototype available);
\item
  DMS-PRTL-REQ-0046 (LVV-9886, Visualization of workspace data) is
  covered at ``demonstration'' level for a couple of FITS image and
  table files, and only for the User File Workspace;
\item
  DMS-PRTL-REQ-0110 (LVV-9954, Tabular data download) is covered at
  ``demonstration'' level, only for catalog data (there was no image
  metadata in the LSP deployment at the time of test), and only for the
  User File Workspace;~
\item
  DMS-PRTL-REQ-0095 (LVV-9932, Saving Displayed Tabular Data) is covered
  at ``demonstration'' level for a simple subset operation in the table
  browser; and
\item
  DMS-PRTL-REQ-0111 (LVV-9951, Image data download) is covered at
  ``demonstration'' level, and only for download from an image display
  screen itself (as LSST-style image metadata services, e.g., ObsTAP,
  were not available in the LSP at the time of testing).
\end{itemize}
}
  
 \newpage 
\subsection{[LVV-9887] DMS-PRTL-REQ-0048-V-01: Alert Visualization\_1 }\label{lvv-9887}

\begin{longtable}{cccc}
\hline
\textbf{Jira Link} & \textbf{Assignee} & \textbf{Status} & \textbf{Test Cases}\\ \hline
\href{https://jira.lsstcorp.org/browse/LVV-9887}{LVV-9887} &
Gregory Dubois-Felsmann & Not Covered &
\begin{tabular}{c}
LVV-T683 \\
\end{tabular}
\\
\hline
\end{longtable}

\textbf{Verification Element Description:} \\
Undefined

{\footnotesize
\begin{longtable}{p{2.5cm}p{13.5cm}}
\hline
\multicolumn{2}{c}{\textbf{Requirement Details}}\\ \hline
Requirement ID & DMS-PRTL-REQ-0048 \\ \cdashline{1-2}
Requirement Description &
\begin{minipage}[]{13cm}
The Portal aspect shall provide for the users a ``property sheet'' for
the contents of an alert packet including, but not necessarily limited
to, the alert postage stamp image, the postage stamp time series, the
photometric time series, the source and object information (e.g.,
position, brightness).
\end{minipage}
\\ \cdashline{1-2}
Requirement Discussion &
\begin{minipage}[]{13cm}
This display is based on a query performed on the alert database. It may
also be desirable to provide this functionality based on an actual
published alert packet, e.g., by permitting the upload of such a packet
for display.\\
The alert property sheet should facilitate further exploration based on,
e.g., the associated Object IDs in the alert.
\end{minipage}
\\ \cdashline{1-2}
Requirement Priority &  \\ \cdashline{1-2}
Upper Level Requirement &
\begin{tabular}{cl}
\end{tabular}
\\ \hline
\end{longtable}
}


\subsubsection{Test Cases Summary}
\begin{longtable}{p{3cm}p{2.5cm}p{2.5cm}p{3cm}p{4cm}}
\toprule
\href{https://jira.lsstcorp.org/secure/Tests.jspa\#/testCase/LVV-T683}{LVV-T683} & \multicolumn{4}{p{12cm}}{ Verify visualization of alerts } \\ \hline
\textbf{Owner} & \textbf{Status} & \textbf{Version} & \textbf{Critical Event} & \textbf{Verification Type} \\ \hline
Jeffrey Carlin & Draft & 1 & false & Inspection \\ \hline
\end{longtable}
{\scriptsize
\textbf{Objective:}\\
Verify that the Portal aspect provides for the users a ``property
sheet'' for the contents of an alert packet including, but not
necessarily limited to, the alert postage stamp image, the postage stamp
time series, the photometric time series, the source and object
information (e.g., position, brightness).
}
  
 \newpage 
\subsection{[LVV-9888] DMS-PRTL-REQ-0047-V-01: Table Row Property Sheet\_1 }\label{lvv-9888}

\begin{longtable}{cccc}
\hline
\textbf{Jira Link} & \textbf{Assignee} & \textbf{Status} & \textbf{Test Cases}\\ \hline
\href{https://jira.lsstcorp.org/browse/LVV-9888}{LVV-9888} &
Gregory Dubois-Felsmann & Not Covered &
\begin{tabular}{c}
LVV-T682 \\
\end{tabular}
\\
\hline
\end{longtable}

\textbf{Verification Element Description:} \\
Undefined

{\footnotesize
\begin{longtable}{p{2.5cm}p{13.5cm}}
\hline
\multicolumn{2}{c}{\textbf{Requirement Details}}\\ \hline
Requirement ID & DMS-PRTL-REQ-0047 \\ \cdashline{1-2}
Requirement Description &
\begin{minipage}[]{13cm}
The Portal aspect shall permit the inspection of all the data in a
single row of a tabular data query result as a ``property sheet'' for
that row, taking advantage of available metadata to supply units and
other semantic information for each column value.
\end{minipage}
\\ \cdashline{1-2}
Requirement Discussion &
\begin{minipage}[]{13cm}
Resources permitting, the property sheet may be elaborated to provide
additional functionality (typically, further queries) associated with
particular data items displayed.\\
Property sheets should, where enabled by metadata, appropriately exhibit
relationships between columns, such by displaying a value and its
uncertainty together.\\
The system must provide a generic property sheet functionality for any
table for which full metadata is available. It may also provide custom
property sheets for commonly-queried tables such as Object,
ForcedSource, Visit, etc. that provide a more scientifically useful
layout, and additional available workflows, than possible just from the
metadata.
\end{minipage}
\\ \cdashline{1-2}
Requirement Priority &  \\ \cdashline{1-2}
Upper Level Requirement &
\begin{tabular}{cl}
\end{tabular}
\\ \hline
\end{longtable}
}


\subsubsection{Test Cases Summary}
\begin{longtable}{p{3cm}p{2.5cm}p{2.5cm}p{3cm}p{4cm}}
\toprule
\href{https://jira.lsstcorp.org/secure/Tests.jspa\#/testCase/LVV-T682}{LVV-T682} & \multicolumn{4}{p{12cm}}{ Verify availability of property sheets for table rows } \\ \hline
\textbf{Owner} & \textbf{Status} & \textbf{Version} & \textbf{Critical Event} & \textbf{Verification Type} \\ \hline
Jeffrey Carlin & Draft & 1 & false & Inspection \\ \hline
\end{longtable}
{\scriptsize
\textbf{Objective:}\\
Verify that the Portal permits inspection of a row in tabular data query
results, summarizing metadata such as units, semantic information, and
relationships between columns.
}
  
 \newpage 
\subsection{[LVV-9889] DMS-PRTL-REQ-0050-V-01: Column Selection of Tabular Data\_1 }\label{lvv-9889}

\begin{longtable}{cccc}
\hline
\textbf{Jira Link} & \textbf{Assignee} & \textbf{Status} & \textbf{Test Cases}\\ \hline
\href{https://jira.lsstcorp.org/browse/LVV-9889}{LVV-9889} &
Gregory Dubois-Felsmann & Not Covered &
\begin{tabular}{c}
LVV-T6 \\
LVV-T685 \\
\end{tabular}
\\
\hline
\end{longtable}

\textbf{Verification Element Description:} \\
Undefined

{\footnotesize
\begin{longtable}{p{2.5cm}p{13.5cm}}
\hline
\multicolumn{2}{c}{\textbf{Requirement Details}}\\ \hline
Requirement ID & DMS-PRTL-REQ-0050 \\ \cdashline{1-2}
Requirement Description &
\begin{minipage}[]{13cm}
The Portal aspect shall provide the capability to select, for display
and downloading, specific columns within the tabular data viewer.
\end{minipage}
\\ \cdashline{1-2}
Requirement Discussion &
\begin{minipage}[]{13cm}
The intent of this requirement is to enable users to decide which
columns are desired for display and download.
\end{minipage}
\\ \cdashline{1-2}
Requirement Priority &  \\ \cdashline{1-2}
Upper Level Requirement &
\begin{tabular}{cl}
\end{tabular}
\\ \hline
\end{longtable}
}


\subsubsection{Test Cases Summary}
\begin{longtable}{p{3cm}p{2.5cm}p{2.5cm}p{3cm}p{4cm}}
\toprule
\href{https://jira.lsstcorp.org/secure/Tests.jspa\#/testCase/LVV-T6}{LVV-T6} & \multicolumn{4}{p{12cm}}{ LSP-00-20: Operation of the UI for interaction with tabular data results } \\ \hline
\textbf{Owner} & \textbf{Status} & \textbf{Version} & \textbf{Critical Event} & \textbf{Verification Type} \\ \hline
Gregory Dubois-Felsmann & Deprecated & 1 & false & Test \\ \hline
\end{longtable}
{\scriptsize
\textbf{Objective:}\\
This test will test the functional requirements to be able to perform
certain basic exploratory data analysis functions on tabular data
results in the Portal Aspect UI:

\begin{itemize}
\tightlist
\item
  Sort tabular results;
\item
  Filter tabular results based on the contents of columns;~
\item
  Perform per-row selections from a table;
\item
  Display 1D histograms of selected attributes;
\item
  Display 2D scatter plots of selected attributes;
\item
  Perform graphical selections of rows from plots; and
\item
  Download tabular query results reflecting sorting and selection.
\end{itemize}

This test does not address the limits of scaling of these capabilities
to large query results. That will be addressed in future test
specifications. The test report should include notes on the sizes of
results that were used.
}
\begin{longtable}{p{3cm}p{2.5cm}p{2.5cm}p{3cm}p{4cm}}
\toprule
\href{https://jira.lsstcorp.org/secure/Tests.jspa\#/testCase/LVV-T685}{LVV-T685} & \multicolumn{4}{p{12cm}}{ Verify column selection from tables } \\ \hline
\textbf{Owner} & \textbf{Status} & \textbf{Version} & \textbf{Critical Event} & \textbf{Verification Type} \\ \hline
Jeffrey Carlin & Draft & 1 & false & Inspection \\ \hline
\end{longtable}
{\scriptsize
\textbf{Objective:}\\
Verify that the Portal provides the capability to select specific
columns from tabular data, for display and download.
}
  
 \newpage 
\subsection{[LVV-9890] DMS-PRTL-REQ-0052-V-01: Copying of Tabular Data\_1 }\label{lvv-9890}

\begin{longtable}{cccc}
\hline
\textbf{Jira Link} & \textbf{Assignee} & \textbf{Status} & \textbf{Test Cases}\\ \hline
\href{https://jira.lsstcorp.org/browse/LVV-9890}{LVV-9890} &
Gregory Dubois-Felsmann & Not Covered &
\begin{tabular}{c}
LVV-T687 \\
\end{tabular}
\\
\hline
\end{longtable}

\textbf{Verification Element Description:} \\
Undefined

{\footnotesize
\begin{longtable}{p{2.5cm}p{13.5cm}}
\hline
\multicolumn{2}{c}{\textbf{Requirement Details}}\\ \hline
Requirement ID & DMS-PRTL-REQ-0052 \\ \cdashline{1-2}
Requirement Description &
\begin{minipage}[]{13cm}
The Portal aspect shall provide the capability of interactively
selecting and copying data within a displayed data table.
\end{minipage}
\\ \cdashline{1-2}
Requirement Discussion &
\begin{minipage}[]{13cm}
The intent of this requirement is to enable users to use the mouse to
select fields within a displayed table and utilize standard copy
mechanisms.
\end{minipage}
\\ \cdashline{1-2}
Requirement Priority &  \\ \cdashline{1-2}
Upper Level Requirement &
\begin{tabular}{cl}
\end{tabular}
\\ \hline
\end{longtable}
}


\subsubsection{Test Cases Summary}
\begin{longtable}{p{3cm}p{2.5cm}p{2.5cm}p{3cm}p{4cm}}
\toprule
\href{https://jira.lsstcorp.org/secure/Tests.jspa\#/testCase/LVV-T687}{LVV-T687} & \multicolumn{4}{p{12cm}}{ Verify capability of copying data in tables } \\ \hline
\textbf{Owner} & \textbf{Status} & \textbf{Version} & \textbf{Critical Event} & \textbf{Verification Type} \\ \hline
Jeffrey Carlin & Draft & 1 & false & Inspection \\ \hline
\end{longtable}
{\scriptsize
\textbf{Objective:}\\
Verify that data can be interactively selected and copied from displayed
tables in the Portal aspect.
}
  
 \newpage 
\subsection{[LVV-9891] DMS-PRTL-REQ-0049-V-01: Display of Tabular Data\_1 }\label{lvv-9891}

\begin{longtable}{cccc}
\hline
\textbf{Jira Link} & \textbf{Assignee} & \textbf{Status} & \textbf{Test Cases}\\ \hline
\href{https://jira.lsstcorp.org/browse/LVV-9891}{LVV-9891} &
Gregory Dubois-Felsmann & Not Covered &
\begin{tabular}{c}
LVV-T6 \\
LVV-T684 \\
LVV-T1334 \\
\end{tabular}
\\
\hline
\end{longtable}

\textbf{Verification Element Description:} \\
Undefined

{\footnotesize
\begin{longtable}{p{2.5cm}p{13.5cm}}
\hline
\multicolumn{2}{c}{\textbf{Requirement Details}}\\ \hline
Requirement ID & DMS-PRTL-REQ-0049 \\ \cdashline{1-2}
Requirement Description &
\begin{minipage}[]{13cm}
The Portal aspect provide the capability to display tabular data in an
interactive environment which displays the tables by columns and rows.
\end{minipage}
\\ \cdashline{1-2}
Requirement Discussion &
\begin{minipage}[]{13cm}
The intent of this requirement is to capture that the database query
returns are displayed.
\end{minipage}
\\ \cdashline{1-2}
Requirement Priority &  \\ \cdashline{1-2}
Upper Level Requirement &
\begin{tabular}{cl}
\end{tabular}
\\ \hline
\end{longtable}
}


\subsubsection{Test Cases Summary}
\begin{longtable}{p{3cm}p{2.5cm}p{2.5cm}p{3cm}p{4cm}}
\toprule
\href{https://jira.lsstcorp.org/secure/Tests.jspa\#/testCase/LVV-T6}{LVV-T6} & \multicolumn{4}{p{12cm}}{ LSP-00-20: Operation of the UI for interaction with tabular data results } \\ \hline
\textbf{Owner} & \textbf{Status} & \textbf{Version} & \textbf{Critical Event} & \textbf{Verification Type} \\ \hline
Gregory Dubois-Felsmann & Deprecated & 1 & false & Test \\ \hline
\end{longtable}
{\scriptsize
\textbf{Objective:}\\
This test will test the functional requirements to be able to perform
certain basic exploratory data analysis functions on tabular data
results in the Portal Aspect UI:

\begin{itemize}
\tightlist
\item
  Sort tabular results;
\item
  Filter tabular results based on the contents of columns;~
\item
  Perform per-row selections from a table;
\item
  Display 1D histograms of selected attributes;
\item
  Display 2D scatter plots of selected attributes;
\item
  Perform graphical selections of rows from plots; and
\item
  Download tabular query results reflecting sorting and selection.
\end{itemize}

This test does not address the limits of scaling of these capabilities
to large query results. That will be addressed in future test
specifications. The test report should include notes on the sizes of
results that were used.
}
\begin{longtable}{p{3cm}p{2.5cm}p{2.5cm}p{3cm}p{4cm}}
\toprule
\href{https://jira.lsstcorp.org/secure/Tests.jspa\#/testCase/LVV-T684}{LVV-T684} & \multicolumn{4}{p{12cm}}{ Verify display of tabular data } \\ \hline
\textbf{Owner} & \textbf{Status} & \textbf{Version} & \textbf{Critical Event} & \textbf{Verification Type} \\ \hline
Jeffrey Carlin & Draft & 1 & false & Inspection \\ \hline
\end{longtable}
{\scriptsize
\textbf{Objective:}\\
Verify that the Portal provides an interactive environment that displays
table data by columns and rows.
}
\begin{longtable}{p{3cm}p{2.5cm}p{2.5cm}p{3cm}p{4cm}}
\toprule
\href{https://jira.lsstcorp.org/secure/Tests.jspa\#/testCase/LVV-T1334}{LVV-T1334} & \multicolumn{4}{p{12cm}}{ LDM-503-10a: Portal Aspect tests for LSP with Authentication and TAP
milestone } \\ \hline
\textbf{Owner} & \textbf{Status} & \textbf{Version} & \textbf{Critical Event} & \textbf{Verification Type} \\ \hline
Gregory Dubois-Felsmann & Defined & 1 & false & Test \\ \hline
\end{longtable}
{\scriptsize
\textbf{Objective:}\\
This test case verifies that the Portal Aspect of the Science Platform
is accessible to authorized users through a login process, and that TAP
searches can be performed from the Portal Aspect UI.\\[2\baselineskip]In
so doing and in conjunction with the other LDM-503-10a test cases
collected under LVV-P48, it addresses all or part of the following
requirements:

\begin{itemize}
\tightlist
\item
  DMS-LSP-REQ-0002, DMS-LSP-REQ-0005, DMS-LSP-REQ-0006,
  DMS-LSP-REQ-0020, DMS-LSP-REQ-0022, DMS-LSP-REQ-0023, DMS-LSP-REQ-0024
\item
  DMS-PRTL-REQ-0001, DMS-PRTL-REQ-0015, DMS-PRTL-REQ-0016,
  DMS-PRTL-REQ-0017, DMS-PRTL-REQ-0020, DMS-PRTL-REQ-0026,
  DMS-PRTL-REQ-0049, and DMS-PRTL-REQ-0095, primarily
\end{itemize}

Note this test was not designed to perform a full verification of the
above requirements, but rather to demonstrate having reached a certain
level of partial capability during construction.
}
  
 \newpage 
\subsection{[LVV-9892] DMS-PRTL-REQ-0051-V-01: Display Order of Columns of Tabular Data\_1 }\label{lvv-9892}

\begin{longtable}{cccc}
\hline
\textbf{Jira Link} & \textbf{Assignee} & \textbf{Status} & \textbf{Test Cases}\\ \hline
\href{https://jira.lsstcorp.org/browse/LVV-9892}{LVV-9892} &
Gregory Dubois-Felsmann & Not Covered &
\begin{tabular}{c}
LVV-T686 \\
\end{tabular}
\\
\hline
\end{longtable}

\textbf{Verification Element Description:} \\
Undefined

{\footnotesize
\begin{longtable}{p{2.5cm}p{13.5cm}}
\hline
\multicolumn{2}{c}{\textbf{Requirement Details}}\\ \hline
Requirement ID & DMS-PRTL-REQ-0051 \\ \cdashline{1-2}
Requirement Description &
\begin{minipage}[]{13cm}
The Portal aspect shall provide the capability to change the display
order of the columns for tabular data.
\end{minipage}
\\ \cdashline{1-2}
Requirement Discussion &
\begin{minipage}[]{13cm}
The intent of this requirement is to enable users to decide which order
to view the columns.
\end{minipage}
\\ \cdashline{1-2}
Requirement Priority &  \\ \cdashline{1-2}
Upper Level Requirement &
\begin{tabular}{cl}
\end{tabular}
\\ \hline
\end{longtable}
}


\subsubsection{Test Cases Summary}
\begin{longtable}{p{3cm}p{2.5cm}p{2.5cm}p{3cm}p{4cm}}
\toprule
\href{https://jira.lsstcorp.org/secure/Tests.jspa\#/testCase/LVV-T686}{LVV-T686} & \multicolumn{4}{p{12cm}}{ Verify capability to re-order columns in displayed tabular data } \\ \hline
\textbf{Owner} & \textbf{Status} & \textbf{Version} & \textbf{Critical Event} & \textbf{Verification Type} \\ \hline
Jeffrey Carlin & Draft & 1 & false & Inspection \\ \hline
\end{longtable}
{\scriptsize
\textbf{Objective:}\\
Verify that the Portal provides capability to change the order in which
columns of tabular data are displayed.
}
  
 \newpage 
\subsection{[LVV-9893] DMS-PRTL-REQ-0054-V-01: Paging of Tabular Data\_1 }\label{lvv-9893}

\begin{longtable}{cccc}
\hline
\textbf{Jira Link} & \textbf{Assignee} & \textbf{Status} & \textbf{Test Cases}\\ \hline
\href{https://jira.lsstcorp.org/browse/LVV-9893}{LVV-9893} &
Gregory Dubois-Felsmann & Not Covered &
\begin{tabular}{c}
LVV-T6 \\
LVV-T689 \\
\end{tabular}
\\
\hline
\end{longtable}

\textbf{Verification Element Description:} \\
Undefined

{\footnotesize
\begin{longtable}{p{2.5cm}p{13.5cm}}
\hline
\multicolumn{2}{c}{\textbf{Requirement Details}}\\ \hline
Requirement ID & DMS-PRTL-REQ-0054 \\ \cdashline{1-2}
Requirement Description &
\begin{minipage}[]{13cm}
The Portal aspect shall provide the capability to display tabular data
in a paged format.
\end{minipage}
\\ \cdashline{1-2}
Requirement Discussion &
\begin{minipage}[]{13cm}
The intent of this requirement is to capture that the database query
returns may be too large to display all at once to the user and a form
of paging will be necessary.
\end{minipage}
\\ \cdashline{1-2}
Requirement Priority &  \\ \cdashline{1-2}
Upper Level Requirement &
\begin{tabular}{cl}
\end{tabular}
\\ \hline
\end{longtable}
}


\subsubsection{Test Cases Summary}
\begin{longtable}{p{3cm}p{2.5cm}p{2.5cm}p{3cm}p{4cm}}
\toprule
\href{https://jira.lsstcorp.org/secure/Tests.jspa\#/testCase/LVV-T6}{LVV-T6} & \multicolumn{4}{p{12cm}}{ LSP-00-20: Operation of the UI for interaction with tabular data results } \\ \hline
\textbf{Owner} & \textbf{Status} & \textbf{Version} & \textbf{Critical Event} & \textbf{Verification Type} \\ \hline
Gregory Dubois-Felsmann & Deprecated & 1 & false & Test \\ \hline
\end{longtable}
{\scriptsize
\textbf{Objective:}\\
This test will test the functional requirements to be able to perform
certain basic exploratory data analysis functions on tabular data
results in the Portal Aspect UI:

\begin{itemize}
\tightlist
\item
  Sort tabular results;
\item
  Filter tabular results based on the contents of columns;~
\item
  Perform per-row selections from a table;
\item
  Display 1D histograms of selected attributes;
\item
  Display 2D scatter plots of selected attributes;
\item
  Perform graphical selections of rows from plots; and
\item
  Download tabular query results reflecting sorting and selection.
\end{itemize}

This test does not address the limits of scaling of these capabilities
to large query results. That will be addressed in future test
specifications. The test report should include notes on the sizes of
results that were used.
}
\begin{longtable}{p{3cm}p{2.5cm}p{2.5cm}p{3cm}p{4cm}}
\toprule
\href{https://jira.lsstcorp.org/secure/Tests.jspa\#/testCase/LVV-T689}{LVV-T689} & \multicolumn{4}{p{12cm}}{ Verify capability to display tabular data in paged format } \\ \hline
\textbf{Owner} & \textbf{Status} & \textbf{Version} & \textbf{Critical Event} & \textbf{Verification Type} \\ \hline
Jeffrey Carlin & Draft & 1 & false & Inspection \\ \hline
\end{longtable}
{\scriptsize
\textbf{Objective:}\\
Verify that the Portal aspect provides the capability to display tabular
data in a paged format, in the case that database queries return results
too large to display on a single page.
}
  
 \newpage 
\subsection{[LVV-9894] DMS-PRTL-REQ-0053-V-01: Row Selection of Tabular Data\_1 }\label{lvv-9894}

\begin{longtable}{cccc}
\hline
\textbf{Jira Link} & \textbf{Assignee} & \textbf{Status} & \textbf{Test Cases}\\ \hline
\href{https://jira.lsstcorp.org/browse/LVV-9894}{LVV-9894} &
Gregory Dubois-Felsmann & Not Covered &
\begin{tabular}{c}
LVV-T6 \\
LVV-T688 \\
\end{tabular}
\\
\hline
\end{longtable}

\textbf{Verification Element Description:} \\
Undefined

{\footnotesize
\begin{longtable}{p{2.5cm}p{13.5cm}}
\hline
\multicolumn{2}{c}{\textbf{Requirement Details}}\\ \hline
Requirement ID & DMS-PRTL-REQ-0053 \\ \cdashline{1-2}
Requirement Description &
\begin{minipage}[]{13cm}
The Portal aspect shall provide the capability to select, for display
and downloading, specific rows within the tabular data.
\end{minipage}
\\ \cdashline{1-2}
Requirement Discussion &
\begin{minipage}[]{13cm}
The intent of this requirement is to enable users to decide which rows
are desired for display and download and these may be different from the
filtered rows (see filtering requirement below).
\end{minipage}
\\ \cdashline{1-2}
Requirement Priority &  \\ \cdashline{1-2}
Upper Level Requirement &
\begin{tabular}{cl}
\end{tabular}
\\ \hline
\end{longtable}
}


\subsubsection{Test Cases Summary}
\begin{longtable}{p{3cm}p{2.5cm}p{2.5cm}p{3cm}p{4cm}}
\toprule
\href{https://jira.lsstcorp.org/secure/Tests.jspa\#/testCase/LVV-T6}{LVV-T6} & \multicolumn{4}{p{12cm}}{ LSP-00-20: Operation of the UI for interaction with tabular data results } \\ \hline
\textbf{Owner} & \textbf{Status} & \textbf{Version} & \textbf{Critical Event} & \textbf{Verification Type} \\ \hline
Gregory Dubois-Felsmann & Deprecated & 1 & false & Test \\ \hline
\end{longtable}
{\scriptsize
\textbf{Objective:}\\
This test will test the functional requirements to be able to perform
certain basic exploratory data analysis functions on tabular data
results in the Portal Aspect UI:

\begin{itemize}
\tightlist
\item
  Sort tabular results;
\item
  Filter tabular results based on the contents of columns;~
\item
  Perform per-row selections from a table;
\item
  Display 1D histograms of selected attributes;
\item
  Display 2D scatter plots of selected attributes;
\item
  Perform graphical selections of rows from plots; and
\item
  Download tabular query results reflecting sorting and selection.
\end{itemize}

This test does not address the limits of scaling of these capabilities
to large query results. That will be addressed in future test
specifications. The test report should include notes on the sizes of
results that were used.
}
\begin{longtable}{p{3cm}p{2.5cm}p{2.5cm}p{3cm}p{4cm}}
\toprule
\href{https://jira.lsstcorp.org/secure/Tests.jspa\#/testCase/LVV-T688}{LVV-T688} & \multicolumn{4}{p{12cm}}{ Verify row selection from tables } \\ \hline
\textbf{Owner} & \textbf{Status} & \textbf{Version} & \textbf{Critical Event} & \textbf{Verification Type} \\ \hline
Jeffrey Carlin & Draft & 1 & false & Inspection \\ \hline
\end{longtable}
{\scriptsize
\textbf{Objective:}\\
Verify that the Portal provides the capability to select specific rows
from tabular data, for display and download.
}
  
 \newpage 
\subsection{[LVV-9895] DMS-PRTL-REQ-0056-V-01: Histograms\_1 }\label{lvv-9895}

\begin{longtable}{cccc}
\hline
\textbf{Jira Link} & \textbf{Assignee} & \textbf{Status} & \textbf{Test Cases}\\ \hline
\href{https://jira.lsstcorp.org/browse/LVV-9895}{LVV-9895} &
Gregory Dubois-Felsmann & Not Covered &
\begin{tabular}{c}
LVV-T6 \\
LVV-T691 \\
\end{tabular}
\\
\hline
\end{longtable}

\textbf{Verification Element Description:} \\
Undefined

{\footnotesize
\begin{longtable}{p{2.5cm}p{13.5cm}}
\hline
\multicolumn{2}{c}{\textbf{Requirement Details}}\\ \hline
Requirement ID & DMS-PRTL-REQ-0056 \\ \cdashline{1-2}
Requirement Description &
\begin{minipage}[]{13cm}
The Portal aspect shall enable the creation and display of 1-dimensional
and 2-dimensional histograms of tabular data.
\end{minipage}
\\ \cdashline{1-2}
Requirement Discussion &
\begin{minipage}[]{13cm}
This requirement is about producing traditional histograms in 1 and 2
dimensions where the number of items within a bin are reported
\end{minipage}
\\ \cdashline{1-2}
Requirement Priority &  \\ \cdashline{1-2}
Upper Level Requirement &
\begin{tabular}{cl}
\end{tabular}
\\ \hline
\end{longtable}
}


\subsubsection{Test Cases Summary}
\begin{longtable}{p{3cm}p{2.5cm}p{2.5cm}p{3cm}p{4cm}}
\toprule
\href{https://jira.lsstcorp.org/secure/Tests.jspa\#/testCase/LVV-T6}{LVV-T6} & \multicolumn{4}{p{12cm}}{ LSP-00-20: Operation of the UI for interaction with tabular data results } \\ \hline
\textbf{Owner} & \textbf{Status} & \textbf{Version} & \textbf{Critical Event} & \textbf{Verification Type} \\ \hline
Gregory Dubois-Felsmann & Deprecated & 1 & false & Test \\ \hline
\end{longtable}
{\scriptsize
\textbf{Objective:}\\
This test will test the functional requirements to be able to perform
certain basic exploratory data analysis functions on tabular data
results in the Portal Aspect UI:

\begin{itemize}
\tightlist
\item
  Sort tabular results;
\item
  Filter tabular results based on the contents of columns;~
\item
  Perform per-row selections from a table;
\item
  Display 1D histograms of selected attributes;
\item
  Display 2D scatter plots of selected attributes;
\item
  Perform graphical selections of rows from plots; and
\item
  Download tabular query results reflecting sorting and selection.
\end{itemize}

This test does not address the limits of scaling of these capabilities
to large query results. That will be addressed in future test
specifications. The test report should include notes on the sizes of
results that were used.
}
\begin{longtable}{p{3cm}p{2.5cm}p{2.5cm}p{3cm}p{4cm}}
\toprule
\href{https://jira.lsstcorp.org/secure/Tests.jspa\#/testCase/LVV-T691}{LVV-T691} & \multicolumn{4}{p{12cm}}{ Verify creation and display of histogram plots } \\ \hline
\textbf{Owner} & \textbf{Status} & \textbf{Version} & \textbf{Critical Event} & \textbf{Verification Type} \\ \hline
Jeffrey Carlin & Draft & 1 & false & Inspection \\ \hline
\end{longtable}
{\scriptsize
\textbf{Objective:}\\
Verify that the Portal provides the capability to create and display
1-dimensional and 2-dimensional histogram plots from tabular data.
}
  
 \newpage 
\subsection{[LVV-9896] DMS-PRTL-REQ-0061-V-01: Multiple XY-Plots on the Same Display\_1 }\label{lvv-9896}

\begin{longtable}{cccc}
\hline
\textbf{Jira Link} & \textbf{Assignee} & \textbf{Status} & \textbf{Test Cases}\\ \hline
\href{https://jira.lsstcorp.org/browse/LVV-9896}{LVV-9896} &
Gregory Dubois-Felsmann & Not Covered &
\begin{tabular}{c}
LVV-T696 \\
\end{tabular}
\\
\hline
\end{longtable}

\textbf{Verification Element Description:} \\
Undefined

{\footnotesize
\begin{longtable}{p{2.5cm}p{13.5cm}}
\hline
\multicolumn{2}{c}{\textbf{Requirement Details}}\\ \hline
Requirement ID & DMS-PRTL-REQ-0061 \\ \cdashline{1-2}
Requirement Description &
\begin{minipage}[]{13cm}
The Portal aspect shall be able to overlay multiple plots on the same
display, differentiated by plotting colors, symbols, line styles, and
shading.
\end{minipage}
\\ \cdashline{1-2}
Requirement Discussion &
\begin{minipage}[]{13cm}
The intent of this requirement is enable the plotting of multiple graphs
on the same plotting canvas. An example would be RA vs Time and Dec vs
Time or u-band mag vs i-z color and r-band mag vs i-z color.
\end{minipage}
\\ \cdashline{1-2}
Requirement Priority &  \\ \cdashline{1-2}
Upper Level Requirement &
\begin{tabular}{cl}
\end{tabular}
\\ \hline
\end{longtable}
}


\subsubsection{Test Cases Summary}
\begin{longtable}{p{3cm}p{2.5cm}p{2.5cm}p{3cm}p{4cm}}
\toprule
\href{https://jira.lsstcorp.org/secure/Tests.jspa\#/testCase/LVV-T696}{LVV-T696} & \multicolumn{4}{p{12cm}}{ Verify visualization of multiple XY plots on the same display } \\ \hline
\textbf{Owner} & \textbf{Status} & \textbf{Version} & \textbf{Critical Event} & \textbf{Verification Type} \\ \hline
Jeffrey Carlin & Draft & 1 & false & Inspection \\ \hline
\end{longtable}
{\scriptsize
\textbf{Objective:}\\
Verify that the Portal provides the capability to display multiple XY
plots on a single display canvas.
}
  
 \newpage 
\subsection{[LVV-9897] DMS-PRTL-REQ-0059-V-01: Plot Asymmetric Quantitative Uncertainties\_1 }\label{lvv-9897}

\begin{longtable}{cccc}
\hline
\textbf{Jira Link} & \textbf{Assignee} & \textbf{Status} & \textbf{Test Cases}\\ \hline
\href{https://jira.lsstcorp.org/browse/LVV-9897}{LVV-9897} &
Gregory Dubois-Felsmann & Not Covered &
\begin{tabular}{c}
LVV-T694 \\
\end{tabular}
\\
\hline
\end{longtable}

\textbf{Verification Element Description:} \\
Undefined

{\footnotesize
\begin{longtable}{p{2.5cm}p{13.5cm}}
\hline
\multicolumn{2}{c}{\textbf{Requirement Details}}\\ \hline
Requirement ID & DMS-PRTL-REQ-0059 \\ \cdashline{1-2}
Requirement Description &
\begin{minipage}[]{13cm}
The Portal aspect shall be able represent uncertainties in the plotting
of data that are unequal in value for the positive and negative
directions.
\end{minipage}
\\ \cdashline{1-2}
Requirement Discussion &
\begin{minipage}[]{13cm}
Uncertainties often have different limits in the positive and negative
directions and as a result representation of the uncertainties will be
different.
\end{minipage}
\\ \cdashline{1-2}
Requirement Priority &  \\ \cdashline{1-2}
Upper Level Requirement &
\begin{tabular}{cl}
\end{tabular}
\\ \hline
\end{longtable}
}


\subsubsection{Test Cases Summary}
\begin{longtable}{p{3cm}p{2.5cm}p{2.5cm}p{3cm}p{4cm}}
\toprule
\href{https://jira.lsstcorp.org/secure/Tests.jspa\#/testCase/LVV-T694}{LVV-T694} & \multicolumn{4}{p{12cm}}{ Verify visualization of asymmetric uncertainties } \\ \hline
\textbf{Owner} & \textbf{Status} & \textbf{Version} & \textbf{Critical Event} & \textbf{Verification Type} \\ \hline
Jeffrey Carlin & Draft & 1 & false & Inspection \\ \hline
\end{longtable}
{\scriptsize
\textbf{Objective:}\\
Verify that the Portal aspect can display uncertainties that are
asymmetric (i.e., differ in the positive and negative directions).~
}
  
 \newpage 
\subsection{[LVV-9898] DMS-PRTL-REQ-0058-V-01: Plot Quantitative Uncertainties\_1 }\label{lvv-9898}

\begin{longtable}{cccc}
\hline
\textbf{Jira Link} & \textbf{Assignee} & \textbf{Status} & \textbf{Test Cases}\\ \hline
\href{https://jira.lsstcorp.org/browse/LVV-9898}{LVV-9898} &
Gregory Dubois-Felsmann & Not Covered &
\begin{tabular}{c}
LVV-T693 \\
\end{tabular}
\\
\hline
\end{longtable}

\textbf{Verification Element Description:} \\
Undefined

{\footnotesize
\begin{longtable}{p{2.5cm}p{13.5cm}}
\hline
\multicolumn{2}{c}{\textbf{Requirement Details}}\\ \hline
Requirement ID & DMS-PRTL-REQ-0058 \\ \cdashline{1-2}
Requirement Description &
\begin{minipage}[]{13cm}
The Portal aspect shall be able represent uncertainties in the plotting
of data.
\end{minipage}
\\ \cdashline{1-2}
Requirement Discussion &
\begin{minipage}[]{13cm}
This flows down from higher-level requirements above, and has
implications for the creation of the necessary metadata to support this
in the first place.
\end{minipage}
\\ \cdashline{1-2}
Requirement Priority &  \\ \cdashline{1-2}
Upper Level Requirement &
\begin{tabular}{cl}
\end{tabular}
\\ \hline
\end{longtable}
}


\subsubsection{Test Cases Summary}
\begin{longtable}{p{3cm}p{2.5cm}p{2.5cm}p{3cm}p{4cm}}
\toprule
\href{https://jira.lsstcorp.org/secure/Tests.jspa\#/testCase/LVV-T693}{LVV-T693} & \multicolumn{4}{p{12cm}}{ Verify visualization of uncertainties in plots } \\ \hline
\textbf{Owner} & \textbf{Status} & \textbf{Version} & \textbf{Critical Event} & \textbf{Verification Type} \\ \hline
Jeffrey Carlin & Draft & 1 & false & Inspection \\ \hline
\end{longtable}
{\scriptsize
\textbf{Objective:}\\
Verify the capability to represent uncertainties in plots of tabular
data.
}
  
 \newpage 
\subsection{[LVV-9899] DMS-PRTL-REQ-0060-V-01: Plot Upper and Lower Quantitative Limits\_1 }\label{lvv-9899}

\begin{longtable}{cccc}
\hline
\textbf{Jira Link} & \textbf{Assignee} & \textbf{Status} & \textbf{Test Cases}\\ \hline
\href{https://jira.lsstcorp.org/browse/LVV-9899}{LVV-9899} &
Gregory Dubois-Felsmann & Not Covered &
\begin{tabular}{c}
LVV-T695 \\
\end{tabular}
\\
\hline
\end{longtable}

\textbf{Verification Element Description:} \\
Undefined

{\footnotesize
\begin{longtable}{p{2.5cm}p{13.5cm}}
\hline
\multicolumn{2}{c}{\textbf{Requirement Details}}\\ \hline
Requirement ID & DMS-PRTL-REQ-0060 \\ \cdashline{1-2}
Requirement Description &
\begin{minipage}[]{13cm}
The Portal aspect shall be able represent upper and lower limits in the
plotting of tabular data.
\end{minipage}
\\ \cdashline{1-2}
Requirement Discussion &
\begin{minipage}[]{13cm}
Often the values are non-detections and a limit is estimated on that
value. The limit can be an upper limit (e.g., flux) or a lower limit
(e.g., magnitude). Typically, the limit is represented by a flat line at
the position of the value and an up or down arrow from that position.\\
For tables with full metadata including UCDs this can be handled
generically: the ``stat.min'' and ``stat.max'' UCDs can be used to
trigger the display of the distinctive symbols.
\end{minipage}
\\ \cdashline{1-2}
Requirement Priority &  \\ \cdashline{1-2}
Upper Level Requirement &
\begin{tabular}{cl}
\end{tabular}
\\ \hline
\end{longtable}
}


\subsubsection{Test Cases Summary}
\begin{longtable}{p{3cm}p{2.5cm}p{2.5cm}p{3cm}p{4cm}}
\toprule
\href{https://jira.lsstcorp.org/secure/Tests.jspa\#/testCase/LVV-T695}{LVV-T695} & \multicolumn{4}{p{12cm}}{ Verify visualization of upper and lower limits in plots } \\ \hline
\textbf{Owner} & \textbf{Status} & \textbf{Version} & \textbf{Critical Event} & \textbf{Verification Type} \\ \hline
Jeffrey Carlin & Draft & 1 & false & Inspection \\ \hline
\end{longtable}
{\scriptsize
\textbf{Objective:}\\
Verify that the Portal is capable of displaying quantities that
represent upper or lower limits (provided, for example, for
non-detections).
}
  
 \newpage 
\subsection{[LVV-9900] DMS-PRTL-REQ-0057-V-01: Symbol Size, Shape, and Color Coding in XY(Z)
Scatter Plots\_1 }\label{lvv-9900}

\begin{longtable}{cccc}
\hline
\textbf{Jira Link} & \textbf{Assignee} & \textbf{Status} & \textbf{Test Cases}\\ \hline
\href{https://jira.lsstcorp.org/browse/LVV-9900}{LVV-9900} &
Gregory Dubois-Felsmann & Not Covered &
\begin{tabular}{c}
LVV-T692 \\
\end{tabular}
\\
\hline
\end{longtable}

\textbf{Verification Element Description:} \\
Undefined

{\footnotesize
\begin{longtable}{p{2.5cm}p{13.5cm}}
\hline
\multicolumn{2}{c}{\textbf{Requirement Details}}\\ \hline
Requirement ID & DMS-PRTL-REQ-0057 \\ \cdashline{1-2}
Requirement Description &
\begin{minipage}[]{13cm}
The Portal aspect shall enable the use of symbol size, shape, and color
as indicators of additional tabular data associated with the XY(Z)-data
plotted.
\end{minipage}
\\ \cdashline{1-2}
Requirement Discussion &
\begin{minipage}[]{13cm}
This is a generalized requirement to enable the visualization of
multi-dimensional data in a 2-d scatter plot environment.
\end{minipage}
\\ \cdashline{1-2}
Requirement Priority &  \\ \cdashline{1-2}
Upper Level Requirement &
\begin{tabular}{cl}
\end{tabular}
\\ \hline
\end{longtable}
}


\subsubsection{Test Cases Summary}
\begin{longtable}{p{3cm}p{2.5cm}p{2.5cm}p{3cm}p{4cm}}
\toprule
\href{https://jira.lsstcorp.org/secure/Tests.jspa\#/testCase/LVV-T692}{LVV-T692} & \multicolumn{4}{p{12cm}}{ Verify capability to change symbol shapes, sizes, and colors in XY(Z)
scatter plots } \\ \hline
\textbf{Owner} & \textbf{Status} & \textbf{Version} & \textbf{Critical Event} & \textbf{Verification Type} \\ \hline
Jeffrey Carlin & Draft & 1 & false & Inspection \\ \hline
\end{longtable}
{\scriptsize
\textbf{Objective:}\\
Verify that users can change the shape, size, and color of symbols in
XY(Z) scatter plots to indicate information from additional dimensions
of tabular data.
}
  
 \newpage 
\subsection{[LVV-9901] DMS-PRTL-REQ-0055-V-01: XY Scatter Plots\_1 }\label{lvv-9901}

\begin{longtable}{cccc}
\hline
\textbf{Jira Link} & \textbf{Assignee} & \textbf{Status} & \textbf{Test Cases}\\ \hline
\href{https://jira.lsstcorp.org/browse/LVV-9901}{LVV-9901} &
Gregory Dubois-Felsmann & Not Covered &
\begin{tabular}{c}
LVV-T6 \\
LVV-T690 \\
\end{tabular}
\\
\hline
\end{longtable}

\textbf{Verification Element Description:} \\
Undefined

{\footnotesize
\begin{longtable}{p{2.5cm}p{13.5cm}}
\hline
\multicolumn{2}{c}{\textbf{Requirement Details}}\\ \hline
Requirement ID & DMS-PRTL-REQ-0055 \\ \cdashline{1-2}
Requirement Description &
\begin{minipage}[]{13cm}
The Portal aspect shall enable the creation and display of 2-dimensional
xy-plots from tabular data.
\end{minipage}
\\ \cdashline{1-2}
Requirement Priority &  \\ \cdashline{1-2}
Upper Level Requirement &
\begin{tabular}{cl}
\end{tabular}
\\ \hline
\end{longtable}
}


\subsubsection{Test Cases Summary}
\begin{longtable}{p{3cm}p{2.5cm}p{2.5cm}p{3cm}p{4cm}}
\toprule
\href{https://jira.lsstcorp.org/secure/Tests.jspa\#/testCase/LVV-T6}{LVV-T6} & \multicolumn{4}{p{12cm}}{ LSP-00-20: Operation of the UI for interaction with tabular data results } \\ \hline
\textbf{Owner} & \textbf{Status} & \textbf{Version} & \textbf{Critical Event} & \textbf{Verification Type} \\ \hline
Gregory Dubois-Felsmann & Deprecated & 1 & false & Test \\ \hline
\end{longtable}
{\scriptsize
\textbf{Objective:}\\
This test will test the functional requirements to be able to perform
certain basic exploratory data analysis functions on tabular data
results in the Portal Aspect UI:

\begin{itemize}
\tightlist
\item
  Sort tabular results;
\item
  Filter tabular results based on the contents of columns;~
\item
  Perform per-row selections from a table;
\item
  Display 1D histograms of selected attributes;
\item
  Display 2D scatter plots of selected attributes;
\item
  Perform graphical selections of rows from plots; and
\item
  Download tabular query results reflecting sorting and selection.
\end{itemize}

This test does not address the limits of scaling of these capabilities
to large query results. That will be addressed in future test
specifications. The test report should include notes on the sizes of
results that were used.
}
\begin{longtable}{p{3cm}p{2.5cm}p{2.5cm}p{3cm}p{4cm}}
\toprule
\href{https://jira.lsstcorp.org/secure/Tests.jspa\#/testCase/LVV-T690}{LVV-T690} & \multicolumn{4}{p{12cm}}{ Verify creation and display of X-Y scatter plots } \\ \hline
\textbf{Owner} & \textbf{Status} & \textbf{Version} & \textbf{Critical Event} & \textbf{Verification Type} \\ \hline
Jeffrey Carlin & Draft & 1 & false & Inspection \\ \hline
\end{longtable}
{\scriptsize
\textbf{Objective:}\\
Verify that the Portal provides the capability to create and display
2-dimensional X-Y scatter plots from tabular data.
}
  
 \newpage 
\subsection{[LVV-9902] DMS-PRTL-REQ-0067-V-01: Display Calibration Image Data Products\_1 }\label{lvv-9902}

\begin{longtable}{cccc}
\hline
\textbf{Jira Link} & \textbf{Assignee} & \textbf{Status} & \textbf{Test Cases}\\ \hline
\href{https://jira.lsstcorp.org/browse/LVV-9902}{LVV-9902} &
Gregory Dubois-Felsmann & Not Covered &
\begin{tabular}{c}
LVV-T701 \\
\end{tabular}
\\
\hline
\end{longtable}

\textbf{Verification Element Description:} \\
Undefined

{\footnotesize
\begin{longtable}{p{2.5cm}p{13.5cm}}
\hline
\multicolumn{2}{c}{\textbf{Requirement Details}}\\ \hline
Requirement ID & DMS-PRTL-REQ-0067 \\ \cdashline{1-2}
Requirement Description &
\begin{minipage}[]{13cm}
The Portal aspect shall have the capability to display the calibration
image data products such as synthetic flats, bias frames, and the like.
\end{minipage}
\\ \cdashline{1-2}
Requirement Discussion &
\begin{minipage}[]{13cm}
Note that these images may not have WCS information.
\end{minipage}
\\ \cdashline{1-2}
Requirement Priority &  \\ \cdashline{1-2}
Upper Level Requirement &
\begin{tabular}{cl}
\end{tabular}
\\ \hline
\end{longtable}
}


\subsubsection{Test Cases Summary}
\begin{longtable}{p{3cm}p{2.5cm}p{2.5cm}p{3cm}p{4cm}}
\toprule
\href{https://jira.lsstcorp.org/secure/Tests.jspa\#/testCase/LVV-T701}{LVV-T701} & \multicolumn{4}{p{12cm}}{ Verify display of calibration images } \\ \hline
\textbf{Owner} & \textbf{Status} & \textbf{Version} & \textbf{Critical Event} & \textbf{Verification Type} \\ \hline
Jeffrey Carlin & Draft & 1 & false & Inspection \\ \hline
\end{longtable}
{\scriptsize
\textbf{Objective:}\\
Verify that the Portal is capable of displaying calibration image data
products, including synthetic flats, bias frames, etc.
}
  
 \newpage 
\subsection{[LVV-9903] DMS-PRTL-REQ-0066-V-01: Display Coadded Image Cutouts / Mosaics\_1 }\label{lvv-9903}

\begin{longtable}{cccc}
\hline
\textbf{Jira Link} & \textbf{Assignee} & \textbf{Status} & \textbf{Test Cases}\\ \hline
\href{https://jira.lsstcorp.org/browse/LVV-9903}{LVV-9903} &
Gregory Dubois-Felsmann & Not Covered &
\begin{tabular}{c}
LVV-T700 \\
\end{tabular}
\\
\hline
\end{longtable}

\textbf{Verification Element Description:} \\
Undefined

{\footnotesize
\begin{longtable}{p{2.5cm}p{13.5cm}}
\hline
\multicolumn{2}{c}{\textbf{Requirement Details}}\\ \hline
Requirement ID & DMS-PRTL-REQ-0066 \\ \cdashline{1-2}
Requirement Description &
\begin{minipage}[]{13cm}
The Portal aspect shall have the capability to display cutouts and
mosaics from coadded image data products, as delivered from the API
aspect.
\end{minipage}
\\ \cdashline{1-2}
Requirement Discussion &
\begin{minipage}[]{13cm}
Corresponds to the ability to request the generation of cutouts and
mosaics. These may cover areas both smaller than and larger than the
native ``patch'' scale.
\end{minipage}
\\ \cdashline{1-2}
Requirement Priority &  \\ \cdashline{1-2}
Upper Level Requirement &
\begin{tabular}{cl}
\end{tabular}
\\ \hline
\end{longtable}
}


\subsubsection{Test Cases Summary}
\begin{longtable}{p{3cm}p{2.5cm}p{2.5cm}p{3cm}p{4cm}}
\toprule
\href{https://jira.lsstcorp.org/secure/Tests.jspa\#/testCase/LVV-T700}{LVV-T700} & \multicolumn{4}{p{12cm}}{ Verify display of coadd cutouts and mosaics } \\ \hline
\textbf{Owner} & \textbf{Status} & \textbf{Version} & \textbf{Critical Event} & \textbf{Verification Type} \\ \hline
Jeffrey Carlin & Draft & 1 & false & Inspection \\ \hline
\end{longtable}
{\scriptsize
\textbf{Objective:}\\
Verify that the Portal aspect has the capability to display cutout or
mosaic images created from coadds.
}
  
 \newpage 
\subsection{[LVV-9904] DMS-PRTL-REQ-0065-V-01: Display Native Coadded Image Data Products\_1 }\label{lvv-9904}

\begin{longtable}{cccc}
\hline
\textbf{Jira Link} & \textbf{Assignee} & \textbf{Status} & \textbf{Test Cases}\\ \hline
\href{https://jira.lsstcorp.org/browse/LVV-9904}{LVV-9904} &
Gregory Dubois-Felsmann & Not Covered &
\begin{tabular}{c}
LVV-T699 \\
\end{tabular}
\\
\hline
\end{longtable}

\textbf{Verification Element Description:} \\
Undefined

{\footnotesize
\begin{longtable}{p{2.5cm}p{13.5cm}}
\hline
\multicolumn{2}{c}{\textbf{Requirement Details}}\\ \hline
Requirement ID & DMS-PRTL-REQ-0065 \\ \cdashline{1-2}
Requirement Description &
\begin{minipage}[]{13cm}
The Portal aspect shall have the capability to display the native
coadded image data products, i.e., the patch-level images.
\end{minipage}
\\ \cdashline{1-2}
Requirement Priority &  \\ \cdashline{1-2}
Upper Level Requirement &
\begin{tabular}{cl}
\end{tabular}
\\ \hline
\end{longtable}
}


\subsubsection{Test Cases Summary}
\begin{longtable}{p{3cm}p{2.5cm}p{2.5cm}p{3cm}p{4cm}}
\toprule
\href{https://jira.lsstcorp.org/secure/Tests.jspa\#/testCase/LVV-T699}{LVV-T699} & \multicolumn{4}{p{12cm}}{ Verify display of native coadd images } \\ \hline
\textbf{Owner} & \textbf{Status} & \textbf{Version} & \textbf{Critical Event} & \textbf{Verification Type} \\ \hline
Jeffrey Carlin & Draft & 1 & false & Inspection \\ \hline
\end{longtable}
{\scriptsize
\textbf{Objective:}\\
Verify that the Portal can display native coadd image products (i.e.,
patch-level images).
}
  
 \newpage 
\subsection{[LVV-9905] DMS-PRTL-REQ-0062-V-01: Display Native Single-Visit Image Data
Products\_1 }\label{lvv-9905}

\begin{longtable}{cccc}
\hline
\textbf{Jira Link} & \textbf{Assignee} & \textbf{Status} & \textbf{Test Cases}\\ \hline
\href{https://jira.lsstcorp.org/browse/LVV-9905}{LVV-9905} &
Gregory Dubois-Felsmann & Not Covered &
\begin{tabular}{c}
LVV-T676 \\
\end{tabular}
\\
\hline
\end{longtable}

\textbf{Verification Element Description:} \\
Undefined

{\footnotesize
\begin{longtable}{p{2.5cm}p{13.5cm}}
\hline
\multicolumn{2}{c}{\textbf{Requirement Details}}\\ \hline
Requirement ID & DMS-PRTL-REQ-0062 \\ \cdashline{1-2}
Requirement Description &
\begin{minipage}[]{13cm}
The Portal aspect shall have the ability to display the native
single-visit image data products, including raw images, Processed Visit
Images (PVIs), and difference images, as well as the standard
single-exposure calibration images used as inputs for flats, bias
frames, etc.
\end{minipage}
\\ \cdashline{1-2}
Requirement Discussion &
\begin{minipage}[]{13cm}
The native raw data will contain amplifier-level data with full
pre-scan, serial overscan, and parallel overscan.\\
The other native single-visit data products will be at CCD level.
\end{minipage}
\\ \cdashline{1-2}
Requirement Priority &  \\ \cdashline{1-2}
Upper Level Requirement &
\begin{tabular}{cl}
\end{tabular}
\\ \hline
\end{longtable}
}


\subsubsection{Test Cases Summary}
\begin{longtable}{p{3cm}p{2.5cm}p{2.5cm}p{3cm}p{4cm}}
\toprule
\href{https://jira.lsstcorp.org/secure/Tests.jspa\#/testCase/LVV-T676}{LVV-T676} & \multicolumn{4}{p{12cm}}{ Verify display of native single-visit images } \\ \hline
\textbf{Owner} & \textbf{Status} & \textbf{Version} & \textbf{Critical Event} & \textbf{Verification Type} \\ \hline
Jeffrey Carlin & Draft & 1 & false & Inspection \\ \hline
\end{longtable}
{\scriptsize
\textbf{Objective:}\\
Verify that the Portal aspect provides a means to display the native
single-visit image data products, including raw images, Processed Visit
Images (PVIs), and difference images, as well as the standard
single-exposure calibration images used as inputs for flats, bias
frames, etc.
}
  
 \newpage 
\subsection{[LVV-9906] DMS-PRTL-REQ-0063-V-01: Display Raft- and Focal-Plane-Level Single-Visit
Image Data\_1 }\label{lvv-9906}

\begin{longtable}{cccc}
\hline
\textbf{Jira Link} & \textbf{Assignee} & \textbf{Status} & \textbf{Test Cases}\\ \hline
\href{https://jira.lsstcorp.org/browse/LVV-9906}{LVV-9906} &
Gregory Dubois-Felsmann & Not Covered &
\begin{tabular}{c}
LVV-T697 \\
\end{tabular}
\\
\hline
\end{longtable}

\textbf{Verification Element Description:} \\
Undefined

{\footnotesize
\begin{longtable}{p{2.5cm}p{13.5cm}}
\hline
\multicolumn{2}{c}{\textbf{Requirement Details}}\\ \hline
Requirement ID & DMS-PRTL-REQ-0063 \\ \cdashline{1-2}
Requirement Description &
\begin{minipage}[]{13cm}
The Portal aspect shall have the capability to generate a synthetic
display of image data at raft level and at full focal plane (FPA) level.
\end{minipage}
\\ \cdashline{1-2}
Requirement Discussion &
\begin{minipage}[]{13cm}
No such data products will exist per se; the requirement is for the
Portal to be able to show, e.g., the coverage of raft-level or FPA-level
visit imaging on the sky.
\end{minipage}
\\ \cdashline{1-2}
Requirement Priority &  \\ \cdashline{1-2}
Upper Level Requirement &
\begin{tabular}{cl}
\end{tabular}
\\ \hline
\end{longtable}
}


\subsubsection{Test Cases Summary}
\begin{longtable}{p{3cm}p{2.5cm}p{2.5cm}p{3cm}p{4cm}}
\toprule
\href{https://jira.lsstcorp.org/secure/Tests.jspa\#/testCase/LVV-T697}{LVV-T697} & \multicolumn{4}{p{12cm}}{ Verify display of raft and full focal-plane single-visit images } \\ \hline
\textbf{Owner} & \textbf{Status} & \textbf{Version} & \textbf{Critical Event} & \textbf{Verification Type} \\ \hline
Jeffrey Carlin & Draft & 1 & false & Inspection \\ \hline
\end{longtable}
{\scriptsize
\textbf{Objective:}\\
Verify that the Portal aspect has the ability to generate a single-visit
image display of a raft and full focal-plane image.
}
  
 \newpage 
\subsection{[LVV-9907] DMS-PRTL-REQ-0064-V-01: Display Single Visit Image Cut-Out\_1 }\label{lvv-9907}

\begin{longtable}{cccc}
\hline
\textbf{Jira Link} & \textbf{Assignee} & \textbf{Status} & \textbf{Test Cases}\\ \hline
\href{https://jira.lsstcorp.org/browse/LVV-9907}{LVV-9907} &
Gregory Dubois-Felsmann & Not Covered &
\begin{tabular}{c}
LVV-T698 \\
\end{tabular}
\\
\hline
\end{longtable}

\textbf{Verification Element Description:} \\
Undefined

{\footnotesize
\begin{longtable}{p{2.5cm}p{13.5cm}}
\hline
\multicolumn{2}{c}{\textbf{Requirement Details}}\\ \hline
Requirement ID & DMS-PRTL-REQ-0064 \\ \cdashline{1-2}
Requirement Description &
\begin{minipage}[]{13cm}
The Portal aspect shall have the capability to display a cutout from a
single visit image.
\end{minipage}
\\ \cdashline{1-2}
Requirement Discussion &
\begin{minipage}[]{13cm}
Corresponds to the ability to request the generation of cutouts and
cross-CCD mosaics.
\end{minipage}
\\ \cdashline{1-2}
Requirement Priority &  \\ \cdashline{1-2}
Upper Level Requirement &
\begin{tabular}{cl}
\end{tabular}
\\ \hline
\end{longtable}
}


\subsubsection{Test Cases Summary}
\begin{longtable}{p{3cm}p{2.5cm}p{2.5cm}p{3cm}p{4cm}}
\toprule
\href{https://jira.lsstcorp.org/secure/Tests.jspa\#/testCase/LVV-T698}{LVV-T698} & \multicolumn{4}{p{12cm}}{ Verify display of cutout from single-visit image } \\ \hline
\textbf{Owner} & \textbf{Status} & \textbf{Version} & \textbf{Critical Event} & \textbf{Verification Type} \\ \hline
Jeffrey Carlin & Draft & 1 & false & Inspection \\ \hline
\end{longtable}
{\scriptsize
\textbf{Objective:}\\
Verify that the Portal is capable of displaying a cutout from a
single-visit image.
}
  
 \newpage 
\subsection{[LVV-9908] DMS-PRTL-REQ-0068-V-01: Display User-provided Images\_1 }\label{lvv-9908}

\begin{longtable}{cccc}
\hline
\textbf{Jira Link} & \textbf{Assignee} & \textbf{Status} & \textbf{Test Cases}\\ \hline
\href{https://jira.lsstcorp.org/browse/LVV-9908}{LVV-9908} &
Gregory Dubois-Felsmann & Not Covered &
\begin{tabular}{c}
LVV-T702 \\
\end{tabular}
\\
\hline
\end{longtable}

\textbf{Verification Element Description:} \\
Undefined

{\footnotesize
\begin{longtable}{p{2.5cm}p{13.5cm}}
\hline
\multicolumn{2}{c}{\textbf{Requirement Details}}\\ \hline
Requirement ID & DMS-PRTL-REQ-0068 \\ \cdashline{1-2}
Requirement Description &
\begin{minipage}[]{13cm}
The Portal aspect shall have the capability to display user-provided
images in widely-used astronomical community formats, including FITS,
and shall properly interpret a variety of commonly-used WCS
specifications in the image headers.
\end{minipage}
\\ \cdashline{1-2}
Requirement Discussion &
\begin{minipage}[]{13cm}
This would be used, for instance, to allow a user to view LSST catalog
data superposed on a user-provided image. FITS is the only currently
supported image file format; others will be considered as community
usage develops or as part of supporting all-sky visualization.
\end{minipage}
\\ \cdashline{1-2}
Requirement Priority &  \\ \cdashline{1-2}
Upper Level Requirement &
\begin{tabular}{cl}
\end{tabular}
\\ \hline
\end{longtable}
}


\subsubsection{Test Cases Summary}
\begin{longtable}{p{3cm}p{2.5cm}p{2.5cm}p{3cm}p{4cm}}
\toprule
\href{https://jira.lsstcorp.org/secure/Tests.jspa\#/testCase/LVV-T702}{LVV-T702} & \multicolumn{4}{p{12cm}}{ Verify display of user-provided images } \\ \hline
\textbf{Owner} & \textbf{Status} & \textbf{Version} & \textbf{Critical Event} & \textbf{Verification Type} \\ \hline
Jeffrey Carlin & Draft & 1 & false & Inspection \\ \hline
\end{longtable}
{\scriptsize
\textbf{Objective:}\\
Verify that the Portal has the capability of displaying user-provided
images in widely-used astronomical data formats, and properly interprets
commonly-used WCS specifications from the image headers. This includes
FITS format, and may be extended to others.~
}
  
 \newpage 
\subsection{[LVV-9909] DMS-PRTL-REQ-0069-V-01: Image Property Sheet\_1 }\label{lvv-9909}

\begin{longtable}{cccc}
\hline
\textbf{Jira Link} & \textbf{Assignee} & \textbf{Status} & \textbf{Test Cases}\\ \hline
\href{https://jira.lsstcorp.org/browse/LVV-9909}{LVV-9909} &
Gregory Dubois-Felsmann & Not Covered &
\begin{tabular}{c}
LVV-T703 \\
\end{tabular}
\\
\hline
\end{longtable}

\textbf{Verification Element Description:} \\
Undefined

{\footnotesize
\begin{longtable}{p{2.5cm}p{13.5cm}}
\hline
\multicolumn{2}{c}{\textbf{Requirement Details}}\\ \hline
Requirement ID & DMS-PRTL-REQ-0069 \\ \cdashline{1-2}
Requirement Description &
\begin{minipage}[]{13cm}
The Portal shall have the ability to display a property sheet for an
image data product or user-provided image, displaying image format and
other header data.
\end{minipage}
\\ \cdashline{1-2}
Requirement Discussion &
\begin{minipage}[]{13cm}
This is information on the image artifact itself, e.g., on the contents
of FITS headers. It should not be confused with the property sheet
associated with an image metadata table entry, though there may be a
strong overlap in content, and for UX purposes they may be displayed in
combined screens.
\end{minipage}
\\ \cdashline{1-2}
Requirement Priority &  \\ \cdashline{1-2}
Upper Level Requirement &
\begin{tabular}{cl}
\end{tabular}
\\ \hline
\end{longtable}
}


\subsubsection{Test Cases Summary}
\begin{longtable}{p{3cm}p{2.5cm}p{2.5cm}p{3cm}p{4cm}}
\toprule
\href{https://jira.lsstcorp.org/secure/Tests.jspa\#/testCase/LVV-T703}{LVV-T703} & \multicolumn{4}{p{12cm}}{ Verify display of image property sheet } \\ \hline
\textbf{Owner} & \textbf{Status} & \textbf{Version} & \textbf{Critical Event} & \textbf{Verification Type} \\ \hline
Jeffrey Carlin & Draft & 1 & false & Inspection \\ \hline
\end{longtable}
{\scriptsize
\textbf{Objective:}\\
Verify that the Portal has the ability to display a property sheet for
an image data product or user-provided image, displaying image format
and other header data.
}
  
 \newpage 
\subsection{[LVV-9910] DMS-PRTL-REQ-0074-V-01: Image Appearance Manipulation\_1 }\label{lvv-9910}

\begin{longtable}{cccc}
\hline
\textbf{Jira Link} & \textbf{Assignee} & \textbf{Status} & \textbf{Test Cases}\\ \hline
\href{https://jira.lsstcorp.org/browse/LVV-9910}{LVV-9910} &
Gregory Dubois-Felsmann & Not Covered &
\begin{tabular}{c}
LVV-T708 \\
\end{tabular}
\\
\hline
\end{longtable}

\textbf{Verification Element Description:} \\
Undefined

{\footnotesize
\begin{longtable}{p{2.5cm}p{13.5cm}}
\hline
\multicolumn{2}{c}{\textbf{Requirement Details}}\\ \hline
Requirement ID & DMS-PRTL-REQ-0074 \\ \cdashline{1-2}
Requirement Description &
\begin{minipage}[]{13cm}
The Portal aspect shall enable the user to change the view of a
displayed image including, but not necessarily limited to, the color
table, the stretch function, and the displayed data range.
\end{minipage}
\\ \cdashline{1-2}
Requirement Priority &  \\ \cdashline{1-2}
Upper Level Requirement &
\begin{tabular}{cl}
\end{tabular}
\\ \hline
\end{longtable}
}


\subsubsection{Test Cases Summary}
\begin{longtable}{p{3cm}p{2.5cm}p{2.5cm}p{3cm}p{4cm}}
\toprule
\href{https://jira.lsstcorp.org/secure/Tests.jspa\#/testCase/LVV-T708}{LVV-T708} & \multicolumn{4}{p{12cm}}{ Verify manipulation of image appearance } \\ \hline
\textbf{Owner} & \textbf{Status} & \textbf{Version} & \textbf{Critical Event} & \textbf{Verification Type} \\ \hline
Jeffrey Carlin & Draft & 1 & false & Inspection \\ \hline
\end{longtable}
{\scriptsize
\textbf{Objective:}\\
Verify that the Portal enables users to manipulate the appearance of
displayed images, including changing the stretch, color table, or
displayed data range.
}
  
 \newpage 
\subsection{[LVV-9911] DMS-PRTL-REQ-0071-V-01: Image Pixel Content Display\_1 }\label{lvv-9911}

\begin{longtable}{cccc}
\hline
\textbf{Jira Link} & \textbf{Assignee} & \textbf{Status} & \textbf{Test Cases}\\ \hline
\href{https://jira.lsstcorp.org/browse/LVV-9911}{LVV-9911} &
Gregory Dubois-Felsmann & Not Covered &
\begin{tabular}{c}
LVV-T705 \\
\end{tabular}
\\
\hline
\end{longtable}

\textbf{Verification Element Description:} \\
Undefined

{\footnotesize
\begin{longtable}{p{2.5cm}p{13.5cm}}
\hline
\multicolumn{2}{c}{\textbf{Requirement Details}}\\ \hline
Requirement ID & DMS-PRTL-REQ-0071 \\ \cdashline{1-2}
Requirement Description &
\begin{minipage}[]{13cm}
The Portal aspect shall have the capability to inspect the pixel content
of an image at the position of the mouse cursor. This capability shall
be integrated with the Point Coordinate Display capability for the
image.
\end{minipage}
\\ \cdashline{1-2}
Requirement Priority &  \\ \cdashline{1-2}
Upper Level Requirement &
\begin{tabular}{cl}
\end{tabular}
\\ \hline
\end{longtable}
}


\subsubsection{Test Cases Summary}
\begin{longtable}{p{3cm}p{2.5cm}p{2.5cm}p{3cm}p{4cm}}
\toprule
\href{https://jira.lsstcorp.org/secure/Tests.jspa\#/testCase/LVV-T705}{LVV-T705} & \multicolumn{4}{p{12cm}}{ Verify image pixel content display } \\ \hline
\textbf{Owner} & \textbf{Status} & \textbf{Version} & \textbf{Critical Event} & \textbf{Verification Type} \\ \hline
Jeffrey Carlin & Draft & 1 & false & Inspection \\ \hline
\end{longtable}
{\scriptsize
\textbf{Objective:}\\
Verify that the Portal provides the capability to inspect the pixel
contents of an image at the cursor position.
}
  
 \newpage 
\subsection{[LVV-9912] DMS-PRTL-REQ-0072-V-01: Image Spatial Manipulation\_1 }\label{lvv-9912}

\begin{longtable}{cccc}
\hline
\textbf{Jira Link} & \textbf{Assignee} & \textbf{Status} & \textbf{Test Cases}\\ \hline
\href{https://jira.lsstcorp.org/browse/LVV-9912}{LVV-9912} &
Gregory Dubois-Felsmann & Not Covered &
\begin{tabular}{c}
LVV-T706 \\
\end{tabular}
\\
\hline
\end{longtable}

\textbf{Verification Element Description:} \\
Undefined

{\footnotesize
\begin{longtable}{p{2.5cm}p{13.5cm}}
\hline
\multicolumn{2}{c}{\textbf{Requirement Details}}\\ \hline
Requirement ID & DMS-PRTL-REQ-0072 \\ \cdashline{1-2}
Requirement Description &
\begin{minipage}[]{13cm}
The Portal aspect shall have the capability to resize, rescale,
re-project, zoom, and crop the image display and save or download the
current view.
\end{minipage}
\\ \cdashline{1-2}
Requirement Priority &  \\ \cdashline{1-2}
Upper Level Requirement &
\begin{tabular}{cl}
\end{tabular}
\\ \hline
\end{longtable}
}


\subsubsection{Test Cases Summary}
\begin{longtable}{p{3cm}p{2.5cm}p{2.5cm}p{3cm}p{4cm}}
\toprule
\href{https://jira.lsstcorp.org/secure/Tests.jspa\#/testCase/LVV-T706}{LVV-T706} & \multicolumn{4}{p{12cm}}{ Verify spatial manipulation of images } \\ \hline
\textbf{Owner} & \textbf{Status} & \textbf{Version} & \textbf{Critical Event} & \textbf{Verification Type} \\ \hline
Jeffrey Carlin & Draft & 1 & false & Inspection \\ \hline
\end{longtable}
{\scriptsize
\textbf{Objective:}\\
Verify that the Portal allows users to spatially manipulate displayed
images, including resizing, rescaling, reprojecting, zooming, and
cropping.
}
  
 \newpage 
\subsection{[LVV-9913] DMS-PRTL-REQ-0073-V-01: Multi-Image Scaling and Aligning\_1 }\label{lvv-9913}

\begin{longtable}{cccc}
\hline
\textbf{Jira Link} & \textbf{Assignee} & \textbf{Status} & \textbf{Test Cases}\\ \hline
\href{https://jira.lsstcorp.org/browse/LVV-9913}{LVV-9913} &
Gregory Dubois-Felsmann & Not Covered &
\begin{tabular}{c}
LVV-T707 \\
\end{tabular}
\\
\hline
\end{longtable}

\textbf{Verification Element Description:} \\
Undefined

{\footnotesize
\begin{longtable}{p{2.5cm}p{13.5cm}}
\hline
\multicolumn{2}{c}{\textbf{Requirement Details}}\\ \hline
Requirement ID & DMS-PRTL-REQ-0073 \\ \cdashline{1-2}
Requirement Description &
\begin{minipage}[]{13cm}
The Portal aspect shall have the capability to display multiple images
on a common astrophysical coordinate scale and aligned on the screen in
a common astrophysical orientation.
\end{minipage}
\\ \cdashline{1-2}
Requirement Discussion &
\begin{minipage}[]{13cm}
The point behind this requirement is to enable viewing the same part of
the sky in different filters aligned to same orientation and scaled to
the same screen resolution for both single-frame and coadded images.
\end{minipage}
\\ \cdashline{1-2}
Requirement Priority &  \\ \cdashline{1-2}
Upper Level Requirement &
\begin{tabular}{cl}
\end{tabular}
\\ \hline
\end{longtable}
}


\subsubsection{Test Cases Summary}
\begin{longtable}{p{3cm}p{2.5cm}p{2.5cm}p{3cm}p{4cm}}
\toprule
\href{https://jira.lsstcorp.org/secure/Tests.jspa\#/testCase/LVV-T707}{LVV-T707} & \multicolumn{4}{p{12cm}}{ Verify multi-image scaling and alignment } \\ \hline
\textbf{Owner} & \textbf{Status} & \textbf{Version} & \textbf{Critical Event} & \textbf{Verification Type} \\ \hline
Jeffrey Carlin & Draft & 1 & false & Inspection \\ \hline
\end{longtable}
{\scriptsize
\textbf{Objective:}\\
Verify that the Portal has the capability to display multiple images on
a common astrophysical coordinate scale, aligned on the screen in a
common orientation.
}
  
 \newpage 
\subsection{[LVV-9914] DMS-PRTL-REQ-0070-V-01: Provide Coordinate Display Tools for Images\_1 }\label{lvv-9914}

\begin{longtable}{cccc}
\hline
\textbf{Jira Link} & \textbf{Assignee} & \textbf{Status} & \textbf{Test Cases}\\ \hline
\href{https://jira.lsstcorp.org/browse/LVV-9914}{LVV-9914} &
Gregory Dubois-Felsmann & Not Covered &
\begin{tabular}{c}
LVV-T704 \\
\end{tabular}
\\
\hline
\end{longtable}

\textbf{Verification Element Description:} \\
Undefined

{\footnotesize
\begin{longtable}{p{2.5cm}p{13.5cm}}
\hline
\multicolumn{2}{c}{\textbf{Requirement Details}}\\ \hline
Requirement ID & DMS-PRTL-REQ-0070 \\ \cdashline{1-2}
Requirement Description &
\begin{minipage}[]{13cm}
The Portal shall provide all the capabilities in the Coordinate Display
Tools section herein for image displays. Specific capabilities will
depend on the availability of WCS information for an image.
\end{minipage}
\\ \cdashline{1-2}
Requirement Priority &  \\ \cdashline{1-2}
Upper Level Requirement &
\begin{tabular}{cl}
\end{tabular}
\\ \hline
\end{longtable}
}


\subsubsection{Test Cases Summary}
\begin{longtable}{p{3cm}p{2.5cm}p{2.5cm}p{3cm}p{4cm}}
\toprule
\href{https://jira.lsstcorp.org/secure/Tests.jspa\#/testCase/LVV-T704}{LVV-T704} & \multicolumn{4}{p{12cm}}{ Verify that coordinate display tools are provided for images } \\ \hline
\textbf{Owner} & \textbf{Status} & \textbf{Version} & \textbf{Critical Event} & \textbf{Verification Type} \\ \hline
Jeffrey Carlin & Draft & 1 & false & Inspection \\ \hline
\end{longtable}
{\scriptsize
\textbf{Objective:}\\
Verify that the Portal provides all the capabilities in the Coordinate
Display Tools section in \citeds{LDM-554} for image displays. Specific
capabilities will depend on the availability of WCS information for an
image.
}
  
 \newpage 
\subsection{[LVV-9915] DMS-PRTL-REQ-0075-V-01: Image Mask and Variance Overlays\_1 }\label{lvv-9915}

\begin{longtable}{cccc}
\hline
\textbf{Jira Link} & \textbf{Assignee} & \textbf{Status} & \textbf{Test Cases}\\ \hline
\href{https://jira.lsstcorp.org/browse/LVV-9915}{LVV-9915} &
Gregory Dubois-Felsmann & Not Covered &
\begin{tabular}{c}
LVV-T709 \\
\end{tabular}
\\
\hline
\end{longtable}

\textbf{Verification Element Description:} \\
Undefined

{\footnotesize
\begin{longtable}{p{2.5cm}p{13.5cm}}
\hline
\multicolumn{2}{c}{\textbf{Requirement Details}}\\ \hline
Requirement ID & DMS-PRTL-REQ-0075 \\ \cdashline{1-2}
Requirement Description &
\begin{minipage}[]{13cm}
The Portal aspect shall enable the overlaying of additional
pixel-oriented data on an image, including image masks (bit planes) and
variance data.
\end{minipage}
\\ \cdashline{1-2}
Requirement Discussion &
\begin{minipage}[]{13cm}
This also enables, but does not require, the overplotting of
two-dimensional density plots such as depth maps.
\end{minipage}
\\ \cdashline{1-2}
Requirement Priority &  \\ \cdashline{1-2}
Upper Level Requirement &
\begin{tabular}{cl}
\end{tabular}
\\ \hline
\end{longtable}
}


\subsubsection{Test Cases Summary}
\begin{longtable}{p{3cm}p{2.5cm}p{2.5cm}p{3cm}p{4cm}}
\toprule
\href{https://jira.lsstcorp.org/secure/Tests.jspa\#/testCase/LVV-T709}{LVV-T709} & \multicolumn{4}{p{12cm}}{ Verify display of image mask and variance overlays } \\ \hline
\textbf{Owner} & \textbf{Status} & \textbf{Version} & \textbf{Critical Event} & \textbf{Verification Type} \\ \hline
Jeffrey Carlin & Draft & 1 & false & Inspection \\ \hline
\end{longtable}
{\scriptsize
\textbf{Objective:}\\
Verify that the Portal enables overlaying pixel-based data on top of
already displayed images, including image masks (bit planes) and
variance data.
}
  
 \newpage 
\subsection{[LVV-9916] DMS-PRTL-REQ-0077-V-01: Image Overlays: Adjustment of Colors and
Positions\_1 }\label{lvv-9916}

\begin{longtable}{cccc}
\hline
\textbf{Jira Link} & \textbf{Assignee} & \textbf{Status} & \textbf{Test Cases}\\ \hline
\href{https://jira.lsstcorp.org/browse/LVV-9916}{LVV-9916} &
Gregory Dubois-Felsmann & Not Covered &
\begin{tabular}{c}
LVV-T711 \\
\end{tabular}
\\
\hline
\end{longtable}

\textbf{Verification Element Description:} \\
Undefined

{\footnotesize
\begin{longtable}{p{2.5cm}p{13.5cm}}
\hline
\multicolumn{2}{c}{\textbf{Requirement Details}}\\ \hline
Requirement ID & DMS-PRTL-REQ-0077 \\ \cdashline{1-2}
Requirement Description &
\begin{minipage}[]{13cm}
The Portal aspect shall have the capability for a user to configure the
annotations, colors, transparency, and positions (where applicable) of
any image overlays, including those resulting from the use of the
Coordinate Tools.
\end{minipage}
\\ \cdashline{1-2}
Requirement Discussion &
\begin{minipage}[]{13cm}
Often the default color and position of the overlay needs to be changed
for clarification.
\end{minipage}
\\ \cdashline{1-2}
Requirement Priority &  \\ \cdashline{1-2}
Upper Level Requirement &
\begin{tabular}{cl}
\end{tabular}
\\ \hline
\end{longtable}
}


\subsubsection{Test Cases Summary}
\begin{longtable}{p{3cm}p{2.5cm}p{2.5cm}p{3cm}p{4cm}}
\toprule
\href{https://jira.lsstcorp.org/secure/Tests.jspa\#/testCase/LVV-T711}{LVV-T711} & \multicolumn{4}{p{12cm}}{ Verify capability to adjust the appearance of plot overlays on images } \\ \hline
\textbf{Owner} & \textbf{Status} & \textbf{Version} & \textbf{Critical Event} & \textbf{Verification Type} \\ \hline
Jeffrey Carlin & Draft & 1 & false & Inspection \\ \hline
\end{longtable}
{\scriptsize
\textbf{Objective:}\\
Verify that the Portal enables users to adjust the annotations, colors,
transparency, and positions of plot overlays displayed on top of
images.~
}
  
 \newpage 
\subsection{[LVV-9917] DMS-PRTL-REQ-0076-V-01: Image Plot Overlays\_1 }\label{lvv-9917}

\begin{longtable}{cccc}
\hline
\textbf{Jira Link} & \textbf{Assignee} & \textbf{Status} & \textbf{Test Cases}\\ \hline
\href{https://jira.lsstcorp.org/browse/LVV-9917}{LVV-9917} &
Gregory Dubois-Felsmann & Not Covered &
\begin{tabular}{c}
LVV-T710 \\
\end{tabular}
\\
\hline
\end{longtable}

\textbf{Verification Element Description:} \\
Undefined

{\footnotesize
\begin{longtable}{p{2.5cm}p{13.5cm}}
\hline
\multicolumn{2}{c}{\textbf{Requirement Details}}\\ \hline
Requirement ID & DMS-PRTL-REQ-0076 \\ \cdashline{1-2}
Requirement Description &
\begin{minipage}[]{13cm}
The Portal aspect shall enable the overlaying of tabular data on an
image, either based on pixel coordinates or astrophysical coordinates,
as supported by the availability of coordinate system information.
\end{minipage}
\\ \cdashline{1-2}
Requirement Discussion &
\begin{minipage}[]{13cm}
More generally, this should work for any two-dimensional plot data that
shares a coordinate system with an image. The catalog data need not be
semantically linked with the image by anything other than the coordinate
system used.
\end{minipage}
\\ \cdashline{1-2}
Requirement Priority &  \\ \cdashline{1-2}
Upper Level Requirement &
\begin{tabular}{cl}
\end{tabular}
\\ \hline
\end{longtable}
}


\subsubsection{Test Cases Summary}
\begin{longtable}{p{3cm}p{2.5cm}p{2.5cm}p{3cm}p{4cm}}
\toprule
\href{https://jira.lsstcorp.org/secure/Tests.jspa\#/testCase/LVV-T710}{LVV-T710} & \multicolumn{4}{p{12cm}}{ Verify display of plot overlays on images } \\ \hline
\textbf{Owner} & \textbf{Status} & \textbf{Version} & \textbf{Critical Event} & \textbf{Verification Type} \\ \hline
Jeffrey Carlin & Draft & 1 & false & Inspection \\ \hline
\end{longtable}
{\scriptsize
\textbf{Objective:}\\
Verify that the Portal has the capability to overlay tabular data on an
image, based on input image or astrophysical coordinates, as supported
by availability of coordinate system information.
}
  
 \newpage 
\subsection{[LVV-9918] DMS-PRTL-REQ-0078-V-01: Display All-Sky HEALPix Image\_1 }\label{lvv-9918}

\begin{longtable}{cccc}
\hline
\textbf{Jira Link} & \textbf{Assignee} & \textbf{Status} & \textbf{Test Cases}\\ \hline
\href{https://jira.lsstcorp.org/browse/LVV-9918}{LVV-9918} &
Gregory Dubois-Felsmann & Not Covered &
\begin{tabular}{c}
LVV-T712 \\
\end{tabular}
\\
\hline
\end{longtable}

\textbf{Verification Element Description:} \\
Undefined

{\footnotesize
\begin{longtable}{p{2.5cm}p{13.5cm}}
\hline
\multicolumn{2}{c}{\textbf{Requirement Details}}\\ \hline
Requirement ID & DMS-PRTL-REQ-0078 \\ \cdashline{1-2}
Requirement Description &
\begin{minipage}[]{13cm}
The Portal aspect shall be able to display an all-sky image in the
HEALPix format.
\end{minipage}
\\ \cdashline{1-2}
Requirement Discussion &
\begin{minipage}[]{13cm}
In addition to flux images such as all-sky coadds, the LSST pipelines
and/or quality assessment processes may generate a variety of all-sky
metrics, diagnostics, and other artifacts in this format.
\end{minipage}
\\ \cdashline{1-2}
Requirement Priority &  \\ \cdashline{1-2}
Upper Level Requirement &
\begin{tabular}{cl}
\end{tabular}
\\ \hline
\end{longtable}
}


\subsubsection{Test Cases Summary}
\begin{longtable}{p{3cm}p{2.5cm}p{2.5cm}p{3cm}p{4cm}}
\toprule
\href{https://jira.lsstcorp.org/secure/Tests.jspa\#/testCase/LVV-T712}{LVV-T712} & \multicolumn{4}{p{12cm}}{ Verify display all-sky HEALPix image } \\ \hline
\textbf{Owner} & \textbf{Status} & \textbf{Version} & \textbf{Critical Event} & \textbf{Verification Type} \\ \hline
Jeffrey Carlin & Draft & 1 & false & Inspection \\ \hline
\end{longtable}
{\scriptsize
\textbf{Objective:}\\
Verify that the Portal aspect is able to display an all-sky image in the
HEALPix format.
}
  
 \newpage 
\subsection{[LVV-9919] DMS-PRTL-REQ-0081-V-01: HEALPix Pixel Selection\_1 }\label{lvv-9919}

\begin{longtable}{cccc}
\hline
\textbf{Jira Link} & \textbf{Assignee} & \textbf{Status} & \textbf{Test Cases}\\ \hline
\href{https://jira.lsstcorp.org/browse/LVV-9919}{LVV-9919} &
Gregory Dubois-Felsmann & Not Covered &
\begin{tabular}{c}
LVV-T715 \\
\end{tabular}
\\
\hline
\end{longtable}

\textbf{Verification Element Description:} \\
Undefined

{\footnotesize
\begin{longtable}{p{2.5cm}p{13.5cm}}
\hline
\multicolumn{2}{c}{\textbf{Requirement Details}}\\ \hline
Requirement ID & DMS-PRTL-REQ-0081 \\ \cdashline{1-2}
Requirement Description &
\begin{minipage}[]{13cm}
The Portal aspect shall enable a user to select individual HEALPix
pixels or groups of pixels and obtain references to them which can be
used in other LSP aspects.
\end{minipage}
\\ \cdashline{1-2}
Requirement Discussion &
\begin{minipage}[]{13cm}
UI selection of pixels, at selectable scales, can be done in the Portal,
with the selections then used in other aspects.
\end{minipage}
\\ \cdashline{1-2}
Requirement Priority &  \\ \cdashline{1-2}
Upper Level Requirement &
\begin{tabular}{cl}
\end{tabular}
\\ \hline
\end{longtable}
}


\subsubsection{Test Cases Summary}
\begin{longtable}{p{3cm}p{2.5cm}p{2.5cm}p{3cm}p{4cm}}
\toprule
\href{https://jira.lsstcorp.org/secure/Tests.jspa\#/testCase/LVV-T715}{LVV-T715} & \multicolumn{4}{p{12cm}}{ Verify selection of HEALPix pixels } \\ \hline
\textbf{Owner} & \textbf{Status} & \textbf{Version} & \textbf{Critical Event} & \textbf{Verification Type} \\ \hline
Jeffrey Carlin & Draft & 1 & false & Inspection \\ \hline
\end{longtable}
{\scriptsize
\textbf{Objective:}\\
Verify that Portal users can select individual HEALPix pixels or groups
of pixels and obtain references from them for use in other LSP aspects.
}
  
 \newpage 
\subsection{[LVV-9920] DMS-PRTL-REQ-0080-V-01: Pan Around on a HEALPix Image\_1 }\label{lvv-9920}

\begin{longtable}{cccc}
\hline
\textbf{Jira Link} & \textbf{Assignee} & \textbf{Status} & \textbf{Test Cases}\\ \hline
\href{https://jira.lsstcorp.org/browse/LVV-9920}{LVV-9920} &
Gregory Dubois-Felsmann & Not Covered &
\begin{tabular}{c}
LVV-T714 \\
\end{tabular}
\\
\hline
\end{longtable}

\textbf{Verification Element Description:} \\
Undefined

{\footnotesize
\begin{longtable}{p{2.5cm}p{13.5cm}}
\hline
\multicolumn{2}{c}{\textbf{Requirement Details}}\\ \hline
Requirement ID & DMS-PRTL-REQ-0080 \\ \cdashline{1-2}
Requirement Description &
\begin{minipage}[]{13cm}
The Portal aspect shall enable a user to move around within a HEALPix
all-sky image when the full image is not displayed on the screen.
\end{minipage}
\\ \cdashline{1-2}
Requirement Discussion &
\begin{minipage}[]{13cm}
The panning is intended to enable a user to move around on the sky with
a fixed zoom level. Panning does not apply if the full all-sky image is
visible on the screen.
\end{minipage}
\\ \cdashline{1-2}
Requirement Priority &  \\ \cdashline{1-2}
Upper Level Requirement &
\begin{tabular}{cl}
\end{tabular}
\\ \hline
\end{longtable}
}


\subsubsection{Test Cases Summary}
\begin{longtable}{p{3cm}p{2.5cm}p{2.5cm}p{3cm}p{4cm}}
\toprule
\href{https://jira.lsstcorp.org/secure/Tests.jspa\#/testCase/LVV-T714}{LVV-T714} & \multicolumn{4}{p{12cm}}{ Verify panning in HEALPix image display } \\ \hline
\textbf{Owner} & \textbf{Status} & \textbf{Version} & \textbf{Critical Event} & \textbf{Verification Type} \\ \hline
Jeffrey Carlin & Draft & 1 & false & Inspection \\ \hline
\end{longtable}
{\scriptsize
\textbf{Objective:}\\
Verify that the Portal enables panning (i.e., moving around within) a
displayed HEALPix image, provided that the entire image is not already
displayed.
}
  
 \newpage 
\subsection{[LVV-9921] DMS-PRTL-REQ-0082-V-01: Retrieve HEALPix-Associated Data\_1 }\label{lvv-9921}

\begin{longtable}{cccc}
\hline
\textbf{Jira Link} & \textbf{Assignee} & \textbf{Status} & \textbf{Test Cases}\\ \hline
\href{https://jira.lsstcorp.org/browse/LVV-9921}{LVV-9921} &
Gregory Dubois-Felsmann & Not Covered &
\begin{tabular}{c}
LVV-T716 \\
\end{tabular}
\\
\hline
\end{longtable}

\textbf{Verification Element Description:} \\
Undefined

{\footnotesize
\begin{longtable}{p{2.5cm}p{13.5cm}}
\hline
\multicolumn{2}{c}{\textbf{Requirement Details}}\\ \hline
Requirement ID & DMS-PRTL-REQ-0082 \\ \cdashline{1-2}
Requirement Description &
\begin{minipage}[]{13cm}
The Portal aspect shall enable a user to retrieve metadata and data
associated with selected HEALPixels and display that information as
tabular or image data as appropriate.
\end{minipage}
\\ \cdashline{1-2}
Requirement Discussion &
\begin{minipage}[]{13cm}
The HEALPix pixels will be associated with metadata (e.g., which objects
are associated with that position on the sky) or data (e.g., what is the
FWHM of all of the sources within that pixel on the sky). That
metadata/data will be retrievable via a selection of the HEALPix pixels.
\end{minipage}
\\ \cdashline{1-2}
Requirement Priority &  \\ \cdashline{1-2}
Upper Level Requirement &
\begin{tabular}{cl}
\end{tabular}
\\ \hline
\end{longtable}
}


\subsubsection{Test Cases Summary}
\begin{longtable}{p{3cm}p{2.5cm}p{2.5cm}p{3cm}p{4cm}}
\toprule
\href{https://jira.lsstcorp.org/secure/Tests.jspa\#/testCase/LVV-T716}{LVV-T716} & \multicolumn{4}{p{12cm}}{ Verify retrieval of HEALPix-associated data } \\ \hline
\textbf{Owner} & \textbf{Status} & \textbf{Version} & \textbf{Critical Event} & \textbf{Verification Type} \\ \hline
Jeffrey Carlin & Draft & 1 & false & Inspection \\ \hline
\end{longtable}
{\scriptsize
\textbf{Objective:}\\
Verify that the Portal enables users to retrieve metadata and data
associated with selected HEALPixels and display that data in tabular or
image form as appropriate.
}
  
 \newpage 
\subsection{[LVV-9922] DMS-PRTL-REQ-0079-V-01: Zoom In and Out on a HEALPix Image\_1 }\label{lvv-9922}

\begin{longtable}{cccc}
\hline
\textbf{Jira Link} & \textbf{Assignee} & \textbf{Status} & \textbf{Test Cases}\\ \hline
\href{https://jira.lsstcorp.org/browse/LVV-9922}{LVV-9922} &
Gregory Dubois-Felsmann & Not Covered &
\begin{tabular}{c}
LVV-T713 \\
\end{tabular}
\\
\hline
\end{longtable}

\textbf{Verification Element Description:} \\
Undefined

{\footnotesize
\begin{longtable}{p{2.5cm}p{13.5cm}}
\hline
\multicolumn{2}{c}{\textbf{Requirement Details}}\\ \hline
Requirement ID & DMS-PRTL-REQ-0079 \\ \cdashline{1-2}
Requirement Description &
\begin{minipage}[]{13cm}
The Portal aspect shall enable a user to zoom in and out on a HEALPix
all-sky image.
\end{minipage}
\\ \cdashline{1-2}
Requirement Discussion &
\begin{minipage}[]{13cm}
The zooming in and out will enable a user to change effective spatial
resolution of the image on the screen, tranferring across levels of the
image hierarchy.
\end{minipage}
\\ \cdashline{1-2}
Requirement Priority &  \\ \cdashline{1-2}
Upper Level Requirement &
\begin{tabular}{cl}
\end{tabular}
\\ \hline
\end{longtable}
}


\subsubsection{Test Cases Summary}
\begin{longtable}{p{3cm}p{2.5cm}p{2.5cm}p{3cm}p{4cm}}
\toprule
\href{https://jira.lsstcorp.org/secure/Tests.jspa\#/testCase/LVV-T713}{LVV-T713} & \multicolumn{4}{p{12cm}}{ Verify ability to zoom in/out on a HEALPix image } \\ \hline
\textbf{Owner} & \textbf{Status} & \textbf{Version} & \textbf{Critical Event} & \textbf{Verification Type} \\ \hline
Jeffrey Carlin & Draft & 1 & false & Inspection \\ \hline
\end{longtable}
{\scriptsize
\textbf{Objective:}\\
Verify that the Portal enables users to zoom in and out on a displayed
HEALPix image, adapting the displayed spatial scale and traversing
different levels of the image hierarchy.
}
  
 \newpage 
\subsection{[LVV-9923] DMS-PRTL-REQ-0087-V-01: Astrophysical Compass Overlay\_1 }\label{lvv-9923}

\begin{longtable}{cccc}
\hline
\textbf{Jira Link} & \textbf{Assignee} & \textbf{Status} & \textbf{Test Cases}\\ \hline
\href{https://jira.lsstcorp.org/browse/LVV-9923}{LVV-9923} &
Gregory Dubois-Felsmann & Not Covered &
\begin{tabular}{c}
LVV-T721 \\
\end{tabular}
\\
\hline
\end{longtable}

\textbf{Verification Element Description:} \\
Undefined

{\footnotesize
\begin{longtable}{p{2.5cm}p{13.5cm}}
\hline
\multicolumn{2}{c}{\textbf{Requirement Details}}\\ \hline
Requirement ID & DMS-PRTL-REQ-0087 \\ \cdashline{1-2}
Requirement Description &
\begin{minipage}[]{13cm}
The Portal aspect shall have the capability to display a North-East
Compass on an image or two-dimensional plot with a known astrophysical
coordinate system.
\end{minipage}
\\ \cdashline{1-2}
Requirement Discussion &
\begin{minipage}[]{13cm}
So the user knows that the (particularly for the generally randomly
rotated single-epoch data) which directions are North and East.
\end{minipage}
\\ \cdashline{1-2}
Requirement Priority &  \\ \cdashline{1-2}
Upper Level Requirement &
\begin{tabular}{cl}
\end{tabular}
\\ \hline
\end{longtable}
}


\subsubsection{Test Cases Summary}
\begin{longtable}{p{3cm}p{2.5cm}p{2.5cm}p{3cm}p{4cm}}
\toprule
\href{https://jira.lsstcorp.org/secure/Tests.jspa\#/testCase/LVV-T721}{LVV-T721} & \multicolumn{4}{p{12cm}}{ Verify astrophysical compass overlay } \\ \hline
\textbf{Owner} & \textbf{Status} & \textbf{Version} & \textbf{Critical Event} & \textbf{Verification Type} \\ \hline
Jeffrey Carlin & Draft & 1 & false & Inspection \\ \hline
\end{longtable}
{\scriptsize
\textbf{Objective:}\\
Verify that the Portal provides the capability to overlay a North-East
compass atop images or 2-dimensional plots with known astrophysical
coordinate systems.
}
  
 \newpage 
\subsection{[LVV-9924] DMS-PRTL-REQ-0083-V-01: Coordinate Display Applicability\_1 }\label{lvv-9924}

\begin{longtable}{cccc}
\hline
\textbf{Jira Link} & \textbf{Assignee} & \textbf{Status} & \textbf{Test Cases}\\ \hline
\href{https://jira.lsstcorp.org/browse/LVV-9924}{LVV-9924} &
Gregory Dubois-Felsmann & Not Covered &
\begin{tabular}{c}
LVV-T717 \\
\end{tabular}
\\
\hline
\end{longtable}

\textbf{Verification Element Description:} \\
Undefined

{\footnotesize
\begin{longtable}{p{2.5cm}p{13.5cm}}
\hline
\multicolumn{2}{c}{\textbf{Requirement Details}}\\ \hline
Requirement ID & DMS-PRTL-REQ-0083 \\ \cdashline{1-2}
Requirement Description &
\begin{minipage}[]{13cm}
The Portal aspect shall have the capability to provide the set of
coordinate system display and measurement tools in this section for any
two-dimensional data display where both coordinates have a spatial
interpretation, except as further specified below.
\end{minipage}
\\ \cdashline{1-2}
Requirement Discussion &
\begin{minipage}[]{13cm}
The knowledge that coordinates in an x-y plot are spatial will in
general depend on the availability of suitable metadata to define the
coordinates.
\end{minipage}
\\ \cdashline{1-2}
Requirement Priority &  \\ \cdashline{1-2}
Upper Level Requirement &
\begin{tabular}{cl}
\end{tabular}
\\ \hline
\end{longtable}
}


\subsubsection{Test Cases Summary}
\begin{longtable}{p{3cm}p{2.5cm}p{2.5cm}p{3cm}p{4cm}}
\toprule
\href{https://jira.lsstcorp.org/secure/Tests.jspa\#/testCase/LVV-T717}{LVV-T717} & \multicolumn{4}{p{12cm}}{ Verify broad applicability of coordinate display } \\ \hline
\textbf{Owner} & \textbf{Status} & \textbf{Version} & \textbf{Critical Event} & \textbf{Verification Type} \\ \hline
Jeffrey Carlin & Draft & 1 & false & Inspection \\ \hline
\end{longtable}
{\scriptsize
\textbf{Objective:}\\
Verify that the Portal aspect provides the coordinate display and
measurement tools for all applicable two-dimensional data displays where
the two coordinates have a spatial interpretation.
}
  
 \newpage 
\subsection{[LVV-9925] DMS-PRTL-REQ-0086-V-01: Coordinate Grid Overlays\_1 }\label{lvv-9925}

\begin{longtable}{cccc}
\hline
\textbf{Jira Link} & \textbf{Assignee} & \textbf{Status} & \textbf{Test Cases}\\ \hline
\href{https://jira.lsstcorp.org/browse/LVV-9925}{LVV-9925} &
Gregory Dubois-Felsmann & Not Covered &
\begin{tabular}{c}
LVV-T720 \\
\end{tabular}
\\
\hline
\end{longtable}

\textbf{Verification Element Description:} \\
Undefined

{\footnotesize
\begin{longtable}{p{2.5cm}p{13.5cm}}
\hline
\multicolumn{2}{c}{\textbf{Requirement Details}}\\ \hline
Requirement ID & DMS-PRTL-REQ-0086 \\ \cdashline{1-2}
Requirement Description &
\begin{minipage}[]{13cm}
The Portal aspect shall have the capability to display one or more
coordinate grids on top of images or two-dimensional plots with a known
astrophysical coordinate system.
\end{minipage}
\\ \cdashline{1-2}
Requirement Discussion &
\begin{minipage}[]{13cm}
The point behind this requirement is to enable viewing of equatorial,
galactic, and ecliptic coordinates overlays at the same time.
\end{minipage}
\\ \cdashline{1-2}
Requirement Priority &  \\ \cdashline{1-2}
Upper Level Requirement &
\begin{tabular}{cl}
\end{tabular}
\\ \hline
\end{longtable}
}


\subsubsection{Test Cases Summary}
\begin{longtable}{p{3cm}p{2.5cm}p{2.5cm}p{3cm}p{4cm}}
\toprule
\href{https://jira.lsstcorp.org/secure/Tests.jspa\#/testCase/LVV-T720}{LVV-T720} & \multicolumn{4}{p{12cm}}{ Verify coordinate grid overlays } \\ \hline
\textbf{Owner} & \textbf{Status} & \textbf{Version} & \textbf{Critical Event} & \textbf{Verification Type} \\ \hline
Jeffrey Carlin & Draft & 1 & false & Inspection \\ \hline
\end{longtable}
{\scriptsize
\textbf{Objective:}\\
Verify that the Portal provides the capability to overlay one or more
coordinate grids atop images or 2-dimensional plots with known
coordinate systems. (For example, it should be possible to overlay
equatorial, Galactic, and ecliptic coordinate grids simultaneously.)
}
  
 \newpage 
\subsection{[LVV-9927] DMS-PRTL-REQ-0088-V-01: Geometric Figure Overlays\_1 }\label{lvv-9927}

\begin{longtable}{cccc}
\hline
\textbf{Jira Link} & \textbf{Assignee} & \textbf{Status} & \textbf{Test Cases}\\ \hline
\href{https://jira.lsstcorp.org/browse/LVV-9927}{LVV-9927} &
Gregory Dubois-Felsmann & Not Covered &
\begin{tabular}{c}
LVV-T722 \\
\end{tabular}
\\
\hline
\end{longtable}

\textbf{Verification Element Description:} \\
Undefined

{\footnotesize
\begin{longtable}{p{2.5cm}p{13.5cm}}
\hline
\multicolumn{2}{c}{\textbf{Requirement Details}}\\ \hline
Requirement ID & DMS-PRTL-REQ-0088 \\ \cdashline{1-2}
Requirement Description &
\begin{minipage}[]{13cm}
The Portal aspect shall enable the drawing, display, and selection of a
closed 2-dimensional polygon on any two-dimensional image.
\end{minipage}
\\ \cdashline{1-2}
Requirement Discussion &
\begin{minipage}[]{13cm}
This is a general requirement that enables the overlay of a polygon on
an image or a plot. A polygon could be a circle, an ellipse, or an
N-vertices polygon. The purpose of this is enable area selection on the
images or plots.
\end{minipage}
\\ \cdashline{1-2}
Requirement Priority &  \\ \cdashline{1-2}
Upper Level Requirement &
\begin{tabular}{cl}
\end{tabular}
\\ \hline
\end{longtable}
}


\subsubsection{Test Cases Summary}
\begin{longtable}{p{3cm}p{2.5cm}p{2.5cm}p{3cm}p{4cm}}
\toprule
\href{https://jira.lsstcorp.org/secure/Tests.jspa\#/testCase/LVV-T722}{LVV-T722} & \multicolumn{4}{p{12cm}}{ Verify geometric figure overlays } \\ \hline
\textbf{Owner} & \textbf{Status} & \textbf{Version} & \textbf{Critical Event} & \textbf{Verification Type} \\ \hline
Jeffrey Carlin & Draft & 1 & false & Inspection \\ \hline
\end{longtable}
{\scriptsize
\textbf{Objective:}\\
Verify that the Portal aspect enables the drawing, display, and
selection of a closed 2-dimensional polygon on any 2-dimensional image.
}
  
 \newpage 
\subsection{[LVV-9928] DMS-PRTL-REQ-0084-V-01: Point Coordinate Display\_1 }\label{lvv-9928}

\begin{longtable}{cccc}
\hline
\textbf{Jira Link} & \textbf{Assignee} & \textbf{Status} & \textbf{Test Cases}\\ \hline
\href{https://jira.lsstcorp.org/browse/LVV-9928}{LVV-9928} &
Gregory Dubois-Felsmann & Not Covered &
\begin{tabular}{c}
LVV-T718 \\
\end{tabular}
\\
\hline
\end{longtable}

\textbf{Verification Element Description:} \\
Undefined

{\footnotesize
\begin{longtable}{p{2.5cm}p{13.5cm}}
\hline
\multicolumn{2}{c}{\textbf{Requirement Details}}\\ \hline
Requirement ID & DMS-PRTL-REQ-0084 \\ \cdashline{1-2}
Requirement Description &
\begin{minipage}[]{13cm}
The Portal aspect shall have the capability to inspect the coordinates
in a two-dimensional display that correspond to the position of the
mouse cursor. When coordinate system conversion information is
available, display of all available coordinates shall be supported. In
particular, when available (e.g., for data associated with single-epoch
and calibration images) the coordinate display shall include focal plane
array electronic as well as spatial coordinates. When available (i.e.,
for an image, when a WCS is present) the coordinate display shall
include astrophysical coordinates.
\end{minipage}
\\ \cdashline{1-2}
Requirement Discussion &
\begin{minipage}[]{13cm}
When applied to images and at high zoom levels, the UI shall make clear
whether pixel center coordinates or continuous cursor-location
coordinates are being displayed.
\end{minipage}
\\ \cdashline{1-2}
Requirement Priority &  \\ \cdashline{1-2}
Upper Level Requirement &
\begin{tabular}{cl}
\end{tabular}
\\ \hline
\end{longtable}
}


\subsubsection{Test Cases Summary}
\begin{longtable}{p{3cm}p{2.5cm}p{2.5cm}p{3cm}p{4cm}}
\toprule
\href{https://jira.lsstcorp.org/secure/Tests.jspa\#/testCase/LVV-T718}{LVV-T718} & \multicolumn{4}{p{12cm}}{ Verify point coordinate display } \\ \hline
\textbf{Owner} & \textbf{Status} & \textbf{Version} & \textbf{Critical Event} & \textbf{Verification Type} \\ \hline
Jeffrey Carlin & Draft & 1 & false & Inspection \\ \hline
\end{longtable}
{\scriptsize
\textbf{Objective:}\\
Verify that the Portal aspect displays the coordinates corresponding to
the position of the mouse cursor. When coordinate conversion information
is available, all available coordinates should be displayed.
}
  
 \newpage 
\subsection{[LVV-9929] DMS-PRTL-REQ-0091-V-01: Calculated Filtering of Tabular Data\_1 }\label{lvv-9929}

\begin{longtable}{cccc}
\hline
\textbf{Jira Link} & \textbf{Assignee} & \textbf{Status} & \textbf{Test Cases}\\ \hline
\href{https://jira.lsstcorp.org/browse/LVV-9929}{LVV-9929} &
Gregory Dubois-Felsmann & Not Covered &
\begin{tabular}{c}
LVV-T725 \\
\end{tabular}
\\
\hline
\end{longtable}

\textbf{Verification Element Description:} \\
Undefined

{\footnotesize
\begin{longtable}{p{2.5cm}p{13.5cm}}
\hline
\multicolumn{2}{c}{\textbf{Requirement Details}}\\ \hline
Requirement ID & DMS-PRTL-REQ-0091 \\ \cdashline{1-2}
Requirement Description &
\begin{minipage}[]{13cm}
The Portal aspect shall provide the capability to filter a table by
single column where the filter has simple arithmetic calculations
applied to the column values, including but not limited to sqrt, log,
log10, exponentials and trigonometric functions.
\end{minipage}
\\ \cdashline{1-2}
Requirement Discussion &
\begin{minipage}[]{13cm}
The intent of this requirement is enable simple arithmetic functions on
the values in the columns prior and then do the filtering on that value
(e.g., X*0.33 \textless{} 10).
\end{minipage}
\\ \cdashline{1-2}
Requirement Priority &  \\ \cdashline{1-2}
Upper Level Requirement &
\begin{tabular}{cl}
\end{tabular}
\\ \hline
\end{longtable}
}


\subsubsection{Test Cases Summary}
\begin{longtable}{p{3cm}p{2.5cm}p{2.5cm}p{3cm}p{4cm}}
\toprule
\href{https://jira.lsstcorp.org/secure/Tests.jspa\#/testCase/LVV-T725}{LVV-T725} & \multicolumn{4}{p{12cm}}{ Verify calculated filtering of tabular data } \\ \hline
\textbf{Owner} & \textbf{Status} & \textbf{Version} & \textbf{Critical Event} & \textbf{Verification Type} \\ \hline
Jeffrey Carlin & Draft & 1 & false & Inspection \\ \hline
\end{longtable}
{\scriptsize
\textbf{Objective:}\\
Verify that the Portal aspect provides the capability to filter a table
by single column where the filter has simple arithmetic calculations
applied to the column values, including but not limited to sqrt, log,
log10, exponentials and trigonometric functions.
}
  
 \newpage 
\subsection{[LVV-9930] DMS-PRTL-REQ-0093-V-01: Calculated Quantities on Tabular Data\_1 }\label{lvv-9930}

\begin{longtable}{cccc}
\hline
\textbf{Jira Link} & \textbf{Assignee} & \textbf{Status} & \textbf{Test Cases}\\ \hline
\href{https://jira.lsstcorp.org/browse/LVV-9930}{LVV-9930} &
Gregory Dubois-Felsmann & Not Covered &
\begin{tabular}{c}
LVV-T727 \\
\end{tabular}
\\
\hline
\end{longtable}

\textbf{Verification Element Description:} \\
Undefined

{\footnotesize
\begin{longtable}{p{2.5cm}p{13.5cm}}
\hline
\multicolumn{2}{c}{\textbf{Requirement Details}}\\ \hline
Requirement ID & DMS-PRTL-REQ-0093 \\ \cdashline{1-2}
Requirement Description &
\begin{minipage}[]{13cm}
The Portal aspect shall provide the capability to add an additional
column to the displayed table based upon an arithmetic operations on
columns within the displayed table and display the new column.
\end{minipage}
\\ \cdashline{1-2}
Requirement Discussion &
\begin{minipage}[]{13cm}
This intent of this requirement is enable the user to combine columns in
the table to form a new column that can be included in the table and
used in the same manner as the intrinsic columns.
\end{minipage}
\\ \cdashline{1-2}
Requirement Priority &  \\ \cdashline{1-2}
Upper Level Requirement &
\begin{tabular}{cl}
\end{tabular}
\\ \hline
\end{longtable}
}


\subsubsection{Test Cases Summary}
\begin{longtable}{p{3cm}p{2.5cm}p{2.5cm}p{3cm}p{4cm}}
\toprule
\href{https://jira.lsstcorp.org/secure/Tests.jspa\#/testCase/LVV-T727}{LVV-T727} & \multicolumn{4}{p{12cm}}{ Verify calculated tabular data columns } \\ \hline
\textbf{Owner} & \textbf{Status} & \textbf{Version} & \textbf{Critical Event} & \textbf{Verification Type} \\ \hline
Jeffrey Carlin & Draft & 1 & false & Inspection \\ \hline
\end{longtable}
{\scriptsize
\textbf{Objective:}\\
Verify that the Portal enables the arithmetic calculation and display of
new tabular data columns based on existing columns in a table.~
}
  
 \newpage 
\subsection{[LVV-9931] DMS-PRTL-REQ-0092-V-01: Filtering of Tabular Data by Multiple Columns\_1 }\label{lvv-9931}

\begin{longtable}{cccc}
\hline
\textbf{Jira Link} & \textbf{Assignee} & \textbf{Status} & \textbf{Test Cases}\\ \hline
\href{https://jira.lsstcorp.org/browse/LVV-9931}{LVV-9931} &
Gregory Dubois-Felsmann & Not Covered &
\begin{tabular}{c}
LVV-T726 \\
\end{tabular}
\\
\hline
\end{longtable}

\textbf{Verification Element Description:} \\
Undefined

{\footnotesize
\begin{longtable}{p{2.5cm}p{13.5cm}}
\hline
\multicolumn{2}{c}{\textbf{Requirement Details}}\\ \hline
Requirement ID & DMS-PRTL-REQ-0092 \\ \cdashline{1-2}
Requirement Description &
\begin{minipage}[]{13cm}
The Portal aspect shall provide the capability to filter tabular data by
multiple columns within the table and redisplay the filtered table.
\end{minipage}
\\ \cdashline{1-2}
Requirement Discussion &
\begin{minipage}[]{13cm}
Apply filters from various columns.
\end{minipage}
\\ \cdashline{1-2}
Requirement Priority &  \\ \cdashline{1-2}
Upper Level Requirement &
\begin{tabular}{cl}
\end{tabular}
\\ \hline
\end{longtable}
}


\subsubsection{Test Cases Summary}
\begin{longtable}{p{3cm}p{2.5cm}p{2.5cm}p{3cm}p{4cm}}
\toprule
\href{https://jira.lsstcorp.org/secure/Tests.jspa\#/testCase/LVV-T726}{LVV-T726} & \multicolumn{4}{p{12cm}}{ Verify filtering data by multiple table columns } \\ \hline
\textbf{Owner} & \textbf{Status} & \textbf{Version} & \textbf{Critical Event} & \textbf{Verification Type} \\ \hline
Jeffrey Carlin & Draft & 1 & false & Inspection \\ \hline
\end{longtable}
{\scriptsize
\textbf{Objective:}\\
Verify that the Portal aspect provides the capability to filter tabular
data by multiple columns within the table and redisplay the filtered
table.
}
  
 \newpage 
\subsection{[LVV-9932] DMS-PRTL-REQ-0095-V-01: Saving Displayed Tabular Data\_1 }\label{lvv-9932}

\begin{longtable}{cccc}
\hline
\textbf{Jira Link} & \textbf{Assignee} & \textbf{Status} & \textbf{Test Cases}\\ \hline
\href{https://jira.lsstcorp.org/browse/LVV-9932}{LVV-9932} &
Gregory Dubois-Felsmann & Not Covered &
\begin{tabular}{c}
LVV-T729 \\
LVV-T1334 \\
LVV-T1818 \\
\end{tabular}
\\
\hline
\end{longtable}

\textbf{Verification Element Description:} \\
Undefined

{\footnotesize
\begin{longtable}{p{2.5cm}p{13.5cm}}
\hline
\multicolumn{2}{c}{\textbf{Requirement Details}}\\ \hline
Requirement ID & DMS-PRTL-REQ-0095 \\ \cdashline{1-2}
Requirement Description &
\begin{minipage}[]{13cm}
The Portal aspect shall provide the capability to save and or download
tabular data as it is displayed in the interface maintaining the
content, filtering, and sorting.
\end{minipage}
\\ \cdashline{1-2}
Requirement Discussion &
\begin{minipage}[]{13cm}
This intent of this requirement is enable the user to manipulate the
table (e.g., sorting, filtering, calculated quantities and save that
table to either workspace or in an offline download).
\end{minipage}
\\ \cdashline{1-2}
Requirement Priority &  \\ \cdashline{1-2}
Upper Level Requirement &
\begin{tabular}{cl}
\end{tabular}
\\ \hline
\end{longtable}
}


\subsubsection{Test Cases Summary}
\begin{longtable}{p{3cm}p{2.5cm}p{2.5cm}p{3cm}p{4cm}}
\toprule
\href{https://jira.lsstcorp.org/secure/Tests.jspa\#/testCase/LVV-T729}{LVV-T729} & \multicolumn{4}{p{12cm}}{ Verify saving of displayed tabular data } \\ \hline
\textbf{Owner} & \textbf{Status} & \textbf{Version} & \textbf{Critical Event} & \textbf{Verification Type} \\ \hline
Jeffrey Carlin & Draft & 1 & false & Inspection \\ \hline
\end{longtable}
{\scriptsize
\textbf{Objective:}\\
Verify that the Portal aspect provides the capability to save and or
download tabular data as it is displayed in the interface maintaining
the content, filtering, and sorting.
}
\begin{longtable}{p{3cm}p{2.5cm}p{2.5cm}p{3cm}p{4cm}}
\toprule
\href{https://jira.lsstcorp.org/secure/Tests.jspa\#/testCase/LVV-T1334}{LVV-T1334} & \multicolumn{4}{p{12cm}}{ LDM-503-10a: Portal Aspect tests for LSP with Authentication and TAP
milestone } \\ \hline
\textbf{Owner} & \textbf{Status} & \textbf{Version} & \textbf{Critical Event} & \textbf{Verification Type} \\ \hline
Gregory Dubois-Felsmann & Defined & 1 & false & Test \\ \hline
\end{longtable}
{\scriptsize
\textbf{Objective:}\\
This test case verifies that the Portal Aspect of the Science Platform
is accessible to authorized users through a login process, and that TAP
searches can be performed from the Portal Aspect UI.\\[2\baselineskip]In
so doing and in conjunction with the other LDM-503-10a test cases
collected under LVV-P48, it addresses all or part of the following
requirements:

\begin{itemize}
\tightlist
\item
  DMS-LSP-REQ-0002, DMS-LSP-REQ-0005, DMS-LSP-REQ-0006,
  DMS-LSP-REQ-0020, DMS-LSP-REQ-0022, DMS-LSP-REQ-0023, DMS-LSP-REQ-0024
\item
  DMS-PRTL-REQ-0001, DMS-PRTL-REQ-0015, DMS-PRTL-REQ-0016,
  DMS-PRTL-REQ-0017, DMS-PRTL-REQ-0020, DMS-PRTL-REQ-0026,
  DMS-PRTL-REQ-0049, and DMS-PRTL-REQ-0095, primarily
\end{itemize}

Note this test was not designed to perform a full verification of the
above requirements, but rather to demonstrate having reached a certain
level of partial capability during construction.
}
\begin{longtable}{p{3cm}p{2.5cm}p{2.5cm}p{3cm}p{4cm}}
\toprule
\href{https://jira.lsstcorp.org/secure/Tests.jspa\#/testCase/LVV-T1818}{LVV-T1818} & \multicolumn{4}{p{12cm}}{ DM-SUIT-8: Verify Portal integration with workspace (via WebDAV) } \\ \hline
\textbf{Owner} & \textbf{Status} & \textbf{Version} & \textbf{Critical Event} & \textbf{Verification Type} \\ \hline
Gregory Dubois-Felsmann & Defined & 1 & false & Demonstration \\ \hline
\end{longtable}
{\scriptsize
\textbf{Objective:}\\
This test case verifies that the Portal Aspect software is capable of
accessing a file-oriented workspace via the WebDAV
protocol.\\[2\baselineskip]In so doing, it partially verifies several
Portal Aspect requirements that relate to this capability -
``partially'' because some of these requirements depend on workspace
capabilities which were not present in the prototype WebDAV service
delivered by the DAX group, because some of the requirements also cover
the User Database Workspace (not relevant to this milestone, and not yet
available), and also because the milestone was not envisioned as an
exhaustive test covering edge cases:

\begin{itemize}
\tightlist
\item
  DMS-PRTL-REQ-0003 (LVV-9846, Portal access to workspace) is covered at
  ``demonstration'' level, with basic tests of saving image and tabular
  data to the workspace, and only for the User File Workspace ~(there is
  currently no User Database Workspace prototype available);
\item
  DMS-PRTL-REQ-0046 (LVV-9886, Visualization of workspace data) is
  covered at ``demonstration'' level for a couple of FITS image and
  table files, and only for the User File Workspace;
\item
  DMS-PRTL-REQ-0110 (LVV-9954, Tabular data download) is covered at
  ``demonstration'' level, only for catalog data (there was no image
  metadata in the LSP deployment at the time of test), and only for the
  User File Workspace;~
\item
  DMS-PRTL-REQ-0095 (LVV-9932, Saving Displayed Tabular Data) is covered
  at ``demonstration'' level for a simple subset operation in the table
  browser; and
\item
  DMS-PRTL-REQ-0111 (LVV-9951, Image data download) is covered at
  ``demonstration'' level, and only for download from an image display
  screen itself (as LSST-style image metadata services, e.g., ObsTAP,
  were not available in the LSP at the time of testing).
\end{itemize}
}
  
 \newpage 
\subsection{[LVV-9933] DMS-PRTL-REQ-0090-V-01: Simple Filtering of Tabular Data\_1 }\label{lvv-9933}

\begin{longtable}{cccc}
\hline
\textbf{Jira Link} & \textbf{Assignee} & \textbf{Status} & \textbf{Test Cases}\\ \hline
\href{https://jira.lsstcorp.org/browse/LVV-9933}{LVV-9933} &
Gregory Dubois-Felsmann & Not Covered &
\begin{tabular}{c}
LVV-T724 \\
\end{tabular}
\\
\hline
\end{longtable}

\textbf{Verification Element Description:} \\
Undefined

{\footnotesize
\begin{longtable}{p{2.5cm}p{13.5cm}}
\hline
\multicolumn{2}{c}{\textbf{Requirement Details}}\\ \hline
Requirement ID & DMS-PRTL-REQ-0090 \\ \cdashline{1-2}
Requirement Description &
\begin{minipage}[]{13cm}
The Portal aspect shall provide the capability to filter tabular data by
a single column, including but not limited to less than (), greater than
or equal (=\textgreater{}), equal (=), not equal ({Unable to render
embedded object: File (=) and not null () not found.}=null).
\end{minipage}
\\ \cdashline{1-2}
Requirement Discussion &
\begin{minipage}[]{13cm}
The intent of this requirement is enable simple one-dimensional
filtering on a single column or a series of columns and-ed together.
\end{minipage}
\\ \cdashline{1-2}
Requirement Priority &  \\ \cdashline{1-2}
Upper Level Requirement &
\begin{tabular}{cl}
\end{tabular}
\\ \hline
\end{longtable}
}


\subsubsection{Test Cases Summary}
\begin{longtable}{p{3cm}p{2.5cm}p{2.5cm}p{3cm}p{4cm}}
\toprule
\href{https://jira.lsstcorp.org/secure/Tests.jspa\#/testCase/LVV-T724}{LVV-T724} & \multicolumn{4}{p{12cm}}{ Verify simple filtering of tabular data } \\ \hline
\textbf{Owner} & \textbf{Status} & \textbf{Version} & \textbf{Critical Event} & \textbf{Verification Type} \\ \hline
Jeffrey Carlin & Draft & 1 & false & Inspection \\ \hline
\end{longtable}
{\scriptsize
\textbf{Objective:}\\
Verify that the Portal aspect provides the capability to filter tabular
data by a single column, including but not limited to less than
(\textless{}), less than or equal (\textless{}=), greater than
(\textgreater{}), greater than or equal (=\textgreater{}), equal (=),
not equal (!=) and not null (!=null).
}
  
 \newpage 
\subsection{[LVV-9934] DMS-PRTL-REQ-0089-V-01: Sorting of Tabular Data by Column\_1 }\label{lvv-9934}

\begin{longtable}{cccc}
\hline
\textbf{Jira Link} & \textbf{Assignee} & \textbf{Status} & \textbf{Test Cases}\\ \hline
\href{https://jira.lsstcorp.org/browse/LVV-9934}{LVV-9934} &
Gregory Dubois-Felsmann & Not Covered &
\begin{tabular}{c}
LVV-T723 \\
\end{tabular}
\\
\hline
\end{longtable}

\textbf{Verification Element Description:} \\
Undefined

{\footnotesize
\begin{longtable}{p{2.5cm}p{13.5cm}}
\hline
\multicolumn{2}{c}{\textbf{Requirement Details}}\\ \hline
Requirement ID & DMS-PRTL-REQ-0089 \\ \cdashline{1-2}
Requirement Description &
\begin{minipage}[]{13cm}
The Portal aspect shall provide the capability to sort tabular data by a
single column within the table and redisplay the sorted table.
\end{minipage}
\\ \cdashline{1-2}
Requirement Discussion &
\begin{minipage}[]{13cm}
This is the traditional sorting by one column.
\end{minipage}
\\ \cdashline{1-2}
Requirement Priority &  \\ \cdashline{1-2}
Upper Level Requirement &
\begin{tabular}{cl}
\end{tabular}
\\ \hline
\end{longtable}
}


\subsubsection{Test Cases Summary}
\begin{longtable}{p{3cm}p{2.5cm}p{2.5cm}p{3cm}p{4cm}}
\toprule
\href{https://jira.lsstcorp.org/secure/Tests.jspa\#/testCase/LVV-T723}{LVV-T723} & \multicolumn{4}{p{12cm}}{ Verify sorting of tabular data by column } \\ \hline
\textbf{Owner} & \textbf{Status} & \textbf{Version} & \textbf{Critical Event} & \textbf{Verification Type} \\ \hline
Jeffrey Carlin & Draft & 1 & false & Inspection \\ \hline
\end{longtable}
{\scriptsize
\textbf{Objective:}\\
Verify that the Portal aspect enables users to sort tabular data by a
single column within the table and redisplay the sorted data.
}
  
 \newpage 
\subsection{[LVV-9935] DMS-PRTL-REQ-0094-V-01: Statistical Measurements on Tabular Data\_1 }\label{lvv-9935}

\begin{longtable}{cccc}
\hline
\textbf{Jira Link} & \textbf{Assignee} & \textbf{Status} & \textbf{Test Cases}\\ \hline
\href{https://jira.lsstcorp.org/browse/LVV-9935}{LVV-9935} &
Gregory Dubois-Felsmann & Not Covered &
\begin{tabular}{c}
LVV-T728 \\
\end{tabular}
\\
\hline
\end{longtable}

\textbf{Verification Element Description:} \\
Undefined

{\footnotesize
\begin{longtable}{p{2.5cm}p{13.5cm}}
\hline
\multicolumn{2}{c}{\textbf{Requirement Details}}\\ \hline
Requirement ID & DMS-PRTL-REQ-0094 \\ \cdashline{1-2}
Requirement Description &
\begin{minipage}[]{13cm}
The Portal aspect shall enable the capability to perform a set of
statistical measurements (e.g., mean, median, RMS, skew, kurtosis) on
tabular data selected by the user.
\end{minipage}
\\ \cdashline{1-2}
Requirement Priority &  \\ \cdashline{1-2}
Upper Level Requirement &
\begin{tabular}{cl}
\end{tabular}
\\ \hline
\end{longtable}
}


\subsubsection{Test Cases Summary}
\begin{longtable}{p{3cm}p{2.5cm}p{2.5cm}p{3cm}p{4cm}}
\toprule
\href{https://jira.lsstcorp.org/secure/Tests.jspa\#/testCase/LVV-T728}{LVV-T728} & \multicolumn{4}{p{12cm}}{ Verify statistical measurements on tabular data } \\ \hline
\textbf{Owner} & \textbf{Status} & \textbf{Version} & \textbf{Critical Event} & \textbf{Verification Type} \\ \hline
Jeffrey Carlin & Draft & 1 & false & Inspection \\ \hline
\end{longtable}
{\scriptsize
\textbf{Objective:}\\
Verify that the Portal aspect enables the capability to perform a set of
statistical measurements (e.g., mean, median, RMS, skew, kurtosis) on
tabular data selected by the user.
}
  
 \newpage 
\subsection{[LVV-9936] DMS-PRTL-REQ-0096-V-01: False-color Images Creation and Display\_1 }\label{lvv-9936}

\begin{longtable}{cccc}
\hline
\textbf{Jira Link} & \textbf{Assignee} & \textbf{Status} & \textbf{Test Cases}\\ \hline
\href{https://jira.lsstcorp.org/browse/LVV-9936}{LVV-9936} &
Gregory Dubois-Felsmann & Not Covered &
\begin{tabular}{c}
LVV-T730 \\
\end{tabular}
\\
\hline
\end{longtable}

\textbf{Verification Element Description:} \\
Undefined

{\footnotesize
\begin{longtable}{p{2.5cm}p{13.5cm}}
\hline
\multicolumn{2}{c}{\textbf{Requirement Details}}\\ \hline
Requirement ID & DMS-PRTL-REQ-0096 \\ \cdashline{1-2}
Requirement Description &
\begin{minipage}[]{13cm}
The Portal aspect shall have the capability to create and display
false-color images composed from any user-selectable set of filters from
multiple filter views of the same region.
\end{minipage}
\\ \cdashline{1-2}
Requirement Priority &  \\ \cdashline{1-2}
Upper Level Requirement &
\begin{tabular}{cl}
\end{tabular}
\\ \hline
\end{longtable}
}


\subsubsection{Test Cases Summary}
\begin{longtable}{p{3cm}p{2.5cm}p{2.5cm}p{3cm}p{4cm}}
\toprule
\href{https://jira.lsstcorp.org/secure/Tests.jspa\#/testCase/LVV-T730}{LVV-T730} & \multicolumn{4}{p{12cm}}{ Verify creation and display of false-color images } \\ \hline
\textbf{Owner} & \textbf{Status} & \textbf{Version} & \textbf{Critical Event} & \textbf{Verification Type} \\ \hline
Jeffrey Carlin & Draft & 1 & false & Inspection \\ \hline
\end{longtable}
{\scriptsize
\textbf{Objective:}\\
Verify that the Portal aspect has the capability to create and display
false-color images composed from any user-selectable set of filters from
multiple filter views of the same region.
}
  
 \newpage 
\subsection{[LVV-9938] DMS-PRTL-REQ-0105-V-01: Brightness Light Curves\_1 }\label{lvv-9938}

\begin{longtable}{cccc}
\hline
\textbf{Jira Link} & \textbf{Assignee} & \textbf{Status} & \textbf{Test Cases}\\ \hline
\href{https://jira.lsstcorp.org/browse/LVV-9938}{LVV-9938} &
Gregory Dubois-Felsmann & Not Covered &
\begin{tabular}{c}
LVV-T739 \\
\end{tabular}
\\
\hline
\end{longtable}

\textbf{Verification Element Description:} \\
Undefined

{\footnotesize
\begin{longtable}{p{2.5cm}p{13.5cm}}
\hline
\multicolumn{2}{c}{\textbf{Requirement Details}}\\ \hline
Requirement ID & DMS-PRTL-REQ-0105 \\ \cdashline{1-2}
Requirement Description &
\begin{minipage}[]{13cm}
The Portal aspect shall have the capability to display graphically the
brightness/magnitude of an LSST Object or Source or Forced Source as a
function of time.
\end{minipage}
\\ \cdashline{1-2}
Requirement Discussion &
\begin{minipage}[]{13cm}
This is a specific implementation of the xy-plot capabilities
\end{minipage}
\\ \cdashline{1-2}
Requirement Priority &  \\ \cdashline{1-2}
Upper Level Requirement &
\begin{tabular}{cl}
\end{tabular}
\\ \hline
\end{longtable}
}


\subsubsection{Test Cases Summary}
\begin{longtable}{p{3cm}p{2.5cm}p{2.5cm}p{3cm}p{4cm}}
\toprule
\href{https://jira.lsstcorp.org/secure/Tests.jspa\#/testCase/LVV-T739}{LVV-T739} & \multicolumn{4}{p{12cm}}{ Verify display of light curves } \\ \hline
\textbf{Owner} & \textbf{Status} & \textbf{Version} & \textbf{Critical Event} & \textbf{Verification Type} \\ \hline
Jeffrey Carlin & Draft & 1 & false & Inspection \\ \hline
\end{longtable}
{\scriptsize
\textbf{Objective:}\\
Verify that the Portal can display graphically the
brightness/flux/magnitude of an LSST Object, Source, or ForcedSource as
a function of time.
}
  
 \newpage 
\subsection{[LVV-9939] DMS-PRTL-REQ-0107-V-01: Data Selection from a Plot or Image\_1 }\label{lvv-9939}

\begin{longtable}{cccc}
\hline
\textbf{Jira Link} & \textbf{Assignee} & \textbf{Status} & \textbf{Test Cases}\\ \hline
\href{https://jira.lsstcorp.org/browse/LVV-9939}{LVV-9939} &
Gregory Dubois-Felsmann & Not Covered &
\begin{tabular}{c}
LVV-T741 \\
\end{tabular}
\\
\hline
\end{longtable}

\textbf{Verification Element Description:} \\
Undefined

{\footnotesize
\begin{longtable}{p{2.5cm}p{13.5cm}}
\hline
\multicolumn{2}{c}{\textbf{Requirement Details}}\\ \hline
Requirement ID & DMS-PRTL-REQ-0107 \\ \cdashline{1-2}
Requirement Description &
\begin{minipage}[]{13cm}
The Portal aspect shall enable the selection of data contained inside or
outside a closed 2-dimensional polygon on an xy-plot, 2-dimensional data
structure (e.g., density plot), and a 2-dimensional image.
\end{minipage}
\\ \cdashline{1-2}
Requirement Discussion &
\begin{minipage}[]{13cm}
This is a general requirement that enables the selection of data from
inside or outside a polygon.
\end{minipage}
\\ \cdashline{1-2}
Requirement Priority &  \\ \cdashline{1-2}
Upper Level Requirement &
\begin{tabular}{cl}
\end{tabular}
\\ \hline
\end{longtable}
}


\subsubsection{Test Cases Summary}
\begin{longtable}{p{3cm}p{2.5cm}p{2.5cm}p{3cm}p{4cm}}
\toprule
\href{https://jira.lsstcorp.org/secure/Tests.jspa\#/testCase/LVV-T741}{LVV-T741} & \multicolumn{4}{p{12cm}}{ Verify capability to select data from a plot or image } \\ \hline
\textbf{Owner} & \textbf{Status} & \textbf{Version} & \textbf{Critical Event} & \textbf{Verification Type} \\ \hline
Jeffrey Carlin & Draft & 1 & false & Inspection \\ \hline
\end{longtable}
{\scriptsize
\textbf{Objective:}\\
Verify that the Portal aspect enables the selection of data contained
inside or outside a closed 2-dimensional polygon on an xy-plot,
2-dimension data structure (e.g., density plot), and a 2-dimensional
image.
}
  
 \newpage 
\subsection{[LVV-9940] DMS-PRTL-REQ-0102-V-01: Display of Camera Artifacts as Overlays\_1 }\label{lvv-9940}

\begin{longtable}{cccc}
\hline
\textbf{Jira Link} & \textbf{Assignee} & \textbf{Status} & \textbf{Test Cases}\\ \hline
\href{https://jira.lsstcorp.org/browse/LVV-9940}{LVV-9940} &
Gregory Dubois-Felsmann & Not Covered &
\begin{tabular}{c}
LVV-T736 \\
\end{tabular}
\\
\hline
\end{longtable}

\textbf{Verification Element Description:} \\
Undefined

{\footnotesize
\begin{longtable}{p{2.5cm}p{13.5cm}}
\hline
\multicolumn{2}{c}{\textbf{Requirement Details}}\\ \hline
Requirement ID & DMS-PRTL-REQ-0102 \\ \cdashline{1-2}
Requirement Description &
\begin{minipage}[]{13cm}
The Portal aspect shall have the capability to display a camera
artifacts including but not limited to image crosstalk matrices, ghost
image identifications, saturation, and column bleeding.
\end{minipage}
\\ \cdashline{1-2}
Requirement Discussion &
\begin{minipage}[]{13cm}
The intent of this requirement is to enable the users to be able to see
where artifacts may be affecting the data. These artifacts may not be
stored in image format and may need to be reconstructed algorithmically.
\end{minipage}
\\ \cdashline{1-2}
Requirement Priority &  \\ \cdashline{1-2}
Upper Level Requirement &
\begin{tabular}{cl}
\end{tabular}
\\ \hline
\end{longtable}
}


\subsubsection{Test Cases Summary}
\begin{longtable}{p{3cm}p{2.5cm}p{2.5cm}p{3cm}p{4cm}}
\toprule
\href{https://jira.lsstcorp.org/secure/Tests.jspa\#/testCase/LVV-T736}{LVV-T736} & \multicolumn{4}{p{12cm}}{ Verify overlay of camera artifacts on images } \\ \hline
\textbf{Owner} & \textbf{Status} & \textbf{Version} & \textbf{Critical Event} & \textbf{Verification Type} \\ \hline
Jeffrey Carlin & Draft & 1 & false & Inspection \\ \hline
\end{longtable}
{\scriptsize
\textbf{Objective:}\\
Verify that the Portal aspect has the capability to display as image
overlays camera artifacts including but not limited to image crosstalk
matrices, ghost image identifications, saturation, and column bleeding.
}
  
 \newpage 
\subsection{[LVV-9941] DMS-PRTL-REQ-0106-V-01: Linked Tables, Plots, and Images\_1 }\label{lvv-9941}

\begin{longtable}{cccc}
\hline
\textbf{Jira Link} & \textbf{Assignee} & \textbf{Status} & \textbf{Test Cases}\\ \hline
\href{https://jira.lsstcorp.org/browse/LVV-9941}{LVV-9941} &
Gregory Dubois-Felsmann & Not Covered &
\begin{tabular}{c}
LVV-T740 \\
\end{tabular}
\\
\hline
\end{longtable}

\textbf{Verification Element Description:} \\
Undefined

{\footnotesize
\begin{longtable}{p{2.5cm}p{13.5cm}}
\hline
\multicolumn{2}{c}{\textbf{Requirement Details}}\\ \hline
Requirement ID & DMS-PRTL-REQ-0106 \\ \cdashline{1-2}
Requirement Description &
\begin{minipage}[]{13cm}
The Portal aspect shall have the capability to have tabular data, plots,
and images with overlays connected via brushing and linking.
\end{minipage}
\\ \cdashline{1-2}
Requirement Discussion &
\begin{minipage}[]{13cm}
Updates to the data in any one visualization tool (e.g., plot, image,
table) creates an update in other visualization tools. For example,
selection of a set of photometry points in a color-color plot causes the
corresponding rows in an Object table and symbols on an image to be
highlighted; or, application of a selection predicate to a table causes
the corresponding points in a plot to be highlighted.
\end{minipage}
\\ \cdashline{1-2}
Requirement Priority &  \\ \cdashline{1-2}
Upper Level Requirement &
\begin{tabular}{cl}
\end{tabular}
\\ \hline
\end{longtable}
}


\subsubsection{Test Cases Summary}
\begin{longtable}{p{3cm}p{2.5cm}p{2.5cm}p{3cm}p{4cm}}
\toprule
\href{https://jira.lsstcorp.org/secure/Tests.jspa\#/testCase/LVV-T740}{LVV-T740} & \multicolumn{4}{p{12cm}}{ Verify linked tables, plots, and images } \\ \hline
\textbf{Owner} & \textbf{Status} & \textbf{Version} & \textbf{Critical Event} & \textbf{Verification Type} \\ \hline
Jeffrey Carlin & Draft & 1 & false & Inspection \\ \hline
\end{longtable}
{\scriptsize
\textbf{Objective:}\\
Verify that the Portal aspect has the capability to have tabular data,
plots, and images with overlays connected via brushing and linking, so
that updates to the data in any one visualization tool (e.g., plot,
image, table) creates an update in other visualization tools.
}
  
 \newpage 
\subsection{[LVV-9942] DMS-PRTL-REQ-0098-V-01: Overlay Catalog of Sources and Objects on
Images\_1 }\label{lvv-9942}

\begin{longtable}{cccc}
\hline
\textbf{Jira Link} & \textbf{Assignee} & \textbf{Status} & \textbf{Test Cases}\\ \hline
\href{https://jira.lsstcorp.org/browse/LVV-9942}{LVV-9942} &
Gregory Dubois-Felsmann & Not Covered &
\begin{tabular}{c}
LVV-T732 \\
\end{tabular}
\\
\hline
\end{longtable}

\textbf{Verification Element Description:} \\
Undefined

{\footnotesize
\begin{longtable}{p{2.5cm}p{13.5cm}}
\hline
\multicolumn{2}{c}{\textbf{Requirement Details}}\\ \hline
Requirement ID & DMS-PRTL-REQ-0098 \\ \cdashline{1-2}
Requirement Description &
\begin{minipage}[]{13cm}
The Portal aspect shall be able to overlay the positions of catalog
sources and objects on a displayed image based upon
astrophysically-based or observatory-based coordinates.
\end{minipage}
\\ \cdashline{1-2}
Requirement Priority &  \\ \cdashline{1-2}
Upper Level Requirement &
\begin{tabular}{cl}
\end{tabular}
\\ \hline
\end{longtable}
}


\subsubsection{Test Cases Summary}
\begin{longtable}{p{3cm}p{2.5cm}p{2.5cm}p{3cm}p{4cm}}
\toprule
\href{https://jira.lsstcorp.org/secure/Tests.jspa\#/testCase/LVV-T732}{LVV-T732} & \multicolumn{4}{p{12cm}}{ Verify overlay of catalog sources/objects on images } \\ \hline
\textbf{Owner} & \textbf{Status} & \textbf{Version} & \textbf{Critical Event} & \textbf{Verification Type} \\ \hline
Jeffrey Carlin & Draft & 1 & false & Inspection \\ \hline
\end{longtable}
{\scriptsize
\textbf{Objective:}\\
Verify that the Portal aspect enables the overlay of positions of
catalog sources and objects on a displayed image based upon
astrophysically-based or observatory-based coordinates.
}
  
 \newpage 
\subsection{[LVV-9943] DMS-PRTL-REQ-0099-V-01: Overlay LSST-Derived Orbits\_1 }\label{lvv-9943}

\begin{longtable}{cccc}
\hline
\textbf{Jira Link} & \textbf{Assignee} & \textbf{Status} & \textbf{Test Cases}\\ \hline
\href{https://jira.lsstcorp.org/browse/LVV-9943}{LVV-9943} &
Gregory Dubois-Felsmann & Not Covered &
\begin{tabular}{c}
LVV-T733 \\
\end{tabular}
\\
\hline
\end{longtable}

\textbf{Verification Element Description:} \\
Undefined

{\footnotesize
\begin{longtable}{p{2.5cm}p{13.5cm}}
\hline
\multicolumn{2}{c}{\textbf{Requirement Details}}\\ \hline
Requirement ID & DMS-PRTL-REQ-0099 \\ \cdashline{1-2}
Requirement Description &
\begin{minipage}[]{13cm}
The Portal aspect shall have the capability to overlay predicted
positions from the orbits of solar system objects in the LSST catalog on
to images.
\end{minipage}
\\ \cdashline{1-2}
Requirement Discussion &
\begin{minipage}[]{13cm}
This is envisioned as the ability to display a specific prediction for a
position along an orbit on a single-epoch image, as well as a set of
predictions for an orbit on a coadded image or all-sky map.\\
It would also be useful to support overlay of predicted positions from
user-supplied orbits in community-standard forms. The capabilities to be
provided in this area will be determined during construction.\\
It might further be useful to be able to overlay intermediate data
products such as tracks and tracklets; whether it is desirable and
feasible to provide this will be determined during construction.
\end{minipage}
\\ \cdashline{1-2}
Requirement Priority &  \\ \cdashline{1-2}
Upper Level Requirement &
\begin{tabular}{cl}
\end{tabular}
\\ \hline
\end{longtable}
}


\subsubsection{Test Cases Summary}
\begin{longtable}{p{3cm}p{2.5cm}p{2.5cm}p{3cm}p{4cm}}
\toprule
\href{https://jira.lsstcorp.org/secure/Tests.jspa\#/testCase/LVV-T733}{LVV-T733} & \multicolumn{4}{p{12cm}}{ Verify overlay of LSST-derived orbits on images } \\ \hline
\textbf{Owner} & \textbf{Status} & \textbf{Version} & \textbf{Critical Event} & \textbf{Verification Type} \\ \hline
Jeffrey Carlin & Draft & 1 & false & Inspection \\ \hline
\end{longtable}
{\scriptsize
\textbf{Objective:}\\
Verify that the Portal aspect has the capability to overlay predicted
positions from the orbits of solar system objects in the LSST catalog on
to images.
}
  
 \newpage 
\subsection{[LVV-9944] DMS-PRTL-REQ-0100-V-01: Overlay User-provided Catalogs on Images\_1 }\label{lvv-9944}

\begin{longtable}{cccc}
\hline
\textbf{Jira Link} & \textbf{Assignee} & \textbf{Status} & \textbf{Test Cases}\\ \hline
\href{https://jira.lsstcorp.org/browse/LVV-9944}{LVV-9944} &
Gregory Dubois-Felsmann & Not Covered &
\begin{tabular}{c}
LVV-T734 \\
\end{tabular}
\\
\hline
\end{longtable}

\textbf{Verification Element Description:} \\
Undefined

{\footnotesize
\begin{longtable}{p{2.5cm}p{13.5cm}}
\hline
\multicolumn{2}{c}{\textbf{Requirement Details}}\\ \hline
Requirement ID & DMS-PRTL-REQ-0100 \\ \cdashline{1-2}
Requirement Description &
\begin{minipage}[]{13cm}
The Portal aspect shall be able to overlay user-provided source lists or
catalogs on images.
\end{minipage}
\\ \cdashline{1-2}
Requirement Priority &  \\ \cdashline{1-2}
Upper Level Requirement &
\begin{tabular}{cl}
\end{tabular}
\\ \hline
\end{longtable}
}


\subsubsection{Test Cases Summary}
\begin{longtable}{p{3cm}p{2.5cm}p{2.5cm}p{3cm}p{4cm}}
\toprule
\href{https://jira.lsstcorp.org/secure/Tests.jspa\#/testCase/LVV-T734}{LVV-T734} & \multicolumn{4}{p{12cm}}{ Verify overlay of user-supplied catalogs on images } \\ \hline
\textbf{Owner} & \textbf{Status} & \textbf{Version} & \textbf{Critical Event} & \textbf{Verification Type} \\ \hline
Jeffrey Carlin & Draft & 1 & false & Inspection \\ \hline
\end{longtable}
{\scriptsize
\textbf{Objective:}\\
Verify that the Portal enables users to overlay the positions of objects
in user-supplied catalogs on top of images.
}
  
 \newpage 
\subsection{[LVV-9945] DMS-PRTL-REQ-0101-V-01: Overlay User-provided Region Files on Images\_1 }\label{lvv-9945}

\begin{longtable}{cccc}
\hline
\textbf{Jira Link} & \textbf{Assignee} & \textbf{Status} & \textbf{Test Cases}\\ \hline
\href{https://jira.lsstcorp.org/browse/LVV-9945}{LVV-9945} &
Gregory Dubois-Felsmann & Not Covered &
\begin{tabular}{c}
LVV-T735 \\
\end{tabular}
\\
\hline
\end{longtable}

\textbf{Verification Element Description:} \\
Undefined

{\footnotesize
\begin{longtable}{p{2.5cm}p{13.5cm}}
\hline
\multicolumn{2}{c}{\textbf{Requirement Details}}\\ \hline
Requirement ID & DMS-PRTL-REQ-0101 \\ \cdashline{1-2}
Requirement Description &
\begin{minipage}[]{13cm}
The Portal aspect shall be able to overlay user-provided region files
(e.g., DS9 region files, focal plane outlines) on images.
\end{minipage}
\\ \cdashline{1-2}
Requirement Priority &  \\ \cdashline{1-2}
Upper Level Requirement &
\begin{tabular}{cl}
\end{tabular}
\\ \hline
\end{longtable}
}


\subsubsection{Test Cases Summary}
\begin{longtable}{p{3cm}p{2.5cm}p{2.5cm}p{3cm}p{4cm}}
\toprule
\href{https://jira.lsstcorp.org/secure/Tests.jspa\#/testCase/LVV-T735}{LVV-T735} & \multicolumn{4}{p{12cm}}{ Verify overlay of user-supplied region files on images } \\ \hline
\textbf{Owner} & \textbf{Status} & \textbf{Version} & \textbf{Critical Event} & \textbf{Verification Type} \\ \hline
Jeffrey Carlin & Draft & 1 & false & Inspection \\ \hline
\end{longtable}
{\scriptsize
\textbf{Objective:}\\
Verify that Portal users can upload a region file and overlay the region
on a displayed image.
}
  
 \newpage 
\subsection{[LVV-9946] DMS-PRTL-REQ-0104-V-01: Position-based Time-Domain Image View\_1 }\label{lvv-9946}

\begin{longtable}{cccc}
\hline
\textbf{Jira Link} & \textbf{Assignee} & \textbf{Status} & \textbf{Test Cases}\\ \hline
\href{https://jira.lsstcorp.org/browse/LVV-9946}{LVV-9946} &
Gregory Dubois-Felsmann & Not Covered &
\begin{tabular}{c}
LVV-T738 \\
\end{tabular}
\\
\hline
\end{longtable}

\textbf{Verification Element Description:} \\
Undefined

{\footnotesize
\begin{longtable}{p{2.5cm}p{13.5cm}}
\hline
\multicolumn{2}{c}{\textbf{Requirement Details}}\\ \hline
Requirement ID & DMS-PRTL-REQ-0104 \\ \cdashline{1-2}
Requirement Description &
\begin{minipage}[]{13cm}
The Portal aspect shall have the capability to view an image time series
that maintains the same physical scale, photometric scale, and image
size display of a specified position on the sky.
\end{minipage}
\\ \cdashline{1-2}
Requirement Discussion &
\begin{minipage}[]{13cm}
If the object moves, then the images stay centered on the sky and the
object appears to move.
\end{minipage}
\\ \cdashline{1-2}
Requirement Priority &  \\ \cdashline{1-2}
Upper Level Requirement &
\begin{tabular}{cl}
\end{tabular}
\\ \hline
\end{longtable}
}


\subsubsection{Test Cases Summary}
\begin{longtable}{p{3cm}p{2.5cm}p{2.5cm}p{3cm}p{4cm}}
\toprule
\href{https://jira.lsstcorp.org/secure/Tests.jspa\#/testCase/LVV-T738}{LVV-T738} & \multicolumn{4}{p{12cm}}{ Verify position-based time-domain image view } \\ \hline
\textbf{Owner} & \textbf{Status} & \textbf{Version} & \textbf{Critical Event} & \textbf{Verification Type} \\ \hline
Jeffrey Carlin & Draft & 1 & false & Inspection \\ \hline
\end{longtable}
{\scriptsize
\textbf{Objective:}\\
Verify that the Portal provides the capability to view an image time
series that maintains the same physical scale, photometric scale, and
image size display of a specified region on the sky. If the object
moves, then the images should stay centered on the sky and the object
will appear to move.
}
  
 \newpage 
\subsection{[LVV-9947] DMS-PRTL-REQ-0108-V-01: Saving Data Selection from a Plot or Image\_1 }\label{lvv-9947}

\begin{longtable}{cccc}
\hline
\textbf{Jira Link} & \textbf{Assignee} & \textbf{Status} & \textbf{Test Cases}\\ \hline
\href{https://jira.lsstcorp.org/browse/LVV-9947}{LVV-9947} &
Gregory Dubois-Felsmann & Not Covered &
\begin{tabular}{c}
LVV-T742 \\
\end{tabular}
\\
\hline
\end{longtable}

\textbf{Verification Element Description:} \\
Undefined

{\footnotesize
\begin{longtable}{p{2.5cm}p{13.5cm}}
\hline
\multicolumn{2}{c}{\textbf{Requirement Details}}\\ \hline
Requirement ID & DMS-PRTL-REQ-0108 \\ \cdashline{1-2}
Requirement Description &
\begin{minipage}[]{13cm}
The Portal aspect shall enable the saving of data selected via a polygon
selection across the linked images, tables, and plots.
\end{minipage}
\\ \cdashline{1-2}
Requirement Discussion &
\begin{minipage}[]{13cm}
An example here is to have an image up; draw a polygon on the image to
select the area on the sky. All the tabular data associated with sources
and objects in that part of the sky would be selected and saved.
\end{minipage}
\\ \cdashline{1-2}
Requirement Priority &  \\ \cdashline{1-2}
Upper Level Requirement &
\begin{tabular}{cl}
\end{tabular}
\\ \hline
\end{longtable}
}


\subsubsection{Test Cases Summary}
\begin{longtable}{p{3cm}p{2.5cm}p{2.5cm}p{3cm}p{4cm}}
\toprule
\href{https://jira.lsstcorp.org/secure/Tests.jspa\#/testCase/LVV-T742}{LVV-T742} & \multicolumn{4}{p{12cm}}{ Verify saving data selection from a plot or image } \\ \hline
\textbf{Owner} & \textbf{Status} & \textbf{Version} & \textbf{Critical Event} & \textbf{Verification Type} \\ \hline
Jeffrey Carlin & Draft & 1 & false & Inspection \\ \hline
\end{longtable}
{\scriptsize
\textbf{Objective:}\\
Verify that the Portal aspect enables the saving of data selected via a
polygon selection across the linked images, tables, and plots.
}
  
 \newpage 
\subsection{[LVV-9948] DMS-PRTL-REQ-0103-V-01: Single-Object Time-Domain Image View\_1 }\label{lvv-9948}

\begin{longtable}{cccc}
\hline
\textbf{Jira Link} & \textbf{Assignee} & \textbf{Status} & \textbf{Test Cases}\\ \hline
\href{https://jira.lsstcorp.org/browse/LVV-9948}{LVV-9948} &
Gregory Dubois-Felsmann & Not Covered &
\begin{tabular}{c}
LVV-T737 \\
\end{tabular}
\\
\hline
\end{longtable}

\textbf{Verification Element Description:} \\
Undefined

{\footnotesize
\begin{longtable}{p{2.5cm}p{13.5cm}}
\hline
\multicolumn{2}{c}{\textbf{Requirement Details}}\\ \hline
Requirement ID & DMS-PRTL-REQ-0103 \\ \cdashline{1-2}
Requirement Description &
\begin{minipage}[]{13cm}
The Portal aspect shall have the capability to view an image time series
that maintains the same physical scale, photometric scale, and image
size display of a cutout area centered on an LSST object
\end{minipage}
\\ \cdashline{1-2}
Requirement Discussion &
\begin{minipage}[]{13cm}
If the object moves, then the images stay centered on the object.
\end{minipage}
\\ \cdashline{1-2}
Requirement Priority &  \\ \cdashline{1-2}
Upper Level Requirement &
\begin{tabular}{cl}
\end{tabular}
\\ \hline
\end{longtable}
}


\subsubsection{Test Cases Summary}
\begin{longtable}{p{3cm}p{2.5cm}p{2.5cm}p{3cm}p{4cm}}
\toprule
\href{https://jira.lsstcorp.org/secure/Tests.jspa\#/testCase/LVV-T737}{LVV-T737} & \multicolumn{4}{p{12cm}}{ Verify single-object time-domain image view } \\ \hline
\textbf{Owner} & \textbf{Status} & \textbf{Version} & \textbf{Critical Event} & \textbf{Verification Type} \\ \hline
Jeffrey Carlin & Draft & 1 & false & Inspection \\ \hline
\end{longtable}
{\scriptsize
\textbf{Objective:}\\
Verify that the Portal provides the capability to view an image time
series that maintains the same physical scale, photometric scale, and
image size display of a cutout area centered on an LSST object. If the
object moves, then the images should stay centered on the object.
}
  
 \newpage 
\subsection{[LVV-9949] DMS-PRTL-REQ-0109-V-01: Access to User Databases\_1 }\label{lvv-9949}

\begin{longtable}{cccc}
\hline
\textbf{Jira Link} & \textbf{Assignee} & \textbf{Status} & \textbf{Test Cases}\\ \hline
\href{https://jira.lsstcorp.org/browse/LVV-9949}{LVV-9949} &
Gregory Dubois-Felsmann & Not Covered &
\begin{tabular}{c}
LVV-T743 \\
\end{tabular}
\\
\hline
\end{longtable}

\textbf{Verification Element Description:} \\
Undefined

{\footnotesize
\begin{longtable}{p{2.5cm}p{13.5cm}}
\hline
\multicolumn{2}{c}{\textbf{Requirement Details}}\\ \hline
Requirement ID & DMS-PRTL-REQ-0109 \\ \cdashline{1-2}
Requirement Description &
\begin{minipage}[]{13cm}
The Portal aspect shall provide read/write access to user databases
(Level 3 tabular data products) and shall implement any access
restrictions placed on such data.
\end{minipage}
\\ \cdashline{1-2}
Requirement Priority &  \\ \cdashline{1-2}
Upper Level Requirement &
\begin{tabular}{cl}
\end{tabular}
\\ \hline
\end{longtable}
}


\subsubsection{Test Cases Summary}
\begin{longtable}{p{3cm}p{2.5cm}p{2.5cm}p{3cm}p{4cm}}
\toprule
\href{https://jira.lsstcorp.org/secure/Tests.jspa\#/testCase/LVV-T743}{LVV-T743} & \multicolumn{4}{p{12cm}}{ Verify access to user databases } \\ \hline
\textbf{Owner} & \textbf{Status} & \textbf{Version} & \textbf{Critical Event} & \textbf{Verification Type} \\ \hline
Jeffrey Carlin & Draft & 1 & false & Inspection \\ \hline
\end{longtable}
{\scriptsize
\textbf{Objective:}\\
Verify that the Portal aspect provides read/write access to user
databases (Level 3 tabular data products) and has implemented any access
restrictions placed on such data.
}
  
 \newpage 
\subsection{[LVV-9950] DMS-PRTL-REQ-0113-V-01: Download Volume Estimation\_1 }\label{lvv-9950}

\begin{longtable}{cccc}
\hline
\textbf{Jira Link} & \textbf{Assignee} & \textbf{Status} & \textbf{Test Cases}\\ \hline
\href{https://jira.lsstcorp.org/browse/LVV-9950}{LVV-9950} &
Gregory Dubois-Felsmann & Not Covered &
\begin{tabular}{c}
LVV-T747 \\
\end{tabular}
\\
\hline
\end{longtable}

\textbf{Verification Element Description:} \\
Undefined

{\footnotesize
\begin{longtable}{p{2.5cm}p{13.5cm}}
\hline
\multicolumn{2}{c}{\textbf{Requirement Details}}\\ \hline
Requirement ID & DMS-PRTL-REQ-0113 \\ \cdashline{1-2}
Requirement Description &
\begin{minipage}[]{13cm}
The Portal aspect shall provide an estimate of the data download volume
prior to a user confirming the download option.
\end{minipage}
\\ \cdashline{1-2}
Requirement Discussion &
\begin{minipage}[]{13cm}
The intent of this requirement is to enable the users to understand how
large a query result or data set may be prior to the full retrieval and
downloading of that data to either the workspace environment or their
own local environment.
\end{minipage}
\\ \cdashline{1-2}
Requirement Priority &  \\ \cdashline{1-2}
Upper Level Requirement &
\begin{tabular}{cl}
\end{tabular}
\\ \hline
\end{longtable}
}


\subsubsection{Test Cases Summary}
\begin{longtable}{p{3cm}p{2.5cm}p{2.5cm}p{3cm}p{4cm}}
\toprule
\href{https://jira.lsstcorp.org/secure/Tests.jspa\#/testCase/LVV-T747}{LVV-T747} & \multicolumn{4}{p{12cm}}{ Verify estimation of data download volume } \\ \hline
\textbf{Owner} & \textbf{Status} & \textbf{Version} & \textbf{Critical Event} & \textbf{Verification Type} \\ \hline
Jeffrey Carlin & Draft & 1 & false & Inspection \\ \hline
\end{longtable}
{\scriptsize
\textbf{Objective:}\\
Verify that the Portal provides an estimate of the volume of a data
download before the user confirms the download option.
}
  
 \newpage 
\subsection{[LVV-9951] DMS-PRTL-REQ-0111-V-01: Image Data Download\_1 }\label{lvv-9951}

\begin{longtable}{cccc}
\hline
\textbf{Jira Link} & \textbf{Assignee} & \textbf{Status} & \textbf{Test Cases}\\ \hline
\href{https://jira.lsstcorp.org/browse/LVV-9951}{LVV-9951} &
Gregory Dubois-Felsmann & Not Covered &
\begin{tabular}{c}
LVV-T745 \\
LVV-T1818 \\
\end{tabular}
\\
\hline
\end{longtable}

\textbf{Verification Element Description:} \\
Undefined

{\footnotesize
\begin{longtable}{p{2.5cm}p{13.5cm}}
\hline
\multicolumn{2}{c}{\textbf{Requirement Details}}\\ \hline
Requirement ID & DMS-PRTL-REQ-0111 \\ \cdashline{1-2}
Requirement Description &
\begin{minipage}[]{13cm}
The Portal aspect shall include mechanisms for a user to download image
data to a remote site or to the Workspace, from both screens displaying
images and screens displaying lists of image metadata.
\end{minipage}
\\ \cdashline{1-2}
Requirement Discussion &
\begin{minipage}[]{13cm}
Again, this should be implemented as a pass-through to the API aspect.
\end{minipage}
\\ \cdashline{1-2}
Requirement Priority &  \\ \cdashline{1-2}
Upper Level Requirement &
\begin{tabular}{cl}
\end{tabular}
\\ \hline
\end{longtable}
}


\subsubsection{Test Cases Summary}
\begin{longtable}{p{3cm}p{2.5cm}p{2.5cm}p{3cm}p{4cm}}
\toprule
\href{https://jira.lsstcorp.org/secure/Tests.jspa\#/testCase/LVV-T745}{LVV-T745} & \multicolumn{4}{p{12cm}}{ Verify image data download } \\ \hline
\textbf{Owner} & \textbf{Status} & \textbf{Version} & \textbf{Critical Event} & \textbf{Verification Type} \\ \hline
Jeffrey Carlin & Draft & 1 & false & Inspection \\ \hline
\end{longtable}
{\scriptsize
\textbf{Objective:}\\
Verify that the Portal aspect includes mechanisms for a user to download
image data to a remote site or to the Workspace, from both screens
displaying images and screens displaying lists of image metadata.
}
\begin{longtable}{p{3cm}p{2.5cm}p{2.5cm}p{3cm}p{4cm}}
\toprule
\href{https://jira.lsstcorp.org/secure/Tests.jspa\#/testCase/LVV-T1818}{LVV-T1818} & \multicolumn{4}{p{12cm}}{ DM-SUIT-8: Verify Portal integration with workspace (via WebDAV) } \\ \hline
\textbf{Owner} & \textbf{Status} & \textbf{Version} & \textbf{Critical Event} & \textbf{Verification Type} \\ \hline
Gregory Dubois-Felsmann & Defined & 1 & false & Demonstration \\ \hline
\end{longtable}
{\scriptsize
\textbf{Objective:}\\
This test case verifies that the Portal Aspect software is capable of
accessing a file-oriented workspace via the WebDAV
protocol.\\[2\baselineskip]In so doing, it partially verifies several
Portal Aspect requirements that relate to this capability -
``partially'' because some of these requirements depend on workspace
capabilities which were not present in the prototype WebDAV service
delivered by the DAX group, because some of the requirements also cover
the User Database Workspace (not relevant to this milestone, and not yet
available), and also because the milestone was not envisioned as an
exhaustive test covering edge cases:

\begin{itemize}
\tightlist
\item
  DMS-PRTL-REQ-0003 (LVV-9846, Portal access to workspace) is covered at
  ``demonstration'' level, with basic tests of saving image and tabular
  data to the workspace, and only for the User File Workspace ~(there is
  currently no User Database Workspace prototype available);
\item
  DMS-PRTL-REQ-0046 (LVV-9886, Visualization of workspace data) is
  covered at ``demonstration'' level for a couple of FITS image and
  table files, and only for the User File Workspace;
\item
  DMS-PRTL-REQ-0110 (LVV-9954, Tabular data download) is covered at
  ``demonstration'' level, only for catalog data (there was no image
  metadata in the LSP deployment at the time of test), and only for the
  User File Workspace;~
\item
  DMS-PRTL-REQ-0095 (LVV-9932, Saving Displayed Tabular Data) is covered
  at ``demonstration'' level for a simple subset operation in the table
  browser; and
\item
  DMS-PRTL-REQ-0111 (LVV-9951, Image data download) is covered at
  ``demonstration'' level, and only for download from an image display
  screen itself (as LSST-style image metadata services, e.g., ObsTAP,
  were not available in the LSP at the time of testing).
\end{itemize}
}
  
 \newpage 
\subsection{[LVV-9952] DMS-PRTL-REQ-0114-V-01: Long Download Completion Notification\_1 }\label{lvv-9952}

\begin{longtable}{cccc}
\hline
\textbf{Jira Link} & \textbf{Assignee} & \textbf{Status} & \textbf{Test Cases}\\ \hline
\href{https://jira.lsstcorp.org/browse/LVV-9952}{LVV-9952} &
Gregory Dubois-Felsmann & Not Covered &
\begin{tabular}{c}
LVV-T748 \\
\end{tabular}
\\
\hline
\end{longtable}

\textbf{Verification Element Description:} \\
Undefined

{\footnotesize
\begin{longtable}{p{2.5cm}p{13.5cm}}
\hline
\multicolumn{2}{c}{\textbf{Requirement Details}}\\ \hline
Requirement ID & DMS-PRTL-REQ-0114 \\ \cdashline{1-2}
Requirement Description &
\begin{minipage}[]{13cm}
The Portal aspect shall notify the user with an estimate of how long a
download is expected to take. The user can continue to monitor the
download; an option shall be provided to notify the user when the
download has completed.
\end{minipage}
\\ \cdashline{1-2}
Requirement Discussion &
\begin{minipage}[]{13cm}
DAX requirement for async
\end{minipage}
\\ \cdashline{1-2}
Requirement Priority &  \\ \cdashline{1-2}
Upper Level Requirement &
\begin{tabular}{cl}
\end{tabular}
\\ \hline
\end{longtable}
}


\subsubsection{Test Cases Summary}
\begin{longtable}{p{3cm}p{2.5cm}p{2.5cm}p{3cm}p{4cm}}
\toprule
\href{https://jira.lsstcorp.org/secure/Tests.jspa\#/testCase/LVV-T748}{LVV-T748} & \multicolumn{4}{p{12cm}}{ Verify notification of long download completion } \\ \hline
\textbf{Owner} & \textbf{Status} & \textbf{Version} & \textbf{Critical Event} & \textbf{Verification Type} \\ \hline
Jeffrey Carlin & Draft & 1 & false & Inspection \\ \hline
\end{longtable}
{\scriptsize
\textbf{Objective:}\\
Verify that the Portal aspect notifies the user with an estimate of how
long a download is expected to take. The user can continue to monitor
the download; verify that an option has been provided to notify the user
when the download has completed.
}
  
 \newpage 
\subsection{[LVV-9953] DMS-PRTL-REQ-0112-V-01: Selected Image Download\_1 }\label{lvv-9953}

\begin{longtable}{cccc}
\hline
\textbf{Jira Link} & \textbf{Assignee} & \textbf{Status} & \textbf{Test Cases}\\ \hline
\href{https://jira.lsstcorp.org/browse/LVV-9953}{LVV-9953} &
Gregory Dubois-Felsmann & Not Covered &
\begin{tabular}{c}
LVV-T746 \\
\end{tabular}
\\
\hline
\end{longtable}

\textbf{Verification Element Description:} \\
Undefined

{\footnotesize
\begin{longtable}{p{2.5cm}p{13.5cm}}
\hline
\multicolumn{2}{c}{\textbf{Requirement Details}}\\ \hline
Requirement ID & DMS-PRTL-REQ-0112 \\ \cdashline{1-2}
Requirement Description &
\begin{minipage}[]{13cm}
The Portal aspect shall support user selection for download of a subset
of the images in an image metadata table or image cutout table.
\end{minipage}
\\ \cdashline{1-2}
Requirement Priority &  \\ \cdashline{1-2}
Upper Level Requirement &
\begin{tabular}{cl}
\end{tabular}
\\ \hline
\end{longtable}
}


\subsubsection{Test Cases Summary}
\begin{longtable}{p{3cm}p{2.5cm}p{2.5cm}p{3cm}p{4cm}}
\toprule
\href{https://jira.lsstcorp.org/secure/Tests.jspa\#/testCase/LVV-T746}{LVV-T746} & \multicolumn{4}{p{12cm}}{ Verify selected image download } \\ \hline
\textbf{Owner} & \textbf{Status} & \textbf{Version} & \textbf{Critical Event} & \textbf{Verification Type} \\ \hline
Jeffrey Carlin & Draft & 1 & false & Inspection \\ \hline
\end{longtable}
{\scriptsize
\textbf{Objective:}\\
Verify that the Portal aspect supports user selection for download of a
subset of the images in an image metadata table or image cutout table.
~~
}
  
 \newpage 
\subsection{[LVV-9954] DMS-PRTL-REQ-0110-V-01: Tabular Data Download\_1 }\label{lvv-9954}

\begin{longtable}{cccc}
\hline
\textbf{Jira Link} & \textbf{Assignee} & \textbf{Status} & \textbf{Test Cases}\\ \hline
\href{https://jira.lsstcorp.org/browse/LVV-9954}{LVV-9954} &
Gregory Dubois-Felsmann & Not Covered &
\begin{tabular}{c}
LVV-T744 \\
LVV-T1818 \\
\end{tabular}
\\
\hline
\end{longtable}

\textbf{Verification Element Description:} \\
Undefined

{\footnotesize
\begin{longtable}{p{2.5cm}p{13.5cm}}
\hline
\multicolumn{2}{c}{\textbf{Requirement Details}}\\ \hline
Requirement ID & DMS-PRTL-REQ-0110 \\ \cdashline{1-2}
Requirement Description &
\begin{minipage}[]{13cm}
The Portal aspect shall include a mechanism for a user to download to a
remote site, Workspace, or to an existing or new user database the
tabular results from a database query, including for catalog or image
metadata.
\end{minipage}
\\ \cdashline{1-2}
Requirement Discussion &
\begin{minipage}[]{13cm}
This may be implemented as a pass-through to the API aspect when applied
to the results of a query against an LSST database.
\end{minipage}
\\ \cdashline{1-2}
Requirement Priority &  \\ \cdashline{1-2}
Upper Level Requirement &
\begin{tabular}{cl}
\end{tabular}
\\ \hline
\end{longtable}
}


\subsubsection{Test Cases Summary}
\begin{longtable}{p{3cm}p{2.5cm}p{2.5cm}p{3cm}p{4cm}}
\toprule
\href{https://jira.lsstcorp.org/secure/Tests.jspa\#/testCase/LVV-T744}{LVV-T744} & \multicolumn{4}{p{12cm}}{ Verify tabular data download } \\ \hline
\textbf{Owner} & \textbf{Status} & \textbf{Version} & \textbf{Critical Event} & \textbf{Verification Type} \\ \hline
Jeffrey Carlin & Draft & 1 & false & Inspection \\ \hline
\end{longtable}
{\scriptsize
\textbf{Objective:}\\
Verify that the Portal aspect includes a mechanism for a user to
download to a remote site, Workspace, or to an existing or new user
database the tabular results from a database query, including for
catalog or image metadata.
}
\begin{longtable}{p{3cm}p{2.5cm}p{2.5cm}p{3cm}p{4cm}}
\toprule
\href{https://jira.lsstcorp.org/secure/Tests.jspa\#/testCase/LVV-T1818}{LVV-T1818} & \multicolumn{4}{p{12cm}}{ DM-SUIT-8: Verify Portal integration with workspace (via WebDAV) } \\ \hline
\textbf{Owner} & \textbf{Status} & \textbf{Version} & \textbf{Critical Event} & \textbf{Verification Type} \\ \hline
Gregory Dubois-Felsmann & Defined & 1 & false & Demonstration \\ \hline
\end{longtable}
{\scriptsize
\textbf{Objective:}\\
This test case verifies that the Portal Aspect software is capable of
accessing a file-oriented workspace via the WebDAV
protocol.\\[2\baselineskip]In so doing, it partially verifies several
Portal Aspect requirements that relate to this capability -
``partially'' because some of these requirements depend on workspace
capabilities which were not present in the prototype WebDAV service
delivered by the DAX group, because some of the requirements also cover
the User Database Workspace (not relevant to this milestone, and not yet
available), and also because the milestone was not envisioned as an
exhaustive test covering edge cases:

\begin{itemize}
\tightlist
\item
  DMS-PRTL-REQ-0003 (LVV-9846, Portal access to workspace) is covered at
  ``demonstration'' level, with basic tests of saving image and tabular
  data to the workspace, and only for the User File Workspace ~(there is
  currently no User Database Workspace prototype available);
\item
  DMS-PRTL-REQ-0046 (LVV-9886, Visualization of workspace data) is
  covered at ``demonstration'' level for a couple of FITS image and
  table files, and only for the User File Workspace;
\item
  DMS-PRTL-REQ-0110 (LVV-9954, Tabular data download) is covered at
  ``demonstration'' level, only for catalog data (there was no image
  metadata in the LSP deployment at the time of test), and only for the
  User File Workspace;~
\item
  DMS-PRTL-REQ-0095 (LVV-9932, Saving Displayed Tabular Data) is covered
  at ``demonstration'' level for a simple subset operation in the table
  browser; and
\item
  DMS-PRTL-REQ-0111 (LVV-9951, Image data download) is covered at
  ``demonstration'' level, and only for download from an image display
  screen itself (as LSST-style image metadata services, e.g., ObsTAP,
  were not available in the LSP at the time of testing).
\end{itemize}
}
  
 \newpage 
\subsection{[LVV-9955] DMS-PRTL-REQ-0115-V-01: APIs for Visualization Components\_1 }\label{lvv-9955}

\begin{longtable}{cccc}
\hline
\textbf{Jira Link} & \textbf{Assignee} & \textbf{Status} & \textbf{Test Cases}\\ \hline
\href{https://jira.lsstcorp.org/browse/LVV-9955}{LVV-9955} &
Gregory Dubois-Felsmann & Not Covered &
\begin{tabular}{c}
LVV-T749 \\
\end{tabular}
\\
\hline
\end{longtable}

\textbf{Verification Element Description:} \\
Undefined

{\footnotesize
\begin{longtable}{p{2.5cm}p{13.5cm}}
\hline
\multicolumn{2}{c}{\textbf{Requirement Details}}\\ \hline
Requirement ID & DMS-PRTL-REQ-0115 \\ \cdashline{1-2}
Requirement Description &
\begin{minipage}[]{13cm}
The Portal aspect shall provide a documented application program
interface that allows users and services at any location to access and
manipulate the Portal's visualization services
\end{minipage}
\\ \cdashline{1-2}
Requirement Discussion &
\begin{minipage}[]{13cm}
This is intended to be enable API control of the visualization
components and tool-level visualization services to be called and
controlled through an API.\\
There will be a Web API as well as a Python wrapper for it.
\end{minipage}
\\ \cdashline{1-2}
Requirement Priority &  \\ \cdashline{1-2}
Upper Level Requirement &
\begin{tabular}{cl}
\end{tabular}
\\ \hline
\end{longtable}
}


\subsubsection{Test Cases Summary}
\begin{longtable}{p{3cm}p{2.5cm}p{2.5cm}p{3cm}p{4cm}}
\toprule
\href{https://jira.lsstcorp.org/secure/Tests.jspa\#/testCase/LVV-T749}{LVV-T749} & \multicolumn{4}{p{12cm}}{ Verify API for visualization components } \\ \hline
\textbf{Owner} & \textbf{Status} & \textbf{Version} & \textbf{Critical Event} & \textbf{Verification Type} \\ \hline
Jeffrey Carlin & Draft & 1 & false & Inspection \\ \hline
\end{longtable}
{\scriptsize
\textbf{Objective:}\\
Verify that the Portal aspect provides a documented application program
interface that allows users and services at any location to access and
manipulate the Portal's visualization services. This is intended to
enable API control of the visualization components and tool-level
visualization services to be called and controlled through an API. There
will be a Web API as well as a Python wrapper for it.
}
  
 \newpage 
\subsection{[LVV-9956] DMS-PRTL-REQ-0117-V-01: Computational Quotas User Interface\_1 }\label{lvv-9956}

\begin{longtable}{cccc}
\hline
\textbf{Jira Link} & \textbf{Assignee} & \textbf{Status} & \textbf{Test Cases}\\ \hline
\href{https://jira.lsstcorp.org/browse/LVV-9956}{LVV-9956} &
Gregory Dubois-Felsmann & Not Covered &
\begin{tabular}{c}
LVV-T751 \\
\end{tabular}
\\
\hline
\end{longtable}

\textbf{Verification Element Description:} \\
Undefined

{\footnotesize
\begin{longtable}{p{2.5cm}p{13.5cm}}
\hline
\multicolumn{2}{c}{\textbf{Requirement Details}}\\ \hline
Requirement ID & DMS-PRTL-REQ-0117 \\ \cdashline{1-2}
Requirement Description &
\begin{minipage}[]{13cm}
The Portal aspect shall provide the user with an understanding of the
current status of their allocations.
\end{minipage}
\\ \cdashline{1-2}
Requirement Discussion &
\begin{minipage}[]{13cm}
This requirement is about the SUIT implementing the quotas defined by
the DM system. Those requirements need to defined and work through by
the Project and the DM.
\end{minipage}
\\ \cdashline{1-2}
Requirement Priority &  \\ \cdashline{1-2}
Upper Level Requirement &
\begin{tabular}{cl}
\end{tabular}
\\ \hline
\end{longtable}
}


\subsubsection{Test Cases Summary}
\begin{longtable}{p{3cm}p{2.5cm}p{2.5cm}p{3cm}p{4cm}}
\toprule
\href{https://jira.lsstcorp.org/secure/Tests.jspa\#/testCase/LVV-T751}{LVV-T751} & \multicolumn{4}{p{12cm}}{ Verify implementation of computational quotas status } \\ \hline
\textbf{Owner} & \textbf{Status} & \textbf{Version} & \textbf{Critical Event} & \textbf{Verification Type} \\ \hline
Jeffrey Carlin & Draft & 1 & false & Inspection \\ \hline
\end{longtable}
{\scriptsize
\textbf{Objective:}\\
Verify that the Portal aspect provides a summary of the current status
of users' allocations of computational resources.
}
  
 \newpage 
\subsection{[LVV-9957] DMS-PRTL-REQ-0118-V-01: Portal Display Preferences\_1 }\label{lvv-9957}

\begin{longtable}{cccc}
\hline
\textbf{Jira Link} & \textbf{Assignee} & \textbf{Status} & \textbf{Test Cases}\\ \hline
\href{https://jira.lsstcorp.org/browse/LVV-9957}{LVV-9957} &
Gregory Dubois-Felsmann & Not Covered &
\begin{tabular}{c}
LVV-T752 \\
\end{tabular}
\\
\hline
\end{longtable}

\textbf{Verification Element Description:} \\
Undefined

{\footnotesize
\begin{longtable}{p{2.5cm}p{13.5cm}}
\hline
\multicolumn{2}{c}{\textbf{Requirement Details}}\\ \hline
Requirement ID & DMS-PRTL-REQ-0118 \\ \cdashline{1-2}
Requirement Description &
\begin{minipage}[]{13cm}
The Portal aspect shall enable a user to establish and save viewing
preferences, including, but not limited to, which tabular data columns
to view, how tables should be sorted by default, which calculated
quantities appear within a table, what image stretch and color tables,
what types of plots are generated, how data are overlaid on images.
\end{minipage}
\\ \cdashline{1-2}
Requirement Discussion &
\begin{minipage}[]{13cm}
The intent behind this requirement is to enable user to set up a working
environment within the default portal or the workspace environment and
be able to save the state of the workflow.
\end{minipage}
\\ \cdashline{1-2}
Requirement Priority &  \\ \cdashline{1-2}
Upper Level Requirement &
\begin{tabular}{cl}
\end{tabular}
\\ \hline
\end{longtable}
}


\subsubsection{Test Cases Summary}
\begin{longtable}{p{3cm}p{2.5cm}p{2.5cm}p{3cm}p{4cm}}
\toprule
\href{https://jira.lsstcorp.org/secure/Tests.jspa\#/testCase/LVV-T752}{LVV-T752} & \multicolumn{4}{p{12cm}}{ Verify saved Portal display preferences } \\ \hline
\textbf{Owner} & \textbf{Status} & \textbf{Version} & \textbf{Critical Event} & \textbf{Verification Type} \\ \hline
Jeffrey Carlin & Draft & 1 & false & Inspection \\ \hline
\end{longtable}
{\scriptsize
\textbf{Objective:}\\
Verify that the Portal aspect enables a user to establish and save
viewing preferences, including, but not limited to, which tabular data
columns to view, how tables should be sorted by default, which
calculated quantities appear within a table, what image stretch and
color tables, what types of plots are generated, how data are overlaid
on images.
}
  
 \newpage 
\subsection{[LVV-9958] DMS-PRTL-REQ-0116-V-01: Storage Quotas User Interface\_1 }\label{lvv-9958}

\begin{longtable}{cccc}
\hline
\textbf{Jira Link} & \textbf{Assignee} & \textbf{Status} & \textbf{Test Cases}\\ \hline
\href{https://jira.lsstcorp.org/browse/LVV-9958}{LVV-9958} &
Gregory Dubois-Felsmann & Not Covered &
\begin{tabular}{c}
LVV-T750 \\
\end{tabular}
\\
\hline
\end{longtable}

\textbf{Verification Element Description:} \\
Undefined

{\footnotesize
\begin{longtable}{p{2.5cm}p{13.5cm}}
\hline
\multicolumn{2}{c}{\textbf{Requirement Details}}\\ \hline
Requirement ID & DMS-PRTL-REQ-0116 \\ \cdashline{1-2}
Requirement Description &
\begin{minipage}[]{13cm}
The Portal aspect shall provide the user with an understanding of the
current status of their storage allocations
\end{minipage}
\\ \cdashline{1-2}
Requirement Discussion &
\begin{minipage}[]{13cm}
This requirement is about the SUIT implementing the quotas defined by
the DM system. Those requirements need to defined and work through by
the Project and the DM.
\end{minipage}
\\ \cdashline{1-2}
Requirement Priority &  \\ \cdashline{1-2}
Upper Level Requirement &
\begin{tabular}{cl}
\end{tabular}
\\ \hline
\end{longtable}
}


\subsubsection{Test Cases Summary}
\begin{longtable}{p{3cm}p{2.5cm}p{2.5cm}p{3cm}p{4cm}}
\toprule
\href{https://jira.lsstcorp.org/secure/Tests.jspa\#/testCase/LVV-T750}{LVV-T750} & \multicolumn{4}{p{12cm}}{ Verify implementation of storage quotas status } \\ \hline
\textbf{Owner} & \textbf{Status} & \textbf{Version} & \textbf{Critical Event} & \textbf{Verification Type} \\ \hline
Jeffrey Carlin & Draft & 1 & false & Inspection \\ \hline
\end{longtable}
{\scriptsize
\textbf{Objective:}\\
Verify that the Portal aspect provides a summary of the current status
of users' storage allocations.
}
  
 \newpage 
\subsection{[LVV-9959] DMS-PRTL-REQ-0127-V-01: Alert Subscription Monitoring\_1 }\label{lvv-9959}

\begin{longtable}{cccc}
\hline
\textbf{Jira Link} & \textbf{Assignee} & \textbf{Status} & \textbf{Test Cases}\\ \hline
\href{https://jira.lsstcorp.org/browse/LVV-9959}{LVV-9959} &
Gregory Dubois-Felsmann & Not Covered &
\begin{tabular}{c}
LVV-T756 \\
\end{tabular}
\\
\hline
\end{longtable}

\textbf{Verification Element Description:} \\
Undefined

{\footnotesize
\begin{longtable}{p{2.5cm}p{13.5cm}}
\hline
\multicolumn{2}{c}{\textbf{Requirement Details}}\\ \hline
Requirement ID & DMS-PRTL-REQ-0127 \\ \cdashline{1-2}
Requirement Description &
\begin{minipage}[]{13cm}
The Portal aspect shall report feedback about the status and performance
of a user's filters in the alert subscription service.
\end{minipage}
\\ \cdashline{1-2}
Requirement Discussion &
\begin{minipage}[]{13cm}
This is a front end to information exposed by an API provided by the
Simple Filtering Service, and is expected to encompass information such
as filter status (enabled/disabled, error), statistics on the number of
alerts seen and the number transmitted by the filter, and optional
debugging log information.
\end{minipage}
\\ \cdashline{1-2}
Requirement Priority &  \\ \cdashline{1-2}
Upper Level Requirement &
\begin{tabular}{cl}
\end{tabular}
\\ \hline
\end{longtable}
}


\subsubsection{Test Cases Summary}
\begin{longtable}{p{3cm}p{2.5cm}p{2.5cm}p{3cm}p{4cm}}
\toprule
\href{https://jira.lsstcorp.org/secure/Tests.jspa\#/testCase/LVV-T756}{LVV-T756} & \multicolumn{4}{p{12cm}}{ Verify monitoring of alert subscription } \\ \hline
\textbf{Owner} & \textbf{Status} & \textbf{Version} & \textbf{Critical Event} & \textbf{Verification Type} \\ \hline
Jeffrey Carlin & Draft & 1 & false & Inspection \\ \hline
\end{longtable}
{\scriptsize
\textbf{Objective:}\\
Verify that the Portal provides feedback about the status and
performance of a user's filters in the alert subscription service.
}
  
 \newpage 
\subsection{[LVV-9960] DMS-PRTL-REQ-0119-V-01: Alert Subscription Service\_1 }\label{lvv-9960}

\begin{longtable}{cccc}
\hline
\textbf{Jira Link} & \textbf{Assignee} & \textbf{Status} & \textbf{Test Cases}\\ \hline
\href{https://jira.lsstcorp.org/browse/LVV-9960}{LVV-9960} &
Gregory Dubois-Felsmann & Not Covered &
\begin{tabular}{c}
LVV-T753 \\
\end{tabular}
\\
\hline
\end{longtable}

\textbf{Verification Element Description:} \\
Undefined

{\footnotesize
\begin{longtable}{p{2.5cm}p{13.5cm}}
\hline
\multicolumn{2}{c}{\textbf{Requirement Details}}\\ \hline
Requirement ID & DMS-PRTL-REQ-0119 \\ \cdashline{1-2}
Requirement Description &
\begin{minipage}[]{13cm}
The Portal aspect shall provide an interface to the alert subscription
service that allows authenticated users with LSST data rights to
subscribe to a stream of alert events.
\end{minipage}
\\ \cdashline{1-2}
Requirement Discussion &
\begin{minipage}[]{13cm}
This is just a UI for the underlying capability developed under
02C.03.03.\\
Note that users without data rights will have to subscribe to alerts
through brokers external to the LSST project.
\end{minipage}
\\ \cdashline{1-2}
Requirement Priority &  \\ \cdashline{1-2}
Upper Level Requirement &
\begin{tabular}{cl}
\end{tabular}
\\ \hline
\end{longtable}
}


\subsubsection{Test Cases Summary}
\begin{longtable}{p{3cm}p{2.5cm}p{2.5cm}p{3cm}p{4cm}}
\toprule
\href{https://jira.lsstcorp.org/secure/Tests.jspa\#/testCase/LVV-T753}{LVV-T753} & \multicolumn{4}{p{12cm}}{ Verify alert subscription service } \\ \hline
\textbf{Owner} & \textbf{Status} & \textbf{Version} & \textbf{Critical Event} & \textbf{Verification Type} \\ \hline
Jeffrey Carlin & Draft & 1 & false & Inspection \\ \hline
\end{longtable}
{\scriptsize
\textbf{Objective:}\\
Verify that the Portal aspect provides an interface to the alert
subscription service that allows authenticated users with LSST data
rights to subscribe to a stream of alert events.
}
  
 \newpage 
\subsection{[LVV-9961] DMS-PRTL-REQ-0120-V-01: Pre-defined Alert Filters\_1 }\label{lvv-9961}

\begin{longtable}{cccc}
\hline
\textbf{Jira Link} & \textbf{Assignee} & \textbf{Status} & \textbf{Test Cases}\\ \hline
\href{https://jira.lsstcorp.org/browse/LVV-9961}{LVV-9961} &
Gregory Dubois-Felsmann & Not Covered &
\begin{tabular}{c}
LVV-T754 \\
\end{tabular}
\\
\hline
\end{longtable}

\textbf{Verification Element Description:} \\
Undefined

{\footnotesize
\begin{longtable}{p{2.5cm}p{13.5cm}}
\hline
\multicolumn{2}{c}{\textbf{Requirement Details}}\\ \hline
Requirement ID & DMS-PRTL-REQ-0120 \\ \cdashline{1-2}
Requirement Description &
\begin{minipage}[]{13cm}
The Portal aspect shall provide an interface to permit alert
subscriptions to be configured with Project-provided alert filters.
\end{minipage}
\\ \cdashline{1-2}
Requirement Priority &  \\ \cdashline{1-2}
Upper Level Requirement &
\begin{tabular}{cl}
\end{tabular}
\\ \hline
\end{longtable}
}


\subsubsection{Test Cases Summary}
\begin{longtable}{p{3cm}p{2.5cm}p{2.5cm}p{3cm}p{4cm}}
\toprule
\href{https://jira.lsstcorp.org/secure/Tests.jspa\#/testCase/LVV-T754}{LVV-T754} & \multicolumn{4}{p{12cm}}{ Verify availability of pre-defined alert filters } \\ \hline
\textbf{Owner} & \textbf{Status} & \textbf{Version} & \textbf{Critical Event} & \textbf{Verification Type} \\ \hline
Jeffrey Carlin & Draft & 1 & false & Inspection \\ \hline
\end{longtable}
{\scriptsize
\textbf{Objective:}\\
Verify that the Portal provides an interface to permit alert
subscriptions to be configured with Project-provided alert filters.
}
  
 \newpage 
\subsection{[LVV-9962] DMS-PRTL-REQ-0121-V-01: User-defined Alert Filters\_1 }\label{lvv-9962}

\begin{longtable}{cccc}
\hline
\textbf{Jira Link} & \textbf{Assignee} & \textbf{Status} & \textbf{Test Cases}\\ \hline
\href{https://jira.lsstcorp.org/browse/LVV-9962}{LVV-9962} &
Gregory Dubois-Felsmann & Not Covered &
\begin{tabular}{c}
LVV-T755 \\
\end{tabular}
\\
\hline
\end{longtable}

\textbf{Verification Element Description:} \\
Undefined

{\footnotesize
\begin{longtable}{p{2.5cm}p{13.5cm}}
\hline
\multicolumn{2}{c}{\textbf{Requirement Details}}\\ \hline
Requirement ID & DMS-PRTL-REQ-0121 \\ \cdashline{1-2}
Requirement Description &
\begin{minipage}[]{13cm}
The Portal aspect shall provide an interface to permit alert
subscriptions to be configured with user-defined alert filters.
\end{minipage}
\\ \cdashline{1-2}
Requirement Priority &  \\ \cdashline{1-2}
Upper Level Requirement &
\begin{tabular}{cl}
\end{tabular}
\\ \hline
\end{longtable}
}


\subsubsection{Test Cases Summary}
\begin{longtable}{p{3cm}p{2.5cm}p{2.5cm}p{3cm}p{4cm}}
\toprule
\href{https://jira.lsstcorp.org/secure/Tests.jspa\#/testCase/LVV-T755}{LVV-T755} & \multicolumn{4}{p{12cm}}{ Verify availability of user-defined alert filters } \\ \hline
\textbf{Owner} & \textbf{Status} & \textbf{Version} & \textbf{Critical Event} & \textbf{Verification Type} \\ \hline
Jeffrey Carlin & Draft & 1 & false & Inspection \\ \hline
\end{longtable}
{\scriptsize
\textbf{Objective:}\\
Verify that the Portal provides an interface to permit alert
subscriptions to be configured with user-provided alert filters.
}
  
 \newpage 
\subsection{[LVV-9963] DMS-PRTL-REQ-0122-V-01: Access to Observatory Documentation\_1 }\label{lvv-9963}

\begin{longtable}{cccc}
\hline
\textbf{Jira Link} & \textbf{Assignee} & \textbf{Status} & \textbf{Test Cases}\\ \hline
\href{https://jira.lsstcorp.org/browse/LVV-9963}{LVV-9963} &
Gregory Dubois-Felsmann & Not Covered &
\begin{tabular}{c}
LVV-T757 \\
\end{tabular}
\\
\hline
\end{longtable}

\textbf{Verification Element Description:} \\
Undefined

{\footnotesize
\begin{longtable}{p{2.5cm}p{13.5cm}}
\hline
\multicolumn{2}{c}{\textbf{Requirement Details}}\\ \hline
Requirement ID & DMS-PRTL-REQ-0122 \\ \cdashline{1-2}
Requirement Description &
\begin{minipage}[]{13cm}
The Portal aspect shall provide access to Project-provided documentation
on the design, construction, and operation of the LSST.
\end{minipage}
\\ \cdashline{1-2}
Requirement Discussion &
\begin{minipage}[]{13cm}
Much of this will be just a link or links to documentation whose
creation is out of the scope of DM. The DM documentation will include
documentation on the code, on the pipeline processing, and on the
delivered data quality. Note that documentation on the Science Platform
itself will also be included.
\end{minipage}
\\ \cdashline{1-2}
Requirement Priority &  \\ \cdashline{1-2}
Upper Level Requirement &
\begin{tabular}{cl}
\end{tabular}
\\ \hline
\end{longtable}
}


\subsubsection{Test Cases Summary}
\begin{longtable}{p{3cm}p{2.5cm}p{2.5cm}p{3cm}p{4cm}}
\toprule
\href{https://jira.lsstcorp.org/secure/Tests.jspa\#/testCase/LVV-T757}{LVV-T757} & \multicolumn{4}{p{12cm}}{ Verify access to survey documentation } \\ \hline
\textbf{Owner} & \textbf{Status} & \textbf{Version} & \textbf{Critical Event} & \textbf{Verification Type} \\ \hline
Jeffrey Carlin & Draft & 1 & false & Inspection \\ \hline
\end{longtable}
{\scriptsize
\textbf{Objective:}\\
Verify that the Portal provides access to Project-provided documentation
on the design, construction, and operation of the LSST.
}
  
 \newpage 
\subsection{[LVV-9964] DMS-PRTL-REQ-0124-V-01: Portal API Documentation\_1 }\label{lvv-9964}

\begin{longtable}{cccc}
\hline
\textbf{Jira Link} & \textbf{Assignee} & \textbf{Status} & \textbf{Test Cases}\\ \hline
\href{https://jira.lsstcorp.org/browse/LVV-9964}{LVV-9964} &
Gregory Dubois-Felsmann & Not Covered &
\begin{tabular}{c}
LVV-T759 \\
\end{tabular}
\\
\hline
\end{longtable}

\textbf{Verification Element Description:} \\
Undefined

{\footnotesize
\begin{longtable}{p{2.5cm}p{13.5cm}}
\hline
\multicolumn{2}{c}{\textbf{Requirement Details}}\\ \hline
Requirement ID & DMS-PRTL-REQ-0124 \\ \cdashline{1-2}
Requirement Description &
\begin{minipage}[]{13cm}
The Portal aspect shall provide reference-manual-style documentation on
its public network and programmatic APIs.
\end{minipage}
\\ \cdashline{1-2}
Requirement Discussion &
\begin{minipage}[]{13cm}
This may include network APIs that allow interaction with a Portal
session's state, Python APIs including wrappers for the network APIs,
and JavaScript APIs for the components of the Portal application.\\
This requirement is somewhat redundant with the general DM requirement
that all code be supplied with reference documentation, but emphasizes a
coherent presentation of the APIs relevant to the Portal.
\end{minipage}
\\ \cdashline{1-2}
Requirement Priority &  \\ \cdashline{1-2}
Upper Level Requirement &
\begin{tabular}{cl}
\end{tabular}
\\ \hline
\end{longtable}
}


\subsubsection{Test Cases Summary}
\begin{longtable}{p{3cm}p{2.5cm}p{2.5cm}p{3cm}p{4cm}}
\toprule
\href{https://jira.lsstcorp.org/secure/Tests.jspa\#/testCase/LVV-T759}{LVV-T759} & \multicolumn{4}{p{12cm}}{ Verify access to Portal API documentation } \\ \hline
\textbf{Owner} & \textbf{Status} & \textbf{Version} & \textbf{Critical Event} & \textbf{Verification Type} \\ \hline
Jeffrey Carlin & Draft & 1 & false & Inspection \\ \hline
\end{longtable}
{\scriptsize
\textbf{Objective:}\\
Verify that the Portal provides access to reference manual-style
documentation of its public network and programmatic APIs.
}
  
 \newpage 
\subsection{[LVV-9965] DMS-PRTL-REQ-0123-V-01: Portal User Documentation\_1 }\label{lvv-9965}

\begin{longtable}{cccc}
\hline
\textbf{Jira Link} & \textbf{Assignee} & \textbf{Status} & \textbf{Test Cases}\\ \hline
\href{https://jira.lsstcorp.org/browse/LVV-9965}{LVV-9965} &
Gregory Dubois-Felsmann & Not Covered &
\begin{tabular}{c}
LVV-T758 \\
\end{tabular}
\\
\hline
\end{longtable}

\textbf{Verification Element Description:} \\
Undefined

{\footnotesize
\begin{longtable}{p{2.5cm}p{13.5cm}}
\hline
\multicolumn{2}{c}{\textbf{Requirement Details}}\\ \hline
Requirement ID & DMS-PRTL-REQ-0123 \\ \cdashline{1-2}
Requirement Description &
\begin{minipage}[]{13cm}
The Portal aspect shall provide user-guide-style documentation on the
use of the Portal.
\end{minipage}
\\ \cdashline{1-2}
Requirement Priority &  \\ \cdashline{1-2}
Upper Level Requirement &
\begin{tabular}{cl}
\end{tabular}
\\ \hline
\end{longtable}
}


\subsubsection{Test Cases Summary}
\begin{longtable}{p{3cm}p{2.5cm}p{2.5cm}p{3cm}p{4cm}}
\toprule
\href{https://jira.lsstcorp.org/secure/Tests.jspa\#/testCase/LVV-T758}{LVV-T758} & \multicolumn{4}{p{12cm}}{ Verify access to Portal documentation } \\ \hline
\textbf{Owner} & \textbf{Status} & \textbf{Version} & \textbf{Critical Event} & \textbf{Verification Type} \\ \hline
Jeffrey Carlin & Draft & 1 & false & Inspection \\ \hline
\end{longtable}
{\scriptsize
\textbf{Objective:}\\
Verify that the Portal provides access to documentation on the use of
the Portal (i.e., a user guide, or similar).
}
  
 \newpage 
\subsection{[LVV-9966] DMS-PRTL-REQ-0126-V-01: System-Busy Indication\_1 }\label{lvv-9966}

\begin{longtable}{cccc}
\hline
\textbf{Jira Link} & \textbf{Assignee} & \textbf{Status} & \textbf{Test Cases}\\ \hline
\href{https://jira.lsstcorp.org/browse/LVV-9966}{LVV-9966} &
Gregory Dubois-Felsmann & Not Covered &
\begin{tabular}{c}
LVV-T761 \\
\end{tabular}
\\
\hline
\end{longtable}

\textbf{Verification Element Description:} \\
Undefined

{\footnotesize
\begin{longtable}{p{2.5cm}p{13.5cm}}
\hline
\multicolumn{2}{c}{\textbf{Requirement Details}}\\ \hline
Requirement ID & DMS-PRTL-REQ-0126 \\ \cdashline{1-2}
Requirement Description &
\begin{minipage}[]{13cm}
The Portal aspect shall provide a means to inform users when elements of
the system are unavailable.
\end{minipage}
\\ \cdashline{1-2}
Requirement Discussion &
\begin{minipage}[]{13cm}
This might be due to maintenance or to excessive load.
\end{minipage}
\\ \cdashline{1-2}
Requirement Priority &  \\ \cdashline{1-2}
Upper Level Requirement &
\begin{tabular}{cl}
\end{tabular}
\\ \hline
\end{longtable}
}


\subsubsection{Test Cases Summary}
\begin{longtable}{p{3cm}p{2.5cm}p{2.5cm}p{3cm}p{4cm}}
\toprule
\href{https://jira.lsstcorp.org/secure/Tests.jspa\#/testCase/LVV-T761}{LVV-T761} & \multicolumn{4}{p{12cm}}{ Verify implementation of system-busy notification } \\ \hline
\textbf{Owner} & \textbf{Status} & \textbf{Version} & \textbf{Critical Event} & \textbf{Verification Type} \\ \hline
Jeffrey Carlin & Draft & 1 & false & Inspection \\ \hline
\end{longtable}
{\scriptsize
\textbf{Objective:}\\
Verify that the Portal provides a means to inform users when the
elements of the system are unavailable due to maintenance or excessive
load.
}
  
 \newpage 
\subsection{[LVV-9967] DMS-PRTL-REQ-0125-V-01: Tolerance of Production Database Changes\_1 }\label{lvv-9967}

\begin{longtable}{cccc}
\hline
\textbf{Jira Link} & \textbf{Assignee} & \textbf{Status} & \textbf{Test Cases}\\ \hline
\href{https://jira.lsstcorp.org/browse/LVV-9967}{LVV-9967} &
Gregory Dubois-Felsmann & Not Covered &
\begin{tabular}{c}
LVV-T760 \\
\end{tabular}
\\
\hline
\end{longtable}

\textbf{Verification Element Description:} \\
Undefined

{\footnotesize
\begin{longtable}{p{2.5cm}p{13.5cm}}
\hline
\multicolumn{2}{c}{\textbf{Requirement Details}}\\ \hline
Requirement ID & DMS-PRTL-REQ-0125 \\ \cdashline{1-2}
Requirement Description &
\begin{minipage}[]{13cm}
The Portal aspect shall be designed to facilitate accommodation of
database expansion and changes and metadata extension and changes
associated with the evolution of the Level 1 data, Level 2 data
releases, and other planned data sources.
\end{minipage}
\\ \cdashline{1-2}
Requirement Discussion &
\begin{minipage}[]{13cm}
The LSP needs to accommodate the database changes associated with Level
1 and Level 2 updates; this also has implications for DAX - see the
``Discovery and Reflection APIs'' requirement - and for the availability
of the data as they are released with the processing (nightly and
annually). This requires stability for the APIs (web and client), and
usability of the Portal across releases (and for multiple releases).
\end{minipage}
\\ \cdashline{1-2}
Requirement Priority &  \\ \cdashline{1-2}
Upper Level Requirement &
\begin{tabular}{cl}
\end{tabular}
\\ \hline
\end{longtable}
}


\subsubsection{Test Cases Summary}
\begin{longtable}{p{3cm}p{2.5cm}p{2.5cm}p{3cm}p{4cm}}
\toprule
\href{https://jira.lsstcorp.org/secure/Tests.jspa\#/testCase/LVV-T760}{LVV-T760} & \multicolumn{4}{p{12cm}}{ Verify tolerance of database changes } \\ \hline
\textbf{Owner} & \textbf{Status} & \textbf{Version} & \textbf{Critical Event} & \textbf{Verification Type} \\ \hline
Jeffrey Carlin & Draft & 1 & false & Inspection \\ \hline
\end{longtable}
{\scriptsize
\textbf{Objective:}\\
Verify that the Portal aspect facilitates accommodation of database
expansion and changes and metadata extension and changes associated with
the evolution of the Level 1 data, Level 2 data releases, and other
planned data sources.
}
  
 \newpage 
\subsection{[LVV-9968] DMS-NB-REQ-0010-V-01: Common Astronomy Package Availability\_1 }\label{lvv-9968}

\begin{longtable}{cccc}
\hline
\textbf{Jira Link} & \textbf{Assignee} & \textbf{Status} & \textbf{Test Cases}\\ \hline
\href{https://jira.lsstcorp.org/browse/LVV-9968}{LVV-9968} &
Gregory Dubois-Felsmann & Not Covered &
\begin{tabular}{c}
LVV-T767 \\
\end{tabular}
\\
\hline
\end{longtable}

\textbf{Verification Element Description:} \\
Undefined

{\footnotesize
\begin{longtable}{p{2.5cm}p{13.5cm}}
\hline
\multicolumn{2}{c}{\textbf{Requirement Details}}\\ \hline
Requirement ID & DMS-NB-REQ-0010 \\ \cdashline{1-2}
Requirement Description &
\begin{minipage}[]{13cm}
The Notebook Aspect shall provide select standard astronomy and data
analysis packages in the interactive environments.
\end{minipage}
\\ \cdashline{1-2}
Requirement Discussion &
\begin{minipage}[]{13cm}
These may include, for example, astropy, pandas, scipy, scikit-learn,
matplotlib, bokeh, and seaborn.
\end{minipage}
\\ \cdashline{1-2}
Requirement Priority &  \\ \cdashline{1-2}
Upper Level Requirement &
\begin{tabular}{cl}
\end{tabular}
\\ \hline
\end{longtable}
}


\subsubsection{Test Cases Summary}
\begin{longtable}{p{3cm}p{2.5cm}p{2.5cm}p{3cm}p{4cm}}
\toprule
\href{https://jira.lsstcorp.org/secure/Tests.jspa\#/testCase/LVV-T767}{LVV-T767} & \multicolumn{4}{p{12cm}}{ Verify availability of standard astronomy software } \\ \hline
\textbf{Owner} & \textbf{Status} & \textbf{Version} & \textbf{Critical Event} & \textbf{Verification Type} \\ \hline
Jeffrey Carlin & Draft & 1 & false & Inspection \\ \hline
\end{longtable}
{\scriptsize
\textbf{Objective:}\\
Verify that the Notebook Aspect provides select standard astronomy
packages in the interactive environments. These may include, for
example, Astropy and S-Extractor.
}
  
 \newpage 
\subsection{[LVV-9969] DMS-NB-REQ-0009-V-01: Data Access Middleware Availability\_1 }\label{lvv-9969}

\begin{longtable}{cccc}
\hline
\textbf{Jira Link} & \textbf{Assignee} & \textbf{Status} & \textbf{Test Cases}\\ \hline
\href{https://jira.lsstcorp.org/browse/LVV-9969}{LVV-9969} &
Gregory Dubois-Felsmann & Not Covered &
\begin{tabular}{c}
LVV-T766 \\
\end{tabular}
\\
\hline
\end{longtable}

\textbf{Verification Element Description:} \\
Undefined

{\footnotesize
\begin{longtable}{p{2.5cm}p{13.5cm}}
\hline
\multicolumn{2}{c}{\textbf{Requirement Details}}\\ \hline
Requirement ID & DMS-NB-REQ-0009 \\ \cdashline{1-2}
Requirement Description &
\begin{minipage}[]{13cm}
Users of the Notebook Aspect shall be able to make use of the LSST
Python I/O middleware layer to perform data discovery, data access and
any other supported functions (e.g., provenance information).
\end{minipage}
\\ \cdashline{1-2}
Requirement Discussion &
\begin{minipage}[]{13cm}
Notably, the Data Butler is available in the Notebook Python
environment, with full access to all authorized data products available
on that instance of the Science Platform.
\end{minipage}
\\ \cdashline{1-2}
Requirement Priority &  \\ \cdashline{1-2}
Upper Level Requirement &
\begin{tabular}{cl}
\end{tabular}
\\ \hline
\end{longtable}
}


\subsubsection{Test Cases Summary}
\begin{longtable}{p{3cm}p{2.5cm}p{2.5cm}p{3cm}p{4cm}}
\toprule
\href{https://jira.lsstcorp.org/secure/Tests.jspa\#/testCase/LVV-T766}{LVV-T766} & \multicolumn{4}{p{12cm}}{ Verify availability of data access middleware } \\ \hline
\textbf{Owner} & \textbf{Status} & \textbf{Version} & \textbf{Critical Event} & \textbf{Verification Type} \\ \hline
Jeffrey Carlin & Draft & 1 & false & Inspection \\ \hline
\end{longtable}
{\scriptsize
\textbf{Objective:}\\
Verify that users of the Notebook Aspect are able to make use of the
LSST Python I/O middleware layer to perform data discovery, data access
and any other supported functions (e.g., provenance information).
Notably, the Data Butler is available in the Notebook Python
environment, with full access to all authorized data products available
on that instance of the Science Platform.
}
  
 \newpage 
\subsection{[LVV-9970] DMS-NB-REQ-0014-V-01: Documentation\_1 }\label{lvv-9970}

\begin{longtable}{cccc}
\hline
\textbf{Jira Link} & \textbf{Assignee} & \textbf{Status} & \textbf{Test Cases}\\ \hline
\href{https://jira.lsstcorp.org/browse/LVV-9970}{LVV-9970} &
Gregory Dubois-Felsmann & Not Covered &
\begin{tabular}{c}
LVV-T771 \\
\end{tabular}
\\
\hline
\end{longtable}

\textbf{Verification Element Description:} \\
Undefined

{\footnotesize
\begin{longtable}{p{2.5cm}p{13.5cm}}
\hline
\multicolumn{2}{c}{\textbf{Requirement Details}}\\ \hline
Requirement ID & DMS-NB-REQ-0014 \\ \cdashline{1-2}
Requirement Description &
\begin{minipage}[]{13cm}
The Notebook Aspect shall provide documentation of each of the
constituent features as well as tutorial notebooks demonstrating the use
of the Aspect.
\end{minipage}
\\ \cdashline{1-2}
Requirement Priority &  \\ \cdashline{1-2}
Upper Level Requirement &
\begin{tabular}{cl}
\end{tabular}
\\ \hline
\end{longtable}
}


\subsubsection{Test Cases Summary}
\begin{longtable}{p{3cm}p{2.5cm}p{2.5cm}p{3cm}p{4cm}}
\toprule
\href{https://jira.lsstcorp.org/secure/Tests.jspa\#/testCase/LVV-T771}{LVV-T771} & \multicolumn{4}{p{12cm}}{ Verify availability of Notebook aspect documentation } \\ \hline
\textbf{Owner} & \textbf{Status} & \textbf{Version} & \textbf{Critical Event} & \textbf{Verification Type} \\ \hline
Jeffrey Carlin & Draft & 1 & false & Inspection \\ \hline
\end{longtable}
{\scriptsize
\textbf{Objective:}\\
Verify that the Notebook Aspect provides documentation of each of the
constituent features as well as tutorial notebooks demonstrating the use
of the Aspect.~
}
  
 \newpage 
\subsection{[LVV-9971] DMS-NB-REQ-0005-V-01: Interactive Python Environment\_1 }\label{lvv-9971}

\begin{longtable}{cccc}
\hline
\textbf{Jira Link} & \textbf{Assignee} & \textbf{Status} & \textbf{Test Cases}\\ \hline
\href{https://jira.lsstcorp.org/browse/LVV-9971}{LVV-9971} &
Gregory Dubois-Felsmann & Not Covered &
\begin{tabular}{c}
LVV-T762 \\
LVV-T1436 \\
\end{tabular}
\\
\hline
\end{longtable}

\textbf{Verification Element Description:} \\
Undefined

{\footnotesize
\begin{longtable}{p{2.5cm}p{13.5cm}}
\hline
\multicolumn{2}{c}{\textbf{Requirement Details}}\\ \hline
Requirement ID & DMS-NB-REQ-0005 \\ \cdashline{1-2}
Requirement Description &
\begin{minipage}[]{13cm}
The Notebook Aspect shall provide an interactive Python environment
through both a notebook interface and via a Python interactive
interpreter.
\end{minipage}
\\ \cdashline{1-2}
Requirement Priority &  \\ \cdashline{1-2}
Upper Level Requirement &
\begin{tabular}{cl}
\end{tabular}
\\ \hline
\end{longtable}
}


\subsubsection{Test Cases Summary}
\begin{longtable}{p{3cm}p{2.5cm}p{2.5cm}p{3cm}p{4cm}}
\toprule
\href{https://jira.lsstcorp.org/secure/Tests.jspa\#/testCase/LVV-T762}{LVV-T762} & \multicolumn{4}{p{12cm}}{ Verify availability of interactive Python environment } \\ \hline
\textbf{Owner} & \textbf{Status} & \textbf{Version} & \textbf{Critical Event} & \textbf{Verification Type} \\ \hline
Simon Krughoff & Draft & 1 & false & Inspection \\ \hline
\end{longtable}
{\scriptsize
\textbf{Objective:}\\
Verify that the Notebook aspect provides an interactive Python
environment through both a notebook interface and via a Python
interactive interpreter.
}
\begin{longtable}{p{3cm}p{2.5cm}p{2.5cm}p{3cm}p{4cm}}
\toprule
\href{https://jira.lsstcorp.org/secure/Tests.jspa\#/testCase/LVV-T1436}{LVV-T1436} & \multicolumn{4}{p{12cm}}{ LDM-503-10a: Notebook Aspect tests for LSP with Authentication and TAP
milestone } \\ \hline
\textbf{Owner} & \textbf{Status} & \textbf{Version} & \textbf{Critical Event} & \textbf{Verification Type} \\ \hline
Gregory Dubois-Felsmann & Defined & 1 & false & Test \\ \hline
\end{longtable}
{\scriptsize
\textbf{Objective:}\\
This test case verifies that the Notebook Aspect of the Science Platform
is accessible to authorized users through a login process, and that TAP
searches can be performed from Python code in the Notebook
Aspect.\\[2\baselineskip]In so doing and in conjunction with the other
LDM-503-10a test cases collected under LVV-P48, it addresses all or part
of the following requirements:

\begin{itemize}
\tightlist
\item
  DMS-LSP-REQ-0003, DMS-LSP-REQ-0005, DMS-LSP-REQ-0006,
  DMS-LSP-REQ-0020, DMS-LSP-REQ-0022, DMS-LSP-REQ-0023, DMS-LSP-REQ-0024
\item
  DMS-NB-REQ-0001, DMS-NB-REQ-0002, DMS-NB-REQ-0005, DMS-NB-REQ-0006,
  DMS-NB-REQ-0013, DMS-NB-REQ-0017, and DMS-NB-REQ-0029, primarily
\end{itemize}

Note this test was not designed to perform a full verification of the
above requirements, but rather to demonstrate having reached a certain
level of partial capability during construction.
}
  
 \newpage 
\subsection{[LVV-9972] DMS-NB-REQ-0015-V-01: New-User Onboarding\_1 }\label{lvv-9972}

\begin{longtable}{cccc}
\hline
\textbf{Jira Link} & \textbf{Assignee} & \textbf{Status} & \textbf{Test Cases}\\ \hline
\href{https://jira.lsstcorp.org/browse/LVV-9972}{LVV-9972} &
Gregory Dubois-Felsmann & Not Covered &
\begin{tabular}{c}
LVV-T772 \\
\end{tabular}
\\
\hline
\end{longtable}

\textbf{Verification Element Description:} \\
Undefined

{\footnotesize
\begin{longtable}{p{2.5cm}p{13.5cm}}
\hline
\multicolumn{2}{c}{\textbf{Requirement Details}}\\ \hline
Requirement ID & DMS-NB-REQ-0015 \\ \cdashline{1-2}
Requirement Description &
\begin{minipage}[]{13cm}
The Notebook Aspect shall provide clear documentation on how to obtain
credentials for accessing the Notebook Aspect.
\end{minipage}
\\ \cdashline{1-2}
Requirement Priority &  \\ \cdashline{1-2}
Upper Level Requirement &
\begin{tabular}{cl}
\end{tabular}
\\ \hline
\end{longtable}
}


\subsubsection{Test Cases Summary}
\begin{longtable}{p{3cm}p{2.5cm}p{2.5cm}p{3cm}p{4cm}}
\toprule
\href{https://jira.lsstcorp.org/secure/Tests.jspa\#/testCase/LVV-T772}{LVV-T772} & \multicolumn{4}{p{12cm}}{ Verify new-user onboarding } \\ \hline
\textbf{Owner} & \textbf{Status} & \textbf{Version} & \textbf{Critical Event} & \textbf{Verification Type} \\ \hline
Jeffrey Carlin & Draft & 1 & false & Inspection \\ \hline
\end{longtable}
{\scriptsize
\textbf{Objective:}\\
Verify that the Notebook Aspect provides clear documentation on how to
obtain credentials for accessing the Notebook Aspect.
}
  
 \newpage 
\subsection{[LVV-9973] DMS-NB-REQ-0013-V-01: Persistent User Home File Space\_1 }\label{lvv-9973}

\begin{longtable}{cccc}
\hline
\textbf{Jira Link} & \textbf{Assignee} & \textbf{Status} & \textbf{Test Cases}\\ \hline
\href{https://jira.lsstcorp.org/browse/LVV-9973}{LVV-9973} &
Gregory Dubois-Felsmann & Not Covered &
\begin{tabular}{c}
LVV-T770 \\
LVV-T1436 \\
\end{tabular}
\\
\hline
\end{longtable}

\textbf{Verification Element Description:} \\
Undefined

{\footnotesize
\begin{longtable}{p{2.5cm}p{13.5cm}}
\hline
\multicolumn{2}{c}{\textbf{Requirement Details}}\\ \hline
Requirement ID & DMS-NB-REQ-0013 \\ \cdashline{1-2}
Requirement Description &
\begin{minipage}[]{13cm}
The Notebook Aspect shall provide a persistent home space such that per
user configuration survives shutdown and restart of the environment.
\end{minipage}
\\ \cdashline{1-2}
Requirement Discussion &
\begin{minipage}[]{13cm}
This space appears as a home directory from Python and in the Unix shell
environment. This includes things like .bashrc, .pythonrc, and user
installed python libs.
\end{minipage}
\\ \cdashline{1-2}
Requirement Priority &  \\ \cdashline{1-2}
Upper Level Requirement &
\begin{tabular}{cl}
\end{tabular}
\\ \hline
\end{longtable}
}


\subsubsection{Test Cases Summary}
\begin{longtable}{p{3cm}p{2.5cm}p{2.5cm}p{3cm}p{4cm}}
\toprule
\href{https://jira.lsstcorp.org/secure/Tests.jspa\#/testCase/LVV-T770}{LVV-T770} & \multicolumn{4}{p{12cm}}{ Verify availability of persistent user home file space } \\ \hline
\textbf{Owner} & \textbf{Status} & \textbf{Version} & \textbf{Critical Event} & \textbf{Verification Type} \\ \hline
Simon Krughoff & Draft & 1 & false & Inspection \\ \hline
\end{longtable}
{\scriptsize
\textbf{Objective:}\\
Verify that the Notebook Aspect provides a persistent home space such
that per user configuration survives shutdown and restart of the
environment. This space appears as a home directory from Python and in
the Unix shell environment. This includes things like .bashrc,
.pythonrc, and user installed python libs.
}
\begin{longtable}{p{3cm}p{2.5cm}p{2.5cm}p{3cm}p{4cm}}
\toprule
\href{https://jira.lsstcorp.org/secure/Tests.jspa\#/testCase/LVV-T1436}{LVV-T1436} & \multicolumn{4}{p{12cm}}{ LDM-503-10a: Notebook Aspect tests for LSP with Authentication and TAP
milestone } \\ \hline
\textbf{Owner} & \textbf{Status} & \textbf{Version} & \textbf{Critical Event} & \textbf{Verification Type} \\ \hline
Gregory Dubois-Felsmann & Defined & 1 & false & Test \\ \hline
\end{longtable}
{\scriptsize
\textbf{Objective:}\\
This test case verifies that the Notebook Aspect of the Science Platform
is accessible to authorized users through a login process, and that TAP
searches can be performed from Python code in the Notebook
Aspect.\\[2\baselineskip]In so doing and in conjunction with the other
LDM-503-10a test cases collected under LVV-P48, it addresses all or part
of the following requirements:

\begin{itemize}
\tightlist
\item
  DMS-LSP-REQ-0003, DMS-LSP-REQ-0005, DMS-LSP-REQ-0006,
  DMS-LSP-REQ-0020, DMS-LSP-REQ-0022, DMS-LSP-REQ-0023, DMS-LSP-REQ-0024
\item
  DMS-NB-REQ-0001, DMS-NB-REQ-0002, DMS-NB-REQ-0005, DMS-NB-REQ-0006,
  DMS-NB-REQ-0013, DMS-NB-REQ-0017, and DMS-NB-REQ-0029, primarily
\end{itemize}

Note this test was not designed to perform a full verification of the
above requirements, but rather to demonstrate having reached a certain
level of partial capability during construction.
}
  
 \newpage 
\subsection{[LVV-9974] DMS-NB-REQ-0007-V-01: Pre-installed Containerized Software Releases\_1 }\label{lvv-9974}

\begin{longtable}{cccc}
\hline
\textbf{Jira Link} & \textbf{Assignee} & \textbf{Status} & \textbf{Test Cases}\\ \hline
\href{https://jira.lsstcorp.org/browse/LVV-9974}{LVV-9974} &
Gregory Dubois-Felsmann & Not Covered &
\begin{tabular}{c}
LVV-T764 \\
\end{tabular}
\\
\hline
\end{longtable}

\textbf{Verification Element Description:} \\
Undefined

{\footnotesize
\begin{longtable}{p{2.5cm}p{13.5cm}}
\hline
\multicolumn{2}{c}{\textbf{Requirement Details}}\\ \hline
Requirement ID & DMS-NB-REQ-0007 \\ \cdashline{1-2}
Requirement Description &
\begin{minipage}[]{13cm}
Users of the Notebook Aspect shall be able to chose from a curated list
of pre-built containers (including version of LSST stack) for their
notebooks (and any other provided interactive environment) to execute
in.
\end{minipage}
\\ \cdashline{1-2}
Requirement Discussion &
\begin{minipage}[]{13cm}
The use of the Docker container technology is anticipated.
\end{minipage}
\\ \cdashline{1-2}
Requirement Priority &  \\ \cdashline{1-2}
Upper Level Requirement &
\begin{tabular}{cl}
\end{tabular}
\\ \hline
\end{longtable}
}


\subsubsection{Test Cases Summary}
\begin{longtable}{p{3cm}p{2.5cm}p{2.5cm}p{3cm}p{4cm}}
\toprule
\href{https://jira.lsstcorp.org/secure/Tests.jspa\#/testCase/LVV-T764}{LVV-T764} & \multicolumn{4}{p{12cm}}{ Verify availability of containerized software releases } \\ \hline
\textbf{Owner} & \textbf{Status} & \textbf{Version} & \textbf{Critical Event} & \textbf{Verification Type} \\ \hline
Simon Krughoff & Draft & 1 & false & Inspection \\ \hline
\end{longtable}
{\scriptsize
\textbf{Objective:}\\
Verify that users of the Notebook aspect are able to choose from a
curated list of pre-built containers (including version of LSST stack)
for their notebooks (and any other provided interactive environment) to
execute in.
}
  
 \newpage 
\subsection{[LVV-9975] DMS-NB-REQ-0008-V-01: Release Deployment Latency\_1 }\label{lvv-9975}

\begin{longtable}{cccc}
\hline
\textbf{Jira Link} & \textbf{Assignee} & \textbf{Status} & \textbf{Test Cases}\\ \hline
\href{https://jira.lsstcorp.org/browse/LVV-9975}{LVV-9975} &
Gregory Dubois-Felsmann & Not Covered &
\begin{tabular}{c}
LVV-T765 \\
\end{tabular}
\\
\hline
\end{longtable}

\textbf{Verification Element Description:} \\
Undefined

{\footnotesize
\begin{longtable}{p{2.5cm}p{13.5cm}}
\hline
\multicolumn{2}{c}{\textbf{Requirement Details}}\\ \hline
Requirement ID & DMS-NB-REQ-0008 \\ \cdashline{1-2}
Requirement Description &
\begin{minipage}[]{13cm}
It shall be possible to add a new environment (with a new version of the
LSST stack) to the curated list of available execution environments in
less than four hours.
\end{minipage}
\\ \cdashline{1-2}
Requirement Priority &  \\ \cdashline{1-2}
Upper Level Requirement &
\begin{tabular}{cl}
\end{tabular}
\\ \hline
\end{longtable}
}


\subsubsection{Test Cases Summary}
\begin{longtable}{p{3cm}p{2.5cm}p{2.5cm}p{3cm}p{4cm}}
\toprule
\href{https://jira.lsstcorp.org/secure/Tests.jspa\#/testCase/LVV-T765}{LVV-T765} & \multicolumn{4}{p{12cm}}{ Verify latency of release deployment } \\ \hline
\textbf{Owner} & \textbf{Status} & \textbf{Version} & \textbf{Critical Event} & \textbf{Verification Type} \\ \hline
Jeffrey Carlin & Draft & 1 & false & Demonstration \\ \hline
\end{longtable}
{\scriptsize
\textbf{Objective:}\\
Verify that it is possible to add a new environment (with a new version
of the LSST stack) to the curated list of available execution
environments in less than four hours.
}
  
 \newpage 
\subsection{[LVV-9976] DMS-NB-REQ-0006-V-01: Unix Shell Access\_1 }\label{lvv-9976}

\begin{longtable}{cccc}
\hline
\textbf{Jira Link} & \textbf{Assignee} & \textbf{Status} & \textbf{Test Cases}\\ \hline
\href{https://jira.lsstcorp.org/browse/LVV-9976}{LVV-9976} &
Gregory Dubois-Felsmann & Not Covered &
\begin{tabular}{c}
LVV-T763 \\
LVV-T1436 \\
\end{tabular}
\\
\hline
\end{longtable}

\textbf{Verification Element Description:} \\
Undefined

{\footnotesize
\begin{longtable}{p{2.5cm}p{13.5cm}}
\hline
\multicolumn{2}{c}{\textbf{Requirement Details}}\\ \hline
Requirement ID & DMS-NB-REQ-0006 \\ \cdashline{1-2}
Requirement Description &
\begin{minipage}[]{13cm}
The Notebook Aspect shall provide command line access to a Unix shell
with the same environment as DMS-NB-REQ-0005.
\end{minipage}
\\ \cdashline{1-2}
Requirement Priority &  \\ \cdashline{1-2}
Upper Level Requirement &
\begin{tabular}{cl}
\end{tabular}
\\ \hline
\end{longtable}
}


\subsubsection{Test Cases Summary}
\begin{longtable}{p{3cm}p{2.5cm}p{2.5cm}p{3cm}p{4cm}}
\toprule
\href{https://jira.lsstcorp.org/secure/Tests.jspa\#/testCase/LVV-T763}{LVV-T763} & \multicolumn{4}{p{12cm}}{ Verify availability of Unix shell access } \\ \hline
\textbf{Owner} & \textbf{Status} & \textbf{Version} & \textbf{Critical Event} & \textbf{Verification Type} \\ \hline
Simon Krughoff & Draft & 1 & false & Inspection \\ \hline
\end{longtable}
{\scriptsize
\textbf{Objective:}\\
Verify that the Notebook aspect provides command-line access to a Unix
shell with the same environment as the interactive Python environment.
}
\begin{longtable}{p{3cm}p{2.5cm}p{2.5cm}p{3cm}p{4cm}}
\toprule
\href{https://jira.lsstcorp.org/secure/Tests.jspa\#/testCase/LVV-T1436}{LVV-T1436} & \multicolumn{4}{p{12cm}}{ LDM-503-10a: Notebook Aspect tests for LSP with Authentication and TAP
milestone } \\ \hline
\textbf{Owner} & \textbf{Status} & \textbf{Version} & \textbf{Critical Event} & \textbf{Verification Type} \\ \hline
Gregory Dubois-Felsmann & Defined & 1 & false & Test \\ \hline
\end{longtable}
{\scriptsize
\textbf{Objective:}\\
This test case verifies that the Notebook Aspect of the Science Platform
is accessible to authorized users through a login process, and that TAP
searches can be performed from Python code in the Notebook
Aspect.\\[2\baselineskip]In so doing and in conjunction with the other
LDM-503-10a test cases collected under LVV-P48, it addresses all or part
of the following requirements:

\begin{itemize}
\tightlist
\item
  DMS-LSP-REQ-0003, DMS-LSP-REQ-0005, DMS-LSP-REQ-0006,
  DMS-LSP-REQ-0020, DMS-LSP-REQ-0022, DMS-LSP-REQ-0023, DMS-LSP-REQ-0024
\item
  DMS-NB-REQ-0001, DMS-NB-REQ-0002, DMS-NB-REQ-0005, DMS-NB-REQ-0006,
  DMS-NB-REQ-0013, DMS-NB-REQ-0017, and DMS-NB-REQ-0029, primarily
\end{itemize}

Note this test was not designed to perform a full verification of the
above requirements, but rather to demonstrate having reached a certain
level of partial capability during construction.
}
  
 \newpage 
\subsection{[LVV-9977] DMS-NB-REQ-0012-V-01: User Development Environment\_1 }\label{lvv-9977}

\begin{longtable}{cccc}
\hline
\textbf{Jira Link} & \textbf{Assignee} & \textbf{Status} & \textbf{Test Cases}\\ \hline
\href{https://jira.lsstcorp.org/browse/LVV-9977}{LVV-9977} &
Gregory Dubois-Felsmann & Not Covered &
\begin{tabular}{c}
LVV-T769 \\
\end{tabular}
\\
\hline
\end{longtable}

\textbf{Verification Element Description:} \\
Undefined

{\footnotesize
\begin{longtable}{p{2.5cm}p{13.5cm}}
\hline
\multicolumn{2}{c}{\textbf{Requirement Details}}\\ \hline
Requirement ID & DMS-NB-REQ-0012 \\ \cdashline{1-2}
Requirement Description &
\begin{minipage}[]{13cm}
The Notebook Aspect environment shall permit a user to edit and build
their own version of any LSST science pipeline package in their
container.
\end{minipage}
\\ \cdashline{1-2}
Requirement Discussion &
\begin{minipage}[]{13cm}
This implies the availability of both a C++ and a Python development
environment.
\end{minipage}
\\ \cdashline{1-2}
Requirement Priority &  \\ \cdashline{1-2}
Upper Level Requirement &
\begin{tabular}{cl}
\end{tabular}
\\ \hline
\end{longtable}
}


\subsubsection{Test Cases Summary}
\begin{longtable}{p{3cm}p{2.5cm}p{2.5cm}p{3cm}p{4cm}}
\toprule
\href{https://jira.lsstcorp.org/secure/Tests.jspa\#/testCase/LVV-T769}{LVV-T769} & \multicolumn{4}{p{12cm}}{ Verify availability of user development environment } \\ \hline
\textbf{Owner} & \textbf{Status} & \textbf{Version} & \textbf{Critical Event} & \textbf{Verification Type} \\ \hline
Jeffrey Carlin & Draft & 1 & false & Inspection \\ \hline
\end{longtable}
{\scriptsize
\textbf{Objective:}\\
Verify that the Notebook Aspect environment permits a user to edit and
build their own version of any LSST science pipeline package in their
container. This implies the availability of both a C++ and a Python
development environment.~
}
  
 \newpage 
\subsection{[LVV-9978] DMS-NB-REQ-0011-V-01: User Package Installation\_1 }\label{lvv-9978}

\begin{longtable}{cccc}
\hline
\textbf{Jira Link} & \textbf{Assignee} & \textbf{Status} & \textbf{Test Cases}\\ \hline
\href{https://jira.lsstcorp.org/browse/LVV-9978}{LVV-9978} &
Gregory Dubois-Felsmann & Not Covered &
\begin{tabular}{c}
LVV-T768 \\
\end{tabular}
\\
\hline
\end{longtable}

\textbf{Verification Element Description:} \\
Undefined

{\footnotesize
\begin{longtable}{p{2.5cm}p{13.5cm}}
\hline
\multicolumn{2}{c}{\textbf{Requirement Details}}\\ \hline
Requirement ID & DMS-NB-REQ-0011 \\ \cdashline{1-2}
Requirement Description &
\begin{minipage}[]{13cm}
The Notebook Aspect shall have a process that allows users to add new
packages to their environment
\end{minipage}
\\ \cdashline{1-2}
Requirement Discussion &
\begin{minipage}[]{13cm}
It is intended that operations like ``pip install'' will be usable.
\end{minipage}
\\ \cdashline{1-2}
Requirement Priority &  \\ \cdashline{1-2}
Upper Level Requirement &
\begin{tabular}{cl}
\end{tabular}
\\ \hline
\end{longtable}
}


\subsubsection{Test Cases Summary}
\begin{longtable}{p{3cm}p{2.5cm}p{2.5cm}p{3cm}p{4cm}}
\toprule
\href{https://jira.lsstcorp.org/secure/Tests.jspa\#/testCase/LVV-T768}{LVV-T768} & \multicolumn{4}{p{12cm}}{ Verify availability of user package installation } \\ \hline
\textbf{Owner} & \textbf{Status} & \textbf{Version} & \textbf{Critical Event} & \textbf{Verification Type} \\ \hline
Simon Krughoff & Draft & 1 & false & Inspection \\ \hline
\end{longtable}
{\scriptsize
\textbf{Objective:}\\
Verify that the Notebook Aspect has a process that allows users to add
new packages to their environment It is intended that operations like
``pip install'' will be usable.
}
  
 \newpage 
\subsection{[LVV-9979] DMS-NB-REQ-0023-V-01: Access to All Data Products\_1 }\label{lvv-9979}

\begin{longtable}{cccc}
\hline
\textbf{Jira Link} & \textbf{Assignee} & \textbf{Status} & \textbf{Test Cases}\\ \hline
\href{https://jira.lsstcorp.org/browse/LVV-9979}{LVV-9979} &
Gregory Dubois-Felsmann & Not Covered &
\begin{tabular}{c}
LVV-T780 \\
\end{tabular}
\\
\hline
\end{longtable}

\textbf{Verification Element Description:} \\
Undefined

{\footnotesize
\begin{longtable}{p{2.5cm}p{13.5cm}}
\hline
\multicolumn{2}{c}{\textbf{Requirement Details}}\\ \hline
Requirement ID & DMS-NB-REQ-0023 \\ \cdashline{1-2}
Requirement Description &
\begin{minipage}[]{13cm}
An authorized user of the Notebook Aspect shall be able to access the
Transformed Engineering and Facilities Database (EFD) and and all other
LSST released data products.
\end{minipage}
\\ \cdashline{1-2}
Requirement Priority &  \\ \cdashline{1-2}
Upper Level Requirement &
\begin{tabular}{cl}
\end{tabular}
\\ \hline
\end{longtable}
}


\subsubsection{Test Cases Summary}
\begin{longtable}{p{3cm}p{2.5cm}p{2.5cm}p{3cm}p{4cm}}
\toprule
\href{https://jira.lsstcorp.org/secure/Tests.jspa\#/testCase/LVV-T780}{LVV-T780} & \multicolumn{4}{p{12cm}}{ Verify access to all data products from Notebook aspect } \\ \hline
\textbf{Owner} & \textbf{Status} & \textbf{Version} & \textbf{Critical Event} & \textbf{Verification Type} \\ \hline
Jeffrey Carlin & Draft & 1 & false & Inspection \\ \hline
\end{longtable}
{\scriptsize
\textbf{Objective:}\\
Verify that an authorized user of the Notebook Aspect is able to access
the reformatted Engineering and Facilities Database (EFD) and and all
other LSST released data products.
}
  
 \newpage 
\subsection{[LVV-9980] DMS-NB-REQ-0017-V-01: Access to the API and Portal Aspects\_1 }\label{lvv-9980}

\begin{longtable}{cccc}
\hline
\textbf{Jira Link} & \textbf{Assignee} & \textbf{Status} & \textbf{Test Cases}\\ \hline
\href{https://jira.lsstcorp.org/browse/LVV-9980}{LVV-9980} &
Gregory Dubois-Felsmann & Not Covered &
\begin{tabular}{c}
LVV-T774 \\
LVV-T1436 \\
\end{tabular}
\\
\hline
\end{longtable}

\textbf{Verification Element Description:} \\
Undefined

{\footnotesize
\begin{longtable}{p{2.5cm}p{13.5cm}}
\hline
\multicolumn{2}{c}{\textbf{Requirement Details}}\\ \hline
Requirement ID & DMS-NB-REQ-0017 \\ \cdashline{1-2}
Requirement Description &
\begin{minipage}[]{13cm}
The Notebook Aspect shall be able to utilise the data access services
provided by other Aspects.
\end{minipage}
\\ \cdashline{1-2}
Requirement Discussion &
\begin{minipage}[]{13cm}
In particular, a Notebook user can use standard VO services to access
LSST Data Releases.
\end{minipage}
\\ \cdashline{1-2}
Requirement Priority &  \\ \cdashline{1-2}
Upper Level Requirement &
\begin{tabular}{cl}
\end{tabular}
\\ \hline
\end{longtable}
}


\subsubsection{Test Cases Summary}
\begin{longtable}{p{3cm}p{2.5cm}p{2.5cm}p{3cm}p{4cm}}
\toprule
\href{https://jira.lsstcorp.org/secure/Tests.jspa\#/testCase/LVV-T774}{LVV-T774} & \multicolumn{4}{p{12cm}}{ Verify API and Portal aspects accessible from Notebook } \\ \hline
\textbf{Owner} & \textbf{Status} & \textbf{Version} & \textbf{Critical Event} & \textbf{Verification Type} \\ \hline
Jeffrey Carlin & Draft & 1 & false & Inspection \\ \hline
\end{longtable}
{\scriptsize
\textbf{Objective:}\\
Verify that the Notebook Aspect is able to utilise the data access
services provided by other Aspects. In particular, a Notebook user can
use standard VO services to access LSST Data Releases.
}
\begin{longtable}{p{3cm}p{2.5cm}p{2.5cm}p{3cm}p{4cm}}
\toprule
\href{https://jira.lsstcorp.org/secure/Tests.jspa\#/testCase/LVV-T1436}{LVV-T1436} & \multicolumn{4}{p{12cm}}{ LDM-503-10a: Notebook Aspect tests for LSP with Authentication and TAP
milestone } \\ \hline
\textbf{Owner} & \textbf{Status} & \textbf{Version} & \textbf{Critical Event} & \textbf{Verification Type} \\ \hline
Gregory Dubois-Felsmann & Defined & 1 & false & Test \\ \hline
\end{longtable}
{\scriptsize
\textbf{Objective:}\\
This test case verifies that the Notebook Aspect of the Science Platform
is accessible to authorized users through a login process, and that TAP
searches can be performed from Python code in the Notebook
Aspect.\\[2\baselineskip]In so doing and in conjunction with the other
LDM-503-10a test cases collected under LVV-P48, it addresses all or part
of the following requirements:

\begin{itemize}
\tightlist
\item
  DMS-LSP-REQ-0003, DMS-LSP-REQ-0005, DMS-LSP-REQ-0006,
  DMS-LSP-REQ-0020, DMS-LSP-REQ-0022, DMS-LSP-REQ-0023, DMS-LSP-REQ-0024
\item
  DMS-NB-REQ-0001, DMS-NB-REQ-0002, DMS-NB-REQ-0005, DMS-NB-REQ-0006,
  DMS-NB-REQ-0013, DMS-NB-REQ-0017, and DMS-NB-REQ-0029, primarily
\end{itemize}

Note this test was not designed to perform a full verification of the
above requirements, but rather to demonstrate having reached a certain
level of partial capability during construction.
}
  
 \newpage 
\subsection{[LVV-9981] DMS-NB-REQ-0021-V-01: Batch System Access\_1 }\label{lvv-9981}

\begin{longtable}{cccc}
\hline
\textbf{Jira Link} & \textbf{Assignee} & \textbf{Status} & \textbf{Test Cases}\\ \hline
\href{https://jira.lsstcorp.org/browse/LVV-9981}{LVV-9981} &
Gregory Dubois-Felsmann & Not Covered &
\begin{tabular}{c}
LVV-T778 \\
\end{tabular}
\\
\hline
\end{longtable}

\textbf{Verification Element Description:} \\
Undefined

{\footnotesize
\begin{longtable}{p{2.5cm}p{13.5cm}}
\hline
\multicolumn{2}{c}{\textbf{Requirement Details}}\\ \hline
Requirement ID & DMS-NB-REQ-0021 \\ \cdashline{1-2}
Requirement Description &
\begin{minipage}[]{13cm}
The Notebook Aspect shall provide access to a batch processing system
via shell access.
\end{minipage}
\\ \cdashline{1-2}
Requirement Discussion &
\begin{minipage}[]{13cm}
This is a cluster of computers scheduled through a standard scheduler
like slurm, condor, or pbs.
\end{minipage}
\\ \cdashline{1-2}
Requirement Priority &  \\ \cdashline{1-2}
Upper Level Requirement &
\begin{tabular}{cl}
\end{tabular}
\\ \hline
\end{longtable}
}


\subsubsection{Test Cases Summary}
\begin{longtable}{p{3cm}p{2.5cm}p{2.5cm}p{3cm}p{4cm}}
\toprule
\href{https://jira.lsstcorp.org/secure/Tests.jspa\#/testCase/LVV-T778}{LVV-T778} & \multicolumn{4}{p{12cm}}{ Verify access to batch system } \\ \hline
\textbf{Owner} & \textbf{Status} & \textbf{Version} & \textbf{Critical Event} & \textbf{Verification Type} \\ \hline
Jeffrey Carlin & Draft & 1 & false & Inspection \\ \hline
\end{longtable}
{\scriptsize
\textbf{Objective:}\\
Verify that the Notebook aspect provides access to a batch processing
system via shell access.
}
  
 \newpage 
\subsection{[LVV-9982] DMS-NB-REQ-0022-V-01: Compute and Storage Quotas\_1 }\label{lvv-9982}

\begin{longtable}{cccc}
\hline
\textbf{Jira Link} & \textbf{Assignee} & \textbf{Status} & \textbf{Test Cases}\\ \hline
\href{https://jira.lsstcorp.org/browse/LVV-9982}{LVV-9982} &
Gregory Dubois-Felsmann & Not Covered &
\begin{tabular}{c}
LVV-T779 \\
\end{tabular}
\\
\hline
\end{longtable}

\textbf{Verification Element Description:} \\
Undefined

{\footnotesize
\begin{longtable}{p{2.5cm}p{13.5cm}}
\hline
\multicolumn{2}{c}{\textbf{Requirement Details}}\\ \hline
Requirement ID & DMS-NB-REQ-0022 \\ \cdashline{1-2}
Requirement Description &
\begin{minipage}[]{13cm}
The Notebook Aspect shall have a quota system for compute and storage
authorized access via an authentication system.
\end{minipage}
\\ \cdashline{1-2}
Requirement Priority &  \\ \cdashline{1-2}
Upper Level Requirement &
\begin{tabular}{cl}
\end{tabular}
\\ \hline
\end{longtable}
}


\subsubsection{Test Cases Summary}
\begin{longtable}{p{3cm}p{2.5cm}p{2.5cm}p{3cm}p{4cm}}
\toprule
\href{https://jira.lsstcorp.org/secure/Tests.jspa\#/testCase/LVV-T779}{LVV-T779} & \multicolumn{4}{p{12cm}}{ Verify implementation of quotas in Notebook aspect } \\ \hline
\textbf{Owner} & \textbf{Status} & \textbf{Version} & \textbf{Critical Event} & \textbf{Verification Type} \\ \hline
Jeffrey Carlin & Draft & 1 & false & Inspection \\ \hline
\end{longtable}
{\scriptsize
\textbf{Objective:}\\
Verify that the Notebook Aspect has a quota system for compute and
storage authorized access via an authentication system.
}
  
 \newpage 
\subsection{[LVV-9983] DMS-NB-REQ-0016-V-01: Shared File Space\_1 }\label{lvv-9983}

\begin{longtable}{cccc}
\hline
\textbf{Jira Link} & \textbf{Assignee} & \textbf{Status} & \textbf{Test Cases}\\ \hline
\href{https://jira.lsstcorp.org/browse/LVV-9983}{LVV-9983} &
Gregory Dubois-Felsmann & Not Covered &
\begin{tabular}{c}
LVV-T773 \\
\end{tabular}
\\
\hline
\end{longtable}

\textbf{Verification Element Description:} \\
Undefined

{\footnotesize
\begin{longtable}{p{2.5cm}p{13.5cm}}
\hline
\multicolumn{2}{c}{\textbf{Requirement Details}}\\ \hline
Requirement ID & DMS-NB-REQ-0016 \\ \cdashline{1-2}
Requirement Description &
\begin{minipage}[]{13cm}
The Notebook Aspect shall provide access to a shared read/write
filesystem visible to all users of an instance of the Science Platform.
\end{minipage}
\\ \cdashline{1-2}
Requirement Discussion &
\begin{minipage}[]{13cm}
In the LDF this is intended to be implemented as .\\
DMS-REQ-0340 means that permissions will be controlled on a variety of
granularity including user and group.
\end{minipage}
\\ \cdashline{1-2}
Requirement Priority &  \\ \cdashline{1-2}
Upper Level Requirement &
\begin{tabular}{cl}
\end{tabular}
\\ \hline
\end{longtable}
}


\subsubsection{Test Cases Summary}
\begin{longtable}{p{3cm}p{2.5cm}p{2.5cm}p{3cm}p{4cm}}
\toprule
\href{https://jira.lsstcorp.org/secure/Tests.jspa\#/testCase/LVV-T773}{LVV-T773} & \multicolumn{4}{p{12cm}}{ Verify availability of shared file space } \\ \hline
\textbf{Owner} & \textbf{Status} & \textbf{Version} & \textbf{Critical Event} & \textbf{Verification Type} \\ \hline
Jeffrey Carlin & Draft & 1 & false & Inspection \\ \hline
\end{longtable}
{\scriptsize
\textbf{Objective:}\\
Verify that the Notebook Aspect provides access to a shared read/write
filesystem visible to all users of an instance of the Science Platform.
}
  
 \newpage 
\subsection{[LVV-9984] DMS-NB-REQ-0020-V-01: User Database Workspace Access\_1 }\label{lvv-9984}

\begin{longtable}{cccc}
\hline
\textbf{Jira Link} & \textbf{Assignee} & \textbf{Status} & \textbf{Test Cases}\\ \hline
\href{https://jira.lsstcorp.org/browse/LVV-9984}{LVV-9984} &
Gregory Dubois-Felsmann & Not Covered &
\begin{tabular}{c}
LVV-T777 \\
\end{tabular}
\\
\hline
\end{longtable}

\textbf{Verification Element Description:} \\
Undefined

{\footnotesize
\begin{longtable}{p{2.5cm}p{13.5cm}}
\hline
\multicolumn{2}{c}{\textbf{Requirement Details}}\\ \hline
Requirement ID & DMS-NB-REQ-0020 \\ \cdashline{1-2}
Requirement Description &
\begin{minipage}[]{13cm}
Users will be able to interact with their User Database through the
Notebook Aspect to insert, delete, and control access to their tables.
\end{minipage}
\\ \cdashline{1-2}
Requirement Discussion &
\begin{minipage}[]{13cm}
This will be possible via TAP, at least, and possibly through
lower-level access.
\end{minipage}
\\ \cdashline{1-2}
Requirement Priority &  \\ \cdashline{1-2}
Upper Level Requirement &
\begin{tabular}{cl}
\end{tabular}
\\ \hline
\end{longtable}
}


\subsubsection{Test Cases Summary}
\begin{longtable}{p{3cm}p{2.5cm}p{2.5cm}p{3cm}p{4cm}}
\toprule
\href{https://jira.lsstcorp.org/secure/Tests.jspa\#/testCase/LVV-T777}{LVV-T777} & \multicolumn{4}{p{12cm}}{ Verify user database workspace access from Notebook aspect } \\ \hline
\textbf{Owner} & \textbf{Status} & \textbf{Version} & \textbf{Critical Event} & \textbf{Verification Type} \\ \hline
Jeffrey Carlin & Draft & 1 & false & Inspection \\ \hline
\end{longtable}
{\scriptsize
\textbf{Objective:}\\
Verify that users are able to interact with their User Database through
the Notebook Aspect to insert, delete, and control access to their
tables. This will be possible via TAP, at least, and possibly through
lower-level access.
}
  
 \newpage 
\subsection{[LVV-9985] DMS-NB-REQ-0018-V-01: User File Workspace Access\_1 }\label{lvv-9985}

\begin{longtable}{cccc}
\hline
\textbf{Jira Link} & \textbf{Assignee} & \textbf{Status} & \textbf{Test Cases}\\ \hline
\href{https://jira.lsstcorp.org/browse/LVV-9985}{LVV-9985} &
Gregory Dubois-Felsmann & Not Covered &
\begin{tabular}{c}
LVV-T775 \\
\end{tabular}
\\
\hline
\end{longtable}

\textbf{Verification Element Description:} \\
Undefined

{\footnotesize
\begin{longtable}{p{2.5cm}p{13.5cm}}
\hline
\multicolumn{2}{c}{\textbf{Requirement Details}}\\ \hline
Requirement ID & DMS-NB-REQ-0018 \\ \cdashline{1-2}
Requirement Description &
\begin{minipage}[]{13cm}
The Notebook Aspect shall be able to access the User File Workspace
available as a POSIX filesystem from within the Python kernels and
shell-prompt sessions it supports.
\end{minipage}
\\ \cdashline{1-2}
Requirement Priority &  \\ \cdashline{1-2}
Upper Level Requirement &
\begin{tabular}{cl}
\end{tabular}
\\ \hline
\end{longtable}
}


\subsubsection{Test Cases Summary}
\begin{longtable}{p{3cm}p{2.5cm}p{2.5cm}p{3cm}p{4cm}}
\toprule
\href{https://jira.lsstcorp.org/secure/Tests.jspa\#/testCase/LVV-T775}{LVV-T775} & \multicolumn{4}{p{12cm}}{ Verify access to User File Workspace } \\ \hline
\textbf{Owner} & \textbf{Status} & \textbf{Version} & \textbf{Critical Event} & \textbf{Verification Type} \\ \hline
Jeffrey Carlin & Draft & 1 & false & Inspection \\ \hline
\end{longtable}
{\scriptsize
\textbf{Objective:}\\
Verify that users of the Notebook Aspect are able to access the User
File Workspace available as a POSIX filesystem from within the Python
kernels and shell-prompt sessions it supports.
}
  
 \newpage 
\subsection{[LVV-9986] DMS-NB-REQ-0019-V-01: VOSpace Access\_1 }\label{lvv-9986}

\begin{longtable}{cccc}
\hline
\textbf{Jira Link} & \textbf{Assignee} & \textbf{Status} & \textbf{Test Cases}\\ \hline
\href{https://jira.lsstcorp.org/browse/LVV-9986}{LVV-9986} &
Gregory Dubois-Felsmann & Not Covered &
\begin{tabular}{c}
LVV-T776 \\
\end{tabular}
\\
\hline
\end{longtable}

\textbf{Verification Element Description:} \\
Undefined

{\footnotesize
\begin{longtable}{p{2.5cm}p{13.5cm}}
\hline
\multicolumn{2}{c}{\textbf{Requirement Details}}\\ \hline
Requirement ID & DMS-NB-REQ-0019 \\ \cdashline{1-2}
Requirement Description &
\begin{minipage}[]{13cm}
The Notebook Aspect shall be able to interact with VOSpace services
available through project or external services.
\end{minipage}
\\ \cdashline{1-2}
Requirement Discussion &
\begin{minipage}[]{13cm}
Users will be able to directly use VOSpace APIs within a Notebook. It is
not yet decided whether there will be support for user-mode mounting of
non-LSP VOSpace (or WebDAV) services as virtual POSIX filesystems.
\end{minipage}
\\ \cdashline{1-2}
Requirement Priority &  \\ \cdashline{1-2}
Upper Level Requirement &
\begin{tabular}{cl}
\end{tabular}
\\ \hline
\end{longtable}
}


\subsubsection{Test Cases Summary}
\begin{longtable}{p{3cm}p{2.5cm}p{2.5cm}p{3cm}p{4cm}}
\toprule
\href{https://jira.lsstcorp.org/secure/Tests.jspa\#/testCase/LVV-T776}{LVV-T776} & \multicolumn{4}{p{12cm}}{ Verify access to VOSpace services from Notebook aspect } \\ \hline
\textbf{Owner} & \textbf{Status} & \textbf{Version} & \textbf{Critical Event} & \textbf{Verification Type} \\ \hline
Jeffrey Carlin & Draft & 1 & false & Inspection \\ \hline
\end{longtable}
{\scriptsize
\textbf{Objective:}\\
Verify that users of the Notebook Aspect are able to interact with
VOSpace services available through project or external services. Users
will be able to directly use VOSpace APIs within a Notebook.
}
  
 \newpage 
\subsection{[LVV-9987] DMS-NB-REQ-0025-V-01: Deployment Workload in Kubernetes\_1 }\label{lvv-9987}

\begin{longtable}{cccc}
\hline
\textbf{Jira Link} & \textbf{Assignee} & \textbf{Status} & \textbf{Test Cases}\\ \hline
\href{https://jira.lsstcorp.org/browse/LVV-9987}{LVV-9987} &
Gregory Dubois-Felsmann & Not Covered &
\begin{tabular}{c}
LVV-T782 \\
\end{tabular}
\\
\hline
\end{longtable}

\textbf{Verification Element Description:} \\
Undefined

{\footnotesize
\begin{longtable}{p{2.5cm}p{13.5cm}}
\hline
\multicolumn{2}{c}{\textbf{Requirement Details}}\\ \hline
Requirement ID & DMS-NB-REQ-0025 \\ \cdashline{1-2}
Requirement Description &
\begin{minipage}[]{13cm}
Given a Kubernetes cluster with a configuration meeting a documented
standard set of specifications, it shall take an engineer with admin
rights no more than 2 days to deploy the Notebook Aspect in that
context.
\end{minipage}
\\ \cdashline{1-2}
Requirement Discussion &
\begin{minipage}[]{13cm}
The specification is expected to constrain factors such as software
versions for Kubernetes and related packages, available storage, a
shared file system, and an authentication system.
\end{minipage}
\\ \cdashline{1-2}
Requirement Priority &  \\ \cdashline{1-2}
Upper Level Requirement &
\begin{tabular}{cl}
\end{tabular}
\\ \hline
\end{longtable}
}


\subsubsection{Test Cases Summary}
\begin{longtable}{p{3cm}p{2.5cm}p{2.5cm}p{3cm}p{4cm}}
\toprule
\href{https://jira.lsstcorp.org/secure/Tests.jspa\#/testCase/LVV-T782}{LVV-T782} & \multicolumn{4}{p{12cm}}{ Verify workload for deployment in Kubernetes } \\ \hline
\textbf{Owner} & \textbf{Status} & \textbf{Version} & \textbf{Critical Event} & \textbf{Verification Type} \\ \hline
Jeffrey Carlin & Draft & 1 & false & Demonstration \\ \hline
\end{longtable}
{\scriptsize
\textbf{Objective:}\\
Given a Kubernetes cluster with a configuration meeting a documented
standard set of specifications, verify that it takes an engineer with
admin rights no more than 2 days to deploy the Notebook Aspect in that
context. The specification is expected to constrain factors such as
software versions for Kubernetes and related packages, available
storage, a shared file system, and an authentication system.
}
  
 \newpage 
\subsection{[LVV-9988] DMS-NB-REQ-0024-V-01: Ease of Deployment\_1 }\label{lvv-9988}

\begin{longtable}{cccc}
\hline
\textbf{Jira Link} & \textbf{Assignee} & \textbf{Status} & \textbf{Test Cases}\\ \hline
\href{https://jira.lsstcorp.org/browse/LVV-9988}{LVV-9988} &
Gregory Dubois-Felsmann & Not Covered &
\begin{tabular}{c}
LVV-T781 \\
\end{tabular}
\\
\hline
\end{longtable}

\textbf{Verification Element Description:} \\
Undefined

{\footnotesize
\begin{longtable}{p{2.5cm}p{13.5cm}}
\hline
\multicolumn{2}{c}{\textbf{Requirement Details}}\\ \hline
Requirement ID & DMS-NB-REQ-0024 \\ \cdashline{1-2}
Requirement Description &
\begin{minipage}[]{13cm}
The Notebook Aspect shall be deployable to multiple instances and
contexts, both private and public.
\end{minipage}
\\ \cdashline{1-2}
Requirement Discussion &
\begin{minipage}[]{13cm}
Such as the Commissioning Cluster and the LDF, but also collaborator
clusters, subject to the underlying resources available in the specific
instance. (What level of effort? 2 days/week/month; one click deployable
on a common standard platform: e.g., Kubernetes.)
\end{minipage}
\\ \cdashline{1-2}
Requirement Priority &  \\ \cdashline{1-2}
Upper Level Requirement &
\begin{tabular}{cl}
\end{tabular}
\\ \hline
\end{longtable}
}


\subsubsection{Test Cases Summary}
\begin{longtable}{p{3cm}p{2.5cm}p{2.5cm}p{3cm}p{4cm}}
\toprule
\href{https://jira.lsstcorp.org/secure/Tests.jspa\#/testCase/LVV-T781}{LVV-T781} & \multicolumn{4}{p{12cm}}{ Verify ease of Notebook aspect deployment } \\ \hline
\textbf{Owner} & \textbf{Status} & \textbf{Version} & \textbf{Critical Event} & \textbf{Verification Type} \\ \hline
Jeffrey Carlin & Draft & 1 & false & Inspection \\ \hline
\end{longtable}
{\scriptsize
\textbf{Objective:}\\
Verify that the Notebook Aspect is deployable to multiple instances and
contexts, both private and public.
}
  
 \newpage 
\subsection{[LVV-9989] DMS-NB-REQ-0026-V-01: System Health Monitoring\_1 }\label{lvv-9989}

\begin{longtable}{cccc}
\hline
\textbf{Jira Link} & \textbf{Assignee} & \textbf{Status} & \textbf{Test Cases}\\ \hline
\href{https://jira.lsstcorp.org/browse/LVV-9989}{LVV-9989} &
Gregory Dubois-Felsmann & Not Covered &
\begin{tabular}{c}
LVV-T783 \\
\end{tabular}
\\
\hline
\end{longtable}

\textbf{Verification Element Description:} \\
Undefined

{\footnotesize
\begin{longtable}{p{2.5cm}p{13.5cm}}
\hline
\multicolumn{2}{c}{\textbf{Requirement Details}}\\ \hline
Requirement ID & DMS-NB-REQ-0026 \\ \cdashline{1-2}
Requirement Description &
\begin{minipage}[]{13cm}
The Notebook Aspect shall provide a service health microservice and a
dynamic web page hostable on separate resources that provides a view of
the health status.
\end{minipage}
\\ \cdashline{1-2}
Requirement Priority &  \\ \cdashline{1-2}
Upper Level Requirement &
\begin{tabular}{cl}
\end{tabular}
\\ \hline
\end{longtable}
}


\subsubsection{Test Cases Summary}
\begin{longtable}{p{3cm}p{2.5cm}p{2.5cm}p{3cm}p{4cm}}
\toprule
\href{https://jira.lsstcorp.org/secure/Tests.jspa\#/testCase/LVV-T783}{LVV-T783} & \multicolumn{4}{p{12cm}}{ Verify monitoring of Notebook system health } \\ \hline
\textbf{Owner} & \textbf{Status} & \textbf{Version} & \textbf{Critical Event} & \textbf{Verification Type} \\ \hline
Jeffrey Carlin & Draft & 1 & false & Inspection \\ \hline
\end{longtable}
{\scriptsize
\textbf{Objective:}\\
Verify that the Notebook Aspect provides a service health microservice
and a dynamic web page hostable on separate resources that provides a
view of the health status.
}
  
 \newpage 
\subsection{[LVV-9990] DMS-NB-REQ-0032-V-01: Image Visualization\_1 }\label{lvv-9990}

\begin{longtable}{cccc}
\hline
\textbf{Jira Link} & \textbf{Assignee} & \textbf{Status} & \textbf{Test Cases}\\ \hline
\href{https://jira.lsstcorp.org/browse/LVV-9990}{LVV-9990} &
Gregory Dubois-Felsmann & Not Covered &
\begin{tabular}{c}
LVV-T784 \\
\end{tabular}
\\
\hline
\end{longtable}

\textbf{Verification Element Description:} \\
Undefined

{\footnotesize
\begin{longtable}{p{2.5cm}p{13.5cm}}
\hline
\multicolumn{2}{c}{\textbf{Requirement Details}}\\ \hline
Requirement ID & DMS-NB-REQ-0032 \\ \cdashline{1-2}
Requirement Description &
\begin{minipage}[]{13cm}
The Notebook Aspect shall provide a tool for displaying image like
datasets produced by LSST stack tools.
\end{minipage}
\\ \cdashline{1-2}
Requirement Discussion &
\begin{minipage}[]{13cm}
This requirement could be satisfied simply by inclusion of afw.display
in the pre-installed stack. However, it is anticpated that we will also
provide a way to use Portal Aspect Javascript components in JupyterLab.
\end{minipage}
\\ \cdashline{1-2}
Requirement Priority &  \\ \cdashline{1-2}
Upper Level Requirement &
\begin{tabular}{cl}
\end{tabular}
\\ \hline
\end{longtable}
}


\subsubsection{Test Cases Summary}
\begin{longtable}{p{3cm}p{2.5cm}p{2.5cm}p{3cm}p{4cm}}
\toprule
\href{https://jira.lsstcorp.org/secure/Tests.jspa\#/testCase/LVV-T784}{LVV-T784} & \multicolumn{4}{p{12cm}}{ Verify visualization of images in Notebook aspect } \\ \hline
\textbf{Owner} & \textbf{Status} & \textbf{Version} & \textbf{Critical Event} & \textbf{Verification Type} \\ \hline
Jeffrey Carlin & Draft & 1 & false & Inspection \\ \hline
\end{longtable}
{\scriptsize
\textbf{Objective:}\\
Verify that the Notebook aspect provides tools for visualization of
images produced by the LSST stack tools.
}
  
 \newpage 
\subsection{[LVV-9991] DMS-NB-REQ-0033-V-01: Scientific Plotting\_1 }\label{lvv-9991}

\begin{longtable}{cccc}
\hline
\textbf{Jira Link} & \textbf{Assignee} & \textbf{Status} & \textbf{Test Cases}\\ \hline
\href{https://jira.lsstcorp.org/browse/LVV-9991}{LVV-9991} &
Gregory Dubois-Felsmann & Not Covered &
\begin{tabular}{c}
LVV-T785 \\
\end{tabular}
\\
\hline
\end{longtable}

\textbf{Verification Element Description:} \\
Undefined

{\footnotesize
\begin{longtable}{p{2.5cm}p{13.5cm}}
\hline
\multicolumn{2}{c}{\textbf{Requirement Details}}\\ \hline
Requirement ID & DMS-NB-REQ-0033 \\ \cdashline{1-2}
Requirement Description &
\begin{minipage}[]{13cm}
The Notebook Aspect shall provide common plotting methods:\\
scatter plots, raster images, histograms, 2D histograms, contours, line
traces, polygons, compositions of these (contours on scatter plots),
density images
\end{minipage}
\\ \cdashline{1-2}
Requirement Discussion &
\begin{minipage}[]{13cm}
This requirement could be satisfied simply by ensuring that matplotlib
is usable within JupyterLab.
\end{minipage}
\\ \cdashline{1-2}
Requirement Priority &  \\ \cdashline{1-2}
Upper Level Requirement &
\begin{tabular}{cl}
\end{tabular}
\\ \hline
\end{longtable}
}


\subsubsection{Test Cases Summary}
\begin{longtable}{p{3cm}p{2.5cm}p{2.5cm}p{3cm}p{4cm}}
\toprule
\href{https://jira.lsstcorp.org/secure/Tests.jspa\#/testCase/LVV-T785}{LVV-T785} & \multicolumn{4}{p{12cm}}{ Verify availability of scientific plotting tools in Notebook aspect } \\ \hline
\textbf{Owner} & \textbf{Status} & \textbf{Version} & \textbf{Critical Event} & \textbf{Verification Type} \\ \hline
Jeffrey Carlin & Draft & 1 & false & Inspection \\ \hline
\end{longtable}
{\scriptsize
\textbf{Objective:}\\
Verify that the Notebook Aspect provides common plotting methods
including scatter plots, raster images, histograms, 2D histograms,
contours, line traces, polygons, compositions of these (contours on
scatter plots), density images.
}
  
 \newpage 
\subsection{[LVV-9992] DMS-NB-REQ-0035-V-01: Visualization Interactivity\_1 }\label{lvv-9992}

\begin{longtable}{cccc}
\hline
\textbf{Jira Link} & \textbf{Assignee} & \textbf{Status} & \textbf{Test Cases}\\ \hline
\href{https://jira.lsstcorp.org/browse/LVV-9992}{LVV-9992} &
Gregory Dubois-Felsmann & Not Covered &
\begin{tabular}{c}
LVV-T787 \\
\end{tabular}
\\
\hline
\end{longtable}

\textbf{Verification Element Description:} \\
Undefined

{\footnotesize
\begin{longtable}{p{2.5cm}p{13.5cm}}
\hline
\multicolumn{2}{c}{\textbf{Requirement Details}}\\ \hline
Requirement ID & DMS-NB-REQ-0035 \\ \cdashline{1-2}
Requirement Description &
\begin{minipage}[]{13cm}
The Notebook Aspect shall provide interactive plots for certain
visualizations:\\
Linked axes on multiple plots, zoom, pan, data point selection
\end{minipage}
\\ \cdashline{1-2}
Requirement Discussion &
\begin{minipage}[]{13cm}
(gpdf is concerned that this is too vaguely defined to be verifiable.
Should this be in a design document instead?)
\end{minipage}
\\ \cdashline{1-2}
Requirement Priority &  \\ \cdashline{1-2}
Upper Level Requirement &
\begin{tabular}{cl}
\end{tabular}
\\ \hline
\end{longtable}
}


\subsubsection{Test Cases Summary}
\begin{longtable}{p{3cm}p{2.5cm}p{2.5cm}p{3cm}p{4cm}}
\toprule
\href{https://jira.lsstcorp.org/secure/Tests.jspa\#/testCase/LVV-T787}{LVV-T787} & \multicolumn{4}{p{12cm}}{ Verify interactivity of visualizations in Notebook aspect } \\ \hline
\textbf{Owner} & \textbf{Status} & \textbf{Version} & \textbf{Critical Event} & \textbf{Verification Type} \\ \hline
Jeffrey Carlin & Draft & 1 & false & Inspection \\ \hline
\end{longtable}
{\scriptsize
\textbf{Objective:}\\
Verify that the Notebook Aspect provides interactive plots for certain
visualizations, including linked axes on multiple plots, zoom, pan, and
data point selection.
}
  
 \newpage 
\subsection{[LVV-9993] DMS-NB-REQ-0034-V-01: Visualization Linkage\_1 }\label{lvv-9993}

\begin{longtable}{cccc}
\hline
\textbf{Jira Link} & \textbf{Assignee} & \textbf{Status} & \textbf{Test Cases}\\ \hline
\href{https://jira.lsstcorp.org/browse/LVV-9993}{LVV-9993} &
Gregory Dubois-Felsmann & Not Covered &
\begin{tabular}{c}
LVV-T786 \\
\end{tabular}
\\
\hline
\end{longtable}

\textbf{Verification Element Description:} \\
Undefined

{\footnotesize
\begin{longtable}{p{2.5cm}p{13.5cm}}
\hline
\multicolumn{2}{c}{\textbf{Requirement Details}}\\ \hline
Requirement ID & DMS-NB-REQ-0034 \\ \cdashline{1-2}
Requirement Description &
\begin{minipage}[]{13cm}
The Notebook Aspect shall provide ``drill down'' functionality in plots:
brushing and linking between plots, interactively discover metadata
about particular points, drill down to imaging from measurements
\end{minipage}
\\ \cdashline{1-2}
Requirement Discussion &
\begin{minipage}[]{13cm}
Metadata can be visit properties for a measurement, git commits, etc.
(gpdf is concerned that this is too vaguely defined to be verifiable.
Should this be in a design document instead?)
\end{minipage}
\\ \cdashline{1-2}
Requirement Priority &  \\ \cdashline{1-2}
Upper Level Requirement &
\begin{tabular}{cl}
\end{tabular}
\\ \hline
\end{longtable}
}


\subsubsection{Test Cases Summary}
\begin{longtable}{p{3cm}p{2.5cm}p{2.5cm}p{3cm}p{4cm}}
\toprule
\href{https://jira.lsstcorp.org/secure/Tests.jspa\#/testCase/LVV-T786}{LVV-T786} & \multicolumn{4}{p{12cm}}{ Verify linkage of visualization tools in Notebook aspect } \\ \hline
\textbf{Owner} & \textbf{Status} & \textbf{Version} & \textbf{Critical Event} & \textbf{Verification Type} \\ \hline
Jeffrey Carlin & Draft & 1 & false & Inspection \\ \hline
\end{longtable}
{\scriptsize
\textbf{Objective:}\\
Verify that the Notebook Aspect provides ``drill down'' functionality in
plots, including brushing and linking between plots, interactive
discovery of metadata about particular points, drill down to imaging
from measurements.
}
  
 \newpage 
\subsection{[LVV-9994] DMS-NB-REQ-0036-V-01: Visualization Scaling\_1 }\label{lvv-9994}

\begin{longtable}{cccc}
\hline
\textbf{Jira Link} & \textbf{Assignee} & \textbf{Status} & \textbf{Test Cases}\\ \hline
\href{https://jira.lsstcorp.org/browse/LVV-9994}{LVV-9994} &
Gregory Dubois-Felsmann & Not Covered &
\begin{tabular}{c}
LVV-T788 \\
\end{tabular}
\\
\hline
\end{longtable}

\textbf{Verification Element Description:} \\
Undefined

{\footnotesize
\begin{longtable}{p{2.5cm}p{13.5cm}}
\hline
\multicolumn{2}{c}{\textbf{Requirement Details}}\\ \hline
Requirement ID & DMS-NB-REQ-0036 \\ \cdashline{1-2}
Requirement Description &
\begin{minipage}[]{13cm}
The Notebook Aspect shall provide interactive plots that scale to
include at least 1E6 datapoints.
\end{minipage}
\\ \cdashline{1-2}
Requirement Discussion &
\begin{minipage}[]{13cm}
This may be done through an adaptive refinement scheme like datashader.
\end{minipage}
\\ \cdashline{1-2}
Requirement Priority &  \\ \cdashline{1-2}
Upper Level Requirement &
\begin{tabular}{cl}
\end{tabular}
\\ \hline
\end{longtable}
}


\subsubsection{Test Cases Summary}
\begin{longtable}{p{3cm}p{2.5cm}p{2.5cm}p{3cm}p{4cm}}
\toprule
\href{https://jira.lsstcorp.org/secure/Tests.jspa\#/testCase/LVV-T788}{LVV-T788} & \multicolumn{4}{p{12cm}}{ Verify interactive scaling of visualizations in Notebook aspect } \\ \hline
\textbf{Owner} & \textbf{Status} & \textbf{Version} & \textbf{Critical Event} & \textbf{Verification Type} \\ \hline
Jeffrey Carlin & Draft & 1 & false & Inspection \\ \hline
\end{longtable}
{\scriptsize
\textbf{Objective:}\\
Verify that the Notebook Aspect provides interactive plots that scale to
include at least 1E6 datapoints. This may be done through an adaptive
refinement scheme like datashader.
}
  
 \newpage 
\subsection{[LVV-9995] DMS-NB-REQ-0030-V-01: Access to Portal Visualization API\_1 }\label{lvv-9995}

\begin{longtable}{cccc}
\hline
\textbf{Jira Link} & \textbf{Assignee} & \textbf{Status} & \textbf{Test Cases}\\ \hline
\href{https://jira.lsstcorp.org/browse/LVV-9995}{LVV-9995} &
Gregory Dubois-Felsmann & Not Covered &
\begin{tabular}{c}
LVV-T790 \\
\end{tabular}
\\
\hline
\end{longtable}

\textbf{Verification Element Description:} \\
Undefined

{\footnotesize
\begin{longtable}{p{2.5cm}p{13.5cm}}
\hline
\multicolumn{2}{c}{\textbf{Requirement Details}}\\ \hline
Requirement ID & DMS-NB-REQ-0030 \\ \cdashline{1-2}
Requirement Description &
\begin{minipage}[]{13cm}
The Notebook Aspect shall provide a mechanism for ``pushing'' specific
types of data to the Portal API.
\end{minipage}
\\ \cdashline{1-2}
Requirement Discussion &
\begin{minipage}[]{13cm}
For instance, this allows a user to plot a catalog of coordinates over
an image display using the Portal's Firefly components. This is
supported by DMS-PRTL-REQ-0115 on the Portal side.
\end{minipage}
\\ \cdashline{1-2}
Requirement Priority &  \\ \cdashline{1-2}
Upper Level Requirement &
\begin{tabular}{cl}
\end{tabular}
\\ \hline
\end{longtable}
}


\subsubsection{Test Cases Summary}
\begin{longtable}{p{3cm}p{2.5cm}p{2.5cm}p{3cm}p{4cm}}
\toprule
\href{https://jira.lsstcorp.org/secure/Tests.jspa\#/testCase/LVV-T790}{LVV-T790} & \multicolumn{4}{p{12cm}}{ Verify access to Portal visualization API from Notebook aspect } \\ \hline
\textbf{Owner} & \textbf{Status} & \textbf{Version} & \textbf{Critical Event} & \textbf{Verification Type} \\ \hline
Jeffrey Carlin & Draft & 1 & false & Inspection \\ \hline
\end{longtable}
{\scriptsize
\textbf{Objective:}\\
Verify that the Notebook Aspect provides a mechanism for ``pushing''
specific types of data to the Portal API. For instance, this allows a
user to plot a catalog of coordinates over an image display using the
Portal's Firefly components. This is supported by DMS-PRTL-REQ-0115 on
the Portal side.
}
  
 \newpage 
\subsection{[LVV-9996] DMS-NB-REQ-0029-V-01: Access to Portal-Initiated Queries\_1 }\label{lvv-9996}

\begin{longtable}{cccc}
\hline
\textbf{Jira Link} & \textbf{Assignee} & \textbf{Status} & \textbf{Test Cases}\\ \hline
\href{https://jira.lsstcorp.org/browse/LVV-9996}{LVV-9996} &
Gregory Dubois-Felsmann & Not Covered &
\begin{tabular}{c}
LVV-T789 \\
LVV-T1436 \\
\end{tabular}
\\
\hline
\end{longtable}

\textbf{Verification Element Description:} \\
Undefined

{\footnotesize
\begin{longtable}{p{2.5cm}p{13.5cm}}
\hline
\multicolumn{2}{c}{\textbf{Requirement Details}}\\ \hline
Requirement ID & DMS-NB-REQ-0029 \\ \cdashline{1-2}
Requirement Description &
\begin{minipage}[]{13cm}
A user of the Notebook Aspect shall have access to search queries they
performed in the Portal Aspect.
\end{minipage}
\\ \cdashline{1-2}
Requirement Discussion &
\begin{minipage}[]{13cm}
This depends on underlying services from the API Aspect.
\end{minipage}
\\ \cdashline{1-2}
Requirement Priority &  \\ \cdashline{1-2}
Upper Level Requirement &
\begin{tabular}{cl}
\end{tabular}
\\ \hline
\end{longtable}
}


\subsubsection{Test Cases Summary}
\begin{longtable}{p{3cm}p{2.5cm}p{2.5cm}p{3cm}p{4cm}}
\toprule
\href{https://jira.lsstcorp.org/secure/Tests.jspa\#/testCase/LVV-T789}{LVV-T789} & \multicolumn{4}{p{12cm}}{ Verify access to Portal queries from Notebook aspect } \\ \hline
\textbf{Owner} & \textbf{Status} & \textbf{Version} & \textbf{Critical Event} & \textbf{Verification Type} \\ \hline
Jeffrey Carlin & Draft & 1 & false & Inspection \\ \hline
\end{longtable}
{\scriptsize
\textbf{Objective:}\\
Verify that a user of the Notebook Aspect can access search queries they
performed in the Portal Aspect.
}
\begin{longtable}{p{3cm}p{2.5cm}p{2.5cm}p{3cm}p{4cm}}
\toprule
\href{https://jira.lsstcorp.org/secure/Tests.jspa\#/testCase/LVV-T1436}{LVV-T1436} & \multicolumn{4}{p{12cm}}{ LDM-503-10a: Notebook Aspect tests for LSP with Authentication and TAP
milestone } \\ \hline
\textbf{Owner} & \textbf{Status} & \textbf{Version} & \textbf{Critical Event} & \textbf{Verification Type} \\ \hline
Gregory Dubois-Felsmann & Defined & 1 & false & Test \\ \hline
\end{longtable}
{\scriptsize
\textbf{Objective:}\\
This test case verifies that the Notebook Aspect of the Science Platform
is accessible to authorized users through a login process, and that TAP
searches can be performed from Python code in the Notebook
Aspect.\\[2\baselineskip]In so doing and in conjunction with the other
LDM-503-10a test cases collected under LVV-P48, it addresses all or part
of the following requirements:

\begin{itemize}
\tightlist
\item
  DMS-LSP-REQ-0003, DMS-LSP-REQ-0005, DMS-LSP-REQ-0006,
  DMS-LSP-REQ-0020, DMS-LSP-REQ-0022, DMS-LSP-REQ-0023, DMS-LSP-REQ-0024
\item
  DMS-NB-REQ-0001, DMS-NB-REQ-0002, DMS-NB-REQ-0005, DMS-NB-REQ-0006,
  DMS-NB-REQ-0013, DMS-NB-REQ-0017, and DMS-NB-REQ-0029, primarily
\end{itemize}

Note this test was not designed to perform a full verification of the
above requirements, but rather to demonstrate having reached a certain
level of partial capability during construction.
}
  
 \newpage 
\subsection{[LVV-9997] DMS-NB-REQ-0031-V-01: Notebook-Launching Interface\_1 }\label{lvv-9997}

\begin{longtable}{cccc}
\hline
\textbf{Jira Link} & \textbf{Assignee} & \textbf{Status} & \textbf{Test Cases}\\ \hline
\href{https://jira.lsstcorp.org/browse/LVV-9997}{LVV-9997} &
Gregory Dubois-Felsmann & Not Covered &
\begin{tabular}{c}
LVV-T791 \\
\end{tabular}
\\
\hline
\end{longtable}

\textbf{Verification Element Description:} \\
Undefined

{\footnotesize
\begin{longtable}{p{2.5cm}p{13.5cm}}
\hline
\multicolumn{2}{c}{\textbf{Requirement Details}}\\ \hline
Requirement ID & DMS-NB-REQ-0031 \\ \cdashline{1-2}
Requirement Description &
\begin{minipage}[]{13cm}
The Notebook Aspect shall provide a means to trigger the opening of a
notebook with access to the results of a query performed in the Portal.
\end{minipage}
\\ \cdashline{1-2}
Requirement Discussion &
\begin{minipage}[]{13cm}
This is intended to permit a Portal user to perform a query and then
quickly obtain a Notebook session with that data available for further
analysis. The UI element for this might be in either the Portal or
Notebook system Uis, depending on implementation issues.
\end{minipage}
\\ \cdashline{1-2}
Requirement Priority &  \\ \cdashline{1-2}
Upper Level Requirement &
\begin{tabular}{cl}
\end{tabular}
\\ \hline
\end{longtable}
}


\subsubsection{Test Cases Summary}
\begin{longtable}{p{3cm}p{2.5cm}p{2.5cm}p{3cm}p{4cm}}
\toprule
\href{https://jira.lsstcorp.org/secure/Tests.jspa\#/testCase/LVV-T791}{LVV-T791} & \multicolumn{4}{p{12cm}}{ Verify ability to launch a notebook with access to Portal query results } \\ \hline
\textbf{Owner} & \textbf{Status} & \textbf{Version} & \textbf{Critical Event} & \textbf{Verification Type} \\ \hline
Jeffrey Carlin & Draft & 1 & false & Inspection \\ \hline
\end{longtable}
{\scriptsize
\textbf{Objective:}\\
Verify that the Notebook Aspect provides a means to trigger the opening
of a notebook with access to the results of a query performed in the
Portal. This is intended to permit a Portal user to perform a query and
then quickly obtain a Notebook session with that data available for
further analysis.
}
  
 \newpage 
\subsection{[LVV-9998] DMS-NB-REQ-0002-V-01: Authentication and Authorization\_1 }\label{lvv-9998}

\begin{longtable}{cccc}
\hline
\textbf{Jira Link} & \textbf{Assignee} & \textbf{Status} & \textbf{Test Cases}\\ \hline
\href{https://jira.lsstcorp.org/browse/LVV-9998}{LVV-9998} &
Gregory Dubois-Felsmann & Not Covered &
\begin{tabular}{c}
LVV-T793 \\
LVV-T1436 \\
\end{tabular}
\\
\hline
\end{longtable}

\textbf{Verification Element Description:} \\
Undefined

{\footnotesize
\begin{longtable}{p{2.5cm}p{13.5cm}}
\hline
\multicolumn{2}{c}{\textbf{Requirement Details}}\\ \hline
Requirement ID & DMS-NB-REQ-0002 \\ \cdashline{1-2}
Requirement Description &
\begin{minipage}[]{13cm}
The Notebook Aspect shall authenticate users for the purpose of
establishing authorized use and only permit access to authenticated
users using the LSST Data Facility authentication and authorisation
service.
\end{minipage}
\\ \cdashline{1-2}
Requirement Priority &  \\ \cdashline{1-2}
Upper Level Requirement &
\begin{tabular}{cl}
\end{tabular}
\\ \hline
\end{longtable}
}


\subsubsection{Test Cases Summary}
\begin{longtable}{p{3cm}p{2.5cm}p{2.5cm}p{3cm}p{4cm}}
\toprule
\href{https://jira.lsstcorp.org/secure/Tests.jspa\#/testCase/LVV-T793}{LVV-T793} & \multicolumn{4}{p{12cm}}{ Verify implementation of authentication and authorization service in
Notebook aspect } \\ \hline
\textbf{Owner} & \textbf{Status} & \textbf{Version} & \textbf{Critical Event} & \textbf{Verification Type} \\ \hline
Jeffrey Carlin & Draft & 1 & false & Inspection \\ \hline
\end{longtable}
{\scriptsize
\textbf{Objective:}\\
Verify that the Notebook Aspect provides a means to authenticate users
for the purpose of establishing authorized use and only permit access to
authenticated users using the LSST Data Facility authentication and
authorization service.
}
\begin{longtable}{p{3cm}p{2.5cm}p{2.5cm}p{3cm}p{4cm}}
\toprule
\href{https://jira.lsstcorp.org/secure/Tests.jspa\#/testCase/LVV-T1436}{LVV-T1436} & \multicolumn{4}{p{12cm}}{ LDM-503-10a: Notebook Aspect tests for LSP with Authentication and TAP
milestone } \\ \hline
\textbf{Owner} & \textbf{Status} & \textbf{Version} & \textbf{Critical Event} & \textbf{Verification Type} \\ \hline
Gregory Dubois-Felsmann & Defined & 1 & false & Test \\ \hline
\end{longtable}
{\scriptsize
\textbf{Objective:}\\
This test case verifies that the Notebook Aspect of the Science Platform
is accessible to authorized users through a login process, and that TAP
searches can be performed from Python code in the Notebook
Aspect.\\[2\baselineskip]In so doing and in conjunction with the other
LDM-503-10a test cases collected under LVV-P48, it addresses all or part
of the following requirements:

\begin{itemize}
\tightlist
\item
  DMS-LSP-REQ-0003, DMS-LSP-REQ-0005, DMS-LSP-REQ-0006,
  DMS-LSP-REQ-0020, DMS-LSP-REQ-0022, DMS-LSP-REQ-0023, DMS-LSP-REQ-0024
\item
  DMS-NB-REQ-0001, DMS-NB-REQ-0002, DMS-NB-REQ-0005, DMS-NB-REQ-0006,
  DMS-NB-REQ-0013, DMS-NB-REQ-0017, and DMS-NB-REQ-0029, primarily
\end{itemize}

Note this test was not designed to perform a full verification of the
above requirements, but rather to demonstrate having reached a certain
level of partial capability during construction.
}
  
 \newpage 
\subsection{[LVV-9999] DMS-NB-REQ-0003-V-01: Secure Implementation\_1 }\label{lvv-9999}

\begin{longtable}{cccc}
\hline
\textbf{Jira Link} & \textbf{Assignee} & \textbf{Status} & \textbf{Test Cases}\\ \hline
\href{https://jira.lsstcorp.org/browse/LVV-9999}{LVV-9999} &
Gregory Dubois-Felsmann & Not Covered &
\begin{tabular}{c}
LVV-T794 \\
\end{tabular}
\\
\hline
\end{longtable}

\textbf{Verification Element Description:} \\
Undefined

{\footnotesize
\begin{longtable}{p{2.5cm}p{13.5cm}}
\hline
\multicolumn{2}{c}{\textbf{Requirement Details}}\\ \hline
Requirement ID & DMS-NB-REQ-0003 \\ \cdashline{1-2}
Requirement Description &
\begin{minipage}[]{13cm}
The Notebook Aspect shall prevent users from circumventing authorisation
controls.
\end{minipage}
\\ \cdashline{1-2}
Requirement Discussion &
\begin{minipage}[]{13cm}
The Notebook Aspect relies on other services, such as authentication,
file system permissions etc to prevent access to unauthorized data. It
should not be possible for a user to spoof another user in a way that
permits access to unauthorized data
\end{minipage}
\\ \cdashline{1-2}
Requirement Priority &  \\ \cdashline{1-2}
Upper Level Requirement &
\begin{tabular}{cl}
\end{tabular}
\\ \hline
\end{longtable}
}


\subsubsection{Test Cases Summary}
\begin{longtable}{p{3cm}p{2.5cm}p{2.5cm}p{3cm}p{4cm}}
\toprule
\href{https://jira.lsstcorp.org/secure/Tests.jspa\#/testCase/LVV-T794}{LVV-T794} & \multicolumn{4}{p{12cm}}{ Verify secure implementation of Notebook aspect } \\ \hline
\textbf{Owner} & \textbf{Status} & \textbf{Version} & \textbf{Critical Event} & \textbf{Verification Type} \\ \hline
Jeffrey Carlin & Draft & 1 & false & Inspection \\ \hline
\end{longtable}
{\scriptsize
\textbf{Objective:}\\
Verify that the Notebook aspect does not allow users to circumvent
authorizing controls.
}
  
 \newpage 
\subsection{[LVV-10000] DMS-NB-REQ-0001-V-01: Secure Protocol\_1 }\label{lvv-10000}

\begin{longtable}{cccc}
\hline
\textbf{Jira Link} & \textbf{Assignee} & \textbf{Status} & \textbf{Test Cases}\\ \hline
\href{https://jira.lsstcorp.org/browse/LVV-10000}{LVV-10000} &
Gregory Dubois-Felsmann & Not Covered &
\begin{tabular}{c}
LVV-T792 \\
LVV-T1436 \\
\end{tabular}
\\
\hline
\end{longtable}

\textbf{Verification Element Description:} \\
Undefined

{\footnotesize
\begin{longtable}{p{2.5cm}p{13.5cm}}
\hline
\multicolumn{2}{c}{\textbf{Requirement Details}}\\ \hline
Requirement ID & DMS-NB-REQ-0001 \\ \cdashline{1-2}
Requirement Description &
\begin{minipage}[]{13cm}
The Notebook Aspect shall be accessible through an HTTPS endpoint.
\end{minipage}
\\ \cdashline{1-2}
Requirement Priority &  \\ \cdashline{1-2}
Upper Level Requirement &
\begin{tabular}{cl}
\end{tabular}
\\ \hline
\end{longtable}
}


\subsubsection{Test Cases Summary}
\begin{longtable}{p{3cm}p{2.5cm}p{2.5cm}p{3cm}p{4cm}}
\toprule
\href{https://jira.lsstcorp.org/secure/Tests.jspa\#/testCase/LVV-T792}{LVV-T792} & \multicolumn{4}{p{12cm}}{ Verify implementation of secure protocol for Notebook aspect } \\ \hline
\textbf{Owner} & \textbf{Status} & \textbf{Version} & \textbf{Critical Event} & \textbf{Verification Type} \\ \hline
Jeffrey Carlin & Draft & 1 & false & Inspection \\ \hline
\end{longtable}
{\scriptsize
\textbf{Objective:}\\
Verify that the Notebook Aspect is accessible through an HTTPS endpoint.
}
\begin{longtable}{p{3cm}p{2.5cm}p{2.5cm}p{3cm}p{4cm}}
\toprule
\href{https://jira.lsstcorp.org/secure/Tests.jspa\#/testCase/LVV-T1436}{LVV-T1436} & \multicolumn{4}{p{12cm}}{ LDM-503-10a: Notebook Aspect tests for LSP with Authentication and TAP
milestone } \\ \hline
\textbf{Owner} & \textbf{Status} & \textbf{Version} & \textbf{Critical Event} & \textbf{Verification Type} \\ \hline
Gregory Dubois-Felsmann & Defined & 1 & false & Test \\ \hline
\end{longtable}
{\scriptsize
\textbf{Objective:}\\
This test case verifies that the Notebook Aspect of the Science Platform
is accessible to authorized users through a login process, and that TAP
searches can be performed from Python code in the Notebook
Aspect.\\[2\baselineskip]In so doing and in conjunction with the other
LDM-503-10a test cases collected under LVV-P48, it addresses all or part
of the following requirements:

\begin{itemize}
\tightlist
\item
  DMS-LSP-REQ-0003, DMS-LSP-REQ-0005, DMS-LSP-REQ-0006,
  DMS-LSP-REQ-0020, DMS-LSP-REQ-0022, DMS-LSP-REQ-0023, DMS-LSP-REQ-0024
\item
  DMS-NB-REQ-0001, DMS-NB-REQ-0002, DMS-NB-REQ-0005, DMS-NB-REQ-0006,
  DMS-NB-REQ-0013, DMS-NB-REQ-0017, and DMS-NB-REQ-0029, primarily
\end{itemize}

Note this test was not designed to perform a full verification of the
above requirements, but rather to demonstrate having reached a certain
level of partial capability during construction.
}
  
 \newpage 
\subsection{[LVV-10001] DMS-NB-REQ-0004-V-01: IPV6 Access\_1 }\label{lvv-10001}

\begin{longtable}{cccc}
\hline
\textbf{Jira Link} & \textbf{Assignee} & \textbf{Status} & \textbf{Test Cases}\\ \hline
\href{https://jira.lsstcorp.org/browse/LVV-10001}{LVV-10001} &
Gregory Dubois-Felsmann & Not Covered &
\begin{tabular}{c}
LVV-T795 \\
\end{tabular}
\\
\hline
\end{longtable}

\textbf{Verification Element Description:} \\
Undefined

{\footnotesize
\begin{longtable}{p{2.5cm}p{13.5cm}}
\hline
\multicolumn{2}{c}{\textbf{Requirement Details}}\\ \hline
Requirement ID & DMS-NB-REQ-0004 \\ \cdashline{1-2}
Requirement Description &
\begin{minipage}[]{13cm}
Access to the Notebook Aspect shall support access using IPv6 protocols.
\end{minipage}
\\ \cdashline{1-2}
Requirement Priority &  \\ \cdashline{1-2}
Upper Level Requirement &
\begin{tabular}{cl}
\end{tabular}
\\ \hline
\end{longtable}
}


\subsubsection{Test Cases Summary}
\begin{longtable}{p{3cm}p{2.5cm}p{2.5cm}p{3cm}p{4cm}}
\toprule
\href{https://jira.lsstcorp.org/secure/Tests.jspa\#/testCase/LVV-T795}{LVV-T795} & \multicolumn{4}{p{12cm}}{ Verify access to Notebook aspect via IPv6 } \\ \hline
\textbf{Owner} & \textbf{Status} & \textbf{Version} & \textbf{Critical Event} & \textbf{Verification Type} \\ \hline
Jeffrey Carlin & Draft & 1 & false & Inspection \\ \hline
\end{longtable}
{\scriptsize
\textbf{Objective:}\\
Verify that the Notebook Aspect supports access using IPv6 protocols.
}
  
 \newpage 
\subsection{[LVV-10002] DMS-API-REQ-0023-V-01: Access to Catalog Data Products\_1 }\label{lvv-10002}

\begin{longtable}{cccc}
\hline
\textbf{Jira Link} & \textbf{Assignee} & \textbf{Status} & \textbf{Test Cases}\\ \hline
\href{https://jira.lsstcorp.org/browse/LVV-10002}{LVV-10002} &
Gregory Dubois-Felsmann & Not Covered &
\begin{tabular}{c}
LVV-T798 \\
LVV-T1437 \\
\end{tabular}
\\
\hline
\end{longtable}

\textbf{Verification Element Description:} \\
Undefined

{\footnotesize
\begin{longtable}{p{2.5cm}p{13.5cm}}
\hline
\multicolumn{2}{c}{\textbf{Requirement Details}}\\ \hline
Requirement ID & DMS-API-REQ-0023 \\ \cdashline{1-2}
Requirement Description &
\begin{minipage}[]{13cm}
The API Aspect shall provide for retrieval of all Prompt and Data
Release catalog data (per \citeds{LSE-163}) via TAP ADQL queries.
\end{minipage}
\\ \cdashline{1-2}
Requirement Priority &  \\ \cdashline{1-2}
Upper Level Requirement &
\begin{tabular}{cl}
\end{tabular}
\\ \hline
\end{longtable}
}


\subsubsection{Test Cases Summary}
\begin{longtable}{p{3cm}p{2.5cm}p{2.5cm}p{3cm}p{4cm}}
\toprule
\href{https://jira.lsstcorp.org/secure/Tests.jspa\#/testCase/LVV-T798}{LVV-T798} & \multicolumn{4}{p{12cm}}{ Verify API access to catalog data products } \\ \hline
\textbf{Owner} & \textbf{Status} & \textbf{Version} & \textbf{Critical Event} & \textbf{Verification Type} \\ \hline
Colin Slater & Draft & 1 & false & Inspection \\ \hline
\end{longtable}
{\scriptsize
\textbf{Objective:}\\
Verify that the API Aspect provides for retrieval of all Prompt and Data
Release catalog data via TAP ADQL queries.~
}
\begin{longtable}{p{3cm}p{2.5cm}p{2.5cm}p{3cm}p{4cm}}
\toprule
\href{https://jira.lsstcorp.org/secure/Tests.jspa\#/testCase/LVV-T1437}{LVV-T1437} & \multicolumn{4}{p{12cm}}{ LDM-503-10a: API Aspect tests for LSP with Authentication and TAP
milestone } \\ \hline
\textbf{Owner} & \textbf{Status} & \textbf{Version} & \textbf{Critical Event} & \textbf{Verification Type} \\ \hline
Gregory Dubois-Felsmann & Defined & 1 & false & Test \\ \hline
\end{longtable}
{\scriptsize
\textbf{Objective:}\\
This test case verifies that the TAP service in the API Aspect of the
Science Platform is accessible to authorized users through a login
process, and that TAP searches can be performed using the IVOA TAP
protocol from remote sites.\\[2\baselineskip]In so doing and in
conjunction with the other LDM-503-10a test cases collected under
LVV-P48, it addresses all or part of the following requirements:

\begin{itemize}
\tightlist
\item
  DMS-LSP-REQ-0004, DMS-LSP-REQ-0005, DMS-LSP-REQ-0006,
  DMS-LSP-REQ-0020, DMS-LSP-REQ-0022, DMS-LSP-REQ-0023, DMS-LSP-REQ-0024
\item
  DMS-API-REQ-0003, DMS-API-REQ-0004, DMS-API-REQ-0006,
  DMS-API-REQ-0007, DMS-API-REQ-0008, DMS-API-REQ-0009,
  DMS-API-REQ-0023, and DMS-API-REQ-0039, primarily
\end{itemize}

Note this test was not designed to perform a full verification of the
above requirements, but rather to demonstrate having reached a certain
level of partial capability during construction.
}
  
 \newpage 
\subsection{[LVV-10003] DMS-API-REQ-0022-V-01: Access to Image and Visit Metadata\_1 }\label{lvv-10003}

\begin{longtable}{cccc}
\hline
\textbf{Jira Link} & \textbf{Assignee} & \textbf{Status} & \textbf{Test Cases}\\ \hline
\href{https://jira.lsstcorp.org/browse/LVV-10003}{LVV-10003} &
Gregory Dubois-Felsmann & Not Covered &
\begin{tabular}{c}
LVV-T797 \\
\end{tabular}
\\
\hline
\end{longtable}

\textbf{Verification Element Description:} \\
Undefined

{\footnotesize
\begin{longtable}{p{2.5cm}p{13.5cm}}
\hline
\multicolumn{2}{c}{\textbf{Requirement Details}}\\ \hline
Requirement ID & DMS-API-REQ-0022 \\ \cdashline{1-2}
Requirement Description &
\begin{minipage}[]{13cm}
The API Aspect shall provide for retrieval of image and visit metadata
via TAP ADQL queries.
\end{minipage}
\\ \cdashline{1-2}
Requirement Priority &  \\ \cdashline{1-2}
Upper Level Requirement &
\begin{tabular}{cl}
\end{tabular}
\\ \hline
\end{longtable}
}


\subsubsection{Test Cases Summary}
\begin{longtable}{p{3cm}p{2.5cm}p{2.5cm}p{3cm}p{4cm}}
\toprule
\href{https://jira.lsstcorp.org/secure/Tests.jspa\#/testCase/LVV-T797}{LVV-T797} & \multicolumn{4}{p{12cm}}{ Verify API access to image and visit metadata } \\ \hline
\textbf{Owner} & \textbf{Status} & \textbf{Version} & \textbf{Critical Event} & \textbf{Verification Type} \\ \hline
Jeffrey Carlin & Draft & 1 & false & Inspection \\ \hline
\end{longtable}
{\scriptsize
\textbf{Objective:}\\
Verify that the API Aspect provides for retrieval of image and visit
metadata via TAP ADQL queries.
}
  
 \newpage 
\subsection{[LVV-10004] DMS-API-REQ-0028-V-01: Access to Image Data in FITS Format\_1 }\label{lvv-10004}

\begin{longtable}{cccc}
\hline
\textbf{Jira Link} & \textbf{Assignee} & \textbf{Status} & \textbf{Test Cases}\\ \hline
\href{https://jira.lsstcorp.org/browse/LVV-10004}{LVV-10004} &
Gregory Dubois-Felsmann & Not Covered &
\begin{tabular}{c}
LVV-T803 \\
\end{tabular}
\\
\hline
\end{longtable}

\textbf{Verification Element Description:} \\
Undefined

{\footnotesize
\begin{longtable}{p{2.5cm}p{13.5cm}}
\hline
\multicolumn{2}{c}{\textbf{Requirement Details}}\\ \hline
Requirement ID & DMS-API-REQ-0028 \\ \cdashline{1-2}
Requirement Description &
\begin{minipage}[]{13cm}
The API Aspect shall deliver image data in FITS format, and \textbf{MAY}
deliver images in additional formats.
\end{minipage}
\\ \cdashline{1-2}
Requirement Priority &  \\ \cdashline{1-2}
Upper Level Requirement &
\begin{tabular}{cl}
\end{tabular}
\\ \hline
\end{longtable}
}


\subsubsection{Test Cases Summary}
\begin{longtable}{p{3cm}p{2.5cm}p{2.5cm}p{3cm}p{4cm}}
\toprule
\href{https://jira.lsstcorp.org/secure/Tests.jspa\#/testCase/LVV-T803}{LVV-T803} & \multicolumn{4}{p{12cm}}{ Verify API access to FITS image data } \\ \hline
\textbf{Owner} & \textbf{Status} & \textbf{Version} & \textbf{Critical Event} & \textbf{Verification Type} \\ \hline
Jeffrey Carlin & Draft & 1 & false & Inspection \\ \hline
\end{longtable}
{\scriptsize
\textbf{Objective:}\\
Verify that the API Aspect delivers image data in FITS format.
}
  
 \newpage 
\subsection{[LVV-10005] DMS-API-REQ-0024-V-01: Access to Observatory Metadata\_1 }\label{lvv-10005}

\begin{longtable}{cccc}
\hline
\textbf{Jira Link} & \textbf{Assignee} & \textbf{Status} & \textbf{Test Cases}\\ \hline
\href{https://jira.lsstcorp.org/browse/LVV-10005}{LVV-10005} &
Gregory Dubois-Felsmann & Not Covered &
\begin{tabular}{c}
LVV-T799 \\
\end{tabular}
\\
\hline
\end{longtable}

\textbf{Verification Element Description:} \\
Undefined

{\footnotesize
\begin{longtable}{p{2.5cm}p{13.5cm}}
\hline
\multicolumn{2}{c}{\textbf{Requirement Details}}\\ \hline
Requirement ID & DMS-API-REQ-0024 \\ \cdashline{1-2}
Requirement Description &
\begin{minipage}[]{13cm}
The API Aspect shall provide for retrieval of observatory metadata
(including the Transformed EFD) via TAP ADQL queries.
\end{minipage}
\\ \cdashline{1-2}
Requirement Priority &  \\ \cdashline{1-2}
Upper Level Requirement &
\begin{tabular}{cl}
\end{tabular}
\\ \hline
\end{longtable}
}


\subsubsection{Test Cases Summary}
\begin{longtable}{p{3cm}p{2.5cm}p{2.5cm}p{3cm}p{4cm}}
\toprule
\href{https://jira.lsstcorp.org/secure/Tests.jspa\#/testCase/LVV-T799}{LVV-T799} & \multicolumn{4}{p{12cm}}{ Verify API access to observatory metadata } \\ \hline
\textbf{Owner} & \textbf{Status} & \textbf{Version} & \textbf{Critical Event} & \textbf{Verification Type} \\ \hline
Jeffrey Carlin & Draft & 1 & false & Inspection \\ \hline
\end{longtable}
{\scriptsize
\textbf{Objective:}\\
Verify that the API Aspect provides for retrieval of observatory
metadata (including the Transformed EFD) via TAP ADQL queries.
}
  
 \newpage 
\subsection{[LVV-10006] DMS-API-REQ-0026-V-01: Access to Reference Catalogs\_1 }\label{lvv-10006}

\begin{longtable}{cccc}
\hline
\textbf{Jira Link} & \textbf{Assignee} & \textbf{Status} & \textbf{Test Cases}\\ \hline
\href{https://jira.lsstcorp.org/browse/LVV-10006}{LVV-10006} &
Gregory Dubois-Felsmann & Not Covered &
\begin{tabular}{c}
LVV-T801 \\
\end{tabular}
\\
\hline
\end{longtable}

\textbf{Verification Element Description:} \\
Undefined

{\footnotesize
\begin{longtable}{p{2.5cm}p{13.5cm}}
\hline
\multicolumn{2}{c}{\textbf{Requirement Details}}\\ \hline
Requirement ID & DMS-API-REQ-0026 \\ \cdashline{1-2}
Requirement Description &
\begin{minipage}[]{13cm}
The API Aspect shall provide for retrieval of all reference catalog data
via TAP ADQL queries. For the purposes of this requirement a ``reference
catalog'' is an externally sourced catalog used during data production
activities.
\end{minipage}
\\ \cdashline{1-2}
Requirement Discussion &
\begin{minipage}[]{13cm}
Is this a more general provenance requirement? Just reference catalogs?
Or also, e.g., relevant calibration images? Or does it just mean that if
we have reference catalogs, they'll also be queryable? FM and GPDF: we
think the latter was meant - i.e., there's no implication that this
requirement mandates linkage.
\end{minipage}
\\ \cdashline{1-2}
Requirement Priority &  \\ \cdashline{1-2}
Upper Level Requirement &
\begin{tabular}{cl}
\end{tabular}
\\ \hline
\end{longtable}
}


\subsubsection{Test Cases Summary}
\begin{longtable}{p{3cm}p{2.5cm}p{2.5cm}p{3cm}p{4cm}}
\toprule
\href{https://jira.lsstcorp.org/secure/Tests.jspa\#/testCase/LVV-T801}{LVV-T801} & \multicolumn{4}{p{12cm}}{ Verify API access to reference catalogs } \\ \hline
\textbf{Owner} & \textbf{Status} & \textbf{Version} & \textbf{Critical Event} & \textbf{Verification Type} \\ \hline
Jeffrey Carlin & Draft & 1 & false & Inspection \\ \hline
\end{longtable}
{\scriptsize
\textbf{Objective:}\\
Verify that the API Aspect provides for retrieval of all reference
catalog data via TAP ADQL queries. For the purposes of this requirement
a ``reference catalog'' is an externally sourced catalog used during
data production activities.
}
  
 \newpage 
\subsection{[LVV-10007] DMS-API-REQ-0027-V-01: Access to Virtual Data Products\_1 }\label{lvv-10007}

\begin{longtable}{cccc}
\hline
\textbf{Jira Link} & \textbf{Assignee} & \textbf{Status} & \textbf{Test Cases}\\ \hline
\href{https://jira.lsstcorp.org/browse/LVV-10007}{LVV-10007} &
Gregory Dubois-Felsmann & Not Covered &
\begin{tabular}{c}
LVV-T802 \\
\end{tabular}
\\
\hline
\end{longtable}

\textbf{Verification Element Description:} \\
Undefined

{\footnotesize
\begin{longtable}{p{2.5cm}p{13.5cm}}
\hline
\multicolumn{2}{c}{\textbf{Requirement Details}}\\ \hline
Requirement ID & DMS-API-REQ-0027 \\ \cdashline{1-2}
Requirement Description &
\begin{minipage}[]{13cm}
The API Aspect shall provide services to initiate regeneration of, and
facilitate retrieval of, virtual data products on demand.
\end{minipage}
\\ \cdashline{1-2}
Requirement Discussion &
\begin{minipage}[]{13cm}
For image data products, this would likely be provided via the SODA
endpoint.
\end{minipage}
\\ \cdashline{1-2}
Requirement Priority &  \\ \cdashline{1-2}
Upper Level Requirement &
\begin{tabular}{cl}
\end{tabular}
\\ \hline
\end{longtable}
}


\subsubsection{Test Cases Summary}
\begin{longtable}{p{3cm}p{2.5cm}p{2.5cm}p{3cm}p{4cm}}
\toprule
\href{https://jira.lsstcorp.org/secure/Tests.jspa\#/testCase/LVV-T802}{LVV-T802} & \multicolumn{4}{p{12cm}}{ Verify API access to virtual data products } \\ \hline
\textbf{Owner} & \textbf{Status} & \textbf{Version} & \textbf{Critical Event} & \textbf{Verification Type} \\ \hline
Jeffrey Carlin & Draft & 1 & false & Inspection \\ \hline
\end{longtable}
{\scriptsize
\textbf{Objective:}\\
Verify that the API Aspect provides services to initiate regeneration
of, and facilitate retrieval of, virtual data products on demand.
}
  
 \newpage 
\subsection{[LVV-10008] DMS-API-REQ-0030-V-01: Catalog Metadata Service\_1 }\label{lvv-10008}

\begin{longtable}{cccc}
\hline
\textbf{Jira Link} & \textbf{Assignee} & \textbf{Status} & \textbf{Test Cases}\\ \hline
\href{https://jira.lsstcorp.org/browse/LVV-10008}{LVV-10008} &
Gregory Dubois-Felsmann & Not Covered &
\begin{tabular}{c}
LVV-T805 \\
\end{tabular}
\\
\hline
\end{longtable}

\textbf{Verification Element Description:} \\
Undefined

{\footnotesize
\begin{longtable}{p{2.5cm}p{13.5cm}}
\hline
\multicolumn{2}{c}{\textbf{Requirement Details}}\\ \hline
Requirement ID & DMS-API-REQ-0030 \\ \cdashline{1-2}
Requirement Description &
\begin{minipage}[]{13cm}
The API Aspect shall provide complete metadata for all tables within
each data release, including per-column a description, IVOA UCD when
appropriate, unit when appropriate, and any relationship with other
columns
\end{minipage}
\\ \cdashline{1-2}
Requirement Priority &  \\ \cdashline{1-2}
Upper Level Requirement &
\begin{tabular}{cl}
\end{tabular}
\\ \hline
\end{longtable}
}


\subsubsection{Test Cases Summary}
\begin{longtable}{p{3cm}p{2.5cm}p{2.5cm}p{3cm}p{4cm}}
\toprule
\href{https://jira.lsstcorp.org/secure/Tests.jspa\#/testCase/LVV-T805}{LVV-T805} & \multicolumn{4}{p{12cm}}{ Verify API provides catalog metadata } \\ \hline
\textbf{Owner} & \textbf{Status} & \textbf{Version} & \textbf{Critical Event} & \textbf{Verification Type} \\ \hline
Jeffrey Carlin & Draft & 1 & false & Inspection \\ \hline
\end{longtable}
{\scriptsize
\textbf{Objective:}\\
Verify that the API Aspect provides complete metadata for all tables
within each data release, including a per-column description, IVOA UCD
when appropriate, units when appropriate, and any relationship with
other columns.
}
  
 \newpage 
\subsection{[LVV-10009] DMS-API-REQ-0025-V-01: Enforcement of Information Classification\_1 }\label{lvv-10009}

\begin{longtable}{cccc}
\hline
\textbf{Jira Link} & \textbf{Assignee} & \textbf{Status} & \textbf{Test Cases}\\ \hline
\href{https://jira.lsstcorp.org/browse/LVV-10009}{LVV-10009} &
Gregory Dubois-Felsmann & Not Covered &
\begin{tabular}{c}
LVV-T800 \\
\end{tabular}
\\
\hline
\end{longtable}

\textbf{Verification Element Description:} \\
Undefined

{\footnotesize
\begin{longtable}{p{2.5cm}p{13.5cm}}
\hline
\multicolumn{2}{c}{\textbf{Requirement Details}}\\ \hline
Requirement ID & DMS-API-REQ-0025 \\ \cdashline{1-2}
Requirement Description &
\begin{minipage}[]{13cm}
The API Aspect shall \textbf{NOT} allow access to Sensitive or Highly
Sensitive (per \citeds{LPM-122}) observatory metadata.
\end{minipage}
\\ \cdashline{1-2}
Requirement Discussion &
\begin{minipage}[]{13cm}
Information classified as ``Internal'' should only be provided to
project staff.
\end{minipage}
\\ \cdashline{1-2}
Requirement Priority &  \\ \cdashline{1-2}
Upper Level Requirement &
\begin{tabular}{cl}
\end{tabular}
\\ \hline
\end{longtable}
}


\subsubsection{Test Cases Summary}
\begin{longtable}{p{3cm}p{2.5cm}p{2.5cm}p{3cm}p{4cm}}
\toprule
\href{https://jira.lsstcorp.org/secure/Tests.jspa\#/testCase/LVV-T800}{LVV-T800} & \multicolumn{4}{p{12cm}}{ Verify API enforcement of information classification } \\ \hline
\textbf{Owner} & \textbf{Status} & \textbf{Version} & \textbf{Critical Event} & \textbf{Verification Type} \\ \hline
Jeffrey Carlin & Draft & 1 & false & Inspection \\ \hline
\end{longtable}
{\scriptsize
\textbf{Objective:}\\
Verify that the API Aspect does NOT allow access to Sensitive or Highly
Sensitive (per \citeds{LPM-122}) observatory metadata.
}
  
 \newpage 
\subsection{[LVV-10010] DMS-API-REQ-0029-V-01: Multiple Data Releases\_1 }\label{lvv-10010}

\begin{longtable}{cccc}
\hline
\textbf{Jira Link} & \textbf{Assignee} & \textbf{Status} & \textbf{Test Cases}\\ \hline
\href{https://jira.lsstcorp.org/browse/LVV-10010}{LVV-10010} &
Gregory Dubois-Felsmann & Not Covered &
\begin{tabular}{c}
LVV-T804 \\
\end{tabular}
\\
\hline
\end{longtable}

\textbf{Verification Element Description:} \\
Undefined

{\footnotesize
\begin{longtable}{p{2.5cm}p{13.5cm}}
\hline
\multicolumn{2}{c}{\textbf{Requirement Details}}\\ \hline
Requirement ID & DMS-API-REQ-0029 \\ \cdashline{1-2}
Requirement Description &
\begin{minipage}[]{13cm}
The API Aspect Web APIs shall provide unambiguous access to data
products and metadata from more than one Data Release simultaneously
\end{minipage}
\\ \cdashline{1-2}
Requirement Discussion &
\begin{minipage}[]{13cm}
The requirement is explicitly silent on the question of whether data
from multiple releases will be available from a single endpoint.
\end{minipage}
\\ \cdashline{1-2}
Requirement Priority &  \\ \cdashline{1-2}
Upper Level Requirement &
\begin{tabular}{cl}
\end{tabular}
\\ \hline
\end{longtable}
}


\subsubsection{Test Cases Summary}
\begin{longtable}{p{3cm}p{2.5cm}p{2.5cm}p{3cm}p{4cm}}
\toprule
\href{https://jira.lsstcorp.org/secure/Tests.jspa\#/testCase/LVV-T804}{LVV-T804} & \multicolumn{4}{p{12cm}}{ Verify API access to multiple data releases } \\ \hline
\textbf{Owner} & \textbf{Status} & \textbf{Version} & \textbf{Critical Event} & \textbf{Verification Type} \\ \hline
Jeffrey Carlin & Draft & 1 & false & Inspection \\ \hline
\end{longtable}
{\scriptsize
\textbf{Objective:}\\
Verify that the API Aspect Web APIs provide unambiguous access to data
products and metadata from more than one Data Release simultaneously.
}
  
 \newpage 
\subsection{[LVV-10011] DMS-API-REQ-0021-V-01: Use of CAOM2\_1 }\label{lvv-10011}

\begin{longtable}{cccc}
\hline
\textbf{Jira Link} & \textbf{Assignee} & \textbf{Status} & \textbf{Test Cases}\\ \hline
\href{https://jira.lsstcorp.org/browse/LVV-10011}{LVV-10011} &
Gregory Dubois-Felsmann & Not Covered &
\begin{tabular}{c}
LVV-T796 \\
\end{tabular}
\\
\hline
\end{longtable}

\textbf{Verification Element Description:} \\
Undefined

{\footnotesize
\begin{longtable}{p{2.5cm}p{13.5cm}}
\hline
\multicolumn{2}{c}{\textbf{Requirement Details}}\\ \hline
Requirement ID & DMS-API-REQ-0021 \\ \cdashline{1-2}
Requirement Description &
\begin{minipage}[]{13cm}
The API Aspect Web APIs shall present image and visit metadata organized
in accordance with the CAOM2 data model.
\end{minipage}
\\ \cdashline{1-2}
Requirement Priority &  \\ \cdashline{1-2}
Upper Level Requirement &
\begin{tabular}{cl}
\end{tabular}
\\ \hline
\end{longtable}
}


\subsubsection{Test Cases Summary}
\begin{longtable}{p{3cm}p{2.5cm}p{2.5cm}p{3cm}p{4cm}}
\toprule
\href{https://jira.lsstcorp.org/secure/Tests.jspa\#/testCase/LVV-T796}{LVV-T796} & \multicolumn{4}{p{12cm}}{ Verify web APIs use CAOM2 } \\ \hline
\textbf{Owner} & \textbf{Status} & \textbf{Version} & \textbf{Critical Event} & \textbf{Verification Type} \\ \hline
Jeffrey Carlin & Draft & 1 & false & Inspection \\ \hline
\end{longtable}
{\scriptsize
\textbf{Objective:}\\
Verify that the API Aspect Web APIs present image and visit metadata
organized in accordance with the CAOM2 data model.
}
  
 \newpage 
\subsection{[LVV-10012] DMS-API-REQ-0009-V-01: ADQL Support\_1 }\label{lvv-10012}

\begin{longtable}{cccc}
\hline
\textbf{Jira Link} & \textbf{Assignee} & \textbf{Status} & \textbf{Test Cases}\\ \hline
\href{https://jira.lsstcorp.org/browse/LVV-10012}{LVV-10012} &
Gregory Dubois-Felsmann & Not Covered &
\begin{tabular}{c}
LVV-T809 \\
LVV-T1437 \\
\end{tabular}
\\
\hline
\end{longtable}

\textbf{Verification Element Description:} \\
Undefined

{\footnotesize
\begin{longtable}{p{2.5cm}p{13.5cm}}
\hline
\multicolumn{2}{c}{\textbf{Requirement Details}}\\ \hline
Requirement ID & DMS-API-REQ-0009 \\ \cdashline{1-2}
Requirement Description &
\begin{minipage}[]{13cm}
The API Aspect TAP endpoint shall support IVOA ADQL 2.1 as a query
language, \textbf{BUT} supported query syntax for database targets
\textbf{MAY} be limited by practical considerations of individual
underlying database technologies
\end{minipage}
\\ \cdashline{1-2}
Requirement Priority &  \\ \cdashline{1-2}
Upper Level Requirement &
\begin{tabular}{cl}
\end{tabular}
\\ \hline
\end{longtable}
}


\subsubsection{Test Cases Summary}
\begin{longtable}{p{3cm}p{2.5cm}p{2.5cm}p{3cm}p{4cm}}
\toprule
\href{https://jira.lsstcorp.org/secure/Tests.jspa\#/testCase/LVV-T809}{LVV-T809} & \multicolumn{4}{p{12cm}}{ Verify availability of ADQL for queries } \\ \hline
\textbf{Owner} & \textbf{Status} & \textbf{Version} & \textbf{Critical Event} & \textbf{Verification Type} \\ \hline
Colin Slater & Draft & 1 & false & Inspection \\ \hline
\end{longtable}
{\scriptsize
\textbf{Objective:}\\
Verify that the API Aspect TAP endpoint supports IVOA ADQL 2.1 as a
query language.
}
\begin{longtable}{p{3cm}p{2.5cm}p{2.5cm}p{3cm}p{4cm}}
\toprule
\href{https://jira.lsstcorp.org/secure/Tests.jspa\#/testCase/LVV-T1437}{LVV-T1437} & \multicolumn{4}{p{12cm}}{ LDM-503-10a: API Aspect tests for LSP with Authentication and TAP
milestone } \\ \hline
\textbf{Owner} & \textbf{Status} & \textbf{Version} & \textbf{Critical Event} & \textbf{Verification Type} \\ \hline
Gregory Dubois-Felsmann & Defined & 1 & false & Test \\ \hline
\end{longtable}
{\scriptsize
\textbf{Objective:}\\
This test case verifies that the TAP service in the API Aspect of the
Science Platform is accessible to authorized users through a login
process, and that TAP searches can be performed using the IVOA TAP
protocol from remote sites.\\[2\baselineskip]In so doing and in
conjunction with the other LDM-503-10a test cases collected under
LVV-P48, it addresses all or part of the following requirements:

\begin{itemize}
\tightlist
\item
  DMS-LSP-REQ-0004, DMS-LSP-REQ-0005, DMS-LSP-REQ-0006,
  DMS-LSP-REQ-0020, DMS-LSP-REQ-0022, DMS-LSP-REQ-0023, DMS-LSP-REQ-0024
\item
  DMS-API-REQ-0003, DMS-API-REQ-0004, DMS-API-REQ-0006,
  DMS-API-REQ-0007, DMS-API-REQ-0008, DMS-API-REQ-0009,
  DMS-API-REQ-0023, and DMS-API-REQ-0039, primarily
\end{itemize}

Note this test was not designed to perform a full verification of the
above requirements, but rather to demonstrate having reached a certain
level of partial capability during construction.
}
  
 \newpage 
\subsection{[LVV-10013] DMS-API-REQ-0008-V-01: Asynchronous TAP Support\_1 }\label{lvv-10013}

\begin{longtable}{cccc}
\hline
\textbf{Jira Link} & \textbf{Assignee} & \textbf{Status} & \textbf{Test Cases}\\ \hline
\href{https://jira.lsstcorp.org/browse/LVV-10013}{LVV-10013} &
Gregory Dubois-Felsmann & Not Covered &
\begin{tabular}{c}
LVV-T808 \\
LVV-T1437 \\
\end{tabular}
\\
\hline
\end{longtable}

\textbf{Verification Element Description:} \\
Undefined

{\footnotesize
\begin{longtable}{p{2.5cm}p{13.5cm}}
\hline
\multicolumn{2}{c}{\textbf{Requirement Details}}\\ \hline
Requirement ID & DMS-API-REQ-0008 \\ \cdashline{1-2}
Requirement Description &
\begin{minipage}[]{13cm}
The API Aspect TAP endpoint shall support asynchronous queries as
described by the IVOA TAP 1.1 specification.
\end{minipage}
\\ \cdashline{1-2}
Requirement Discussion &
\begin{minipage}[]{13cm}
Asynchronous queries are expected to be the primary means for carrying
out user-driven queries from the Portal aspect (so that the query is
assigned an ID that enables multiple retrievals of its results, e.g., in
both the Portal and Notebook Aspects). (The Portal Aspect implementation
will still use synchronous queries for internal functions such as
retrieving metadata needed for page configuration, etc.) Asynchronous
queries will also be used across all aspects for queries where large
result sets and/or long run times are expected.
\end{minipage}
\\ \cdashline{1-2}
Requirement Priority &  \\ \cdashline{1-2}
Upper Level Requirement &
\begin{tabular}{cl}
\end{tabular}
\\ \hline
\end{longtable}
}


\subsubsection{Test Cases Summary}
\begin{longtable}{p{3cm}p{2.5cm}p{2.5cm}p{3cm}p{4cm}}
\toprule
\href{https://jira.lsstcorp.org/secure/Tests.jspa\#/testCase/LVV-T808}{LVV-T808} & \multicolumn{4}{p{12cm}}{ Verify asynchronous TAP queries } \\ \hline
\textbf{Owner} & \textbf{Status} & \textbf{Version} & \textbf{Critical Event} & \textbf{Verification Type} \\ \hline
Jeffrey Carlin & Draft & 1 & false & Inspection \\ \hline
\end{longtable}
{\scriptsize
\textbf{Objective:}\\
Verify that the API Aspect TAP endpoint supports asynchronous queries as
described by the IVOA TAP 1.1 specification.
}
\begin{longtable}{p{3cm}p{2.5cm}p{2.5cm}p{3cm}p{4cm}}
\toprule
\href{https://jira.lsstcorp.org/secure/Tests.jspa\#/testCase/LVV-T1437}{LVV-T1437} & \multicolumn{4}{p{12cm}}{ LDM-503-10a: API Aspect tests for LSP with Authentication and TAP
milestone } \\ \hline
\textbf{Owner} & \textbf{Status} & \textbf{Version} & \textbf{Critical Event} & \textbf{Verification Type} \\ \hline
Gregory Dubois-Felsmann & Defined & 1 & false & Test \\ \hline
\end{longtable}
{\scriptsize
\textbf{Objective:}\\
This test case verifies that the TAP service in the API Aspect of the
Science Platform is accessible to authorized users through a login
process, and that TAP searches can be performed using the IVOA TAP
protocol from remote sites.\\[2\baselineskip]In so doing and in
conjunction with the other LDM-503-10a test cases collected under
LVV-P48, it addresses all or part of the following requirements:

\begin{itemize}
\tightlist
\item
  DMS-LSP-REQ-0004, DMS-LSP-REQ-0005, DMS-LSP-REQ-0006,
  DMS-LSP-REQ-0020, DMS-LSP-REQ-0022, DMS-LSP-REQ-0023, DMS-LSP-REQ-0024
\item
  DMS-API-REQ-0003, DMS-API-REQ-0004, DMS-API-REQ-0006,
  DMS-API-REQ-0007, DMS-API-REQ-0008, DMS-API-REQ-0009,
  DMS-API-REQ-0023, and DMS-API-REQ-0039, primarily
\end{itemize}

Note this test was not designed to perform a full verification of the
above requirements, but rather to demonstrate having reached a certain
level of partial capability during construction.
}
  
 \newpage 
\subsection{[LVV-10014] DMS-API-REQ-0007-V-01: Synchronous TAP Support\_1 }\label{lvv-10014}

\begin{longtable}{cccc}
\hline
\textbf{Jira Link} & \textbf{Assignee} & \textbf{Status} & \textbf{Test Cases}\\ \hline
\href{https://jira.lsstcorp.org/browse/LVV-10014}{LVV-10014} &
Gregory Dubois-Felsmann & Not Covered &
\begin{tabular}{c}
LVV-T807 \\
LVV-T1437 \\
\end{tabular}
\\
\hline
\end{longtable}

\textbf{Verification Element Description:} \\
Undefined

{\footnotesize
\begin{longtable}{p{2.5cm}p{13.5cm}}
\hline
\multicolumn{2}{c}{\textbf{Requirement Details}}\\ \hline
Requirement ID & DMS-API-REQ-0007 \\ \cdashline{1-2}
Requirement Description &
\begin{minipage}[]{13cm}
The API Aspect TAP endpoint shall support synchronous queries as
described by the IVOA TAP 1.1 specification.
\end{minipage}
\\ \cdashline{1-2}
Requirement Discussion &
\begin{minipage}[]{13cm}
Synchronous queries are primarily expected to be used for small results.
It is TBD whether this mode will be suitable for use with shared-scan
queries.
\end{minipage}
\\ \cdashline{1-2}
Requirement Priority &  \\ \cdashline{1-2}
Upper Level Requirement &
\begin{tabular}{cl}
\end{tabular}
\\ \hline
\end{longtable}
}


\subsubsection{Test Cases Summary}
\begin{longtable}{p{3cm}p{2.5cm}p{2.5cm}p{3cm}p{4cm}}
\toprule
\href{https://jira.lsstcorp.org/secure/Tests.jspa\#/testCase/LVV-T807}{LVV-T807} & \multicolumn{4}{p{12cm}}{ Verify synchronous TAP queries } \\ \hline
\textbf{Owner} & \textbf{Status} & \textbf{Version} & \textbf{Critical Event} & \textbf{Verification Type} \\ \hline
Jeffrey Carlin & Defined & 1 & false & Inspection \\ \hline
\end{longtable}
{\scriptsize
\textbf{Objective:}\\
Verify that the API Aspect TAP endpoint supports synchronous queries as
described by the IVOA TAP 1.1 specification. ~ ~
}
\begin{longtable}{p{3cm}p{2.5cm}p{2.5cm}p{3cm}p{4cm}}
\toprule
\href{https://jira.lsstcorp.org/secure/Tests.jspa\#/testCase/LVV-T1437}{LVV-T1437} & \multicolumn{4}{p{12cm}}{ LDM-503-10a: API Aspect tests for LSP with Authentication and TAP
milestone } \\ \hline
\textbf{Owner} & \textbf{Status} & \textbf{Version} & \textbf{Critical Event} & \textbf{Verification Type} \\ \hline
Gregory Dubois-Felsmann & Defined & 1 & false & Test \\ \hline
\end{longtable}
{\scriptsize
\textbf{Objective:}\\
This test case verifies that the TAP service in the API Aspect of the
Science Platform is accessible to authorized users through a login
process, and that TAP searches can be performed using the IVOA TAP
protocol from remote sites.\\[2\baselineskip]In so doing and in
conjunction with the other LDM-503-10a test cases collected under
LVV-P48, it addresses all or part of the following requirements:

\begin{itemize}
\tightlist
\item
  DMS-LSP-REQ-0004, DMS-LSP-REQ-0005, DMS-LSP-REQ-0006,
  DMS-LSP-REQ-0020, DMS-LSP-REQ-0022, DMS-LSP-REQ-0023, DMS-LSP-REQ-0024
\item
  DMS-API-REQ-0003, DMS-API-REQ-0004, DMS-API-REQ-0006,
  DMS-API-REQ-0007, DMS-API-REQ-0008, DMS-API-REQ-0009,
  DMS-API-REQ-0023, and DMS-API-REQ-0039, primarily
\end{itemize}

Note this test was not designed to perform a full verification of the
above requirements, but rather to demonstrate having reached a certain
level of partial capability during construction.
}
  
 \newpage 
\subsection{[LVV-10015] DMS-API-REQ-0006-V-01: TAP Service for Tabular Queries\_1 }\label{lvv-10015}

\begin{longtable}{cccc}
\hline
\textbf{Jira Link} & \textbf{Assignee} & \textbf{Status} & \textbf{Test Cases}\\ \hline
\href{https://jira.lsstcorp.org/browse/LVV-10015}{LVV-10015} &
Gregory Dubois-Felsmann & Not Covered &
\begin{tabular}{c}
LVV-T806 \\
LVV-T1437 \\
\end{tabular}
\\
\hline
\end{longtable}

\textbf{Verification Element Description:} \\
Undefined

{\footnotesize
\begin{longtable}{p{2.5cm}p{13.5cm}}
\hline
\multicolumn{2}{c}{\textbf{Requirement Details}}\\ \hline
Requirement ID & DMS-API-REQ-0006 \\ \cdashline{1-2}
Requirement Description &
\begin{minipage}[]{13cm}
The API Aspect Web APIs shall include an endpoint conforming to IVOA TAP
1.1 for the purpose of accessing tabularly structured data.
\end{minipage}
\\ \cdashline{1-2}
Requirement Priority &  \\ \cdashline{1-2}
Upper Level Requirement &
\begin{tabular}{cl}
\end{tabular}
\\ \hline
\end{longtable}
}


\subsubsection{Test Cases Summary}
\begin{longtable}{p{3cm}p{2.5cm}p{2.5cm}p{3cm}p{4cm}}
\toprule
\href{https://jira.lsstcorp.org/secure/Tests.jspa\#/testCase/LVV-T806}{LVV-T806} & \multicolumn{4}{p{12cm}}{ Verify availability of TAP service } \\ \hline
\textbf{Owner} & \textbf{Status} & \textbf{Version} & \textbf{Critical Event} & \textbf{Verification Type} \\ \hline
Jeffrey Carlin & Draft & 1 & false & Inspection \\ \hline
\end{longtable}
{\scriptsize
\textbf{Objective:}\\
Verify that the API Aspect Web APIs include an endpoint conforming to
IVOA TAP 1.1 for the purpose of accessing tabularly structured data.
}
\begin{longtable}{p{3cm}p{2.5cm}p{2.5cm}p{3cm}p{4cm}}
\toprule
\href{https://jira.lsstcorp.org/secure/Tests.jspa\#/testCase/LVV-T1437}{LVV-T1437} & \multicolumn{4}{p{12cm}}{ LDM-503-10a: API Aspect tests for LSP with Authentication and TAP
milestone } \\ \hline
\textbf{Owner} & \textbf{Status} & \textbf{Version} & \textbf{Critical Event} & \textbf{Verification Type} \\ \hline
Gregory Dubois-Felsmann & Defined & 1 & false & Test \\ \hline
\end{longtable}
{\scriptsize
\textbf{Objective:}\\
This test case verifies that the TAP service in the API Aspect of the
Science Platform is accessible to authorized users through a login
process, and that TAP searches can be performed using the IVOA TAP
protocol from remote sites.\\[2\baselineskip]In so doing and in
conjunction with the other LDM-503-10a test cases collected under
LVV-P48, it addresses all or part of the following requirements:

\begin{itemize}
\tightlist
\item
  DMS-LSP-REQ-0004, DMS-LSP-REQ-0005, DMS-LSP-REQ-0006,
  DMS-LSP-REQ-0020, DMS-LSP-REQ-0022, DMS-LSP-REQ-0023, DMS-LSP-REQ-0024
\item
  DMS-API-REQ-0003, DMS-API-REQ-0004, DMS-API-REQ-0006,
  DMS-API-REQ-0007, DMS-API-REQ-0008, DMS-API-REQ-0009,
  DMS-API-REQ-0023, and DMS-API-REQ-0039, primarily
\end{itemize}

Note this test was not designed to perform a full verification of the
above requirements, but rather to demonstrate having reached a certain
level of partial capability during construction.
}
  
 \newpage 
\subsection{[LVV-10016] DMS-API-REQ-0016-V-01: SIA Service for Image Availability\_1 }\label{lvv-10016}

\begin{longtable}{cccc}
\hline
\textbf{Jira Link} & \textbf{Assignee} & \textbf{Status} & \textbf{Test Cases}\\ \hline
\href{https://jira.lsstcorp.org/browse/LVV-10016}{LVV-10016} &
Gregory Dubois-Felsmann & Not Covered &
\begin{tabular}{c}
LVV-T810 \\
\end{tabular}
\\
\hline
\end{longtable}

\textbf{Verification Element Description:} \\
Undefined

{\footnotesize
\begin{longtable}{p{2.5cm}p{13.5cm}}
\hline
\multicolumn{2}{c}{\textbf{Requirement Details}}\\ \hline
Requirement ID & DMS-API-REQ-0016 \\ \cdashline{1-2}
Requirement Description &
\begin{minipage}[]{13cm}
The API Aspect Web APIs shall include an endpoint conforming to IVOA SIA
V2 for the purpose of locating available images
\end{minipage}
\\ \cdashline{1-2}
Requirement Priority &  \\ \cdashline{1-2}
Upper Level Requirement &
\begin{tabular}{cl}
\end{tabular}
\\ \hline
\end{longtable}
}


\subsubsection{Test Cases Summary}
\begin{longtable}{p{3cm}p{2.5cm}p{2.5cm}p{3cm}p{4cm}}
\toprule
\href{https://jira.lsstcorp.org/secure/Tests.jspa\#/testCase/LVV-T810}{LVV-T810} & \multicolumn{4}{p{12cm}}{ Verify SIA service for image availability } \\ \hline
\textbf{Owner} & \textbf{Status} & \textbf{Version} & \textbf{Critical Event} & \textbf{Verification Type} \\ \hline
Jeffrey Carlin & Draft & 1 & false & Inspection \\ \hline
\end{longtable}
{\scriptsize
\textbf{Objective:}\\
Verify that the API Aspect Web APIs include an endpoint conforming to
IVOA SIA V2 for the purpose of locating available images.
}
  
 \newpage 
\subsection{[LVV-10017] DMS-API-REQ-0018-V-01: Cutout Service\_1 }\label{lvv-10017}

\begin{longtable}{cccc}
\hline
\textbf{Jira Link} & \textbf{Assignee} & \textbf{Status} & \textbf{Test Cases}\\ \hline
\href{https://jira.lsstcorp.org/browse/LVV-10017}{LVV-10017} &
Gregory Dubois-Felsmann & Not Covered &
\begin{tabular}{c}
LVV-T812 \\
\end{tabular}
\\
\hline
\end{longtable}

\textbf{Verification Element Description:} \\
Undefined

{\footnotesize
\begin{longtable}{p{2.5cm}p{13.5cm}}
\hline
\multicolumn{2}{c}{\textbf{Requirement Details}}\\ \hline
Requirement ID & DMS-API-REQ-0018 \\ \cdashline{1-2}
Requirement Description &
\begin{minipage}[]{13cm}
The API Aspect SODA enpoint shall support performing cutouts on all
released image data types, \textbf{BUT} supported filter predicates
\textbf{MAY} exclude POLYGON
\end{minipage}
\\ \cdashline{1-2}
Requirement Priority &  \\ \cdashline{1-2}
Upper Level Requirement &
\begin{tabular}{cl}
\end{tabular}
\\ \hline
\end{longtable}
}


\subsubsection{Test Cases Summary}
\begin{longtable}{p{3cm}p{2.5cm}p{2.5cm}p{3cm}p{4cm}}
\toprule
\href{https://jira.lsstcorp.org/secure/Tests.jspa\#/testCase/LVV-T812}{LVV-T812} & \multicolumn{4}{p{12cm}}{ Verify API SODA cutout image support } \\ \hline
\textbf{Owner} & \textbf{Status} & \textbf{Version} & \textbf{Critical Event} & \textbf{Verification Type} \\ \hline
Colin Slater & Draft & 1 & false & Inspection \\ \hline
\end{longtable}
{\scriptsize
\textbf{Objective:}\\
Verify that the API Aspect SODA endpoint supports performing cutouts on
all released image data types.
}
  
 \newpage 
\subsection{[LVV-10018] DMS-API-REQ-0017-V-01: SODA Service for Image Data\_1 }\label{lvv-10018}

\begin{longtable}{cccc}
\hline
\textbf{Jira Link} & \textbf{Assignee} & \textbf{Status} & \textbf{Test Cases}\\ \hline
\href{https://jira.lsstcorp.org/browse/LVV-10018}{LVV-10018} &
Gregory Dubois-Felsmann & Not Covered &
\begin{tabular}{c}
LVV-T811 \\
\end{tabular}
\\
\hline
\end{longtable}

\textbf{Verification Element Description:} \\
Undefined

{\footnotesize
\begin{longtable}{p{2.5cm}p{13.5cm}}
\hline
\multicolumn{2}{c}{\textbf{Requirement Details}}\\ \hline
Requirement ID & DMS-API-REQ-0017 \\ \cdashline{1-2}
Requirement Description &
\begin{minipage}[]{13cm}
The API Aspect Web APIs shall include an endpoint conforming to IVOA
SODA 1.0 for the purpose of retrieving image data.
\end{minipage}
\\ \cdashline{1-2}
Requirement Priority &  \\ \cdashline{1-2}
Upper Level Requirement &
\begin{tabular}{cl}
\end{tabular}
\\ \hline
\end{longtable}
}


\subsubsection{Test Cases Summary}
\begin{longtable}{p{3cm}p{2.5cm}p{2.5cm}p{3cm}p{4cm}}
\toprule
\href{https://jira.lsstcorp.org/secure/Tests.jspa\#/testCase/LVV-T811}{LVV-T811} & \multicolumn{4}{p{12cm}}{ Verify availability of SODA service for image data } \\ \hline
\textbf{Owner} & \textbf{Status} & \textbf{Version} & \textbf{Critical Event} & \textbf{Verification Type} \\ \hline
Jeffrey Carlin & Draft & 1 & false & Inspection \\ \hline
\end{longtable}
{\scriptsize
\textbf{Objective:}\\
~Verify that the API Aspect Web APIs include an endpoint conforming to
IVOA SODA 1.0 for the purpose of retrieving image data.
}
  
 \newpage 
\subsection{[LVV-10019] DMS-API-REQ-0039-V-01: Cached Query Result Retrieval\_1 }\label{lvv-10019}

\begin{longtable}{cccc}
\hline
\textbf{Jira Link} & \textbf{Assignee} & \textbf{Status} & \textbf{Test Cases}\\ \hline
\href{https://jira.lsstcorp.org/browse/LVV-10019}{LVV-10019} &
Gregory Dubois-Felsmann & Not Covered &
\begin{tabular}{c}
LVV-T814 \\
LVV-T1437 \\
\end{tabular}
\\
\hline
\end{longtable}

\textbf{Verification Element Description:} \\
Undefined

{\footnotesize
\begin{longtable}{p{2.5cm}p{13.5cm}}
\hline
\multicolumn{2}{c}{\textbf{Requirement Details}}\\ \hline
Requirement ID & DMS-API-REQ-0039 \\ \cdashline{1-2}
Requirement Description &
\begin{minipage}[]{13cm}
The API Aspect shall provide for the caching of results of queries for a
limited time, and their retrieval based on information from the query
history or on query identifiers previously returned from asynchronous
query services.
\end{minipage}
\\ \cdashline{1-2}
Requirement Discussion &
\begin{minipage}[]{13cm}
Caching is subject to resource contraints. The system may use a
combination of a central buffer and quota from the user's Workspace to
implement caching; the details are still under design.
\end{minipage}
\\ \cdashline{1-2}
Requirement Priority &  \\ \cdashline{1-2}
Upper Level Requirement &
\begin{tabular}{cl}
\end{tabular}
\\ \hline
\end{longtable}
}


\subsubsection{Test Cases Summary}
\begin{longtable}{p{3cm}p{2.5cm}p{2.5cm}p{3cm}p{4cm}}
\toprule
\href{https://jira.lsstcorp.org/secure/Tests.jspa\#/testCase/LVV-T814}{LVV-T814} & \multicolumn{4}{p{12cm}}{ Verify availability of cached query result retrieval } \\ \hline
\textbf{Owner} & \textbf{Status} & \textbf{Version} & \textbf{Critical Event} & \textbf{Verification Type} \\ \hline
Jeffrey Carlin & Draft & 1 & false & Inspection \\ \hline
\end{longtable}
{\scriptsize
\textbf{Objective:}\\
Verify that the API Aspect provides for the caching of results of
queries for a limited time, and their retrieval based on information
from the query history or on query identifiers previously returned from
asynchronous query services.
}
\begin{longtable}{p{3cm}p{2.5cm}p{2.5cm}p{3cm}p{4cm}}
\toprule
\href{https://jira.lsstcorp.org/secure/Tests.jspa\#/testCase/LVV-T1437}{LVV-T1437} & \multicolumn{4}{p{12cm}}{ LDM-503-10a: API Aspect tests for LSP with Authentication and TAP
milestone } \\ \hline
\textbf{Owner} & \textbf{Status} & \textbf{Version} & \textbf{Critical Event} & \textbf{Verification Type} \\ \hline
Gregory Dubois-Felsmann & Defined & 1 & false & Test \\ \hline
\end{longtable}
{\scriptsize
\textbf{Objective:}\\
This test case verifies that the TAP service in the API Aspect of the
Science Platform is accessible to authorized users through a login
process, and that TAP searches can be performed using the IVOA TAP
protocol from remote sites.\\[2\baselineskip]In so doing and in
conjunction with the other LDM-503-10a test cases collected under
LVV-P48, it addresses all or part of the following requirements:

\begin{itemize}
\tightlist
\item
  DMS-LSP-REQ-0004, DMS-LSP-REQ-0005, DMS-LSP-REQ-0006,
  DMS-LSP-REQ-0020, DMS-LSP-REQ-0022, DMS-LSP-REQ-0023, DMS-LSP-REQ-0024
\item
  DMS-API-REQ-0003, DMS-API-REQ-0004, DMS-API-REQ-0006,
  DMS-API-REQ-0007, DMS-API-REQ-0008, DMS-API-REQ-0009,
  DMS-API-REQ-0023, and DMS-API-REQ-0039, primarily
\end{itemize}

Note this test was not designed to perform a full verification of the
above requirements, but rather to demonstrate having reached a certain
level of partial capability during construction.
}
  
 \newpage 
\subsection{[LVV-10020] DMS-API-REQ-0038-V-01: Query History Retrieval\_1 }\label{lvv-10020}

\begin{longtable}{cccc}
\hline
\textbf{Jira Link} & \textbf{Assignee} & \textbf{Status} & \textbf{Test Cases}\\ \hline
\href{https://jira.lsstcorp.org/browse/LVV-10020}{LVV-10020} &
Gregory Dubois-Felsmann & Not Covered &
\begin{tabular}{c}
LVV-T813 \\
\end{tabular}
\\
\hline
\end{longtable}

\textbf{Verification Element Description:} \\
Undefined

{\footnotesize
\begin{longtable}{p{2.5cm}p{13.5cm}}
\hline
\multicolumn{2}{c}{\textbf{Requirement Details}}\\ \hline
Requirement ID & DMS-API-REQ-0038 \\ \cdashline{1-2}
Requirement Description &
\begin{minipage}[]{13cm}
The API aspect shall provide interfaces for retrieving the history of
queries for a user.
\end{minipage}
\\ \cdashline{1-2}
Requirement Discussion &
\begin{minipage}[]{13cm}
This capability is essential for cross-Aspect linking, where a query is
created in one Aspect and accessed, or re-executed, in another.
\end{minipage}
\\ \cdashline{1-2}
Requirement Priority &  \\ \cdashline{1-2}
Upper Level Requirement &
\begin{tabular}{cl}
\end{tabular}
\\ \hline
\end{longtable}
}


\subsubsection{Test Cases Summary}
\begin{longtable}{p{3cm}p{2.5cm}p{2.5cm}p{3cm}p{4cm}}
\toprule
\href{https://jira.lsstcorp.org/secure/Tests.jspa\#/testCase/LVV-T813}{LVV-T813} & \multicolumn{4}{p{12cm}}{ Verify query history retrieval } \\ \hline
\textbf{Owner} & \textbf{Status} & \textbf{Version} & \textbf{Critical Event} & \textbf{Verification Type} \\ \hline
Jeffrey Carlin & Draft & 1 & false & Inspection \\ \hline
\end{longtable}
{\scriptsize
\textbf{Objective:}\\
Verify that the API aspect provides interfaces for retrieving the
history of queries for a user.
}
  
 \newpage 
\subsection{[LVV-10021] DMS-API-REQ-0040-V-01: Query Specification Retrieval\_1 }\label{lvv-10021}

\begin{longtable}{cccc}
\hline
\textbf{Jira Link} & \textbf{Assignee} & \textbf{Status} & \textbf{Test Cases}\\ \hline
\href{https://jira.lsstcorp.org/browse/LVV-10021}{LVV-10021} &
Gregory Dubois-Felsmann & Not Covered &
\begin{tabular}{c}
LVV-T815 \\
\end{tabular}
\\
\hline
\end{longtable}

\textbf{Verification Element Description:} \\
Undefined

{\footnotesize
\begin{longtable}{p{2.5cm}p{13.5cm}}
\hline
\multicolumn{2}{c}{\textbf{Requirement Details}}\\ \hline
Requirement ID & DMS-API-REQ-0040 \\ \cdashline{1-2}
Requirement Description &
\begin{minipage}[]{13cm}
The API Aspect shall provide interfaces that return an artifact
containing a complete specification for a query, and that permit that
artifact to be used at a later time to re-execute the same query.
\end{minipage}
\\ \cdashline{1-2}
Requirement Discussion &
\begin{minipage}[]{13cm}
The results of re-executing the same query depend on the database(s)
being queried. For Level 2 (Data Releases) the content of a data release
is nominally frozen upon release and the same query repeated later
should always return the same result. (We assume that if a correctable
error is found in a data release after its release it will only be
repaired, if at all, by adding additional database(s) or table(s) with
the corrected data, or just by providing correction recipies that could
be applied to the results of a query.)\\
For the continuously updated Level 1 database(s) update times will be
recorded that will permit queries to be repeated precisely, or,
optionally, with new data taken into account.\\
For user databases no guarantees can be made.
\end{minipage}
\\ \cdashline{1-2}
Requirement Priority &  \\ \cdashline{1-2}
Upper Level Requirement &
\begin{tabular}{cl}
\end{tabular}
\\ \hline
\end{longtable}
}


\subsubsection{Test Cases Summary}
\begin{longtable}{p{3cm}p{2.5cm}p{2.5cm}p{3cm}p{4cm}}
\toprule
\href{https://jira.lsstcorp.org/secure/Tests.jspa\#/testCase/LVV-T815}{LVV-T815} & \multicolumn{4}{p{12cm}}{ Verify retrieval of query specifications } \\ \hline
\textbf{Owner} & \textbf{Status} & \textbf{Version} & \textbf{Critical Event} & \textbf{Verification Type} \\ \hline
Jeffrey Carlin & Draft & 1 & false & Inspection \\ \hline
\end{longtable}
{\scriptsize
\textbf{Objective:}\\
Verify that the API Aspect provides interfaces that return an artifact
containing a complete specification for a query, and that permit that
artifact to be used at a later time to re-execute the same query.
}
  
 \newpage 
\subsection{[LVV-10022] DMS-API-REQ-0034-V-01: Butler Interface to Data Products\_1 }\label{lvv-10022}

\begin{longtable}{cccc}
\hline
\textbf{Jira Link} & \textbf{Assignee} & \textbf{Status} & \textbf{Test Cases}\\ \hline
\href{https://jira.lsstcorp.org/browse/LVV-10022}{LVV-10022} &
Gregory Dubois-Felsmann & Not Covered &
\begin{tabular}{c}
LVV-T816 \\
\end{tabular}
\\
\hline
\end{longtable}

\textbf{Verification Element Description:} \\
Undefined

{\footnotesize
\begin{longtable}{p{2.5cm}p{13.5cm}}
\hline
\multicolumn{2}{c}{\textbf{Requirement Details}}\\ \hline
Requirement ID & DMS-API-REQ-0034 \\ \cdashline{1-2}
Requirement Description &
\begin{minipage}[]{13cm}
The API Aspect shall provide connection between the Data Butler
(Generation 3) instances within notebooks hosted in a LDF instance and
backend file system, database, and object data stores within that same
LDF instance, for the purpose of allowing notebook aspect users to
access data release data products and user generated data products as
Python objects
\end{minipage}
\\ \cdashline{1-2}
Requirement Discussion &
\begin{minipage}[]{13cm}
See \citeds{LDM-556}.
\end{minipage}
\\ \cdashline{1-2}
Requirement Priority &  \\ \cdashline{1-2}
Upper Level Requirement &
\begin{tabular}{cl}
\end{tabular}
\\ \hline
\end{longtable}
}


\subsubsection{Test Cases Summary}
\begin{longtable}{p{3cm}p{2.5cm}p{2.5cm}p{3cm}p{4cm}}
\toprule
\href{https://jira.lsstcorp.org/secure/Tests.jspa\#/testCase/LVV-T816}{LVV-T816} & \multicolumn{4}{p{12cm}}{ Verify Butler interface to data products } \\ \hline
\textbf{Owner} & \textbf{Status} & \textbf{Version} & \textbf{Critical Event} & \textbf{Verification Type} \\ \hline
Jeffrey Carlin & Draft & 1 & false & Inspection \\ \hline
\end{longtable}
{\scriptsize
\textbf{Objective:}\\
Verify that the API Aspect provides a connection between the Data Butler
(Generation 3) instances within notebooks hosted in a LDF instance and
backend file system, database, and object data stores within that same
LDF instance, for the purpose of allowing notebook aspect users to
access data release data products and user generated data products as
Python objects.
}
  
 \newpage 
\subsection{[LVV-10023] DMS-API-REQ-0019-V-01: VOSpace Service\_1 }\label{lvv-10023}

\begin{longtable}{cccc}
\hline
\textbf{Jira Link} & \textbf{Assignee} & \textbf{Status} & \textbf{Test Cases}\\ \hline
\href{https://jira.lsstcorp.org/browse/LVV-10023}{LVV-10023} &
Gregory Dubois-Felsmann & Not Covered &
\begin{tabular}{c}
LVV-T817 \\
\end{tabular}
\\
\hline
\end{longtable}

\textbf{Verification Element Description:} \\
Undefined

{\footnotesize
\begin{longtable}{p{2.5cm}p{13.5cm}}
\hline
\multicolumn{2}{c}{\textbf{Requirement Details}}\\ \hline
Requirement ID & DMS-API-REQ-0019 \\ \cdashline{1-2}
Requirement Description &
\begin{minipage}[]{13cm}
The API Aspect Web APIs shall include an endpoint conforming to IVOA
VOSpace 2.0 for the purpose of persistence and retrieval of
user-generated file-oriented data products in the User Workspace defined
in DMS-LSP-REQ-0011.
\end{minipage}
\\ \cdashline{1-2}
Requirement Priority &  \\ \cdashline{1-2}
Upper Level Requirement &
\begin{tabular}{cl}
\end{tabular}
\\ \hline
\end{longtable}
}


\subsubsection{Test Cases Summary}
\begin{longtable}{p{3cm}p{2.5cm}p{2.5cm}p{3cm}p{4cm}}
\toprule
\href{https://jira.lsstcorp.org/secure/Tests.jspa\#/testCase/LVV-T817}{LVV-T817} & \multicolumn{4}{p{12cm}}{ Verify availability of VOSpace service } \\ \hline
\textbf{Owner} & \textbf{Status} & \textbf{Version} & \textbf{Critical Event} & \textbf{Verification Type} \\ \hline
Jeffrey Carlin & Draft & 1 & false & Inspection \\ \hline
\end{longtable}
{\scriptsize
\textbf{Objective:}\\
Verify that the API Aspect Web APIs include an endpoint conforming to
IVOA VOSpace 2.0 for the purpose of persistence and retrieval of
user-generated file-oriented data products in the User Workspace defined
in DMS-LSP-REQ-0011.
}
  
 \newpage 
\subsection{[LVV-10024] DMS-API-REQ-0020-V-01: WebDAV Service\_1 }\label{lvv-10024}

\begin{longtable}{cccc}
\hline
\textbf{Jira Link} & \textbf{Assignee} & \textbf{Status} & \textbf{Test Cases}\\ \hline
\href{https://jira.lsstcorp.org/browse/LVV-10024}{LVV-10024} &
Gregory Dubois-Felsmann & Not Covered &
\begin{tabular}{c}
LVV-T818 \\
\end{tabular}
\\
\hline
\end{longtable}

\textbf{Verification Element Description:} \\
Undefined

{\footnotesize
\begin{longtable}{p{2.5cm}p{13.5cm}}
\hline
\multicolumn{2}{c}{\textbf{Requirement Details}}\\ \hline
Requirement ID & DMS-API-REQ-0020 \\ \cdashline{1-2}
Requirement Description &
\begin{minipage}[]{13cm}
The API Aspect Web APIs shall include an endpoint conforming to WebDAV
for the purpose of persistence and retrieval of user-generated
file-oriented data products in the User Workspace defined in
DMS-LSP-REQ-0011.
\end{minipage}
\\ \cdashline{1-2}
Requirement Priority &  \\ \cdashline{1-2}
Upper Level Requirement &
\begin{tabular}{cl}
\end{tabular}
\\ \hline
\end{longtable}
}


\subsubsection{Test Cases Summary}
\begin{longtable}{p{3cm}p{2.5cm}p{2.5cm}p{3cm}p{4cm}}
\toprule
\href{https://jira.lsstcorp.org/secure/Tests.jspa\#/testCase/LVV-T818}{LVV-T818} & \multicolumn{4}{p{12cm}}{ Verify availability of WebDAV service } \\ \hline
\textbf{Owner} & \textbf{Status} & \textbf{Version} & \textbf{Critical Event} & \textbf{Verification Type} \\ \hline
Jeffrey Carlin & Draft & 1 & false & Inspection \\ \hline
\end{longtable}
{\scriptsize
\textbf{Objective:}\\
Verify that the API Aspect Web APIs include an endpoint conforming to
WebDAV for the purpose of persistence and retrieval of user-generated
file-oriented data products in the User Workspace defined in
DMS-LSP-REQ-0011.
}
  
 \newpage 
\subsection{[LVV-10025] DMS-API-REQ-0014-V-01: CSV Output for TAP\_1 }\label{lvv-10025}

\begin{longtable}{cccc}
\hline
\textbf{Jira Link} & \textbf{Assignee} & \textbf{Status} & \textbf{Test Cases}\\ \hline
\href{https://jira.lsstcorp.org/browse/LVV-10025}{LVV-10025} &
Gregory Dubois-Felsmann & Not Covered &
\begin{tabular}{c}
LVV-T823 \\
\end{tabular}
\\
\hline
\end{longtable}

\textbf{Verification Element Description:} \\
Undefined

{\footnotesize
\begin{longtable}{p{2.5cm}p{13.5cm}}
\hline
\multicolumn{2}{c}{\textbf{Requirement Details}}\\ \hline
Requirement ID & DMS-API-REQ-0014 \\ \cdashline{1-2}
Requirement Description &
\begin{minipage}[]{13cm}
The API Aspect TAP endpoint shall support CSV as and alternative
available output format. This output format is not required to meet
requirements otherwise in force on the return of table and column
metadata.
\end{minipage}
\\ \cdashline{1-2}
Requirement Discussion &
\begin{minipage}[]{13cm}
The CSV format inherently is unsuitable for returning rich metadata with
a table.
\end{minipage}
\\ \cdashline{1-2}
Requirement Priority &  \\ \cdashline{1-2}
Upper Level Requirement &
\begin{tabular}{cl}
\end{tabular}
\\ \hline
\end{longtable}
}


\subsubsection{Test Cases Summary}
\begin{longtable}{p{3cm}p{2.5cm}p{2.5cm}p{3cm}p{4cm}}
\toprule
\href{https://jira.lsstcorp.org/secure/Tests.jspa\#/testCase/LVV-T823}{LVV-T823} & \multicolumn{4}{p{12cm}}{ Verify CSV support for TAP outputs } \\ \hline
\textbf{Owner} & \textbf{Status} & \textbf{Version} & \textbf{Critical Event} & \textbf{Verification Type} \\ \hline
Jeffrey Carlin & Draft & 1 & false & Inspection \\ \hline
\end{longtable}
{\scriptsize
\textbf{Objective:}\\
Verify that the API Aspect TAP endpoint supports CSV as an alternative
available output format. This output format is not required to meet
requirements otherwise in force on the return of table and column
metadata.
}
  
 \newpage 
\subsection{[LVV-10026] DMS-API-REQ-0013-V-01: JSON Output for TAP\_1 }\label{lvv-10026}

\begin{longtable}{cccc}
\hline
\textbf{Jira Link} & \textbf{Assignee} & \textbf{Status} & \textbf{Test Cases}\\ \hline
\href{https://jira.lsstcorp.org/browse/LVV-10026}{LVV-10026} &
Gregory Dubois-Felsmann & Not Covered &
\begin{tabular}{c}
LVV-T822 \\
\end{tabular}
\\
\hline
\end{longtable}

\textbf{Verification Element Description:} \\
Undefined

{\footnotesize
\begin{longtable}{p{2.5cm}p{13.5cm}}
\hline
\multicolumn{2}{c}{\textbf{Requirement Details}}\\ \hline
Requirement ID & DMS-API-REQ-0013 \\ \cdashline{1-2}
Requirement Description &
\begin{minipage}[]{13cm}
The API Aspect TAP endpoint shall support JSON as an alternative
available output format
\end{minipage}
\\ \cdashline{1-2}
Requirement Priority &  \\ \cdashline{1-2}
Upper Level Requirement &
\begin{tabular}{cl}
\end{tabular}
\\ \hline
\end{longtable}
}


\subsubsection{Test Cases Summary}
\begin{longtable}{p{3cm}p{2.5cm}p{2.5cm}p{3cm}p{4cm}}
\toprule
\href{https://jira.lsstcorp.org/secure/Tests.jspa\#/testCase/LVV-T822}{LVV-T822} & \multicolumn{4}{p{12cm}}{ Verify JSON support for TAP outputs } \\ \hline
\textbf{Owner} & \textbf{Status} & \textbf{Version} & \textbf{Critical Event} & \textbf{Verification Type} \\ \hline
Jeffrey Carlin & Draft & 1 & false & Inspection \\ \hline
\end{longtable}
{\scriptsize
\textbf{Objective:}\\
Verify that the API Aspect TAP endpoint supports JSON as an alternative
available output format.
}
  
 \newpage 
\subsection{[LVV-10027] DMS-API-REQ-0015-V-01: SQLite Output for TAP\_1 }\label{lvv-10027}

\begin{longtable}{cccc}
\hline
\textbf{Jira Link} & \textbf{Assignee} & \textbf{Status} & \textbf{Test Cases}\\ \hline
\href{https://jira.lsstcorp.org/browse/LVV-10027}{LVV-10027} &
Gregory Dubois-Felsmann & Not Covered &
\begin{tabular}{c}
LVV-T824 \\
\end{tabular}
\\
\hline
\end{longtable}

\textbf{Verification Element Description:} \\
Undefined

{\footnotesize
\begin{longtable}{p{2.5cm}p{13.5cm}}
\hline
\multicolumn{2}{c}{\textbf{Requirement Details}}\\ \hline
Requirement ID & DMS-API-REQ-0015 \\ \cdashline{1-2}
Requirement Description &
\begin{minipage}[]{13cm}
The API Aspect TAP endpoint SHOULD support SQLite as an alternative
available output format
\end{minipage}
\\ \cdashline{1-2}
Requirement Discussion &
\begin{minipage}[]{13cm}
The mandatory status of this requirement should be settled one way or
the other as soon as possible. It is a candidate for a efficient and
readily consumed format for large results with rich metadata.
\end{minipage}
\\ \cdashline{1-2}
Requirement Priority &  \\ \cdashline{1-2}
Upper Level Requirement &
\begin{tabular}{cl}
\end{tabular}
\\ \hline
\end{longtable}
}


\subsubsection{Test Cases Summary}
\begin{longtable}{p{3cm}p{2.5cm}p{2.5cm}p{3cm}p{4cm}}
\toprule
\href{https://jira.lsstcorp.org/secure/Tests.jspa\#/testCase/LVV-T824}{LVV-T824} & \multicolumn{4}{p{12cm}}{ Verify SQLite support for TAP outputs } \\ \hline
\textbf{Owner} & \textbf{Status} & \textbf{Version} & \textbf{Critical Event} & \textbf{Verification Type} \\ \hline
Jeffrey Carlin & Draft & 1 & false & Inspection \\ \hline
\end{longtable}
{\scriptsize
\textbf{Objective:}\\
Verify that the API Aspect TAP endpoint supports SQLite as an
alternative available output format.
}
  
 \newpage 
\subsection{[LVV-10028] DMS-API-REQ-0012-V-01: VOTable BINARY2 Payload\_1 }\label{lvv-10028}

\begin{longtable}{cccc}
\hline
\textbf{Jira Link} & \textbf{Assignee} & \textbf{Status} & \textbf{Test Cases}\\ \hline
\href{https://jira.lsstcorp.org/browse/LVV-10028}{LVV-10028} &
Gregory Dubois-Felsmann & Not Covered &
\begin{tabular}{c}
LVV-T821 \\
\end{tabular}
\\
\hline
\end{longtable}

\textbf{Verification Element Description:} \\
Undefined

{\footnotesize
\begin{longtable}{p{2.5cm}p{13.5cm}}
\hline
\multicolumn{2}{c}{\textbf{Requirement Details}}\\ \hline
Requirement ID & DMS-API-REQ-0012 \\ \cdashline{1-2}
Requirement Description &
\begin{minipage}[]{13cm}
API Aspect services that support returning results in VOTable format
shall support the return of a VOTable data payload in the BINARY2
serialization.
\end{minipage}
\\ \cdashline{1-2}
Requirement Discussion &
\begin{minipage}[]{13cm}
This payload form supports larger tabular results but is most likely
still not optimal for the largest queries. The API Aspect team is still
investigating additional options, including the use of the FITS
serialization, as well as the use of non-VOTable formats, for efficient
and metadata-rich tabular results.
\end{minipage}
\\ \cdashline{1-2}
Requirement Priority &  \\ \cdashline{1-2}
Upper Level Requirement &
\begin{tabular}{cl}
\end{tabular}
\\ \hline
\end{longtable}
}


\subsubsection{Test Cases Summary}
\begin{longtable}{p{3cm}p{2.5cm}p{2.5cm}p{3cm}p{4cm}}
\toprule
\href{https://jira.lsstcorp.org/secure/Tests.jspa\#/testCase/LVV-T821}{LVV-T821} & \multicolumn{4}{p{12cm}}{ Verify support for VOTable BINARY2 payload } \\ \hline
\textbf{Owner} & \textbf{Status} & \textbf{Version} & \textbf{Critical Event} & \textbf{Verification Type} \\ \hline
Jeffrey Carlin & Draft & 1 & false & Inspection \\ \hline
\end{longtable}
{\scriptsize
\textbf{Objective:}\\
Verify that the API Aspect services that support returning results in
VOTable format support the return of a VOTable data payload in the
BINARY2 serialization.
}
  
 \newpage 
\subsection{[LVV-10029] DMS-API-REQ-0010-V-01: VOTable Output for TAP\_1 }\label{lvv-10029}

\begin{longtable}{cccc}
\hline
\textbf{Jira Link} & \textbf{Assignee} & \textbf{Status} & \textbf{Test Cases}\\ \hline
\href{https://jira.lsstcorp.org/browse/LVV-10029}{LVV-10029} &
Gregory Dubois-Felsmann & Not Covered &
\begin{tabular}{c}
LVV-T819 \\
\end{tabular}
\\
\hline
\end{longtable}

\textbf{Verification Element Description:} \\
Undefined

{\footnotesize
\begin{longtable}{p{2.5cm}p{13.5cm}}
\hline
\multicolumn{2}{c}{\textbf{Requirement Details}}\\ \hline
Requirement ID & DMS-API-REQ-0010 \\ \cdashline{1-2}
Requirement Description &
\begin{minipage}[]{13cm}
The API Aspect TAP endpoint shall support IVOA VOTable 1.3 as an
available output format
\end{minipage}
\\ \cdashline{1-2}
Requirement Priority &  \\ \cdashline{1-2}
Upper Level Requirement &
\begin{tabular}{cl}
\end{tabular}
\\ \hline
\end{longtable}
}


\subsubsection{Test Cases Summary}
\begin{longtable}{p{3cm}p{2.5cm}p{2.5cm}p{3cm}p{4cm}}
\toprule
\href{https://jira.lsstcorp.org/secure/Tests.jspa\#/testCase/LVV-T819}{LVV-T819} & \multicolumn{4}{p{12cm}}{ Verify VOTable 1.3 support } \\ \hline
\textbf{Owner} & \textbf{Status} & \textbf{Version} & \textbf{Critical Event} & \textbf{Verification Type} \\ \hline
Colin Slater & Draft & 1 & false & Inspection \\ \hline
\end{longtable}
{\scriptsize
\textbf{Objective:}\\
Verify that the API Aspect TAP endpoint supports IVOA VOTable 1.3 as an
available output format.
}
  
 \newpage 
\subsection{[LVV-10030] DMS-API-REQ-0011-V-01: VOTable TABLEDATA Payload\_1 }\label{lvv-10030}

\begin{longtable}{cccc}
\hline
\textbf{Jira Link} & \textbf{Assignee} & \textbf{Status} & \textbf{Test Cases}\\ \hline
\href{https://jira.lsstcorp.org/browse/LVV-10030}{LVV-10030} &
Gregory Dubois-Felsmann & Not Covered &
\begin{tabular}{c}
LVV-T820 \\
\end{tabular}
\\
\hline
\end{longtable}

\textbf{Verification Element Description:} \\
Undefined

{\footnotesize
\begin{longtable}{p{2.5cm}p{13.5cm}}
\hline
\multicolumn{2}{c}{\textbf{Requirement Details}}\\ \hline
Requirement ID & DMS-API-REQ-0011 \\ \cdashline{1-2}
Requirement Description &
\begin{minipage}[]{13cm}
API Aspect services that support returning results in VOTable format
shall support the return of a VOTable data payload in the XML-based
TABLEDATA serialization.
\end{minipage}
\\ \cdashline{1-2}
Requirement Discussion &
\begin{minipage}[]{13cm}
This payload form is not intended for large tables, but is provided for
compatibility with community tools. The API Aspect TAP service may place
an upper bound on the size of a table that can be returned in this
format.
\end{minipage}
\\ \cdashline{1-2}
Requirement Priority &  \\ \cdashline{1-2}
Upper Level Requirement &
\begin{tabular}{cl}
\end{tabular}
\\ \hline
\end{longtable}
}


\subsubsection{Test Cases Summary}
\begin{longtable}{p{3cm}p{2.5cm}p{2.5cm}p{3cm}p{4cm}}
\toprule
\href{https://jira.lsstcorp.org/secure/Tests.jspa\#/testCase/LVV-T820}{LVV-T820} & \multicolumn{4}{p{12cm}}{ Verify support for VOTable TABLEDATA payload } \\ \hline
\textbf{Owner} & \textbf{Status} & \textbf{Version} & \textbf{Critical Event} & \textbf{Verification Type} \\ \hline
Jeffrey Carlin & Draft & 1 & false & Inspection \\ \hline
\end{longtable}
{\scriptsize
\textbf{Objective:}\\
Verify that API Aspect services that support returning results in
VOTable format support the return of a VOTable data payload in the
XML-based TABLEDATA serialization.
}
  
 \newpage 
\subsection{[LVV-10031] DMS-API-REQ-0033-V-01: Deletion from Workspace\_1 }\label{lvv-10031}

\begin{longtable}{cccc}
\hline
\textbf{Jira Link} & \textbf{Assignee} & \textbf{Status} & \textbf{Test Cases}\\ \hline
\href{https://jira.lsstcorp.org/browse/LVV-10031}{LVV-10031} &
Gregory Dubois-Felsmann & Not Covered &
\begin{tabular}{c}
LVV-T827 \\
\end{tabular}
\\
\hline
\end{longtable}

\textbf{Verification Element Description:} \\
Undefined

{\footnotesize
\begin{longtable}{p{2.5cm}p{13.5cm}}
\hline
\multicolumn{2}{c}{\textbf{Requirement Details}}\\ \hline
Requirement ID & DMS-API-REQ-0033 \\ \cdashline{1-2}
Requirement Description &
\begin{minipage}[]{13cm}
The API Aspect shall provide a capability for users to drop previously
uploaded user catalog data products
\end{minipage}
\\ \cdashline{1-2}
Requirement Priority &  \\ \cdashline{1-2}
Upper Level Requirement &
\begin{tabular}{cl}
\end{tabular}
\\ \hline
\end{longtable}
}


\subsubsection{Test Cases Summary}
\begin{longtable}{p{3cm}p{2.5cm}p{2.5cm}p{3cm}p{4cm}}
\toprule
\href{https://jira.lsstcorp.org/secure/Tests.jspa\#/testCase/LVV-T827}{LVV-T827} & \multicolumn{4}{p{12cm}}{ Verify ability to drop catalogs from Workspace } \\ \hline
\textbf{Owner} & \textbf{Status} & \textbf{Version} & \textbf{Critical Event} & \textbf{Verification Type} \\ \hline
Jeffrey Carlin & Draft & 1 & false & Inspection \\ \hline
\end{longtable}
{\scriptsize
\textbf{Objective:}\\
Verify that the API Aspect provides a capability for users to drop
previously uploaded user catalog data products.
}
  
 \newpage 
\subsection{[LVV-10032] DMS-API-REQ-0031-V-01: Tabular Result Download to Workspace\_1 }\label{lvv-10032}

\begin{longtable}{cccc}
\hline
\textbf{Jira Link} & \textbf{Assignee} & \textbf{Status} & \textbf{Test Cases}\\ \hline
\href{https://jira.lsstcorp.org/browse/LVV-10032}{LVV-10032} &
Gregory Dubois-Felsmann & Not Covered &
\begin{tabular}{c}
LVV-T825 \\
\end{tabular}
\\
\hline
\end{longtable}

\textbf{Verification Element Description:} \\
Undefined

{\footnotesize
\begin{longtable}{p{2.5cm}p{13.5cm}}
\hline
\multicolumn{2}{c}{\textbf{Requirement Details}}\\ \hline
Requirement ID & DMS-API-REQ-0031 \\ \cdashline{1-2}
Requirement Description &
\begin{minipage}[]{13cm}
The API Aspect shall provide a capability for users to save their query
results as VOTables in their allocated VOSpace, subject to limitations
of a resource quota system
\end{minipage}
\\ \cdashline{1-2}
Requirement Discussion &
\begin{minipage}[]{13cm}
Or any of the other supported formats?
\end{minipage}
\\ \cdashline{1-2}
Requirement Priority &  \\ \cdashline{1-2}
Upper Level Requirement &
\begin{tabular}{cl}
\end{tabular}
\\ \hline
\end{longtable}
}


\subsubsection{Test Cases Summary}
\begin{longtable}{p{3cm}p{2.5cm}p{2.5cm}p{3cm}p{4cm}}
\toprule
\href{https://jira.lsstcorp.org/secure/Tests.jspa\#/testCase/LVV-T825}{LVV-T825} & \multicolumn{4}{p{12cm}}{ Verify support for tabular result download to Workspace } \\ \hline
\textbf{Owner} & \textbf{Status} & \textbf{Version} & \textbf{Critical Event} & \textbf{Verification Type} \\ \hline
Jeffrey Carlin & Draft & 1 & false & Inspection \\ \hline
\end{longtable}
{\scriptsize
\textbf{Objective:}\\
Verify that the API Aspect provides a capability for users to save their
query results as VOTables in their allocated VOSpace, subject to
limitations of a resource quota system.
}
  
 \newpage 
\subsection{[LVV-10033] DMS-API-REQ-0032-V-01: Tabular Upload to Workspace\_1 }\label{lvv-10033}

\begin{longtable}{cccc}
\hline
\textbf{Jira Link} & \textbf{Assignee} & \textbf{Status} & \textbf{Test Cases}\\ \hline
\href{https://jira.lsstcorp.org/browse/LVV-10033}{LVV-10033} &
Gregory Dubois-Felsmann & Not Covered &
\begin{tabular}{c}
LVV-T826 \\
\end{tabular}
\\
\hline
\end{longtable}

\textbf{Verification Element Description:} \\
Undefined

{\footnotesize
\begin{longtable}{p{2.5cm}p{13.5cm}}
\hline
\multicolumn{2}{c}{\textbf{Requirement Details}}\\ \hline
Requirement ID & DMS-API-REQ-0032 \\ \cdashline{1-2}
Requirement Description &
\begin{minipage}[]{13cm}
The API Aspect shall provide a capability for users to upload catalog
data products (formatted as VOTables) residing within their allocated
VOSpace, such that the catalog products after upload may be joined in
queries against data release catalog products, subject to limitations of
a resource quota system
\end{minipage}
\\ \cdashline{1-2}
Requirement Discussion &
\begin{minipage}[]{13cm}
Or any of the other supported formats?
\end{minipage}
\\ \cdashline{1-2}
Requirement Priority &  \\ \cdashline{1-2}
Upper Level Requirement &
\begin{tabular}{cl}
\end{tabular}
\\ \hline
\end{longtable}
}


\subsubsection{Test Cases Summary}
\begin{longtable}{p{3cm}p{2.5cm}p{2.5cm}p{3cm}p{4cm}}
\toprule
\href{https://jira.lsstcorp.org/secure/Tests.jspa\#/testCase/LVV-T826}{LVV-T826} & \multicolumn{4}{p{12cm}}{ Verify support for tabular upload to Workspace } \\ \hline
\textbf{Owner} & \textbf{Status} & \textbf{Version} & \textbf{Critical Event} & \textbf{Verification Type} \\ \hline
Jeffrey Carlin & Draft & 1 & false & Inspection \\ \hline
\end{longtable}
{\scriptsize
\textbf{Objective:}\\
Verify that the API Aspect provides a capability for users to upload
catalog data products (formatted as VOTables) residing within their
allocated VOSpace, such that the catalog products after upload may be
joined in queries against data release catalog products, subject to
limitations of a resource quota system.
}
  
 \newpage 
\subsection{[LVV-10034] DMS-API-REQ-0003-V-01: Authentication\_1 }\label{lvv-10034}

\begin{longtable}{cccc}
\hline
\textbf{Jira Link} & \textbf{Assignee} & \textbf{Status} & \textbf{Test Cases}\\ \hline
\href{https://jira.lsstcorp.org/browse/LVV-10034}{LVV-10034} &
Gregory Dubois-Felsmann & Not Covered &
\begin{tabular}{c}
LVV-T829 \\
LVV-T1437 \\
\end{tabular}
\\
\hline
\end{longtable}

\textbf{Verification Element Description:} \\
Undefined

{\footnotesize
\begin{longtable}{p{2.5cm}p{13.5cm}}
\hline
\multicolumn{2}{c}{\textbf{Requirement Details}}\\ \hline
Requirement ID & DMS-API-REQ-0003 \\ \cdashline{1-2}
Requirement Description &
\begin{minipage}[]{13cm}
The API Aspect Web APIs shall accept authenticated requests for the
purpose of establishing user identity.
\end{minipage}
\\ \cdashline{1-2}
Requirement Priority &  \\ \cdashline{1-2}
Upper Level Requirement &
\begin{tabular}{cl}
\end{tabular}
\\ \hline
\end{longtable}
}


\subsubsection{Test Cases Summary}
\begin{longtable}{p{3cm}p{2.5cm}p{2.5cm}p{3cm}p{4cm}}
\toprule
\href{https://jira.lsstcorp.org/secure/Tests.jspa\#/testCase/LVV-T829}{LVV-T829} & \multicolumn{4}{p{12cm}}{ Verify API authentication } \\ \hline
\textbf{Owner} & \textbf{Status} & \textbf{Version} & \textbf{Critical Event} & \textbf{Verification Type} \\ \hline
Jeffrey Carlin & Draft & 1 & false & Inspection \\ \hline
\end{longtable}
{\scriptsize
\textbf{Objective:}\\
Verify that the API Aspect Web APIs accept authenticated requests for
the purpose of establishing user identity. ~
}
\begin{longtable}{p{3cm}p{2.5cm}p{2.5cm}p{3cm}p{4cm}}
\toprule
\href{https://jira.lsstcorp.org/secure/Tests.jspa\#/testCase/LVV-T1437}{LVV-T1437} & \multicolumn{4}{p{12cm}}{ LDM-503-10a: API Aspect tests for LSP with Authentication and TAP
milestone } \\ \hline
\textbf{Owner} & \textbf{Status} & \textbf{Version} & \textbf{Critical Event} & \textbf{Verification Type} \\ \hline
Gregory Dubois-Felsmann & Defined & 1 & false & Test \\ \hline
\end{longtable}
{\scriptsize
\textbf{Objective:}\\
This test case verifies that the TAP service in the API Aspect of the
Science Platform is accessible to authorized users through a login
process, and that TAP searches can be performed using the IVOA TAP
protocol from remote sites.\\[2\baselineskip]In so doing and in
conjunction with the other LDM-503-10a test cases collected under
LVV-P48, it addresses all or part of the following requirements:

\begin{itemize}
\tightlist
\item
  DMS-LSP-REQ-0004, DMS-LSP-REQ-0005, DMS-LSP-REQ-0006,
  DMS-LSP-REQ-0020, DMS-LSP-REQ-0022, DMS-LSP-REQ-0023, DMS-LSP-REQ-0024
\item
  DMS-API-REQ-0003, DMS-API-REQ-0004, DMS-API-REQ-0006,
  DMS-API-REQ-0007, DMS-API-REQ-0008, DMS-API-REQ-0009,
  DMS-API-REQ-0023, and DMS-API-REQ-0039, primarily
\end{itemize}

Note this test was not designed to perform a full verification of the
above requirements, but rather to demonstrate having reached a certain
level of partial capability during construction.
}
  
 \newpage 
\subsection{[LVV-10035] DMS-API-REQ-0004-V-01: Authorization\_1 }\label{lvv-10035}

\begin{longtable}{cccc}
\hline
\textbf{Jira Link} & \textbf{Assignee} & \textbf{Status} & \textbf{Test Cases}\\ \hline
\href{https://jira.lsstcorp.org/browse/LVV-10035}{LVV-10035} &
Gregory Dubois-Felsmann & Not Covered &
\begin{tabular}{c}
LVV-T830 \\
LVV-T1437 \\
\end{tabular}
\\
\hline
\end{longtable}

\textbf{Verification Element Description:} \\
Undefined

{\footnotesize
\begin{longtable}{p{2.5cm}p{13.5cm}}
\hline
\multicolumn{2}{c}{\textbf{Requirement Details}}\\ \hline
Requirement ID & DMS-API-REQ-0004 \\ \cdashline{1-2}
Requirement Description &
\begin{minipage}[]{13cm}
The API Aspect Web APIs shall interact with project authorization
infrastructure for the purpose of establishing authorized use.
\end{minipage}
\\ \cdashline{1-2}
Requirement Priority &  \\ \cdashline{1-2}
Upper Level Requirement &
\begin{tabular}{cl}
\end{tabular}
\\ \hline
\end{longtable}
}


\subsubsection{Test Cases Summary}
\begin{longtable}{p{3cm}p{2.5cm}p{2.5cm}p{3cm}p{4cm}}
\toprule
\href{https://jira.lsstcorp.org/secure/Tests.jspa\#/testCase/LVV-T830}{LVV-T830} & \multicolumn{4}{p{12cm}}{ Verify API uses project authorization infrastructure } \\ \hline
\textbf{Owner} & \textbf{Status} & \textbf{Version} & \textbf{Critical Event} & \textbf{Verification Type} \\ \hline
Jeffrey Carlin & Draft & 1 & false & Inspection \\ \hline
\end{longtable}
{\scriptsize
\textbf{Objective:}\\
Verify that the API Aspect Web APIs interact with project authorization
infrastructure for the purpose of establishing authorized use.
}
\begin{longtable}{p{3cm}p{2.5cm}p{2.5cm}p{3cm}p{4cm}}
\toprule
\href{https://jira.lsstcorp.org/secure/Tests.jspa\#/testCase/LVV-T1437}{LVV-T1437} & \multicolumn{4}{p{12cm}}{ LDM-503-10a: API Aspect tests for LSP with Authentication and TAP
milestone } \\ \hline
\textbf{Owner} & \textbf{Status} & \textbf{Version} & \textbf{Critical Event} & \textbf{Verification Type} \\ \hline
Gregory Dubois-Felsmann & Defined & 1 & false & Test \\ \hline
\end{longtable}
{\scriptsize
\textbf{Objective:}\\
This test case verifies that the TAP service in the API Aspect of the
Science Platform is accessible to authorized users through a login
process, and that TAP searches can be performed using the IVOA TAP
protocol from remote sites.\\[2\baselineskip]In so doing and in
conjunction with the other LDM-503-10a test cases collected under
LVV-P48, it addresses all or part of the following requirements:

\begin{itemize}
\tightlist
\item
  DMS-LSP-REQ-0004, DMS-LSP-REQ-0005, DMS-LSP-REQ-0006,
  DMS-LSP-REQ-0020, DMS-LSP-REQ-0022, DMS-LSP-REQ-0023, DMS-LSP-REQ-0024
\item
  DMS-API-REQ-0003, DMS-API-REQ-0004, DMS-API-REQ-0006,
  DMS-API-REQ-0007, DMS-API-REQ-0008, DMS-API-REQ-0009,
  DMS-API-REQ-0023, and DMS-API-REQ-0039, primarily
\end{itemize}

Note this test was not designed to perform a full verification of the
above requirements, but rather to demonstrate having reached a certain
level of partial capability during construction.
}
  
 \newpage 
\subsection{[LVV-10036] DMS-API-REQ-0005-V-01: Secure Implementation\_1 }\label{lvv-10036}

\begin{longtable}{cccc}
\hline
\textbf{Jira Link} & \textbf{Assignee} & \textbf{Status} & \textbf{Test Cases}\\ \hline
\href{https://jira.lsstcorp.org/browse/LVV-10036}{LVV-10036} &
Gregory Dubois-Felsmann & Not Covered &
\begin{tabular}{c}
LVV-T831 \\
\end{tabular}
\\
\hline
\end{longtable}

\textbf{Verification Element Description:} \\
Undefined

{\footnotesize
\begin{longtable}{p{2.5cm}p{13.5cm}}
\hline
\multicolumn{2}{c}{\textbf{Requirement Details}}\\ \hline
Requirement ID & DMS-API-REQ-0005 \\ \cdashline{1-2}
Requirement Description &
\begin{minipage}[]{13cm}
The API Aspect Web APIs shall prevent users from circumventing
authorization controls.
\end{minipage}
\\ \cdashline{1-2}
Requirement Priority &  \\ \cdashline{1-2}
Upper Level Requirement &
\begin{tabular}{cl}
\end{tabular}
\\ \hline
\end{longtable}
}


\subsubsection{Test Cases Summary}
\begin{longtable}{p{3cm}p{2.5cm}p{2.5cm}p{3cm}p{4cm}}
\toprule
\href{https://jira.lsstcorp.org/secure/Tests.jspa\#/testCase/LVV-T831}{LVV-T831} & \multicolumn{4}{p{12cm}}{ Verify secure implementation of APIs } \\ \hline
\textbf{Owner} & \textbf{Status} & \textbf{Version} & \textbf{Critical Event} & \textbf{Verification Type} \\ \hline
Jeffrey Carlin & Draft & 1 & false & Inspection \\ \hline
\end{longtable}
{\scriptsize
\textbf{Objective:}\\
Verify that the API Aspect Web APIs prevent users from circumventing
authorization controls.
}
  
 \newpage 
\subsection{[LVV-10037] DMS-API-REQ-0001-V-01: Secure Protocols\_1 }\label{lvv-10037}

\begin{longtable}{cccc}
\hline
\textbf{Jira Link} & \textbf{Assignee} & \textbf{Status} & \textbf{Test Cases}\\ \hline
\href{https://jira.lsstcorp.org/browse/LVV-10037}{LVV-10037} &
Gregory Dubois-Felsmann & Not Covered &
\begin{tabular}{c}
LVV-T828 \\
LVV-T1437 \\
\end{tabular}
\\
\hline
\end{longtable}

\textbf{Verification Element Description:} \\
Undefined

{\footnotesize
\begin{longtable}{p{2.5cm}p{13.5cm}}
\hline
\multicolumn{2}{c}{\textbf{Requirement Details}}\\ \hline
Requirement ID & DMS-API-REQ-0001 \\ \cdashline{1-2}
Requirement Description &
\begin{minipage}[]{13cm}
The API Aspect Web APIs shall be accessible through HTTPS endpoints.
\end{minipage}
\\ \cdashline{1-2}
Requirement Priority &  \\ \cdashline{1-2}
Upper Level Requirement &
\begin{tabular}{cl}
\end{tabular}
\\ \hline
\end{longtable}
}


\subsubsection{Test Cases Summary}
\begin{longtable}{p{3cm}p{2.5cm}p{2.5cm}p{3cm}p{4cm}}
\toprule
\href{https://jira.lsstcorp.org/secure/Tests.jspa\#/testCase/LVV-T828}{LVV-T828} & \multicolumn{4}{p{12cm}}{ Verify API uses secure protocols } \\ \hline
\textbf{Owner} & \textbf{Status} & \textbf{Version} & \textbf{Critical Event} & \textbf{Verification Type} \\ \hline
Jeffrey Carlin & Draft & 1 & false & Inspection \\ \hline
\end{longtable}
{\scriptsize
\textbf{Objective:}\\
Verify that the API Aspect Web APIs are accessible through HTTPS
endpoints.
}
\begin{longtable}{p{3cm}p{2.5cm}p{2.5cm}p{3cm}p{4cm}}
\toprule
\href{https://jira.lsstcorp.org/secure/Tests.jspa\#/testCase/LVV-T1437}{LVV-T1437} & \multicolumn{4}{p{12cm}}{ LDM-503-10a: API Aspect tests for LSP with Authentication and TAP
milestone } \\ \hline
\textbf{Owner} & \textbf{Status} & \textbf{Version} & \textbf{Critical Event} & \textbf{Verification Type} \\ \hline
Gregory Dubois-Felsmann & Defined & 1 & false & Test \\ \hline
\end{longtable}
{\scriptsize
\textbf{Objective:}\\
This test case verifies that the TAP service in the API Aspect of the
Science Platform is accessible to authorized users through a login
process, and that TAP searches can be performed using the IVOA TAP
protocol from remote sites.\\[2\baselineskip]In so doing and in
conjunction with the other LDM-503-10a test cases collected under
LVV-P48, it addresses all or part of the following requirements:

\begin{itemize}
\tightlist
\item
  DMS-LSP-REQ-0004, DMS-LSP-REQ-0005, DMS-LSP-REQ-0006,
  DMS-LSP-REQ-0020, DMS-LSP-REQ-0022, DMS-LSP-REQ-0023, DMS-LSP-REQ-0024
\item
  DMS-API-REQ-0003, DMS-API-REQ-0004, DMS-API-REQ-0006,
  DMS-API-REQ-0007, DMS-API-REQ-0008, DMS-API-REQ-0009,
  DMS-API-REQ-0023, and DMS-API-REQ-0039, primarily
\end{itemize}

Note this test was not designed to perform a full verification of the
above requirements, but rather to demonstrate having reached a certain
level of partial capability during construction.
}
  
 \newpage 
\subsection{[LVV-10038] DMS-API-REQ-0035-V-01: Containerized Deployment\_1 }\label{lvv-10038}

\begin{longtable}{cccc}
\hline
\textbf{Jira Link} & \textbf{Assignee} & \textbf{Status} & \textbf{Test Cases}\\ \hline
\href{https://jira.lsstcorp.org/browse/LVV-10038}{LVV-10038} &
Gregory Dubois-Felsmann & Not Covered &
\begin{tabular}{c}
LVV-T832 \\
\end{tabular}
\\
\hline
\end{longtable}

\textbf{Verification Element Description:} \\
Undefined

{\footnotesize
\begin{longtable}{p{2.5cm}p{13.5cm}}
\hline
\multicolumn{2}{c}{\textbf{Requirement Details}}\\ \hline
Requirement ID & DMS-API-REQ-0035 \\ \cdashline{1-2}
Requirement Description &
\begin{minipage}[]{13cm}
The API Aspect services shall be deliverered as containerized
applications.
\end{minipage}
\\ \cdashline{1-2}
Requirement Priority &  \\ \cdashline{1-2}
Upper Level Requirement &
\begin{tabular}{cl}
\end{tabular}
\\ \hline
\end{longtable}
}


\subsubsection{Test Cases Summary}
\begin{longtable}{p{3cm}p{2.5cm}p{2.5cm}p{3cm}p{4cm}}
\toprule
\href{https://jira.lsstcorp.org/secure/Tests.jspa\#/testCase/LVV-T832}{LVV-T832} & \multicolumn{4}{p{12cm}}{ Verify containerized deployment of API services } \\ \hline
\textbf{Owner} & \textbf{Status} & \textbf{Version} & \textbf{Critical Event} & \textbf{Verification Type} \\ \hline
Jeffrey Carlin & Draft & 1 & false & Inspection \\ \hline
\end{longtable}
{\scriptsize
\textbf{Objective:}\\
Verify that the API Aspect services are delivered as containerized
applications.~
}
  
 \newpage 
\subsection{[LVV-10039] DMS-API-REQ-0037-V-01: Logging and Monitoring\_1 }\label{lvv-10039}

\begin{longtable}{cccc}
\hline
\textbf{Jira Link} & \textbf{Assignee} & \textbf{Status} & \textbf{Test Cases}\\ \hline
\href{https://jira.lsstcorp.org/browse/LVV-10039}{LVV-10039} &
Gregory Dubois-Felsmann & Not Covered &
\begin{tabular}{c}
LVV-T835 \\
\end{tabular}
\\
\hline
\end{longtable}

\textbf{Verification Element Description:} \\
Undefined

{\footnotesize
\begin{longtable}{p{2.5cm}p{13.5cm}}
\hline
\multicolumn{2}{c}{\textbf{Requirement Details}}\\ \hline
Requirement ID & DMS-API-REQ-0037 \\ \cdashline{1-2}
Requirement Description &
\begin{minipage}[]{13cm}
The API Aspect services shall provide logging and monitoring
capabilities for the purpose of supporting service operators
\end{minipage}
\\ \cdashline{1-2}
Requirement Priority &  \\ \cdashline{1-2}
Upper Level Requirement &
\begin{tabular}{cl}
\end{tabular}
\\ \hline
\end{longtable}
}


\subsubsection{Test Cases Summary}
\begin{longtable}{p{3cm}p{2.5cm}p{2.5cm}p{3cm}p{4cm}}
\toprule
\href{https://jira.lsstcorp.org/secure/Tests.jspa\#/testCase/LVV-T835}{LVV-T835} & \multicolumn{4}{p{12cm}}{ Verify API logging and monitoring } \\ \hline
\textbf{Owner} & \textbf{Status} & \textbf{Version} & \textbf{Critical Event} & \textbf{Verification Type} \\ \hline
Jeffrey Carlin & Draft & 1 & false & Inspection \\ \hline
\end{longtable}
{\scriptsize
\textbf{Objective:}\\
Verify that the API Aspect services provide logging and monitoring
capabilities for the purpose of supporting service operators.
}
  
 \newpage 
\subsection{[LVV-10040] DMS-API-REQ-0002-V-01: Result Compression\_1 }\label{lvv-10040}

\begin{longtable}{cccc}
\hline
\textbf{Jira Link} & \textbf{Assignee} & \textbf{Status} & \textbf{Test Cases}\\ \hline
\href{https://jira.lsstcorp.org/browse/LVV-10040}{LVV-10040} &
Gregory Dubois-Felsmann & Not Covered &
\begin{tabular}{c}
LVV-T833 \\
\end{tabular}
\\
\hline
\end{longtable}

\textbf{Verification Element Description:} \\
Undefined

{\footnotesize
\begin{longtable}{p{2.5cm}p{13.5cm}}
\hline
\multicolumn{2}{c}{\textbf{Requirement Details}}\\ \hline
Requirement ID & DMS-API-REQ-0002 \\ \cdashline{1-2}
Requirement Description &
\begin{minipage}[]{13cm}
The API Aspect Web APIs shall support gzip HTTP content-encoding for the
purpose of returning compressed data.
\end{minipage}
\\ \cdashline{1-2}
Requirement Priority &  \\ \cdashline{1-2}
Upper Level Requirement &
\begin{tabular}{cl}
\end{tabular}
\\ \hline
\end{longtable}
}


\subsubsection{Test Cases Summary}
\begin{longtable}{p{3cm}p{2.5cm}p{2.5cm}p{3cm}p{4cm}}
\toprule
\href{https://jira.lsstcorp.org/secure/Tests.jspa\#/testCase/LVV-T833}{LVV-T833} & \multicolumn{4}{p{12cm}}{ Verify support for compression of API results } \\ \hline
\textbf{Owner} & \textbf{Status} & \textbf{Version} & \textbf{Critical Event} & \textbf{Verification Type} \\ \hline
Jeffrey Carlin & Draft & 1 & false & Inspection \\ \hline
\end{longtable}
{\scriptsize
\textbf{Objective:}\\
Verify that the API Aspect Web APIs support gzip HTTP content-encoding
for the purpose of returning compressed data.
}
  
 \newpage 
\subsection{[LVV-10041] DMS-API-REQ-0036-V-01: Upgradability\_1 }\label{lvv-10041}

\begin{longtable}{cccc}
\hline
\textbf{Jira Link} & \textbf{Assignee} & \textbf{Status} & \textbf{Test Cases}\\ \hline
\href{https://jira.lsstcorp.org/browse/LVV-10041}{LVV-10041} &
Gregory Dubois-Felsmann & Not Covered &
\begin{tabular}{c}
LVV-T834 \\
\end{tabular}
\\
\hline
\end{longtable}

\textbf{Verification Element Description:} \\
Undefined

{\footnotesize
\begin{longtable}{p{2.5cm}p{13.5cm}}
\hline
\multicolumn{2}{c}{\textbf{Requirement Details}}\\ \hline
Requirement ID & DMS-API-REQ-0036 \\ \cdashline{1-2}
Requirement Description &
\begin{minipage}[]{13cm}
The API Aspect service software shall be upgradable in place with
minimal user downtime.
\end{minipage}
\\ \cdashline{1-2}
Requirement Priority &  \\ \cdashline{1-2}
Upper Level Requirement &
\begin{tabular}{cl}
\end{tabular}
\\ \hline
\end{longtable}
}


\subsubsection{Test Cases Summary}
\begin{longtable}{p{3cm}p{2.5cm}p{2.5cm}p{3cm}p{4cm}}
\toprule
\href{https://jira.lsstcorp.org/secure/Tests.jspa\#/testCase/LVV-T834}{LVV-T834} & \multicolumn{4}{p{12cm}}{ Verify API upgradeability } \\ \hline
\textbf{Owner} & \textbf{Status} & \textbf{Version} & \textbf{Critical Event} & \textbf{Verification Type} \\ \hline
Jeffrey Carlin & Draft & 1 & false & Inspection \\ \hline
\end{longtable}
{\scriptsize
\textbf{Objective:}\\
Verify that the API Aspect service software are upgradable in place with
minimal user downtime.
}
  
 \newpage 
\subsection{[LVV-18222] DMS-REQ-0384-V-01: Export MOCs As FITS\_1 }\label{lvv-18222}

\begin{longtable}{cccc}
\hline
\textbf{Jira Link} & \textbf{Assignee} & \textbf{Status} & \textbf{Test Cases}\\ \hline
\href{https://jira.lsstcorp.org/browse/LVV-18222}{LVV-18222} &
Leanne Guy & Not Covered &
\begin{tabular}{c}
LVV-T1524 \\
\end{tabular}
\\
\hline
\end{longtable}

\textbf{Verification Element Description:} \\
Undefined

{\footnotesize
\begin{longtable}{p{2.5cm}p{13.5cm}}
\hline
\multicolumn{2}{c}{\textbf{Requirement Details}}\\ \hline
Requirement ID & DMS-REQ-0384 \\ \cdashline{1-2}
Requirement Description &
\begin{minipage}[]{13cm}
\textbf{Specification:} The Data Management system shall provide a means
for exporting the LSST-generated MOCs in the FITS serialization form
defined in the IVOA MOC Recommendation.
\end{minipage}
\\ \cdashline{1-2}
Requirement Discussion &
\begin{minipage}[]{13cm}
\textbf{Discussion:} The external endpoint for this should be designed
to be conformant with relevant community practice and any IVOA standards
that may emerge in this area.
\end{minipage}
\\ \cdashline{1-2}
Requirement Priority & 1b \\ \cdashline{1-2}
Upper Level Requirement &
\begin{tabular}{cl}
OSS-REQ-0391 & Data Product Conventions \\
\end{tabular}
\\ \hline
\end{longtable}
}


\subsubsection{Test Cases Summary}
\begin{longtable}{p{3cm}p{2.5cm}p{2.5cm}p{3cm}p{4cm}}
\toprule
\href{https://jira.lsstcorp.org/secure/Tests.jspa\#/testCase/LVV-T1524}{LVV-T1524} & \multicolumn{4}{p{12cm}}{ Verify Implementation of Exporting MOCs as FITS } \\ \hline
\textbf{Owner} & \textbf{Status} & \textbf{Version} & \textbf{Critical Event} & \textbf{Verification Type} \\ \hline
Jeffrey Carlin & Draft & 1 & false & Demonstration \\ \hline
\end{longtable}
{\scriptsize
\textbf{Objective:}\\
Verify that the Data Management system provides a means for exporting
the LSST-generated MOCs in the FITS serialization form defined in the
IVOA MOC Recommendation.
}
  
 \newpage 
\subsection{[LVV-18223] DMS-REQ-0381-V-01: HiPS Linkage to Coadds\_1 }\label{lvv-18223}

\begin{longtable}{cccc}
\hline
\textbf{Jira Link} & \textbf{Assignee} & \textbf{Status} & \textbf{Test Cases}\\ \hline
\href{https://jira.lsstcorp.org/browse/LVV-18223}{LVV-18223} &
Leanne Guy & Not Covered &
\begin{tabular}{c}
LVV-T1525 \\
\end{tabular}
\\
\hline
\end{longtable}

\textbf{Verification Element Description:} \\
Undefined

{\footnotesize
\begin{longtable}{p{2.5cm}p{13.5cm}}
\hline
\multicolumn{2}{c}{\textbf{Requirement Details}}\\ \hline
Requirement ID & DMS-REQ-0381 \\ \cdashline{1-2}
Requirement Description &
\begin{minipage}[]{13cm}
\textbf{Specification:} The HiPS maps produced by the Data Management
system shall provide for straightforward linkage from the HiPS data to
the underlying LSST coadded images. ~This SHOULD be implemented using a
mechanism supported by both the LSST Science Platform and by community
tools.
\end{minipage}
\\ \cdashline{1-2}
Requirement Discussion &
\begin{minipage}[]{13cm}
\textbf{Discussion:} It is intended that this be done using the ``HiPS
Progenitor'' mechanism introduced at the May 2018 IVOA meeting, or an
evolution of it that emerges from the IVOA standardization process.
\end{minipage}
\\ \cdashline{1-2}
Requirement Priority & 2 \\ \cdashline{1-2}
Upper Level Requirement &
\begin{tabular}{cl}
OSS-REQ-0122 & Provenance \\
OSS-REQ-0061 & Data Visualization \\
\end{tabular}
\\ \hline
\end{longtable}
}


\subsubsection{Test Cases Summary}
\begin{longtable}{p{3cm}p{2.5cm}p{2.5cm}p{3cm}p{4cm}}
\toprule
\href{https://jira.lsstcorp.org/secure/Tests.jspa\#/testCase/LVV-T1525}{LVV-T1525} & \multicolumn{4}{p{12cm}}{ Verify Implementation of Linkage Between HiPS Maps and Coadded Images } \\ \hline
\textbf{Owner} & \textbf{Status} & \textbf{Version} & \textbf{Critical Event} & \textbf{Verification Type} \\ \hline
Jeffrey Carlin & Draft & 1 & false & Demonstration \\ \hline
\end{longtable}
{\scriptsize
\textbf{Objective:}\\
Verify that the HiPS maps produced by the Data Management system provide
for straightforward linkage from the HiPS data to the underlying LSST
coadded images, and that this has been implemented using a mechanism
supported by both the LSST Science Platform and by community tools.
}
  
 \newpage 
\subsection{[LVV-18224] DMS-REQ-0380-V-01: HiPS Service\_1 }\label{lvv-18224}

\begin{longtable}{cccc}
\hline
\textbf{Jira Link} & \textbf{Assignee} & \textbf{Status} & \textbf{Test Cases}\\ \hline
\href{https://jira.lsstcorp.org/browse/LVV-18224}{LVV-18224} &
Leanne Guy & Not Covered &
\begin{tabular}{c}
LVV-T1526 \\
\end{tabular}
\\
\hline
\end{longtable}

\textbf{Verification Element Description:} \\
Undefined

{\footnotesize
\begin{longtable}{p{2.5cm}p{13.5cm}}
\hline
\multicolumn{2}{c}{\textbf{Requirement Details}}\\ \hline
Requirement ID & DMS-REQ-0380 \\ \cdashline{1-2}
Requirement Description &
\begin{minipage}[]{13cm}
\textbf{Specification:} The Data Management system shall include a
secure and authenticated Internet endpoint for an IVOA-compliant HiPS
service. ~This service shall be advertised via Registry as well as in
the HiPS community mechanism operated by CDS, or whatever equivalent
mechanism may exist in the LSST operations era.
\end{minipage}
\\ \cdashline{1-2}
Requirement Discussion &
\begin{minipage}[]{13cm}
\textbf{Discussion:} The DM HiPS service will be available only to data
rights holders. ~LSST EPO will also operate a world-public HiPS service,
but with its spatial resolution limited to approximately one arcsecond.
\end{minipage}
\\ \cdashline{1-2}
Requirement Priority & 1b \\ \cdashline{1-2}
Upper Level Requirement &
\begin{tabular}{cl}
OSS-REQ-0176 & Data Access \\
\end{tabular}
\\ \hline
\end{longtable}
}


\subsubsection{Test Cases Summary}
\begin{longtable}{p{3cm}p{2.5cm}p{2.5cm}p{3cm}p{4cm}}
\toprule
\href{https://jira.lsstcorp.org/secure/Tests.jspa\#/testCase/LVV-T1526}{LVV-T1526} & \multicolumn{4}{p{12cm}}{ Verify Availability of Secure and Authenticated HiPS Service } \\ \hline
\textbf{Owner} & \textbf{Status} & \textbf{Version} & \textbf{Critical Event} & \textbf{Verification Type} \\ \hline
Jeffrey Carlin & Draft & 1 & false & Demonstration \\ \hline
\end{longtable}
{\scriptsize
\textbf{Objective:}\\
Verify that the Data Management system includes a secure and
authenticated Internet endpoint for an IVOA-compliant HiPS service.
~Confirm that this service is advertised via Registry as well as in the
HiPS community mechanism operated by CDS, or whatever equivalent
mechanism may exist in the LSST operations era.
}
  
 \newpage 
\subsection{[LVV-18226] DMS-REQ-0385-V-01: MOC Visualization\_1 }\label{lvv-18226}

\begin{longtable}{cccc}
\hline
\textbf{Jira Link} & \textbf{Assignee} & \textbf{Status} & \textbf{Test Cases}\\ \hline
\href{https://jira.lsstcorp.org/browse/LVV-18226}{LVV-18226} &
Leanne Guy & Not Covered &
\begin{tabular}{c}
LVV-T1528 \\
\end{tabular}
\\
\hline
\end{longtable}

\textbf{Verification Element Description:} \\
Undefined

{\footnotesize
\begin{longtable}{p{2.5cm}p{13.5cm}}
\hline
\multicolumn{2}{c}{\textbf{Requirement Details}}\\ \hline
Requirement ID & DMS-REQ-0385 \\ \cdashline{1-2}
Requirement Description &
\begin{minipage}[]{13cm}
\textbf{Specification:} The LSST Science Platform shall support the
visualization of the LSST-generated MOCs as well as other MOCs which
satisfy the IVOA MOC Recommendation.
\end{minipage}
\\ \cdashline{1-2}
Requirement Discussion &
\begin{minipage}[]{13cm}
\textbf{Discussion:} We are considering the provision of services which
allow computations based on MOCs, e.g., the use of a MOC from another
mission or survey to define a query on the LSST data, but this is not
ready for codification at this time.
\end{minipage}
\\ \cdashline{1-2}
Requirement Priority & 1b \\ \cdashline{1-2}
Upper Level Requirement &
\begin{tabular}{cl}
OSS-REQ-0033 & Survey Planning and Performance Monitoring \\
OSS-REQ-0061 & Data Visualization \\
\end{tabular}
\\ \hline
\end{longtable}
}


\subsubsection{Test Cases Summary}
\begin{longtable}{p{3cm}p{2.5cm}p{2.5cm}p{3cm}p{4cm}}
\toprule
\href{https://jira.lsstcorp.org/secure/Tests.jspa\#/testCase/LVV-T1528}{LVV-T1528} & \multicolumn{4}{p{12cm}}{ Verify Visualization of MOCs via Science Platform } \\ \hline
\textbf{Owner} & \textbf{Status} & \textbf{Version} & \textbf{Critical Event} & \textbf{Verification Type} \\ \hline
Jeffrey Carlin & Draft & 1 & false & Demonstration \\ \hline
\end{longtable}
{\scriptsize
\textbf{Objective:}\\
Verify that the LSST Science Platform supports the visualization of the
LSST-generated MOCs as well as other MOCs which satisfy the IVOA MOC
Recommendation.
}
  
 \newpage 
\subsection{[LVV-18230] DMS-REQ-0386-V-01: Archive Processing Provenance\_1 }\label{lvv-18230}

\begin{longtable}{cccc}
\hline
\textbf{Jira Link} & \textbf{Assignee} & \textbf{Status} & \textbf{Test Cases}\\ \hline
\href{https://jira.lsstcorp.org/browse/LVV-18230}{LVV-18230} &
Leanne Guy & Not Covered &
\begin{tabular}{c}
LVV-T1560 \\
\end{tabular}
\\
\hline
\end{longtable}

\textbf{Verification Element Description:} \\
Undefined

{\footnotesize
\begin{longtable}{p{2.5cm}p{13.5cm}}
\hline
\multicolumn{2}{c}{\textbf{Requirement Details}}\\ \hline
Requirement ID & DMS-REQ-0386 \\ \cdashline{1-2}
Requirement Description &
\begin{minipage}[]{13cm}
\textbf{Specification:} The Data Management System shall archive all
processing provenance associated with archived data products, including
relevant data from other subsystems.
\end{minipage}
\\ \cdashline{1-2}
Requirement Priority & 1b \\ \cdashline{1-2}
Upper Level Requirement &
\begin{tabular}{cl}
OSS-REQ-0172 & Provenance Archiving \\
\end{tabular}
\\ \hline
\end{longtable}
}


\subsubsection{Test Cases Summary}
\begin{longtable}{p{3cm}p{2.5cm}p{2.5cm}p{3cm}p{4cm}}
\toprule
\href{https://jira.lsstcorp.org/secure/Tests.jspa\#/testCase/LVV-T1560}{LVV-T1560} & \multicolumn{4}{p{12cm}}{ Verify archiving of processing provenance } \\ \hline
\textbf{Owner} & \textbf{Status} & \textbf{Version} & \textbf{Critical Event} & \textbf{Verification Type} \\ \hline
Jeffrey Carlin & Draft & 1 & false & Inspection \\ \hline
\end{longtable}
{\scriptsize
\textbf{Objective:}\\
Verify that provenance information related to data processing, including
relevant data from other subsystems, has been archived.
}
  
 \newpage 
\subsection{[LVV-18231] DMS-REQ-0387-V-01: Serve Archived Provenance\_1 }\label{lvv-18231}

\begin{longtable}{cccc}
\hline
\textbf{Jira Link} & \textbf{Assignee} & \textbf{Status} & \textbf{Test Cases}\\ \hline
\href{https://jira.lsstcorp.org/browse/LVV-18231}{LVV-18231} &
Leanne Guy & Not Covered &
\begin{tabular}{c}
LVV-T1561 \\
\end{tabular}
\\
\hline
\end{longtable}

\textbf{Verification Element Description:} \\
Undefined

{\footnotesize
\begin{longtable}{p{2.5cm}p{13.5cm}}
\hline
\multicolumn{2}{c}{\textbf{Requirement Details}}\\ \hline
Requirement ID & DMS-REQ-0387 \\ \cdashline{1-2}
Requirement Description &
\begin{minipage}[]{13cm}
\textbf{Specification:} The Data Management System shall make the
archived provenance data available to science users together with the
associated science data products.
\end{minipage}
\\ \cdashline{1-2}
Requirement Priority & 1b \\ \cdashline{1-2}
Upper Level Requirement &
\begin{tabular}{cl}
OSS-REQ-0172 & Provenance Archiving \\
\end{tabular}
\\ \hline
\end{longtable}
}


\subsubsection{Test Cases Summary}
\begin{longtable}{p{3cm}p{2.5cm}p{2.5cm}p{3cm}p{4cm}}
\toprule
\href{https://jira.lsstcorp.org/secure/Tests.jspa\#/testCase/LVV-T1561}{LVV-T1561} & \multicolumn{4}{p{12cm}}{ Verify provenance availability to science users } \\ \hline
\textbf{Owner} & \textbf{Status} & \textbf{Version} & \textbf{Critical Event} & \textbf{Verification Type} \\ \hline
Jeffrey Carlin & Draft & 1 & false & Inspection \\ \hline
\end{longtable}
{\scriptsize
\textbf{Objective:}\\
Verify that archived provenance data is available to science users
together with the associated science data products.
}
  
 \newpage 
\subsection{[LVV-18232] DMS-REQ-0388-V-01: Provide Re-Run Tools\_1 }\label{lvv-18232}

\begin{longtable}{cccc}
\hline
\textbf{Jira Link} & \textbf{Assignee} & \textbf{Status} & \textbf{Test Cases}\\ \hline
\href{https://jira.lsstcorp.org/browse/LVV-18232}{LVV-18232} &
Leanne Guy & Not Covered &
\begin{tabular}{c}
LVV-T1562 \\
\end{tabular}
\\
\hline
\end{longtable}

\textbf{Verification Element Description:} \\
Undefined

{\footnotesize
\begin{longtable}{p{2.5cm}p{13.5cm}}
\hline
\multicolumn{2}{c}{\textbf{Requirement Details}}\\ \hline
Requirement ID & DMS-REQ-0388 \\ \cdashline{1-2}
Requirement Description &
\begin{minipage}[]{13cm}
\textbf{Specification:} The Data Management System shall provide tools
to re-run a data processing operation under the same conditions as a
previous run of that operation, based on provenance data recorded by the
system.
\end{minipage}
\\ \cdashline{1-2}
Requirement Discussion &
\begin{minipage}[]{13cm}
\textbf{Discussion:} The ``conditions'' include the LSST software, its
configuration parameters, and support data such as calibration frames.
\end{minipage}
\\ \cdashline{1-2}
Requirement Priority & 1b \\ \cdashline{1-2}
Upper Level Requirement &
\begin{tabular}{cl}
OSS-REQ-0122 & Provenance \\
OSS-REQ-0123 & Reproducibility \\
OSS-REQ-0172 & Provenance Archiving \\
\end{tabular}
\\ \hline
\end{longtable}
}


\subsubsection{Test Cases Summary}
\begin{longtable}{p{3cm}p{2.5cm}p{2.5cm}p{3cm}p{4cm}}
\toprule
\href{https://jira.lsstcorp.org/secure/Tests.jspa\#/testCase/LVV-T1562}{LVV-T1562} & \multicolumn{4}{p{12cm}}{ Verify availability of re-run tools } \\ \hline
\textbf{Owner} & \textbf{Status} & \textbf{Version} & \textbf{Critical Event} & \textbf{Verification Type} \\ \hline
Jeffrey Carlin & Draft & 1 & false & Demonstration \\ \hline
\end{longtable}
{\scriptsize
\textbf{Objective:}\\
Verify that tools are provided to use the archived provenance data to
re-run a data processing operation under the same conditions (including
LSST software version, its configuration parameters, and supporting data
such as calibration frames) as a previous run of that operation.
}
  
 \newpage 
\subsection{[LVV-18271] OCS-EFD-HS-0001-V-01: Fulfill requirements of a Commandable SAL
Component (CSC)\_1 }\label{lvv-18271}

\begin{longtable}{cccc}
\hline
\textbf{Jira Link} & \textbf{Assignee} & \textbf{Status} & \textbf{Test Cases}\\ \hline
\href{https://jira.lsstcorp.org/browse/LVV-18271}{LVV-18271} &
Leanne Guy & Not Covered &
\begin{tabular}{c}
\end{tabular}
\\
\hline
\end{longtable}

\textbf{Verification Element Description:} \\
Undefined

{\footnotesize
\begin{longtable}{p{2.5cm}p{13.5cm}}
\hline
\multicolumn{2}{c}{\textbf{Requirement Details}}\\ \hline
Requirement ID & OCS-EFD-HS-0001 \\ \cdashline{1-2}
Requirement Description &
\begin{minipage}[]{13cm}
\textbf{Specification:} The Header Service shall behave as a Commandable
SAL Component (CSC) following the command patterns described in \citeds{LSE-70}
and \citeds{LSE-209}.
\end{minipage}
\\ \cdashline{1-2}
Requirement Discussion &
\begin{minipage}[]{13cm}
\textbf{Discussion:} The Header Service is not expected to have any
CSC-specific commands, only common cross-subsystem commands such as
``start'' and ``enable''.
\end{minipage}
\\ \cdashline{1-2}
Requirement Priority &  \\ \cdashline{1-2}
Upper Level Requirement &
\begin{tabular}{cl}
\end{tabular}
\\ \hline
\end{longtable}
}


  
 \newpage 
\subsection{[LVV-18272] OCS-EFD-HS-0002-V-01: Critical System\_1 }\label{lvv-18272}

\begin{longtable}{cccc}
\hline
\textbf{Jira Link} & \textbf{Assignee} & \textbf{Status} & \textbf{Test Cases}\\ \hline
\href{https://jira.lsstcorp.org/browse/LVV-18272}{LVV-18272} &
Leanne Guy & Not Covered &
\begin{tabular}{c}
\end{tabular}
\\
\hline
\end{longtable}

\textbf{Verification Element Description:} \\
Undefined

{\footnotesize
\begin{longtable}{p{2.5cm}p{13.5cm}}
\hline
\multicolumn{2}{c}{\textbf{Requirement Details}}\\ \hline
Requirement ID & OCS-EFD-HS-0002 \\ \cdashline{1-2}
Requirement Description &
\begin{minipage}[]{13cm}
\textbf{Specification:} The Header Service instances shall be considered
a critical system for observatory operations and shall reside within the
EFD computer cluster.
\end{minipage}
\\ \cdashline{1-2}
Requirement Priority &  \\ \cdashline{1-2}
Upper Level Requirement &
\begin{tabular}{cl}
\end{tabular}
\\ \hline
\end{longtable}
}


  
 \newpage 
\subsection{[LVV-18273] OCS-EFD-HS-0003-V-01: Write Headers for all images taken by all Cameras
supported by LSST\_1 }\label{lvv-18273}

\begin{longtable}{cccc}
\hline
\textbf{Jira Link} & \textbf{Assignee} & \textbf{Status} & \textbf{Test Cases}\\ \hline
\href{https://jira.lsstcorp.org/browse/LVV-18273}{LVV-18273} &
Leanne Guy & Not Covered &
\begin{tabular}{c}
\end{tabular}
\\
\hline
\end{longtable}

\textbf{Verification Element Description:} \\
Undefined

{\footnotesize
\begin{longtable}{p{2.5cm}p{13.5cm}}
\hline
\multicolumn{2}{c}{\textbf{Requirement Details}}\\ \hline
Requirement ID & OCS-EFD-HS-0003 \\ \cdashline{1-2}
Requirement Description &
\begin{minipage}[]{13cm}
\textbf{Specification:} The Header Service instances shall write header
files for all (100\%) of the images taken and announced by the Camera
Control System for the camera for which each instance is configured
(LSSTCam, ComCam, AuxTel or Test Stand) while the instance is enabled,
including information for every CCD configured for that camera in its
observing mode such as science and wavefront CCDs for LSSTCam.
\end{minipage}
\\ \cdashline{1-2}
Requirement Discussion &
\begin{minipage}[]{13cm}
\textbf{Discussion:} The header files must be generated and written at
the cadence required for different observing mode (bias, flats,
science).
\end{minipage}
\\ \cdashline{1-2}
Requirement Priority &  \\ \cdashline{1-2}
Upper Level Requirement &
\begin{tabular}{cl}
\end{tabular}
\\ \hline
\end{longtable}
}


  
 \newpage 
\subsection{[LVV-18274] OCS-EFD-HS-0004-V-01: Ability to capture metadata at the beginning of
exposure\_1 }\label{lvv-18274}

\begin{longtable}{cccc}
\hline
\textbf{Jira Link} & \textbf{Assignee} & \textbf{Status} & \textbf{Test Cases}\\ \hline
\href{https://jira.lsstcorp.org/browse/LVV-18274}{LVV-18274} &
Leanne Guy & Not Covered &
\begin{tabular}{c}
\end{tabular}
\\
\hline
\end{longtable}

\textbf{Verification Element Description:} \\
Undefined

{\footnotesize
\begin{longtable}{p{2.5cm}p{13.5cm}}
\hline
\multicolumn{2}{c}{\textbf{Requirement Details}}\\ \hline
Requirement ID & OCS-EFD-HS-0004 \\ \cdashline{1-2}
Requirement Description &
\begin{minipage}[]{13cm}
\textbf{Specification:} The Header Service shall be able to capture and
store Events or Telemetry before the start of an integration.
\end{minipage}
\\ \cdashline{1-2}
Requirement Discussion &
\begin{minipage}[]{13cm}
\textbf{Discussion:} An example is configuration telemetry from other
CSCs.
\end{minipage}
\\ \cdashline{1-2}
Requirement Priority &  \\ \cdashline{1-2}
Upper Level Requirement &
\begin{tabular}{cl}
\end{tabular}
\\ \hline
\end{longtable}
}


  
 \newpage 
\subsection{[LVV-18275] OCS-EFD-HS-0005-V-01: Ability to capture metadata during of exposure
integration\_1 }\label{lvv-18275}

\begin{longtable}{cccc}
\hline
\textbf{Jira Link} & \textbf{Assignee} & \textbf{Status} & \textbf{Test Cases}\\ \hline
\href{https://jira.lsstcorp.org/browse/LVV-18275}{LVV-18275} &
Leanne Guy & Not Covered &
\begin{tabular}{c}
\end{tabular}
\\
\hline
\end{longtable}

\textbf{Verification Element Description:} \\
Undefined

{\footnotesize
\begin{longtable}{p{2.5cm}p{13.5cm}}
\hline
\multicolumn{2}{c}{\textbf{Requirement Details}}\\ \hline
Requirement ID & OCS-EFD-HS-0005 \\ \cdashline{1-2}
Requirement Description &
\begin{minipage}[]{13cm}
\textbf{Specification:} The Header Service shall be able to capture and
store Events or Telemetry that happen during the image integration time.
\end{minipage}
\\ \cdashline{1-2}
Requirement Discussion &
\begin{minipage}[]{13cm}
\textbf{Discussion:} An example is shutter motion events.
\end{minipage}
\\ \cdashline{1-2}
Requirement Priority &  \\ \cdashline{1-2}
Upper Level Requirement &
\begin{tabular}{cl}
\end{tabular}
\\ \hline
\end{longtable}
}


  
 \newpage 
\subsection{[LVV-18276] OCS-EFD-HS-0006-V-01: Ability to capture metadata at end of readout\_1 }\label{lvv-18276}

\begin{longtable}{cccc}
\hline
\textbf{Jira Link} & \textbf{Assignee} & \textbf{Status} & \textbf{Test Cases}\\ \hline
\href{https://jira.lsstcorp.org/browse/LVV-18276}{LVV-18276} &
Leanne Guy & Not Covered &
\begin{tabular}{c}
\end{tabular}
\\
\hline
\end{longtable}

\textbf{Verification Element Description:} \\
Undefined

{\footnotesize
\begin{longtable}{p{2.5cm}p{13.5cm}}
\hline
\multicolumn{2}{c}{\textbf{Requirement Details}}\\ \hline
Requirement ID & OCS-EFD-HS-0006 \\ \cdashline{1-2}
Requirement Description &
\begin{minipage}[]{13cm}
\textbf{Specification:} The Header Service shall be able to capture and
store Events or Telemetry that happen at or slightly after the end of
readout, up to the receipt of the end-of-telemetry event.
\end{minipage}
\\ \cdashline{1-2}
Requirement Discussion &
\begin{minipage}[]{13cm}
\textbf{Discussion:} An example is the end-of-readout time.
\end{minipage}
\\ \cdashline{1-2}
Requirement Priority &  \\ \cdashline{1-2}
Upper Level Requirement &
\begin{tabular}{cl}
\end{tabular}
\\ \hline
\end{longtable}
}


  
 \newpage 
\subsection{[LVV-18277] OCS-EFD-HS-0007-V-01: Write header and Publish Event after end of
telemetry event\_1 }\label{lvv-18277}

\begin{longtable}{cccc}
\hline
\textbf{Jira Link} & \textbf{Assignee} & \textbf{Status} & \textbf{Test Cases}\\ \hline
\href{https://jira.lsstcorp.org/browse/LVV-18277}{LVV-18277} &
Leanne Guy & Not Covered &
\begin{tabular}{c}
\end{tabular}
\\
\hline
\end{longtable}

\textbf{Verification Element Description:} \\
Undefined

{\footnotesize
\begin{longtable}{p{2.5cm}p{13.5cm}}
\hline
\multicolumn{2}{c}{\textbf{Requirement Details}}\\ \hline
Requirement ID & OCS-EFD-HS-0007 \\ \cdashline{1-2}
Requirement Description &
\begin{minipage}[]{13cm}
\textbf{Specification:} The Header Service shall begin to write the
header file(s) immediately after receiving the end-of-telemetry Event
from the Camera Control System and, when complete, emit one or more
LargeFileObjectAvailable Events that will notify the EFD of the
existence of the new header file(s).
\end{minipage}
\\ \cdashline{1-2}
Requirement Discussion &
\begin{minipage}[]{13cm}
\textbf{Discussion:} The Event anouncing that a Large File Object (LFO)
is available will contain the image name, a unique id that will be used
to match images (pixels) from the DAQ with the header meta-data.
\end{minipage}
\\ \cdashline{1-2}
Requirement Priority &  \\ \cdashline{1-2}
Upper Level Requirement &
\begin{tabular}{cl}
\end{tabular}
\\ \hline
\end{longtable}
}


  
 \newpage 
\subsection{[LVV-18278] OCS-EFD-HS-0008-V-01: Write header and Publish Event within specified
time of the end-of-telemetry Event\_1 }\label{lvv-18278}

\begin{longtable}{cccc}
\hline
\textbf{Jira Link} & \textbf{Assignee} & \textbf{Status} & \textbf{Test Cases}\\ \hline
\href{https://jira.lsstcorp.org/browse/LVV-18278}{LVV-18278} &
Leanne Guy & Not Covered &
\begin{tabular}{c}
\end{tabular}
\\
\hline
\end{longtable}

\textbf{Verification Element Description:} \\
Undefined

{\footnotesize
\begin{longtable}{p{2.5cm}p{13.5cm}}
\hline
\multicolumn{2}{c}{\textbf{Requirement Details}}\\ \hline
Requirement ID & OCS-EFD-HS-0008 \\ \cdashline{1-2}
Requirement Description &
\begin{minipage}[]{13cm}
\textbf{Specification:} The Header Service shall complete the writing of
the header file and emit the LFO Event within 200 milliseconds.
\end{minipage}
\\ \cdashline{1-2}
Requirement Discussion &
\begin{minipage}[]{13cm}
\textbf{Discussion:} This constraint is required because the Header
Service is in the critical timing path for visualization of images,
quality control processes, and the Alert Production. ~The Header Service
must keep up with the image cadence even for bias frames.
\end{minipage}
\\ \cdashline{1-2}
Requirement Priority &  \\ \cdashline{1-2}
Upper Level Requirement &
\begin{tabular}{cl}
\end{tabular}
\\ \hline
\end{longtable}
}


  
 \newpage 
\subsection{[LVV-18279] OCS-EFD-HS-0009-V-01: Adherence to the FITS Standard\_1 }\label{lvv-18279}

\begin{longtable}{cccc}
\hline
\textbf{Jira Link} & \textbf{Assignee} & \textbf{Status} & \textbf{Test Cases}\\ \hline
\href{https://jira.lsstcorp.org/browse/LVV-18279}{LVV-18279} &
Leanne Guy & Not Covered &
\begin{tabular}{c}
\end{tabular}
\\
\hline
\end{longtable}

\textbf{Verification Element Description:} \\
Undefined

{\footnotesize
\begin{longtable}{p{2.5cm}p{13.5cm}}
\hline
\multicolumn{2}{c}{\textbf{Requirement Details}}\\ \hline
Requirement ID & OCS-EFD-HS-0009 \\ \cdashline{1-2}
Requirement Description &
\begin{minipage}[]{13cm}
\textbf{Specification:} The contents of the file(s) written by the
Header Service will be consistent with the data needed to generate
compliant FITS headers.
\end{minipage}
\\ \cdashline{1-2}
Requirement Discussion &
\begin{minipage}[]{13cm}
\textbf{Discussion:} Details of the file format and number of files are
given in Appendix B.
\end{minipage}
\\ \cdashline{1-2}
Requirement Priority &  \\ \cdashline{1-2}
Upper Level Requirement &
\begin{tabular}{cl}
\end{tabular}
\\ \hline
\end{longtable}
}


  
 \newpage 
\subsection{[LVV-18280] OCS-EFD-HS-0010-V-01: Configuration of Header Keywords and source\_1 }\label{lvv-18280}

\begin{longtable}{cccc}
\hline
\textbf{Jira Link} & \textbf{Assignee} & \textbf{Status} & \textbf{Test Cases}\\ \hline
\href{https://jira.lsstcorp.org/browse/LVV-18280}{LVV-18280} &
Leanne Guy & Not Covered &
\begin{tabular}{c}
\end{tabular}
\\
\hline
\end{longtable}

\textbf{Verification Element Description:} \\
Undefined

{\footnotesize
\begin{longtable}{p{2.5cm}p{13.5cm}}
\hline
\multicolumn{2}{c}{\textbf{Requirement Details}}\\ \hline
Requirement ID & OCS-EFD-HS-0010 \\ \cdashline{1-2}
Requirement Description &
\begin{minipage}[]{13cm}
\textbf{Specification:} The Header Service shall be configurable as to
the keywords used to identify metadata that goes into the header as well
as configurable as to the source of that metadata. The sources may be
Events or Telemetry to which the Header Service will subscribe or
elements of the Header Service's own configuration.
\end{minipage}
\\ \cdashline{1-2}
Requirement Discussion &
\begin{minipage}[]{13cm}
\textbf{Discussion:} This configuration is expected to be performed via
YAML files that are easy to read and maintained under version control.
Slowly-changing information can be stored statically in these files.
Configurations will also vary by the associated camera and the observing
mode (science versus calibration).
\end{minipage}
\\ \cdashline{1-2}
Requirement Priority &  \\ \cdashline{1-2}
Upper Level Requirement &
\begin{tabular}{cl}
\end{tabular}
\\ \hline
\end{longtable}
}


  
 \newpage 
\subsection{[LVV-18281] OCS-EFD-HS-0011-V-01: Produce header even if some meta-data not
avaiable\_1 }\label{lvv-18281}

\begin{longtable}{cccc}
\hline
\textbf{Jira Link} & \textbf{Assignee} & \textbf{Status} & \textbf{Test Cases}\\ \hline
\href{https://jira.lsstcorp.org/browse/LVV-18281}{LVV-18281} &
Leanne Guy & Not Covered &
\begin{tabular}{c}
\end{tabular}
\\
\hline
\end{longtable}

\textbf{Verification Element Description:} \\
Undefined

{\footnotesize
\begin{longtable}{p{2.5cm}p{13.5cm}}
\hline
\multicolumn{2}{c}{\textbf{Requirement Details}}\\ \hline
Requirement ID & OCS-EFD-HS-0011 \\ \cdashline{1-2}
Requirement Description &
\begin{minipage}[]{13cm}
\textbf{Specification:} The Header Service shall write headers even with
faulty or missing Telemetry.
\end{minipage}
\\ \cdashline{1-2}
Requirement Discussion &
\begin{minipage}[]{13cm}
\textbf{Discussion:} In the case that some telemetry is missing or not
broadcast the Header Service will still write files and use
FITS-compliant mechanisms for specifying undefined values for the
missing metadata.
\end{minipage}
\\ \cdashline{1-2}
Requirement Priority &  \\ \cdashline{1-2}
Upper Level Requirement &
\begin{tabular}{cl}
\end{tabular}
\\ \hline
\end{longtable}
}


  
 \newpage 
\subsection{[LVV-18282] OCS-EFD-HS-0012-V-01: Publish an Event if monitoring detects any failure
of the service.\_1 }\label{lvv-18282}

\begin{longtable}{cccc}
\hline
\textbf{Jira Link} & \textbf{Assignee} & \textbf{Status} & \textbf{Test Cases}\\ \hline
\href{https://jira.lsstcorp.org/browse/LVV-18282}{LVV-18282} &
Leanne Guy & Not Covered &
\begin{tabular}{c}
\end{tabular}
\\
\hline
\end{longtable}

\textbf{Verification Element Description:} \\
Undefined

{\footnotesize
\begin{longtable}{p{2.5cm}p{13.5cm}}
\hline
\multicolumn{2}{c}{\textbf{Requirement Details}}\\ \hline
Requirement ID & OCS-EFD-HS-0012 \\ \cdashline{1-2}
Requirement Description &
\begin{minipage}[]{13cm}
\textbf{Specification:} The Header Service shall publish an Event
message describing the type of problem if it detects that its service is
degraded in some way.
\end{minipage}
\\ \cdashline{1-2}
Requirement Priority &  \\ \cdashline{1-2}
Upper Level Requirement &
\begin{tabular}{cl}
\end{tabular}
\\ \hline
\end{longtable}
}


  
 \newpage 
\subsection{[LVV-18283] OCS-EFD-HS-0013-V-01: Extract metadata from published configuration\_1 }\label{lvv-18283}

\begin{longtable}{cccc}
\hline
\textbf{Jira Link} & \textbf{Assignee} & \textbf{Status} & \textbf{Test Cases}\\ \hline
\href{https://jira.lsstcorp.org/browse/LVV-18283}{LVV-18283} &
Leanne Guy & Not Covered &
\begin{tabular}{c}
\end{tabular}
\\
\hline
\end{longtable}

\textbf{Verification Element Description:} \\
Undefined

{\footnotesize
\begin{longtable}{p{2.5cm}p{13.5cm}}
\hline
\multicolumn{2}{c}{\textbf{Requirement Details}}\\ \hline
Requirement ID & OCS-EFD-HS-0013 \\ \cdashline{1-2}
Requirement Description &
\begin{minipage}[]{13cm}
\textbf{Specification:} The Header Service shall be able to extract
metadata from the configuration information published by other CSCs such
as the Camera Control System and the Telescope Control System.
\end{minipage}
\\ \cdashline{1-2}
Requirement Discussion &
\begin{minipage}[]{13cm}
\textbf{Discussion:} Some metadata that changes at nightly rate might be
easier to acquire via configuration information published by individual
CSCs.
\end{minipage}
\\ \cdashline{1-2}
Requirement Priority &  \\ \cdashline{1-2}
Upper Level Requirement &
\begin{tabular}{cl}
\end{tabular}
\\ \hline
\end{longtable}
}


  
 \newpage 
\subsection{[LVV-18284] OCS-EFD-HS-0014-V-01: Metadata Capture\_1 }\label{lvv-18284}

\begin{longtable}{cccc}
\hline
\textbf{Jira Link} & \textbf{Assignee} & \textbf{Status} & \textbf{Test Cases}\\ \hline
\href{https://jira.lsstcorp.org/browse/LVV-18284}{LVV-18284} &
Leanne Guy & Not Covered &
\begin{tabular}{c}
\end{tabular}
\\
\hline
\end{longtable}

\textbf{Verification Element Description:} \\
Undefined

{\footnotesize
\begin{longtable}{p{2.5cm}p{13.5cm}}
\hline
\multicolumn{2}{c}{\textbf{Requirement Details}}\\ \hline
Requirement ID & OCS-EFD-HS-0014 \\ \cdashline{1-2}
Requirement Description &
\begin{minipage}[]{13cm}
\textbf{Specification:} The Header Service shall capture at a minimum
all metadata required by Prompt Processing, Archiving, and any relevant
Summit systems.
\end{minipage}
\\ \cdashline{1-2}
Requirement Discussion &
\begin{minipage}[]{13cm}
\textbf{Discussion:} Appendix A includes a list of items to be captured;
additional items may be added via normal change control processes.
\end{minipage}
\\ \cdashline{1-2}
Requirement Priority &  \\ \cdashline{1-2}
Upper Level Requirement &
\begin{tabular}{cl}
\end{tabular}
\\ \hline
\end{longtable}
}


  
 \newpage 
\subsection{[LVV-18285] OCS-EFD-HS-0015-V-01: Generate on-the-fly additional metadata requested
by the Project Science Team.\_1 }\label{lvv-18285}

\begin{longtable}{cccc}
\hline
\textbf{Jira Link} & \textbf{Assignee} & \textbf{Status} & \textbf{Test Cases}\\ \hline
\href{https://jira.lsstcorp.org/browse/LVV-18285}{LVV-18285} &
Leanne Guy & Not Covered &
\begin{tabular}{c}
\end{tabular}
\\
\hline
\end{longtable}

\textbf{Verification Element Description:} \\
Undefined

{\footnotesize
\begin{longtable}{p{2.5cm}p{13.5cm}}
\hline
\multicolumn{2}{c}{\textbf{Requirement Details}}\\ \hline
Requirement ID & OCS-EFD-HS-0015 \\ \cdashline{1-2}
Requirement Description &
\begin{minipage}[]{13cm}
\textbf{Specification:} The Header Service shall be able to do
light-weight computations to generate additional metadata as requested
by the project in case it is not directly provided by other CSCs.
\end{minipage}
\\ \cdashline{1-2}
Requirement Discussion &
\begin{minipage}[]{13cm}
\textbf{Discussion:} For example, calculating the exposure time or dark
time.
\end{minipage}
\\ \cdashline{1-2}
Requirement Priority &  \\ \cdashline{1-2}
Upper Level Requirement &
\begin{tabular}{cl}
\end{tabular}
\\ \hline
\end{longtable}
}


  
 \newpage 
\subsection{[LVV-18849] CA-DM-CON-ICD-0020-V-02: Archiving service availability\_DM\_2 }\label{lvv-18849}

\begin{longtable}{cccc}
\hline
\textbf{Jira Link} & \textbf{Assignee} & \textbf{Status} & \textbf{Test Cases}\\ \hline
\href{https://jira.lsstcorp.org/browse/LVV-18849}{LVV-18849} &
Leanne Guy & Not Covered &
\begin{tabular}{c}
\end{tabular}
\\
\hline
\end{longtable}

\textbf{Verification Element Description:} \\
Undefined

{\footnotesize
\begin{longtable}{p{2.5cm}p{13.5cm}}
\hline
\multicolumn{2}{c}{\textbf{Requirement Details}}\\ \hline
Requirement ID & CA-DM-CON-ICD-0020 \\ \cdashline{1-2}
Requirement Description &
\begin{minipage}[]{13cm}
\textbf{Specification:} The archiving service shall be available no
later than the start of Observatory commissioning, i.e., supporting the
Commissioning Camera.
\end{minipage}
\\ \cdashline{1-2}
Requirement Priority &  \\ \cdashline{1-2}
Upper Level Requirement &
\begin{tabular}{cl}
\end{tabular}
\\ \hline
\end{longtable}
}


  
 \newpage 
\subsection{[LVV-18852] CA-DM-CON-ICD-0022-V-02: Archiving service during maintenance\_DM\_2 }\label{lvv-18852}

\begin{longtable}{cccc}
\hline
\textbf{Jira Link} & \textbf{Assignee} & \textbf{Status} & \textbf{Test Cases}\\ \hline
\href{https://jira.lsstcorp.org/browse/LVV-18852}{LVV-18852} &
Leanne Guy & Not Covered &
\begin{tabular}{c}
\end{tabular}
\\
\hline
\end{longtable}

\textbf{Verification Element Description:} \\
Undefined

{\footnotesize
\begin{longtable}{p{2.5cm}p{13.5cm}}
\hline
\multicolumn{2}{c}{\textbf{Requirement Details}}\\ \hline
Requirement ID & CA-DM-CON-ICD-0022 \\ \cdashline{1-2}
Requirement Description &
\begin{minipage}[]{13cm}
\textbf{Specification:} Data Management shall archive data from the
Camera upon a request from the Camera, as long as Data Management is not
performing incompatible maintenance activities of its own.
\end{minipage}
\\ \cdashline{1-2}
Requirement Priority &  \\ \cdashline{1-2}
Upper Level Requirement &
\begin{tabular}{cl}
\end{tabular}
\\ \hline
\end{longtable}
}


  
 \newpage 
\subsection{[LVV-18855] CA-DM-CON-ICD-0023-V-02: Archiving service during outages\_DM\_2 }\label{lvv-18855}

\begin{longtable}{cccc}
\hline
\textbf{Jira Link} & \textbf{Assignee} & \textbf{Status} & \textbf{Test Cases}\\ \hline
\href{https://jira.lsstcorp.org/browse/LVV-18855}{LVV-18855} &
Leanne Guy & Not Covered &
\begin{tabular}{c}
\end{tabular}
\\
\hline
\end{longtable}

\textbf{Verification Element Description:} \\
Undefined

{\footnotesize
\begin{longtable}{p{2.5cm}p{13.5cm}}
\hline
\multicolumn{2}{c}{\textbf{Requirement Details}}\\ \hline
Requirement ID & CA-DM-CON-ICD-0023 \\ \cdashline{1-2}
Requirement Description &
\begin{minipage}[]{13cm}
\textbf{Specification:} Data Management shall normally provide access to
this archiving service for at least part of every day's maintenance
time, except during scheduled long outages.
\end{minipage}
\\ \cdashline{1-2}
Requirement Priority &  \\ \cdashline{1-2}
Upper Level Requirement &
\begin{tabular}{cl}
\end{tabular}
\\ \hline
\end{longtable}
}


  
 \newpage 
\subsection{[LVV-18858] CA-DM-CON-ICD-0021-V-02: Archiving service storage duration\_DM\_2 }\label{lvv-18858}

\begin{longtable}{cccc}
\hline
\textbf{Jira Link} & \textbf{Assignee} & \textbf{Status} & \textbf{Test Cases}\\ \hline
\href{https://jira.lsstcorp.org/browse/LVV-18858}{LVV-18858} &
Leanne Guy & Not Covered &
\begin{tabular}{c}
\end{tabular}
\\
\hline
\end{longtable}

\textbf{Verification Element Description:} \\
Undefined

{\footnotesize
\begin{longtable}{p{2.5cm}p{13.5cm}}
\hline
\multicolumn{2}{c}{\textbf{Requirement Details}}\\ \hline
Requirement ID & CA-DM-CON-ICD-0021 \\ \cdashline{1-2}
Requirement Description &
\begin{minipage}[]{13cm}
\textbf{Specification:} The archiving service shall permit the storage
of camera image data, covering the entire focal plane including the
corner rafts, taken during Camera-specific engineering activities for
the life of the survey.
\end{minipage}
\\ \cdashline{1-2}
Requirement Priority &  \\ \cdashline{1-2}
Upper Level Requirement &
\begin{tabular}{cl}
\end{tabular}
\\ \hline
\end{longtable}
}


  
 \newpage 
\subsection{[LVV-18911] DMS-REQ-0391-V-02: Alert Stream Distribution Latency }\label{lvv-18911}

\begin{longtable}{cccc}
\hline
\textbf{Jira Link} & \textbf{Assignee} & \textbf{Status} & \textbf{Test Cases}\\ \hline
\href{https://jira.lsstcorp.org/browse/LVV-18911}{LVV-18911} &
Leanne Guy & Not Covered &
\begin{tabular}{c}
LVV-T1868 \\
\end{tabular}
\\
\hline
\end{longtable}

\textbf{Verification Element Description:} \\
This VE satisfies the requirement on OTT1=1 minute. The related VE
LVV-81297Â~pertains to the number of streams.

{\footnotesize
\begin{longtable}{p{2.5cm}p{13.5cm}}
\hline
\multicolumn{2}{c}{\textbf{Requirement Details}}\\ \hline
Requirement ID & DMS-REQ-0391 \\ \cdashline{1-2}
Requirement Description &
\begin{minipage}[]{13cm}
\textbf{Specification:} LSST shall be capable of supporting the
transmission of at least \textbf{numStreams} full alert streams out of
the alert distribution system within \textbf{OTT1}.
\end{minipage}
\\ \cdashline{1-2}
Requirement Parameters & {[}\textbf{numStreams = 5{{[}integer{]}}} The minimum number of full
streams that can be transmitted out of the alert distribution system.,
\textbf{OTT1 = 1{{[}minute{]}}} The latency of reporting optical
transients following the completion of readout of the last image of a
visit{]} \\ \cdashline{1-2}
Requirement Priority &  \\ \cdashline{1-2}
Upper Level Requirement &
\begin{tabular}{cl}
OSS-REQ-0184 & Transient Alert Publication \\
OSS-REQ-0127 & Level 1 Data Product Availability \\
\end{tabular}
\\ \hline
\end{longtable}
}


\subsubsection{Test Cases Summary}
\begin{longtable}{p{3cm}p{2.5cm}p{2.5cm}p{3cm}p{4cm}}
\toprule
\href{https://jira.lsstcorp.org/secure/Tests.jspa\#/testCase/LVV-T1868}{LVV-T1868} & \multicolumn{4}{p{12cm}}{ Verify implementation of alert streams distributed within latency limit } \\ \hline
\textbf{Owner} & \textbf{Status} & \textbf{Version} & \textbf{Critical Event} & \textbf{Verification Type} \\ \hline
Jeffrey Carlin & Draft & 1 & false & Test \\ \hline
\end{longtable}
{\scriptsize
\textbf{Objective:}\\
Verify that the LSST system supports the transmission of full alert
streams out of the alert distribution system within \textbf{OTT1=1
minute}.
}
  

\newpage
\appendix
\section{Traceability}
\label{sec:trace}

\begin{longtable}{ccc}
\hline
\textbf{Requirements} & \textbf{Verification Elements} & \textbf{Test Cases} \\ \hline
 DMS-REQ-0074  &
 LVV-32  &
LVV-T20 \\
 &
 &
LVV-T37 \\
\hline
 DMS-REQ-0077  &
 LVV-34  &
LVV-T150 \\
\hline
 DMS-REQ-0078  &
 LVV-35  &
LVV-T151 \\
 &
 &
LVV-T1232 \\
\hline
 DMS-REQ-0094  &
 LVV-37  &
LVV-T152 \\
\hline
 DMS-REQ-0103  &
 LVV-45  &
LVV-T63 \\
\hline
 DMS-REQ-0119  &
 LVV-47  &
LVV-T117 \\
\hline
 DMS-REQ-0122  &
 LVV-50  &
LVV-T204 \\
\hline
 DMS-REQ-0123  &
 LVV-51  &
LVV-T205 \\
\hline
 DMS-REQ-0127  &
 LVV-55  &
LVV-T208 \\
\hline
 DMS-REQ-0131  &
 LVV-58  &
LVV-T106 \\
\hline
 DMS-REQ-0155  &
 LVV-60  &
Verified By LVV-129 \\
 &
 &
Verified By LVV-130 \\
 &
 &
Verified By LVV-131 \\
\hline
 DMS-REQ-0156  &
 LVV-61  &
Verified By LVV-133 \\
 &
 &
Verified By LVV-134 \\
 &
 &
Verified By LVV-135 \\
\hline
 DMS-REQ-0160  &
 LVV-63  &
LVV-T131 \\
 &
 &
LVV-T368 \\
 &
 &
LVV-T368 \\
\hline
 DMS-REQ-0298  &
 LVV-129  &
LVV-T136 \\
 &
 &
LVV-T368 \\
 &
 &
LVV-T374 \\
 &
 &
LVV-T368 \\
\hline
 DMS-REQ-0308  &
 LVV-139  &
LVV-T10 \\
 &
 &
LVV-T17 \\
 &
 &
LVV-T124 \\
 &
 &
LVV-T216 \\
 &
 &
LVV-T362 \\
 &
 &
LVV-T363 \\
\hline
 DMS-REQ-0312  &
 LVV-143  &
LVV-T157 \\
\hline
 DMS-REQ-0313  &
 LVV-144  &
LVV-T158 \\
\hline
 DMS-REQ-0320  &
 LVV-151  &
LVV-T92 \\
\hline
 DMS-REQ-0340  &
 LVV-171  &
LVV-T123 \\
\hline
 DMS-REQ-0341  &
 LVV-172  &
LVV-T160 \\
 \cdashline{2-3} DMS-REQ-0341 & LVV-9749 &  \\ \hline
\hline
 DMS-REQ-0342  &
 LVV-173  &
LVV-T112 \\
 &
 &
LVV-T218 \\
\hline
 DMS-REQ-0343  &
 LVV-174  &
LVV-T113 \\
 &
 &
LVV-T218 \\
 \cdashline{2-3}  &
 LVV-9748  &
LVV-T1252 \\
\hline
 DMS-REQ-0004  &
 LVV-175  &
LVV-T35 \\
 &
 &
LVV-T95 \\
 \cdashline{2-3}  &
 LVV-9740  &
LVV-T1276 \\
 \cdashline{2-3}  &
 LVV-9803  &
LVV-T102 \\
\hline
 DMS-REQ-0346  &
 LVV-177  &
LVV-T27 \\
 &
 &
LVV-T286 \\
\hline
 DMS-REQ-0353  &
 LVV-184  &
LVV-T60 \\
\hline
DMS-REQ-0355 & LVV-186 &  \\ \hline
 \cdashline{2-3} DMS-REQ-0355 & LVV-9784 &  \\ \hline
\hline
DMS-REQ-0356 & LVV-187 &  \\ \hline
 \cdashline{2-3} DMS-REQ-0356 & LVV-9785 &  \\ \hline
 \cdashline{2-3}  &
 LVV-9786  &
LVV-T1089 \\
 &
 &
LVV-T1090 \\
 \cdashline{2-3}  &
 LVV-9787  &
LVV-T1085 \\
 &
 &
LVV-T1089 \\
 &
 &
LVV-T1090 \\
\hline
 DMS-REQ-0364  &
 LVV-190  &
LVV-T163 \\
 \cdashline{2-3} DMS-REQ-0364 & LVV-9750 &  \\ \hline
\hline
 DMS-REQ-0365  &
 LVV-191  &
LVV-T164 \\
\hline
 DMS-REQ-0366  &
 LVV-192  &
LVV-T165 \\
\hline
 DMS-REQ-0368  &
 LVV-194  &
LVV-T167 \\
\hline
 DMS-REQ-0369  &
 LVV-195  &
LVV-T168 \\
\hline
 DMS-REQ-0370  &
 LVV-196  &
LVV-T169 \\
\hline
 DMS-REQ-0371  &
 LVV-197  &
LVV-T170 \\
\hline
 DMS-REQ-0377  &
 LVV-3394  &
LVV-T385 \\
 \cdashline{2-3}  &
 LVV-9797  &
LVV-T1332 \\
\hline
DMS-REQ-0374 & LVV-3395 &  \\ \hline
 \cdashline{2-3} DMS-REQ-0374 & LVV-9790 &  \\ \hline
 \cdashline{2-3} DMS-REQ-0374 & LVV-9791 &  \\ \hline
\hline
DMS-REQ-0376 & LVV-3396 &  \\ \hline
 \cdashline{2-3} DMS-REQ-0376 & LVV-9795 &  \\ \hline
 \cdashline{2-3} DMS-REQ-0376 & LVV-9796 &  \\ \hline
\hline
DMS-REQ-0373 & LVV-3397 &  \\ \hline
 \cdashline{2-3} DMS-REQ-0373 & LVV-9789 &  \\ \hline
\hline
DMS-REQ-0375 & LVV-3398 &  \\ \hline
 \cdashline{2-3} DMS-REQ-0375 & LVV-9792 &  \\ \hline
 \cdashline{2-3} DMS-REQ-0375 & LVV-9793 &  \\ \hline
 \cdashline{2-3} DMS-REQ-0375 & LVV-9794 &  \\ \hline
\hline
 DMS-REQ-0358  &
 LVV-3400  &
LVV-T1250 \\
 \cdashline{2-3}  &
 LVV-9788  &
LVV-T1251 \\
\hline
 DMS-REQ-0361  &
 LVV-3403  &
LVV-T1088 \\
 &
 &
LVV-T1089 \\
 &
 &
LVV-T1090 \\
\hline
CA-DM-DAQ-ICD-0094 & LVV-4669 &  \\ \hline
 \cdashline{2-3} CA-DM-DAQ-ICD-0094 & LVV-4670 &  \\ \hline
\hline
CA-DM-DAQ-ICD-0082 & LVV-4675 &  \\ \hline
 \cdashline{2-3} CA-DM-DAQ-ICD-0082 & LVV-4676 &  \\ \hline
\hline
CA-DM-DAQ-ICD-0093 & LVV-4729 &  \\ \hline
 \cdashline{2-3} CA-DM-DAQ-ICD-0093 & LVV-4730 &  \\ \hline
\hline
CA-DM-DAQ-ICD-0097 & LVV-4735 &  \\ \hline
 \cdashline{2-3} CA-DM-DAQ-ICD-0097 & LVV-4736 &  \\ \hline
\hline
CA-DM-DAQ-ICD-0059 & LVV-4747 &  \\ \hline
 \cdashline{2-3} CA-DM-DAQ-ICD-0059 & LVV-4748 &  \\ \hline
\hline
CA-DM-DAQ-ICD-0060 & LVV-4753 &  \\ \hline
 \cdashline{2-3} CA-DM-DAQ-ICD-0060 & LVV-4754 &  \\ \hline
\hline
CA-DM-DAQ-ICD-0081 & LVV-4759 &  \\ \hline
 \cdashline{2-3} CA-DM-DAQ-ICD-0081 & LVV-4760 &  \\ \hline
\hline
CA-DM-DAQ-ICD-0047 & LVV-4765 &  \\ \hline
 \cdashline{2-3} CA-DM-DAQ-ICD-0047 & LVV-4766 &  \\ \hline
\hline
CA-DM-DAQ-ICD-0098 & LVV-4771 &  \\ \hline
 \cdashline{2-3} CA-DM-DAQ-ICD-0098 & LVV-4772 &  \\ \hline
\hline
CA-DM-DAQ-ICD-0100 & LVV-4777 &  \\ \hline
 \cdashline{2-3} CA-DM-DAQ-ICD-0100 & LVV-4778 &  \\ \hline
\hline
CA-DM-DAQ-ICD-0092 & LVV-4784 &  \\ \hline
\hline
CA-DM-DAQ-ICD-0084 & LVV-4789 &  \\ \hline
 \cdashline{2-3} CA-DM-DAQ-ICD-0084 & LVV-4790 &  \\ \hline
\hline
CA-DM-DAQ-ICD-0099 & LVV-4795 &  \\ \hline
 \cdashline{2-3} CA-DM-DAQ-ICD-0099 & LVV-4796 &  \\ \hline
\hline
CA-DM-DAQ-ICD-0085 & LVV-4801 &  \\ \hline
 \cdashline{2-3} CA-DM-DAQ-ICD-0085 & LVV-4802 &  \\ \hline
\hline
CA-DM-DAQ-ICD-0086 & LVV-4807 &  \\ \hline
 \cdashline{2-3} CA-DM-DAQ-ICD-0086 & LVV-4808 &  \\ \hline
\hline
CA-DM-DAQ-ICD-0091 & LVV-4819 &  \\ \hline
 \cdashline{2-3} CA-DM-DAQ-ICD-0091 & LVV-4820 &  \\ \hline
\hline
CA-DM-DAQ-ICD-0075 & LVV-4825 &  \\ \hline
 \cdashline{2-3} CA-DM-DAQ-ICD-0075 & LVV-4826 &  \\ \hline
\hline
CA-DM-DAQ-ICD-0080 & LVV-4831 &  \\ \hline
 \cdashline{2-3} CA-DM-DAQ-ICD-0080 & LVV-4832 &  \\ \hline
\hline
CA-DM-CON-ICD-0003 & LVV-4843 &  \\ \hline
 \cdashline{2-3} CA-DM-CON-ICD-0003 & LVV-4844 &  \\ \hline
\hline
CA-DM-CON-ICD-0004 & LVV-4849 &  \\ \hline
 \cdashline{2-3} CA-DM-CON-ICD-0004 & LVV-4850 &  \\ \hline
\hline
CA-DM-CON-ICD-0019 & LVV-4855 &  \\ \hline
 \cdashline{2-3} CA-DM-CON-ICD-0019 & LVV-4856 &  \\ \hline
\hline
CA-DM-CON-ICD-0008 & LVV-4861 &  \\ \hline
 \cdashline{2-3} CA-DM-CON-ICD-0008 & LVV-4862 &  \\ \hline
\hline
CA-DM-CON-ICD-0002 & LVV-4873 &  \\ \hline
 \cdashline{2-3} CA-DM-CON-ICD-0002 & LVV-4874 &  \\ \hline
\hline
CA-DM-CON-ICD-0005 & LVV-4879 &  \\ \hline
 \cdashline{2-3} CA-DM-CON-ICD-0005 & LVV-4880 &  \\ \hline
\hline
CA-DM-CON-ICD-0001 & LVV-4885 &  \\ \hline
 \cdashline{2-3} CA-DM-CON-ICD-0001 & LVV-4886 &  \\ \hline
\hline
CA-DM-CON-ICD-0018 & LVV-4897 &  \\ \hline
 \cdashline{2-3} CA-DM-CON-ICD-0018 & LVV-4898 &  \\ \hline
\hline
CA-DM-CON-ICD-0007 & LVV-4903 &  \\ \hline
 \cdashline{2-3} CA-DM-CON-ICD-0007 & LVV-4904 &  \\ \hline
\hline
CA-DM-CON-ICD-0016 & LVV-4909 &  \\ \hline
 \cdashline{2-3} CA-DM-CON-ICD-0016 & LVV-4910 &  \\ \hline
\hline
CA-DM-CON-ICD-0014 & LVV-4915 &  \\ \hline
 \cdashline{2-3} CA-DM-CON-ICD-0014 & LVV-4916 &  \\ \hline
\hline
CA-DM-CON-ICD-0015 & LVV-4921 &  \\ \hline
 \cdashline{2-3} CA-DM-CON-ICD-0015 & LVV-4922 &  \\ \hline
\hline
OCS-DM-COM-ICD-0040 & LVV-5237 &  \\ \hline
 \cdashline{2-3} OCS-DM-COM-ICD-0040 & LVV-5238 &  \\ \hline
\hline
OCS-DM-COM-ICD-0009 & LVV-5243 &  \\ \hline
 \cdashline{2-3} OCS-DM-COM-ICD-0009 & LVV-5244 &  \\ \hline
\hline
OCS-DM-COM-ICD-0013 & LVV-5249 &  \\ \hline
 \cdashline{2-3} OCS-DM-COM-ICD-0013 & LVV-5250 &  \\ \hline
\hline
OCS-DM-COM-ICD-0015 & LVV-5255 &  \\ \hline
 \cdashline{2-3} OCS-DM-COM-ICD-0015 & LVV-5256 &  \\ \hline
\hline
OCS-DM-COM-ICD-0014 & LVV-5261 &  \\ \hline
 \cdashline{2-3} OCS-DM-COM-ICD-0014 & LVV-5262 &  \\ \hline
\hline
OCS-DM-COM-ICD-0038 & LVV-5267 &  \\ \hline
 \cdashline{2-3} OCS-DM-COM-ICD-0038 & LVV-5268 &  \\ \hline
\hline
OCS-DM-COM-ICD-0039 & LVV-5273 &  \\ \hline
 \cdashline{2-3} OCS-DM-COM-ICD-0039 & LVV-5274 &  \\ \hline
\hline
OCS-DM-COM-ICD-0037 & LVV-5279 &  \\ \hline
 \cdashline{2-3} OCS-DM-COM-ICD-0037 & LVV-5280 &  \\ \hline
\hline
OCS-DM-COM-ICD-0036 & LVV-5285 &  \\ \hline
 \cdashline{2-3} OCS-DM-COM-ICD-0036 & LVV-5286 &  \\ \hline
\hline
OCS-DM-COM-ICD-0012 & LVV-5291 &  \\ \hline
 \cdashline{2-3} OCS-DM-COM-ICD-0012 & LVV-5292 &  \\ \hline
\hline
OCS-DM-COM-ICD-0003 & LVV-5297 &  \\ \hline
 \cdashline{2-3} OCS-DM-COM-ICD-0003 & LVV-5298 &  \\ \hline
\hline
OCS-DM-COM-ICD-0034 & LVV-5303 &  \\ \hline
 \cdashline{2-3} OCS-DM-COM-ICD-0034 & LVV-5304 &  \\ \hline
\hline
OCS-DM-COM-ICD-0032 & LVV-5309 &  \\ \hline
 \cdashline{2-3} OCS-DM-COM-ICD-0032 & LVV-5310 &  \\ \hline
\hline
OCS-DM-COM-ICD-0006 & LVV-5315 &  \\ \hline
 \cdashline{2-3} OCS-DM-COM-ICD-0006 & LVV-5316 &  \\ \hline
\hline
OCS-DM-COM-ICD-0004 & LVV-5321 &  \\ \hline
 \cdashline{2-3} OCS-DM-COM-ICD-0004 & LVV-5322 &  \\ \hline
\hline
OCS-DM-COM-ICD-0008 & LVV-5327 &  \\ \hline
 \cdashline{2-3} OCS-DM-COM-ICD-0008 & LVV-5328 &  \\ \hline
\hline
OCS-DM-COM-ICD-0033 & LVV-5333 &  \\ \hline
 \cdashline{2-3} OCS-DM-COM-ICD-0033 & LVV-5334 &  \\ \hline
\hline
OCS-DM-COM-ICD-0005 & LVV-5339 &  \\ \hline
 \cdashline{2-3} OCS-DM-COM-ICD-0005 & LVV-5340 &  \\ \hline
\hline
OCS-DM-COM-ICD-0035 & LVV-5345 &  \\ \hline
 \cdashline{2-3} OCS-DM-COM-ICD-0035 & LVV-5346 &  \\ \hline
\hline
OCS-DM-COM-ICD-0007 & LVV-5351 &  \\ \hline
 \cdashline{2-3} OCS-DM-COM-ICD-0007 & LVV-5352 &  \\ \hline
\hline
OCS-DM-COM-ICD-0048 & LVV-5357 &  \\ \hline
 \cdashline{2-3} OCS-DM-COM-ICD-0048 & LVV-5358 &  \\ \hline
\hline
OCS-DM-COM-ICD-0055 & LVV-5363 &  \\ \hline
 \cdashline{2-3} OCS-DM-COM-ICD-0055 & LVV-5364 &  \\ \hline
\hline
OCS-DM-COM-ICD-0054 & LVV-5369 &  \\ \hline
 \cdashline{2-3} OCS-DM-COM-ICD-0054 & LVV-5370 &  \\ \hline
\hline
OCS-DM-COM-ICD-0019 & LVV-5375 &  \\ \hline
 \cdashline{2-3} OCS-DM-COM-ICD-0019 & LVV-5376 &  \\ \hline
\hline
OCS-DM-COM-ICD-0017 & LVV-5381 &  \\ \hline
 \cdashline{2-3} OCS-DM-COM-ICD-0017 & LVV-5382 &  \\ \hline
\hline
OCS-DM-COM-ICD-0018 & LVV-5387 &  \\ \hline
 \cdashline{2-3} OCS-DM-COM-ICD-0018 & LVV-5388 &  \\ \hline
\hline
OCS-DM-COM-ICD-0021 & LVV-5393 &  \\ \hline
 \cdashline{2-3} OCS-DM-COM-ICD-0021 & LVV-5394 &  \\ \hline
\hline
OCS-DM-COM-ICD-0020 & LVV-5399 &  \\ \hline
 \cdashline{2-3} OCS-DM-COM-ICD-0020 & LVV-5400 &  \\ \hline
\hline
OCS-DM-COM-ICD-0047 & LVV-5405 &  \\ \hline
 \cdashline{2-3} OCS-DM-COM-ICD-0047 & LVV-5406 &  \\ \hline
\hline
OCS-DM-COM-ICD-0046 & LVV-5411 &  \\ \hline
 \cdashline{2-3} OCS-DM-COM-ICD-0046 & LVV-5412 &  \\ \hline
\hline
OCS-DM-COM-ICD-0045 & LVV-5417 &  \\ \hline
 \cdashline{2-3} OCS-DM-COM-ICD-0045 & LVV-5418 &  \\ \hline
\hline
OCS-DM-COM-ICD-0043 & LVV-5423 &  \\ \hline
 \cdashline{2-3} OCS-DM-COM-ICD-0043 & LVV-5424 &  \\ \hline
\hline
OCS-DM-COM-ICD-0044 & LVV-5429 &  \\ \hline
 \cdashline{2-3} OCS-DM-COM-ICD-0044 & LVV-5430 &  \\ \hline
\hline
OCS-DM-COM-ICD-0052 & LVV-5435 &  \\ \hline
 \cdashline{2-3} OCS-DM-COM-ICD-0052 & LVV-5436 &  \\ \hline
\hline
OCS-DM-COM-ICD-0051 & LVV-5441 &  \\ \hline
 \cdashline{2-3} OCS-DM-COM-ICD-0051 & LVV-5442 &  \\ \hline
\hline
OCS-DM-COM-ICD-0056 & LVV-5447 &  \\ \hline
 \cdashline{2-3} OCS-DM-COM-ICD-0056 & LVV-5448 &  \\ \hline
\hline
OCS-DM-COM-ICD-0050 & LVV-5453 &  \\ \hline
 \cdashline{2-3} OCS-DM-COM-ICD-0050 & LVV-5454 &  \\ \hline
\hline
OCS-DM-COM-ICD-0053 & LVV-5459 &  \\ \hline
 \cdashline{2-3} OCS-DM-COM-ICD-0053 & LVV-5460 &  \\ \hline
\hline
OCS-DM-COM-ICD-0022 & LVV-5465 &  \\ \hline
 \cdashline{2-3} OCS-DM-COM-ICD-0022 & LVV-5466 &  \\ \hline
\hline
OCS-DM-COM-ICD-0049 & LVV-5471 &  \\ \hline
 \cdashline{2-3} OCS-DM-COM-ICD-0049 & LVV-5472 &  \\ \hline
\hline
OCS-DM-COM-ICD-0023 & LVV-5477 &  \\ \hline
 \cdashline{2-3} OCS-DM-COM-ICD-0023 & LVV-5478 &  \\ \hline
\hline
OCS-DM-COM-ICD-0025 & LVV-5483 &  \\ \hline
 \cdashline{2-3} OCS-DM-COM-ICD-0025 & LVV-5484 &  \\ \hline
\hline
OCS-DM-COM-ICD-0029 & LVV-5489 &  \\ \hline
 \cdashline{2-3} OCS-DM-COM-ICD-0029 & LVV-5490 &  \\ \hline
\hline
OCS-DM-COM-ICD-0042 & LVV-5495 &  \\ \hline
 \cdashline{2-3} OCS-DM-COM-ICD-0042 & LVV-5496 &  \\ \hline
\hline
OCS-DM-COM-ICD-0030 & LVV-5501 &  \\ \hline
 \cdashline{2-3} OCS-DM-COM-ICD-0030 & LVV-5502 &  \\ \hline
\hline
OCS-DM-COM-ICD-0028 & LVV-5513 &  \\ \hline
 \cdashline{2-3} OCS-DM-COM-ICD-0028 & LVV-5514 &  \\ \hline
\hline
OCS-DM-COM-ICD-0041 & LVV-5519 &  \\ \hline
 \cdashline{2-3} OCS-DM-COM-ICD-0041 & LVV-5520 &  \\ \hline
\hline
OCS-DM-COM-ICD-0031 & LVV-5531 &  \\ \hline
 \cdashline{2-3} OCS-DM-COM-ICD-0031 & LVV-5532 &  \\ \hline
\hline
OCS-DM-COM-ICD-0002 & LVV-5537 &  \\ \hline
 \cdashline{2-3} OCS-DM-COM-ICD-0002 & LVV-5538 &  \\ \hline
\hline
OCS-DM-COM-ICD-0001 & LVV-5543 &  \\ \hline
 \cdashline{2-3} OCS-DM-COM-ICD-0001 & LVV-5544 &  \\ \hline
\hline
DM-TS-CON-ICD-0003 & LVV-5628 &  \\ \hline
 \cdashline{2-3} DM-TS-CON-ICD-0003 & LVV-5629 &  \\ \hline
\hline
DM-TS-CON-ICD-0010 & LVV-5634 &  \\ \hline
 \cdashline{2-3} DM-TS-CON-ICD-0010 & LVV-5635 &  \\ \hline
\hline
DM-TS-CON-ICD-0006 & LVV-5652 &  \\ \hline
 \cdashline{2-3} DM-TS-CON-ICD-0006 & LVV-5653 &  \\ \hline
\hline
DM-TS-CON-ICD-0007 & LVV-5658 &  \\ \hline
 \cdashline{2-3} DM-TS-CON-ICD-0007 & LVV-5659 &  \\ \hline
\hline
DM-TS-CON-ICD-0009 & LVV-5664 &  \\ \hline
 \cdashline{2-3} DM-TS-CON-ICD-0009 & LVV-5665 &  \\ \hline
\hline
DM-TS-CON-ICD-0008 & LVV-5670 &  \\ \hline
 \cdashline{2-3} DM-TS-CON-ICD-0008 & LVV-5671 &  \\ \hline
\hline
DM-TS-CON-ICD-0004 & LVV-5676 &  \\ \hline
 \cdashline{2-3} DM-TS-CON-ICD-0004 & LVV-5677 &  \\ \hline
\hline
CA-DM-SUP-ICD-0026 & LVV-6140 &  \\ \hline
 \cdashline{2-3} CA-DM-SUP-ICD-0026 & LVV-6141 &  \\ \hline
\hline
CA-DM-SUP-ICD-0027 & LVV-6146 &  \\ \hline
 \cdashline{2-3} CA-DM-SUP-ICD-0027 & LVV-6147 &  \\ \hline
\hline
CA-DM-SUP-ICD-0024 & LVV-6152 &  \\ \hline
 \cdashline{2-3} CA-DM-SUP-ICD-0024 & LVV-6153 &  \\ \hline
\hline
CA-DM-SUP-ICD-0023 & LVV-6158 &  \\ \hline
 \cdashline{2-3} CA-DM-SUP-ICD-0023 & LVV-6159 &  \\ \hline
\hline
CA-DM-SUP-ICD-0025 & LVV-6164 &  \\ \hline
 \cdashline{2-3} CA-DM-SUP-ICD-0025 & LVV-6165 &  \\ \hline
\hline
CA-DM-SUP-ICD-0022 & LVV-6170 &  \\ \hline
 \cdashline{2-3} CA-DM-SUP-ICD-0022 & LVV-6171 &  \\ \hline
\hline
CA-DM-SUP-ICD-0021 & LVV-6176 &  \\ \hline
 \cdashline{2-3} CA-DM-SUP-ICD-0021 & LVV-6177 &  \\ \hline
\hline
CA-DM-SUP-ICD-0028 & LVV-6182 &  \\ \hline
 \cdashline{2-3} CA-DM-SUP-ICD-0028 & LVV-6183 &  \\ \hline
\hline
CA-DM-SUP-ICD-0029 & LVV-6188 &  \\ \hline
 \cdashline{2-3} CA-DM-SUP-ICD-0029 & LVV-6189 &  \\ \hline
\hline
CA-DM-SUP-ICD-0031 & LVV-6194 &  \\ \hline
 \cdashline{2-3} CA-DM-SUP-ICD-0031 & LVV-6195 &  \\ \hline
\hline
CA-DM-SUP-ICD-0030 & LVV-6200 &  \\ \hline
 \cdashline{2-3} CA-DM-SUP-ICD-0030 & LVV-6201 &  \\ \hline
\hline
CA-DM-SUP-ICD-0008 & LVV-6206 &  \\ \hline
 \cdashline{2-3} CA-DM-SUP-ICD-0008 & LVV-6207 &  \\ \hline
\hline
CA-DM-SUP-ICD-0007 & LVV-6212 &  \\ \hline
 \cdashline{2-3} CA-DM-SUP-ICD-0007 & LVV-6213 &  \\ \hline
\hline
CA-DM-SUP-ICD-0009 & LVV-6218 &  \\ \hline
 \cdashline{2-3} CA-DM-SUP-ICD-0009 & LVV-6219 &  \\ \hline
\hline
CA-DM-SUP-ICD-0010 & LVV-6224 &  \\ \hline
 \cdashline{2-3} CA-DM-SUP-ICD-0010 & LVV-6225 &  \\ \hline
\hline
CA-DM-SUP-ICD-0020 & LVV-6230 &  \\ \hline
 \cdashline{2-3} CA-DM-SUP-ICD-0020 & LVV-6231 &  \\ \hline
\hline
CA-DM-SUP-ICD-0019 & LVV-6236 &  \\ \hline
 \cdashline{2-3} CA-DM-SUP-ICD-0019 & LVV-6237 &  \\ \hline
\hline
CA-DM-SUP-ICD-0005 & LVV-6242 &  \\ \hline
 \cdashline{2-3} CA-DM-SUP-ICD-0005 & LVV-6243 &  \\ \hline
\hline
CA-DM-SUP-ICD-0006 & LVV-6248 &  \\ \hline
 \cdashline{2-3} CA-DM-SUP-ICD-0006 & LVV-6249 &  \\ \hline
\hline
CA-DM-SUP-ICD-0002 & LVV-6254 &  \\ \hline
 \cdashline{2-3} CA-DM-SUP-ICD-0002 & LVV-6255 &  \\ \hline
\hline
CA-DM-SUP-ICD-0003 & LVV-6260 &  \\ \hline
 \cdashline{2-3} CA-DM-SUP-ICD-0003 & LVV-6261 &  \\ \hline
\hline
CA-DM-SUP-ICD-0004 & LVV-6266 &  \\ \hline
 \cdashline{2-3} CA-DM-SUP-ICD-0004 & LVV-6267 &  \\ \hline
\hline
CA-DM-SUP-ICD-0016 & LVV-6272 &  \\ \hline
 \cdashline{2-3} CA-DM-SUP-ICD-0016 & LVV-6273 &  \\ \hline
\hline
CA-DM-SUP-ICD-0015 & LVV-6278 &  \\ \hline
 \cdashline{2-3} CA-DM-SUP-ICD-0015 & LVV-6279 &  \\ \hline
\hline
CA-DM-SUP-ICD-0017 & LVV-6284 &  \\ \hline
 \cdashline{2-3} CA-DM-SUP-ICD-0017 & LVV-6285 &  \\ \hline
\hline
CA-DM-SUP-ICD-0014 & LVV-6290 &  \\ \hline
 \cdashline{2-3} CA-DM-SUP-ICD-0014 & LVV-6291 &  \\ \hline
\hline
CA-DM-SUP-ICD-0013 & LVV-6296 &  \\ \hline
 \cdashline{2-3} CA-DM-SUP-ICD-0013 & LVV-6297 &  \\ \hline
\hline
CA-DM-SUP-ICD-0011 & LVV-6302 &  \\ \hline
 \cdashline{2-3} CA-DM-SUP-ICD-0011 & LVV-6303 &  \\ \hline
\hline
CA-DM-SUP-ICD-0012 & LVV-6308 &  \\ \hline
 \cdashline{2-3} CA-DM-SUP-ICD-0012 & LVV-6309 &  \\ \hline
\hline
CA-DM-SUP-ICD-0018 & LVV-6314 &  \\ \hline
 \cdashline{2-3} CA-DM-SUP-ICD-0018 & LVV-6315 &  \\ \hline
\hline
CA-DM-SUP-ICD-0001 & LVV-6320 &  \\ \hline
 \cdashline{2-3} CA-DM-SUP-ICD-0001 & LVV-6321 &  \\ \hline
\hline
EP-DM-CON-ICD-0004 & LVV-6324 &  \\ \hline
 \cdashline{2-3} EP-DM-CON-ICD-0004 & LVV-6325 &  \\ \hline
\hline
EP-DM-CON-ICD-0021 & LVV-6330 &  \\ \hline
 \cdashline{2-3} EP-DM-CON-ICD-0021 & LVV-6331 &  \\ \hline
\hline
EP-DM-CON-ICD-0009 & LVV-6342 &  \\ \hline
 \cdashline{2-3} EP-DM-CON-ICD-0009 & LVV-6343 &  \\ \hline
\hline
EP-DM-CON-ICD-0034 & LVV-6348 &  \\ \hline
 \cdashline{2-3} EP-DM-CON-ICD-0034 & LVV-6349 &  \\ \hline
\hline
EP-DM-CON-ICD-0031 & LVV-6360 &  \\ \hline
 \cdashline{2-3} EP-DM-CON-ICD-0031 & LVV-6361 &  \\ \hline
\hline
EP-DM-CON-ICD-0019 & LVV-6372 &  \\ \hline
 \cdashline{2-3} EP-DM-CON-ICD-0019 & LVV-6373 &  \\ \hline
\hline
EP-DM-CON-ICD-0002 & LVV-6378 &  \\ \hline
 \cdashline{2-3} EP-DM-CON-ICD-0002 & LVV-6379 &  \\ \hline
\hline
EP-DM-CON-ICD-0033 & LVV-6384 &  \\ \hline
 \cdashline{2-3} EP-DM-CON-ICD-0033 & LVV-6385 &  \\ \hline
\hline
EP-DM-CON-ICD-0032 & LVV-6390 &  \\ \hline
 \cdashline{2-3} EP-DM-CON-ICD-0032 & LVV-6391 &  \\ \hline
\hline
EP-DM-CON-ICD-0020 & LVV-6402 &  \\ \hline
 \cdashline{2-3} EP-DM-CON-ICD-0020 & LVV-6403 &  \\ \hline
\hline
DM-TS-AUX-ICD-0020 & LVV-6420 &  \\ \hline
 \cdashline{2-3} DM-TS-AUX-ICD-0020 & LVV-6421 &  \\ \hline
\hline
DM-TS-AUX-ICD-0029 & LVV-6426 &  \\ \hline
 \cdashline{2-3} DM-TS-AUX-ICD-0029 & LVV-6427 &  \\ \hline
\hline
DM-TS-AUX-ICD-0027 & LVV-6432 &  \\ \hline
 \cdashline{2-3} DM-TS-AUX-ICD-0027 & LVV-6433 &  \\ \hline
\hline
DM-TS-AUX-ICD-0025 & LVV-6456 &  \\ \hline
 \cdashline{2-3} DM-TS-AUX-ICD-0025 & LVV-6457 &  \\ \hline
\hline
DM-TS-AUX-ICD-0026 & LVV-6462 &  \\ \hline
 \cdashline{2-3} DM-TS-AUX-ICD-0026 & LVV-6463 &  \\ \hline
\hline
DM-TS-AUX-ICD-0024 & LVV-6468 &  \\ \hline
 \cdashline{2-3} DM-TS-AUX-ICD-0024 & LVV-6469 &  \\ \hline
\hline
DM-TS-AUX-ICD-0037 & LVV-6474 &  \\ \hline
 \cdashline{2-3} DM-TS-AUX-ICD-0037 & LVV-6475 &  \\ \hline
\hline
DM-TS-AUX-ICD-0002 & LVV-6480 &  \\ \hline
 \cdashline{2-3} DM-TS-AUX-ICD-0002 & LVV-6481 &  \\ \hline
\hline
DM-TS-AUX-ICD-0001 & LVV-6486 &  \\ \hline
 \cdashline{2-3} DM-TS-AUX-ICD-0001 & LVV-6487 &  \\ \hline
\hline
DM-TS-AUX-ICD-0007 & LVV-6492 &  \\ \hline
 \cdashline{2-3} DM-TS-AUX-ICD-0007 & LVV-6493 &  \\ \hline
\hline
DM-TS-AUX-ICD-0008 & LVV-6498 &  \\ \hline
 \cdashline{2-3} DM-TS-AUX-ICD-0008 & LVV-6499 &  \\ \hline
\hline
DM-TS-AUX-ICD-0004 & LVV-6528 &  \\ \hline
 \cdashline{2-3} DM-TS-AUX-ICD-0004 & LVV-6529 &  \\ \hline
\hline
DM-TS-AUX-ICD-0003 & LVV-6534 &  \\ \hline
 \cdashline{2-3} DM-TS-AUX-ICD-0003 & LVV-6535 &  \\ \hline
\hline
DM-TS-AUX-ICD-0034 & LVV-6540 &  \\ \hline
 \cdashline{2-3} DM-TS-AUX-ICD-0034 & LVV-6541 &  \\ \hline
\hline
DM-TS-AUX-ICD-0036 & LVV-6546 &  \\ \hline
 \cdashline{2-3} DM-TS-AUX-ICD-0036 & LVV-6547 &  \\ \hline
\hline
DM-TS-AUX-ICD-0019 & LVV-6552 &  \\ \hline
 \cdashline{2-3} DM-TS-AUX-ICD-0019 & LVV-6553 &  \\ \hline
\hline
DM-TS-AUX-ICD-0018 & LVV-6558 &  \\ \hline
 \cdashline{2-3} DM-TS-AUX-ICD-0018 & LVV-6559 &  \\ \hline
\hline
DM-TS-AUX-ICD-0014 & LVV-6564 &  \\ \hline
 \cdashline{2-3} DM-TS-AUX-ICD-0014 & LVV-6565 &  \\ \hline
\hline
DM-TS-AUX-ICD-0012 & LVV-6570 &  \\ \hline
 \cdashline{2-3} DM-TS-AUX-ICD-0012 & LVV-6571 &  \\ \hline
\hline
DM-TS-AUX-ICD-0028 & LVV-6576 &  \\ \hline
 \cdashline{2-3} DM-TS-AUX-ICD-0028 & LVV-6577 &  \\ \hline
\hline
DM-TS-AUX-ICD-0035 & LVV-6594 &  \\ \hline
 \cdashline{2-3} DM-TS-AUX-ICD-0035 & LVV-6595 &  \\ \hline
\hline
DM-TS-AUX-ICD-0033 & LVV-6600 &  \\ \hline
 \cdashline{2-3} DM-TS-AUX-ICD-0033 & LVV-6601 &  \\ \hline
\hline
DM-TS-AUX-ICD-0032 & LVV-6606 &  \\ \hline
 \cdashline{2-3} DM-TS-AUX-ICD-0032 & LVV-6607 &  \\ \hline
\hline
EP-DM-CON-ICD-0036 & LVV-6751 &  \\ \hline
 \cdashline{2-3} EP-DM-CON-ICD-0036 & LVV-6752 &  \\ \hline
\hline
EP-DM-CON-ICD-0035 & LVV-6757 &  \\ \hline
 \cdashline{2-3} EP-DM-CON-ICD-0035 & LVV-6758 &  \\ \hline
\hline
EP-DM-CON-ICD-0037 & LVV-6763 &  \\ \hline
 \cdashline{2-3} EP-DM-CON-ICD-0037 & LVV-6764 &  \\ \hline
\hline
SYS-ALL-COM-ICD-0047 & LVV-6771 &  \\ \hline
 \cdashline{2-3} SYS-ALL-COM-ICD-0047 & LVV-6772 &  \\ \hline
\hline
SYS-ALL-COM-ICD-0048 & LVV-6777 &  \\ \hline
 \cdashline{2-3} SYS-ALL-COM-ICD-0048 & LVV-6778 &  \\ \hline
\hline
SYS-ALL-COM-ICD-0043 & LVV-6783 &  \\ \hline
 \cdashline{2-3} SYS-ALL-COM-ICD-0043 & LVV-6784 &  \\ \hline
\hline
SYS-ALL-COM-ICD-0046 & LVV-6789 &  \\ \hline
 \cdashline{2-3} SYS-ALL-COM-ICD-0046 & LVV-6790 &  \\ \hline
\hline
SYS-ALL-COM-ICD-0044 & LVV-6795 &  \\ \hline
 \cdashline{2-3} SYS-ALL-COM-ICD-0044 & LVV-6796 &  \\ \hline
\hline
SYS-ALL-COM-ICD-0045 & LVV-6801 &  \\ \hline
 \cdashline{2-3} SYS-ALL-COM-ICD-0045 & LVV-6802 &  \\ \hline
\hline
SYS-ALL-COM-ICD-0042 & LVV-6807 &  \\ \hline
 \cdashline{2-3} SYS-ALL-COM-ICD-0042 & LVV-6808 &  \\ \hline
\hline
SYS-ALL-COM-ICD-0029 & LVV-6813 &  \\ \hline
 \cdashline{2-3} SYS-ALL-COM-ICD-0029 & LVV-6814 &  \\ \hline
\hline
SYS-ALL-COM-ICD-0028 & LVV-6819 &  \\ \hline
 \cdashline{2-3} SYS-ALL-COM-ICD-0028 & LVV-6820 &  \\ \hline
\hline
SYS-ALL-COM-ICD-0005 & LVV-6825 &  \\ \hline
 \cdashline{2-3} SYS-ALL-COM-ICD-0005 & LVV-6826 &  \\ \hline
\hline
SYS-ALL-COM-ICD-0030 & LVV-6831 &  \\ \hline
 \cdashline{2-3} SYS-ALL-COM-ICD-0030 & LVV-6832 &  \\ \hline
\hline
SYS-ALL-COM-ICD-0026 & LVV-6837 &  \\ \hline
 \cdashline{2-3} SYS-ALL-COM-ICD-0026 & LVV-6838 &  \\ \hline
\hline
SYS-ALL-COM-ICD-0027 & LVV-6843 &  \\ \hline
 \cdashline{2-3} SYS-ALL-COM-ICD-0027 & LVV-6844 &  \\ \hline
\hline
SYS-ALL-COM-ICD-0050 & LVV-6849 &  \\ \hline
 \cdashline{2-3} SYS-ALL-COM-ICD-0050 & LVV-6850 &  \\ \hline
\hline
SYS-ALL-COM-ICD-0049 & LVV-6855 &  \\ \hline
 \cdashline{2-3} SYS-ALL-COM-ICD-0049 & LVV-6856 &  \\ \hline
\hline
SYS-ALL-COM-ICD-0031 & LVV-6861 &  \\ \hline
 \cdashline{2-3} SYS-ALL-COM-ICD-0031 & LVV-6862 &  \\ \hline
\hline
SYS-ALL-COM-ICD-0033 & LVV-6867 &  \\ \hline
 \cdashline{2-3} SYS-ALL-COM-ICD-0033 & LVV-6868 &  \\ \hline
\hline
SYS-ALL-COM-ICD-0035 & LVV-6873 &  \\ \hline
 \cdashline{2-3} SYS-ALL-COM-ICD-0035 & LVV-6874 &  \\ \hline
\hline
SYS-ALL-COM-ICD-0037 & LVV-6879 &  \\ \hline
 \cdashline{2-3} SYS-ALL-COM-ICD-0037 & LVV-6880 &  \\ \hline
\hline
SYS-ALL-COM-ICD-0040 & LVV-6885 &  \\ \hline
 \cdashline{2-3} SYS-ALL-COM-ICD-0040 & LVV-6886 &  \\ \hline
\hline
SYS-ALL-COM-ICD-0036 & LVV-6891 &  \\ \hline
 \cdashline{2-3} SYS-ALL-COM-ICD-0036 & LVV-6892 &  \\ \hline
\hline
SYS-ALL-COM-ICD-0041 & LVV-6897 &  \\ \hline
 \cdashline{2-3} SYS-ALL-COM-ICD-0041 & LVV-6898 &  \\ \hline
\hline
SYS-ALL-COM-ICD-0038 & LVV-6903 &  \\ \hline
 \cdashline{2-3} SYS-ALL-COM-ICD-0038 & LVV-6904 &  \\ \hline
\hline
SYS-ALL-COM-ICD-0034 & LVV-6909 &  \\ \hline
 \cdashline{2-3} SYS-ALL-COM-ICD-0034 & LVV-6910 &  \\ \hline
\hline
SYS-ALL-COM-ICD-0032 & LVV-6915 &  \\ \hline
 \cdashline{2-3} SYS-ALL-COM-ICD-0032 & LVV-6916 &  \\ \hline
\hline
SYS-ALL-COM-ICD-0039 & LVV-6921 &  \\ \hline
 \cdashline{2-3} SYS-ALL-COM-ICD-0039 & LVV-6922 &  \\ \hline
\hline
CPT-OCS-INT-ICD-0001 & LVV-6927 &  \\ \hline
 \cdashline{2-3} CPT-OCS-INT-ICD-0001 & LVV-6928 &  \\ \hline
\hline
CPT-OCS-INT-ICD-0005 & LVV-6933 &  \\ \hline
 \cdashline{2-3} CPT-OCS-INT-ICD-0005 & LVV-6934 &  \\ \hline
\hline
CPT-OCS-INT-ICD-0006 & LVV-6939 &  \\ \hline
 \cdashline{2-3} CPT-OCS-INT-ICD-0006 & LVV-6940 &  \\ \hline
\hline
CPT-OCS-INT-ICD-0008 & LVV-6945 &  \\ \hline
 \cdashline{2-3} CPT-OCS-INT-ICD-0008 & LVV-6946 &  \\ \hline
\hline
CPT-OCS-INT-ICD-0040 & LVV-6951 &  \\ \hline
 \cdashline{2-3} CPT-OCS-INT-ICD-0040 & LVV-6952 &  \\ \hline
\hline
CPT-OCS-INT-ICD-0041 & LVV-6957 &  \\ \hline
 \cdashline{2-3} CPT-OCS-INT-ICD-0041 & LVV-6958 &  \\ \hline
\hline
CPT-OCS-INT-ICD-0042 & LVV-6963 &  \\ \hline
 \cdashline{2-3} CPT-OCS-INT-ICD-0042 & LVV-6964 &  \\ \hline
\hline
CPT-OCS-INT-ICD-0002 & LVV-6969 &  \\ \hline
 \cdashline{2-3} CPT-OCS-INT-ICD-0002 & LVV-6970 &  \\ \hline
\hline
CPT-OCS-INT-ICD-0003 & LVV-6975 &  \\ \hline
 \cdashline{2-3} CPT-OCS-INT-ICD-0003 & LVV-6976 &  \\ \hline
\hline
CPT-OCS-INT-ICD-0009 & LVV-6981 &  \\ \hline
 \cdashline{2-3} CPT-OCS-INT-ICD-0009 & LVV-6982 &  \\ \hline
\hline
CPT-OCS-INT-ICD-0072 & LVV-6987 &  \\ \hline
 \cdashline{2-3} CPT-OCS-INT-ICD-0072 & LVV-6988 &  \\ \hline
\hline
CPT-OCS-INT-ICD-0010 & LVV-6993 &  \\ \hline
 \cdashline{2-3} CPT-OCS-INT-ICD-0010 & LVV-6994 &  \\ \hline
\hline
CPT-OCS-INT-ICD-0012 & LVV-6999 &  \\ \hline
 \cdashline{2-3} CPT-OCS-INT-ICD-0012 & LVV-7000 &  \\ \hline
\hline
CPT-OCS-INT-ICD-0004 & LVV-7005 &  \\ \hline
 \cdashline{2-3} CPT-OCS-INT-ICD-0004 & LVV-7006 &  \\ \hline
\hline
CPT-OCS-INT-ICD-0007 & LVV-7011 &  \\ \hline
 \cdashline{2-3} CPT-OCS-INT-ICD-0007 & LVV-7012 &  \\ \hline
\hline
CPT-OCS-INT-ICD-0011 & LVV-7017 &  \\ \hline
 \cdashline{2-3} CPT-OCS-INT-ICD-0011 & LVV-7018 &  \\ \hline
\hline
CPT-OCS-INT-ICD-0049 & LVV-7023 &  \\ \hline
 \cdashline{2-3} CPT-OCS-INT-ICD-0049 & LVV-7024 &  \\ \hline
\hline
CPT-OCS-INT-ICD-0071 & LVV-7029 &  \\ \hline
 \cdashline{2-3} CPT-OCS-INT-ICD-0071 & LVV-7030 &  \\ \hline
\hline
CPT-OCS-INT-ICD-0046 & LVV-7035 &  \\ \hline
 \cdashline{2-3} CPT-OCS-INT-ICD-0046 & LVV-7036 &  \\ \hline
\hline
CPT-OCS-INT-ICD-0045 & LVV-7041 &  \\ \hline
 \cdashline{2-3} CPT-OCS-INT-ICD-0045 & LVV-7042 &  \\ \hline
\hline
CPT-OCS-INT-ICD-0048 & LVV-7047 &  \\ \hline
 \cdashline{2-3} CPT-OCS-INT-ICD-0048 & LVV-7048 &  \\ \hline
\hline
CPT-OCS-INT-ICD-0043 & LVV-7053 &  \\ \hline
 \cdashline{2-3} CPT-OCS-INT-ICD-0043 & LVV-7054 &  \\ \hline
\hline
CPT-OCS-INT-ICD-0044 & LVV-7059 &  \\ \hline
 \cdashline{2-3} CPT-OCS-INT-ICD-0044 & LVV-7060 &  \\ \hline
\hline
CPT-OCS-INT-ICD-0047 & LVV-7065 &  \\ \hline
 \cdashline{2-3} CPT-OCS-INT-ICD-0047 & LVV-7066 &  \\ \hline
\hline
CPT-OCS-INT-ICD-0061 & LVV-7071 &  \\ \hline
 \cdashline{2-3} CPT-OCS-INT-ICD-0061 & LVV-7072 &  \\ \hline
\hline
CPT-OCS-INT-ICD-0057 & LVV-7077 &  \\ \hline
 \cdashline{2-3} CPT-OCS-INT-ICD-0057 & LVV-7078 &  \\ \hline
\hline
CPT-OCS-INT-ICD-0052 & LVV-7083 &  \\ \hline
 \cdashline{2-3} CPT-OCS-INT-ICD-0052 & LVV-7084 &  \\ \hline
\hline
CPT-OCS-INT-ICD-0050 & LVV-7089 &  \\ \hline
 \cdashline{2-3} CPT-OCS-INT-ICD-0050 & LVV-7090 &  \\ \hline
\hline
CPT-OCS-INT-ICD-0053 & LVV-7095 &  \\ \hline
 \cdashline{2-3} CPT-OCS-INT-ICD-0053 & LVV-7096 &  \\ \hline
\hline
CPT-OCS-INT-ICD-0054 & LVV-7101 &  \\ \hline
 \cdashline{2-3} CPT-OCS-INT-ICD-0054 & LVV-7102 &  \\ \hline
\hline
CPT-OCS-INT-ICD-0055 & LVV-7107 &  \\ \hline
 \cdashline{2-3} CPT-OCS-INT-ICD-0055 & LVV-7108 &  \\ \hline
\hline
CPT-OCS-INT-ICD-0051 & LVV-7113 &  \\ \hline
 \cdashline{2-3} CPT-OCS-INT-ICD-0051 & LVV-7114 &  \\ \hline
\hline
CPT-OCS-INT-ICD-0073 & LVV-7119 &  \\ \hline
 \cdashline{2-3} CPT-OCS-INT-ICD-0073 & LVV-7120 &  \\ \hline
\hline
CPT-OCS-INT-ICD-0058 & LVV-7125 &  \\ \hline
 \cdashline{2-3} CPT-OCS-INT-ICD-0058 & LVV-7126 &  \\ \hline
\hline
CPT-OCS-INT-ICD-0059 & LVV-7131 &  \\ \hline
 \cdashline{2-3} CPT-OCS-INT-ICD-0059 & LVV-7132 &  \\ \hline
\hline
CPT-OCS-INT-ICD-0060 & LVV-7137 &  \\ \hline
 \cdashline{2-3} CPT-OCS-INT-ICD-0060 & LVV-7138 &  \\ \hline
\hline
CPT-OCS-INT-ICD-0056 & LVV-7143 &  \\ \hline
 \cdashline{2-3} CPT-OCS-INT-ICD-0056 & LVV-7144 &  \\ \hline
\hline
CPT-OCS-INT-ICD-0063 & LVV-7149 &  \\ \hline
 \cdashline{2-3} CPT-OCS-INT-ICD-0063 & LVV-7150 &  \\ \hline
\hline
CPT-OCS-INT-ICD-0064 & LVV-7155 &  \\ \hline
 \cdashline{2-3} CPT-OCS-INT-ICD-0064 & LVV-7156 &  \\ \hline
\hline
CPT-OCS-INT-ICD-0065 & LVV-7161 &  \\ \hline
 \cdashline{2-3} CPT-OCS-INT-ICD-0065 & LVV-7162 &  \\ \hline
\hline
CPT-OCS-INT-ICD-0066 & LVV-7167 &  \\ \hline
 \cdashline{2-3} CPT-OCS-INT-ICD-0066 & LVV-7168 &  \\ \hline
\hline
CPT-OCS-INT-ICD-0062 & LVV-7173 &  \\ \hline
 \cdashline{2-3} CPT-OCS-INT-ICD-0062 & LVV-7174 &  \\ \hline
\hline
CPT-OCS-INT-ICD-0067 & LVV-7179 &  \\ \hline
 \cdashline{2-3} CPT-OCS-INT-ICD-0067 & LVV-7180 &  \\ \hline
\hline
CPT-OCS-INT-ICD-0068 & LVV-7185 &  \\ \hline
 \cdashline{2-3} CPT-OCS-INT-ICD-0068 & LVV-7186 &  \\ \hline
\hline
CPT-OCS-INT-ICD-0069 & LVV-7191 &  \\ \hline
 \cdashline{2-3} CPT-OCS-INT-ICD-0069 & LVV-7192 &  \\ \hline
\hline
CPT-OCS-INT-ICD-0070 & LVV-7197 &  \\ \hline
 \cdashline{2-3} CPT-OCS-INT-ICD-0070 & LVV-7198 &  \\ \hline
\hline
 DMS-REQ-0372  &
 LVV-9637  &
LVV-T1264 \\
\hline
DMS-REQ-0271 & LVV-9742 &  \\ \hline
\hline
 DMS-REQ-0344  &
 LVV-9744  &
LVV-T1866 \\
\hline
 DMS-LSP-REQ-0007  &
 LVV-9806  &
LVV-T605 \\
\hline
 DMS-LSP-REQ-0001  &
 LVV-9807  &
LVV-T2 \\
 &
 &
LVV-T598 \\
\hline
 DMS-LSP-REQ-0004  &
 LVV-9808  &
LVV-T3 \\
 &
 &
LVV-T602 \\
 &
 &
LVV-T1437 \\
\hline
 DMS-LSP-REQ-0005  &
 LVV-9809  &
LVV-T2 \\
 &
 &
LVV-T603 \\
 &
 &
LVV-T1334 \\
 &
 &
LVV-T1436 \\
 &
 &
LVV-T1437 \\
\hline
 DMS-LSP-REQ-0003  &
 LVV-9810  &
LVV-T601 \\
 &
 &
LVV-T1436 \\
\hline
 DMS-LSP-REQ-0002  &
 LVV-9811  &
LVV-T5 \\
 &
 &
LVV-T600 \\
 &
 &
LVV-T1334 \\
\hline
 DMS-LSP-REQ-0006  &
 LVV-9812  &
LVV-T604 \\
 &
 &
LVV-T1334 \\
 &
 &
LVV-T1436 \\
 &
 &
LVV-T1437 \\
\hline
 DMS-LSP-REQ-0009  &
 LVV-9813  &
LVV-T607 \\
\hline
 DMS-LSP-REQ-0008  &
 LVV-9814  &
LVV-T8 \\
 &
 &
LVV-T9 \\
 &
 &
LVV-T606 \\
\hline
 DMS-LSP-REQ-0010  &
 LVV-9815  &
LVV-T608 \\
\hline
 DMS-LSP-REQ-0012  &
 LVV-9816  &
LVV-T610 \\
\hline
 DMS-LSP-REQ-0011  &
 LVV-9817  &
LVV-T609 \\
\hline
 DMS-LSP-REQ-0013  &
 LVV-9818  &
LVV-T611 \\
\hline
 DMS-LSP-REQ-0014  &
 LVV-9819  &
LVV-T5 \\
 &
 &
LVV-T6 \\
 &
 &
LVV-T7 \\
 &
 &
LVV-T612 \\
\hline
 DMS-LSP-REQ-0018  &
 LVV-9820  &
LVV-T7 \\
 &
 &
LVV-T616 \\
\hline
 DMS-LSP-REQ-0017  &
 LVV-9821  &
LVV-T6 \\
 &
 &
LVV-T615 \\
\hline
 DMS-LSP-REQ-0016  &
 LVV-9822  &
LVV-T614 \\
\hline
 DMS-LSP-REQ-0015  &
 LVV-9823  &
LVV-T613 \\
\hline
 DMS-LSP-REQ-0028  &
 LVV-9824  &
LVV-T4 \\
 &
 &
LVV-T617 \\
\hline
 DMS-LSP-REQ-0029  &
 LVV-9825  &
LVV-T4 \\
 &
 &
LVV-T618 \\
\hline
 DMS-LSP-REQ-0030  &
 LVV-9826  &
LVV-T619 \\
\hline
 DMS-LSP-REQ-0031  &
 LVV-9827  &
LVV-T620 \\
\hline
 DMS-LSP-REQ-0019  &
 LVV-9828  &
LVV-T621 \\
\hline
 DMS-LSP-REQ-0025  &
 LVV-9829  &
LVV-T627 \\
\hline
 DMS-LSP-REQ-0020  &
 LVV-9830  &
LVV-T622 \\
 &
 &
LVV-T1334 \\
 &
 &
LVV-T1436 \\
 &
 &
LVV-T1437 \\
\hline
 DMS-LSP-REQ-0022  &
 LVV-9831  &
LVV-T624 \\
 &
 &
LVV-T1334 \\
 &
 &
LVV-T1436 \\
 &
 &
LVV-T1437 \\
\hline
 DMS-LSP-REQ-0021  &
 LVV-9832  &
LVV-T623 \\
\hline
 DMS-LSP-REQ-0027  &
 LVV-9833  &
LVV-T629 \\
\hline
 DMS-LSP-REQ-0023  &
 LVV-9834  &
LVV-T625 \\
 &
 &
LVV-T1334 \\
 &
 &
LVV-T1436 \\
 &
 &
LVV-T1437 \\
\hline
 DMS-LSP-REQ-0024  &
 LVV-9835  &
LVV-T626 \\
 &
 &
LVV-T1334 \\
 &
 &
LVV-T1436 \\
 &
 &
LVV-T1437 \\
\hline
 DMS-LSP-REQ-0026  &
 LVV-9836  &
LVV-T628 \\
 &
 &
LVV-T1436 \\
\hline
 DMS-LSP-REQ-0033  &
 LVV-9837  &
LVV-T631 \\
\hline
 DMS-LSP-REQ-0034  &
 LVV-9838  &
LVV-T632 \\
\hline
 DMS-LSP-REQ-0032  &
 LVV-9839  &
LVV-T630 \\
\hline
 DMS-LSP-REQ-0035  &
 LVV-9840  &
LVV-T633 \\
\hline
 DMS-PRTL-REQ-0001  &
 LVV-9841  &
LVV-T634 \\
 &
 &
LVV-T1334 \\
\hline
 DMS-PRTL-REQ-0005  &
 LVV-9842  &
LVV-T638 \\
\hline
 DMS-PRTL-REQ-0007  &
 LVV-9843  &
LVV-T640 \\
\hline
 DMS-PRTL-REQ-0008  &
 LVV-9844  &
LVV-T641 \\
\hline
 DMS-PRTL-REQ-0006  &
 LVV-9845  &
LVV-T639 \\
\hline
 DMS-PRTL-REQ-0003  &
 LVV-9846  &
LVV-T636 \\
 &
 &
LVV-T1818 \\
\hline
 DMS-PRTL-REQ-0002  &
 LVV-9847  &
LVV-T635 \\
\hline
 DMS-PRTL-REQ-0004  &
 LVV-9848  &
LVV-T8 \\
 &
 &
LVV-T637 \\
\hline
 DMS-PRTL-REQ-0010  &
 LVV-9849  &
LVV-T643 \\
\hline
 DMS-PRTL-REQ-0013  &
 LVV-9850  &
LVV-T646 \\
\hline
 DMS-PRTL-REQ-0012  &
 LVV-9851  &
LVV-T645 \\
\hline
 DMS-PRTL-REQ-0014  &
 LVV-9852  &
LVV-T647 \\
\hline
 DMS-PRTL-REQ-0011  &
 LVV-9853  &
LVV-T644 \\
\hline
 DMS-PRTL-REQ-0009  &
 LVV-9854  &
LVV-T642 \\
\hline
 DMS-PRTL-REQ-0017  &
 LVV-9855  &
LVV-T650 \\
 &
 &
LVV-T1334 \\
\hline
 DMS-PRTL-REQ-0016  &
 LVV-9856  &
LVV-T5 \\
 &
 &
LVV-T649 \\
 &
 &
LVV-T1334 \\
\hline
 DMS-PRTL-REQ-0015  &
 LVV-9857  &
LVV-T648 \\
 &
 &
LVV-T1334 \\
\hline
 DMS-PRTL-REQ-0018  &
 LVV-9858  &
LVV-T651 \\
\hline
 DMS-PRTL-REQ-0028  &
 LVV-9859  &
LVV-T5 \\
 &
 &
LVV-T652 \\
\hline
 DMS-PRTL-REQ-0029  &
 LVV-9860  &
LVV-T653 \\
\hline
 DMS-PRTL-REQ-0030  &
 LVV-9861  &
LVV-T654 \\
\hline
 DMS-PRTL-REQ-0022  &
 LVV-9862  &
LVV-T5 \\
 &
 &
LVV-T657 \\
\hline
 DMS-PRTL-REQ-0023  &
 LVV-9863  &
LVV-T658 \\
\hline
 DMS-PRTL-REQ-0024  &
 LVV-9864  &
LVV-T659 \\
\hline
 DMS-PRTL-REQ-0021  &
 LVV-9865  &
LVV-T5 \\
 &
 &
LVV-T656 \\
\hline
 DMS-PRTL-REQ-0020  &
 LVV-9866  &
LVV-T655 \\
 &
 &
LVV-T1334 \\
\hline
 DMS-PRTL-REQ-0027  &
 LVV-9868  &
LVV-T5 \\
 &
 &
LVV-T662 \\
\hline
 DMS-PRTL-REQ-0019  &
 LVV-9870  &
LVV-T663 \\
\hline
 DMS-PRTL-REQ-0034  &
 LVV-9871  &
LVV-T668 \\
\hline
 DMS-PRTL-REQ-0033  &
 LVV-9872  &
LVV-T667 \\
\hline
 DMS-PRTL-REQ-0032  &
 LVV-9873  &
LVV-T666 \\
\hline
 DMS-PRTL-REQ-0031  &
 LVV-9874  &
LVV-T664 \\
\hline
 DMS-PRTL-REQ-0039  &
 LVV-9875  &
LVV-T673 \\
\hline
 DMS-PRTL-REQ-0037  &
 LVV-9876  &
LVV-T671 \\
\hline
 DMS-PRTL-REQ-0036  &
 LVV-9877  &
LVV-T670 \\
\hline
 DMS-PRTL-REQ-0035  &
 LVV-9878  &
LVV-T669 \\
\hline
 DMS-PRTL-REQ-0038  &
 LVV-9879  &
LVV-T672 \\
\hline
 DMS-PRTL-REQ-0041  &
 LVV-9880  &
LVV-T7 \\
 &
 &
LVV-T674 \\
\hline
 DMS-PRTL-REQ-0040  &
 LVV-9881  &
LVV-T7 \\
 &
 &
LVV-T675 \\
\hline
 DMS-PRTL-REQ-0044  &
 LVV-9882  &
LVV-T679 \\
\hline
 DMS-PRTL-REQ-0043  &
 LVV-9883  &
LVV-T678 \\
\hline
 DMS-PRTL-REQ-0042  &
 LVV-9884  &
LVV-T677 \\
\hline
 DMS-PRTL-REQ-0045  &
 LVV-9885  &
LVV-T680 \\
\hline
 DMS-PRTL-REQ-0046  &
 LVV-9886  &
LVV-T681 \\
 &
 &
LVV-T1818 \\
\hline
 DMS-PRTL-REQ-0048  &
 LVV-9887  &
LVV-T683 \\
\hline
 DMS-PRTL-REQ-0047  &
 LVV-9888  &
LVV-T682 \\
\hline
 DMS-PRTL-REQ-0050  &
 LVV-9889  &
LVV-T6 \\
 &
 &
LVV-T685 \\
\hline
 DMS-PRTL-REQ-0052  &
 LVV-9890  &
LVV-T687 \\
\hline
 DMS-PRTL-REQ-0049  &
 LVV-9891  &
LVV-T6 \\
 &
 &
LVV-T684 \\
 &
 &
LVV-T1334 \\
\hline
 DMS-PRTL-REQ-0051  &
 LVV-9892  &
LVV-T686 \\
\hline
 DMS-PRTL-REQ-0054  &
 LVV-9893  &
LVV-T6 \\
 &
 &
LVV-T689 \\
\hline
 DMS-PRTL-REQ-0053  &
 LVV-9894  &
LVV-T6 \\
 &
 &
LVV-T688 \\
\hline
 DMS-PRTL-REQ-0056  &
 LVV-9895  &
LVV-T6 \\
 &
 &
LVV-T691 \\
\hline
 DMS-PRTL-REQ-0061  &
 LVV-9896  &
LVV-T696 \\
\hline
 DMS-PRTL-REQ-0059  &
 LVV-9897  &
LVV-T694 \\
\hline
 DMS-PRTL-REQ-0058  &
 LVV-9898  &
LVV-T693 \\
\hline
 DMS-PRTL-REQ-0060  &
 LVV-9899  &
LVV-T695 \\
\hline
 DMS-PRTL-REQ-0057  &
 LVV-9900  &
LVV-T692 \\
\hline
 DMS-PRTL-REQ-0055  &
 LVV-9901  &
LVV-T6 \\
 &
 &
LVV-T690 \\
\hline
 DMS-PRTL-REQ-0067  &
 LVV-9902  &
LVV-T701 \\
\hline
 DMS-PRTL-REQ-0066  &
 LVV-9903  &
LVV-T700 \\
\hline
 DMS-PRTL-REQ-0065  &
 LVV-9904  &
LVV-T699 \\
\hline
 DMS-PRTL-REQ-0062  &
 LVV-9905  &
LVV-T676 \\
\hline
 DMS-PRTL-REQ-0063  &
 LVV-9906  &
LVV-T697 \\
\hline
 DMS-PRTL-REQ-0064  &
 LVV-9907  &
LVV-T698 \\
\hline
 DMS-PRTL-REQ-0068  &
 LVV-9908  &
LVV-T702 \\
\hline
 DMS-PRTL-REQ-0069  &
 LVV-9909  &
LVV-T703 \\
\hline
 DMS-PRTL-REQ-0074  &
 LVV-9910  &
LVV-T708 \\
\hline
 DMS-PRTL-REQ-0071  &
 LVV-9911  &
LVV-T705 \\
\hline
 DMS-PRTL-REQ-0072  &
 LVV-9912  &
LVV-T706 \\
\hline
 DMS-PRTL-REQ-0073  &
 LVV-9913  &
LVV-T707 \\
\hline
 DMS-PRTL-REQ-0070  &
 LVV-9914  &
LVV-T704 \\
\hline
 DMS-PRTL-REQ-0075  &
 LVV-9915  &
LVV-T709 \\
\hline
 DMS-PRTL-REQ-0077  &
 LVV-9916  &
LVV-T711 \\
\hline
 DMS-PRTL-REQ-0076  &
 LVV-9917  &
LVV-T710 \\
\hline
 DMS-PRTL-REQ-0078  &
 LVV-9918  &
LVV-T712 \\
\hline
 DMS-PRTL-REQ-0081  &
 LVV-9919  &
LVV-T715 \\
\hline
 DMS-PRTL-REQ-0080  &
 LVV-9920  &
LVV-T714 \\
\hline
 DMS-PRTL-REQ-0082  &
 LVV-9921  &
LVV-T716 \\
\hline
 DMS-PRTL-REQ-0079  &
 LVV-9922  &
LVV-T713 \\
\hline
 DMS-PRTL-REQ-0087  &
 LVV-9923  &
LVV-T721 \\
\hline
 DMS-PRTL-REQ-0083  &
 LVV-9924  &
LVV-T717 \\
\hline
 DMS-PRTL-REQ-0086  &
 LVV-9925  &
LVV-T720 \\
\hline
 DMS-PRTL-REQ-0088  &
 LVV-9927  &
LVV-T722 \\
\hline
 DMS-PRTL-REQ-0084  &
 LVV-9928  &
LVV-T718 \\
\hline
 DMS-PRTL-REQ-0091  &
 LVV-9929  &
LVV-T725 \\
\hline
 DMS-PRTL-REQ-0093  &
 LVV-9930  &
LVV-T727 \\
\hline
 DMS-PRTL-REQ-0092  &
 LVV-9931  &
LVV-T726 \\
\hline
 DMS-PRTL-REQ-0095  &
 LVV-9932  &
LVV-T729 \\
 &
 &
LVV-T1334 \\
 &
 &
LVV-T1818 \\
\hline
 DMS-PRTL-REQ-0090  &
 LVV-9933  &
LVV-T724 \\
\hline
 DMS-PRTL-REQ-0089  &
 LVV-9934  &
LVV-T723 \\
\hline
 DMS-PRTL-REQ-0094  &
 LVV-9935  &
LVV-T728 \\
\hline
 DMS-PRTL-REQ-0096  &
 LVV-9936  &
LVV-T730 \\
\hline
 DMS-PRTL-REQ-0105  &
 LVV-9938  &
LVV-T739 \\
\hline
 DMS-PRTL-REQ-0107  &
 LVV-9939  &
LVV-T741 \\
\hline
 DMS-PRTL-REQ-0102  &
 LVV-9940  &
LVV-T736 \\
\hline
 DMS-PRTL-REQ-0106  &
 LVV-9941  &
LVV-T740 \\
\hline
 DMS-PRTL-REQ-0098  &
 LVV-9942  &
LVV-T732 \\
\hline
 DMS-PRTL-REQ-0099  &
 LVV-9943  &
LVV-T733 \\
\hline
 DMS-PRTL-REQ-0100  &
 LVV-9944  &
LVV-T734 \\
\hline
 DMS-PRTL-REQ-0101  &
 LVV-9945  &
LVV-T735 \\
\hline
 DMS-PRTL-REQ-0104  &
 LVV-9946  &
LVV-T738 \\
\hline
 DMS-PRTL-REQ-0108  &
 LVV-9947  &
LVV-T742 \\
\hline
 DMS-PRTL-REQ-0103  &
 LVV-9948  &
LVV-T737 \\
\hline
 DMS-PRTL-REQ-0109  &
 LVV-9949  &
LVV-T743 \\
\hline
 DMS-PRTL-REQ-0113  &
 LVV-9950  &
LVV-T747 \\
\hline
 DMS-PRTL-REQ-0111  &
 LVV-9951  &
LVV-T745 \\
 &
 &
LVV-T1818 \\
\hline
 DMS-PRTL-REQ-0114  &
 LVV-9952  &
LVV-T748 \\
\hline
 DMS-PRTL-REQ-0112  &
 LVV-9953  &
LVV-T746 \\
\hline
 DMS-PRTL-REQ-0110  &
 LVV-9954  &
LVV-T744 \\
 &
 &
LVV-T1818 \\
\hline
 DMS-PRTL-REQ-0115  &
 LVV-9955  &
LVV-T749 \\
\hline
 DMS-PRTL-REQ-0117  &
 LVV-9956  &
LVV-T751 \\
\hline
 DMS-PRTL-REQ-0118  &
 LVV-9957  &
LVV-T752 \\
\hline
 DMS-PRTL-REQ-0116  &
 LVV-9958  &
LVV-T750 \\
\hline
 DMS-PRTL-REQ-0127  &
 LVV-9959  &
LVV-T756 \\
\hline
 DMS-PRTL-REQ-0119  &
 LVV-9960  &
LVV-T753 \\
\hline
 DMS-PRTL-REQ-0120  &
 LVV-9961  &
LVV-T754 \\
\hline
 DMS-PRTL-REQ-0121  &
 LVV-9962  &
LVV-T755 \\
\hline
 DMS-PRTL-REQ-0122  &
 LVV-9963  &
LVV-T757 \\
\hline
 DMS-PRTL-REQ-0124  &
 LVV-9964  &
LVV-T759 \\
\hline
 DMS-PRTL-REQ-0123  &
 LVV-9965  &
LVV-T758 \\
\hline
 DMS-PRTL-REQ-0126  &
 LVV-9966  &
LVV-T761 \\
\hline
 DMS-PRTL-REQ-0125  &
 LVV-9967  &
LVV-T760 \\
\hline
 DMS-NB-REQ-0010  &
 LVV-9968  &
LVV-T767 \\
\hline
 DMS-NB-REQ-0009  &
 LVV-9969  &
LVV-T766 \\
\hline
 DMS-NB-REQ-0014  &
 LVV-9970  &
LVV-T771 \\
\hline
 DMS-NB-REQ-0005  &
 LVV-9971  &
LVV-T762 \\
 &
 &
LVV-T1436 \\
\hline
 DMS-NB-REQ-0015  &
 LVV-9972  &
LVV-T772 \\
\hline
 DMS-NB-REQ-0013  &
 LVV-9973  &
LVV-T770 \\
 &
 &
LVV-T1436 \\
\hline
 DMS-NB-REQ-0007  &
 LVV-9974  &
LVV-T764 \\
\hline
 DMS-NB-REQ-0008  &
 LVV-9975  &
LVV-T765 \\
\hline
 DMS-NB-REQ-0006  &
 LVV-9976  &
LVV-T763 \\
 &
 &
LVV-T1436 \\
\hline
 DMS-NB-REQ-0012  &
 LVV-9977  &
LVV-T769 \\
\hline
 DMS-NB-REQ-0011  &
 LVV-9978  &
LVV-T768 \\
\hline
 DMS-NB-REQ-0023  &
 LVV-9979  &
LVV-T780 \\
\hline
 DMS-NB-REQ-0017  &
 LVV-9980  &
LVV-T774 \\
 &
 &
LVV-T1436 \\
\hline
 DMS-NB-REQ-0021  &
 LVV-9981  &
LVV-T778 \\
\hline
 DMS-NB-REQ-0022  &
 LVV-9982  &
LVV-T779 \\
\hline
 DMS-NB-REQ-0016  &
 LVV-9983  &
LVV-T773 \\
\hline
 DMS-NB-REQ-0020  &
 LVV-9984  &
LVV-T777 \\
\hline
 DMS-NB-REQ-0018  &
 LVV-9985  &
LVV-T775 \\
\hline
 DMS-NB-REQ-0019  &
 LVV-9986  &
LVV-T776 \\
\hline
 DMS-NB-REQ-0025  &
 LVV-9987  &
LVV-T782 \\
\hline
 DMS-NB-REQ-0024  &
 LVV-9988  &
LVV-T781 \\
\hline
 DMS-NB-REQ-0026  &
 LVV-9989  &
LVV-T783 \\
\hline
 DMS-NB-REQ-0032  &
 LVV-9990  &
LVV-T784 \\
\hline
 DMS-NB-REQ-0033  &
 LVV-9991  &
LVV-T785 \\
\hline
 DMS-NB-REQ-0035  &
 LVV-9992  &
LVV-T787 \\
\hline
 DMS-NB-REQ-0034  &
 LVV-9993  &
LVV-T786 \\
\hline
 DMS-NB-REQ-0036  &
 LVV-9994  &
LVV-T788 \\
\hline
 DMS-NB-REQ-0030  &
 LVV-9995  &
LVV-T790 \\
\hline
 DMS-NB-REQ-0029  &
 LVV-9996  &
LVV-T789 \\
 &
 &
LVV-T1436 \\
\hline
 DMS-NB-REQ-0031  &
 LVV-9997  &
LVV-T791 \\
\hline
 DMS-NB-REQ-0002  &
 LVV-9998  &
LVV-T793 \\
 &
 &
LVV-T1436 \\
\hline
 DMS-NB-REQ-0003  &
 LVV-9999  &
LVV-T794 \\
\hline
 DMS-NB-REQ-0001  &
 LVV-10000  &
LVV-T792 \\
 &
 &
LVV-T1436 \\
\hline
 DMS-NB-REQ-0004  &
 LVV-10001  &
LVV-T795 \\
\hline
 DMS-API-REQ-0023  &
 LVV-10002  &
LVV-T798 \\
 &
 &
LVV-T1437 \\
\hline
 DMS-API-REQ-0022  &
 LVV-10003  &
LVV-T797 \\
\hline
 DMS-API-REQ-0028  &
 LVV-10004  &
LVV-T803 \\
\hline
 DMS-API-REQ-0024  &
 LVV-10005  &
LVV-T799 \\
\hline
 DMS-API-REQ-0026  &
 LVV-10006  &
LVV-T801 \\
\hline
 DMS-API-REQ-0027  &
 LVV-10007  &
LVV-T802 \\
\hline
 DMS-API-REQ-0030  &
 LVV-10008  &
LVV-T805 \\
\hline
 DMS-API-REQ-0025  &
 LVV-10009  &
LVV-T800 \\
\hline
 DMS-API-REQ-0029  &
 LVV-10010  &
LVV-T804 \\
\hline
 DMS-API-REQ-0021  &
 LVV-10011  &
LVV-T796 \\
\hline
 DMS-API-REQ-0009  &
 LVV-10012  &
LVV-T809 \\
 &
 &
LVV-T1437 \\
\hline
 DMS-API-REQ-0008  &
 LVV-10013  &
LVV-T808 \\
 &
 &
LVV-T1437 \\
\hline
 DMS-API-REQ-0007  &
 LVV-10014  &
LVV-T807 \\
 &
 &
LVV-T1437 \\
\hline
 DMS-API-REQ-0006  &
 LVV-10015  &
LVV-T806 \\
 &
 &
LVV-T1437 \\
\hline
 DMS-API-REQ-0016  &
 LVV-10016  &
LVV-T810 \\
\hline
 DMS-API-REQ-0018  &
 LVV-10017  &
LVV-T812 \\
\hline
 DMS-API-REQ-0017  &
 LVV-10018  &
LVV-T811 \\
\hline
 DMS-API-REQ-0039  &
 LVV-10019  &
LVV-T814 \\
 &
 &
LVV-T1437 \\
\hline
 DMS-API-REQ-0038  &
 LVV-10020  &
LVV-T813 \\
\hline
 DMS-API-REQ-0040  &
 LVV-10021  &
LVV-T815 \\
\hline
 DMS-API-REQ-0034  &
 LVV-10022  &
LVV-T816 \\
\hline
 DMS-API-REQ-0019  &
 LVV-10023  &
LVV-T817 \\
\hline
 DMS-API-REQ-0020  &
 LVV-10024  &
LVV-T818 \\
\hline
 DMS-API-REQ-0014  &
 LVV-10025  &
LVV-T823 \\
\hline
 DMS-API-REQ-0013  &
 LVV-10026  &
LVV-T822 \\
\hline
 DMS-API-REQ-0015  &
 LVV-10027  &
LVV-T824 \\
\hline
 DMS-API-REQ-0012  &
 LVV-10028  &
LVV-T821 \\
\hline
 DMS-API-REQ-0010  &
 LVV-10029  &
LVV-T819 \\
\hline
 DMS-API-REQ-0011  &
 LVV-10030  &
LVV-T820 \\
\hline
 DMS-API-REQ-0033  &
 LVV-10031  &
LVV-T827 \\
\hline
 DMS-API-REQ-0031  &
 LVV-10032  &
LVV-T825 \\
\hline
 DMS-API-REQ-0032  &
 LVV-10033  &
LVV-T826 \\
\hline
 DMS-API-REQ-0003  &
 LVV-10034  &
LVV-T829 \\
 &
 &
LVV-T1437 \\
\hline
 DMS-API-REQ-0004  &
 LVV-10035  &
LVV-T830 \\
 &
 &
LVV-T1437 \\
\hline
 DMS-API-REQ-0005  &
 LVV-10036  &
LVV-T831 \\
\hline
 DMS-API-REQ-0001  &
 LVV-10037  &
LVV-T828 \\
 &
 &
LVV-T1437 \\
\hline
 DMS-API-REQ-0035  &
 LVV-10038  &
LVV-T832 \\
\hline
 DMS-API-REQ-0037  &
 LVV-10039  &
LVV-T835 \\
\hline
 DMS-API-REQ-0002  &
 LVV-10040  &
LVV-T833 \\
\hline
 DMS-API-REQ-0036  &
 LVV-10041  &
LVV-T834 \\
\hline
 DMS-REQ-0384  &
 LVV-18222  &
LVV-T1524 \\
\hline
 DMS-REQ-0381  &
 LVV-18223  &
LVV-T1525 \\
\hline
 DMS-REQ-0380  &
 LVV-18224  &
LVV-T1526 \\
\hline
 DMS-REQ-0385  &
 LVV-18226  &
LVV-T1528 \\
\hline
 DMS-REQ-0386  &
 LVV-18230  &
LVV-T1560 \\
\hline
 DMS-REQ-0387  &
 LVV-18231  &
LVV-T1561 \\
\hline
 DMS-REQ-0388  &
 LVV-18232  &
LVV-T1562 \\
\hline
OCS-EFD-HS-0001 & LVV-18271 &  \\ \hline
\hline
OCS-EFD-HS-0002 & LVV-18272 &  \\ \hline
\hline
OCS-EFD-HS-0003 & LVV-18273 &  \\ \hline
\hline
OCS-EFD-HS-0004 & LVV-18274 &  \\ \hline
\hline
OCS-EFD-HS-0005 & LVV-18275 &  \\ \hline
\hline
OCS-EFD-HS-0006 & LVV-18276 &  \\ \hline
\hline
OCS-EFD-HS-0007 & LVV-18277 &  \\ \hline
\hline
OCS-EFD-HS-0008 & LVV-18278 &  \\ \hline
\hline
OCS-EFD-HS-0009 & LVV-18279 &  \\ \hline
\hline
OCS-EFD-HS-0010 & LVV-18280 &  \\ \hline
\hline
OCS-EFD-HS-0011 & LVV-18281 &  \\ \hline
\hline
OCS-EFD-HS-0012 & LVV-18282 &  \\ \hline
\hline
OCS-EFD-HS-0013 & LVV-18283 &  \\ \hline
\hline
OCS-EFD-HS-0014 & LVV-18284 &  \\ \hline
\hline
OCS-EFD-HS-0015 & LVV-18285 &  \\ \hline
\hline
CA-DM-CON-ICD-0020 & LVV-18849 &  \\ \hline
\hline
CA-DM-CON-ICD-0022 & LVV-18852 &  \\ \hline
\hline
CA-DM-CON-ICD-0023 & LVV-18855 &  \\ \hline
\hline
CA-DM-CON-ICD-0021 & LVV-18858 &  \\ \hline
\hline
 DMS-REQ-0391  &
 LVV-18911  &
LVV-T1868 \\
\hline
\end{longtable}

Note that some of the requirements listed in this traceability table may be related with additional
Verification Elements not in the scope of \textit{ Service } Verification,
and therefore not listed here.


\newpage
\section{References\label{sect:references}}
\renewcommand{\refname}{}
\bibliography{lsst,refs,books,refs_ads,local.bib}

\newpage
\section{Acronyms \label{sect:acronyms}} % include acronyms.tex generated by the generateAcronyms.py (in texmf/scripts)
\input{acronyms}


\end{document}
